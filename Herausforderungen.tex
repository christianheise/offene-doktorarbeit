\chapter{Herausforderungen in der wissenschaftlichen Kommunikation}

Im Folgenden werden wesentliche Herausforderungen im bestehenden System wissenschaftlicher Kommunikation in Ergänzung zu den Grundlagen dargestellt. In einem weiteren Schritt werden die Erkenntnisse zusammengefasst sowie die Katalysatoren und Hindernisse für die Öffnung von Wissenschaft identifiziert, extrapoliert und in der Gesamtbetrachtung der Arbeit zusammengeführt und strukturiert dargestellt. Zur Analyse und Darstellung der Herausforderungen wird aus ausgewählten Texten das Spektrum der unterschiedlichen Debatten rund um den Themenkomplex der Komunikation in Wissenschaft und Forschung herausgearbeitet, die Debatten verdichtet und für die Befragung der Wissenschaftler und Wissenschaftlerinnen verschiedener Fachbereiche zusammengefasst.

\section{Herausforderungen im bestehenden System wissenschaftlicher Kommunikation}

Diese Kanalisierung des Wissens im Rahmen der wissenschaftlichen Kommunikation und die Wirksamkeit, wie auch Zweckmäßigkeit dieses wissenschaftlichen Kommunikationssystems sind seit Jahrzehnten Bestandeil von Debatten in der Literatur \cite{Simon_2009} in denen diese immer wieder hinterfragt und als begrenzt geeignet bezeichnet werden \cite{Hornbostel_1997} \cite{Hicks_1996} \cite{Havemann_2002} \cite{Warnke_2012} \cite{Brembs_2013a}. Wie im Grundlagenkapitel dargestellt, sind die Debatten zwischen den Akteuren und ihre Fließrichtungen sehr vielfältig. Die folgende stukturierte Einteilung in verschiedene Bereiche der Kritik und Darstellung dieser dient der Einordung der Herausforderungen und zur Eingrenzung der Debatten um das aktuelle System der wissenschaftlichen Kommunikation. Sie werden im Verlauf der Arbeit bei der Betrachtung von Hindernissen und Katalysatoren sowie im Rahmen der Befragung an späterer Stelle erneut aufgegriffen.

Die Herausforderungen im bestehenden System formeller wissenschaftlicher Kommunikation beziehen sich vor allem auf neun Bereiche:
\begin{enumerate}
\item Leistungsbewertung wissenschaftlicher Arbeit
\item Geschwindigkeit im Kommunikationsprozess
\item Wahrung der Freiheit von Wissenschaft und Forschung
\item Effizienz
\item Fehlerresistenz und Qualitätssicherung
\item Verbreitung und Zugänglichkeit
\item Digitalisierung
\item Möglichkeiten der Überprüfbarkeit des Wissens/der wissenschaftlichen Güte
\item Verhinderung von Missbrauch und wissenschaftliches Fehlverhalten
\end{enumerate}

\subsection{Leistungsbewertung wissenschaftlicher Arbeit}

Die Verlage haben in den letzten Dekaden mit den wissenschaftlichen Journalen und Monografien ein zentrales Steuerungs- und Bewertungssystem in der Wissenschaft etablieren können. In diesem System werden die Grundprinzipien der Wissenschaft für die verlegerischen Verwertungsinteressen (aus)genutzt und das, obwohl diese "wissenschaftlichen Grundprinzipien und Normen eigentlich ökonomischen Verwertungsinteressen zu widersprechen scheinen" \cite{Hanekop_2006}. Darüber agieren die Forscherinnen und Forscher in einem Umfeld, in dem sie in vielen Fällen wenig oder keine Verantwortung für den Einkauf der wissenschaftlichen Informationen haben, die er oder sie im Rahmen der Veröffentlichung "verschenkt" \cite{Steele_2006}.

Die Einführung der quantitativer Bewertungsindikatoren wie das Zitationsregister und die Impact Faktoren, sowie die Definition der Kernzeitschriften, führte zu einer weitgehenden Erstarrung des wissenschaftliche Zeitschriftenmarktes und gleichzeitig zu einem Anstieg der Kapazität der kommerziellen Verlagen, sowie deren Gewinnmargen \cite{CREATe_2014}. Die Steuerungsmechanismen werden über die Messbarkeit mittels Methoden direkt oder indirekt ausgeübt. Dabei stehen insbesondere die Methoden, die auf der quantitativen Grundlage der Zitationsraten wissenschaftlicher Publikationen gemessen werden in der Kritik \cite{Brembs_2013} \cite{Dong_2005} und auch andere Indikatoren für die Messung von Forschungsleistungen sind hoch umstritten \cite{Hornbostel_1997} \cite{Hicks_1996} \cite{Havemann_2002} \cite{Warnke_2012}. Die Verfahren, um die Wirkung von Wissenschaft und damit auch die Reputation von Wissenschaftlern zu messen, beruhen auf einer "fragwürdigen wissenschaftlichen Grundlage" \cite[:10]{Osterloh_2008}. Darüber hinaus sind weder "importance noch impact noch quality (...) direkt messbar" und man kann sich ihnen nur "nähern" \cite[:188]{Hornbostel_1997}. Das führt unter anderem dazu, dass der jährlich aus der "Zahl der Zitationen auf die Beiträge einer Zeitschrift ermittelte" \cite[:26]{Weishaupt_2009} Impact Factor nicht als perfektes Werkzeug betrachtet werden kann, um die Qualität der Artikel zu messen \cite{Garfield_1999} und "selbst die grundlegendsten wissenschaftlichen Standards verletzt" \cite{Brembs_2013}. Trotzdem wird er zur Bewertung von Wissenschaft genutzt, denn es gibt nichts Besseres und er hat den Vorteil, dass er allein durch seine lange Existenz eine gute Technik für die wissenschaftliche Bewertung darstellt \cite{Garfield_1999} \cite{Weishaupt_2009}.

Die Kritik am Impact Factor lässt sich laut der Bibliotheks- und Informationswissenschaftlerin Karin Weishaupt am Beispiel des "Thomson Reuters Journal Citation Factors" in sechs Punkten zusammenfassen \cite{Weishaupt_2009}:
\begin{enumerate}
\item Der Impact Factor bezieht sich immer auf die gesamte Zeitschrift und hat somit keine Aussagekraft über die "Rezeption oder Qualität des einzelnen Artikels" \cite{Weishaupt_2009}.
\item Der Impact Factor berücksichtigt nur die Zeitschriften, die im eigenen Index gelistet sind und enthält weder Monografien, Tagungsbeiträge, sonstige Beiträge oder Internetquellen.
\item Durch Selbstzitierungen sind Manipulationen möglich.
\item Es werden nur Zitate aus den letzten beiden Jahren berücksichtigt und je nach Fachgebiet ist es von Vorteil wenn im eigenen Gebiet die Verwertungszyklen kürzer sind.
\item Publikationen, die nicht in englischer Sprache verfasst sind, weisen überwiegend eine geringere Sichtbarkeit und Popularität auf, da englische Journale überproportional vertreten sind
\item Spezialisierte Zeitschriften sind ebenfalls systematisch benachteiligt gegenüber Journalen großer Fach-Communities oder Journalen mit Übersichtsartikeln.
\end{enumerate}

Der neue Managerialismus an Universitäten setzen dabei auf die quantitative Leistungsmessung und die wissenschaftlichen Kommunikation wird zunehmend anhand quantitativer bibliometrischer Methoden evaluiert \cite[:40]{Frost_2014}. Seit der Entwicklung des Science Citation Index (SCI) sowie des Aufkommens systematischer Wissenschaftsevaluation in Form von Rankings wird das zunehmend von Autoren, Wissenschaftlern, Lesern, Verlagen und Herausgebern für die Evaluation der Wirkung der Kommunikation akzeptiert und adoptiert \cite[:2]{Haustein_2012}. Diese rein quantitative Betrachtungen können eine Tendenz zu Fehlanreizen darstellen \cite{Wissenschaftsrat_2015} die dazu führt, dass zunehmend messbarer Output ein wichtigeres Ziel für Wissenschaftler und Wissenschaftlerinnen darstellt, als die eigentliche Kreation und Produktion von originellen und innovativen Wissen nach den Kriterien der guten wissenschaftlichen Praxis \cite[:41]{Frost_2014}.

Hier offenbart sich ein "Generaldilemma wettbewerblicher Wissenschaft" \cite[:37]{Wissenschaftsrat_2015}. Die Idee, dass Wettbewerb in der Wissenschaft zu mehr Qualität führt steht dem Überdruck und der Beschleunigung im System gegenüber, "was Qualitätsverluste und eine Gefährdung wissenschaftlicher Integrität zufolge haben kann" \cite[:37]{Wissenschaftsrat_2015}. In der praktischen Auslegung der Entwicklungen von Universitäten wird in diesem Zusammenhang auch von der Entmythologisierung der Humboldt’schen "Einheit von Forschung und Lehre" gesprochen \cite{Binswanger_2014} \cite[:299]{Schimank_2001} \cite[:343]{Kruecken_2001} und es ist nicht zu verleugnen, dass in der Wissenschaft zunehmend ein Zusammenhang zwischen ökonomischer Effizienz, Kontrollmechanismen und Öffentlichkeit herrscht \cite[:27]{Reinhart_2006} \cite{Foucault_1977}  \cite{Meier_2009}. Diese hat jedoch nicht erst mit dem steigenden Kosten- und Effizienzdruck, der Frage nach der Verwertbarkeit von Wissenschaft und Forschung sowie der Modernisierung der Steuerungsmechanismen stattgefunden, sondern schon viel früher wurde die Ausrichtung der Universität auf die Verwertbarkeit wissenschaftlichen Wissens kritisiert \cite{Huber_2005}. Die Idee der Einheit von Forschung und Lehre auf Grundlage eines völligen Verzichts auf Differenzierung \cite{kittler_2004}, lässt sich somit grundsätzlich nur in Ausnahmefällen realisieren \cite{Schimank_2001}. Als realistische Lesart kann im vorherrschenden System nur eine situative Differenzierung stattfinden, bei der die Mittel der Grundausstattung nicht nach beiden Aufgaben separiert sind \cite{Schimank_2001}.

Dennoch ist die Lesart der Humboldt’schen Idee noch immer hegemonialer Rahmen der aktuellen Hochschulreformen \cite{Huber_2005}. Das Recht auf Freiheit von Lehre und Forschung und die Humboldt’sche Idee der Universität wird und wurde immer für die Erhaltung des "organisationellen Status Quo", die Absicherung der "Institution Universität" und die Wahrung der "Staatsunabhängigkeit" angebracht \cite{Huber_2005}. Diese Autonomie der Wissenschaft und Forschung gilt auch heute als "hohes Gut, das es gegen externe Anforderungen zu verteidigen gilt" \cite{Kaldewey_2010}. Auch im Zusammenhang mit den Veränderungen der wissenschaftlichen Kommunikation sowie die Konsequenzen der Veränderung auf die Steurung von Wissenschaft ist auf diese Ausprägungen zu achten, "will man diese Ent-
kopplung entweder befördern oder verhindern" \cite[:57]{Meier_2009}.

Bisher bleibt festzuhalten, dass die im aktuellen wissenschaftlichen System genutzten Indikatoren die komplexe Realität der Leistungsbewertung in der Wissenschaft nicht abbilden können und sie eine eigene Realität konstruieren \cite[:188]{Hornbostel_1997}. Versteht man Wissenschaft als soziales System, so stellen Reputation sowie "die Stabilisierung eines guten Rufes" und nicht die Wahrheit der Beobachtungen und Erklärungen "nicht selten auch eingestandenes vorrangiges Ziel wissenschaftlicher Tätigkeit" dar \cite[:237]{Luhmann_1970}. Wie gering der Wirkungsgrad und die Methoden zur Messung "zur Reproduktion des traditionellen wissenschaftlichen Diskurses ausfällt, wird von dem Moment an klar, an dem ein neues Kommunikationsmedium wie das Internet als Alternative zur Verfügung steht" \cite{Rost_1998}.

\subsection{Geschwindigkeit im Kommunikationsprozess}

Einen weiterer Aspekt der Debatte betrifft die Kritik an der Geschwindigkeit zwischen der Fertigstellung einer wissenschaftlichen Arbeit durch den Autoren, der Einreichung zur Veröffentlichung und der finalen Veröffentlichung der Ergebnisse. Trotz der Beschleunigung der Prozesse bei der Qualitätssicherung und Bewertung wissenschaftlicher Arbeiten durch die Digitalisierung der Kommunikation zwischen Wissenschaftlern, Gutachtern und Verlagen kann es mehrere Jahre dauern, bevor ein Text veröffentlicht wird \cite{Curry_2015} \cite{Nosek_2012} \cite{Smith_2006}. Diese Verzögerung beruht unter anderem auf folgenden Umständen:

\begin{enumerate}
\item Gutachter/innen können aufgrund der Ausführung dieser Funktion als Nebentätigkeit meist Termine nicht einhalten \cite{Bar_2009}.
\item Es gibt weder Anreiz- noch Sanktionsmöglichkeiten für Gutachter und Gutachterinnen.
\item Die wissenschaftlichen Zeitschriften erscheinen größtenteils noch immer als Periodika und wissenschaftlichen Bücher orientieren sich am Druck. Sie sind damit für einen bestimmten Zeitraum der Veröffentlichung terminiert.
\end{enumerate}

Eine Möglichkeit, die wissenschaftlichen Inhalte schneller zugänglich zu machen, ohne den sehr zeitaufwändigen Begutachtungsprozess strukturell oder inhaltlich zu verändern, ist die Veröffentlichung der wissenschaftlichen Arbeit als digitalen Pre-Print, sowie eine umfangreichere Kommunikation des Erkenntnisprozesses vor der finalen Veröffentlichung. Von dieser Möglichkeit machen immer mehr Wissenschaftler und Wissenschaftlerinnen gebrauch \cite{Curry_2015}. Eine weitere Möglichkeit stellt die offene Begutachtung (Open Peer Commentary) dar, bei der ein Text anonymisiert (vorab) veröffentlicht und kommuniziert, sowie von der wissenschaftlichen Gemeinschaft kollaborativ bewertet wird \cite{Mueller_2009} \cite{Smith_2006}. Dabei darf der Wunsch nach einer erhöhten Geschwindigkeit nicht über den Wunsch nach einem ausgewogenen Qualitätssicherungsprozess gestellt werden \cite{Beall_2012}.

\begin{figure}[h!]
\includegraphics{smalltableid:HW9dy}
\caption{Eigenschaften und Ausprägungen von OPC-Verfahren mit entsprechenden Beispielen}
\end{figure}

Insgesamt behindert die Trägheit des tradierten Systems der wissenschaftlichen Kommunikation den wissenschaftlichen Fortschritt und wird den Möglichkeiten für die digitale Informationsversorgung nicht gerecht. Dabei ist die schnelle und umfassende Verbreitung von wissenschaftlichen Informationen und Daten im Rahmen des kumulativ orientieren wissenschaftlichen Erkenntnisprozesses von grundlegender Bedeutung. Forscher und Forscherinnen würden in vielfacher Hinsicht davon profitieren, wenn sie gegenseitig schneller auf die Ergebnisse ihrer Arbeit zugreifen könnten \cite{Nosek_2012} \cite{Winkler_2011}.

\subsection{Wahrung der Freiheit von Wissenschaft und Forschung}

Die freie Verbreitung von Informationen und offene Diskussion ist ein wesentlicher Teil des wissenschaftlichen Prozesses \cite{Edsall_1976}. Das Recht auf Wissenschaftsfreiheit ist ein "Erbe der deutschen Achtundvierzigerrevolution \cite{Kempny_2013}. Neben der Freiheit der Lehre bildet die Freiheit der Forschung den zweiten Pfeiler der Wissenschaftsfreiheit \cite[:46]{Thurnherr_2014} \cite{Meier_2009}. Die Forschungsfreiheit ist in Deutschland grundrechtlich nach Artikel 5 Absatz 3 Grundgesetz (GG) geschützt und ist auch europäisches Verfassungsgut \cite{Kempny_2013}. Sie ist eine "Freiheit schlechthin, nicht Freiheit zu bestimmten Zielen oder Zwecken" \cite[:1530]{Boeckenfoerde_1974} und ihr Schutzbereich umfasst auch die Bewertung der Forschungsergebnisse sowie ihre Verbreitung \cite[:429]{Pfeiffer_2013}.

Die Wissenschaft unterliegt mannigfaltigen externen Einflüssen, operiert aber dennoch autonom \cite{Luhmann_1998}. So greifen "andere Funktionssysteme [...] in die Wissenschaft zwar ein, wenn sie in Erfüllung ihrer eigenen Funktionen operieren und ihren eigenen Codes folgen. Aber sie können, jedenfalls unter den Bedingungen der modernen Gesellschaft, nicht selbst festlegen, was wahr und was unwahr ist" \cite[:293]{Luhmann_1998}. Dabei ist die vorgabenfreie Erarbeitung und Veröffentlichung neuer Erkenntnisse die Grundlage für wissenschaftlichen Fortschritt. "Die Autonomie der Wissenschaft wird nach außen durch die Abhängigkeit der Universität vom Staat und universitätsintern durch die Einheit von Wissenschaft und Forschung gesichert" \cite{Huber_2005}. Diese Wahrung ist im Artikel 5 Absatz 3 GG als garantiertes Grundrecht wie folgt festgehalten: "Wissenschaft, Forschung und Lehre sind frei" \cite{Grundgesetz_2015}.

Dieses Recht ist nicht nur ein Grundrecht auf wissenschaftliche Meinungsfreiheit, sondern auch eine rechtliche Garantie. "Jeder, der in Wissenschaft, Forschung und Lehre tätig ist, hat – vorbehaltlich der Treuepflicht gemäß Art. 5 Absatz 3 Satz 2 GG – ein Recht auf Abwehr jeder staatlichen Einwirkung auf den Prozess der Gewinnung und Vermittlung wissenschaftlicher Erkenntnisse" \cite{BVerfGE_1973}. Das garantiert einerseits die Einrichtung wissenschaftlicher Hochschulen mit Anspruch auf Selbstverwaltung, die staatliche Finanzierung und die Absicherung ihrer Arbeit, andererseits richtet es sich als "Abwehrrecht auf die Abwehr von Eingriffen in die wissenschaftliche Betätigung" gegen staatliche Eingriffe \cite{Mayen_1992} \cite{Spindler_2006}. Jede Form der wissenschaftlichen Betätigung ist durch dieses Abwehrrecht geschützt. Dazu zählen laut Urteil des Bundesverfassungsgerichts "vor allem die auf wissenschaftlichen Eigengesetzlichkeiten beruhenden Prozesse, Verhaltensweisen und Entscheidungen bei dem Auffinden von Erkenntnissen, ihrer Deutung und Weitergabe" \cite{BVerfGE_1973}.

Christopher Kelty bedient sich bei der grundlegenden Einordnung von Freiheit des Konzepts der positiven und negativen Freiheit \cite{Kelty_2014}. Die positive Freiheit definiert dabei die Freiheit, die es aktiv erlaubt etwas zu tun. Die negative Freiheit beschreibt demgegenüber die Freiheit von bestimmten (meist unerwünschten) Einflüssen. Damit eignet sich dieses Konzept von Freiheit als ein analytisches Werkzeug für die Erforschung der Auswirkungen von neuen Technologien \cite{Kelty_2014}. Das betrifft auch die freie Entscheidung über die Art und Weise der Veröffentlichung von Forschungsergebnissen (positive Publikationsfreiheit) \cite{Fangerau_2014} \cite[:190]{Fehling_2014} oder eben die Freiheit der Nicht-Veröffentlichung von Inhalten (negative Publikationsfreiheit).

Somit steht es allen an öffentlichen Forschungseinrichtungen tätigen Wissenschaftlern und Wissenschaftlerinnen frei, "zu entscheiden, ob und in welcher Form sie ihre dort erbrachten wissenschaftlichen Leistungen veröffentlichen" \cite{Schmidt_2009}. Auch die Wahl zwischen einer Veröffentlichung in einem kostenpflichtigen Journal oder in einem Open-Access-Journal fällt damit unter die positive Publikationsfreiheit \cite[:190]{Fehling_2014}. Diese Publikationsfreiheit
im Rahmen der individuellen Wissenschaftsfreiheit ist zwar im aktuellen System des wissenschaftlichen Austauschs nicht direkt gefährdet, wird aber durch indirekte Faktoren und Anreize stark eingeschränkt \cite{Binswanger_2014}. So fördert das System insbesondere die Publikationsformen und -kanäle, die von der wissenschaftlichen Gemeinschaft der jeweiligen Fachdisziplin als etabliert und als förderungsfähig betrachtet werden. Neue Formen und Kanäle hingegen werden nur selten im Rahmen der formellen Kommunikation berücksichtigt. Für sie ist es besonders schwer im bestehenden Reputationssystem Fuß zu fassen.

Wissenschaftliche Freiheit bezieht sich demnach auf der einen Seite auf die selbstbestimmte und unabhängige Wahl der Themen, Methodik, die freie Wahl Verbreitungs- und Publikationskanal sowie des Zeitpunkts und betrifft die Selbstorganisation bei der Durchführung und Steuerung der wissenschaftlichen Arbeit \cite[:190]{Fehling_2014}. Auf der anderen Seite beschreibt sie die Freiheit von inhaltlichen und methodischen Richtlinien und Vorgaben \cite[:140]{Goetting_2015}. Diese beiden Garantien beziehen sich in abgeleiteter Form auch auf die unterschiedlichen Organisationen und Institutionen von Wissenschaft. Wer "diese Freiheit der Wissenschaft beschneidet, behindert das Bemühen um Wahrheit und damit den Zweck der Wissenschaft selbst" \cite[:69]{Oezmen_2015}.

In Hinblick auf die wissenschaftliche Publikation kann festgehalten werden, dass Hochschullehrer nicht von der Hochschule oder anderen staatlichen Institutionen gezwungen werden können, über einen bestimmten Weg oder Kanal zu veröffentlichen \cite{Spindler_2006} \cite{Dorschel_2006}. Eine Ausnahme stellen nur die privatfinanzierten Drittmittelprojekte dar, da sich der Hochschullehrer hier nicht auf die Wissenschaftsfreiheit als Abwehrrecht gegen den Staat berufen kann \cite{Spindler_2006}. Wissenschaftliche Mitarbeiter und Mitarbeiterinnen "müssen ihrer Hochschule die Nutzungsrechte an ihrer Publikation einräumen", es sein denn, sie haben sie nicht nach Weisung des Lehrstuhl- oder Institutsleiters erarbeitet oder es handelt sich um eine Dissertation oder Habilitation \cite{Spindler_2006}. Ein direkter staatlicher Eingriff im Rahmen einer Richtlinie zum Publikationszwang über einen bestimmten Weg scheint demnach mit der Wissenschafts- und Publikationsfreiheit nicht vereinbar.

Dennoch kann der Staat Anreizsysteme oder Rahmenbedingungen schaffen, die die Öffnung des wissenschaftlichen Kommunikations- und Publikationssystems befördern. In der rechtlichen Auseinandersetzung mit dem Thema zielen die diskutierten Ansätze meist darauf ab, "den Autor eines öffentlich finanzierten wissenschaftlichen Werkes zu zwingen, die Allgemeinheit in gewissem Umfang an diesem partizipieren zu lassen und den Verlagen die Möglichkeit zu nehmen, durch einseitige Vertragsgestaltungen eine solche (kostenlose) Partizipation zu verhindern" \cite{Dorschel_2006}.

Im bestehenden System kann auch eine Art Nötigung zur Veröffentlichung auf dem tradierten Weg vermutet werden, die den Wissenschaftler und die Wissenschaftlerin indirekt in seiner/ihrer Freiheit einschränken, den Publikationsweg, den er oder sie für richtig halten, frei zu wählen. Die Forderung der International Association of STM Publishers, "Autoren sollten in einem gesunden, unverzerrten freien Markt frei wählen können, wo sie publizieren" \cite{Brussels_Declaration_2007}, zeigt deutlich diesen Bias in der Argumentation im Rahmen des bestehenden Systems.

Diese grundsätzliche Darstellung, dass die Wissenschaft als Prozess der Wissensbildung und Wissensvermittlung in Deutschland durch das Grundgesetz abgesichert ist, zeigt, dass Freiheit von Wissenschaft und Forschung eine Bedingung für die Wahrheitssuche der Wissenschaft ist \cite{Oezmen_2015}. Neben diesem rechtlichen Schutz sichern das wissenschaftliche Ethos und die Regeln des wissenschaftlichen Diskurses, auf die im Verlauf der Arbeit eingegangen wurde, die Autonomie und die Unabhängigkeit der Wissenschaft von politischen und gesellschaftlichen Interessenslagen \cite[:67]{Oezmen_2015}: "Politik gehört nicht in den Hörsaal" \cite[:494]{Weber_1992}. Weitere Anknüpfungspunkte zur Forderung nach Öffnung wissenschaftlicher Kommunikation im Spannungsfeld der Freiheit von Wissenschaft und Forschung stellen in diesem Zusammenhang die Dual-Use-Problematik und der Umgang mit Datenschutz dar \cite{Fritsch_2015}.

\subsection{Kosten und Effizienz}

An dem Kosten-Nutzen-Verhältnis des aktuellen wissenschaftlichen Kommunikationssystems und auch an dem praktizierten Peer-Review-Prozess \cite{Smith_2006} gibt es seit Jahren detaillierte und grundsätzliche Zweifel \cite{Brembs_2013a}. Für die Veröffentlichungen einzelner Texte ergeben sich je nach Schätzungen unterschiedlich hohe Kosten. Berechnungen des Wissenschaftsjournalisten Richard Van Noorden ergaben Kosten von 4.871 Dollar pro veröffentlichtem Text im tradierten Print- und Online-Subskriptionsmodell ohne freien Zugang, von 3.509 Dollar bei der reinen Online-Veröffentlichung im Subskriptionsmodell ohne freien Zugang und von 2.289 Dollar unter den Bedingungen von Open Access \cite{Van_Noorden_2013}. Wissenschaftliches Wissen kann für das wissenschaftliche System allerdings nur dann als umfassend effizient betrachtet werden, wenn das neue Wissen frei und offen für andere Forscher und Forscherinnen zur Verfügung steht. Im analogen System war dies aufgrund der Bindung des Wissens an das Speichermedium Druckerzeugnis nur durch hohe Kosten für die Erstellung, den Vertrieb, die Sicherung und Verbreitung möglich.

Mit Beginn der Verbreitung elektronischer Publikationen kam es zu einer Umkehr des Bring- zum Holprinzip bei der Verbreitung wissenschaftlicher Publikationen. Die Erwartungen an die neuen Kanäle richten sich vor allem darauf, mit elektronischen Publikationen die Publikations- und Vertriebszyklen kostengünstiger und effizienter zu machen \cite{Brueggemann-Klein_1995}. Die Vermutung Ende der 1990er Jahre: "Einsparungen in Zeit, Raum und Kosten werden erheblich sein, wenn zunehmend Schreib- und Publikationstätigkeiten in den elektronischen Raum verlegt werden" \cite{Roberts_1999}. Doch nach mehreren Dekaden der Verfügbarkeit dieser "elektronischen Räume" hat sich herausgestellt, dass es sich beim wissenschaftlichen Kommunikationssystem um ein "sozial ineffizientes" System \cite[:47]{Mueller-Langer_2010} handelt, bei dem die Publikations- und Vertriebszyklen weder kostengünstiger noch merklich effizienter geworden sind.

Obwohl die zunehmende Verbreitung digitaler Systeme im wissenschaftlichen Alltag die Möglichkeit eröffnet hat, nicht nur Publikationen schnell und umfassend zu veröffentlichen, sondern auch Daten und Informationen hinter Publikationen zu veröffentlichen, stehen Publikationen und Daten selten für die digitale Informationsversorgung offen für die Gesamtgesellschaft zur Verfügung. Dennoch wird eine Effizienzsteigerung durch die Möglichkeit der Zweitnutzung und Weiterverwendung von Daten, die während des wissenschaftlichen Erkenntnisprozesses entstehen, vermutet \cite{RIN_2010}.

Der restriktive und geschlossene Umgang mit Publikationen, Daten und wissenschaftlichen Informationen im aktuelle System verhindert nicht nur die wissenschaftsinterne, sonder auch die gesamtgesellschaftliche Nutzung der neuen Möglichkeiten für kollaborative Arbeit und den umfassenden Zugriff auf zusätzliche Forschungsergebnisse, bessere Bildung, neue Möglichkeiten und Nutzungsszenarien und eine umfassendere Aufzeichnung, Evaluation und Darstellung von Wissen.

Weder die Kosten für das System der wissenschaftlichen Kommunikation noch die Effizienz im Rahmen der Produktion von neuem Wissen aus bestehendem Wissen werden im gegenwärtigen Kommunikationspraktiken optimal genutzt. Die Auswirkungen dieser Ineffizienz führen zu einem erhöhtem (Zeit)Aufwand seitens der am Kommunikationssystem beteiligten Akteure und zur Verschwendung von Ressourcen \cite{Nosek_2012}.

\subsection{Fehlerresistenz und Qualitätssicherung}

Damit der Erkenntnisfortschritt im Kommunikationsprozess gelingt braucht es Verlässlichkeit bei der Vermeidung von Fehlern im wissenschaftlichen Erkenntnisprozess \cite{Bargheer_2015}. Trotz des aufwändigen wissenschaftlichen Qualitätssicherungssystems kommt es immer wieder zu Fehlern und falschen Aussagen bei der Veröffentlichung wissenschaftlicher Erkenntnisse und Ergebnisse \cite{Brembs_2015} \cite{Luescher_2014} \cite{Smith_2006}. Die Gründe für diese Fehler sind vielfältig und erstrecken sich von Nachlässigkeit über Fahrlässigkeit bis hin zu Vorsatz.

In der Literatur werden unter anderem folgende Faktoren als Herausforderungen für die Absicherung der Fehlerresistenz genannt:
\begin{enumerate}
\item Geschlossene Begutachtungsverfahren ermöglichen nur eine kleinen Anzahl an Gutachtern wissenschaftliche Inhalte auf Fehler zu prüfen \cite{Smith_2006}
\item Nichtverfügbare Methoden und Daten hinter den Publikationen behindern die Qualitätssicherung und Reproduzierbarkeit von Wissen \cite{Nosek_2015} \cite[:9]{Gruber_2005}
\item Nicht dokumentierte und veröffentlichte Kommunikation während des wissenschaftlichen Erkenntnisprozesses, macht es unmöglich Fehler bereits bei der Erstellung der Publikation sichtbar und transparent nachvollziehbar zu machen \cite{Nosek_2015}
\end{enumerate}

Die Fehlerresistenz des wissenschaftlichen Kommunikationssystems ist demnach durch seine Geschlossenheit beeinträchtigt. Hier gibt es einen weiteren Anknüpfungspunkt zu Open-Source-Bewegung im Rahmen der Softwareentwicklung, bei der die Öffnung des Quellcodes von Software eine Möglichkeit der Sicherung der gewünschten Funktionstüchtigkeit und Sicherheit darstellt \cite[:7]{Hoepman_2007}. Darüber hinaus werden durch die Öffnung auch langfristig die Fehler einseh, reproduzier- und nachverfolgbar \cite{Nosek_2015}, die durch Nachlässigkeit oder Fahrlässigkeit aber auch durch Vorsatz entstanden sind. Das ermöglicht eine bisher nicht mögliche Berücksichtigung durch andere Wissenschaftler und Wissenschaftlerinnen im kummulativen Prozess der Generierung von neuem Wissen und stellt einen neuen Ansatz zum Erkenntnisgewinnung dar, der im geschlossenen Kommunikationssystem nicht möglich ist.

Wenn die Quelldokumente und Daten auch zum Zeitpunkt der Erstellung offengelegt sind, können interessierten Akteure die Informationen auf Fehler testen und gegebenenfalls Fehler schnell und umfassend bereinigen \cite[:10]{Gruber_2005} \cite{Curry_2015}. Dadurch ist nicht nur eine Erhöhung der Qualität von wissenschaftlichen Inhalten sondern auch eine Erhhöhung der Fehlerresistenz bei Abschluss des jeweiligen wissenschaftlichen Erkenntnisprozesses zu erwarten.

\subsection{Verbreitung und Zugänglichkeit}

Ebenso, wie die Frage nach der optimalen Geschwindigkeit des aktuellen wissenschaftlichen Kommunikationssystems, stellt sich die Frage nach der optimalen Verbreitung und der möglichst freien Zugänglichkeit \cite[:10]{Gruber_2005} wissenschaftlicher Informationen. Während die Geschwindigkeit auf die zeitliche Komponente von der Herstellung bis zum Vertrieb des Wissens abzielt, geht es bei der Frage nach der Verbreitung um die Verfügbarkeit des Wissens für eine möglichst große Rezipientengruppe. Es gibt erhebliche Zweifel daran, dass es sich bei dem aktuellen System um ein System mit optimalen Voraussetzungen für eine möglichst hohe Verbreitung von neuem Wissen an die Gesamtgesellschaft \cite{Curry_2015} oder nur innerhalb einer bestimmten Gruppe handelt .

Noch heute ist das gedruckte Werk neben dem persönlichen Austausch auf Konferenzen oder Kongressen \cite{Winkler_2011} für die Wissenschaftler und Wissenschaftlerinnen eine der maßgeblichen Informationsquellen. Analoge Publikationen und Verbreitungswege sind allerdings beim ortsübergreifenden Austausch stark beschränkt. Und selbst bei der Verbreitung der Informationen die bereits digitalisiert worden sind, oder bereits bei Erstellung digital vorlagen werden sie im aktuellen System noch immer häufig durch Zugangsbarrieren wie Bezahlschranken gehemmt und damit die Zirkulation von Wissen eingeschränkt.

Die Herausforderung im aktuellen System besteht zum Einen aus der Bereitstellung der wissenschaftlichen Informationen über die unterschiedlichen Kommunikationskanäle hinweg und zum anderen in der langfristigen Sicherung und Bereitstellung dieser Informationen. Der digitale Transformationsprozess stellt in diesem Zusammenhang eine weitere Herausforderung und einen Ausweg zugleich dar, denn obwohl die Verarbeitung digitaler Daten heute ein wesentlicher Bestandteil der allermeisten wissenschaftlichen Vorhaben ist \cite{Winkler_2011}, müssen die Informationen meist auf dem gedruckten und digitalen Speichermedium vorgehalten werden. Auch die vornehmlich durch Verlage praktizierte reine Digitalisierung des analogen Subskriptionsmodells für den Zugriff auf wissenschaftliche Inhalte \cite{Hanekop_2014} \cite{BOAI_2012} stellt eine Barriere für den Zugang zu den Informationen auch außerhalb der wissenschaftlichen Institutionen dar, da digitalisiertes Wissen weiterhin auf den Ort des analogen Wissens beschränkt bleibt.

\subsection{Digitalisierung}

Wie im Kapitel "Wissenschaftliche Kommunikation" beschrieben, ist die Verarbeitung digitaler Daten heute ein wesentlicher Bestandteil der meisten wissenschaftlichen Vorhaben. Obwohl die wissenschaftliche Arbeit und die wissenschaftliche Kommunikation überwiegend an digitalen Geräten stattfinden, wird noch immer für den Druck produziert. Während Wissenschaftler und Wissenschaftlerinnen schon seit dem Ende des letzten Jahrhunderts überwiegend mit Hilfe von Textsystemen schreiben \cite{Brueggemann-Klein_1995} \cite{Bjoerk_2004} haben Verlage erst mit großer Verzögerung auf die elektronische Produktion von Wissen reagiert.

Auch die wissenschaftlichen Rohdaten und Informationen werden bei Abschluss des Erkenntnisprozesses (Publikation der Ergebnisse) umkodiert um analog publiziert zu werden und auch die rein digitalen Versionen der Publikationen entstehen überwiegend noch immer aus Informationen die für die analoge Publikation kodiert worden sind. In diesem Prozess kann ein Großteil der erzeugten Daten nicht weiter genutzt werden und viele der Informationen gehen verloren, beziehungsweise stehen nur selten für die Nachnutzung zur Verfügung.

Auch im Rahmen des Vertriebs beschränkt sich die Digitalisierung der wissenschaftlichen Kommunikation bisher in vielen Fällen noch immer darauf, dass die analog gedruckten und bewährten Journale, sowie andere Publikationsformen der großen wissenschaftlichen Verlage mit nahezu unverändertem Geschäftsmodell digital verbreitet werden \cite{Hanekop_2014} \cite[:179]{Fehling_2014}. Die digitale Distribution wird in diesem Zusammenhang als weiterer Kanal nach dem Drucken der Informationen verstanden.

Die Möglichkeiten, die die Digitalisierung für die wissenschaftliche Informationsversorgung bietet, sind damit bei Weitem nicht ausgeschöpft. Es stehen zwar zunehmend nicht nur digitalisierte Informationen ehemals analoger Veröffentlichungen orts- und zeitunabhängig zur Verfügung, sondern auch wissenschaftliche Sammlungen. Ebenso wird den Metadaten oder Digitalisaten relevanter Objekte ein großes Potenzial für die Wissenschaft zugesprochen \cite{Winkler_2011}. Die Anzahl dieser Daten ist aktuell jedoch noch stark begrenzt.

Die Herausforderungen im bestehenden System formeller wissenschaftlicher Kommunikation bezieht sich bei der Digitalisierung vor allem auf die ungenutzten Potenziale einer umfassenderen Verbreitung und Kommunikation wissenschaftlicher Erkenntnisse. Diese werden im aktuellen System bei der Veröffentlichung nur selten genutzt und das derzeitige System der wissenschaftlichen Veröffentlichungen arbeitet noch immer gegen die maximale Verbreitung der wissenschaftlicher Informationen und Daten hinter den eigentlichen Publikationen \cite{Molloy_2011}.

\subsection{Überprüfbarkeit der wissenschaftlichen Güte: Objektivität, Reliabilität und Validität}

Wissenschaftliches Wissen zeichnet sich gegenüber anderen Formen des Wissens dadurch aus dass es Prüfprozeduren gibt, mit denen das spezifisch wissenschaftliche Wissen geprüft wird \cite{Luhmann_1998}. Bisher wurde durch die formelle Publikation festgeschrieben, was nach den Kriterien des jeweiligen Fachs beziehungsweise der jeweiligen Disziplin als geprüftes Wissen gelten kann \cite[:11]{BBAW_2015}. Hier werden beispielhaft die Herausforderungen an die Prüfbarkeit der Gütekiterien Objektivität, Reliabilität und Validität im wissenschaftlichen  Kommunikationssystem dargestellt.

Unabhängigkeit (Objektivität) in der Wissenschaft gilt für die Sammlung, Aufzeichnung, Analyse, Interpretation, gemeinsame Nutzung und Speicherung von Daten, sowie andere wichtige Verfahren in der Wissenschaft, wie zum Beispiel die Veröffentlichungspraxis und Peer-Review \cite{Resnik_2005}. "Ohne Zorn und auch ohne persönliche Präferenzen sind die wissenschaftlichen Gegenstände sachlich und neutral zu behandeln" \cite[:9]{Gruber_2005}. Die Kenntnis von Eigenschaften der Autoren durch die Gutachter stellt eine der größten Herausforderungen für die Wahrung der Objektivität und Unabhängigkeit im wissenschaftlichen Qualitätssicherungsprozess dar. Aber auch bei anderen Formen der wissenschaftlichen Bewertung können Unabhängigkeit und Objektivität nicht immer uneingeschränkt gewährleistet werden. In der Literatur finden sich Beiträge, die mehrheitlich zu dem Ergebnis kommen, dass die Objektivität und Unabhängigkeit im bestehenden System nur schwer bis nicht gesichert werden können \cite{Binswanger_2014}.

Resnik beschreibt diesbezüglich folgende Herausforderungen an das bestehende geschlossenen System der wissenschaftlichen Kommunikation und an die Wahrung der Objektivität und an das selbstkorrigierende System der Wissenschaft \cite{Resnik_2005}:
\begin{enumerate}
\item Präzision der wissenschaftlichen Arbeit
\item Ehrlichkeit bei der Datenerhebung und Darstellung der Ergebnisse
\item Vermeidung von Fehlverhalten
\item Vermeidung von Fehlern und Selbsttäuschung
\item Offenlegung Interessenskonflikte
\item Offenheit bezüglich Daten, Ideen, Theorien und  Ergebnissen
\item bewusstes Datenmanagement und Dokumentation
\end{enumerate}

Die Zuverlässigkeit (Reliabilität) des Kommunikationssystems kann anhand dessen geprüft werden, ob die Einreichung einer Arbeit über unterschiedliche Wege den selben Erfolg hat beziehungsweise, wie stark Zufallsfaktoren den Erfolg der Veröffentlichung wissenschaftlicher Erkenntnisse beeinflussen. Hier bestehen im aktuellen System wenig Möglichkeiten der Überprüfung. Die Verbreitung der Informationstechnologien ermöglicht zwar ein umfassenderes Monitoring der Aktivitäten von Wissenschaftlerinnen und Wissenschaftlern, eindeutige Sicherheit kann jedoch nicht gewährleistet werden.

Im Gegenteil, die umfassende Replizierbarkeit und Zuverlässigkeit von Ergebnissen kann aktuell kritisiert und angezweifelt werden \cite{Luescher_2014}. Das liegt zum einen an den Herausforderungen im Zusammenhang mit der meist nicht praktizierten Veröffentlichung von (Roh-)Daten, zum anderen an der Verwendung von geschlossenen Systemen und Formaten sowie fehlender Transparenz im Rahmen der genutzten Methoden und Verfahren. Die Transparenz muss dabei nicht zwangsläufig ein Widerspruch zur Notwendigkeit von Unabhängigkeit und Objektivität verstanden werden, da offene Verfahren auch anonym stattfinden können. Als weitere kritische Faktoren für die Wahrung der Zuverlässigkeit im Kommunikationssystem werden in der Literatur unter anderem Lücken im Qualitätssicherungsprozess (siehe auch "Fehlerresistenz") \cite{Bar_2009} und der zunehmende zeitliche Druck im Rahmen der Qualitätssicherung \cite{Luescher_2014} genannt.

Die Herausforderungen an die Überprüfbarkeit der Gültigkeit (Validität) der für den Druck bestimmten wissenschaftlichen Arbeiten und deren Ergebnisse schließen nahtlos an die anderen genannten Kriterien der Güte an. Im Unterschied zur Zuverlässigkeit ermöglicht die Überprüfung der Validität die Eignung der eingesetzten Meßverfahren zur Beantwortung der wissenschaftlichen Fragestellungen und Zielsetzungen. Auch hier sind die Möglichkeiten bei gedruckten Publikationen durch den fehlenden Zugang zu Daten und Informationen, die während des wissenschaftlichen Erkenntnisprozess entstehen bisher eingeschränkt.

\subsection{Verhinderung von Missbrauch und wissenschaftliches Fehlverhalten}

Neben der Notwendigkeit für eine umfassenden Überprüfbarkeit des Wissens, stellen die ethischen Grundsätze in der wissenschaftlichen Debatte von Beginn an eine Besonderheit dar. Vertrauen, das Interesse aller Akteure an optimaler Kommunikation zwischen den Wissenschaftlern, Ehrlichkeit und der Ausschluss von Interessenskonflikten sind Grundpfeiler im wissenschaftlichen Forschungs- und Kommunikationsprozess \cite{Bargheer_2015} \cite{Wissenschaftsrat_2015}. "Betrug ist dabei zwingend an die Absicht zu täuschen gebunden" \cite{Luescher_2014}.

Es muss das Anliegen jedes Forschers sein, "die Wahrheit und nichts als die Wahrheit zu suchen und zu berichten" \cite{Luescher_2014}. Drüber hinaus gilt: "Ohne Vertrauen in die Ehrlichkeit von Forschern gäbe es keine Wissenschaft mehr" \cite[:18]{Hagner_2015}. Vertrauen und Redlichkeit bilden die Grundlage der Wissenschaft \cite{Bargheer_2015} auch wenn diese auf einer "delikaten Struktur weitgehend ungeschriebenen Regeln" \cite{Grand_2012} beruhen.

Auch wenn die Wissenschaft "eine besondere ethische Verantwortung" hat, sind Formen von "Fehlverhalten, Betrugsfälle und Nachlässigkeiten, die in anderen Lebensbereichen geschehen können, auch in der Wissenschaft möglich" \cite{Wissenschaftsrat_2015}. Diesem wissenschaftlichem Ethos stehen die Beispiele gegenüber, bei denen bewusster Missbrauch durch Akteure des Kommunikationssystems zu Verwirklichung partikularer Interessen oder konkreten Einfluss auf wirtschaftliche Aspekte geführt haben \cite{Luescher_2014} \cite{Binswanger_2014} \cite{Beall_2012}.

Margo Bargheer und Birgit Schmidt klassifizieren wissenschaftliches Fehlverhalten wie folgt \cite{Bargheer_2015}:
\begin{enumerate}
\item Unlauterer Umgang mit Ergebnissen (z.B. erfundene Ergebnisse)
\item Unlauteres Forschungsverhalten (z.B. Unzulässige Forschungsmethoden)
\item Fehlverhalten im Datenmanagement (z.B. Zurückhalten von Daten wider besseres Wissen )
\item Fehlverhalten im Publikationsprozess (z.B. unangemessene Partitionierung von Ergebnissen, "Salamitaktik" \cite{Binswanger_2014})
\item Soziales Fehlverhalten (z.B. Sabotage oder Behinderung der Arbeit anderer)
\item Administratives Fehlverhalten (z.B. Verstoß gegen Verwendungsrichtlinien)
\end{enumerate}

Gegen ein solches Fehlverhalten im Rahmen der wissenschaftlichen Kommunikation wurden die internationalen Leitlinien "Principles of Transparency and Best Practice in Scholarly Publishing" \cite{Redhead_2013} veröffentlicht, "sie sollen die Qualitätsstandards im Publikationswesen und zugleich die Filterfunktion der initiierenden Mitgliedsorganisationen stärken" \cite{Bargheer_2015}. Bisher kommen die wenigen vorhandenen Studien zu dem Ergebnis, dass abgelehnte Manuskripte, sofern sie andernorts veröffentlicht wurden, deutlich weniger zitiert wurden \cite[:208]{Hornbostel_1997}. Mit Blick auf die neuen Möglichkeiten der ergänzenden Veröffentlichung von Meta-Informationen und Daten zusätzlich zur finalen Publikation ist zu vermuten, dass die Möglichkeiten zur Sicherung der Qualität im bestehenden System optimiert werden können, so zum Beispiel im Bereich der Replizierbarkeit von wissenschaftlichen Ergebnissen. Hier bestünde durch eine offene, möglichst umfassende Bereitstellung der wissenschaftlichen Kommunikation viel Potenzial für die Verbesserung der Mechanismen zur Selbstkorrektur \cite{Nosek_2015} sowie für die Verhinderung von Missbrauch und wissenschaftlichem Fehlverhalten.

Auch wenn noch nie zuvor über Betrug in der Wissenschaft so intensiv berichtet worden ist \cite{Brembs_2015} wie in den letzten Jahren, ist es "keineswegs ausgemacht, dass die Intensität der Berichterstattung allein auf die tatsächlich gestiegene Inzidenz von Betrug" \cite{Weingart_2005}, sondern eher auf den Anstieg medialer Beobachtung zurückzuführen ist. Dennoch stehen die intransparenten Verfahren und die bisher mangelhafte Veröffentlichung von Supplementen und (Roh-)Daten der Verhinderung von Missbrauch und wissenschaftlichem Fehlverhalten entgegen. Demnach ist zu vermuten, dass die Bereitschaft der Forschenden, positive wie negative Daten zu teilen, zurückgezogene Artikel sichtbar zu machen und den wissenschaftlichen Erkenntnisprozess zu öffnen, helfen können, die notwendigen effektiven Mechanismen zur Verfolgung wissenschaftlichen Fehlverhaltens \cite[:14]{Wissenschaftsrat_2015} \cite{Chan_2015} \cite[:86]{Chalmers_2009} zu installieren und die bestehenden Mechanismen zur Selbstkorrektur zu stärken.

\section{Ableitungen: Katalysatoren und Hindernisse für die Öffnung wissenschaftlicher Kommunikation}

Viele der unterschiedlichen Erklärungsansätze für die Forderung nach einem Wandel der wissenschaftlichen Kommunikation hin zur Öffnung der Wissenschaft gehen von Annahmen aus, bei denen ein direkter Zusammenhang von technischen Entwicklungen und (wissenschafts-)politischen und kulturellen Bewegungen angenommen wird. Diese Perspektive ist in ihren Wegen und Kanälen sehr fragmentiert und beschränkt sich in ihrer Klarheit bisher ausschließlich auf das gemeinsame Ziel, den Zugang zu wissenschaftlichen Ergebnissen offener zu gestalten und weniger auf die Öffnung des gesamten Prozesses sowie den daraus resultierenden Konsequenzen für das gesamte Wissenschaftssystem zu achten.

Die theoretische Auseinandersetzung zwischen der Geschlossenheit des wissenschaftlichen Diskurses auf der einen und den Treibern und Bremsern im realen wissenschaftlichen Prozess auf der anderen Seite wird in der Literatur bisher nur ungenügend berücksichtigt. Insbesondere wird die Verbindung zwischen wissenschaftlicher Reputation, der Motivation, das etablierte System zu unterstützen, und der Geschlossenheit des Wissensproduktionsprozesses nur selten erörtert. Die Debatten über die Veränderungen des wissenschaftlichen Publikationswesens werden von beiden Seiten mit teilweise "heftiger Polemik" \cite[:12]{Naeder_2010} geführt und bedienen sich bei den unterschiedlichsten Ansätzen von Stevan Harnad \cite{Harnad_1995} über die von Richard Stallman \cite{Stallman_2002} bis hin zu denen von Roland Reuß \cite{Reuss_2009}. Eine weitere Unzukänglichkeit besteht darin, dass "die Deliberation und die Verbreitung von Wissen ein stabiles Set von Infrastrukturen braucht" \cite{Kelty_2004}, nach denen man heute noch immer vergeblich sucht. Das Potenzial bei der Verwendung digitaler Technologien und der Wille, Wissenschaft offen zu teilen, ist nicht annährend ausgeschöpft und es "besteht eine erhebliche Diskrepanz zwischen der Idee der offenen Wissenschaft und wissenschaftliche Realität" \cite{Scheliga_2014}. Dabei ist die (geistes-)wissenschaftliche Alltagspraxis "längst von digitalen Recherche- und Kommunikationsformen durchsetzt" \cite{Hagner_2015}.

Openness kann als "schwimmender Signifikant (...) ohne eindeutige Definition, adaptierbar von unterschiedlichen politischen Ideologien" verstanden werden \cite{Adema_2014}. Der Begriff Open Access wird in der neoliberalen Rhetorik als effizientes Wettbewerbsmodell, verbunden mit den Ideen von Transparenz und Effizienz von Unternehmen und Regierung, eingesetzt \cite{Tkacz_2012}. Darüber hinaus muss die Öffnung von wissenschaftlicher Kommunikation auch im Rahmen des Versuchs betrachtet werden, einen Martkmodus als dominante Governanceform der Gesellschaft auch in der Wissenschaft zu verankern \cite[:152]{Troy_2012}. Über diesen Ansatz wird mittels Openness der wissenschaftlichen Prozess outputorientierter und seine Ergebnisse effektiver zu Gunsten des Marktes gestaltet, überwacht und gesteuert \cite{Adema_2010}. Dabei stehen diese neoliberalen Ansätze den Ideealen der Öffnung des gesamten wissenschaftlichen Prozesses gegenüber, "denn die Position funktioniert nur dann ökonomisch effizient, wenn innovatives technisches Wissen nicht nur patentrechtlich sondern auch marktmäßig gehandelt wird" \cite[:179]{Troy_2012}.

Diese Entwicklung bedroht zudem das System der Universität als Produzent, Archivar und bei der Distribution von Wissen. Die Öffnung von Wissenschaft und Forschung kann demnach als Möglichkeit dafür genutzt werden, dass die Universität selbst wieder zu dem (primären) Ort der Wissensproduktion, -speicherung und -vermittlung wird, der sie mal gewesen ist \cite{Kittler_2004}. Um diese Veränderungen voranzutreiben, werden in der Literatur zwei Herangehensweisen für die Etablierung von Offenheit in Wissenschaft und Forschung unterschieden \cite{Schulze_2013}:
\begin{enumerate}
\item "Top-down durch Förderstrategien, Vorgaben und Empfehlungen": Hierbei können durch die Bereitstellung zusätzlicher Mittel im Rahmen der Forschungsförderung konkrete Anreize für die offene Veröffentlichung und die Publikation von Forschungsergebnissen geschaffen werden. Eine weitere Möglichkeit der "Top-Down"-Etablierung von Offenheit in Wissenschaft und Forschung stellen Empfehlungen dar, bei denen Institutionen, Organisationen oder Gruppen nicht bindende Empfehlungen aussprechen, anhand derer Wissenschaftler und Wissenschaftlerinnen überzeugt werden sollen, ihre wissenschaftlichen Ergebnisse offen zu veröffentlichen. Sind weder Anreize noch Empfehlungen als Top-Down-Ansatz erfolgreich, können bindende Vorgaben etabliert werden, um eine Verhaltensänderung der Wissenschaftler und Wissenschaftlerinnen zu erzwingen.
\item "Bottom-up durch Graswurzelprojekte und den Einsatz von Evangelisten":
Im Gegensatz zur Strategie von "oben" gibt es auch Bestrebungen, die von einzelnen Wissenschaftlern, Wissenschaftlerinnen oder Gruppen initiiert werden. Sie sind überwiegend informell und zielen auf die Verbreitung von Verhaltensänderungen oder die Etablierung von Richtlinien ab. Bottum-up-Projekte kommen aus dem wissenschaftlichen Alltag und erfahren überwiegend keine politische oder monetäre Incentivierung für die Öffnung von Wissenschaft und Forschung. Der Einsatz von Evangelisten basiert auf der Idee einer konkreten Stelle oder Position, um eine Änderung zu begleiten oder einen Multiplikator innerhalb und außerhalb von Institutionen oder Organisationen zu etablieren, der das gewünschte Ziel pro aktiv kommuniziert und verbreitet. Evangelisten können helfen, die Befindlichkeiten und Vorbehalte auszutarieren und die teils diffusen, teils realen Ängste bezüglich der Entwicklung von Offenheit und Transparenz der Wissenschaft innerhalb und außerhalb der wissenschaftlichen Gemeinschaft zu beseitigen.
\end{enumerate}

Ergänzend dazu sehen die Rechtwissenschaftler Götting und Lauber-Rönsberg vier konkrete, rechtliche und faktische Maßnahmen zur Förderung der Öffnung wissenschaftlicher Kommunikation \cite[:138]{Goetting_2015}:
\begin{enumerate}
\item Verpflichtungen durch das Hochschulrecht, zum Beispiel eine rechtliche Verpflichtung steuerfinanzierte wissenschaftliche Werke unter einer offenen Lizenz zu veröffentlichen
\item Maßnahmen der Hochschulen, zum Beispiel durch institutionelle Selbstverpflichtungen oder finanzielle und andere faktische Anreizsysteme
\item Maßnahmen der öffentlichen Forschungsförderung, zum Beispiel Verpflichtung im Rahmen der Drittmittelfinanzierung von Forschungsvorhaben oder direkte Förderungsinstrumente für den Aufbau oder die Refinanzierung offener Publikationen
\item Urheberrechtliche Maßnahmen, zum Beispiel Vorhaben steuerfinanzierte wissenschaftliche Werke vom urheberrechtlichen Schutz auszunehmen oder Schrankenregelungen beziehungsweise Zwangslizenzen für öffentlich-finanzierte Werke einzuführen
\end{enumerate}

Im Folgenden werden die Katalysatoren und die Hindernisse für die Etablierung der Öffnung wissenschaftlicher Kommunikation den Indikatoren für die Reputationsverteilung im aktuellen wissenschaftlichen System gegenübergestellt. Diese Ausarbeitung zielt auf die Beantwortung der Forschungsfragen ab und stellt eine Grundlage für die darauffolgende Befragung der wissenschaftlichen Akteure im Publikations- und Kommunikationssystem dar.

\subsection{Katalysatoren für die Öffnung wissenschaftlicher Kommunikation}

In den analysierten wissenschaftlichen Beiträgen zu Open Access und Open Science wurden die mehrheitlich positiven Auswirkungen der Forderungen nach Offenheit im wissenschaftlichen Kommunikationssystem, aber auch antizipierte negative Effekte der Öffnung auf das wissenschaftliche Kommunikationssystem dargestellt. Grundlage für die Darstellung der Vorteile war die umfassende Erarbeitung der Herausforderungen und Unzulänglichkeiten im bestehenden wissenschaftlichen Kommunikationssystem \cite{Herb_2012a}.

Grundsätzlich steht und fällt der Erfolg bei der Etablierung von Verhaltensänderungen damit, ob sich der jeweiligen Zielgruppe ein unmittelbarer Mehrwert und Nutzen erschließen wird \cite{Schulze_2013}. Bisher scheint dieser eher gering zu sein, denn rechtlich steht es bereits nach der heutigen Rechtslage Wissenschaftlerinnen und Wissenschaftlern frei, "sich für eine Erstveröffentlichung ihrer Werke im Wege des Open Access zu entscheiden" \cite[:146]{Goetting_2015}, auch wenn konkrete Möglichkeiten und Grenzen von Open-Access-Publikationsverpflichtungen noch immer wesentlich durch die urheberrechtlichen Rahmenbedingungen beeinflusst werden \cite[:211]{Fehling_2014}.

Für die weiterführende Gruppierung der Argumente für die Öffnung von Wissen wurde die folgende Kategorisierung vorgenommen. Sie beschreibt die grundlegenden Katalysatoren und Argumente für die Öffnung des wissenschaftlichen Kommunikationssystems:
\begin{enumerate}
\item \textbf{Transition}: Die Nutzung der neuen Möglichkeiten für eine offene Wissensverbreitung neben den konventionellen Wegen der nicht-elektronischen Publikationen \cite{Hall_2008} \cite{Berliner_Erklaerung_2003}. Voraussetzung ist die Aufbereitung des Wissens als strukturierte Daten zur Wissensweiterverwendung und -verarbeitung über alle Kanäle.
\item \textbf{Speed and Circulation}: Offene Publikationsverfahren bieten die Chance wissenschaftliche Inhalte schneller und umfassender der wissenschaftlichen Community zur Verfügung zu stellen \cite{Mueller_2010} \cite{RIN_2010} \cite{Hall_2008} \cite{European_Commission_2006}. Wenn das Wissen schneller zur Verfügung steht, kann es auch schneller zirkulieren und effizienter genutzt werden \cite{Woelfle_2011}. In den tradierten Verfahren wird die Wissensverbreitung künstlich durch Embargos und ineffiziente Validierungs- und Qualitätssicherungssysteme zurückgehalten. Die Digitalisierung und Verbreitung über elektronische Kanäle stellt einen Vorteil für die Wissensverbreitung und -verwertung dar. Eine offene Veröffentlichung erreicht potenziell eine größere Leserschaft als es bei Subskriptionsmodellen der Fall ist \cite{Cope_2014}.
\item \textbf{Higher Impact and Citation}: Die uneingeschränkte und globale Verfügbarkeit der offenen wissenschaftlichen Informationen führt zu einem wesentlich höheren Verbreitungsgrad und Einfluss von Wissenschaft \cite{Davis_2011} \cite{Mueller_2010} \cite{Baggs_2006} \cite{Willinsky_2006} \cite{Kurtz_2005}. Der Verbreitungsgrad kann einen positiven Einfluss auf die Zitierhäufigkeit haben \cite{Mueller_2010} \cite{European_Commission_2006} \cite{Hajjem_2005}. Die Zitationsrate wissenschaftlicher Publikationen, die nach den Kriterien von Offenheit veröffentlicht werden ist damit potenziell höher \cite{Bernius_2009}. Diese Kausalität wird "access-citation effect"\cite{Davis_2011} genannt und ist durch bedeutsame Untersuchungen bestätigt worden \cite{Lawrence_2001} \cite{Jeffrey_2008} \cite{Hajjem_2005} \cite{Eysenbach_2006} \cite{Antelman_2004}. Dennoch gibt es Gründe diesen Effekt genau zu hinterfragen und im Detail mögliche Abschwächungseffekte zu berücksichtigen \cite{Davis_2011}.
\item \textbf{Tax-Payer}: Die Kosten des traditionellen Publikationsverfahrens werden im Wesentlichen durch die öffentliche Hand getragen \cite{Mueller_2010}. Dem Steuerzahler ist die konventionelle wissenschaftliche Kommunikation jedoch nur selten unentgeltlich zugänglich, obwohl er de facto im Rahmen öffentlich geförderter Forschungsprogramme die Forschung bereits (mit-)finanziert hat \cite{Suber_2003b} \cite{Resnik_2005} \cite{Baggs_2006} \cite{Woelfle_2011} \cite{Beverungen_2012} \cite{Adema_2014}. Da die Mittel nach intransparenten Kriterien verteilt werden ist im aktuellen Kommunikationssystem unklar, ob wissenschaftliche Kommunikation nach dem bestmöglichen Einsatz der monetären Ressourcen für Wissenschaft und Forschung abläuft \cite{Glasziou_2014} \cite{Altman_1994}. Die Europäische Union und die Organisation für wirtschaftliche Zusammenarbeit und Entwicklung (OECD) kommen in diesem Zusammenhang zu dem Ergebnis, dass der volkswirtschaftliche Nutzen von Open Access die Kosten signifikant übersteigen wird \cite{Cloes_2009} \cite{OECD_2015} \cite{eu_council_2007}.
\item \textbf{Economic Promotion}: Bisher profitieren Wirtschaftsunternehmen nur unzureichend von staatlich finanzierter wissenschaftlicher Kommunikation. Eine schnellere, kommerziell verwertbare und umfassendere Bereitstellung wissenschaftlicher Inhalte kann einen Beitrag zur non-monetären Wirtschaftsförderung und Innovation leisten \cite{European_Commission_2015a} \cite{OECD_2015} \cite{Heise_2012b} \cite{OECD_2004}. Im Rahmen der offenen und schnelleren Verbreitung wissenschaftlicher Informationen sind darüberhinaus auch neue Geschäftsmodelle denkbar.
\item \textbf{Digital Divide}: Der offene Zugang zu Wissenschaft eröffnet neue Chancen sowohl für die Überwindung sozialer, nationaler und globaler Wissenskluften, als auch zwischen bildungsferneren und -affineren Bevölkerungsteilen und -schichten der Welt \cite{BOAI_2012}. Darüber hinaus ist der Mehrwert und die Chance von wissenschaftlichen Informationen für die schulische Bildung und für die Bewegung der offenen Bildungsmaterialien bisher ebenfalls noch nicht vollumfänglich ausgeschöpft \cite{Heise_2013b}.
\item \textbf{Validation, Quality and Reputation}: Offenheit in Wissenschaft und Forschung ermöglicht die Entwicklung neuer Verfahren, die die Aktivität und Qualität eines Forschers oder einer Forscherin umfassender, transparenter und demokratischer messbar und kommunizierbar machen, als es im bestehenden Reputations- und Förderungssystem möglich ist \cite{Grand_2012}. \cite{Chalmers_2009}. Da Wissenschaft "per Definition die Bemühung um integre Information ist" \cite{Umstaetter_2007} wird vermutet dass Wissenschaftsevaluation durch den offenen Zugang und die daraus resultierenden Möglichkeiten der Verifizierung von Wissen effizienter wird \cite{Nosek_2015}. Die Falsifikation ist nur dann umfassend und einfach möglich, wenn der Aufwand für die Falsifikation gering beziehungsweise der Zugriff auf die wissenschaftlichen Informationen überhaupt gegeben \cite{Umstaetter_2007} und offen ist \cite{Peters_2014}. "Offenheit verhindert, dass Wissenschaft dogmatisch, unkritisch und voreingenommen wird" \cite{Resnik_2005}.
\item \textbf{Information Paradox}: Überwindung des bestehenden Informationsparadoxons bei der Verbreitung und Vermarktung wissenschaftlicher Inhalte. Hierbei handelt es sich um die Herausforderung im Rahmen kommerzieller Be- und Verwertung wissenschaftlicher Informationen, ohne zu viel über Inhalt und Qualität auszusagen. Eine im Rahmen von Offenheit angestrebte Entkommerzialisierung des Zugangs zu Wissen würde dieses Informationsparadoxon aufheben.
\item \textbf{Science Communication Crisis}: Durch die Öffnung wissenschaftlicher Kommunikations- und Reputationsprozesse entsteht die Möglichkeit, der vorherrschenden Zeitschriften- und Monografienkrise durch neue Geschäftsmodelle zu begegnen \cite{Mueller_2010} \cite{Naeder_2010}.
\item \textbf{Interdiscipline and International Exchange/Collaboration}: Die Globalisierung führt auch in der Wissenschaft zunehmend zu internationalem Austausch und zur transnationalen Zusammenarbeit von Wissenschaftlern \cite{Waltman_2011}. Das gilt nicht nur für die grenzüberschreitende Zusammenarbeit in Bezug auf die lokale Verortung, sondern auch für die Interdisziplinarität der Forschungsvorhaben. Die Öffnung der Wissenschaft ermöglicht auch fachfremden Wissenschaftlern Zugriff auf Publikationen und damit auf Wissensressourcen für die eigenen Arbeiten.
\item \textbf{Sustainable Access and Archiving}: Nur Offenheit im Sinne von Verwertbarkeit ermöglicht es, in dezentralen Strukturen wie der des Internets alle Informationen nachhaltig und unabhängig voneinander zu speichern. Im Falle von Natur- oder anderen Katastrophen ermöglicht die digitale Ablage auf mehreren Kontinenten eine Präservierung von Wissen unabhängig von lokalen Gegebenheiten oder Bedingungen.
\item \textbf{Dataquality}: Die Veröffentlichung und das offene Teilen der Daten hinter den wissenschaftlichen Publikationen kann zu einer umfassenden Erhöhung der Datenqualität und -integrität von wissenschaftlichen Erkenntnissen führen. Es wird vermutet, dass bei der Weiterverwendung durch Dritte mögliche Fehler schneller identifiziert werden und die offene Bereitstellung zu mehr Disziplin bei der Dokumentation der Datenbereitsteller führt. Ähnliche Erfahrungen wurden bereits im Bereich der Veröffentlichung von Daten der Verwaltung und bei der Entwicklungszusammenarbeit gemacht \cite{Heise_2014}.
\end{enumerate}

\subsection{Hindernisse für die Öffnung wissenschaftlicher Kommunikation}

Differenzierte Ansätze für den Umgang mit den Fragestellungen rund um die Öffnungsprozesse von Wissenschaft und Forschung sind wichtig, um einen "weniger ideologisch-aufgeregten Umgang mit dem Sujet" \cite[:13]{Naeder_2010} bei der Ausarbeitung der Arbeit zu erreichen. Im Folgenden werden die Prozesse dargestellt, die entweder zu einer Verlangsamung der Entwicklung führen oder sie in einigen Teilbereichen sogar ganz zum Erliegen bringen können. Dabei soll explizit keine Position für oder gegen die Veränderung des bestehenden Publikationssystems bezogen werden.

Grundsätzlich lassen sich bei der Öffnung der wissenschaftlichen Kommunikation strukturelle Hindernisse und individuelle Hindernisse unterscheiden \cite{Scheliga_2014}. Strukturelle Hindernisse beziehen sich dabei auf generelle Herausforderungen bei der Etablierung einer Verhaltensänderung im Rahmen der wissenschaftlichen Kommunikation. Dazu gehören zum Beispiel:
\begin{itemize}
\item Fehlende Anreizsysteme für Wissenschaftler und Wissenschaftlerinnen auf regionaler, nationaler und internationaler Ebene
\item Führungs- und Planlosigkeit der Bewegung für Offenheit in Wissenschaft und Forschung
\item Mangelhafte Infrastrukturen und nicht-disponible Applikationen für die Durchführung offener wissenschaftlicher Kommunikation
\end{itemize}

Der Fokus dieser Arbeit liegt auf den wissenschaftlichen Akteuren des Kommunikationssystems. Im Folgenden werden, auch wenn es sich lohnt, das Augenmerk auf "diejenigen Vorteile zu legen, von denen Wissenschaftler selbst profitieren können" \cite{Mueller_2010}, die individuellen Hindernisse für die Öffnung von Wissenschaft und Forschung betrachtet. Folgende individuelle Hindernisse bei und Argumente gegen die Öffnung der wissenschaftlichen Prozesse und Publikationen wurden identifiziert:
\begin{enumerate}
\item \textbf{Quality}: Der erste Hindernisbereich umschreibt die Befürchtung, dass die Qualität von offener wissenschaftlicher Kommunikation unter schlechter oder nicht vorhandener wissenschaftlicher Überprüfungsmechanismen leidet \cite{Chibnik_2015} \cite{Beall_2012}. Dabei wird argumentiert, dass ein durch Autorengebühren finanziertes Publikationsmodell keinen klaren Anreiz für Ablehnung bieten könnte \cite[:257]{Jubb_2011}.
\item \textbf{Renommee}: Die Möglichkeit zur Erlangung von wissenschaftlicher Reputation ist ein grundlegender Motivationsfaktor für Wissenschaftler und Wissenschaftlerinnen, die Ergebnisse ihrer Arbeit zu veröffentlichen. Eine Veröffentlichung hat nur dann Einfluss auf die Reputation, wenn sie im Rahmen von renommierten Publikationskanälen stattfindet. Offene Publikationsplattformen und Journale können aufgrund des kurzen Zeitraums ihres Bestehens und aufgrund von Vorbehalten dieses Renommee nur selten vorweisen. Die Renommeefrage stellt eine der größten Hürden für die offene wissenschaftliche Kommunikation dar \cite{Weishaupt_2009} \cite{Woelfle_2011}.
\item \textbf{Archiving- and Sustainability}: Den grundsätzlichen Vorteilen des elektronischen Publizierens stehen Probleme und Zweifel an der langfristigen Verfügbarkeit und Langzeitarchivierung \cite{Weishaupt_2009} gegenüber. Einige Autoren und Autorinnen kritisieren, dass die Sicherstellung der Langzeitarchivierung und die langfristige Auffindbarkeit sowie die Bereitstellung der Dokumente bisher nicht vollumfänglich durch digitale Strukturen gewährleistet werden kann \cite{Umstaetter_2007} \cite{Gersmann_2007}.
\item \textbf{Authenticity- or Integrity}: Ein weiteres Problem stellt die Sicherung der Authentizität der offen publizierten wissenschaftlichen Informationen dar \cite{Umstaetter_2007} \cite{Weishaupt_2009} \cite{Grand_2012}. Weil elektronische Dokumente oft innerhalb weniger Tage oder Wochen in mehreren Versionen zugänglich sind, wird befürchtet, dass Texte und Arbeiten im Zeitablauf inhaltlich nicht mehr unverändert ihrem Autor beziehungsweise ihrer Autorin zuzuordnen sind. Das gilt, "solange sie nicht in Digitalen Bibliotheken mit gesicherter Authentizität abgeliefert" werden \cite{Umstaetter_2007}.
\item \textbf{Rightsmanagement} Eine generelle Verpflichtung für Mitarbeiter staatlich finanzierter Forschungsinstitutionen, alle Texte und Daten elektronisch frei und offen zu publizieren, wird von einigen Autoren und Autorinnen kritisch hinterfragt \cite{Peukert_2013}. In dem 2009 veröffentlichten "Heidelberger Appell" \cite{Heidelberger_Appell_2009} kritisieren zahlreiche Autoren, Wissenschaftler, Verleger und Publizisten, dass das "verfassungsmäßig verbürgte Grundrecht von Urhebern auf freie und selbstbestimmte Publikation" ... "derzeit massiven Angriffen ausgesetzt und nachhaltig bedroht" ist. Weiter sehen die Unterzeichner "weitreichende Eingriffe in die Presse- und Publikationsfreiheit, deren Folgen grundgesetzwidrig wären" \cite{ITK_2009}. Rechtliche Bedenken und die Befürchtung vor kostspieligen juristischen Fehltritten stellen einen weiteren Vorbehalt gegen die offene Veröffentlichung von Forschung und Forschungsergebnissen dar \cite{Weishaupt_2009}.
\item \textbf{(Re-)Financing}: Die unklare Refinanzierung der Kosten, die im Rahmen der offenen wissenschaftlichen Kommunikation vermutet wird, wird als weiteres Kernargument gegen das offene Publizieren von Arbeiten und Daten angeführt \cite{Chibnik_2015}. Die Befürchtung ist, dass die umfassende Öffnung des wissenschaftlichen Systems überhaupt nicht finanziert werden kann, konnte bisher nicht ganz ausgeräumt werden \cite{Weishaupt_2009}.
\item \textbf{Ressource-Allocation}: Von der fachlichen Anerkennung hängen auch der Zugang zu Forschungsressourcen ab \cite[:14]{Buss_2001}. Dieses Hindernis bezieht sich demnach auf die Annahme, dass den Herausforderungen bei der Vergabe von Fördermitteln und bei den reputationsbildenden Maßnahmen im offenen System nicht ausreichend Rechnung getragen werden kann. Das Argument beruht auf der Befürchtung, dass die Öffnung des wissenschaftlichen Prozesses einen einseitig-negativen Einfluss auf Mittel- und Reputationsvergabe hat \cite{Grand_2012}, sie ausschließlich zugunsten populärer Forschung stattfindet und sie zu einer Aushöhlung der wissenschaftlichen Fächer- und Facettenvielfalt führt.
\item \textbf{Open-Caring}: Wissenschaftler und Wissenschaftlerinnen befürchten durch den Zwang zur umfassenden Bereitstellung ihrer Publikationen und gegebenenfalls sogar der Quelldaten sowie des genutzten Softwarecodes einen nicht unwesentlichen zeitlichen und finanziellen Mehraufwand \cite[:27]{BBAW_2015} \cite{Mennes_2013} \cite{Grand_2012}. Der nötige Aufwand, den die umfassende Öffnung der wissenschaftlichen Daten im Alltag des Wissenschaftlers mit sich bringen würde, ist bisher kaum evaluiert \cite{Osterloh_2008}.
\item \textbf{Scientific-Freedom/Loss of Idea-Diversity}: Dieses Argument betrifft zwei Ebenen: Die Sorge, dass durch Offenheit und Transparenz sowie Forschungsförderung und Öffentlichkeit die bestehenden Steuerungsmechanismen der Wissenschaft ausgehebelt werden und infolgedessen nur die wissenschaftlichen Projekte gefördert und unterstützt werden, die von der Allgemeinheit verstanden werden. Diese Befürchtung ruht auf der Annahme, dass die Gewinnung von Wissen in der Grundlagenforschung ein "öffentliches Gut" darstellt, "dessen Wert von der Öffentlichkeit nur schwer beurteilt werden kann" \cite{Osterloh_2008}. Darüber hinaus wird in der Literatur die Befürchtung geäußert, dass durch die Öffnung die Freiheit von Forschung und Lehre im Sinne der Publikations- und Veröffentlichungsfreiheit gefährdet sein wird \cite{Jochum_2009}. Damit ist die Wahl des Publikationsmediums gemeint, die bei den Wissenschaftler und Wissenschaftlerinnen liegen sollte \cite{BBAW_2015}. Infolgedessen wird an vielen Stellen die Befürchtung geäußert, dass im Rahmen zunehmender Kollaboration über digitale Kanäle sowie durch die Effizienz der elektronischen Suche die Diversität von wissenschaftlichen Meinungen und Projekten zu einem gleichen oder ähnlichen Thema eingeschränkt werden könnte \cite{Evans_2008}. Diese Betrachtung ist wiederum nicht unumstritten \cite{Lariviere_2009}.
\item \textbf{Misinterpretation}: Eine weitere Sorge, die den Öffnungsprozess bremst, ist die Angst der wissenschaftlichen Community vor Fehlinterpretationen \cite{Grand_2012} sowie vor dem Verlust der Kontrolle über die Informationssteuerung \cite{Gibbons_1994}. Dabei steht vor allem die Befürchtung im Vordergrund, dass die frei verfügbaren veröffentlichten Arbeiten genutzt werden, um die Arbeit der Wissenschaft zu diskreditieren oder sie gezielt zur Falschinformation der Öffentlichkeit zu nutzen.
\item \textbf{Transparent-Research-Intentions}: Die Forderung nach Offenlegung des gesamten Forschungsprozesses beinhaltet auch die Forderung nach "Transparenz der Interaktion zwischen Sponsoren (insbesondere kommerzielle Förderer wie die Pharma- und Medizinprodukteindustrie) und Auftragnehmern" \cite{Stengel_2013}
\end{enumerate}

---- TODO: Liste überarbeiten  bzw. mit Befragung final abgleichen ----

Die erarbeiteten Hindernisse für die Verbreitung der Öffnung wissenschaftlicher Kommunikation werden im Verlauf der Arbeit im experimentellen Teil im Rahmen der Befragung aufgegriffen. Die möglichen Irritationspotenziale durch die Ausweitung des Zugangs zu oder Zugriffs auf wissenschaftliche Kommunikation sowie die Kritik an digitalen Medien werden nur in Zusammenhang mit den Forschungsfragen berücksichtigt.

\subsection{Indikatoren für die Reputationsverteilung im wissenschaftlichen Kommunikationssystem}

Um die Anreize für das Verhalten der wissenschaftlichen Akteure im Kommunikationssystem besser zu verstehen werden im Folgenden die Indikatoren für die Reputationsverteilung herausgearbeitet. Die Publikation von Erkenntnissen ist in diesem Rahmen nur einer von vielen Indikatoren für die Reputationsverteilung ist \cite{Hirschauer_2004}. Im Gegensatz zu den Modellen, die eine Verpflichtung von oben für ein bestimmtes Verhalten beinhalten und die wissenschaftliche Selbstständigkeit beeinflussen könnten, werden hier vor allem die Indikatoren betrachtet, die Anreize für ein bestimmtes Verhalten darstellen.

Aus der Literatur wurden folgender Indikatoren für die Verteilung von Reputation herausgearbeitet. Die vorgenommene Kategorisierung ist dabei an Heidemarie Hanekop \cite{Hanekop_2008} und die Befragung durch das SOFI 2007 \cite{SOFI_2007} angelehnt:
\begin{enumerate}
\item \textbf{Anzahl der wissenschaftlichen Aufsätze / Beiträge}: Die Anzahl der Texte, die Wissenschaftler und Wissenschaftlerinnen im Rahmen ihrer Tätigkeit publizieren ist ein wesentlicher Faktor der Bewertung wissenschaftlicher Reputation \cite{Warnke_2012} \cite{Clapham_2005} \cite{Luhmann_1970}. Zum Beispiel erhöht die Anzahl an Texten die Chance, durch die andere Mitglieder der wissenschaftlichen Community zitiert zu werden und damit die Möglichkeit auf die Erlangung von Reputation. Durch den zunehmenden Wettbewerb in der Wissenschaft muss sich der einzelne Wissenschaftler entscheiden, "zu publizieren oder im wissenschaftlichen System zu scheitern" \cite{Suess_2006}. Dadurch entsteht im wissenschaftlichen Kommunikationssystem ein konstanter Publikationsdruck, bei dem die Relevanz der publizierten Ergebnisse nicht immer im Vordergrund steht \cite{Hamilton_1990}. Die Anzahl der veröffentlichten Artikel hat einen Einfluss auf die Vergabe von Ressourcen und finanzieller Mittel für weitere Forschung an Institutionen und Individuen \cite{Warnke_2012} \cite{Hamilton_1990}.
\item \textbf{Relevanz der publizierten Ergebnisse}: Die Relevanz der publizierten Ergebnisse ist für das Wissenschaftssystem ein wesentlicher Katalysator für den Prozess der Wissensgewinnung. Relevante Erkenntnisse sind die Grundlage für die Produktion von neuem Wissen und damit Grundlage für den gesellschaftlichen Auftrag des Wissenschaftssystems \cite{hanekop_2008}. Die Relevanz der publizierten Ergebnisse, so wird postuliert, übt einen direkten Einfluss auf die wissenschaftliche Reputation aus.
\item \textbf{Anzahl Monografien}: Die Anzahl der veröffentlichten Monografien ist ein wesentlicher Reputationsfaktor. Das gilt für die Disziplinen, in denen diese Publikationsform wichtig ist, die Geistes- und Sozialwissenschaften. In den anderen wissenschaftlichen Fachrichtungen spielt die Anzahl der Veröffentlichungen von Artikeln in wissenschaftlichen Journalen eine wichtige Rolle.
\item \textbf{Drittmittelprojekte}: Drittmittel sind, so der deutsche Wissenschaftsrat, "solche Mittel, die zur Förderung der Forschung und Entwicklung sowie des wissenschaftlichen Nachwuchses und der Lehre zusätzlich zum regulären Hochschulhaushalt (Grundausstattung) von öffentlichen oder privaten Stellen eingeworben werden" \cite{Wissenschaftsrat_2014}. Die Drittmitteleinwerbung hat sich in Deutschland als "meist gebrauchter Maßstab der Messung von Forschungsqualität durchgesetzt" \cite{Muench_2006}. Diese Entwicklung geht mit einer zunehmenden Finanzierung der Forschung über Drittmittel einher \cite{Neidhardt_2010} \cite{Jansen_2007} \cite{Simon_2009}. Durch die zunehmende Knappheit öffentlicher Ressourcen für Wissenschaft und Forschung, ist die Akquise von Drittmitteln zu einem kritisch zu betrachtenden Kernziel geworden \cite{Jansen_2007}. Das führt zu der Vermutung, dass zunehmend direkte finanzielle und administrative Kontrolle der Forschung eine Rolle spielen \cite{Barloesius_2008}. Dabei ist die Frage relevant, ob die Publikationen, die im Rahmen der Drittmittelfinanzierung als wissenschaftliche Erkenntnisse veröffentlicht werden und ob der Antrag um Drittmitteleinwerbung selbst, "zum Erkenntnisfortschritt in der wissenschaftlichen Gemeinschaft beiträgt" \cite{Muench_2006}. Die wissenschaftliche Community befürchtet durch die zunehmende Relevanz der Anzahl von Drittmittelprojekten bei der Erlangung von wissenschaftlicher Reputation eine Einschränkung der Freiheit von Wissenschaft und Forschung.
\item \textbf{Patente}: Im Gegensatz zu Urheberrechten, werden Patente nur auf Antrag und nach Prüfung staatlich erteilt \cite[:152]{Troy_2012}. Es handelt sich dabei um ein "vom Staat verliehene Schutzrecht für eine technische Erfindung, welches dem Patentinhaber für eine bestimmte Zeit die ausschließliche wirtschaftliche Nutzung der Erfindung vorbehält" \cite{Greif_2003}. Diese Kommodifizierung von Wissen in Form von Patenten ist dabei exemplarisch für die Privatisierung von Wissen \cite[:152]{Troy_2012}. Die Anzahl dieser Schutzrechte im Hochschulbereich nimmt seit den 1970er konstant zu \cite[:168]{Troy_2012}. \cite{Schmoch_2003} \cite{Fabrizio_2008}. Vor allem in den technischen Fachdisziplinen wird eine Patentschrift "als funktionales Äquivalent zur wissenschaftlichen Publikation begriffen" und bewertet \cite{Mersch_2014}. Die deutsche Hochschulrektorenkonferenz hält fasst die Rolle des Patentwesen an den Hochschulen wie folgt zusammen: "Patente leisten einen Beitrag zur Förderung der Wissenschaft, die Grundlagen des Patentwesens sind daher dem wissenschaftlichen Nachwuchs über entsprechende Lehrangebote zu vermitteln" \cite{Greif_2003}. Die Befürchtung, dass Patente einen negativen Effekt auf die Erstellung und Veröffentlichung fundamentaler Forschungsergebnisse haben, konnte nicht abschließend bestätigt werden \cite{Fabrizio_2008}.
\item \textbf{Vorträge}: Vorträge dienen der Verbreitung der Forschungserkenntnisse, sowie Zwischenständen und ermöglichen das Vermitteln des Wissens an andere \cite{Rassenhoevel_2010}. Vorträge stellen eine informelle und schnelle Form für die Verbreitung neuer wissenschaftlicher Erkenntnisse und Ergebnisse dar. Die in einem Vortrag vermittelten Inhalten müssen nicht immer genauer belegt werden und die kommunizierten Inhalte lassen sich gegebenenfalls später schriftlich konkretisieren oder korrigieren \cite{Haberle_2002}. Vorträge bieten die Möglichkeit bereits vor der eigentlichen Publikation von wissenschaftlichen Erkenntnissen Anregungen und Reaktionen einzuholen.
\item\textbf{Anwendungsrelevanz bzw. Verwertbarkeit}: Ein vergleichsweise neuer Indikator für die Reputation von Hochschulen und außeruniversitärer Forschungsinstitute ist die Anwendungsrelevanz der Erkenntnisse von Wissenschaft und Forschung \cite{Simon_2009}. Sie tritt
neben die akademischen Mechanismen der Qualitäts- und Leistungskontrolle \cite[:8]{Buss_2001} und bezieht sich auf einen Outputfaktor, der primär auf den konkreten Einsatz der gewonnenen wissenschaftlichen Erkenntnisse und auf die Verwertbarkeit wirtschaftlicher Produkte oder Patente und weniger auf die eigentliche wissenschaftliche Veröffentlichung abzielt.
\item \textbf{Netzwerke und Kontakte}: Netzwerke beschreiben formelle und informelle Verbundsysteme zwischen Wissenschaftlern. Sie erlauben den schnellen Austausch und können Grundlage für Aktivitäten zur Steigerung der wissenschaftlichen Reputation darstellen. Diese Aktivitäten umfassen zum Beispiel gemeinsame Publikationsvorhaben und den Austausch wissenschaftlicher Erkenntnisse. Kontakte und Netzwerke schaffen soziale Beziehungen, die für eine erfolgreiche Integration an der Hochschule und der Fachcommunity sorgen, Zugang zu wissenschaftlicher Kommunikation ermöglichen und somit einen Einfluss auf die Anerkennung eines Wissenschaftler oder einer Wissenschaftlerin haben können.
\item \textbf{Öffentliche Aufmerksamkeit}: Die öffentliche Aufmerksamkeit stellt zum einen eine Möglichkeit des Wissenstransfers außerhalb der wissenschaftlichen (Fach-)Community dar, zum anderen ermöglicht sie die Einflussnahme auf die politische Relevanz wissenschaftlicher Forschungsthemen. Die Veröffentlichung wissenschaftlicher Informationen zu einem bestimmten Thema des öffentlichen Interesses stellt eine Möglichkeit dar, dieses Thema öffentlichkeitswirksam zu katalysieren. Öffentliche Aufmerksamkeit im Rahmen wissenschaftlicher Tätigkeit stellt eine kritisch zu hinterfragende Möglichkeit für die alternative Ressourcengewinnung dar.
\item \textbf{Politische Relevanz}: Die wissenschaftliche Tätigkeit mit politischer Relevanz stellt eine weitere Möglichkeit dar, wissenschaftliche Inhalte außerhalb der Wissenschaft anwendbar zu machen und führt zu Anerkennung der wissenschaftlichen Arbeit. Daraus ergeben sich allerdings grundsätzliche "Verständigungsprobleme und Interessenkonflikte", da  "Wissenschaft und Politik aufgrund unterschiedlicher Rationalitäten handeln, einander aber zugleich brauchen" \cite{Mayntz_1996}. Während es im Wissenschaftssystem "um Erwerb und Erhalt von Wissen" geht, zielt die Politik auf "Erwerb und Erhalt von Macht" \cite{Mayntz_1996} ab. Dennoch wirkt Wissenschaft durch wissenschaftliche Beratung auf Politik und  Politik beeinflusst Wissenschaft durch Wissenschaftspolitik \cite[:10]{Brown_2014}. Die daraus resultierenden Interessenkonflikte können jedoch die Legitimität der Wissenschaft beeinträchtigen \cite{Weingart_2005} \cite[:494]{Weber_1992} und gegebenenfalls zu "gegenseitigen Enttäuschungen" führen, vor allem in der "forschungspolitischen Beziehung" \cite{Mayntz_1996}.
\item \textbf{Renommee der Forschungseinrichtung}: Das Renommee einer Forschungseinrichtung ist die Wahrnehmung der Einrichtung innerhalb und außerhalb der wissenschaftlichen (Fach-)Community. Sie hat für Wissenschaftler und die Wissenschaftlerin eine besondere Bedeutung \cite{Mayntz_2008}. Sie basiert auf dem Konzept der "Ansteckung" \cite{Luhmann_1970}. Diese Ansteckung kann dazu führen, dass renommierte Professoren den Ruf einer Fakultät und eine renommierte Fakultät auch den Ruf von Professoren aufbessern können. Übertragen auf das wissenschaftliche Publizieren profitiert ein Autor oder eine Autorin bei der "Ansteckung" von dem Renommee einer Einrichtung, wenn er durch die Publikationsorgane der renommierten Institution veröffentlicht \cite{Lutz_2012} \cite{Buss_2001}.
\item \textbf{Renommee von Herausgebern oder Mitautoren} Der Herausgeber organisiert den Begutachtungsprozess und sichert bestimmte Qualitätskriterien mit seiner Reputation und seinem Namen \cite{Mueller_2009}. Auch hier kommt es im Rahmen des symbolischen wissenschaftlichen Kapitals zu einer Übertragung der Reputation der Herausgeber oder Mitautoren auf die anderen veröffentlichenden Autoren.
\item \textbf{Personelle und materielle Ausstattung}: Die materielle Ausstattung beschreibt die Rahmenbedingungen, in der ein Wissenschaftler arbeitet. Diese Rahmenbedingungen haben eine herausragende Bedeutung bei der Entscheidung über einen Wirkungsort von Wissenschaftlern \cite{Mayntz_2008}. Insbesondere die materielle und personelle Ausstattung sind bei traditionellen Berufungsverfahren deutscher Professorinnen und Professoren von besonderem Belang \cite{Himpele_2011}, da sie die Arbeitsfähigkeit und die Anerkennung beeinflussen \cite{Buss_2001}. Wie die materielle Ausstattung gilt auch die personelle Ausstattung als ein reputationsstiftendes Merkmal für Wissenschaftler und die Institution, an denen sie arbeiten \cite{Mayntz_2008}. Bei der Ausstattung handelt es sich um einen bilateralen Indikator, der zum einen aus der Bewertung der wissenschaftlichen Arbeit (im Rahmen der Forschungsförderung) resultiert \cite{Herb_2008} und  zum anderen Reputation innerhalb der Community schafft \cite{Mayntz_2008}.
\item \textbf{Gutachtertätigkeit und Herausgeberschaft}: Gutachter werden zum Beispiel in Peer-Review-Verfahren Autoren des entsprechenden Fachgebietes zugeordnet und entscheiden über die Veröffentlichung des Textes \cite{Frey_2005}. Bei manchen Publikationen wird ein Text mehrmals abgelehnt und eine weitere Überarbeitung durch den Autoren oder die Autorin eingefordert, bevor der Artikel final akzeptiert und daraufhin publiziert wird \cite{Frey_2005}. In diesem Zusammenhang wirkt sich die Reputation der mit diesem Verfahren betrauten Gutachter auch auf das Image des Verlages aus und umgekehrt. Die Gutachtertätigkeit ist aber nicht nur Kernbestandteil des wissenschaftlichen Qualitätssicherungs- und interdependenten Reputationssystems, sondern stellt auch einen informellen Weg der Kommunikation dar. Er ermöglicht den Gutachtern die Vorabsichtung neuester wissenschaftlicher Informationen und Erkenntnisse. Ähnlich wie die Gutachtertätigkeit ist auch die Herausgeberschaft fester Bestandteil des interdependenten wissenschaftlichen Reputationssystems \cite{Frey_2005}: Herausgeber profitieren von den publizierten Inhalten und Erkenntnissen der Autoren, Autoren von der Reputation Herausgebern und der Verlag von beiden.
\item \textbf{Funktion}: Die jeweilige Funktion oder die (universitäre) Stellenbezeichnung ist ein weiterer Faktor für wissenschaftliche Reputation. Zum wissenschaftlichen Personal zählen Professoren, Juniorprofessoren, wissenschaftliche und künstlerische Mitarbeiter, sowie Lehrkräfte \cite{Erhardt_2011}. Eine Weiterentwicklung und der "Aufstieg" in der wissenschaftlichen Hierarchie zielt auf das akademische Streben nach einer Professur \cite{Klecha_2008}.
\item \textbf{Awards und Preise}: Preise sind ein weitere Indikator für das wissenschaftliche Belohnungs- und Bewertungssystem. "Die Praxis der Award-Verleihung beruht auf dem Konzept, dass Ressourcen von unabhängigen Dritten auf Qualität geprüft und (...) zertifiziert werden" \cite{Bargheer_2002}. Wissenschaftler und Wissenschaftlerinnen, die Preise oder Awards gewinnen, erfahren Anerkennung. Diese Anerkennungen können jedoch nicht automatisch als "Garant für wissenschaftsrelevante Qualität" \cite{Bargheer_2002} verstanden werden. Die Ehrung mit einem Preis weckt andererseits gegebenenfalls Erwartungen und führt zu dem Anspruch eines stetigen Nachschubs an Anerkennung für den Wissenschaftler oder die Wissenschaftlerin.
\end{enumerate}
