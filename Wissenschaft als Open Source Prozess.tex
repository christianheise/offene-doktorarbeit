\subsubsection{Wissenschaft als Open-Source-Prozeß}

Open Source ist ein Begriff aus der Softwareentwicklung der als Gegensatz zum “Verfahren der Wissenssicherung”  zugunsten einer quelloffenen Handhabe von Softwarecode verstanden werden will. Der Ende der 90iger Jahre des letzten Jahrhunderts eingeführte Begriff beschreibt, auch wenn es im Detail Unterschiede im Konzept gibt , das gleiche wie der Begriff “freie Software“ . Besonders der Entwicklungsprozess von Open-Source, in Ergänzung zum reinen Zugang und damit mit Open Science konvergent , unterscheidet sich von den klassisch-traditionellen closed-source Prozessen. Dabei folgt Open Source der Maxime, dass die Kernsteuerungsinformationen und -befehle (Quelltext) von Software öffentlich einsehbar und zugänglich und je nach gewähltem Lizenzmodell modifiziert, kopiert oder weitergegeben werden müssen . Es gibt aber auch unterschiede, so betont Steven Weber den Unterschied zwischen Open-Source-Software und dem traditionellen Modell des geistigen Eigentums mit der Feststellung, dass Open-Source-Software macht das Prinzip der Exklusivität des geistigen Eigentums auf den Kopf, weil diese Software 'um das Recht auf Vertrieb konfiguriert, nicht auszuschließen. "
Als Maurer und Scotchmer angemerkt haben, Open-Source-Software-Entwicklung Rechtsmittel ein Defekt der Schutz des geistigen Eigentums, die nicht allgemein zu fördern hat die Offenlegung des Quellcodes. 

Ebenso wie die Open Definition, gibt es festgelegte Kriterien für die Klassifizierung von Open Source Produkten. So heißt es in der Open Source Definition :
\begin{enumerate}
\item Freie Weitergabe
\item Quellcode, das Programm muss den Quellcode beinhalten, bzw. muss dieser offen zur Verfügung gestellt werden
\item Verwendete Lizenz muss Derivate zulassen
\item Unversehrtheit des Quellcodes des Autors muss garantiert werden
\item Auschluss von Diskriminierung von Personen oder Gruppen
\item Keine Enschränkung des Einsatzfeldes
\item Lizenz muss weitergegeben werden könnne
\item Lizenz muss auf das Produktpaket angewandt werden
\item Lizenz darf die Weitergabe zusammen mit anderer Software nicht einschränken
\end{enumerate}

Im Vergleich zum klassischen wissenschaftlichen Entwicklungsprozess gelten dabei folgende charakteristische Merkmale :
\begin{enumerate}
\item “Anzahl der beteiligten Entwickler: Im Vergleich zu traditioneller Softwareentwicklung ist eine weitaus größere Anzahl von Entwicklern beteiligt. Zudem gibt es keine klare Grenze zwischen Entwicklern und Anwendern, da die Hürden für eine Partizipation im Entwicklungsprozess sehr gering sind. Auch wenn ein großer Teil der Entwicklungsarbeit von Freiwilligen übernommen wird, gibt es dennoch den Trend zum Einsatz bezahlter Entwickler.
\item Zuteilung der Arbeit: Im OSP wird die Entwicklungsarbeit nicht länger von einer definierten Instanz zugeteilt, sondern die Teilnehmer wählen ihre Arbeitspakete selbst aus.
\item Architektur: In der Regel orientierten sich die Teilnehmer eines OSP nicht an einer vorgegebenen System-Architektur. Die Gestaltung der Architektur geschieht dezentral und ist oftmals häufigen Richtungswechseln unterworfen.
\item Koordination: Es gibt wenig oder keine institutionalisierten traditionellen Koordinationsmechanismen, wie z.B. Projekt- und Zeitpläne, Lasten- und Pflichtenhefte u.ä.”
\end{enumerate}

Vergleicht man diese mit dem traditionellen Wissenschaftsprozess (siehe 2.2.1.), ergeben sich gewisse Parallelen. Adaptiert man also den Open-Source-Prozess auf Wissenschaft und versteht wissenschaftliche Publikationen als Quellcode, ist das Konzept übertragbar. Der deutsche Literaturwissenschaftler und Medientheoretiker Friedrich Kittler sieht den Gedanken hinter Open-Source fest verankert und äussert in seinem Beitrag “Wissenschaft als Open-Source-Prozeß” die Sorge, “daß mit der Freiheit von Quellcode auch die Freiheit der Wissenschaft steht und fällt” . Wie Wissenschaft als Open-Source-Prozess verstanden werden kann soll in diesem Kapitel genauer erläutert werden.  
