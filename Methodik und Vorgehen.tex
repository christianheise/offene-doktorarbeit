\chapter{Methodische und Vorgehen}
Die Verortung der Fragestellung dieser Arbeit, die von den Kulturwissenschaften über die Wirtschaftswissenschaften bis hin zu den Medienwissenschaften reicht, erfordert einen transdisziplinären Zugang zur wissenschaftlichen Bearbeitung. 
Drei wissenschaftliche Methoden werden in dieser Arbeit angewandt: das Konzeptionelle/Theoretische im Rahmen der Literaturanalyse für die Begriffsbestimmung, das Ethnographische im Rahmen der Befragung zur Identifikation der Treiber und Bremser für die Öffnung von wissenschaftlicher Informationen und Prozesse sowie das Experimentelle. 
Die Herangehensweise folgt dabei der Auffassung des Medientheoretikers Geert Lovink, der diese dreifache Methode zur Erforschung der digitalen Kultur für zwingend notwendig erachtet\cite{suchen}. 
Ziel ist es, zu einem vertieften theoretischen Verständnis der empirischen Ergebnisse zu gelangen. --- TODO: was mache ich methodisch ---- So können mögliche Verallgemeinerungsmodelle im Rahmen der definierten Fragestellungen theoretisch entwickelt und praktisch geprüft werden.

\section{Vorüberlegungen zur Methodenwahl}

Die Methodendiskussion in den Kulturwissenschaften wird von 
. 

\section{Methode der Literaturanalyse}
Die unterschiedliche Verwendung der Begriffe Open Science und Open Access in der wissenschaftlichen Auseinandersetzung machen es notwendig, eine Begriffsbestimmungen für Open Science und Open Access vorzunehmen und zu konkretisieren. In Ergänzug zur Literaturanalyse von Benedikt Fecher und Sascha Friesike für den Begriff "Open Science"\cite{cite:9} sowie der Litaraturanalyse von Giancarlo Frosio und Estelle Derclaye "Open Access Publishing" \cite{CREATe_2014} soll für diesen Zweck auch eine systematischen Literaturanalyse für die Begriffe "Open Access" und "Open Science" inklusive der Treiber und Bremser der Öffnung von Wissenschaft im Kontext des Begriffs "wissenschaftliche Reputation" durchgeführt werden. Neben der Berücksichtigung von Arbeiten aus den Medienwissenschaften im engeren Sinn sollen auch Arbeiten aus den Wirtschaftswissenschaften und den Kulturwissenschaften berücksichtigt werden.


\section{Methode der Onlinebefragung}
Um der Entwicklung der Öffnungvon Wissenschaft sowie deren Treiber und Bremser nachgehen zu können, soll eine Onlinebefragung unter den beteiligten Stakeholdern des akademischen Publizierens an wissenschaftlichen Institutionen explorativ durchgeführt werden. Dies ist nicht zuletzt deshalb für diese Arbeit relevant, weil theoretische Vorannahmen im Rahmen der Definition und Abgrenzung sowie der Literaturanalyse bestehen. Somit sollen die bestehenden Hypothesen getestet, beziehungsweise neue Hypothesen generiert werden. Durch einen Vergleich mit der Studie "Neue Formen des Wissenschaftlichen Publizierens" aus dem Jahr 2007 und 2008 vom Soziologisches Forschungsinstitut Göttingen (SOFI) soll darüber hinaus ein Einblick in die historische Entwicklung der Thematik im deutschsprachigen Raum ermöglicht werden. Die Befragung aus Göttingen bildet ausserdem die Grundlage für die Fragebogenkonstruktion dieser Erhebung. 

Die umfrangreiche Befragung aus den Jahren 2007 und 2008 entstand im Rahmen eines BMBF geförderten Verbundprojekts zwischen SOFI Göttingen und der Universitätsbibliothek Göttingen. Sie basierte auf einer "Vollerhebung der Wissenschaftler an den Instituten und Einrichtungen an fünf deutschen Standorten, die differenziert nach Fächern, Alters- und Statusgruppen (n=6500) erfasst wurden" \cite{Hanekop_2014}. Ziel der Befragung war es, die "Veränderungen beim Zugang zur Literatur wie auch bei den Veröffentlichungsstrategie" \cite{Hanekop_Wittke_2007_Fragebogen} zu untersuchen. In die Teilnehmer der Studie wurden anhand von Webseiten der Forschungseinrichtungen identifiziert und per Email zur Teilnahme aufgefordert. 

\section{Das Experiment als wissenschaftliche Methode: Offenes Schreiben dieser Arbeit}
Zur weiteren Erkenntnisgewinnung und für das Ziel der Arbeit Handlungsempfehlungen für das offene Schreiben von Dissertationen erstellen zu können sowie die Kriterien und Argumente für oder gegen das offene Publizieren prüfen zu können, wurde für diese Arbeit selber eine offene Schreibweise gewählt. “Offen” bedeutet in diesem Fall, dass diese Arbeit direkt und unmittelbar bei der Erstellung für jeden, jederzeit frei zugänglich auf einer Webseite im Internet unter einer freien Lizenz (CC-BY-SA) veröffentlicht wurde. Der aktuelle Stand der Arbeit entsprach zu jedem Zeitpunkt dem Stand auf der Webseite. 

Um trotzdem den Anforderungen der Prüfungsordnung in allem Umfang gerecht zu werden, wurde in einem Schreiben an die Promotionskomission am 8. Januar 2013 alle betreffende Punkte in der Promotionsordung der Fakultät Kultutwissenschaften (Stand: 02.02.2011) hervorgehoben und versucht zu begründen, warum diese nicht im Widerspruch zur offenen Schreibweise meiner Arbeit stehen. Um die selbstständige, wissenschaftlicher Arbeit sicherzustellen, hatte kein anderer die Möglichkeit, den erstellten Inhalt zu editieren oder zu kommentieren. Die Transparenz während der Erstellung stellt in diesem Fall kein Widerspruch zu der Selbständigkeit bei der Ausarbeitung dar. Im Gegenteil, sie ermöglichte eine neue Form, die Eigenständigkeit direkt während der wissenschaftlichen Arbeit und Erstellung des Inhalts sicherzustellen. Dem Gesuch die Arbeit "offen" verfassen zu dürfen, wurde seitens der Promotionskommission am 12. Dezember 2013 mehrheitlich entsprochen.

\section{Forschungsfragen} 
Folgende Forschungsfragen sollen bei der Literaturanalyse genauer betrachtet werden:
\begin{itemize}
\item Wie können Open Science und Open Access definiert und voneinander abegrenzt werden? 
\item Warum kommt es zu der Bestrebungen hin zur Öffnung von Wissenschaft? 
\item Welche Pro- und Contraargumente gibt es für die Öffnung von Wissenschaft - ist Offenheit in der Wissenschaft gut oder schlecht? 
\item Warum ist die Öffnung von Wissen in den verschiedenen wissenschaftlichen Disziplinen unterschiedlich weit verbreitet? 
\item Was bedeutet Offenheit und freier Zugang im Rahmen des wissenschaftlichen Diskurs-, Reputations- und Machtbegriffs?
\end{itemize}	

\section{Kritische Betrachtung der Vorgehensweise} 
