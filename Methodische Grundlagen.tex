\chapter{Methodische Grundlagen}
Die Verortung der Fragestellung dieser Arbeit, die von den Kulturwissenschaften über die Wirtschaftswissenschaften bis hin zu den Medienwissenschaften reicht, erfordert einen transdisziplinären Zugang zur wissenschaftlichen Bearbeitung. 
Drei wissenschaftliche Methodologien werden in dieser Arbeit angewandt: das Konzeptionelle/Theoretische im Rahmen der Literaturanalyse für die Begriffsbestimmung, das Ethnographische im Rahmen der Befragung zur Identifikation der Treiber und Bremser für die Öffnung von wissenschaftlicher Informationen und Prozesse sowie das Experimentelle. 
Die Herangehensweise folgt dabei der Auffassung des Medientheoretikers Geert Lovink, der diese dreifache Methodik für die Erforschung der digitalen Kultur für zwingend notwendig erachtet . 
Ziel ist es, letztlich zu einem vertieften theoretischen Verständnis der empirischen Ergebnisse zu gelangen. Im Rahmen der Arbeit am Inkubator bietet es sich hier an, weitere Hypothesen anhand von Experimenten zu gewinnen und diese mit neuen Geschäftsmodellen und politischen Prozessen forscherisch zu begleiten. So können mögliche Verallgemeinerungsmodelle im Rahmen der in Kapitel 3 definierten Fragestellungen theoretisch entwickelt und praktisch geprüft werden.
\section{Methode der Inhaltsanalyse}
Die unterschiedliche Verwendung der Begriffe Open Science und Open Access in der wissenschaftlichen Auseinandersetzung machen es notwendig die Begriffsbestimmungen für Open Science und Open Access im Rahmen dieser Arbeit vorzunehmen und zu konkretisieren. In Ergänzug zu der Literaturanalyse von Benedikt Fecher und Sascha Friesike für den Begriff "Open Science"\cite{cite:9} sowie der Litaraturanalyse von Giancarlo Frosio und Estelle Derclaye "Open Access Publishing" \cite{CREATe_2014} soll für diesen Zweck auch eine systematischen Literaturanalyse für die Begriffe "Open Access" und "Open Science" inklusive der Treiber und Bremser der Öffnung von Wissenschaft im Kontext des Begriffs "wissenschaftliche Reputation" durchgeführt werden. Neben der Berücksichtigung von Arbeiten aus den Medienwissenschaften im engeren Sinn sollen auch Arbeiten aus den Wirtschaftswissenschaften und den Kulturwissenschaften berücksichtigt werden.
\subsection{Forschungsfragen} 
Folgende Forschungsfragen sollen bei der Inhalsanalyse genauer analysiert werden:
\begin{itemize}
\item Warum kommt es zu der Bestrebungen hin zur Öffnung von Wissenschaft? 
\item Wie werden Open Science und Open Access definiert und voneinander abegrenzt? 
\item Welche Pro- und Contraargumente gibt es für die Öffnung von Wissenschaft - ist Offenheit in der Wissenschaft gut oder schlecht? 
\item Wo sind die Grenzen der Öffnung? 
\item Warum ist die Öffnung von Wissen in den verschiedenen wissenschaftlichen Disziplinen unterschiedlich weit verbreitet? 
\item Was bedeutet Offenheit und freier Zugang im Rahmen des wissenschaftlichen Diskurs-, Reputations- und Machtbegriffs?
\end{itemize}	

\subsection{Erhebungsmethode und Umfang} 
tbd

\subsection{Analyse der Definitionen von Open Access} 
tbd

\subsection{Analyse der Meinungen und Kritik an Open Access}

In diesem Teil der Arbeit soll im Rahmen der Literaturanalyse eine Auflistung der Kritikpunkte an der Open Access Bewegung in Wissenschaft und Forschung dokumentiert werden. Die Auswahl der berücksichtigten Werke bezieht sich auf dei genannten Fragestellungen und soll als verständlicher Überblick über den vorherrschenden Diskurs im Rahmen von Open Access und Open Science verstanden werden.

\subsubsection{Kritik am ökonomischen Modell}

Ein Kritikpunkt an dem Open Access Modell bezieht sich auf das Kostenargument und die frühe Hoffnung, dass die technologischen Treiber gesteuert und organisiert von der Forschungs Community selbst, anstatt durch Fachverlage, die durchschnittlichen Kosten für einen publizierten Artikel signifikant senke könnten. In einigen Beiträgen wurdne schon früh Kostensenkungen von bis zu 90 Prozent\cite{hilf_2004} prognostiziert. Grundlage dafür war die Fragestellung, dass "aus der Sicht des individuellen Nutzenkalküls von Wissenschaftlern, Verlagen und weiteren Einrichtungen wie Bibliotheken als auch aus Sicht gesamtwirtschaftlicher Wohlfahrtsüberlegunge (...) ob der Markt der Wissenschaftskommunikation nicht effizienter organisiert werden könnte."\cite{Hess_2006} Folgende Punkte schürten darüber hinaus die Hoffung, dass System leistungsfähiger zu machen und "von seinen durch den Papierdruck auferlegten Fesseln" zu befreien \cite{hilf_2004}:
\begin{itemize}
\item langer Zeitverzug vom Einreichen eines Manuskriptes bis zum Gelesen werden,
\item komplizierter Vertriebsweg vom Verlag über Grossisten zu Bibliotheken,
\item horrende Kosten (ca. 3.000,- Euro für die gesamte Verlagsarbeit je Artikel) mit den daraus folgenden horrenden Zeitschriftenpreisen,
\item und daraus folgend wenige Leser, auch noch ungleich in der Welt verteilt (digital divide),
\item unvollständige Information (aus Platzmangel), was Nachnutzungen und das Nachprüfen erschwert und somit auch Fälschungen erleichtert,
\item nur anonymes Referieren vor der Veröffentlichung, was den Missbrauch erleichtert. 
\end{itemize}

\subsubsection{Sicherung von Freiheit von Forschung und Lehre sowie Forschungsdiversität}

Eine Öffnung der wissenschaftlichen Kommunikation hat weitreichende Implikationen, nicht nur auf die Frage wie geforscht wird, sondern auch was geforscht wird \cite{suchen}. Da ein Großteil der Wissenschaft durch die öffentliche Hand finanziert wird, stecken hinter den Steuerungsmechanismen von Wissenschaft und Wissenschaftsförderung immer auch politische Interessen. Zwar soll die Vermischung dieser Interessen in Deutschland durch die Unabhängigkeit der Deutsche Froschungsgemeinschaft verhindert werden und Mittel völlig frei von politischer Couleur verteilt werden \cite{suchen}, dennoch kann, so die Befürchtung einiger Autoren \cite{suchen}, nicht sichergestellt werden, dass eine Einbeziehung der Öffentlichkeit nicht doch einen Einfluss auf die Mittelvergabe hätte. Drastischer ausgedrückt sieht Hagner in dem Beitrag "Open access als Traum der Verwaltungen" dass es im Rahmen des Öffnungsprozesses auch auf eine vollends verwaltete Forschung hinausläuft \cite{suchen}. Grundlagenforschung sowie andere komplexe oder explorative Forschungsbereiche würden in Zukunft weniger Berücksichtigung finden und die Freiheit von Wissenschaft und Forschung endgültig gefährdet \cite{suchen}, so der düstere Ausblick einiger Wissenschaftler \cite{suchen} \cite{suchen}. 

Um diese Aspekte beziehungsweise Prognosen über die Implikationen von Open Access zu evaluieren werden in diesem Teil der Arbeit auf Grundlage von Textbeispielen die Kritik an der Öffnung von Wissenschaft und der (forschungs-)politischen, rechtlichen und freiheitlichen Entwicklungen beleuchtet.

\subsubsection{Beispiel: Der "Heidelberger Apell" für Publikationsfreiheit und die Wahrung der Urheberrechte }

Am 22. März 2009 wurde auf der Webseite der „Frankfurter Allgemeinen Zeitung“ der Artikel "Geistiges Eigentum: Autor darf Freiheit über sein Werk nicht verlieren" \cite{faz_heidelberger_apell_2009} veröffentlicht. Im Anhang zu dem Artikel fand sich ein Aufruf, auch der "Heidelberger Appell" genannt. Vorangegangen war eine öffentlich ausgetragene Diskussion zwischen dem Literaturwissenschaftler Prof. Dr. Roland Reuß sowie weiteren Wissenschaftlern in einem Spezial der Onlineausgabe der Frankfurter Allgemeinen Zeitung: "Die Debatte über Open Access".

Der Appell richtete sich vor allem an "die Bundesregierung und die Regierungen der Länder, das bestehende Urheberrecht, die Publikationsfreiheit und die Freiheit von Forschung und Lehre entschlossen und mit allen zu Gebote stehenden Mitteln zu verteidigen" \cite{ITK_2009}. Die Autoren forderten Politik, Öffentlichkeit und weitere Kreative auf, sich für die "Wahrung der Urheberrechte", unter anderem in Bezug auf die Google Buchsuche "gegen eine angebliche „Enteignung“ der Autoren durch das Vorgehen von Google einerseits und durch das Publikationsmodell Open Access andererseits" \cite{WD_bundestag_2009} zu engagieren. 

Die Kritik am urheberrechtlichem Aspekt der Google Buchsuche soll in dieser Arbeit nicht berücksichtigt werden. Hier soll nur untersucht werden, inwiefern die Kritik am Publikationsmodell Open Access berechtigt ist. Der Apell unterscheidet dabei in zwei Ebenen: \textit{International} kritisieren die Autoren "die nach deutschem Recht illegale Veröffentlichung urheberrechtlich geschützter Werke geistiges Eigentum auf Plattformen wie GoogleBooks und YouTube" und die Entwendung dieser "ohne strafrechtliche Konsequenzen". \textit{National} werden die "Eingriffe in die Presse- und Publikationsfreiheit, deren Folgen grundgesetzwidrig wären" durch die "»Allianz der deutschen Wissenschaftsorganisationen« (Mitglieder: Wissenschaftsrat, Deutsche Forschungsgemeinschaft, Leibniz-Gesellschaft, Max Planck-Institute u. a.)" angeprangert.\cite{ITK_2009}

Die Kritik der Autoren des Heidelberger Apells bezieht laut einer Untersuchung des Wissenschafltichen Diensts des Bundestags insbesondere auf drei Aspekte \cite{WD_bundestag_2009}:
\begin{enumerate}
\item Erzwungene Vertriebswege
"Eine Forschung, der man diktieren könnte, wo ihre Ergebnisse publiziert werden sollen, sei nicht mehr frei." Die Verpflichtung auf "bestimmte Publikationsform (...) dient nicht der Verbesserung der wissenschaftlichen Information" \cite{ITK_2009}.
\item Abhängigkeitsverhältnis
\item Subventionierung von Vertriebswegen
\end{enumerate}

Der Appell "hat eine außergewöhnlich heftige Diskussion über die urheberrechtliche Problematik im Hinblick auf die aktuellen Entwicklungen im Internet ausgelöst. Er hat auch viele Parlamentarier und Politiker für das Thema sensibilisiert"\cite{WD_bundestag_2009}. An vielen Stellen widerlegt der Wissenschaftliche Dienst die Befürchtungen der Autoren des Heidelberger Apells. Beim Kritikpunkt der "Erzwungene Vertriebswege" widerspricht der Wissenschaftliche Dienst mit dem Verweis auf Gudrun Gersmann, weil "auch (Anmerkung: unter Open Access) eine Veröffentlichung bei einem Verlag mit einfachem Nutzungsrecht weiterhin möglich sei". In Bezug auf die im Apell erwähnte Kritik am neuen Abhängigkeitsverhältnis halten die wissenschaftlichen Autoren des Bundestags Reuß entgegen, dass es im bisherigen System "zwischen Autor und Fachzeitschriftverlag oft ein einseitiges Abhängigkeitsverhältnis zu Lasten des Autors gibt" und Wissenschaftler "oftmals alle Rechte an ihren Beiträgen abtreten" \cite{WD_bundestag_2009} müssen. "Der Befürchtung im Heidelberger Appell, das Publikationsmodell Open Access gefährde Fachzeitschriftenverlage", laut Autoren dritter Aspekt der Kritik an Open Access im Apell, "wird entgegengehalten, dass die digitale Plattform auf lange Sicht auch ein Ausweg aus der Zeitschriftenkrise sein könnte" \cite{WD_bundestag_2009}.

Dabei ist die Kritik im Rahmen des Apells mindestens an zwei Punkten berechtigt, so ist es erstens wahr, dass man seitens der Forschungsförderer nicht besonders bemüht war und ist \cite{suchen}, sich "ein genaues Bild von den Nebenwirkungen (Anmerkung: von Open Access)" \cite{Reuss_2009} zu verschaffen und zweitens stellt die Sicherung von Freiheit von Forschung und Lehre sowie die Anpassung der Steuerungsmechanismen eine Herausforderung an die Bestrebungen zur Öffnung von Wissenschaft und Forschung dar \cite{suchen}.

\subsubsection{Analyse der Definitionen von Open Access} 

Eine eindeutige Klassifizierung von Open Access gelingt derzeit nicht. Es "keine formelle Struktur, keine offizelle Organisation und kein ernannter Führer" gibt, der die Open Access Bewegung antreibt\cite{poynder_2011_suber}. Einzig und allein die Open Definition - open definition schreiben -

\subsubsection{Treiber und Bremser für Open Access} 

In den wissenschaftlichen Beiträgen zu Open Access werden viele positive Folgen aufgelistet. Folgende Treiber für eine Veränderung und Öffnung des wissenschaftlichen Kommunikationssystems werden dabei besonders häufig genannt:

\begin{itemize}
\item Verbreitung und Nutzungsmöglichkeiten der digitalen Infrastrukturen
\item Vorteile des grenzüberschreitenden Austauschs im Rahmen der Globalisierung von Wissenschaft und Forschung
\item ...
\end{itemize}

Neben den Aspekten die die Verbreitung von Open Access in den letzten Decaden unterstützt haben, gibt es aber auch einige Kriterien, die entweder zu einer Verlangsamung der Entwicklung geführt haben, oder sie in einigen Teilbereichen ganz zum erliegen gebracht haben. Dazu gehören:

\begin{itemize}
\item Fehlende Richtlinien auf regionaler, nationaler und internationaler Ebene
\item Führungslosigkeit der Open Access Bewegung
\item ...
\end{itemize}

\subsection{Analyse der Definitionen von Open Science} 

--- TODO ---- Michael Nielsen: “Open science is the idea that scientific knowledge of all kinds should be openly shared as early as is practical in the discovery process.”  https://lists.okfn.org/pipermail/open-science/2011-July/000907.html
http://www.openscience.org/blog/?p=454,

Research Information Network: “science carried out and communicated in a manner which allows others to contribute, collaborate and add to the research effort, with all kinds of data, results and protocols made freely available at different stages of the research process.” http://www.rin.ac.uk/our-work/data-management-and-curation/open-science-case-studies

Fecher/Friesike 5 Schulen von Open Science http://blogs.lse.ac.uk/impactofsocialsciences/2013/06/20/open-science-new-perspectives-for-scholarly-communication/ 
Siehe "Open Science"-Teil @ https://docs.google.com/document/d/1qDkQV-M_2VazjWwncRq_udo9tQqrjuZZkdLeKFc3cpI/edit#heading=h.1ahb76xafkbm

"Open science is the concept of making the whole research process as transparent and accessible as possible."\cite{Scheliga_2014}

Open science can be seen as a mechanism of cumulative knowledge production whereby scientists draw upon knowledge derived at by "prior researchers" and make their discoveries available to "future researchers". \cite{Scheliga_2014} auf Grundlage von \cite{Mukherjee_2009}
--- TODO ---

Es gibt zahlreiche Open Science Initiativen \cite{Scheliga_2014} viele von Ihnen erreichen aber keine kritische Masse \cite{wrap_2010} und enden eher als "virtuelle Geisterstädte" \cite{Nielsen_2011}.

\subsection{Analyse der Definitionen Treiber und Bremser für Open Science} 
tbd

\subsection{Analyse der Meinungen und Kritik an Open Science}

Während viele Wissenschaftler und Wissenschaftlerinnen Offenheit in der Forschung als wertvoll erachten, sind nur wenige sind wirklich bereit, die zusätzliche Zeit und Mühe zu investieren und potenziellen Risiken einzugehen, ihre Forschung offen und zugänglich zu machen \cite{Scheliga_2014} \cite{Procter_2010}. Forscherinnen und Forscher, die offene Wissenschaft pratizieren wollen, sind mit einer Reihe von Hindernissen konfrontiert \cite{Scheliga_2014}: 
\begin{enumerate}
\item individuelle Hindernisse: Angst vor Trittbrettfahren, Mehraufwand an Zeit und Mühe, Herausforderungen bei der Nutzung der digitalen Dienste, fehlender Anstoß negative Ergebnisse zu veröffentlichen, Herausforderung den Datenschutz sicherzustellen, Abneigung den Code zu teilen
\item systematische Hindernisse: Evaluationskriterien behindern Offenheit, kulturelle und institutionelle Einschränkungen, ineffektive (politische) Richtlinien, Mangel an Standards für das Teilen von Forschungsmaterialien, Mangel an rechtlicher Klarheit, finanzielle Aspekte der Offenheit
\end{enumerate}

Betrachtet wie Scheliga und Friesike das Phänomen Open Science an Hand des Konzepts des Soziale Dilemmatas, wird deutlich, dass was im kollektiven Interesse der wissenschaftlichen Gemeinschaft ist, nicht unbedingt im Interesse des einzelnen Wissenschaftlers ist und "wenn alle Wissenschaftler ihr Wissen nur in den Situationen teilen, in denen sie erwarten, dass sie selbst davon profitieren, ist die gemeinsame Wissenspool fragmentiert und alle Wissenschaftler stehen schlechter dar"\cite{Scheliga_2014}. 

Demgegenüber stehen dem gegenüber  --- TODO: ausarbeiten ---

\section{Methode der Onlinebefragung}
Um der Entwicklung der Öffnungvon Wissenschaft sowie deren Treiber und Bremser nachgehen zu können, soll eine Onlinebefragung unter den beteiligten Stakeholdern des akademischen Publizierens an wissenschaftlichen Institutionen explorativ durchgeführt werden. Dies ist nicht zuletzt deshalb für diese Arbeit relevant, weil theoretische Vorannahmen im Rahmen der Definition und Abgrenzung sowie der Literaturanalyse bestehen. Somit sollen die bestehenden Hypothesen getestet, beziehungsweise neue Hypothesen generiert werden. Durch einen Vergleich mit der Studie "Neue Formen des Wissenschaftlichen Publizierens" aus dem Jahr 2007 und 2008 vom Soziologisches Forschungsinstitut Göttingen (SOFI) soll darüber hinaus ein Einblick in die historische Entwicklung der Thematik im deutschsprachigen Raum ermöglicht werden. Die Befragung aus Göttingen bildet ausserdem die Grundlage für die Fragebogenkonstruktion dieser Erhebung. 

Die umfrangreiche Befragung aus den Jahren 2007 und 2008 entstand im Rahmen eines BMBF geförderten Verbundprojekts zwischen SOFI Göttingen und der Universitätsbibliothek Göttingen. Sie basierte auf einer "Vollerhebung der Wissenschaftler an den Instituten und Einrichtungen an fünf deutschen Standorten, die differenziert nach Fächern, Alters- und Statusgruppen (n=6500) erfasst wurden" \cite{Hanekop_2014}. Ziel der Befragung war es, die "Veränderungen beim Zugang zur Literatur wie auch bei den Veröffentlichungsstrategie" \cite{Hanekop_Wittke_2007_Fragebogen} zu untersuchen. In die Teilnehmer der Studie wurden anhand von Webseiten der Forschungseinrichtungen identifiziert und per Email zur Teilnahme aufgefordert. 



\section{Das Experiment als wissenschaftliche Methode: Offenes Schreiben dieser Arbeit}
Zur weiteren Erkenntnisgewinnung und für das Ziel der Arbeit Handlungsempfehlungen für das offene Schreiben von Dissertationen erstellen zu können sowie die Kriterien und Argumente für oder gegen das offene Publizieren prüfen zu können, wurde für diese Arbeit selber eine offene Schreibweise gewählt. “Offen” bedeutet in diesem Fall, dass diese Arbeit direkt und unmittelbar bei der Erstellung für jeden, jederzeit frei zugänglich auf einer Webseite im Internet unter einer freien Lizenz (CC-BY-SA) veröffentlicht wurde. Der aktuelle Stand der Arbeit entsprach zu jedem Zeitpunkt dem Stand auf der Webseite. 

Um trotzdem den Anforderungen der Prüfungsordnung in allem Umfang gerecht zu werden, wurde in einem Schreiben an die Promotionskomission am 8. Januar 2013 alle betreffende Punkte in der Promotionsordung der Fakultät Kultutwissenschaften (Stand: 02.02.2011) hervorgehoben und versucht zu begründen, warum diese nicht im Widerspruch zur offenen Schreibweise meiner Arbeit stehen. Um die selbstständige, wissenschaftlicher Arbeit sicherzustellen, hatte kein anderer die Möglichkeit, den erstellten Inhalt zu editieren oder zu kommentieren. Die Transparenz während der Erstellung stellt in diesem Fall kein Widerspruch zu der Selbständigkeit bei der Ausarbeitung dar. Im Gegenteil, sie ermöglichte eine neue Form, die Eigenständigkeit direkt während der wissenschaftlichen Arbeit und Erstellung des Inhalts sicherzustellen. Dem Gesuch die Arbeit "offen" verfassen zu dürfen, wurde seitens der Promotionskommission am 12. Dezember 2013 mehrheitlich entsprochen.

