\chapter{Analyse des Forschungsstands}
Das Geschäftsmodell hinter der wissenschaftlichen Kommunikation ermöglicht den Verlegern Betriebsgewinnmargen von über 35 Prozent \cite{russell_2008_business} und hohe jährliche Wachstumsraten \cite{Wellcome_Trust_2003}. Sucht man nach Gründen für die Unterstützung des bisherigen Modells durch die Wissenschaftsgemeinschaft, wird deutlich, dass vor allem die in --- TODO in welchen Kapitel? --- beschriebene wissenschaftliche Reputation einen zentralen, extrinsischen Motivationsfaktor für Wissenschaftler darstellt \cite{minssen_2012_arbeit}. Die akademische Reputation „ist [dabei] die zentrale Kommunikationsform, die das Wissenschaftssystem charakterisiert“ \cite{Rutenfranz_1997}. Die Ergebnisse aus wissenschaftlicher Forschung werden dabei als Publikationen vor allen Mitgliedern der Wissenschaft präsentiert, „um diese intern von der Wissenschaftsgemeinde als wissenschaftlich beziehungsweise unwissenschaftlich zertifizieren zu lassen" \cite{Rutenfranz_1997}. 

Der aktuelle Forschungsstand zur Öffnung von Wissenschaft, zu den Treibern und Bremsern dieser Entwicklung und dem damiteinhergehenden Paradigmenwechsel mit Fokus auf den Themenbereich der wissenschaftlichen Reputation ist von besonderem Interesse. Grundlage dessen sind ausgewählte relevante und aktuelle Werke der Fachdiskussion die sich mit dem Phänomen Öffnung von Wissenschaft beschäftigen. Ziel des Kapitels ist die Entwicklung einer geeigneten wissenschaftlichen Fragestellung.

\section{Beschreibung des Forschungsstands}
Anhand einer Literaturanalyse wird dargestellt, welche Argumentationen es für und gegen, sowie welche Möglichkeiten und Grenzen für die Öffnung der Wissenschaft angeführt werden. Eine kritische Analyse soll dabei Pro- und Kontraargumente zusammenfassen und einen Überblick über die aktuelle Debatte um Open Science und Open Access ermöglichen. Diese Analyse basiert auf der Annahme, dass sich Open Access in einer Übergangsphase von der reinen offenen Bereitstellung wissenschaftlicher Publikationen und dem damit verbundenen offenen Zugang zur Wissenschaft zur umfassenden und offenen Wissensverteilung und dem damit einherdehnenden Zugriff auf Wissenschaft für die Gesamtgesellschaft (Open Science) befindet. Darüber hinaus sollen medienkulturwissenschaftlich Open Science und Open Access in ihren technischen als auch in ihren gesellschaftlichen und politischen Aspekten sowie die kulturellen Auswirkungen der Medienbrüchen im Rahmen von hybridem Publizieren reflektiert werden. Abschließend werden Treiber und Bremser für die Öffnung von Wissenschaft erhoben und in der Gesamtbetrachtung der Arbeit zusammengeführt.

Die \textbf{Forderung nach Öffnung} adressiert mehrere Unzulänglichkeiten am bestehenden wissenschaftlichen Kommunikationssystem:
\begin{enumerate}
\item Transition-Argument
Die Nutzung der neuen Möglichkeiten für eine offene Wissensverbreitung neben den konventionellen Wegen der nicht-elektronischen Publikationen . Dabei gilt die Grundvoraussetzung der Aufbereitung des Wissens als strukturierte Daten zur Wissensweiterverwendung und -verarbeitung über alle Kanäle.
\item Speed & Circulation-Argument
Wissensverbreitung wird künstlich durch Embargos und ineffiziente Validationssysteme zurückgehalten. Die Digitalsierung und Verbreitung über elektronische Kanäle stellt einen Vorteil für Wissensverbreitung und -verwertung dar. Wenn das Wissen schneller zur Verfügungsteht wird es schneller zirkulieren und effizienter genutzt werden können \cite{Woelfle_2011}.  
\item Higher Impact & Citation-Argument
Ein Hauptargument der Open Acces-Befürworter ist die höhere Zitationsrate von wissenschaftlichen Publikationen, die unter den Kriterien von Open Access veröffentlicht wurden\cite{cite:21a}. In der einschlägigen Literatur findet man viele Untersuchugnen, die das zum Untersuchungsgegenstand gemacht haben und zu einem positiven Ergbniss kommen \cite{Lawrence_2001}\cite{Jeffrey_2008}\cite{Eysenbach_2006}\cite{Antelman_2004}
\item Tax-Payer-Agrument
Durch Steuergelder finanzierte Forschung ist dem Steuerzahler im Rahmen konventioneller wissenschaftlicher Kommunikation nicht immer unentgeldlich zugänglich, obwohl er im Rahmen öffentlich-geförderter Forschungsprogramme die Forschung bereits finanziert hat. Darüber hinaus stellt sich die Frage nach dem bestmöglichen Einsatz der monetären Ressourcen \cite{Glasziou_2014} \cite{altman_1994_scandal}.
\item Economic Promotion Argument
Bisher profitieren wirtschaftliche Unternehmungen nur unzureichend von staatlich-finanzierter wissenschaftlicher Kommunikation, dabei könnte eine schnellere, kommerziell verwertbare und umfassendere Bereitstellung der wissenschaftlichen Inhalte einen eklatanten Beitrag zur non-monetären Wirtschaftsförderung darstellen. Im Rahmen der offenen und schnelleren Verbreitung von wissenschaftlichen Informationen können neue Geschäftmodelle entstehen.
\item Digital Divide Argument
Der offene Zugang zu Publikationen ermöglicht neue Möglichkeiten für die Überwindung der sozialen, nationalen und globalen Wissenskluften  zwischen bildungsfernereren und -affineren Bevölkerungsteilen und -schichten der Welt . Der Mehrwert und die Chance von wissenschaftlichen Informationen für die Bewegung der offenen Bildungsmaterialien ist bisher auch noch nicht ausgeschöpft\cite{heise_lernen_2013}.
\item Validation & Reputation-Argument
Die Entwicklung neuer Verfahren, die die Aktivität und Qualität eines Forschers umfassender, transparenter und demokratischer messbar und kommunizierber machen, als im bestehenden Reputations- und Förderungssystem \cite{chalmers_2009_avoidable_waste}. Wissenschaftsevaluation wird durch Offenheit effizienter.
\item Paradoxon of Information Argument
Überwindung des bestehenden Informationsparadoxons bei der Verbreitung und Vermarktung von wissenschaftlichen Inhalten. Hierbei handelt es sich um das Problem, dass es schwer ist eine Information kommerziell zu verwerten ohne zu viel über Inhalt und Qualität auszusagen. Eine Entkommerzialisierung des Vertriebs von Wissen  würde das Informationsparadoxon aufheben.
\item Science communication Crisis-Argument
Durch die Öffnung von der wissenschaftlichen Kommunikations- und Reputationsprozesse besteht die Möglichkeit der vorherrschende Zeitschriften- und Monographienkrise durch neue Geschäftsmodelle zu begegnen.
\item Interdicipline & International Exchange/Collaboration Argument
Die Globalisierung in der Wissenschaft führt immerstärker zu internationalem Austausch und zur internationalen Zusammenarbeit von Wissenschaftlern . Doch das gilt nicht nur für die grenzenüberschreitende Zusammenarbeit in Bezug auf die lokale Verortung sondern auch für die Interdisziplinarität der Forschungsvorhaben. Die Öffnung von Wissenschaft ermöglicht also auch Fächerfremden Wissenschaftlern Zugruff auf Publikationen und damit auf Wissensressourcen für die eigene Arbeit .
\item Sustainable Access & Archiving Argument
Nur Offenheit im Sinne von Verwertbarkeit ermöglicht es in dezentralen Strukturen wie der des Internets alle Informationen nachhaltig und unabhängig voneinander zu speichern. Im Falle von Natur- oder anderen Katastrophen ermöglicht die digitale Ablage auf mehreren Kontinenten eine präservierung von Wissen undabhängig von lokalen Gegebenheiten oder Bedingungen.
\end{enumerate}

\textbf{Demgegenüber} stehen aber auch Argumente gegen die Öffnung der wissenschaftlichen Prozesse und Publikationen:
\begin{enumerate}
\item 	Quality-Argument
Die Befürchutung, das die Qualität auf Grund von schlechten oder nicht vorhandenen wissenschaftlichen Überprüfungsmechanismen leidet. Hauptargument ist das durch ein Autorengebühren finanziertes Publikationsmodell keinen klaren Anreiz für Ablehnung bietet.
\item Archiving-Argument
Die Sicherstellung der Langzeitarchivierung und die Garantierung der langfristigen Auffindbarkeit sowie Bereitstellung der Dokumente kann im Auge der Kritiker von Offenheit in Wissenschaft und Forschung nicht durch alternative digitale Strukturen gewährleistet werden. 
\item Authenticity-Argument
Forscherinnen und Forscher befürchten durch die dezentrale und offene Handhabung ihrer Texte und Arbeiten, dass diese im Zeitablauf inhaltlich nicht mehr unverändert zuordnenbar ihrem Autor sind.
\item Rightsmanagement-Argument
Hierbei handelt es sich um die Verpflichtung für Mitarbeiter staatlich finanzierter Forschungsinsitutionen alle Texte elektronisch frei und offen zu publizieren. In dem 2009 veröffentlichten "Heidelberger Appell" \cite{faz_heidelberger_apell_2009} kritisieren zahlreiche Autoren, Wissenschaftler, Verleger und Publizisten, dass das “verfassungsmäßig verbürgte Grundrecht von Urhebern auf freie und selbstbestimmte Publikation” … “derzeit massiven Angriffen ausgesetzt und nachhaltig bedroht” ist. Weiter sehen die Unterzeichner „weitreichende Eingriffe in die Presse- und Publikationsfreiheit, deren Folgen grundgesetzwidrig wären“ \cite{ITK_2009} und die Befürchtung, dass die Freiheit von Forschung und Lehre gefährdet ist \cite{Jochum_2009}. 
\item (Re-)Financing-Argument
Die unklare Refinanzierung der Öffnung von Wissenschaft ist eines der Kernargumente gegen das offene Publizieren von Arbeiten und Daten. Die Befürchtung ist, das ein solches System überhaupt nicht finanziert werden kann, konnte bisher nicht ausgeräumt werden.
\item Sustainability-Argument
\item Ressource-Allocation-Argument
Die Befürchtung, dass die Vergabe von Fördermittel und für die Karriere wichtige Aspekte der Reputationsbildung durch offenen System nicht Rechnung getragen wern kann ist eine weiteres Argument der Kritiker der Öffnung von Wissenschaft und Forschung. Eine Mittelvergabe zu gunsten populärer Forschung und damit eine Aushöhlung des wissenschaftlichen Systems in Ihrer Fächer und Facettenvielfalt wäre eine unmittelbare Folge dessen.
\item Open-Caring-Argument
Wissenschaftlerinnen und Wissenschaftler fürchten durch den Zwang zu umfassenderen Bereitstellung von Publikationen und gegebenenfalls soagar Daten einen nicht unwesentlichen zeitlichen Mehraufwand für die Öffnung ihrer Arbeiten. Sie möchten aber möglichst wenig Zeit für die Veröffentlichung, Bereithaltung und Verbreitungung ihrer wissenschaftlichen Arbeiten aufbringen.
	Aufwand für Offenheit im Alltag des Wissenschaftlers
\item Scientific-Freedom/Loss of Idea-Diversity-Argument
Angst dass durch Offenheit und Transparenz Forschungsförderung und Öffentlichkeit nur die wissenschaftlichen Projekte fördern und unterstützen, die von der Öffentlichkeit verstanden werden. Dabei stellt Wissen, vorallem aber Grundlagenwissen ein "öffentliches Gut" dar, "dessen Wert von der Öffentlichkeit nur schwer beurteilt werden kann"\cite{osterloh2008anreize}. Darüber hinaus herrscht die Annahme, dass im Rahmen von zunehmender Kollaboration und der Effizienz der elektronischen Suche die Diversität von wissenschaftlichen Meinungen und Projekten zu einem gleichen oder ähnlichem Thema eingeschränkt wird\cite{Evans_2008}.
\item Interpretations-Argument
Eine der weiteren Ängste der wissenschaftlichen Community ist die Angst vor der Fehlinterpretation ihres kommunizierten Wissens sowie der Verlust der Kontrolle über die Informationen\cite{gibbons_1994}. Dabei steht vor allem die Befürchtung im Vordergrund, dass die offen veröffentlichten Arbeiten genutzt werden um die Arbeit zu miskreditieren oder gezielt zur Falschinfromation der Öffentlichkeit zu nutzen.
\item Transparent-Research-Intentions-Argument
Mit den Forderung nach Offenlegung des gesamten Forschungsprozess erfolgt auch die Forderung nach "Transparenz der Interaktion zwischen Sponsoren (insbesondere kommerzielle Förderer wie die Pharma- und Medizinprodukteindustrie) und Auftragnehmern" \cite{Stengel_2013} 
\end{enumerate}


\section{Defizite}
Viele der Erklärungsansätze für den Paradigmenwechsel hin zur Öffnung der Wissenschaft basieren auf Annahmen, in denen ein direkter Zusammenhang von technischen Entwicklungen unmittelbar auf (wissenschafts-)politische und kulturelle Bewegungen geschlossen werden. Darüber hinaus beschränkt sich die Perspektive primär auf den Zugang zum Ergebnis von Wissenschaft und weniger auf die Öffnung des gesamten Prozesses. Die theoretische Auseinandersetzung mit der Geschlossenheit des wissenschaftlichen Diskurses  auf der Einen und mit den Treibern und Bremsern im realen wissenschaftlichen Prozess werden in der gängigen Literatur auf der anderen Seite, wird nicht genügend berücksichtigt. Hier wird vor allem die Verbindung zwischen wissenschaftlicher Reputation und Geschlossenheit des Wissensproduktionsprozesses nur selten erörtert. Als weiteres Manko kann angeführt werden, "dass die Deliberation bezüglich und die Verbreitung von Wissen ein stabiles Set von Infrastrukturen braucht"\cite{kelty_2004}, nach denen man heute noch vergeblich sucht.
\section{Entwicklung der Fragestellung}
Nach der theoretischen Diskussion sollen in diesem Teil die offenen inhaltliche Fragestellungen dargestellt werden.  Die Fragestellung dieser Arbeit soll dabei drei Kriterien entsprechen :
\begin{enumerate}
\item aus ihrer Formulierung soll klar hervorgehen wie sie verstanden werden kann
\item sie soll im Kontext der wissenschaftlichen Disziplin einen klaren definierten Ort haben 
\item ihr Gegenstand muss eindeutig sein.
\end{enumerate}
Die vorläufige forschungsleitende Hypothese in dieser Arbeit ist (siehe auch 3.1), dass Open Access sich in der Übergangsphase zu Open Science befindet. Die daraus ableitende Fragestellung umfasst dabei zum einen die theoretische Bedeutung von Offenheit im Rahmen des wissenschaftlichen Diskurs- und Machtbegriffs (Kapitel 4.2) aber auch die empirische Frage nach den Motiven und Beweggründen für Wissenschaftler, Verlagen und Universitäten diese Offenheit auf Wissenschaft in den unterschiedlichen Disziplinen zu ermöglichen oder zu verhindern (Kapitel 4.1). Abschließend soll erörtert werden, welche mögliche Auswirkungen auf Selbstverständnis von Wissenschaft, auf das wissenschaftliche Kapital sowie auf die unterschiedlichen Disziplinen durch diesen Prozess der Öffnung zu erwarten sind (Kapitel 5).
