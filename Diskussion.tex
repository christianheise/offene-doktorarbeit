\chapter{Diskussion: Wissenschaftliche Kommunikation im Wandel - Scientific Steady State vs. Second Scientific Revolution}

\begin{quote}
\textbf{"The sciences are too good merely to avert attention from what science does."}
\end{quote} \cite{Kittler_2004}

Der digitale Wandel im wissenschaftlichen Publikationssystem stellt eine Chance dar, das tradierte System so zu justieren, dass es ohne Qualitätsverlust, ohne Einschränkung der Wissenschafts- oder Publikationsfreiheit und vorbehaltlich einer angemessenen Zuordnung der Urheberschaft zu einer umfassenderen und schnelleren Verteilung von Wissen in der wissenschaftlichen Gemeinschaft auch die gesamte Gesellschaft erreichen kann – so die Hoffnung der Befürworter und Befürworterinnen des Wandels hin zur Öffnung wissenschaftlicher Kommunikation im Rahmen von Open Access. Der Mediziner und Wissenschaftshistoriker Michael Hagner fasst diese Chance wie folgt zusammen: "Zum ersten Mal überhaupt in der Geschichte der Wissenschaften verfügen diese über ein Medium, das ihnen eine auf ihre Interessen hin zugeschnittene Tagesaktualität offeriert, die dem nahekommt, was Massenmedien wie Tageszeitung, Radio und Fernsehen früher bereits der Allgemeinheit anbieten konnten" \cite[:50]{Hagner_2015}.

Das Potenzial der Digitalisierung und die damit einhergehenden Möglichkeiten der Öffnung wissenschaftlicher Kommunikation sind jedoch noch viel umfassender und weitreichender. Neben der Chance für einen freien und offenen Zugang zu finalen wissenschaftlichen Publikationen eröffnet sich im Rahmen dieser Entwicklung erstmals auch die Möglichkeit für den umfassenden Zugriff auf Daten anderer Wissenschaftler und Wissenschaftlerinnen und die Öffnung des gesamten wissenschaftlichen Erkenntnisprozesses. Im Verlauf der Arbeit wurde gezeigt, das in diesem Zusammenhang umfangreiche Konsequenzen auf die tradierte Klassifikation, die Abgrenzung und Einordnung von wissenschaftlicher Kommunikation, die bestehenden Paradigmen wissenschaftlicher Praxis, die Ressourcenverteilung, die Kriterien für den Reputationserwerb und auf die regulierenden Prinzipien im Rahmen wissenschaftlicher Diskurse zu vermuten sind. Somit scheint die Überlegung legitim, ob es sich bei diesen teilweise noch bevorstehenden neuen Möglichkeiten und Veränderungen der Tradition wissenschaftlicher Praxis "nur" um eine Anpassung der bestehenden wissenschaftlichen Praxis oder um eine wissenschaftliche Revolution handelt \cite{Kuhn_2012}.

Ein Ziel dieser Arbeit war die Darstellung und Analyse der theoretischen Annahmen und unterschiedlichen Definitionsversuche rund um die Etablierung des Zugangs zu und des Zugriffs auf wissenschaftliche Erkenntnisprozesse. Es wurden Hindernisse und Katalysatoren für die Etablierung der beiden Konzepte anhand theoretischer und empirischer Betrachtungen ausgeführt, Grundannahmen aus der Literatur extrahiert, um die Auffassungen und Meinungen der befragten Wissenschaftler und Wissenschaftlerinnen ergänzt und den eigenen Erfahrungen in einem Selbstversuch unter den praktischen Gegebenheiten im wissenschaftlichen Alltag gegenübergestellt. Dabei konnte eine Diskrepanz zwischen der Idee der Öffnung wissenschaftlicher Kommunikation und dem Interesse der wissenschaftlichen Gemeinschaft an dieser Idee sowie den Möglichkeiten und der gelebten Praxis in der wissenschaftlichen Realität belegt werden.

In diesem Kapitel werden die Ergebnisse der durchgeführten Befragung, des Selbstexperiments, die Argumentationsstränge der Debatten zum Wandel der wissenschaftlichen Kommunikation in der Literatur im Rahmen der Digitalisierung und die zu Beginn der Arbeit formulierten Forschungsfragen abschließend diskutiert. Ziel ist es, zu einer differenzierten, zusammenfassenden Betrachtung der Ergebnisse der theoretischen Ausarbeitung, der Befragung und des Experiments zu gelangen und diese in den Kontext der Fragestellungen der Arbeit zu stellen und kritisch zu diskutieren.

\section{Wesentliche Aspekte von Open Access und Open Science}

Die Literaturrecherche machte deutlich, dass die unterschiedlichsten Definitionsversuche der Begriffe Open Access und Open Science bestehen. Im Folgenden wird eruiert, welche Annäherungen an die Begrifflichkeiten als besonders sinnvoll erachtet werden, welche Vermutungen in der Literatur darüber vorherrschen, warum die Öffnung von Wissen in den verschiedenen wissenschaftlichen Disziplinen unterschiedlich stark etabliert ist, und welchen Einfluss Offenheit und der freie Zugang auf das wissenschaftliche Reputationssystem haben kann.

Eine eng gefasste begriffliche Abgrenzung der beiden Konzepte ist aktuell noch nicht möglich. Zu facettenreich sind die unterschiedlichen Anknüpfungspunkte sowie die Entwicklungen und Eigenheiten in den unterschiedlichen wissenschaftlichen Disziplinen und Arbeitsbereichen. Als zentraler Annäherungspunkt dient in dieser Arbeit deshalb die präzise Definition des gemeinsamen Attributs "Open" und eine damit einhergehende Verknüpfung mit der Open-Definition \cite{Open_Definition_2014}. Die Open-Definition dient dabei als Rahmen für die rechtliche, technische und politische Ausrichtung der beiden Konzepte und beschreibt vor allem die Grundlagen, wann zum Beispiel ein wissenschaftliches Werk, ein beliebiger Teil des wissenschaftlichen Erkenntnisprozesses oder ein Datensatz den Bedingungen entspricht, um kompatibel mit der Idee von Offenheit zu sein.

\subsection{Open Access}

Open Access hat sich in den letzten 25 Jahren zu der meistgenannten Lösung für die beschriebenen Herausforderungen im wissenschaftlichen Kommunikations- und Publikationssystem entwickelt \cite{Brembs_2015}. Seit den ersten Experimenten mit der Öffnung des Zugangs zu wissenschaftlichen Publikationen in der zweiten Hälfte des 20. Jahrhunderts existieren mehrere heterogene Definitionsansätze von Open Access und es bestehen unterschiedliche Auffassungen über die verschiedenen Modelle und Wege hin zu dem Ziel der Öffnung wissenschaftlicher Kommunikation. War die Eingrenzung von Open Access bis Anfang der 2000er Jahre noch sehr vage, haben die "drei Bs" (Budapest Open Access Initiative, Bethesda Stellungnahme und Berliner Erklärung) einen Beitrag zur Vereinheitlichung der Forderungen geleistet. Die Erklärungen stimmen in wesentlichen Merkmalen überein \cite{Albert_2006}, unterscheiden sich aber in ihrem Ausgestaltungs-, Auswirkungs- und Bezugsrahmen \cite{Naeder_2010}. Das erschwert die Abgrenzung des Begriffs "Open Access" und die Reaktion auf die Forderungen nach der Öffnung des Zugangs zu wissenschaftlichen Erkenntnissen nachhaltig.

Näder fasst diese Situation wie folgt zusammen \cite{Naeder_2010}:
"Selbst nach der vollständigen Rezeption aller drei Erklärungen bleibt schließlich ein gewisser Interpretationsspielraum: Etwa hinsichtlich der Frage, ob ein Dokument auch dann dem Open-Access-Gedanken entspricht, wenn die in den beiden jüngeren Erklärungen geforderten Zusatzmaterialien nicht mitgeliefert werden und das Dokument erst nach Ablauf der vertraglichen Schutzfrist online zugänglich gemacht wird."

Will man dennoch zu einer gemeinsamen Rahmendefinition von Open Access kommen, könnte ein alternativer Ansatz das Streben nach einer gemeinsamen und eindeutigen Definition des Attributs "Open" sein. Das erscheint sinnvoll, um die ideelle Entwicklung von Offenheit in Wissenschaft und Forschung in ihrer ursprünglich gedachten Form auch in anderen Bereichen des offenen Wissens, in denen der Begriff "Open" verwendet wird, wie zum Beispiel im Rahmen offener Verwaltungs- und Regierungsdaten, nicht nachhaltig zu gefährden.

Eine Grundlage dafür bietet die Open-Definition \cite{Open_Definition_2014}, die im Gegensatz zu den meisten Erklärungen von Open Access ständig weiterentwickelt wird und eine klare Abgrenzung zu "nicht open" beinhaltet, ohne Bezug auf konkrete Publikationsformen, Prozesse oder den Kern ihrer Aussagen zu verwässern. Nach dieser Definition ist Wissen dann als "open" zu bezeichnen, "wenn jede/r darauf frei zugreifen, es nutzen, verändern und teilen kann – eingeschränkt höchstens durch Maßnahmen, die Ursprung und Offenheit des Wissens bewahren" \cite{Open_Definition_2014}. In der Definition sind die Bereiche "Offene Werke", "Offene Lizenzen" und "Akzeptable Bedingungen" klar beschrieben. Was die Definition auslässt, ist ein zeitlicher Horizont zwischen Erstellung und Veröffentlichung der Inhalte.

Eine weitere Alternative könnte die Definition des Gegensatzes von "Open" darstellen. Zum Beispiel über die Definition von "Close Access" und die Präzision von dem, was im Rahmen der Digitalisierung und Öffnung von Wissenschaft und Forschung seitens der wissenschaftlichen Gemeinschaft als nicht-wünschenswert gilt, könnte das Dilemma der unklaren Definition von "Open Access" benannt werden und trotzdem an den ursprünglichen kulturellen, politischen und gesellschaftlichen Werten der Öffnung von wissenschaftlicher Kommunikation sowie an den Eigenheiten der unterschiedlichen Fächer und Arbeitsweisen festgehalten werden.

Diese Herangehensweise, eine exkludierende Definition zu erstellen, erscheint auch deshalb als eine zielführende Alternative, weil die aktuellen Herausforderungen im Rahmen der Öffnung von wissenschaftlicher Kommunikation einfach zu heterogen erscheinen, als dass sie nur über die Forderungen der ursprünglichen Erklärungen der "drei B's" und die Definition, was Open Access ist und was nicht, gemeistert werden können. Darüber hinaus könnte eine solche Definition aus der wissenschaftlichen Gemeinschaft heraus die veränderten Rahmenbedingungen der wissenschaftlichen Traditionslinien berücksichtigen unter denen Offenheit möglicherweise auch weitere Bezugspunkte aufweist als die genannten Debatten um die Öffnung wissenschaftlicher Inhalte für die Gesamtgesellschaft.

Bisher hat die Zeit, die seit der ersten Grundsatzerklärung zu Open Access in Budapest vergangen ist, gezeigt, dass die Abhängigkeit und Stabilität des wissenschaftlichen Kommunikationssystems größer ist, als zunächst angenommen und der "Faustische Pakt (...) stabiler ist als gedacht" \cite[:75]{Hagner_2015}. Als Konsequenz zeigt das tradierte wissenschaftliche Publikationssystem bisher eine gewisse Veränderungsresistenz im wissenschaftlichen Kommunikationssystem und auch nach zwei Jahrzehnten bleibt das etablierte Publikationssystem der Verlage weitgehend stabil \cite{Hanekop_2014}. Die Thesen von der Veränderungsresistenz und von der Stabilität des aktuellen wissenschaftlichen Kommunikationssystems konnten auch acht Jahre nach der letzten Befragung \cite{Hanekop_2007} erneut durch die Ergebnisse der Befragung im Rahmen dieser Arbeit bestätigt werden (siehe Kapitel 5).

\subsection{Open Science}

Open Science umfasst weitreichende Modernisierungsvorhaben und betrifft neben dem reinen Zugang zu finalen wissenschaftlichen Erkenntnissen (Open Access) die größtmögliche (und wünschenswerte) Öffnung des gesamten wissenschaftlichen Erkenntnisgewinnungsprozesses. Open Science greift demnach die politischen Ideale von Open Access auf und ergänzt sie um die notwendigen weiteren praktischen Aspekte zur Veränderung des wissenschaftlichen Kommunikationssystems. Die Probleme bei der Umsetzung des Konzepts ähneln dabei den Herausforderungen von Open Access und beziehen sich ebenfalls auf Unklarheiten bei der konkreten Ausgestaltung von "Offenheit". Diese sind bei Open Science jedoch noch umfassender, da sie sich nicht nur auf den Zugang zu publiziertem Wissen und finalen wissenschaftlichen Erkenntnissen, sondern auch auf den Zugriff auf den gesamten wissenschaftlichen Prozess beziehen. Open Science beeinflusst nicht nur das Thema Zugang zu publikationsfertigem Wissen, sondern ebenso die Themen Steuerung, wissenschaftliche Prozesse, Datenschutz, Qualitätssicherung und Archivierung sowie grundlegende Aspekte wie die Freiheit der Wissenschaft.

Im Gegensatz zu Open Access gibt es bisher nur wenige Erklärungen und Statements, die eine Annäherung an eine klare Definition von Open Science ermöglichen. Die Ziele der Öffnung wissenschaftlicher Forschungsprojekte nach den Kriterien von Open Science bauen überwiegend auf vergleichbaren Ansätzen der Deklarationen von Open Access auf und können als Weiterentwicklung dieser betrachtet werden. Wie bei dem Konzept zur Öffnung des Zugangs zu finalen wissenschaftlichen Publikationen ist eine solche fächerübergreifende Definition im Moment noch sehr unpraktikabel. Daher scheint auch bei Open Science eine Annäherung an die Open-Definition als Rahmen zielführender zu sein. Die Konzentration auf die Open-Definition als Rahmen und eine fehlende Aushandlung der konkreten Wege und der Kriterien von Open Science birgt jedoch auch die Gefahr, dass das genannte Ziel von Open Science in der konkreten Umsetzung verwässert. Andererseits zeigen die Entwicklungen der letzten Jahre, dass genau dieser Rahmen wichtig ist, um durch experimentelle Ansätze (wie dem offenen Verfassen von wissenschaftlichen Arbeiten) die konkreten Kriterien für Open Science zu verhandeln, ohne dabei das Kernziel des möglichst umfassenden Zugriffs auf den wissenschaftlichen Prozess durch alltägliche Herausforderungen bei der Umsetzung aus den Augen zu verlieren.

Die Umsetzung solcher Open-Science-Experimente und Initiativen kann und darf dabei nicht nur auf der Grundlage von Vorgaben, Gesetzen oder Richtlinien erfolgen, sondern muss primär den Aufgaben dienen, verbesserte Rahmenbedingungen dafür zu schaffen, neues Wissen zu produzieren sowie zu verbreiten und dem gesellschaftlichen Auftrag des Wissenschaftssystems gerecht zu werden. Dazu müssen Anreize für die einzelnen Wissenschaftler und Wissenschaftlerinnen so gesetzt werden, dass deren Eigeninteresse mit dem Wohl der Wissenschaft und damit dem der Öffentlichkeit harmonieren \cite{Brembs_2015}. Diese Anreize sollten dabei nicht nur wissenschaftsextern definiert und praktiziert werden, sondern müssen von der wissenschaftlichen Gemeinschaft selbst koordiniert und in das bestehende System der wissenschaftlichen Kommunikation eingebunden werden.

Bezüglich der zeitlichen Dimension der Veröffentlichung von wissenschaftlichen Inhalten, die die Open-Definition nicht abdeckt und die auch in den Erklärungen von Budapest, Bethesda und Berlin nur unzureichend definiert ist, könnte Open Science eine Lösung darstellen. Denn es wird in diesem Konzept eine unmittelbare, möglichst umfassende und für jeden frei verfügbare Veröffentlichung der Informationen im Rahmen wissenschaftlicher Erkenntnisprozesse angestrebt. Während bei Open Access die finale Publikation, maximal angereichert durch die darin referenzierten Daten, im Vordergrund steht, die in vielen Fällen auch erst nach einer gewissen Embargofrist zu veröffentlichen ist, greift die Verbreitung von Inhalten im Sinne von Open Science viel früher und weiter. Die finale Publikation ist dabei eher als nachgelagertes und abschließendes Ergebnis zu betrachten.

Folgt man dieser Betrachtungsweise, sind die zeitlichen und rechtlichen Herausforderungen bei der Veröffentlichung finaler wissenschaftlicher Erkenntnisse in Form von Open-Access-Publikationen bei Open Science als nachgelagert zu betrachten. Bei einer Realisierung des Konzepts von Open Science geht es um die konstante Veröffentlichung wissenschaftlicher Ergebnisse im Rahmen von öffentlich-finanzierten Arbeitsumgebungen unter den Bedingungen der Open-Definition. Dabei ist die möglichst umfassende Akkumulation von Beweisen ein wesentlicher Teil der wissenschaftlichen Methode der Selbstkorrektur \cite{Nosek_2015}. Eine größtmögliche Offenheit des wissenschaftlichen Erkenntnisprozesses ist demnach auch zentral für das Ziel der Wahrheitsfindung.

Angelehnt an die Kurzzusammenfassung der Sprachwissenschaftlerin Gabriele Graefen für die Entwicklung von Wissenschaft, gilt dennoch, dass für eine erfolgreiche Implementierung von Open Science in den wissenschaftlichen Alltag ein gesamtgesellschaftlicher Bedarf für das zu öffnende Wissen bestehen und die entsprechende Leistung von Individuen zu persönlichen Vorteilen führen muss. Nur dann kann eine Umorientierung von geschlossener, individueller wissenschaftlicher Betätigung hin zu offener, gesellschaftlich anerkannter und zur Kenntnis genommener, kollektiv betriebener Wissenschaft stattfinden \cite{Graefen_2007}. Ergänzend muss die detaillierte Ausgestaltung dieser Entwicklung unter Beteiligung der Wissenschaftler und Wissenschaftlerinnen vonstattengehen, da das System sonst Gefahr läuft, sich zu "sporadischer individueller wissenschaftlicher Betätigung" \cite{Graefen_2007} zu­rück­zuent­wi­ckeln.

Eine Voraussetzung für Open Science ist folglich, dass "wissenschaftliche Daten, Codes und Volltexte sowie zum Teil das Lese-, Gutachter- und Lehrverhalten der Wissenschaftler zu einem großen Teil transparent werden" \cite{Brembs_2015}. Von dieser Forderung ausgeschlossen sind die Informationen, die die Privatsphäre der Wissenschaftler und Wissenschaftlerinnen sowie der Untersuchungssubjekte betreffen und den Datenschutz im Rahmen der Datenerhebung, Verbreitung und Verarbeitung negativ beeinflussen. Als weitere Voraussetzungen sollten nach Graefen \cite{Graefen_2007}:
\begin{itemize}
\item die technischen Möglichkeiten der schnellen Erstellung, Vervielfältigung und Verbreitung von wissenschaftlicher Kommunikation,
\item die Zugänglichkeit zum gesamten Forschungsprozess,
\item die Möglichkeit Wissenschaft(en) von anderen Formen der Information und Kommunikation zu unterscheiden,
\item die ökonomische und politische Nutzbarkeit von Wissenselementen, so dass Forschung als ein Mittel der Verwertung gelten kann,
\item die durch Ausbildungsprozesse gefestigte Existenz einer beruflich mit Forschung befassten 'Schicht' von Fachleuten, Wissenschaftlern und Wissenschaftlerinnen
\end{itemize}
betrachtet werden.

Die Etablierung von Open Science hängt demnach nicht von der Ausgestaltung einer abstrakten Definition ab, sondern vielmehr von der konkreten Gestaltung der offenen wissenschaftlichen Praxis und ihrer Grenzen. Die offene Erstellung dieser Arbeit als Experiment und die Entwicklung von Handlungsempfehlungen für die offene Anfertigung wissenschaftlicher Arbeiten (siehe Kapitel 6) leisten hierfür einen konkreten Beitrag für die Gestaltung und die Durchführung von Open Science. Dass diese Arbeit die erste Doktorarbeit ist, die mit dem Anspruch an einen größtmöglichen Grad an Offenheit und unter den Bedingungen der Open Definition erstellt wurde, zeigt dabei auch einen Mangel an Experimentierfreudigkeit seitens der wissenschaftlichen Gemeinschaft für die Gestaltung des Aushandlungsprozess über die Zukunft der wissenschaftlichen Kommunikation im Rahmen der Forderungen nach Offenheit und unter den veränderten Bedingungen der Digitalisierung.

\section{Haupteinflussfaktoren, die die Öffnung von Wissenschaft und Forschung beeinflussen}

Ein weiteres Ziel dieser Arbeit war die Identifikation von Faktoren, welche die Öffnung von Wissenschaft und Forschung beeinflussen. Diese Faktoren wurden anhand der Literatur herausgearbeitet und im Rahmen der durchgeführten Umfrage abgefragt. Die erhobenen Daten wurden statistisch ausgewertet und zusammenfassend dargestellt.

In der Literatur und von den Befragten wurden als Argumente für die Öffnung vor allem die beschleunigte Wissensverbreitung und die neuen Möglichkeiten für die Kommunikation innerhalb und außerhalb der wissenschaftlichen Gemeinschaft genannt. Demgegenüber wurden als Hindernisse neben den fehlenden etablierten Reputationskriterien für die Bewertung von offener Wissenschaft auch die Gefahr der Fehlinterpretation und Falschinformation durch die freie und umfassende Verfügbarkeit von wissenschaftlichen Informationen genannt.

Der ebenfalls häufig genannte erhöhte zeitliche Mehraufwand für die Bereitstellung der wissenschaftlichen Publikationen und Forschungsdaten sowie die bisher geringe Veröffentlichung von offenen Inhalten stärkt die Vermutung, dass Wissenschaftler und Wissenschaftlerinnen selbst keinen großen Druck verspüren, ihr Veröffentlichungsverhalten zu verändern. Das mag auch an dem Umstand liegen, dass sie trotz der Publikationskrise und anderer Faktoren, die zu einem Marktungleichgewicht im wissenschaftlichen Kommunikationssystem geführt haben, in einer komfortablen Situation sind, beziehungsweise von den Auswirkungen der Krise bisher kaum oder (noch) nicht betroffen sind.

In der Debatte und in den Erhebungen rund um die Öffnung von Wissenschaft und Forschung werden viele Faktoren genannt, die den bisher für die Gesamtgesellschaft intransparenten und geschlossenen wissenschaftlichen Erkenntnisprozess hin zu einer Öffnung in unterschiedlichster Weise beeinflussen. Die Argumente erscheinen dabei als sehr vielseitig und facettenreich, konzentrieren sich aber im Kern auf acht Aspekte:

\begin{itemize}
\item Im Rahmen der Alltagsbetrachtung beeinflussen rechtliche Bedenken die offene wissenschaftliche Kommunikation (siehe Ergebnisse der Befragung in Kapitel 5).
\item Fehlende Reputations- und Qualitätssicherungsmechanismen hindern Wissenschaftler und Wissenschaftlerinnen daran, die eigenen wissenschaftlichen Erkenntnisprozesse zu öffnen oder die Ergebnisse frei zur Verfügung stellen \cite{herb_2015}.
\item Fehlende Infrastrukturen beziehungsweise die fehlende Integration in bestehende Strukturen sowie fehlendes Know-how der Wissenschaftlern und Wissenschaftlerinnen stellen weitere Gründe für den geringen Öffnungsgrad in Wissenschaft und Forschung dar \cite{European_Commission_2015b}.
\item In den wissenschaftstheoretischen Diskursen wird die Wissenschafts-, Presse- und Publikationsfreiheit oft als Argument gegen den Druck wissenschaftliche Inhalte zu öffnen, angeführt \cite{Fehling_2014}. Mithilfe der Befragung konnte diese Annahme allerdings nicht mehrheitlich bestätigt werden (siehe Ergebnisse der Befragung in Kaptiel 5).
\item Ergänzend zu anderen Studien ist festgestellt worden, dass die Verfügbarkeit von digitalen Technologien, deren erhöhte Kapazität und Verbreitung einen wesentlichen Einflussfaktor auf die Entwicklung von Open Access und Open Science darstellen \cite{European_Commission_2015b}. Der Wunsch nach größerer Verbreitung wissenschaftlicher Erkenntnisse und nach mehr Möglichkeiten zum kollaborativen Arbeiten spielt bei dieser Betrachtung eine herausragende Rolle und begünstigt die Entwicklung (siehe Ergebnisse der Befragung in Kaptiel 5).
\item Dem Interesse an der Öffnung wissenschaftlicher Kommunikation steht allerdings die Befürchtung eines hohen Mehraufwands für die wissenschaftliche Praxis gegenüber. Die Befragung hat gezeigt, dass ein großer Teil der befragten Wissenschaftler und Wissenschaftlerinnen bei der Öffnung der Kommunikation wissenschaftlicher Prozesse einen großen Mehraufwand (siehe Kapitel 5) erwartet.
\item Im Rahmen des offenen Verfassens dieser Arbeit ist deutlich geworden (siehe Kapitel 6), dass bisher für die möglichst umfangreiche und offene Bereitstellung wissenschaftlicher Kommunikation nur wenige Tools und adäquate Dienste zur Verfügung stehen. Neben der geringen Anzahl an kommerziellen Lösungen ist der größte Teil der wissenschaftlichen Open-Source-Forschungssoftware qualitativ eher minderwertig, ineffizient, undokumentiert und wird nur selten weiterentwickelt oder gepflegt \cite{Hey_2015}.
\item Dass die befragten Wissenschaftler und Wissenschaftlerinnen die vorherrschenden Zeitschriften- und Monografienkrise als einen eher unwichtigen Katalysator für die Öffnung wissenschaftlicher Kommunikation identifiziert haben (siehe Kaptiel 5), unterstützt die Annahme des geringen Interesses an dem Thema und dass sich viele Akteure der "Konflikthaftigkeit" \cite{Kaldewey_2010} der Situation, in der sich das System befindet, nicht bewusst sind. Sie nährt die Befürchtung, man könne trotz effizienterer Verfahren und neuer Technologien nichts an dem System ändern \cite{Parks_2002}.
\end{itemize}

Die Ergebnisse und der Vergleich mit der SOFI-Studie aus 2007 zeigen dabei deutlich, dass die Verschärfung der Situation in den letzten zehn Jahren bisher nur vereinzelt zu Verhaltensänderungen im Publikations- und Kommunikationsverhalten der Wissenschaftler und Wissenschaftlerinnen geführt haben. Dabei gibt es zwar Unterschiede zwischen den Disziplinen, dennoch beeinflussen maßgeblich die fehlenden Anreizsysteme für Wissenschaftler und Wissenschaftlerinnen auf regionaler, nationaler und internationaler Ebene den Prozess vom geschlossenen System wissenschaftlicher Kommunikation hin zu einer Öffnung negativ.

\section{Bedeutung der Konzepte von Open Access und Open Science im Rahmen wissenschaftlicher Reputation}

Wissenschaftliche Reputation basiert, wie im zweiten Kapitel dargestellt, auf einem System der Verbreitung neuer und der Adaption bestehender wissenschaftlicher Erkenntnisse. Unter dieser Prämisse sollte Offenheit und freier Zugang zu wissenschaftlicher Kommunikation als Grundpfeiler des wissenschaftlichen Diskurses und des Reputations- und Machtsystems selbstverständlich sein.

In der Praxis ist dieses System aber selbstreferentiell und auf die wissenschaftliche Gemeinschaft beschränkt. Es haben nur diejenigen Zugriff auf das System, die bereits Teil des Systems sind oder die beginnen, sich den Normen und Regeln des Systems anzupassen. Die Gesamtgesellschaft ist von dem Diskurs in vielen Fällen ausgeschlossen oder hat erst nach einer gewissen Zeit die Möglichkeit, auf Informationen und Erkenntnisse zuzugreifen. Das aktuelle System festigt demnach die Machtkonstellation derjenigen, die das System derzeit beeinflussen.

Im Rahmen der Identifikation von Katalysatoren und Hindernissen für den Prozess zur Öffnung der wissenschaftlichen Kommunikation stellen die fehlenden Reputationsmechanismen eine der größten Beschränkungen für die Verbreitung dar. Ferner wurden fehlende Qualitätssicherungsmaßnahmen in der Befragung im Rahmen dieser Arbeit (siehe Kapitel 5) und anderer Erhebungen \cite{European_Commission_2015b} besonders häufig bei den Argumenten gegen die Verbreitung von offener wissenschaftlicher Kommunikation erwähnt. Die bisher genutzten bibliometrischen Verfahren erfreuen sich noch immer großer Beliebtheit und haben eine Absicherungsfunktion bezüglich der vermuteten Qualität einer Publikation, die bisher noch kein digitales Äquivalent gefunden hat. Da auch die Forschungsförderung auf diesen tradierten Verfahren der Evaluation von Forschung aufbaut, ist diesbezüglich noch kein Wandel absehbar.

Es ist davon auszugehen, dass die Möglichkeit der Erlangung wissenschaftlicher Reputation einen Einfluss auf die Verbreitung des Konzepts von offenem Zugang zu Wissenschaft hat und nicht umgekehrt. Die bestehenden digitalen Bewertungssysteme haben in der Wissenschaftssteuerung allerdings bisher nur begrenzt Einzug gehalten. So kann noch keine abschließende Beantwortung der Frage stattfinden, wie genau die Öffnung der Kommunikation einen Einfluss auf das Konzept und die Kriterien der wissenschaftlichen Reputation innerhalb und außerhalb der wissenschaftlichen Community haben wird.

Auffällig bei der Auswertung der durchgeführten Befragung war die geringe Befürwortung und Praktizierung von Öffnung wissenschaftlicher Kommunikation bei den jüngeren Altersgruppen und denen, die am Anfang ihrer wissenschaftlichen Karriere stehen. Demgegenüber zeigte diese Altersgruppe das verhältnismäßig größte Interesse an den Forschungsdaten anderer Wissenschaftler und Wissenschaftlerinnen. Gerade bei den "reputationshungrigen" Nachwuchswissenschaftlern und -wissenschaftlerinnen scheint zwar ein Interessen an Offenheit, allerdings auch Zurückhaltung bei der Bereitschaft eigene wissenschaftliche Inhalte frei zu veröffentlichen, vorzuherrschen. Demnach ist davon auszugehen, dass die Sensibilisierung der Nachwuchswissenschaftler und -wissenschaftlerinnen für die Wichtigkeit der offenen Kommunikation ausbaufähig ist.

\section{Aufwand für die Öffnung des gesamten wissenschaftlichen Erkenntnisprozesses}

Im Vergleich zum Publizieren von Texten in einer geschlossenen Umgebung ist das offene Verfassen einer wissenschaftlichen Publikation noch immer mit Mehraufwand verbunden und teilweise ohne technische Vorkenntnisse nicht ohne Weiteres durchführbar – wie im Selbstexperiment gezeigt. Es ist zum einen darin begründet, dass die genutzten Softwareprodukte und -plattformen die Veröffentlichung der Arbeit und der gesamten Erkenntnis- und Kommunikationsprozesse noch nicht vollumfänglich und einfach ermöglichen. Zum anderen sind die Richtlinien, Rahmenbedingungen, Anreizsysteme und Vorgaben für wissenschaftliche Arbeiten an Universitäten und Forschungseinrichtungen nicht darauf ausgelegt "offen" angewendet zu werden.

34 Prozent der Befragten sahen einen erhöhten zeitlichen Mehraufwand für die Bereitstellung der wissenschaftlichen Publikationen und/oder der Forschungsdaten. 31 Prozent schätzten den Gesamtaufwand für das offene Publizieren als "gering" ein. Unter den Teilnehmern und Teilnehmerinnen im Rahmen der Online-Befragung gaben demgegenüber zwar nur 15 Prozent an, dass sie der (Mehr-)Aufwand davon abhält die eigenen wissenschaftlichen Inhalte ohne finanzielle, rechtliche oder technische Barrieren öffentlich zur Verfügung zu stellen, doch die tatsächliche Anzahl der veröffentlichen wissenschaftlichen Inhalte durch Umfrageteilnehmer ist bedeutend geringer. Rund 20 Prozent der befragten Wissenschaftler und Wissenschaftlerinnen waren nicht in der Lage den Aufwand zu bewerten. 28 Prozent der Befragten schätzten den Aufwand für die Veröffentlichung von Publikationen unter den Kriterien von Open Access als mittelgroß oder groß ein, für die Veröffentlichung von Forschungsdaten vermuteten sogar 55 Prozent einen großen Aufwand.

Diese Unsicherheit und die Hervorhebung des vermuteten Mehraufwands sowie fehlende Anreizsysteme hemmen die Weiterentwicklung hin zu einer möglichst umfassenden Öffnung des Systems. Es ist davon auszugehen, dass allein die Veränderungen im Rahmen der Forschungsförderung nicht ausreichen werden, um Wissenschaftler und Wissenschaftlerinnen perspektivisch davon zu überzeugen den gesamten Erkenntnisprozess zu öffnen. Weitere Experimente, wie die offene Ausarbeitung sowie die Erarbeitung von Handlungsempfehlungen für das offene Verfassen wissenschaftlicher Arbeiten im Rahmen der vorliegenden Arbeit, sind nötig, um die Auseinandersetzung mit Offenheit wissenschaftlicher Kommunikation zu befördern.

\section{Von Open Access zu Open Science: Anpassung der bestehenden wissenschaftlichen Praxis oder wissenschaftliche Revolution?}

Die forschungsleitende Hypothese dieser Arbeit lautete, dass sich Open Access in einer Übergangsphase befindet, die derzeit noch überwiegend von der reinen offenen Bereitstellung wissenschaftlicher Publikationen geprägt wird, langfristig aber zur Öffnung weiterer Teile der wissenschaftlichen Kommunikation als wesentliche Grundlage für den Wissenszuwachs in der Gesamtgesellschaft (Open Science) führen wird. Nach der Analyse der Debatten um die Öffnung von Wissenschaft und Forschung, der Ergebnisse der Befragung und den gewonnenen Erkenntnissen kann diese Hypothese bisher nicht bestätigt werden.

Trotz des mehrheitlichen Interesses an dem Zugang zu wissenschaftlichen Daten und Informationen anderer Wissenschaftler und Wissenschaftlerinnen, der grundsätzlichen Bereitschaft, Forschungsdaten und alle weiteren Informationen, die während der wissenschaftlichen Arbeit anfallen öffentlich zur Verfügung zu stellen, einer mehrheitlichen Unterstützung der Forderungen nach kostenfreiem Zugang zu allen wissenschaftlichen Publikationen für Leserinnen und Leser, fehlt es noch immer an einem Aushandlungsprozess über die genaue Ausgestaltung an "Offenheit". Bisher ist unklar, auf welche Bereiche der wissenschaftlichen Arbeit sich die Forderungen nach Öffnung genau beziehen, wie diese Offenheit unter realen Bedingungen praktiziert werden kann, und wie sich die Kriterien für die wissenschaftliche Arbeit sowie die Qualitäts- und Leistungsbemessung durch diese Entwicklung verändern sollen.

Zwar begünstigen die technologischen Entwicklungen auch ohne diese Ausgestaltung die Möglichkeiten für die Verbreitung und das Teilen von wissenschaftlichen Daten und Informationen im Erkenntnisprozess, dennoch ist dieser Prozess bisher eher als Begleiterscheinung der Digitalisierung zu betrachten. Der Kern der wissenschaftlichen Arbeit ist davon bisher nur begrenzt beeinflusst und die tatsächliche Verbreitung offener wissenschaftlicher Kommunikationsverfahren nach den Kriterien von Open Science und unter den Bedingungen der Open Definition bleibt bisher gering. Seitens der Produzenten des Wissens werden vor allem das Fehlen von Reputations- und Anerkennungsmechanismen, funktionierenden Geschäftsmodellen, die Angst vor Ideendiebstahl und das Fehlen von standardisierten Qualitätssicherungsmaßnahmen als Hindernisse für die weitere Verbreitung angeführt. In der Praxis fehlen darüber hinaus einfache und standardisierte Möglichkeiten der offenen Erstellung wissenschaftlicher Arbeiten und der Aufwand für die Öffnung wissenschaftlicher Erkenntnisprozesse übersteigt weiterhin den bisher üblichen Aufwand wissenschaftlicher Arbeiten. Die dargestellten Herausforderungen im aktuellen wissenschaftlichen Kommunikationssystem und identifizierten Katalysatoren für die Etablierung der Öffnung wissenschaftlicher Kommunikation haben demnach bisher (noch) nicht zu einem grundsätzlichen Umdenken im Handeln der wissenschaftlichen Akteure und zu dem Gestaltungsprozess der Kriterien für die wissenschaftliche Arbeit unter den veränderten Bedingungen beigetragen.

Die Debatten um die Veränderungen im wissenschaftlichen Kommunikationssystem fokussieren sich bislang vor allem auf die Steigerung von Effizienz im Rahmen von anwendungsrelevanter Forschung unter den Bedingungen der bestehenden wissenschaftlichen Praxis, die Förderung von Wissenstransfer für die wirtschaftliche Verwertung und die Schaffung von Rahmenbedingungen für privatwirtschaftliche Aneignung von wissenschaftlichen Erkenntnissen. Die Betrachtung der Hindernisse und Katalysatoren für eine wirkliche Öffnung wissenschaftlicher Kommunikation erfolgt in diesem Zusammenhang nur unter den etablierten Kriterien für den Reputationserwerb. Eine grundlegende Auseinandersetzung und Diskussion innerhalb und außerhalb der wissenschaftlichen Gemeinschaft mit den Konsequenzen und möglichen Nebenfolgen der Veränderungen in der wissenschaftlichen Kommunikation im Rahmen eines Wandels der Modi gesellschaftlicher Wissensproduktion ist bisher kaum erkennbar \cite{Buss_2001}.
