\chapter{Diskussion: Scientific Steady State vs. Second Scientific Revolution}

Die unbeschränkte und offene wissenschaftliche Kommunikation ist für das Wissenschaftssystem unverzichtbar. Der digitale Wandel im wissenschaftlichen Publikationssystem stellt eine Chance dar, das tradierte System so zu justieren, dass es ohne Qualitätsverlust, ohne Einschränkung der Wissens- oder Publikationsfreiheit und vorbehaltlich einer angemessenen Zuordnung der Urheberschaft zu einer umfassenderen Verteilung von Wissen die gesamten Gesellschaft zu Gute kommt.

Anhand der Inhaltsanalyse wurde evaluiert, wie Open Access und Open Science definiert werden können und welche Treiber für diesen Wandels verantwortlich sind und welche Faktoren diesen Wandel behindern.

Die durchgeführte empirische Studie, bei der mehr als 4.000 Wissenschaftlerinnen und Wissenschaftler (siehe Kapitel --- TODO --- Kapitel nennen ---) zeigt, dass in der überwiegenden Mehrzahl ein Verständnis für die Forderung nach Offenheit in der wissenschaftlichen Kommunikation unter den Befragten vorherrscht.

--- TODO: weiter ausarbeiten ---

\section{Welche Definition von Open Access und Open Science sind am häufigsten verbreitet?}

\subsection{Open Access}

\subsection{Open Science}

\section{Was bedeutet Offenheit und freier Zugang im Rahmen des wissenschaftlichen Diskurs-, Reputations- und Machtbegriffs?}

\section{Welche Argumente für und gegen die Ö̈ffnung wissenschaftliche formeller Kommunikation spielen in der Debatte eine Rolle?}

\section{Was sind die Treiber und Bremser der Entwicklung um die Forderung von Open Access und Open Science?}

\section{Wie stark sind offene formelle Kommunikationsformen in den verschiedenen wissenschaftlichen Disziplinen verbreitet}

\section{Welche Aufwand bedeutet das offene Verfassen einer Doktorarbeit?}

\section{Welche Handlungsempfehlungen für das Verfassen einer offenen Doktorarbeit können gegeben werden?} 
