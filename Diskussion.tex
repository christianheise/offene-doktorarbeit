\chapter{Diskussion: Wissenschaftliche Kommunikation im Wandel - Scientific Steady State vs. Second Scientific Revolution}

\begin{quote}
\textbf{"The sciences are too good merely to avert attention from what science does."}
\end{quote} \cite{kittler_2004}

Die unbeschränkte und offene wissenschaftliche Kommunikation scheint für das Wissenschaftssystem theoretisch unverzichtbar. Wissenschaft und Forschung sind demnach traditionell eng mit den Normen der schnellen Weitergabe von Forschungsergebnissen, einer Umgegbung des Wissensaustauschs, Co-Autorenschaft und das kumulative lernen und Innovation verbunden \cite{Partha_1994_economics_science}. Der digitale Wandel im wissenschaftlichen Publikationssystem stellt dabei Chance dar, das tradierte System so zu justieren, dass es ohne Qualitätsverlust, ohne Einschränkung der Wissens- oder Publikationsfreiheit und vorbehaltlich einer angemessenen Zuordnung der Urheberschaft zu einer umfassenderen und schnelleren Verteilung von Wissen in der wissenschaftlichen Gemeinschaft aber auch an die gesamten Gesellschaft kommen kann - so die Hoffnung der Befürworter des Wandels hin zur Öffnung wissenschaftlicher Kommunikation. Der Mediziner und Wissenschaftshistoriker Michael Hagner fasst diese Chance wie folgt zusammen: "Zum ersten Mal überhaupt in der Geschichte der Wissenschaften verfügen diese über ein Medium, das ihnen eine auf ihre Interessen hin zugeschnitten Tagesaktualität offeriert, die dem nahekommt, was Massenmedien wie Tageszeitung, Radio und Fernsehen früher bereits der Allgemeinheit anbieten konnte" \cite{hagner_2015_sache_buches}.

In der wissenschaftlichen Realität basiert die aktuelle wissenschaftliche Praxis noch immer auf einem insich geschlossenen System und auf der tradierten Annahme, das "was nicht gedruckt wird, hat kaum Chancen, die Entwicklung des Faches zu beeinflussen" \cite{luhmann_1997_gesellschaft}.

Ziel dieser Arbeit war die Darstellung und Analyse der theoretischen Annahmen und Definitionen rund um die Etablierung und Praktizierung offener wissenschaftlicher Kommunikation mit den praktischen Gegebenheiten im wissenschaftlichen Alltag. Die Diskrepanz zwischen der Idee der Öffnung von wissenschaftlicher Kommunikation und der wissenschaftliche Realität \cite{Scheliga_2014} konnte im Rahmen der Auswertung Befragung belegt werden. Die Gründe für die schleppende Umsetzung der Konzepte rund um die Öffnung von Wissenschaft wurden erarbeitet und die Erfahrungen und Meinungen der Wissenschaftler und Wissenschaftlerinnen den eigenen Erfahrungen aus einem Selbstversuch gegenübergestellt.

In diesem Kapitel werden die Ergebnisse der durchgeführten Befragung, die Argumentationsstänge der Debatten in der Literatur um den Wandel der wissenschaftlichen Kommunikation im Rahmen der Digitalisierung abschließend diskutiert und um die eigene Sichtweise beim offenen Verfassen dieser Arbeit ergänzt. Ziel ist es zu einer differenzierten Betrachtung der Ergebnisse der Befragung und des Experiments zu gelangen und diese in den Kontext der Fragestellungen der Arbeit zu stellen und kritisch zu betrachten.

---- TODO: weiter ausarbeiten "diskussion" anhand von Literatur erklären ----

\section{Annäherung an die Begriffsbestimmung von Open Access und Open Science und gängige Debatten}

Anhand der Literaturrecherche wurden unterschiedliche Definitionen von Open Access und Open Science evaluiert und eruiert, welche Aspekte von Öffnung in den meisten dieser Definitionen enthalten sind. Es wurde außerdem herausgearbeitet, welche Vermutungen in der Literatur darüber vorherrschen, warum die Öffnung von Wissen in den verschiedenen wissenschaftlichen Disziplinen unterschiedlich stark etabliert ist und welche Bedeutung von Offenheit und freien Zugang im Rahmen des wissenschaftlichen Reputationssystems vorherrscht.

---- TODO: weiter ausarbeiten und Ziel des Kapitels konkreter heraustellen ----

\subsection{Open Access}

Open Access hat sich in den letzen 25 Jahren zu der meistgenannten Lösung für die beschriebenen Herausforderungen im wissenschaftlichen Kommunikations- und Publikationssystem entwickelt \cite{brembs2015open}.

Seit den ersten Experimenten mit der Öffnung des Zugangs zu wissenschaftlichen Publikationen,  existieren mehrere Definitionen von Open Access und es bestehen unterschiedliche Auffassungen über die verschiedenen Modelle. Die genaue Definition von Open Access, war bis Anfang der 2000er Jahre vage. Die drei "B"s strebten an, das zu ändern und stimmen in den den wesentlichsten Merkmalen auch überein \cite{albert_2006_open_implications}, unterscheiden sich aber in ihrem Auswirkungs und Bezugsrahmen \cite{naeder_2010_open}. Das stellt die Etablierung nachhaltig vor Herausforderungen.

---- TODO: weiter diskutieren und warum beantworten ----

Johannes Näder fasst die Situation wie folgt zusammen \cite{naeder_2010_open}:
Selbst nach der vollständigen Rezeption aller drei Erklärungen bleibt schließlich ein gewisser Interpretationsspielraum: Etwa hinsichtlich der Frage, ob ein Dokument auch dann dem Open-Access-Gedanken entspricht, wenn die in den beiden jüngeren Erklärungen geforderten Zusatzmaterialien nicht mitgeliefert werden und das Dokument erst nach Ablauf der vertraglichen Schutzfrist online zugänglich gemacht wird."

Will man dennoch zu einer gemeinsamen Definition von Open Access kommen, kann ein Ansatz das anstreben einer gemeinsamen und eindeutige Definition von "open" sein. Das ist auch deshalb sinnvoll um die ideele Entwicklung von Offenheit in Wissenschaft und Forschung in ihrer ursprünglich gedachten Form auch in anderen Bereichen des offenen Wissens, in denen der Begriff "Open" verwendet wird, wie zum Beispiel im Rahmen offener Verwaltungs- und Regierungsdaten nicht nachhaltig zu gefährden.

Als Grundlage hierfür kann die Open Definition \cite{open_definition} gelten, die im Gegensatz zu den meisten Erklärungen rund um Open Access ständig weiterentwickelt wird und eine klare Abgrenzung zu nicht-open beinhaltet, ohne den Kern ihrer Aussagen zu verwässern. Nach dieser Definition ist Wissen dann als "open" zu bezeichnen, "wenn jedeR darauf frei zugreifen, es nutzen, verändern und teilen kann – eingeschränkt höchstens durch Maßnahmen, die Ursprung und Offenheit des Wissens bewahren" \cite{open_definition}. In der Definition sind die Bereiche "Offene Werke", "Offene Lizenzen" und "Akzeptable Bedingungen" klar definiert. Was die Definition auslässt, ist der zeitliche Horizont zwischen Erstellung und zur Verfügungstellung der Inhalte.

Die Zeit die seit der ersten Grundsatzerklärung zu Open Access in Budapest vergangen ist, hat gezeigt, dass die Abhängigkeit und Stabilität des wissenschaftlichen Kommunikationssystems größer ist, als zuerst angenommen: Der "Faustische Pakt stabiler ist als gedacht" \cite{hagner_2015_sache_buches} und die etablierte Publikationssystem der Verlage sind auch nach "zwei Jahrzehnten weitgehend stabil" \cite{Hanekop_2014} geblieben.

---- TODO: weiter diskutieren - Warum? ----

Aus der Forderung nach "unbeschränkten Zugang zur gesamten wissenschaftlichen Zeitschriftenliteratur" \cite{boai_2012} ist ein gesamtgesellschaftliches und umfassendes Modernisierungsvorhaben der Wissenschaft geworden, dass neben den Aspekten bezüglich der Zungänglichkeit zu Wissen und Wissenschaft eine Vielszahl an Unzulänglichkeiten adreessiert die den Fortbestand öffentlicher Forschung insgesamt gefährden \cite{brembs2015open}. Mit Blick auf die Umsetzung werden dabei, anders als in es ursprünglich bei Open Access intendiert war, Einschränkungen der akademischen Freieheit vermutet \cite{hagner_2015_sache_buches}.

---- TODO: weiter diskutieren - Auswirkungen und warum? ----

\subsection{Open Science}

Open Science addressiert diese umfassendere Modernisierungsvorhaben und betrifft den gesamten wissenschaftlichen Erkenntnisgewinnungsprozess. Open Science greift die politischen Ideale von Open Access auf und ergänzt sie um die notwendigen praktischen Aspekte zur Veränderung des wissenschaftlichen Systems. Die Probleme bei der Umsetzung können dabei von Open Access abgeleitet aber dennoch umfassender adressiert werden, da sie sich eben nicht nur auf den Zugang zu publizierten Wissen, sondern auf den Zugriff auf den gesamten wissenschaftlichen Prozess beziehen. Open Science adressiert also ebenso die Themen Steureung, Qualitätssicherung und Archivierung, sowie die Aspekte der Freiheit der Wissenschaft.

Die Umsetzung solcher Open Science Initiativen kann dabei nicht nur auf Grundlage von Vorgaben, Gesetzen oder Richtlinien erfolgen, sondern muss primär der Aufgabe neues Wissen zu produzieren und damit dem gesellschaftlichen Auftrag des Wissenschaftssystems gerecht werden. Dazu müssen Anreize für die einzelnen Wissenschaftler so gesetzt werden, dass deren Eigeninteresse mit dem Wohl der Wissenschaft und damit dem der Öffentlichkeit harmonieren \cite{brembs2015open}.

---- TODO: weiter diskutieren - Auswirkungen und warum? ----

Im Gegensatz zu Open Access gibt es bisher wenige Erklärungen und Statements die zur Definition von Open Science herangezogen werden können. Demnach kann auch hier die Open Definition als Rahmen dienen, wobei sich Open Science nicht nur auf die fertige wissenschaftliche Publikation, sondern auf den gesammten wissenschaftlichen Erkenntnsiprozess

Bezuglüch der zeitlichen Dimension der Veröffentlichung von wissenschaftlichen Inhalten, die die Open Definition nicht abdeckt und die auch in den Erklärungen von Budapest, Bethesda und Berlin nur unzureichend definiert sind, könnte Open Science als eine Lösung bereitstehen. Hier wird eine unmittelbare Veröffentlichung der Informationen im wissenschaftlichen Erkenntnisprozess angestrebt. Damit entfällt die zeitliche Komponente. Während bei Open Access die finale Publikation im Vordergrund steht, greift Open Science viel früher. Die Publikation ist dabei eher als nachgelagertes Ergebnis zu betrachten, bei dem die Frage der Veröffentlichung

Folgt man dieser Betrachtungsweise, sind die zeitlichen und rechtlichen Herausforderungen bei der Veröffentlichung von wissenschaftlichen Erkenntnissen in Form von Open Access Publikationen nachgelagert. Bei einer Realisierung des Konzepts von Open Science. Kritisch wäre hier nur, dass die Argumentation der Ergebniserstellung im Rahmen von steuerfinanzierten Arbeitsumgebungen ebenfalls unter den Bedinungen der Open Definition stattfinden müsste.

---- TODO: weiter diskutieren - Auswirkungen und warum? ----

Angeleht an den Kurzusammenfassung der Sprachwissenschaftlerin Graefen für die Entwicklung von Wissenschaft, gilt auch bei für eine erfolgreiche Implementierung von Open Science in den wissenschaftlichen Alltag, ein gesamtgesellschaftlicher Bedarf für das zu öffnende Wissen bestehen muss und die entsprechende Leistungen von Individuen zu persönlichen führen. Nur dann kann eine Umorientierung von geschlossener individueller wissenschaftlicher Betätigung hin zu offener, gesellschaftlich anerkannter und zur Kenntnis genommener, kollektiv betriebener Wissenschaft stattfinden \cite{graefen2007_wissenschaftliche_artikel}. Ergänzend muss diese Entwicklung unter Beteiligung der Wissenschaft heraus gestaltet werden, da das System sonst gefahr läuft, sich zu "sporadischer individueller wissenschaftlicher Betätigung" \cite{graefen2007_wissenschaftliche_artikel} zurückzuentwickeln. Eine Vorraussetzung für Open Science ist, dass "wissenschaftlichen Daten, Codes und Volltexte sowie zum Teil das Lese-, Gutachter- und Lehrverhalten der Wissenschaftler zu einem großen Teil transparent werden" \cite{brembs2015open}. Einzige Einschränkung sollte die Privatssphäre der Wissenschaftler und Wissenschaftlerinnen sowie der Untersuchten sein.

Weitere Voraussetzungen diesbezüglich sind:
\begin{itemize}
\item die technischen Möglichkeiten der schnellen Erstellung, Vervielfältigung und Verbreitung von wissenschaftlicher Kommunikation,
\item die Zugänglichkeit zum gesamten Forschungsprozesses
\item die Möglichkeit Wissenschaft(en) von anderen Formen der Information und Kommunikation zu unterscheiden
\item "die ökonomische und politische Nutzbarkeit von Wissenselementen, so daß
Forschung ein Mittel der Verwertung werden konnte",
\item die durch Ausbildungsprozesse gefestigte Existenz einer beruflich mit For-
schung befaßten 'Schicht' von Fachleuten und Wissenschaftlern.
\end{itemize}

---- TODO: ausarbeiten und diskutieren ----

\section{Verbreitung von und das Interesse an Offenheit}

---- TODO: Einleiten ----

\subsection{Verbreitung von und das Interesse an Offenheit nach Alter}

Das Interesse an Forschungsdaten in der Befragung war grundsätzlich hoch und 71 Prozent der Befragten bekundete Interesse an den Forschungsdaten anderer. 64 Prozent aller Befragten konnte sich grundsätzlich vorstellen unter unter bestimmten Bedinungen ihre orschungsdaten und alle weiteren Informationen, die während Ihrer wissenschaftliche Arbeit anfallen unter Berücksichtigung von Datenschutz öffentlich zur Verfügung zu stellen, 28 Prozent sogar nur unter den Einschränkungen des Datenschutz. Dabei varierten die Unterschiede zwischen den Altersgruppen nur leicht.

In anderen Befragungen wurde vor allem bei jüngeren Altersgruppen befragter Forscherinnen und Forscher eine spezielles Interesse identifiziert, die Daten nicht ohne Einschränkungen zu veröffentlichen, während die über 50-Jährigen weniger bedenken äusserten \cite{Tenopir_2011}. Es wurde angenommen, dass das mit Bedenken hinsichtlich der Besitzverhältnisse und der beruflichen Entwicklung zusammenhängt \cite{Tenopir_2011}.

---- TODO: Grafik mit Altersvergleich Vorstellung Daten zu veröffentlichen bauen  ----



\subsection{Verbreitung von und das Interesse an Offenheit in den verschiedenen Disziplinen}

Das Interesse an Forschungsdaten anderer Wissenschaftler war in allen Disziplinen ähnlich stark ausgeprägt. Die Vision von Open Access hingegen fand auf hohem Niveau unterschiedlich viel Unterstützung. Im Detail gibt es aber gravierende Unterschiede zwischen den Fachgruppen bei der Bewertung und der praktischen Umsetzung von offener wissenschaftlicher Kommunikation. Diese Entwicklung und eventuelle Gründe werden im Folgenden dargestellt und diskutiert.

---- TODO: Grafik mit Fachgruppenvergleich der Unterstützung von Forderung OA und Interesse and Forschungsdaten bauen  ----

Die durchgeführte empirische Studie zeigt, dass bei der überwiegenden Mehrzahl der Befragten ein grundsätzliches Verständnis für die Forderung nach Offenheit in der wissenschaftlichen Kommunikation vorherrscht (96 Prozent) und 75 Prozent eine gängige Definitionen von Open Access befürworten. 71 Prozent der Befragten zeigte zudem Interesse am Zugang zu Forschungsdaten anderer Wissenschaftler_innen zeigt. 29 Prozent der Befragten gaben an, kein Interesse an den Daten anderer zu haben.

49 Prozent gaben an, gelegentlich und 32 Prozent häufig nicht auf die digitale/Online-Version eines Textes zugreifen zu können. Demgegenüber haben nur 36 Prozent der Befragten angegeben, Aufsätze, Texte oder Bücher publiziert zu haben, die frei zugänglich waren und 38 Prozent stellen laut eigenen Angaben Volltexte auf den eigenen oder Insitutswebseiten zur Verfügung. 32 Prozent bewerteten die Zugänglichkeit zu ihren Veröffentlichungen für potentielle Leser als gut.

Die Zahlen stützen die Annahmen in der Literatur, nach denen Wissenschaftler und Wissenschaftler  Open Access als Rezipienten mehrheitlich bejahen, haben als Autoren jedoch wenig oder nur partiell genuines Interesse an Open Access haben \cite{wein_2010_erwerbung}.

---- TODO: weiter ausarbeiten ----

\subsubsection{Geisteswissenschaften}

Dass die Geisteswissenschaften am geringsten unter allen befragen Fachgruppen, aber dennoch mehrheitlich, die Forderung nach kostenfreiem Zugang zu allen wissenschaftlichen Publikationen für Leser (Open Access) zustimmen, deckt sich mit dem Stand der Verbreitung von Open Access in der Fachrichtung. Dabei spielt die Publikationsform der Monografien nur in den Geistes- und Sozialwissenschaften eine wichtige Rolle, ebenfalls wurden nur in dieser Fachgruppe deutsche Zeitschriften als wichtig erachtet  (65 Prozent). Das deckt sich mit den Aussagen in der Literatur \cite{hagner_2015_sache_buches} \cite{naeder_2010_open} \cite{hollricher_wandel_2009} \cite{Lossau_oa_2007}. Die Auswertung der Ergebnisse der Befragung zeigt aber auch bei den Geistes- und Sozialwissenschaftler und -wissenschaftlerinnen ein mehrheitliches Interesse der an den Forschungsdaten anderer (70 Prozent). Betrachtet man den Austausch von Daten als eine erweiterte Möglichkeit, Wissen zu überprüfen und Verzerrungen und Fehler zu beseitigen, erscheint es dennoch verwunderlich, dass 30 Prozent der befragten Wissenschaftler und Wissenschaftlerinnen der Fachgruppe kein Interesse daran haben.

---- TODO: Grafik bauen ----

Auch insgesamt zeichnet die Auswertung der Antworten von 418 Wissenschaftlern und Wissenschaftlerinnen aus der geistes- und sozialwissenschaftliche Fachrichtung ein eher ambivalentes Bild bezüglich dem Wunsch nach Öffnung wissenschaftlicher Kommunikation und der tatsächlich praktizierten Offenheit. So erachten nur 25 Prozent der Befragten Geisteswissenschaftler ihre eigenen Beiträge als gut zugänglich. Dass die überwiegende Mehrheit der Befragten die freie Verfügbarkeit des eigenen Volltexts im Internet (62 Prozent) und die Veröffentlichung unter einer Open Access Lizenz (70 Prozent) als eher weniger wichtig oder unwichtig erachtet, lässt dennoch auf ein gewisses Desinteresse schließen. Für alle anderen Fachgruppen hat die freie Verfügbarkeit der eigenen Texte eine höheren Stellenwert.

---- TODO: Grafik bauen ----

Demgegenüber gaben 43 Prozent der Befragten an, Volltexte selber auf Webseiten zur Verfügung zu stellen oder stellen zu lassen und 37 Prozent der Teilnehmer und Teilnehmerinnen haben schon mindestens einmal ihre Inhalte frei zugänglich publiziert. Erstaunlich ist, dass Sie auch die Gruppe derer stellen, die sich am Besten vorstellen kann, Forschungsdaten und alle weiteren Informationen, die während Ihrer wissenschaftliche Arbeit anfallen unter bestimmten Bedingungen, öffentlich zur Verfügung zu stellen (67 Prozent).

---- TODO: Grafik bauen  ----

Daraus lässt sich schließen, dass unter den Wissenschaftlern und Wissenschaftlerinnen dieser Fachrichtung zwar ein grundsätzliches Interesse am Zugang zu Forschungsdaten anderer Forscher und Forscherinnen besteht (70 Prozent) und auch mehrheitliche, wenn auch unter allen Fachgruppen am geringsten Ausgeprägte Zustimmung zu der Forderung nach kostenfreiem Zugang zu allen wissenschaftlichen Publikationen für Leser (Open Access) vorherrscht (68 Prozent), diese aber in der praktischen Arbeit keine große Rolle spielt (37 Prozent). Das mag dadurch begründet sein, dass die eigene Zugänglichkeit zu Publikationen durch die Wissenschaftler in der Fachgruppe überwiegend als gut oder sehr gut bewertet wird und dass die Entwicklungen rund um die Öffnung von Wissenschaft und Forschung eher aus den STM-Fächern kommt ---- TODO: Verweis auf Kapitel ----.

Dennoch verwundert diese ambivalente Haltung, denn gerade für die Geistes- und Sozialwissenschaften lässt sich ein besonderes Interesse an die Verbreitung von Wissen innerhalb der wissenschaftlichen Community und auch an die Gesamtgesellschaft vermuten \cite{suchen}. Diese Vermutung wird dadurch bestärkt, dass in den Geistes- und Sozialwissenschaften 83 Prozent der Befragten die Anzahl der Aufsätze und Beiträge, am stärksten unter allen befragten Fachgruppen, als wichtigen Faktor für Reputation in ihrer Disziplin erachtet haben.

\subsubsection{Lebenswissenschaften}

In der Gruppe der Lebenswissenschaften findet die Forderung nach nach kostenfreiem Zugang zu allen wissenschaftlichen Publikationen für Leser (Open Access), die stärkste Zustimmung unter allen vier Fachgruppen (88 Prozent). Mit 98 Prozent sind in den Lebenswissenschaften internationale Zeitschriften wichtigste Publikationsform. Die Befragten bewerteten die Zugänglichkeit zu ihren eigenen Beiträge mehrheitlich als nicht so gut, schlecht oder teil/teils (68 Prozent). Zwei Drittel der Befragten Lebenswissenschaftler und -wissenschaftlerinnen gab an, Interesse am Zugang zu Forschungsdaten anderer zu haben. Auch hier ist es verwunderlich, dass ein Drittel kein Interesse an dem Zugang zu den Daten anderer Forscher hat.

---- TODO: Grafik bauen  ----

Die 197 Befragten aus den Lebenswissenschaften bekennen sich eindeutiger zur Öffnung der wissenschaftlichen Kommunikation als die Geistes- und Sozialwissenschaftler. Der freie Zugang zu den eigenen wissenschaftlichen Beitragen wird in der Fachgruppe von 61 Prozent als wichtig oder sehr wichtig betrachtet und 54 Prozent der Lebenswissenschaftler gaben an, dass es ihnen mindestens wichtig ist, unter einer Open-Access-Lizenz zu veröffentlichen. Demnach ist in dieser Fachgruppe nicht nur die stärkste Zustimmung zur Öffnung wissenschaftlicher Kommunikation zu verzeichnen, sondern auch der stärkste praktische Verbreitungsgrad. Dass 53 Prozent der Befragten bereits unter frei zugänglich, zum Beispiel unter einer Open-Access-Lizenz, publiziert haben, bestätigt diese Einschätzung.

---- TODO: Grafik bauen  ----

Der Zustand kann darauf zurückgeführt werden, dass die Zugangsmöglichkeiten insgesamt zu wissenschaftlichen Publikationen durch die Wissenschaftler in der Fachgruppe überwiegend als schlecht beurteilt werden. Das stützt ebenfalls die These aus der Literatur \cite{suchen}, dass die Öffnung am stärksten in den Fachgruppen vorangetrieben wird, bei denen sich die Krisen am stärksten für die Wissenschaftler und Wissenschaftlerinnen bemerkbar machen. Dennoch nur 36 Prozent erachten die eigenen Veröffentlichung als gut zugänglich.

---- TODO: Vergleich Entwicklung mit SOFI-Studie einbauen als Beleg ----

\subsubsection{Naturwissenschaften}

In den Naturwissenschaften unterstützten 82 Prozent der 322 Befragten die Forderung nach Open Access. Wie in den Lebenswissenschaften wurde auch in den Naturwissenschaften die internationale Zeitschrift als wichtigste Publikationsform genannt (98 Prozent). 39 Prozent der Befragten gaben an, dass sie ihre Veröffentlichungen für potentielle Leser als gut zugänglich bewerten. Mit nur 69 Prozent gab im Fachgruppenvergleich die kleinste Gruppe der Befragten an, Interesse am Zugang zu Forschungsdaten anderer Wissenschaftler und Wissenschaftlerinnen zu haben.

---- TODO: Grafik bauen  ----

36 Prozent erachten es als wichtig oder sehr wichtig unter einer Open-Lizenz zu veröffentlichen und für 52 Prozent spielt der freie Zugang und die Veröffentlichung im Internet zu den eigenen wissenschaftlichen Beitragen eine wichtige oder sehr wichtige Rolle. 36 Prozent gaben an bereits frei zugänglich publiziert zu haben.

Insgesamt befinden sich die Gruppe der Naturwissenshaftler und -wissenschaftlerinnen damit im Fachgruppenvergleich auf niedrigem Niveau im Mittelfweld, wenn es um die Verbreitung offener Kommunikation und die tatsächlichen Umsetzung geht.

Das ist auf der einen Seite verwunderlich, denn die Entwicklung der Forderung von Offenheit in Wissenschaft und Forschung wird neben den Fächern der Lebenswissenschaften auch den Naturwissenschaften und auf der Publikationsform Zeitschrift zugeschrieben wird \cite{suchen}, auf der anderen gaben 78 Prozent der Befragten an, über gute oder sehr gute Zugangsmöglichkeiten zu wissenschaftlichen Online-Zeitschriften über eine Lizenz ihrer Forschungseinrichtung zu erhalten (93 Prozent) und scheinen somit keinen direkten Veränderungsdruck zu erleben.

---- TODO: Vergleich Entwicklung mit SOFI-Studie einbauen als Beleg ----

\subsubsection{Ingenieurwissenschaften}

Die befragten Wissenschaftler und Wissenschaftlerinnen unterstützen ebenfalls mehrheitlich die Forderung nach Open Access (72 Prozent). Knapp einem Drittel (32 Prozent) ist es wichtig oder sehr wichtig unter einer Open-Lizenz zu veröffentlichen, während 42 Prozent der freie Zugang zum Volltext im Internet zu den eigenen wissenschaftlichen Beitragen wichtig oder sehr wichtig.

34 Prozent Prozent haben bereits frei zugänglich publiziert und 32 Prozent finden, dass Ihre Veröffentlichungen in Zeitschriften oder Büchern für potentielle Leser gut zugänglich. Wie bei den Lebenswissenschaften gaben zwei Drittel der Befragten an, Interesse am Zugang zu Forschungsdaten anderer Wissenschaftler und Wissenschaftlerinnen zu haben.

Auch in den Ingenieurwissenschaften ist mit 90 Prozent die internationale Zeitschrift die wichtigste Publikationsform. Nur 37 Prozent erachten deutschsprachige Zeitschriften als wichtig. Monografien spielen nur für ein Drittel (35 Prozent) der Befragten eine wichtige Rolle.

---- TODO: weiter ausarbeiten ----

\section{Katalysatoren und Hindernisse für die Öffnung wissenschaftlicher Kommunikation}

In der Auswertung der Befragung von 1.112 Wissenschaftlern konnte ein mehrheitlich stark verbreitetes Verständnis von Open Access und die mehrheitliche, grundsätzliche Unterstützung der Forderung nach Öffnung von Wissenschaft sowie Interesse an Forschungsdaten anderer nachgewiesen werden.

Von den Befragten wurden als Argumente für die Öffnung vor allem die beschleunigte Wissensverbreitung und die neue Möglichkeiten für die Kommunikation genannt. Demgegenüber stand neben den fehlenden, etablierten Reputationskriterien für die Bewertung von offener Wissenschaft, Gefahr der Fehlinterpretation und Falschinformation durch Wissenschaft genannt.

Der vermutete erhöhte zeitliche Mehraufwand für die Bereitstellung der wissenschaftlichen Publikationen und Forschungsdaten untermalen den Eindruck, dass Wissenschaftler selbst noch keinen großen Druck verspüren, ihr Veröffentlichungsverhalten zu verändern. Das mag auch an dem Umstand liegen, dass sie trotz der Publikationskrise und anderen Faktoren, die zu einem Marktungleichgewicht im wissenschaftlichen Kommunikationssystem geführt haben, in einer komfortablen Situation sind beziehungsweise von den Auswirkungen der Krise bisher kaum oder nicht betroffen sind.

In der Debatte und in Erhebungen rund um die Öffnung von Wissenschaft und Forschung werden viele Faktoren genannt, die diesen Prozess vom geschlossenen System wissenschaftlicher Kommunikation hin zu einer Öffnung in unterschiedlichster Weise beeinflussen. Die Argumente erscheinen dabei als sehr vielseitig und facettenreich, konzentrieren sich aber im Kern auf 5 Aspekte:

---- TODO: ausarbeiten ----

Bei der Alltagsbetrachtung stehen stehen hauptsächlich rechtliche Bedenken, fehlenden Reputations- und Qualitätssicherungsmechanismen für die offene wissenschaftliche Kommunikation im Vordergrund. Das bestehende System macht es weder einfach, noch wird es honoriert, wenn Wissenschaftler und Wissenschaftlerinnen die eigene wissenschaftliche Kommunikation öffnen oder Ergebnisse frei zur Verfügung stellen.

Andere Studien führen darüber hinaus, fehlende Infrastrukturen beziehungsweise die fehlende Integration in bestehende Strukturen sowie fehlendes Know-How bei den Wissenschaftler und Wissenschaftlern als Grund Argument für den geringen Öffnungsgrad in Wissenschaft und Forschung an \cite{eu_open_science_2015}.

In den wissenschaftstheoretischen Diskursen wird die Wissenschafts-, Presse- und Publikationsfreiheit oft als Argument gegen den Druck wissenschaftliche Inhalte zu öffnen angeführt. Mit Hilfe der Befragung konnte diese Annahme allerdings nicht mehrheitlich bestätigt werden.

---- TODO: weiter ausarbeiten und diskutieren ----

\section{Haupteinflussfaktoren für die Entwicklung um die Forderung von Open Access und Open Science}

Ein weiteres Ziel dieser Arbeit war die Identifikation von Faktoren, welche die Öffnung von Wissenschaft und Forschung begünstigen. Diese Faktoren wurden anhand der Literatur herausgearbeitet und im Rahmen der durchgeführten Umfrage abgefragt. Die erhobenen Daten wurden statistisch ausgewertet und zusammenfassend dargestellt.

Wie in anderen Studien festgestellt, wurden die Verfügbarkeit von digitalen Technologien und deren erhöhte Kapazität als einer der wesentlichen Einflussfaktoren für die Entwicklung um die Forderung von Open Access und Open Science identifiziert \cite{eu_open_science_2015}. Der Wunsch nach größerer Verbreitung von wissenschaflichen Erkenntnisen und der Wunsch nach mehr möglichkeiten zum kollaborativen Arbeiten spielen bei der Betrachtung der Haupteinflussfaktoren ebenfalls eine hervorgehobene Rolle.

Dem Interesse, an der Öffnung wissenschaftlicher Kommunikation, steht allerdings die Befürchtung hoher Mehraufwände für die wissenschaftliche Praxis gegenüber. Die Befragung hat gezeigt, dass praktisch nur ein geringer Teil der befragten Wissenschaftler und Wissenschaftlerinnen tatsächlich bereit sind, ihre gelebte Kommunikationspraxis zu ändern und ihre wissenschaftlichen Prozesse offen zu kommunizieren. In der direkten Begründung wurden vor allem rechtliche Unsicherheit, fehlende Mechanismen zu Reputationsbildung durch eine offene Arbeitsweise und sowie Unsicherheit bei der Zulässigkeit eines solchen Handelns genannt.

Im Rahmen des offenen Verfassens dieser Arbeit ist aber auch deutlich geworden, dass für die offene Bereitstellung der wissenschaftlichen Kommunikation nur wenige und aufwändige Tools und Dienste zur Verfügung stehen. Die meiste wissenschaftliche Open Source Forschungssoftware ist qualitativ minderwertig, ineffizient, undokumentiert und wird nur selten weitereentwickelt oder gepflegt \cite{hey_2015_open}. Ein Grundlegenden Problems ist, dass Wissenschaftler und Wissenschaftlerinnen selten gut ausgestattet sind, nur sehr grundlegende Ausbildung und Erfahrung im Bereich Programmierung, Design, Testing, Debugging oder bei der Pflege von Software haben \cite{hey_2015_open}.

Das die Befragten Wissenschaftler und Wissenschaftlerinnen die vorherrschenden Zeitschriften- und Monographienkrise als einen eher unwichtigen Katalysatoren für die Öffnung wissenschschaftlicher Kommunikation identifiziert haben, unterstützt die Annahme des geringen Interesses an dem Thema und dass viele sich der Situation in der sich das System befindet nicht bewusst sind. Sie nährt die Befürchtung, dass sich trotz efizienterer Verfahren und neuer Technologien nichts an dem System ändern wird \cite{Parks_2002_acadamic_faust}.

Die Ergebnisse zeigen deutlich, dass die Verschärfung der Situation in den letzten 10 Jahren bisher nur vereinzelt zu einer Verhaltensänderungen im Publikationsverhalten der Wissenschaftler geführt hat. Dabei gibt es unterschiede zwischen den Disziplinen und bei der

---- TODO: weiter ausarbeiten, prüfen ob altersabhängig oder Statusabhängig und diskutieren ----

\section{Bedeutung der Konzepte von Open Access und Open Science im Rahmen wissenschaftlicher Reputation}

Wissenschaftliche Reputation basiert, wie in Kapitel xxxx dargestellt auf einem System der Verbreitung neuer und der Adaption bestehender wissenschaftlicher Erkenntnisse. Unter dieser Prämisse scheint es selbstverständlich, dass Offenheit und freier Zugang zu wissenschaftlicher Kommunikation Grundpfeiler des wissenschaftlichen Diskurses und des Reputations- und Machtsystems sind.

In der Praxis sind diese System aber selbstreferenziell und auf die wissenschaftliche Gemeinschaft beschränkt. Es haben nur die Zugriff auf das System, die bereits Teil des Systems sind, oder die sich beginnen den Normen und Regeln des Systems anpassen. Die Gesamtgesellschaft ist von dem Diskurs in vielen Fällen ausgeschlossen, oder hat erst nach einer gewissen Zeit die Möglichkeit auf die Informationen und Erkenntnisse zuzugreifen. Somit manifestiert das System auch die Machtkonstelation derjenigen, die das System im Moment beeinflussen.

Im Rahmen der Idenifikation von Treibern und Bremsern für die Verhaltensänderung hin zur Öffnung der wissenschaftlichen Kommunikation, stellen die fehlenden Reputationsmechanismen eine der größten Herausforderungen für die Verbreitung dar. Ebenso wurden fehlende Qualitätssicherungsmaßnahmen besonders häufig bei den Argumenten gegen die Verbreitung von Open Science erwähnt. Das bestätigten auch andere Studien \cite{eu_open_science_2015}.

Die bisher genutzten bibliometrischen Verfahren erfreuen sich noch immer großer Beliebtheit und haben eine Absicherungsfunktion bezüglich der vermuteten Qualität einer Publikation, die bisher noch kein digitales Equivalent gefunden hat. Da auch die Forschungsförderung auf den tradierten Verfahren der Evaluation von Forschung aufetzt, ist diesbezüglich noch kein Wandel abzusehen.

Es ist davon auszugehen, dass die Möglichkeit der Erlangung von wissenschaftlicher Reputation einen Einfluss auf die Verbreitung des Konzepts von offenem Zugang zu Wissenschaft hat, nicht umgekehrt. Die bestehenden digitalen Bewertungssysteme haben in der Wissenschaftssteuerung bisher nur begrenzt einzug gehalten. So kann noch keine abschließende Beantwirtung der Frage stattfinden, ob die Öffnung der Kommunikation einen Einfluss auf das Konzept der wissenschaftlichen Reputation innerhalb und außerhalb der wissenschaftlichen Community hat.

---- TODO: weiter ausarbeiten, prüfen ob altersabhängig oder Statusabhängig und diskutieren ----

\section{Einfluss der Entwicklungen um die Öffnung wissenschaftlicher Kommunikation auf die Massifizierung und Neoliberalisierung der Universität}

---- TODO: ausarbeiten und diskutieren ----

\section{Aufwand für die Öffnung des gesamten wissenschaftlichen Erkenntnisprozess}

Im Vergleich zum Publizieren von Texten in einer geschlossenen Umgebung, ist das offene Verfassen einer wissenschaftlichen Publikation noch immer mit viel Mehraufwand verbunden. Das liegt zum Einen daran, dass die genutzten Softwareprodukte die Veröffentlichung der Arbeit und der Kommunikationsprozesse noch nicht vollumfänglich und einfach ermöglichen, zum Anderen sind die Richtlinien und Vorgaben für wissenschaftliche Arbeiten an Universitäten und Forschungseinrichtungen nicht darauf ausgelegt offen zu arbeiten.

Die Kommunikation von Zwischenständen, die Veröffentlichung von Daten und die Dokumentation des offenen wissenschaftlichen Arbeitens erfordert ebenfalls mehr Aufwand als beim tradierten Arbeiten in geschlossenen Systemen. Der Aufwand ist dabei eng an die Vorkenntnisse im Umgang mit den verwendeten Systemen und Kanälen zur Dokumentation verbunden.

---- TODO: weiter ausarbeiten und anhand der eigenen Erfahrung diskutieren ----

\section{Handlungsempfehlungen für das offene Verfassen wissenschaftlicher Arbeiten}

---- TODO: an Ergebnis aus dem Experiment anknüpfen, ausarbeiten und anhand der eigenen Erfahrung diskutieren ----
