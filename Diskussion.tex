\chapter{Diskussion: Scientific Steady State vs. Second Scientific Revolution}

Die unbeschränkte und offene wissenschaftliche Kommunikation ist für das Wissenschaftssystem eigentlich unverzichtbar. Der digitale Wandel im wissenschaftlichen Publikationssystem stellt eine Chance dar, das tradierte System so zu justieren, dass es ohne Qualitätsverlust, ohne Einschränkung der Wissens- oder Publikationsfreiheit und vorbehaltlich einer angemessenen Zuordnung der Urheberschaft zu einer umfassenderen Verteilung von Wissen in der wissenschaftlichen Gemeinschaft aber auch an die gesamten Gesellschaft kommen kann.

--- TODO: weiter ausarbeiten ---

\section{Welche Definition oder Aspekte in Definitionen von Open Access und Open Science sind am häufigsten verbreitet?}

Anhand der Inhaltsanalyse wurde evaluiert, wie Open Access und Open Science definiert werden können, beziehungsweise welche Aspekte von Öffnung in den meisten Definitionen enthalten sind. Im Weiteren wurden herausgearbeitet, welche Gründe für die Forderung nach Öffnung der wissenschaftlichen Kommunikation angeführt werden und welche Faktoren diesen Wandel unterstützen oder welche Faktoren diesen Wandel behindern. Es wurde außerdem herausgearbeitet, welche Vermutungen in der Literatur darüber vorherrschen, warum die Öffnung von Wissen in den verschiedenen wissenschaftlichen Disziplinen unterschiedlich stark etabliert ist.

Abschließend wurde für die theoretische Betrachtung des Bestrebens nach Öffnung in der Wissenschaft die Bedeutung von Offenheit und freien Zugang im Rahmen des wissenschaftlichen Diskurs-, Reputations- und Machtbegriffs herausgearbeitet.

\subsection{Open Access}

\subsection{Open Science}

\subsection{Verbreitung von Offenheit in den Disziplinen}

\subsection{Bedeutung von Offenheit und freien Zugang im Rahmen des wissenschaftlichen Diskurs-, Reputations- und Machtbegriffs}

\section{Interesse an der Öffnung wissenschaftlicher Kommunikation in der wissenschaftlichen Gemeinschaft}

Die durchgeführte empirische Studie zeigt, dass bei der überwiegenden Mehrzahl der Befragten ein Verständnis für die Forderung nach Offenheit in der wissenschaftlichen Kommunikation vorherrscht.

--- TODO: weiter ausarbeiten ---

\section{Was bedeutet Offenheit und freier Zugang im Rahmen des wissenschaftlichen Diskurs-, Reputations- und Machtbegriffs?}

\section{Welche Argumente für und gegen die Öffnung wissenschaftlicher formeller Kommunikation spielen in der Debatte eine Rolle?}

In der Debatte um die Öffnung von Wissenschaft und Forschung werden einige Faktoren genannt, die diesen Prozess beeinflussen. Die Presse- und Publikationsfreiheit wird in der Literatur häufig als Argument gegen diese Entwicklung angeführt. In der Befragung konnte diese Annahme nicht eindeutig bestätigt werden.

--- TODO: weiter ausarbeiten ---

\section{Welche sind die Haupteinflussfaktoren der Entwicklung um die Forderung von Open Access und Open Science?}

Ein weiteres Ziel dieser Arbeit war die Identifikation von Faktoren, welche die Öffnung von Wissenschaft und Forschung begünstigen. Diese Faktoren wurden anhand der Literatur herausgearbeitet und im Rahmen der durchgeführten Umfrage abgefragt. Die erhobenen Daten wurden statistisch ausgewertet und zusammenfassend dargestellt. In Bezug auf die Faktoren konnten folgende Feststellungen erarbeitet werden.

Dem überwiegenden Interesse, an der Öffnung wissenschaftlicher Kommunikation, steht die Befürchtung hoher Mehraufwände für die wissenschaftliche Praxis gegenüber. Die Befragung hat gezeigt, dass praktisch nur ein Bruchteil der Wissenschaftler und Wissenschaftlerinnen bereit sind, ihre gelebte Kommunikationspraxis zu ändern und ihre wissenschaftlichen Prozesse offen zu kommunizieren.

In der direkten Begründung werden laut der Umfrage vor allem rechtliche Unsicherheit, fehlende Mechanismen zu Reputationsbildung durch eine offene Arbeitsweise und sowie Unsicherheit bei der Zulässigkeit eines solchen Handelns genannt.

Von den Befragten wurden als Argumente für die Öffnung vor allem die beschleunigte Wissensverbreitung und die neue Möglichkeiten für die Kommunikation genannt. Demgegenüber stand neben den fehlenden, etablierten Reputationskriterien für die Bewertung von offener Wissenschaft, Gefahr der Fehlinterpretation und Falschinformation durch Wissenschaft genannt. Der vermutete erhöhte zeitliche Mehraufwand für die Bereitstellung der wissenschaftlichen Publikationen und Forschungsdaten untermalt den Eindruck, dass Wissenschaftler selbst noch keinen großen Druck verspüren, ihr Veröffentlichungsverhalten zu verändern. Das mag auch an dem Umstand liegen, dass sie trotz der Publikationskrise und anderen Faktoren, die zu einem Marktungleichgewicht im wissenschaftlichen Kommunikationssystem geführt haben, in einer komfortablen Situation sind beziehungsweise von den Auswirkungen der Krise bisher kaum oder nicht betroffen sind.

Die Ergebnisse zeigen aber auch, dass dieses Verständnis noch nicht zu einer Verhaltensänderung geführt hat.

--- TODO: weiter ausarbeiten ---

\section{Welche Aufwand bedeutet das offene Verfassen einer wissenschaftlichen Publikation in den Geistes- und Sozialwissenschaften (hier Doktorarbeit)?}


\section{Welche Handlungsempfehlungen für das Verfassen einer offenen Doktorarbeit können gegeben werden?}
