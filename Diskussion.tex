\chapter{Diskussion: Scientific Steady State vs. Second Scientific Revolution}

Die unbeschränkte und offene wissenschaftliche Kommunikation ist für das Wissenschaftssystem eigentlich unverzichtbar. Der digitale Wandel im wissenschaftlichen Publikationssystem stellt eine Chance dar, das tradierte System so zu justieren, dass es ohne Qualitätsverlust, ohne Einschränkung der Wissens- oder Publikationsfreiheit und vorbehaltlich einer angemessenen Zuordnung der Urheberschaft zu einer umfassenderen und schnelleren Verteilung von Wissen in der wissenschaftlichen Gemeinschaft aber auch an die gesamten Gesellschaft kommen kann - so die ursprüngliche Hoffnung.

Der Mediziner und Wissenschaftshistoriker Michael Hagner fasst diese Chance wie folgt zusammen: "Zum ersten Mal überhaupt in der Geschichte der Wissenschaften verfügen diese über ein Medium, das ihnen eine auf ihre Interessen hin zugeschnitten Tagesaktualität offeriert, die dem nahekommt, was Massenmedien wie Tageszeitung, Radio und Fernsehen früher bereits der Allgemeinheit anbieten konnte" \cite{hagner_2015_sache_buches}.

Wissenschaft und Forschung sind traditionell eng mit den Normen der schnellen Weitergabe von Forschungsergebnissen, einer Umgegbung des Wissensaustauschs, Co-Autorenschaft und das kumulative lernen und Innovation verbunden \cite{Partha_1994_economics_science}. Die aktuelle wissenschaftliche Praxis basiert aber auf einem insich geschlossenen System. In diesem Kapitel werden die Ergebnisse der Befragung, die Diskussionsstänge der Debatten in der Literatur um den Wandel der wissenschaftlichen Kommunikation im Rahmen der Digitalisierung diskutiert und um die eigene Sichtweise beim offenen Verfassen dieser Arbeit ergänzt.

Ziel ist es zu einer differenzierten Betrachtung der Ergebnisse der Befragung und des Experiments zu gelangen und diese in den Kontext der Fragestellungen der Arbeit zu stellen.

--- TODO: weiter ausarbeiten ----

\section{Welche Definition oder Aspekte in Definitionen von Open Access und Open Science sind am häufigsten verbreitet?}

Anhand der Literaturrecherche wurden unterschiedliche Definitionen von Open Access und Open Science evaluiert und eruiert, welche Aspekte von Öffnung in den meisten dieser Definitionen enthalten sind. Es wurde außerdem herausgearbeitet, welche Vermutungen in der Literatur darüber vorherrschen, warum die Öffnung von Wissen in den verschiedenen wissenschaftlichen Disziplinen unterschiedlich stark etabliert ist.

Für die theoretische Betrachtung des Bestrebens nach Öffnung in der Wissenschaft wurde ausserdem die Bedeutung von Offenheit und freien Zugang im Rahmen des wissenschaftlichen Reputationssystems herausgearbeitet.

--- TODO: weiter ausarbeiten ----

\subsection{Open Access}

Open Access hat sich in den letzen 25 Jahren zu der meistgenannten Lösung für die beschriebenen Krisen im wissenschaftlichen Kommunikations- und Publikationssystem entwickelt.

Seit den ersten Experimenten mit der Öffnung des Zugangs zu wissenschaftlichen Publikationen,  existieren mehrere Definitionen von Open Access und es bestehen unterschiedliche Auffassungen über die verschiedenen Modelle. Die genaue Definition von Open Access, war bis Anfang der 2000er Jahre vage. Die xxx ---- TODO: weiter ausarbeiten ---- stimmen in den den wesentlichsten Merkmalen überein\cite{albert_2006_open_implications}. Seit den Erklärungen hat sich kein

Folgende Aspekte werden am häufigsten genannt:

---- TODO: weiter ausarbeiten ----

Aus der Forderung nach "unbeschränkten Zugang zur gesamten wissenschaftlichen Zeitschriftenliteratur" ist ein gesamtgesellschaftliches und umfassendes Modernisierungsvorhaben der Wissenschaft geworden. Mit Blick auf die Umsetzung sind, anders als in es ursprünglich bei Open Access intendiert war, Einschränkungen der akademischen Freieheit absehbar \cite{hagner_2015_sache_buches}.

Die Zeit die seit der ersten Grundsatzerklärung zu Open Access in Budapest vergangen ist, hat gezeigt, dass die Abhängigkeit und Stabilität des wissenschaftlichen Kommunikationssystems größer ist, als zuerst angenommen: Der "Faustische Pakt stabiler ist als gedacht" \cite{hagner_2015_sache_buches} und die etablierte Publikationssystem der Verlage sind auch nach "zwei Jahrzehnten weitgehend stabil" \cite{Hanekop_2014} geblieben.

---- TODO: ausarbeiten ----

\subsection{Open Science}

Open Science geht über die reine Veröffentlichung der Ergebnisse hinaus und adressiert den gesamten Erkenntnisgewinnungsprozess. Die Herausforderung bei der Umsetzung sind dabei ähnlich denen von Open Access. Sie addressieren die Themen Steureung, Qualitätssicherung, Archivierung und Freiheit der Wissenschaft sind aber durch den Grad der Forderung nach Öffnung weitreichender. Die Lösung dieser Herausforderung kann nicht nur auf Grundlage von Vorgaben, Gesetzen oder Richtlinien erfolgen, sondern muss der Aufgabe neues Wissen zu produzieren und damit dem gesellschaftlichen Auftrag des Wissenschaftssystems gerecht zu werden.

Im Gegensatz zu Open Access gibt es bisher wenige Erklärungen die zur Definition von Open Access herangezogen werden können.

Angeleht an den Kurzusammenfassung der Sprachwissenschaftlerin Graefen für die Entwicklung von Wissenschaft, gilt auch bei für eine erfolgreiche Implementierung von Open Science in den wissenschaftlichen Alltag, ein gesamtgesellschaftlicher Bedarf für das zu öffnende Wissen bestehen muss und die entsprechende Leistungen von Individuen zu persönlichen führen. Nur dann kann eine Umorientierung von geschlossener individueller wissenschaftlicher Betätigung hin zu offener, gesellschaftlich anerkannter und zur Kenntnis genommener, kollektiv betriebener Wissenschaft stattfinden \cite{graefen2007_wissenschaftliche_artikel}. Ergänzend muss diese Entwicklung unter Beteiligung der Wissenschaft heraus gestaltet werden, da das System sonst gefahr läuft, sich zu "sporadischer individueller wissenschaftlicher Betätigung"\cite{graefen2007_wissenschaftliche_artikel} zurückzuentwickeln.

Weitere Voraussetzungen, ebenfalls angeleht an Graefen, sind:
\begin{itemize}
\item die technischen Möglichkeiten der schnellen Erstellung, Vervielfältigung und Verbreitung von wissenschaftlicher Kommunikation,
\item die Möglichkeit Wissenschaft(en) von anderen Formen der Information und Kommunikation zu unterscheiden
\item "die ökonomische und politische Nutzbarkeit von Wissenselementen, so daß
Forschung ein Mittel der Verwertung werden konnte",
\item die durch Ausbildungsprozesse gefestigte Existenz einer beruflich mit For-
schung befaßten 'Schicht' von Fachleuten und Wissenschaftlern.
\end{itemize}

---- TODO: ausarbeiten ----

\subsection{Verbreitung von und dem Interesse an Offenheit in den Disziplinen}

Das Interesse an Forschungsdaten anderer Wissenschaftler war in allen Disziplinen ähnlich stark ausgeprägt. Die Vision von Open Access hingegen fand auf hohem Niveau unterschiedlich viel Unterstützung. Die Ergebnisse der Befragung bezüglich der Verbreitung von und dem Interesse an Offenheit in den unterschiedlichen Disziplinen decken sich mit der tatsächlichen Akzeptanz und dem Grad an praktizierter Offenheit bei der wissenschaftlichen Kommunikation in der jeweiligen Disziplin.

---- TODO: Weiter ausarbeiten und Grafik aus DOAJ zum Vergleich bauen  ----

\subsubsection{Geisteswissenschaften}

---- TODO: Quelle Verbreitung OA in Geisteswissenschaften suchen  ----

Dass die Geisteswissenschaften am geringsten unter allen befragen Fachgruppen, aber dennoch mehrheitlich, die Forderung nach kostenfreiem Zugang zu allen wissenschaftlichen Publikationen für Leser (Open Access) zustimmen, deckt sich mit dem Stand der Verbreitung von Open Access in der Fachrichtung. Sie zeigen dennoch mehrheitlich Interesse auch Interesse an Forschungsdaten anderer. Die weiteren Zahlen lassen eine eher vorsichtig positive Haltung der Geisteswissenschaftler gegenüber den Forderungen von Open Access vermuten.

Die Auswertung der Antworten von Wissenschaftlern und Wissenschaftlerinnen aus der geistes- und sozialwissenschaftliche Fachrichtung zeichnen insgesamt ein ambivalentes Bild bezüglich dem Wunsch nach Öffnung wissenschaftlicher Kommunikation und der tatsächlich praktizierten Offenheit. So sind sich die Befragten Geisteswissenschaftler über die schlechte Zugänglichkeit Ihrer eigenen Beiträge für die gesamtgesellschaft bewusst. Die überwiegende Mehrheit der Befragten erachtet dennoch die freie Verfügbarkeit des eigenen Volltexts im Internet als eher weniger wichtig oder unwichtig. Für alle anderen Diziplinen hat die freie Verfügbarkeit der eigenen Texte eine höheren Stellenwert.

Nur ein geringer prozenteil der Befragten gab an, seine Volltexte selber auf Webseiten zur Verfügung zu stellen oder stellen zu lassen. Demgegenüber stellten die Geisteswissenschaftler die zweitgrößte Gruppe derer, die Inhalte schon einmal frei zugänglich publiziert haben und sie können sich unter allen Befragten Fachkollegien am besten Vorstellen, Forschungsdaten und alle weiteren Informationen, die während Ihrer wissenschaftliche Arbeit anfallen unter bestimmten Bedingungen, öffentlich zur Verfügung zu stellen.
Dabei spielt die Publikationsform der Monografien nur in den Geistes- und Sozialwissenschaften eine wichtige Rolle.

---- TODO: Auswertung Forschungsdaten zur Verfügung stellen  ----

Daraus kann man schließen, dass unter den Wissenschaftlern und Wissenschaftlerinnen dieser Fachrichtung zwar ein grundsätzliches Interesse an Öffnung wissenschaftlichehr Kommunikation besteht, diese aber in der praktischen Arbeit keine große Rolle spielt. Das mag dadurch begründet sein, dass die eigene Zugänglichkeit zu Publikationen durch die Wissenschaftler in der Fachgruppe überwiegend als gut oder sehr gut bewertet wird.

Das verwundert, denn gerade für die Geisteswissenschaften besteht an die Verbreitung von Wissen innerhalb der wissenschaftlichen Community und an die Gesamtgesellschaft, auch wenn die wird die Anzahl der Publikationen als überdurchschnittlich wichtiger Faktor für wissenschaftliche Reputation angesehen wird.

\subsubsection{Lebenswissenschaften}

---- TODO: weiter ausarbeiten ----

\subsubsection{Naturwissenschaften}

---- TODO: weiter ausarbeiten ----

\subsubsection{Ingenieurwissenschaften}

---- TODO: weiter ausarbeiten ----

\subsection{Bedeutung von Offenheit und freien Zugang im Rahmen des wissenschaftlichen Diskurs-, Reputations- und Machtbegriffs}

Wissenschaftliche Reputation basiert, wie in XXXXXs dargestellt auf einem System der Verbreitung neuer und der Adaption bestehender wissenschaftlicher Erkenntnisse. Unter dieser Prämisse scheint es selbstverständlich, dass Offenheit und freier Zugang zu wissenschaftlicher Kommunikation Grundpfeiler des wissenschaftlichen Diskurses und des Reputations- und Machtsystems sind.

In der Praxis sind diese System aber selbstreferenziell und auf die wissenschaftliche Gemeinschaft beschränkt. Es haben nur die Zugriff auf das System, die bereits Teil des Systems sind, oder die sich beginnen den Normen und Regeln des Systems anpassen. Die Gesamtgesellschaft ist von dem Diskurs in vielen Fällen ausgeschlossen, oder hat erst nach einer gewissen Zeit die Möglichkeit auf die Informationen und Erkenntnisse zuzugreifen. Somit manifestiert das System auch die Machtkonstelation derjenigen, die das System im Moment beeinflussen.

---- TODO: ausarbeiten ----

\section{Interesse an der Öffnung wissenschaftlicher Kommunikation in der wissenschaftlichen Gemeinschaft}


Die durchgeführte empirische Studie zeigt, dass bei der überwiegenden Mehrzahl der Befragten ein Verständnis für die Forderung nach Offenheit in der wissenschaftlichen Kommunikation vorherrscht.

Als ein positiven Effekt kann allerdings die nun notwendige Auseinandersetzung der Wissenschaftler und Wissenschaftlerinnen mit Rechten und der Verfügbarkeit der Publikationen gesehen werden.

---- TODO: weiter ausarbeiten ----


\section{Welche Argumente für und gegen die Öffnung wissenschaftlicher Kommunikation spielen in der Debatte eine Rolle?}

In der Auswertung der Befragung von 1.112 Wissenschaftlern konnte ein mehrheitlich stark verbreitetes Verständnis von Open Access und die mehrheitliche, grundsätzliche Unterstützung der Forderung nach Öffnung von Wissenschaft sowie Interesse an Forschungsdaten anderer nachgewiesen werden.

Von den Befragten wurden als Argumente für die Öffnung vor allem die beschleunigte Wissensverbreitung und die neue Möglichkeiten für die Kommunikation genannt. Demgegenüber stand neben den fehlenden, etablierten Reputationskriterien für die Bewertung von offener Wissenschaft, Gefahr der Fehlinterpretation und Falschinformation durch Wissenschaft genannt. Der vermutete erhöhte zeitliche Mehraufwand für die Bereitstellung der wissenschaftlichen Publikationen und Forschungsdaten untermalt den Eindruck, dass Wissenschaftler selbst noch keinen großen Druck verspüren, ihr Veröffentlichungsverhalten zu verändern. Das mag auch an dem Umstand liegen, dass sie trotz der Publikationskrise und anderen Faktoren, die zu einem Marktungleichgewicht im wissenschaftlichen Kommunikationssystem geführt haben, in einer komfortablen Situation sind beziehungsweise von den Auswirkungen der Krise bisher kaum oder nicht betroffen sind.

In der Debatte um die Öffnung von Wissenschaft und Forschung werden viele Faktoren genannt, die diesen Prozess vom geschlossenen System wissenschaftlicher Kommunikation hin zu einer Öffnung in unterschiedlichster Weise beeinflussen.

Bei der Alltagsbetrachtung stehen stehen hauptsächlich rechtliche Bedenken und fehlenden Reputationsmechanismen für die offene wissenschaftliche Kommunikation im Vordergrund. Das bestehende System macht es weder einfach, noch wird es honoriert, wenn Wissenschaftler und Wissenschaftlerinnen die eigene wissenschaftliche Kommunikation öffnen oder Ergebnisse frei zur Verfügung stellen.

In den wissenschaftstheoretischen Diskursen wird die Wissenschafts-, Presse- und Publikationsfreiheit oft als Argument gegen den Druck wissenschaftliche Inhalte zu öffnen angeführt. Mit Hilfe der Befragung konnte diese Annahme allerdings nicht mehrheitlich bestätigt werden.

---- TODO: weiter ausarbeiten ----

\section{Welche sind die Haupteinflussfaktoren für die Entwicklung um die Forderung von Open Access und Open Science?}

Ein weiteres Ziel dieser Arbeit war die Identifikation von Faktoren, welche die Öffnung von Wissenschaft und Forschung begünstigen. Diese Faktoren wurden anhand der Literatur herausgearbeitet und im Rahmen der durchgeführten Umfrage abgefragt. Die erhobenen Daten wurden statistisch ausgewertet und zusammenfassend dargestellt.

Dem überwiegenden Interesse, an der Öffnung wissenschaftlicher Kommunikation, steht die Befürchtung hoher Mehraufwände für die wissenschaftliche Praxis gegenüber. Die Befragung hat gezeigt, dass praktisch nur ein geringer Teil der befragten Wissenschaftler und Wissenschaftlerinnen tatsächlich bereit sind, ihre gelebte Kommunikationspraxis zu ändern und ihre wissenschaftlichen Prozesse offen zu kommunizieren. In der direkten Begründung wurden vor allem rechtliche Unsicherheit, fehlende Mechanismen zu Reputationsbildung durch eine offene Arbeitsweise und sowie Unsicherheit bei der Zulässigkeit eines solchen Handelns genannt.

Im Rahmen des offenen Verfassens dieser Arbeit ist aber auch deutlich geworden, dass für die offene Bereitstellung der wissenschaftlichen Kommunikation nur wenige und aufwändige Tools und Dienste zur Verfügung stehen.

Das die Befragten Wissenschaftler und Wissenschaftlerinnen die vorherrschenden Zeitschriften- und Monographienkrise als einen eher unwichtigen Treiber für die Öffnung wissenschschaftlicher Kommunikation identifiziert haben, unterstützt die Annahme des geringen Interesses an dem Thema und dass viele sich der Situation in der sich das System befindet nicht bewusst sind.

Die Ergebnisse zeigen deutlich, dass die Verschärfung der Situation in den letzten 10 Jahren kaum zu einer Verhaltensänderungen im Publikationsverhalten der Wissenschaftler geführt hat.

---- TODO: weiter ausarbeiten ----

\section{Welche Aufwand bedeutet das offene Verfassen einer wissenschaftlichen Publikation in den Geistes- und Sozialwissenschaften (hier Doktorarbeit)?}

Im Vergleich zum Publizieren von Texten in einer geschlossenen Umgebung, ist das offene Verfassen einer wissenschaftlichen Publikation noch immer mit viel Mehraufwand verbunden. Das liegt zum Einen daran, dass die genutzten Softwareprodukte die Veröffentlichung der Arbeit und der Kommunikationsprozesse noch nicht vollumfänglich und einfach ermöglichen, zum Anderen sind die Richtlinien und Vorgaben für wissenschaftliche Arbeiten an Universitäten und Forschungseinrichtungen nicht darauf ausgelegt offen zu arbeiten.

Die Kommunikation von Zwischenständen, die Veröffentlichung von Daten und die Dokumentation des offenen wissenschaftlichen Arbeitens erfordert ebenfalls mehr Aufwand als beim tradierten Arbeiten in geschlossenen Systemen. Der Aufwand ist dabei eng an die Vorkenntnisse im Umgang mit den verwendeten Systemen und Kanälen zur Dokumentation verbunden.

--- TODO: weiter ausarbeiten ----

\section{Welche Handlungsempfehlungen können für das Verfassen einer offenen wissenschaftlichen Arbeit gegeben werden?}

Im Rahmen dieser Arbeit wurden insgesamt 10 Empfehlungen erarbeitet, die für das Verfassen einer offenen wissenschaftlichen Arbeit berücksichtigt werden sollten:

--- TODO: Empfehlungen ausarbeiten ----

Die Auswahl der Software für die Text- und Datenverarbeitung spielt eine wichtige Rolle. Die Autorinnen und Autoren sollten von Beginn an, eine Lösung wählen, die es Ihnen einfach ermöglicht, den Text zeitnah im Internet veröffentlichen zu können. Der Aufwand und die Motivation die Arbeit zum Abschluss zu bringen hängt eng mit der genutzten Software zusammen.

Die zeitnahe Veröffentlichung von Daten ist aufwendig.

Vor dem Veröffentlichen des jeweils aktuellen Stands der Arbeit, sollten die Wissenschaftler sichergehen, dass sie diese Art und Weise der Publikation mit den Richtlinien der Institution oder den Voraussetzungen des jeweiligen finalen Veröffentlichungskanals vereinbar ist. Falls das nicht der Fall ist, sollte eine schriftliche Erlaubnis angefordert werden. Das gilt vor allem für wissenschaftliche Qualifikationsarbeiten.

Die Erwartungen an die Vorteile beim offenen Verfassen der Arbeit sollten nicht zu hoch gesteckt werden. Wer das offene Verfassen nutzen will um während des Schreibens zusätzliches Feedback über die Inhalte einzuholen, der darf sich (noch) nicht darauf verlassen. Das hängt auch mit dem Umstand zusammen, dass es sich um eine relativ neue Art der wissenschaftlichen Arbeit handelt. Vorteile liegen aber sicher in der Qualitätssicherung zum Beispiel durch offene und transparente Reviewverfahren bei der Veröffentlichung von Datensätzen.

--- TODO: weiter be- und ausarbeiten ----
