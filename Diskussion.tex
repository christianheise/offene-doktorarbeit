\chapter{Diskussion der Ergebnisse: Wissenschaftliche Kommunikation im Wandel - Scientific Steady State vs. Second Scientific Revolution}

\begin{quote}
\textbf{"The sciences are too good merely to avert attention from what science does."}
\end{quote} \cite{kittler_2004}

Der digitale Wandel im wissenschaftlichen Publikationssystem stellt eine Chance dar, das tradierte System so zu justieren, dass es ohne Qualitätsverlust, ohne Einschränkung der Wissens- oder Publikationsfreiheit und vorbehaltlich einer angemessenen Zuordnung der Urheberschaft zu einer umfassenderen und schnelleren Verteilung von Wissen in der wissenschaftlichen Gemeinschaft, auch an die gesamten Gesellschaft kommen kann - so die Hoffnung der Befürworter und Befürworterinnen des Wandels hin zur Öffnung wissenschaftlicher Kommunikation im Rahmen von Open Access. Der Mediziner und Wissenschaftshistoriker Michael Hagner fasst diese Chance wie folgt zusammen: "Zum ersten Mal überhaupt in der Geschichte der Wissenschaften verfügen diese über ein Medium, das ihnen eine auf ihre Interessen hin zugeschnitten Tagesaktualität offeriert, die dem nahekommt, was Massenmedien wie Tageszeitung, Radio und Fernsehen früher bereits der Allgemeinheit anbieten konnte" \cite{hagner_2015_sache_buches}.

Das Potenzial der Digitalisierung und die damit einhergehenden Möglichkeiten der Öffnung wissenschaftlicher Kommunikation sind jedoch noch viel umfassender und weitreichender. Sie stellen neben der Chance für einen freien und offenen Zugang zu finalen wissenschaftlichen Publikationen, auch erstmals die Möglichkeit für den Zugriff auf Daten und die Öffnung des gesamten wissenschaftlichen Erkenntnisprozesses dar. Wie im Verlauf der Arbeit dargestellt, hat das auch umfangreiche Konsequenzen auf die tradierte Klassifikation, Abgrenzung und Einordnung von wissenschaftlicher Kommunikation.

Ein Ziel dieser Arbeit war die Darstellung und Analyse der theoretischen Annahmen und unterschiedlichen Definitionsversuche rund um die Etablierung des Zugangs zu und des Zugriffs auf wissenschaftliche Erkenntnisprozesse. Es wurden Hindernisse und Katalysatoren für die Etablierung der beiden Konzepte anhand theoretischer und empririscher Betrachtungen ausgeführt, Grundannahmen aus der Literatur extrahiert sowie den Auffassungen und Meinungen der befragten Wissenschaftler und Wissenschaftlerinnen den eigenen Erfahrungen in einem Selbstversuch den praktischen Gegebenheiten im wissenschaftlichen Alltag gegenübergestellt. Dabei konnte eine Diskrepanz zwischen der Idee der Öffnung wissenschaftlicher Kommunikation und dem Interesse sowie den Möglichkeiten in der wissenschaftlichen Realität belegt werden.

In diesem Kapitel werden die Ergebnisse der durchgeführten Befragung, des Selbstexperiments, die Argumentationsstränge der Debatten zum Wandel der wissenschaftlichen Kommunikation in der Literatur im Rahmen der Digitalisierung und die zu Beginn der Arbeit formulierten Forschungsfragen abschließend diskutiert. Ziel ist es, zu einer differenzierten, zusammenfassenden Betrachtung der Ergebnisse der theoretischen Ausarbeitung, der Befragung und des Experiments zu gelangen und diese in den Kontext der Fragestellungen der Arbeit zu stellen und kritisch zu diskutieren.

\section{Welche Aspekte von Open Access und Open Science sind aktuell am häufigsten verbreitet?}

Anhand der Literaturrecherche wurde deutlich, dass die unterschiedlichsten Definitionsversuche der Begriffe Open Access und Open Science bestehen. Im folgenden wird eruiert, welche Annähreungsversuche an die Begrifflichkeiten als besonders sinnvoll erachtet werden, welche Vermutungen in der Literatur darüber vorherrschen, warum die Öffnung von Wissen in den verschiedenen wissenschaftlichen Disziplinen unterschiedlich stark etabliert ist und welchen Einfluss Offenheit und der freie Zugang auf das wissenschaftlichen Reputationssystems haben kann.

Eine eng gefasste begriffliche Abgrenzung der beiden Konzepte ist aktuell noch nicht möglich. Zu facettenreich sind die unterschiedlichen Anknüpfungspunkte, sowie die Entwicklungen und Eigenheiten in den unterschiedlichen wissenschaftlichen Disziplinen und Arbeitsbereichen. Als zentraler Annäherungspunkt dient in dieser Arbeit deshalb die präzise Definition des gemeinsame Attributs "Open" und eine damit einhergehende Verknüpfung mit der Open Definition \cite{open_definition}. Die Open Definition dient dabei als Rahmen für die rechtliche, technische und politische Ausrichtung der beiden Konzepte und addressiert vor allem die Grundlagen, wann zum Beispiel ein wissenschaftliches Werk, ein beliebiger Teil des wissenschaftlichen Erkenntnisprozesses oder ein Datensatz den Bedingungen entspricht um kompatibel mit der Idee von Offenheit zu sein.

\subsection{Open Access}

Open Access hat sich in den letzten 25 Jahren zu der meistgenannten Lösung für die beschriebenen Herausforderungen im wissenschaftlichen Kommunikations- und Publikationssystem entwickelt \cite{brembs2015open}. Seit den ersten Experimenten mit der Öffnung des Zugangs zu wissenschaftlichen Publikationen, existieren mehrere heterogen Definitionsansätze von Open Access und es bestehen unterschiedliche Auffassungen über die verschiedenen Modelle und Wege hin zu dem Ziel der Öffnung wissenschafticher Kommunikation. War die Eingrenzung von Open-Access bis Anfang der 2000er Jahre noch sehr vage, haben die drei "B"s einen Beitrag zur vereinheitlichung geleistet. Die Erklärungen stimmen in wesentlichen Merkmalen überein \cite{albert_2006_open_implications}, unterscheiden sich aber in ihrem Ausgestaltungs-, Auswirkungs- und Bezugsrahmen \cite{naeder_2010_open}. Das stellt die Etablierung von "Open Access" nachhaltig vor große Herausforderungen.

Johannes Näder fasst diese Situation wie folgt zusammen \cite{naeder_2010_open}:
Selbst nach der vollständigen Rezeption aller drei Erklärungen bleibt schließlich ein gewisser Interpretationsspielraum: Etwa hinsichtlich der Frage, ob ein Dokument auch dann dem Open-Access-Gedanken entspricht, wenn die in den beiden jüngeren Erklärungen geforderten Zusatzmaterialien nicht mitgeliefert werden und das Dokument erst nach Ablauf der vertraglichen Schutzfrist online zugänglich gemacht wird."

Will man dennoch zu einer gemeinsamen Rahmendefinition von Open Access kommen, könnte ein alternativer Ansatz das Anstreben einer gemeinsamen und eindeutige Definition von "Open" sein. Das ist auch deshalb sinnvoll um die ideelle Entwicklung von Offenheit in Wissenschaft und Forschung in ihrer ursprünglich gedachten Form auch in anderen Bereichen des offenen Wissens, in denen der Begriff "Open" verwendet wird, wie zum Beispiel im Rahmen offener Verwaltungs- und Regierungsdaten nicht nachhaltig zu gefährden.

Eine Grundlage dafür bietet die Open Definition \cite{open_definition}, die im Gegensatz zu den meisten Erklärungen von Open Access ständig weiterentwickelt wird und eine klare Abgrenzung zu nicht-open beinhaltet, ohne Bezug auf konkrete Publikationsformen, Prozesse oder den Kern ihrer Aussagen zu verwässern. Nach dieser Definition ist Wissen dann als "open" zu bezeichnen, "wenn jede/r darauf frei zugreifen, es nutzen, verändern und teilen kann – eingeschränkt höchstens durch Maßnahmen, die Ursprung und Offenheit des Wissens bewahren" \cite{open_definition}. In der Definition sind die Bereiche "Offene Werke", "Offene Lizenzen" und "Akzeptable Bedingungen" klar definiert. Was die Definition auslässt, ist einen zeitlichen Horizont zwischen Erstellung und Veröffentlichung der Inhalte.

Eine weitere Alternative könnte die Definition des Gegensatzes von "Open Access" darstellen. Über eine Definition von "Close Access" und die Präzision von dem was im Rahmen der Digitalisierung und Öffnung von Wissenschaft und Forschung seitens der wissenschaftlichen Gemeinschaft als nicht-wünschenswert gilt, könnte das Dilemma der unklaren Definition von "Open Access" benannt werden und trotzdem an den ursprünglichen kulturellen, politischen, und gesellschaftlichen Werten der Öffnung von wissenschaftlicher Kommunikation sowie den Eigenheiten der unterschiedlichen Fächer und Arbeitsweisen festgehalten werden.

Die Herangehensweise eine ausschließende Definition zu erstellen, erscheint auch deshalb als eine zielführende Alternative, weil die aktuellen Herausforderungen im Rahmen der Öffnung von wissenschaftlicher Kommunikation einfach zu heterogen erscheinen als dass sie nur über die Forderungen der ursprünglichen Erklärungen der "three B's" und die Definition was Open Access ist und was nicht, gemeistert werden können.

Insgesamt hat die Zeit, die seit der ersten Grundsatzerklärung zu Open Access in Budapest vergangen ist, gezeigt, dass die Abhängigkeit und Stabilität des wissenschaftlichen Kommunikationssystems größer ist, als zuerst angenommen und der "faustische Pakt stabiler ist als gedacht" \cite{hagner_2015_sache_buches}. Als Konsequenz zeig das tradierte wissenschaftlichen Publikationssystem bisher eine gewissen Veränderungsresistenz im wissenschaftlichen Kommunikationssystem und auch nach zwei Jahrzehnten bleibt das etablierte Publikationssystem der Verlage weitgehend stabil\cite{Hanekop_2014}.

\subsection{Open Science}

Open Science adressiert diese umfassenden Modernisierungsvorhaben und betrifft neben dem reinen Zugang zu wissenschaftlicher Kommunikation die größtmögliche Öffnung des gesamten wissenschaftlichen Erkenntnisgewinnungsprozess. Open Science greift demnach die politischen Ideale von Open Access auf und ergänzt sie um die notwendigen weitere praktische Aspekte zur Veränderung des wissenschaftlichen Kommunikationssystems. Die Probleme bei der Umsetzung des Konzepts ähneln dabei den Herausforderungen von Open Access sind dabei aber noch umfassender, da sie sich eben nicht nur auf den Zugang zu publizierten Wissen und auf die reine Kommunikation, sondern auf den Zugriff auf den gesamten wissenschaftlichen Prozess beziehen. Open Science adressiert also nicht nur das Thema Zugang zu Wissen, sondern ebenso die Themen Steuerung, Qualitätssicherung und Archivierung, sowie grundlegende Aspekte wie die Freiheit der Wissenschaft.

Im Gegensatz zu Open Access gibt es bisher nur wenige Erklärungen und Statements die eine Annährung an eine klare Definition von Open Science ermöglichen. Die Ziele bauen überwiegend auf vergleichbaren Ansätzen der Deklarationen von Open Access auf und können als weiterentwicklung dieser betrachtet werden. Wie bei dem Konzept um die Öffnung des Zugangs zu finalen wissenschaftlichen Publikationen, scheint eine fächerübergreifende Definition im Moment noch sehr impraktikabel. Daher scheint auch bei Open Science eine Annäherung an die Open Definition als Rahmen als zielführender zu sein. Diese Konzentration auf einen Rahmen stellt ebenfalls eine Herausforderung dar, da eine fehlende einheitliche und klare Definition der Wege und der Kriterien von Open Science zur Folge haben kann, dass das genannte Ziel von Open Science verwässert. Andererseits zeigen die Entwicklungen der letzten Jahre das genau dieser Rahmen wichtig ist, um nicht das Kernziel des möglichst umfassenden Zugriffs auf den wissenschaftlichen Prozess durch alltägliche Herausforderungen bei der Umsetzung aus den Augen zu verlieren.

Die Umsetzung solcher Open Science Initiativen kann und darf dabei nicht nur auf Grundlage von Vorgaben, Gesetzen oder Richtlinien erfolgen, sondern muss primär der Aufgabe dienen verbesserte Rahmenbedingungen dafür zu schaffen neues Wissen zu produzieren sowie zu verbreiten und dem gesellschaftlichen Auftrag des Wissenschaftssystems gerecht zu werden. Dazu müssen Anreize für die einzelnen Wissenschaftler und Wissenschaftlerinnen so gesetzt werden, dass deren Eigeninteresse mit dem Wohl der Wissenschaft und damit dem der Öffentlichkeit harmonieren \cite{brembs2015open}. Diese Anreize sollten dabei nicht nur wissenschaftsextern definiert und praktiziert werden, sonder müssen von der wissenschaftlichen Gemeinschaft selbst koordiniert und in das bestehende System der wissenschaftlichen Kommunikation eingebunden werden.

Bezüglich der zeitlichen Dimension der Veröffentlichung von wissenschaftlichen Inhalten, die die Open Definition nicht abdeckt und die auch in den Erklärungen von Budapest, Bethesda und Berlin nur unzureichend definiert sind, könnte Open Science eine Lösung darstellen. Denn es wird in diesem Konzept eine unmittelbare, möglichst umfassende und für jeden frei-verfügbare Veröffentlichung der Informationen im Rahmen wissenschaftlicher Erkenntnisprozesse angestrebt. Während bei Open Access die finale Publikation, maximal angereichert durch die darin referenzieren Daten, im Vordergrund steht, die in vielen Fällen auch erst nach einer gewissen Embargofrist zu veröffentlichen ist, greift die Verbreitung von Inhalten im Sinne von Open Science viel früher und weiter. Die ursprüngliche Publikation ist dabei eher als nachgelagertes und abschließendes Ergebnis zu betrachten.

Folgt man dieser Betrachtungsweise, sind die zeitlichen und rechtlichen Herausforderungen bei der Veröffentlichung finaler wissenschaftlicher Erkenntnisse in Form von Open Access Publikationen bei Open Science als nachgelagert zu betrachten. Bei einer Realisierung des Konzepts von Open Science geht es um die konstante Veröffentlichung wissenschaftlicher Ergebnisse im Rahmen von öffentlich-finanzierter Arbeitsumgebungen unter den Bedingungen der Open Definition. Dabei ist die möglichst umfassende Akkumulation von Beweisen ein wesentlicher Teil der wissenschaftliche Methode der Selbstkorrektur \cite{Nosek_2015}. Eine größtmögliche Offenheit des wissenschaftlichen Erkenntnisprozesses ist demnach auch zentral für das Ziel der Wahrheitsfindung.

Angelehnt an die Kurzzusammenfassung der Sprachwissenschaftlerin Graefen für die Entwicklung von Wissenschaft, gilt dennoch dass für eine erfolgreiche Implementierung von Open Science in den wissenschaftlichen Alltag, ein gesamtgesellschaftlicher Bedarf für das zu öffnende Wissen bestehen muss und die entsprechende Leistungen von Individuen zu persönlichen Vorteilen führen. Nur dann kann eine Umorientierung von geschlossener individueller wissenschaftlicher Betätigung hin zu offener, gesellschaftlich anerkannter und zur Kenntnis genommener, kollektiv betriebener Wissenschaft stattfinden \cite{graefen2007_wissenschaftliche_artikel}. Ergänzend muss die detailierte Ausgestaltung dieser Entwicklung unter Beteiligung der Wissenschaftlerinnen und Wissenschaftler gestaltet werden, da das System sonst Gefahr läuft, sich zu "sporadischer individueller wissenschaftlicher Betätigung" \cite{graefen2007_wissenschaftliche_artikel} zurückzuentwickeln.

Eine Voraussetzung für Open Science ist folglich, dass "wissenschaftlichen Daten, Codes und Volltexte sowie zum Teil das Lese-, Gutachter- und Lehrverhalten der Wissenschaftler zu einem großen Teil transparent werden" \cite{brembs2015open}. Einzige Einschränkung sollte die Privatsphäre der Wissenschaftler und Wissenschaftlerinnen sowie der Untersuchungssubjekte sowie der Datenschutz im Rahmen der Datenerhebung, Verbreitung und Verarbeitung sein. Weitere Voraussetzungen sind nach Graefen \cite{graefen2007_wissenschaftliche_artikel}:
\begin{itemize}
\item die technischen Möglichkeiten der schnellen Erstellung, Vervielfältigung und Verbreitung von wissenschaftlicher Kommunikation,
\item die Zugänglichkeit zum gesamten Forschungsprozesses
\item die Möglichkeit Wissenschaft(en) von anderen Formen der Information und Kommunikation zu unterscheiden
\item "die ökonomische und politische Nutzbarkeit von Wissenselementen, so daß
Forschung ein Mittel der Verwertung werden konnte",
\item die durch Ausbildungsprozesse gefestigte Existenz einer beruflich mit Forschung befassten 'Schicht' von Fachleuten und Wissenschaftlern.
\end{itemize}

\section{Was sind die Haupteinflussfaktoren für die Entwicklung um die Forderung von Open Access und Open Science?}

Ein weiteres Ziel dieser Arbeit war die Identifikation von Faktoren, welche die Öffnung von Wissenschaft und Forschung begünstigen. Diese Faktoren wurden anhand der Literatur herausgearbeitet und im Rahmen der durchgeführten Umfrage abgefragt. Die erhobenen Daten wurden statistisch ausgewertet und zusammenfassend dargestellt.

Von den Befragten wurden als Argumente für die Öffnung vor allem die beschleunigte Wissensverbreitung und die neue Möglichkeiten für die Kommunikation genannt. Demgegenüber stand neben den fehlenden, etablierten Reputationskriterien für die Bewertung von offener Wissenschaft, Gefahr der Fehlinterpretation und Falschinformation durch Wissenschaft genannt.

Der ebenfalls häufig genannte erhöhte zeitliche Mehraufwand für die Bereitstellung der wissenschaftlichen Publikationen und Forschungsdaten sowie die bisher geringe Veröffentlichung von offenen Inhalten stärkt die Vermutung, dass Wissenschaftler selbst noch keinen großen Druck verspüren, ihr Veröffentlichungsverhalten zu verändern. Das mag auch an dem Umstand liegen, dass sie trotz der Publikationskrise und anderen Faktoren, die zu einem Marktungleichgewicht im wissenschaftlichen Kommunikationssystem geführt haben, in einer komfortablen Situation sind beziehungsweise von den Auswirkungen der Krise bisher kaum oder nicht betroffen sind.

In der Debatte und in Erhebungen rund um die Öffnung von Wissenschaft und Forschung werden viele Faktoren genannt, die diesen Prozess vom geschlossenen System wissenschaftlicher Kommunikation hin zu einer Öffnung in unterschiedlichster Weise beeinflussen. Die Argumente erscheinen dabei als sehr vielseitig und facettenreich, konzentrieren sich aber im Kern auf 5 Aspekte:

---- TODO: ausarbeiten ----

Bei der Alltagsbetrachtung stehen hauptsächlich rechtliche Bedenken, fehlenden Reputations- und Qualitätssicherungsmechanismen für die offene wissenschaftliche Kommunikation im Vordergrund. Das bestehende System macht es weder einfach, noch wird es honoriert, wenn Wissenschaftler und Wissenschaftlerinnen die eigene wissenschaftliche Kommunikation öffnen oder Ergebnisse frei zur Verfügung stellen.

Andere Studien führen darüber hinaus, fehlende Infrastrukturen beziehungsweise die fehlende Integration in bestehende Strukturen sowie fehlendes Know-How bei den Wissenschaftler und Wissenschaftlern als Grund Argument für den geringen Öffnungsgrad in Wissenschaft und Forschung an \cite{eu_open_science_2015}.

In den wissenschaftstheoretischen Diskursen wird die Wissenschafts-, Presse- und Publikationsfreiheit oft als Argument gegen den Druck wissenschaftliche Inhalte zu öffnen angeführt. Mit Hilfe der Befragung konnte diese Annahme allerdings nicht mehrheitlich bestätigt werden.

---- TODO: weiter ausarbeiten und diskutieren ----

Wie bereits in anderen Studien festgestellt worden ist, wurden die Verfügbarkeit von digitalen Technologien und deren erhöhte Kapazität als einer der wesentlichen Einflussfaktoren für die Entwicklung um die Forderung von Open Access und Open Science identifiziert \cite{eu_open_science_2015}. Der Wunsch nach größerer Verbreitung von wissenschaftlichen Erkenntnissen und der Wunsch nach mehr Möglichkeiten zum kollaborativen Arbeiten spielen bei der Betrachtung der Haupteinflussfaktoren eine herausragende Rolle.

Dem Interesse an der Öffnung wissenschaftlicher Kommunikation, steht allerdings die Befürchtung hoher Mehraufwände für die wissenschaftliche Praxis gegenüber. Die Befragung hat gezeigt, dass nur ein geringer Teil der befragten Wissenschaftler und Wissenschaftlerinnen tatsächlich bereit sind, ihre gelebte Kommunikationspraxis zu ändern und ihre wissenschaftlichen Prozesse umfassend offen zu kommunizieren. In der direkten Begründung wurden insbesondere rechtliche Unsicherheit, fehlende Mechanismen zu Reputationsbildung durch eine offene Arbeitsweise sowie Unsicherheit bei der Zulässigkeit eines solchen Handelns genannt.

Im Rahmen des offenen Verfassens dieser Arbeit ist andererseits deutlich geworden, dass für die offene Bereitstellung der wissenschaftlichen Kommunikation nur wenige und aufwändige Tools und Dienste zur Verfügung stehen. Die meiste wissenschaftliche Open Source Forschungssoftware ist qualitativ minderwertig, ineffizient, undokumentiert und wird nur selten weiterentwickelt oder gepflegt \cite{hey_2015_open}. Ein grundlegendes Problem ist, dass Wissenschaftler und Wissenschaftlerinnen selten mit genügend Ressourcen für die professionelle Softwareentwicklung ausgestattet sind und nur überwiegend nur eine grundlegende Ausbildung und Erfahrung im Bereich Programmierung, Design, Testing, Debugging oder bei der Pflege von Software haben \cite{hey_2015_open}.

Das die Befragten Wissenschaftler und Wissenschaftlerinnen die vorherrschenden Zeitschriften- und Monographienkrise als einen eher unwichtigen Katalysatoren für die Öffnung wissenschaftlicher Kommunikation identifiziert haben, unterstützt die Annahme des geringen Interesses an dem Thema und dass viele sich der Situation in der sich das System befindet nicht bewusst sind. Sie nährt die Befürchtung, dass sich trotz effizienterer Verfahren und neuer Technologien nichts an dem System ändern wird \cite{Parks_2002_acadamic_faust}.

Die Ergebnisse zeigen dabei deutlich, dass die Verschärfung der Situation in den letzten 10 Jahren bisher nur vereinzelt zu einer Verhaltensänderungen im Publikationsverhalten der Wissenschaftler geführt hat. Dabei gibt es unterschiede zwischen den Disziplinen und bei der

---- TODO: weiter ausarbeiten, prüfen ob altersabhängig oder Statusabhängig und diskutieren ----

\section{Welche Bedeutung haben die Konzepte von Open Access und Open Science im Rahmen wissenschaftlicher Reputation?}

Wissenschaftliche Reputation basiert, wie in Kapitel xxxx dargestellt auf einem System der Verbreitung neuer und der Adaption bestehender wissenschaftlicher Erkenntnisse. Unter dieser Prämisse scheint es selbstverständlich, dass Offenheit und freier Zugang zu wissenschaftlicher Kommunikation Grundpfeiler des wissenschaftlichen Diskurses und des Reputations- und Machtsystems sind.

In der Praxis ist dieses System aber selbstreferenziell und auf die wissenschaftliche Gemeinschaft beschränkt. Es haben nur die Zugriff auf das System, die bereits Teil des Systems sind, oder die sich beginnen den Normen und Regeln des Systems anpassen. Die Gesamtgesellschaft ist von dem Diskurs in vielen Fällen ausgeschlossen, oder hat erst nach einer gewissen Zeit die Möglichkeit auf die Informationen und Erkenntnisse zuzugreifen. Somit manifestiert das System auch die Machtkonstellation derjenigen, die das System im Moment beeinflussen.

Im Rahmen der Identifikation von Treibern und Bremsern für die Verhaltensänderung hin zur Öffnung der wissenschaftlichen Kommunikation, stellen die fehlenden Reputationsmechanismen eine der größten Herausforderungen für die Verbreitung dar. Ebenso wurden fehlende Qualitätssicherungsmaßnahmen besonders häufig bei den Argumenten gegen die Verbreitung von Open Science erwähnt. Das bestätigten auch andere Studien \cite{eu_open_science_2015}.

Die bisher genutzten bibliometrischen Verfahren erfreuen sich noch immer großer Beliebtheit und haben eine Absicherungsfunktion bezüglich der vermuteten Qualität einer Publikation, die bisher noch kein digitales Äquivalent gefunden hat. Da auch die Forschungsförderung auf den tradierten Verfahren der Evaluation von Forschung aufsetzt, ist diesbezüglich noch kein Wandel abzusehen.

Es ist davon auszugehen, dass die Möglichkeit der Erlangung von wissenschaftlicher Reputation einen Einfluss auf die Verbreitung des Konzepts von offenem Zugang zu Wissenschaft hat, nicht umgekehrt. Die bestehenden digitalen Bewertungssysteme haben in der Wissenschaftssteuerung bisher nur begrenzt Einzug gehalten. So kann noch keine abschließende Beantwortung der Frage stattfinden, ob die Öffnung der Kommunikation einen Einfluss auf das Konzept der wissenschaftlichen Reputation innerhalb und außerhalb der wissenschaftlichen Community hat.

---- TODO: weiter ausarbeiten, prüfen ob altersabhängig oder Statusabhängig und diskutieren ----

\section{Welchen Einfluss haben die Entwicklungen um die Öffnung wissenschaftlicher Kommunikation auf die Rolle der Universität?}

In der praktischen Auslegung der Entwicklungen um die Universität wird von der Entmythologisierung der Humboldt’schen "Einheit von Forschung und Lehre" gesprochen \cite{binswanger_2014_excellence} \cite[:299]{Schimank_2001} \cite[:343]{Kruecken_2001} und es ist nicht zu verleugnen, dass in der Wissenschaft zunehmend ein Zusammenhang zwischen ökonomischer Effizienz, Kontrollmechanismen und Öffentlichkeit herrscht \cite[:27]{Reinhart_intransparenz_2006} \cite{foucault_1977_uberwachen}. Diese hat jedoch nicht erst mit dem steigenden Kosten- und Effizienzdruck, der Verwertbarkeit von Wissenschaft und Forschung, sowie der Modernisierung der Steuerungsmechanismen stattgefunden, sondern schon viel früher wurden die Ausrichtung der Universität auf die Verwertbarkeit wissenschaftlichen Wissens kritisiert \cite{Huber_2005}. Die Idee der Einheit von Forschung und Lehre, auf Grundlage eines völligen Verzichts auf Differenzierung \cite{kittler_2004}, lässt sich grundsätzlich somit nur in Ausnahmefällen realisieren \cite{Schimank_2001}. Als realistische Lesart kann im vorherrschenden System nur eine situative Differenzierung stattfinden, bei der die Mittel der Grundausstattung nicht nach beiden Aufgaben separiert sind \cite{Schimank_2001}.

Diese Lesart der Humboldt’schen Idee ist noch immer hegemonialer Rahmen der aktuellen Hochschulreformen \cite{Huber_2005}. Das Recht auf Freiheit von Lehre und Forschung und die humboldtsche Idee der Universität wird und wurde immer für die Erhaltung des "organisationellen Status Quo", die Absicherung der "Institution Universität" und die Wahrung der "Staatsunabhängigkeit" angebracht \cite{Huber_2005}. Diese Autonomie der Wissenschaft und Forschung gilt auch heute als "hohes Gut, das es gegen externe Anforderungen zu verteidigen gilt"\cite{kaldewey_2010}.

---- TODO: ausarbeiten und diskutieren ----

\section{Welcher Aufwand entsteht bei der Öffnung des gesamten wissenschaftlichen Erkenntnisprozess?}

Im Vergleich zum Publizieren von Texten in einer geschlossenen Umgebung, ist das offene Verfassen einer wissenschaftlichen Publikation noch immer mit Mehraufwand verbunden und teilweise ohne technische Vorkenntnisse nicht ohne weiteres durchführbar - das hat auch das Selbstexperiment gezeigt. Das liegt zum Einen daran, dass die genutzten Softwareprodukte und -plattformen die Veröffentlichung der Arbeit und der gesamten Erkenntnis- und Kommunikationsprozesse noch nicht vollumfänglich und einfach ermöglichen, zum Anderen sind die Richtlinien, Rahmenbedungen, Anreize und Vorgaben für wissenschaftliche Arbeiten an Universitäten und Forschungseinrichtungen nicht darauf ausgelegt "offen" durchgeführt zu werden.

34 Prozent der Befragen sahen einen erhöhten zeitlichen Mehraufwand für die Bereitstellung der wissenschaftlichen Publikationen und/oder Forschungsdaten. 30 Prozent schäzten den letzendlichen Aufwand um Publikationen und bei Forschungsdaten nur 10 Prozent als "gering" ein. Unter den Teilnehmerinnen und Teilnehmern im Rahmen der Online-Befragung gaben demgegenüber zwar nur 15 Prozent an, dass sie der Aufwand davon abhält die eigenen wissenschaftlichen Inhalte ohne finanzielle, rechtliche oder technische Barrieren öffentlich zur Verfügung zu stellen, doch die tatsächliche Anzahl der veröffentlichen Informationen der Umfrageteilnehmer ist bedeutent geringer. Rund 20 Prozent der befragten Wissenschaftler und Wissenschaftlerinnen war nicht in der Lage den Aufwand zu bewerten und weitere 33 Prozent vermuteten sogar einen großen Aufwand  und 22 Prozent.

Diese Unsicherheit und der vermutete erhöhte Aufwand sowie fehlende Anreizsysteme hemmen die Weiterentwicklung hin zur Öffnung des Systems. Perspektivisch werden die Veränderungen im Rahmen der Forschungsförderung werden nicht aussreichen, um Wissenschaftlerinnen und Wissenschaftler dazu zu überzeugen den gesamten Erkenntnisprozess zu öffnen.

---- TODO: weiter ausarbeiten und anhand der eigenen Erfahrung diskutieren; vorallem prüfen wie schätzen die den aufwand ein, die schon forschungsdaten veröffentlicht haben vs. die, die noch keine veröffentlicht haben ----
