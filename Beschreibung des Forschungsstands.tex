\subsection{Beschreibung des Forschungsstands}
In diesem Teil der Arbeit soll der Forschungsstand umfassend analysiert und beschrieben werden. Es soll mit Hilfe einer ausgewogenen und umfassenden Literaturanalyse dargestellt werden, welche Argumentationen es für und gegen sowie welche Möglichkeiten und Grenzen es für die Öffnung der Wissenschaft im Rahmen von Veröffentlichung angeführt werden, soweit das der derzeitige sehr lückenhafte Forschungsstand erlaubt. Eine kritische Analyse soll dabei Pro- und Kontraargumente zusammenfassen und einen Überblick über die aktuelle Debatte um Open Science und Open Access ermöglichen. Diese Analyse wird auf der Annahme durchgeführt, dass sich Open Access in einer Übergangsphasen von der reinen offenen Bereitstellung wissenschaftlicher Publikationen und dem damit verbundenen offenen Zugang zur Wissenschaft zur umfassenden und offenen Wissensverteilung und dem damit einherdehnenden Zugriff auf Wissenschaft an die Gesamtgesellschaft (Open Science) befindet. Darüber hinaus sollen medienkulturwissenschaftlich Open Science und Open Access in ihren technischen als auch in ihren gesellschaftlichen und politischen Aspekten sowie die kulturellen Auswirkungen der Medienbrüchen im Rahmen von hybridem Publizieren auf hohem theoretischem Niveau reflektiert werden. Abschließend sollen Treiber und Bremser für die Öffnung von Wissenschaft empirisch erhoben werden und in der Gesamtbetrachtung der Arbeit zusammengeführt werden.