\subsection{Wissenschaftliche Reputation}
Ein Grundprinzip des Wissenschaftssystems basiert auf der "gegenseitigen Beurteilung und Anerkennung der jeweils neuen Ergebnisse ihrer Fachkollegen (Peers) durch die Wissenschaftler selbst"\cite{Hanekop_2014}. In dem Peer-Review Prozess "werden eingereichte Beiträge von fachlich versierten Wissenschaftlern (...) beurteilt und gemäß den qualitativen Anforderungen der Forschungs-Community zur Veröffentlichung angenommen oder abgelehnt" \cite{Hess_2006}. Dabei bilden "Publikationen im Hinblick auf die Funktion der Reputationsverteilung eine Art Telos wissenschaftlicher Kommunikation"\cite{hirschauer2004peer}. Im Rahmen von Reputation ist wissenschaftliche Arbeit besonders auf funktionierendes Peer-Review-System angewiesen\cite{suchen}. Dennoch haben qualitatives Peer Review-Systeme und quantitative bibliometrischen Verfahren viele Mängel\cite{osterloh2008anreize}.

In diesem Kapitel soll wissenschaftliche Reputation genau definiert und ihre Abhängigkeit zum bestehenden wissenschaftlichen Kommunikationssystem herausgestellt werden. Wissenschaftliche Reputation soll hier als Währung verstanden werden, mittels derer “Status und Ressourcen verteilt werden”\cite{hanekop_2006}. Wissenschaftliche Reputation verteilt sich auf Einrichtungen und einzelne Personen, die wissenschaftlich tätig sind\cite{suchen}. 

Die Evaluation wissenschaftlicher Einrichtungen findet dabei über “Beobachtungen und Gespräche mit den Wissenschaftlern vor Ort sowie über den Austausch über die Eindrücke innerhalb der Begehungsgruppe und die gemeinsame Verständigung”\cite{Barl_sius_2008} statt. Diese wird in dieser Arbeit nur beiläufig betrachtet.

Die Reputation einzelner Wissenschaftler steht in dieser Arbeit im Vordergrund. Für sie sind Publikationen und die damit einhergehende Verbreitung von wissenschaftlichen Erkenntnissen sehr entscheidend \cite{Hess_2006}. Vereinfacht lässt sich das System der Wechselbeziehungen der Reputationsverteilung im Rahmen von Publikationen wie folgt darstellen\cite{cite:21a}: \href{http://www.eap-journal.com/archive/v39_i1_8_bernius.pdf}{Grafik aus Text von Bernius}

Bernius et al. unterscheiden in diesem Zusammenhang zwischen drei koordinierenden Marktmechanismen aufeinandertreffen: die Reputation und die Nutzung von wissenschaftlichen Publikationen, sowie der Preis für den Erwerb. Während die Reputation ein Aushandlungsmechanismus zwischen den Verlagen und wissenschaftlichen Autoren darstellt, findet die Preisdefinition zwischen den Bibliotheken und den Verlagen statt. Das Zusammenspiel zwischen Wissenschaftlern und Bibliotheken beschränkt sich dabei auf die Frage der Nutzung.\cite{cite:21a}

Doch nicht jede Publikation hat die gleiche Wertigkeit\cite{suchen} und damit den gleichen Einfluss auf die Reputation. Die neuen Möglichkeiten der Verbreitung von Informationen lassen deshalb einen vergleichbaren Veränderungsprozess der wissenschaftlichen Verbreitung und damit auch Anerkennung vermuten, die wie Entwicklung des Drucks.\cite{hanekop_2006} Neben dem Publizieren müssen auch folgende weitere Indikatoren für wissenschaftliche Reputation für wissenschaftliche Institutionen und Personen genannt werden\cite{hanekop_2008}:
\begin{enumerate}
\item Drittmittelprojekte
\item Patente
\item Vorträge
\item Anwendungsrelevanz bzw. Verwertbarkeit
\item Netzwerke
\item öffentliche Aufmerksamkeit sowie politische Relevanz 
\item Renommee der Forschungseinrichtung
\item materielle Ausstattung, Großgeräte etc.
\item personelle Ausstattung
\item Gutachtertätigkeit
\item Herausgeberschaft
\item Funktion
\end{enumerate}
In diesem Rahmen wurden durch den US-amerikanische Soziologe Robert K. Merton vier Grundprinzipien als normative Struktur der Wissenschaft beschrieben\cite{Merton_1985}, auf deren Grundlage in diesem Kaptitel wissenschaftliche Reputation und das damit einhergehende Anreizsystem, sowie deren Veränderungsprozess durch die Öffnung von Wissenschaft erläutert werden soll. 

\subsubsection{Wissenschaftliches Kapital}
Im Rahmen der Betrachtung von Steuerungs- und Reputationsmethoden für Wissenschaft ist der Begriff wissenschaftliches Kapital als zentral zu sehen\cite{suchen}. Wissenschaftliches Kapital kann dabei als eine Ausprägung des kulturellen Kapitals und als symbolisches Kapital\cite{irmer2011} verstanden werden. “Scientia donum dei est, unde vendi non potest", daß die Wissenschaft nicht verkauft werden kann, weil Wissen eine Gabe Gottes ist, sollen die Gelehrten des Mittelalters gerufen haben\cite{suchen}. Bereits damals war aber abzusehen, daß dieses Verständnis nicht lange halten würde.

Wie bereits in der Einleitung erwähnt, basiert die “Gewährung wissenschaftlichen Kapitals” im wissenschaftlichen System heute auf einer engen Verbindung zwischen den Verlagen und den publizierenden Wissenschaftlern\cite{herb_2006}. Dabei steht die Wissenschaft in einer klaren Abhängigkeit zu den Verlagen. Ulrich Herb definiert in diesem Zusammenhang wissenschaftliches Kapital mit Hilfe Pierre Bourdieus als das “Ergebnis einer Investition (...), die sich auszahlen muss” ist. “Diejenigen, die diese Berechtigungsscheine in der Hand halten, verteidigen ihr 'Kapital' und ihre 'Profite', in dem sie diejenigen Institutionen verteidigen, die ihnen dieses 'Kapital' garantieren.”\cite{Bourdieu_1992} Wissenschaftler sind also von einem Ressourcenzufluss abhängig um Ihre Aufgaben dauerhaft wahrzunehmen\cite{Suess_2006}. Herb kommt zu dem Schluss, dass die Öffnung von Wissenschaft dabei bisher nicht wissenschaftlicher Logik folgt, “sondern einer feldunabhängigen Logik der Akkumulation von Kapital”\cite{herb_2006}. Hinzu kommt, dass vor allem das deutsche Wissenschaftssystem durch durch die Einführung an Outputs orientierter Anreizsystem gekennzeichnt ist.\cite{osterloh2008anreize}

Als Beispiel kann der Performanzindikator "Drittmittel"\cite{Jansen_2007} dienen, in der Wissenschaft neben der Sicherung der Qualität von Forschung und Lehre zunehmend direkte finanzielle und administrative Kontrolle eine Rolle spielen\cite{Barl_sius_2008},. Daraus resultiert die Gefahr, dass nicht nur die Erwartungen an die Bewertung von Wissenschaft zu hoch gegriffen sind, sondern auch, dass sich Wissenschaft zu sehr an diesen Erwartungen orientiert und die Legitimität öffentlicher Ausgaben über den Zweck gestellt werden. Vor allem die Verknüpfung von wissenschaftlicher Reputation und der damit einhergehenden Verteilung von Mittel und Stellen stellt eine Herausforderung an das Wissenschaftsystem, “dessen Währung [ursprünglich] nicht Geld ist”\cite{hanekop_2006} dar. In diesem Teil der Arbeit soll das Konzept des wissenschaftlichen Kapitals vor dem Hintergrund des Widerspruchs gegenüber den Grundprinzipien der Wissenschaft erläutert und dargestellt werden, welche Möglichkeiten die Öffnung von Wissenschaft für die Legitimität öffentlicher Ausgaben darstellt, ohne das diese über den eigentlichen Zweck gestellt werden.
\subsubsection{Die Ökonomie des wissenschaftlichen Kommunizierens}
Die klassische Ökonomie der wissenschaftlichen Kommunikation beruht auf der Durchsetzung von Urheberrechten, die den Zugriff und die Wiederverwendung von geschützten Inhalten beschränken und die Zahlung einer Gebühr durch den Leser verlangen um Zugang zu der Veröffentlichung zu erhalten.\cite{CREATe_2014} Dieses Modell baisert auf einer sozial ineffizientem Ebene\cite{mueller-langer_2010}. Diese ungewöhnlichen Ökonomie der Wissenschaftsverlage ist nicht neu und hat sich im Laufe der Zeit entwickelt, die starke Wahrnehmung der Ungerechtigkeit des Systems, vorallem an den Preismodellen für wissenschaftliche Publikationen\cite{King_2008}, findet aber erst seit kurzem statt\cite{CREATe_2014}.

Als weitere wesentliche Besonderheit der Wissenschaftskommunikation ist die Organisation des Marktes, die von spezifischen Akteuren und Prozessen geprägt ist \cite{Hess_2006}, zu nennen. Vereinfacht kann der klassische wissenschaftliche Kommunikationsprozess im Rahmen von Publikationen wie folgt unterteilt werden\cite{cite:11b} \cite{Hess_2006}:
\begin{enumerate}
\item Erstellung durch Wissenschaftler - Inhalte erzeugen: 
Nach der Entwicklung eines konkreten Forschungsvorhabens sowie einer wissenschaftlichen Fragestellung enstehen im Rahmen der wissenschaftlichen Forschung oder der jewiligen Untersuchung Informationen\cite{cite:11c}, die im Forschungsprozess gesammelt, analysiert, ausgewertet, aufbereitet und verschriftlicht wurden\cite{cite:11d}. Diese Infromationen werden strukturiert zusammengefasst und niedergeschrieben \cite{Hess_2006}.
\item Qualitätskontrolle durch Wissenschaftler - Inhalte bewerten: 
Die Qualitätskontrolle ist einer der wesentlichen Bestandteile der wissenschaftlichen Kommunikation. Sie sichert die gewonnen Erkenntnisse\cite{cite:11e} und stellt einen klaren Abrenzungsaspekt zu nicht-wissenschaftlichen Informationen und Erkenntnissen dar\cite{cite:11f}. Sie findet im Kommunikationsprozess an zwei Stellen statt. Hier ist die erste Stelle gemeint, in der vor der Produktion der Informationen in Form der Publikation, die Erkenntnisse von anderen Wissenschaftlern überprüft und gesichert werden (Peer-Review) \cite{Hess_2006}.
\item Bündlung durch Verlage - Inhalte auswählen:
\item Publikation durch Verlage - Inhalte distribuieren: 
Nach Erstellung und Erkenntnissicherung findet die für die Distribution notwendige Publikation der Informationen statt. Vor der digitalen Revolution bestand dieser Schritt ausschließlich in dem Druck auf Papier.\cite{cite:11h}
\item Distribution durch Verlage: 
Der Vertrieb und die Verbreitung der Inhalte ermöglicht den Zugriff auf die Information der Forschung durch andere Wissenschaftler. Der Schritt stellt einen wichtigen Teil zur Zirkulation des neu gewonnen Wissens dar\cite{cite:11i}. Er sichert die Verfügbarkeit und den Zugriff auf die Informationen und ist essentieller Teil des Selektionsprozesses für die Erschaffung neuen Wissens.\cite{cite:11l}
\item Suppoert und Archivierung durch Bibliotheken
\item Konsum beziehungsweise Rezeption durch Wissenschaftler: 
Der nächste Schritt im wissenschaftlichen Kommunikationsprozess, der wiederum den gesamten Prozess von vore beginnen lässt ist die Rezeption der veröffentlichten Inhalte. Hier geht es zum einen um die Rezeption der wissenschaftlichen Vorschung aus der "Erstellung", zum anderen kommt hier auch die zweite Stufe der Qualitätsicherung zum Tragen.\cite{cite:11j} Der Konsum von wissenschaftlicher Informationen ist dabei auch als Grundlage für die "Erstellung" neuen Wissens zu betrachten. Somit ist der Endpunkt des wissenschaftlichen Kommunikationsprozess auch gleichzeitig Ausgangspunkt für einen neuen Prozess\cite{cite:11k}.
\end{enumerate}

An diesem Prozess sind vor allem drei Gruppen beteiligt: erstens die Wissenschaftler, als Produzenten und Konsumenten der Informationen, zweitens die Verleger, die als Intermediäre wissenschaftliche Informationen sammeln, bündeln und verkaufen, sowie drittens die Bibliotheken, die die Informationen wieder den Wissenschaftlern zur Verfügung stellen \cite{Odlyzko_1997}. Aus diesem Prozess und den beteiligten Gruppen, werden folgende Problemfelder ersichtlich:


klassisches Geschäftsmodell/Wertschöpfungskette vs. Open Access Geschäftsmodell/Wertschöpfungskette \cite{Hess_2006}

\subsubsection{Messbarkeit wissenschaftlicher Qualität vs. Publikationsquantität}
Wissenschaft ist ein Prozess, bei dem aus “unterschiedlichen Inputfaktoren, mittels verschiedener Transformationen Beiträge zur Schaffung neuer wissenschaftlicher Erkenntnisse als Output entstehen”\cite{Jansen_2007}. Die Bewertungen des jeweiligen Outputs führt “zur Ausage über die Forschungsperformanz”. Neben den Indikatoren für den Output wissenschaftlicher Perfomanz, müssen aber auch intermediäre Aspekte berücksichtigt werden\cite{schmoch_2009}. Nach diesem Ansatz etablierten sich nach dem zweiten Weltkrieg die ersten Indikatoren für die Effizienzmessung wissenschaftlicher Wissensproduktion. Seit den 1960er Jahren wird diese Messung in der Gestalt von Indikatoren, die die Forschungsleistung quantifizieren sollen, durchgeführt. Seit den 1990er Jahren fällt diese Bewertung in Gestalt von Zahlen als unkontrollierte Nebenprodukte digitaler Wissenskommunikation an\cite{angermueller_2010}. Heute zählen in der Wissenschaft vor allem die wissenschaftliche Reputation und die als Impact bezeichnete Wirkung wissenschaftlicher Publikationen\cite{herb_open_2013}. Die Wirkung wird dabei anhand der Zitationen der jeweiligen Publikation gemessen. Eine häufige Zitation stellt dabei einen Indikator für einen große Wirkung der wissenschaftlichen Arbeit dar. In diesem Kapitel werden die gängigen Methoden und Möglichkeiten der Messbarkeit von wissenschaftlicher Reputation und Wirkung im Kontext der Arbeit dargestellt.
\subsubsection{Wissenschaftliche Diskurse, nach dem Diskurs- und Machtbegriff}
Nach Niklas Luhmann operiert der wissenschaftliche Diskurs funktional eigenständig und alles was durch Wissenschaft kommuniziert wird, ist “entweder wahr oder unwahr”\cite{Luhmann1998}. Der wissenschaftliche Diskurs gründet dabei aber nur zum Teil auf der Forschung und kann auch nicht nur als “Kontaktglied zwischen dem Denken und dem Sprechen”\cite{foucault_ordnung_2003} definiert werden. In der Foucault'schen Diskursanalyse wird der Diskurs deshalb als die Fähigkeit definiert, die “Beziehungen” zwischen “Institutionen, ökonomischen und gesellschaftlichen Prozessen, Verhaltensformen, Normsystemen, Techniken, Klassifikationstypen und Charakterisierungsweisen herzustellen”\cite{foucault_archaologie_1981}. Foucault beschäftigt sich in diesem Zusammenhang vor allem mit den Grenzen des Diskurses sowie dessen institutioneller und praktischer Verortung. In diesem Zusammenhang soll in dieser Arbeit auch adressiert werden inwiefern Macht, regulierende Prinzipien wie Verknappung sowie die Ein- und Ausgrenzung bezüglich des wissenschaftlichen Diskurses, nach dem Diskurs- und Machtbegriff von Michel Foucault, mit den Modellen der Open Initiatives in der wissenschaftlichen Kommunikation vereinbar sind oder dem gegenüberstehen. Im Gegensatz zu innerdiziplinären Betrachtung eignet sich Foucaults “Werkzeugkiste”\cite{Honneth_2003} dabei besonders um die transdisziplinäre Öffnung von wissenschaftlichen Prozessen und den damit einhergehende Öffnung des Diskurses theoretisch zu hinterfragen. 
In diesem Kapitel soll deshalb der Diskursbegriff in den Kontext der Thematik der Öffnung des Zugriff auf den wissenschaftlichen Prozess erläutert werden.
\subsubsection{Kritik}
Die Verlage haben mit Hilfe von wissenschaftlichen Journalen ein zentrales Steuerungs- und Bewertungssystem in der Wissenschaft etablieren können. Dabei werden die Grundprinzipien der Wissenschaft für die verlegerischen Verwertungsinteressen genutzt und das obwohl diese “wissenschaftlichen Grundprinzipien und Normen eigentlich ökonomischen Verwertungsinteressen zu widersprechen scheinen”\cite{hanekop_2006}. Darüber hinaus haben die Forscher in vielen Fällen wenig oder keine Verantwortung für den Einkauf von der wissenschaftlichen Informationen, die er oder sie "verschenkt"\cite{steele_2006}. Spätestens durch die Einführung von Zitationsregistern und Impact Faktoren sowie die Definition der Kernzeitschriften hat den wissenschaftliche Zeitschriftenmarkt extrem unelastisch gemacht und gleichzeitig die Kapazität der kommerziellen Verlagen sowie deren Gewinnmargen ansteigen lassen.\cite{CREATe_2014} Die Steuerungsmechanismen werden über die Messbarkeit an Hand der in 2.3.4 beschriebenen Methoden direkt oder indirekt ausgeübt. Dabei stehen insbesondere die Methoden, die auf der quantitativen Grundlage der Zitationsraten wissenschaftlicher Publikationen gemessen werden in der Kritik\cite{Dong_2005} und auch andere Indikatoren für die Messung von Forschungsleistungen sind hoch umstritten\cite{Hornbostel_1997}\cite{Hicks_1996}\cite{Havemann_2002}. Der Hauptkritikpunkt: Die Verfahren, um die Wirkung von Wissenschaft und damit auch die Reputation von Wissenschaftlern zu messen, sind kein eigentliches wissenschaftliches Produkt\cite{suchen} und erfassen zum Beispiel die Tätigkeit einzelner Forschergruppen zu stark\cite{schmoch_2009}. Das führt unter anderem dazu, dass der Impact Factor “kein perfektes Werkzeug (ist) um die Qualität der Artikel zu messen” und trotzdem wird er zur Bewertung von Wissenschaft genutzt, denn “(...) es gibt nichts Besseres, und er hat den Vorteil, dass er bereits lange existiert und ist daher eine gute Technik für die wissenschaftliche Bewertung”\cite{garfield_1999}. Wie “gering der Wirkungsgrad” und die Methoden zur Messung “zur Reproduktion des traditionellen wissenschaftlichen Diskurses ausfall(en), wird von dem Moment an klar, an dem ein neues und offenes Kommunikationsmedium wie das Internet als alternativer Publikations- und Verbreitungskanal für Wissenschaft zur Verfügung steht\cite{Rost_1998}. Im folgenden soll aufgezeigt werden, welche Kritik es an dem System der wissenschaftlichen Reputation sowie dem Steuerungsmodell durch die Verlage im Rahmen der Veränderung der wissenschaftlichen Publikationskanälen gibt.