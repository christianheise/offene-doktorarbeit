\subsection{Wissenschaftliche Reputation}
In diesem Kapitel soll wissenschaftliche Reputation genau definiert und ihre Abhängigkeit zum bestehenden wissenschaftlichen Kommunikationssystem herausgestellt werden. Wissenschaftliche Reputation soll hier als Währung verstanden werden, mittels derer “Status und Ressourcen verteilt werden.”  Wissenschaftliche Reputation verteilt sich auf Einrichtungen und einzelne Personen, die wissenschaftlich tätig sind. Die Evaluation wissenschaftlicher Einrichtungen findet über “Beobachtungen und Gespräche mit den Wissenschaftlern vor Ort” sowie über den “Austausch über die Eindrücke innerhalb der Begehungsgruppe und die gemeinsame Verständigung”  statt. 
Für die Reputation von einzelnen Wissenschaftlern sind Publikationen und die damit einhergehende Verbreitung von wissenschaftlichen Erkenntnissen sehr entscheidend . Doch nicht jede Publikation die gleiche Wertigkeit  und Einfluss auf die Reputation. Die neuen Möglichkeiten der Verbreitung von Informationen lassen deshalb einen vergleichbaren Veränderungsprozess der wissenschaftlichen Verbreitung und damit auch Anerkennung vermuten, die wie Entwicklung des Drucks.  Neben dem Publizieren müssen auch folgende weitere Indikatoren für wissenschaftliche Reputation für wissenschaftliche Institutionen und Personen genannt werden :
\begin{enumerate}
\item Drittmittelprojekte
\item Patente
\item Vorträge
\item Anwendungsrelevanz bzw. Verwertbarkeit
\item Netzwerke
\item öffentliche Aufmerksamkeit sowie politische Relevanz 
\item Renommee der Forschungseinrichtung
\item materielle Ausstattung, Großgeräte etc.
\item personelle Ausstattung
\item Gutachtertätigkeit
\item Herausgeberschaft
\item Funktion
\end{enumerate}
In diesem Rahmen wurden durch den US-amerikanische Soziologe Robert K. Merton vier Grundprinzipien als normative Struktur der Wissenschaft beschrieben , auf deren Grundlage in diesem Kaptitel wissenschaftliche Reputation und das damit einhergehende Anreizsystem, sowie deren Veränderungsprozess durch die Öffnung von Wissenschaft erläutert werden soll. 
