\subsubsection{Chronologie der Bewegung}
Um Open Access verstehen und einordnen zu können ist eine historische Betrachtung der Entwicklung von wissenschaftlicher Kommunikation, aber auch von der Forderung nach Offenheit in der wissenschaftlichen Kommunikation unabdingbar. 

Die Debatte über die Zukunft des wissenschaftliche Publizierens und Kommunizierens neigt dazu, Konzepte um offene Wissenschaft als einen bisher beispiellosen und noch nie dagewesenen Paradigmenwechsel dar zu stellen \cite{cite:17a} \cite{cite:17b}. Allerdings wurden schon im antiken Griechenland, und in vielen anderen pre-modernen Zivilisationen, Wissen und Informationen als nicht besitzbare Ware angesehen\cite{cite:18} dennoch war der Austausch im Vergleich zu den heutigen Möglichkeiten heute stark beschränkt \cite{cite:17c}.

Die Geschichte von Open Access ist also auch eine Geschichte, die eng mit der Digitalisierung von Kommunikationsprozessen verknüpft ist \cite{albert_2006_open_implications}. Open Access ist dabei kein Selbstzweck\cite{cite:17d}, sondern ein Symptom für tiefergehende Prozesse die mit der wachsenden Bedeutung der Digitalisierung in unserer Zivilisation sowie die damit einhergehenden Möglichkeiten für tiefgreifende Veränderungen im Machtgefüge zusammenhängen\cite{cite:17e}. Denn obwohl es vorher schon vereinzelte Versuche in der Wissenschaft gab komplett Informationen und Publikationen offen und frei zu kommunizieren war Open Access im Printzeitalter physisch und ökonomisch über lokale Grenzen hinaus schier unmöglich \cite{cite:18a}. Trotz dieser Grenzen gehen die ersten Experimente mit offenem Zugang und freien Lizenzen für Publikationen in der Wissenschaft bis in die 60er Jahre des vorherigen Jahrhunderts und somit schon vor der Zeit der Erfindung des Internets, zurück \cite{cite:18b}. 

In Deutschland nahmen bis Anfang der 1990er Jahre die wissenschaftlichen Verlage in Deutschland eine marktbeherrschende Stellung ein und waren exklusiver Dienstleister bei der Veröffentlichung wissenschaftlicher Informationen \cite{schloegl_2005} \cite{offhaus_2012_institutionelle_repos}. Die Entwicklung basiert auf dem in der Welt des geistigen Eigentums ungewöhnlichen Umstand, dass seit dem Beginn des wissenschaftlichen Journals im Jahr 1665, wissenschaftliche Autoren nicht finanzielle Belohnung profitierten sondern durch die weite Verbreitung und Hinweise auf ihre Arbeit sowie die Infromationen dahinter \cite{albert_2006_open_implications}.

Die Vormachtstellung der Verlagen im wissenschaftlichen Publikationssystem ist dabei strukturell auf drei Säulen begründet \cite{offhaus_2012_institutionelle_repos} \cite{bargheer_2006_open}: 
\begin{enumerate}
\item "Urheberrecht, wonach Verlage [...] weitgehende Ansprüche an dem veröffentlichten Werk erwerben“;
\item "redaktionelle Themenbündelung (bundling)“;
\item "Qualitätssicherung durch Begutachtung (Peer Review)"
\end{enumerate}

Infolgedessen kam es kurz vor der Jahrtausendwende zur sogenannten Zeitschriftenkrise \cite{suchen}, in der die Velage die Situation nutzten um teilweise drastischen Preiserhöhungen durchzusetzen. Gleichzeitig standen und stehen die Wissenschaftler unter einem starken Publikationszwang, der mit "Publish or Perish" \cite{CLAPHAM_2005} beziehungsweise "impact factor fever" \cite{Cherubini_2008} und "impact factor race"\cite{Brischoux_2009} beschrieben wurde \cite{offhaus_2012_institutionelle_repos}. Bereits Anfang der 1990er gründete der Physiker Paul Ginsparg mit arXiv den ersten wissenschaftliche Preprint-Dienst des Internets, dass es Wissenschaftlern ermöglichen sollte Ideen vor der gedrukten Veröffentlichung zu teilen} und Steven Harnad forderte 1994 Wissenschaftler dazu auf sofort mit der digitalen Selbstarchivierung und öffentlichen Zurverfügungstellung ihrer Beiträge zu beginnen \cite{albert_2006_open_implications, um "den Barrieren, die zwischen ihrer Arbeit und ihren (kleinen) Leserschaft aufgestellt werden, zu entkommen" \cite{harnad_1995_subversive_proposal}. 1997 wurde mit SPARC

Spätestens jedoch im Jahr 2001 erschien Open Access als eigenes und öffentlichkeitswirksames Thema im wissenschaftlichen Diskurs \cite{cite:19}. Die Public Library of Science (PLoS), gegründet im Oktober 2000, foderte in einem offenen Brief im Mai 2001 \cite{cite:20} dazu auf, ab September 2001 nur noch in den Zeitschriften zu veröffentlichen beziehungsweise nur noch die Zeitschriften zu reviewen, zu editieren und zu abonnieren deren Beiträge spätestens sechs Monate nach ihrer Erstveröffentlichung für jedermann im Internet kostenlos und unentgeltlich einsehbar sind. Schon nach kurzer Zeit unterzeichneten (nach eigenen Angaben \cite{cite:19a}) rund 38.000 Wissenschaftler aus 180 Nationen das Schreiben. Diese Brief kann als Auftakt zu einem 20-monatigen theoretischen Schub gesehen werden, der in drei der bis heute wichtigsten Erklärungen im Bereich der Öffnung des Zugangs zu wissenschaftlicher Kommunikation gesehen werden: Die Erklärung der Budapest Open Access Initiative, die Berliner Erklärung und die Bethesda Erklärung. \cite{CREATe_2014}

