\subsection{Chronologie der Bewegung}
Um Open Access verstehen und einordnen zu können ist eine historische Betrachtung der Entwicklung von wissenschaftlicher Kommunikation, aber auch von der Forderung nach Offenheit in der wissenschaftlichen Kommunikation unabdingbar. 

Die Debatte über die Zukunft des wissenschaftliche Publizierens und Kommunizierens neigt dazu, Konzepte um offene Wissenschaft als einen bisher beispiellosen und noch nie dagewesenen Paradigmenwechsel dar zu stellen \cite{cite:17a} \cite{cite:17b}. Allerdings wurden schon im antiken Griechenland, und in vielen anderen pre-modernen Zivilisationen, Wissen und Informationen als nicht besitzbare Ware angesehen\cite{cite:18} dennoch war der Austausch im Vergleich zu den heutigen Möglichkeiten heute stark beschränkt \cite{cite:17c}.

Die Geschichte von Open Access ist also auch eine Geschichte, die eng mit der Digitalisierung von Kommunikationsprozessen verknüpft ist \cite{albert_2006_open_implications}. Open Access ist dabei kein Selbstzweck\cite{cite:17d}, sondern ein Symptom für tiefergehende Prozesse die mit der wachsenden Bedeutung der Digitalisierung in unserer Zivilisation sowie die damit einhergehenden Möglichkeiten für tiefgreifende Veränderungen im Machtgefüge zusammenhängen\cite{cite:17e}. Denn obwohl es vorher schon vereinzelte Versuche in der Wissenschaft gab komplett Informationen und Publikationen offen und frei zu kommunizieren war Open Access im Printzeitalter physisch und ökonomisch über lokale Grenzen hinaus schier unmöglich \cite{cite:18a}. Trotz dieser Grenzen gehen die ersten Experimente mit offenem Zugang und freien Lizenzen für Publikationen in der Wissenschaft bis in die 60er Jahre des vorherigen Jahrhunderts und somit schon vor der Zeit der Erfindung des Internets, zurück \cite{cite:18b}. 

In Deutschland nahmen bis Anfang der 1990er Jahre die wissenschaftlichen Verlage in Deutschland eine marktbeherrschende Stellung ein und waren exklusiver Dienstleister bei der Veröffentlichung wissenschaftlicher Informationen \cite{schloegl_2005} \cite{offhaus_2012_institutionelle_repos}. Die Entwicklung basiert auf dem in der Welt des geistigen Eigentums ungewöhnlichen Umstand, dass seit dem Beginn des wissenschaftlichen Journals im Jahr 1665, wissenschaftliche Autoren nicht finanzielle Belohnung profitierten sondern durch die weite Verbreitung und Hinweise auf ihre Arbeit sowie die Infromationen dahinter \cite{albert_2006_open_implications}. Darüber hinaus beruht das System auf der Eigenheit, dass Wissenschaftler sowohl Produzenten als auch Konsumenten der Wissenschaftskommunikation sind und damit Ihre eigene Zielgruppe darstellen\cite{Hess_2006}.

Die Vormachtstellung der Verlagen im wissenschaftlichen Publikationssystem ist dabei strukturell auf drei Säulen begründet \cite{offhaus_2012_institutionelle_repos} \cite{bargheer_2006_open}: 
\begin{enumerate}
\item "Urheberrecht, wonach Verlage [...] weitgehende Ansprüche an dem veröffentlichten Werk erwerben“;
\item "redaktionelle Themenbündelung (bundling)“;
\item "Qualitätssicherung durch Begutachtung (Peer Review)"
\end{enumerate}

Infolgedessen kam es kurz vor der Jahrtausendwende zur sogenannten Zeitschriftenkrise \cite{suchen}, in der die Velage die Situation nutzten um teilweise drastischen Preiserhöhungen durchzusetzen. Gleichzeitig standen und stehen die Wissenschaftler unter einem starken Publikationszwang, der mit "Publish or Perish" \cite{CLAPHAM_2005} beziehungsweise "impact factor fever" \cite{Cherubini_2008} und "impact factor race" \cite{Brischoux_2009} beschrieben wurde \cite{offhaus_2012_institutionelle_repos}. Bereits Anfang der 1990er gründete der Physiker Paul Ginsparg mit arXiv den ersten wissenschaftliche Preprint-Dienst des Internets, dass es Wissenschaftlern ermöglichen sollte Ideen vor der gedrukten Veröffentlichung zu teilen und Steven Harnad forderte 1994 Wissenschaftler dazu auf sofort mit der digitalen Selbstarchivierung und öffentlichen Zurverfügungstellung ihrer Beiträge zu beginnen \cite{albert_2006_open_implications}, um "den Barrieren, die zwischen ihrer Arbeit und ihren (kleinen) Leserschaft aufgestellt werden, zu entkommen" \cite{harnad_1995_subversive_proposal}. 1998 wurde mit Scholarly Publishing and Academic Resources Coalition (SPARC) einer der späteren "major player" der OA Bewegung\cite{russell2008business} gegründet. Als Konsequenz aus der Zeitschriftenkrise sollte diese Allianz zwischen Universitäten und wissenschaftlichen Bibliotheken, dafür sorgen, dass die Kosten für wissenschaftliche Zeitschriften reduziert oder durch die Bereitstellung kostengünstiger oder freier, nicht-kommerzieller, Peer-Review-Fachzeitschriften ersetzt werden. Durch Weiterbildung, politische Arbeit und der Förderung von alternativen Geschäftsmodellen sollte die Initiativen des offenen wissenschaftlichen Publizierens stimuliert werden\cite{suchen}.

Spätestens jedoch im Jahr 2001 erschien Open Access als eigenes und öffentlichkeitswirksames Thema im wissenschaftlichen Diskurs \cite{cite:19}. Die Public Library of Science (PLoS), gegründet im Oktober 2000, foderte in einem offenen Brief im Mai 2001 \cite{cite:20} dazu auf, ab September 2001 nur noch in den Zeitschriften zu veröffentlichen beziehungsweise nur noch die Zeitschriften zu reviewen, zu editieren und zu abonnieren deren Beiträge spätestens sechs Monate nach ihrer Erstveröffentlichung für jedermann im Internet kostenlos und unentgeltlich einsehbar sind. Schon nach kurzer Zeit unterzeichneten (nach eigenen Angaben \cite{cite:19a}) rund 38.000 Wissenschaftler aus 180 Nationen das Schreiben. Diese Brief kann als Auftakt zu einem 20-monatigen theoretischen Schub gesehen werden, der in drei der bis heute wichtigsten Erklärungen im Bereich der Öffnung des Zugangs zu wissenschaftlicher Kommunikation gesehen werden: Die Erklärung der Budapest Open Access Initiative, die Berliner Erklärung und die Bethesda Erklärung. \cite{CREATe_2014}

Im gleichen Jahr wie der PLoS-Brief, wurden mit der “Budapest Open Access Initative” (BOAI)\cite{boai_2012} erstmals die Bemühungen um Open Access in einer eigenen Erklärung zusammengefasst\cite{cite:21a}, die Rahmen einer Konferenz des Open Society Institutes in Budapest entstand. In dem Zentrum stand die Forderung nach dem freien Zugang zu wissenschaftlichen Publikationen. Sie manifestiert erstmals, dass wissenschaftliche Peer-Review-Fachliteratur “kostenfrei und öffentlich im Internet zugänglich sein sollte, so dass Interessenten die Volltexte lesen, herunterladen, kopieren, verteilen, drucken, in ihnen suchen, auf sie verweisen und sie auch sonst auf jede denkbare legale Weise benutzen können, ohne finanzielle, gesetzliche oder technische Barrieren jenseits von denen, die mit dem Internet-Zugang selbst verbunden sind. In allen Fragen des Wiederabdrucks und der Verteilung und in allen Fragen des Copyrights überhaupt sollte die einzige Einschränkung darin bestehen, den Autoren Kontrolle über ihre Arbeit zu belassen und deren Recht zu sichern, dass ihre Arbeit angemessen anerkannt und zitiert wird."\cite{boai_2012} 

Zum zehnten Jahrestages der BOAI, wurde von der Open Society Foundation mit der BOAI 10 die usrprüngliche Erklärung bestärkt und anhand von weitere Richtlinien und Empfehlungen die Entwicklungen und Herausforderungen in seiner zehnjährigem Bestehen addressiert. Die Autoren kommen auch zu dem Schluss, dass  "noch immer Zugangsbeschränkungen zu Peer-Review-Forschungsliteratur, meist eher zugunsten der Verlage als zugunsten der Autoren, Reviewer oder Redakteure und damit auch auf Kosten der Forschung, Forscher und Forschungseinrichtungen" bestehen und "nichts aus den letzten zehn Jahren darauf schließen, dass das ursprüngliche Ziel von OA weniger sinnvoll oder erstrebenswert erscheint. Im Gegenteil, die Notwendigkeit, dass Wissen für jeden, der es nutzen, anwenden oder darauf aufbauen kann, offen verfügbar sein sollte, ist dringlicher als je zuvor".\cite{boai_2012}

Nicht mal zwei Jahre später, im Juni 2003, verabschiedeten im US-Bundesstaat Maryland eine Gruppe von Forschungsförderer, wissenschaftlichen Gesellschaften, Verleger, Bibliothekare, Forschungseinrichtungen und einzelnen Wissenschaftlern das "Bethesda Statement on Open Access Publishing".\cite{suchen} Ziel der Erklärung war die Stimulation der Diskussion in der biomedizinischen Forschung, "wie man schnellstmöglich den offenen Zugang zu der primären wissenschaftlichen Literatur in der Biomedizin erreichen könnte"\cite{suchen}. Wie bereits in der Erklärung der BOAI erklärten die Autoren des "Bethesda Statement on Open Access Publishing" Bedingungen an diese Art des offenen Zugangs\cite{suchen}:
\begin{enumerate}
\item Erstens müssen Autor(en) und Copyright-Inhaber für alle Benutzer eine freies, unwiderrufliches, weltweites, unbefristetes Recht auf den Zugang zu und verwenden eine Lizenz die das Kopieren, Nutzen, Verbreiten, Übertragen und öffentliche Anzeigen der Publikation ermöglichen. Darüber hinaus muss es erlaubt sein abgeleitete Werke zu verteilen, in jedem digitalen Medium für jeden Zweck zu veröffentlichen, vorbehaltlich einer angemessenen Zuordnung der Urheberschaft. Das beinhaltet auch das das Recht auf eine kleine Anzahl von gedruckten Kopien für den persönlichen Gebrauch zu machen. 
\item Zweitens, muss eine vollständige Version der Arbeit und allen ergänzenden Materialien, einschließlich einer Kopie der Genehmigung, wie oben erwähnt, in einem geeigneten elektronischen Standardformat sofort bei der ersten Veröffentlichung in mindestens einem Online-Repositorium, das von einer wissenschaftlichen Einrichtung unterstützt wird hinterlegt werden. Dieses Repositorium muss von einer wissenschaftlichen Gesellschaft, Regierungsbehörde oder einer anderen etablierten Organisation akzeptiert und die sich zu einen offenen Zugang, uneingeschränkte Verbreitung sowie Interoperabilität und Langzeitarchivierung (für die biomedizinischen Wissenschaften, PubMed Central ist ein solches Repository) einsetzen.
\end{enumerate}

Ein weiterer Meilenstein für die Verbreitung von Open Access auf dem europäischen Kontinent waren die "Berlin Konferenzen"\cite{CREATe_2014}. Die erste Tagung wurde 2003 von der Max-Planck-Gesellschaft und dem Projekt European Cultural Heritage Online (ECHO) organisiert, um über "Zugangsmöglichkeiten zu Forschungsergebnissen" zu diskutieren. In diesem Rahmen entstand auch die "Berliner Erklärung über den offenen Zugang zu wissenschaftlichem Wissen"\cite{berliner_erklaerung_2003}, in der die Verfasser über die Budapester Erklärung hinaus gehen und neben dem kostenlosen und freien Zugang zu wissenschaftlichem Wissen in Form von Publikationen auch den freien und offenen Zugang zu den Daten fordern: „Open Access-Veröffentlichungen umfassen originäre wissenschaftliche Forschungsergebnisse ebenso wie Ursprungsdaten, Metadaten, Quellenmaterial, digitale Darstellungen von Bild- und Graphik-Material und wissenschaftliches Material in multimedialer Form.“\cite{berliner_erklaerung_2003} Sie symbolisiert damit auch ein erweitertes Verständnis von Open Access und bildet die Grundlage für ein erstes Ansatzpunkt zur Definition von Open Science, konzentriert sich aber dennoch ausschließlich auf den abgeschlossenen wissenschaftlichen Prozess\cite{suchen}.

Alle drei Erklärungen, auch die "three B's"\cite{suber_2004_praising_oa}genannt, gelten als die angesehensten Definitionen von Open Access und sind in wesentlichen Merkmalen in sich stimmig\cite{albert_2006_open_implications}.