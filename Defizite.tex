\subsection{Defizite}
Viele der Erklärungsansätze für den Paradigmenwechsel hin zur Öffnung der Wissenschaft basieren auf Annahmen, in denen ein direkter Zusammenhang von technischen Entwicklungen unmittelbar auf (wissenschafts-)politische und kulturelle Bewegungen geschlossen werden. Darüber hinaus beschränkt sich die Perspektive primär auf den Zugang zum Ergebnis von Wissenschaft und weniger auf die Öffnung des gesamten Prozesses. Die theoretische Auseinandersetzung mit der Geschlossenheit des wissenschaftlichen Diskurses  auf der Einen und mit den Treibern und Bremsern im realen wissenschaftlichen Prozess werden in der gängigen Literatur auf der anderen Seite, wird nicht genügend berücksichtigt. Hier wird vor allem die Verbindung zwischen wissenschaftlicher Reputation und Geschlossenheit des Wissensproduktionsprozesses nur selten erörtert.