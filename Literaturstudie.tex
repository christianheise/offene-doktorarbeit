\chapter{Chronologie, Entwicklungen, Definitionen und Debatten um Open Access und Open Science}

Wie in den vorangegangenen Kapiteln deutlich wurde, sind die Entwicklungen, Definitionen und Debatten um die Öffnung von Kommunikation in Wissenschaft und Forschung in der gängigen Literatur weder einheitlich dargestellt und abgegrenzt, noch unumstritten \cite{muller_2010_open} \cite{schulze_2013_open}. Von hervorzuhebendem Interesse sind im Rahmen der Analyse die Entwicklungen wissenschaftlicher Kommunikation, die Herausforderungen im aktuellen Kommunikationssystem, die Debatte zur Öffnung von Wissenschaft und die Treiber und Bremser dieser Entwicklung und dem damit einhergehenden Wandel mit Fokus auf den Bereich wissenschaftliche Reputation.

Im Folgenden werden wesentliche Anknüpfungspunkte an der Open-Access und Open-Science-Bewegung in Wissenschaft und Forschung dargestellt und erläutert. Die Auswahl der berücksichtigten Werke bezieht sich auf die für die Fragestellungen relevanten Beiträge und wird um die Betrachtung der Debatten von Open Access und Open Science ergänzt. Dabei gibt es nicht die eine Debatte um die Öffnung von wissenschaftlicher Kommunikation, es sind viel mehr eine Vielzahl um Auseinandersetzungen bezüglich unterschiedlicher Bedenken und Interessen von einer fluktuierenden Gruppe von Akteuren \cite{Beals_2013} mit nicht selten polemisch geführte Diskussionen \cite{Lossau_oa_2007}.

Wie in den Grundlagen dargestellt, befindet sich das wissenschaftliche Kommunikationssystem in der Krise. Seitdem wesentliche Bestandteile des wissenschaftlichen Kommunikationsprozess privatisiert wurden, haben sich akademischen Ziele und die Marktinteressen der Verlage immer weiter voneinander entfernt. Zum Beispiel sind im Zeitraum von 1986 bis 2012 die Ausgaben für Bibliotheksbestände in den Vereinigten Staaten um 322 Prozent gestiegen \cite{lewis_2015_future}. Das derzeitige Geschäftsmodell von Verlagen ermöglicht den Verlegern Betriebsgewinnmargen von über 35 Prozent \cite{russell_2008_business} \cite{cope2014future} und hohe jährliche Wachstumsraten \cite{Martin_2013} \cite{Wellcome_Trust_2003}. Die drei größten Wissenschaftsverlage vereinen bereits 42 Prozent aller Journale und trotz der internationalen Finanzkrise stiegen die Umsätze ungebremst. In den Jahren zwischen 2008 und 2011 stiegen die Umsätze um 11,7 Prozent und die Gewinne von 1,6 Milliarden auf 1,9 Milliarden Dollar (17 Prozent) \cite{cope2014future}. Das entspricht einer Umsatzrendite von 35,8 Prozent. Zum Vergleich, die durchschnittliche Umsatzrendite im Wirtschaftszweig "Verlagswesen" bei deutschen Firmen mit mehr als 50 Mitarbeitern lag laut der Bundesbank dem im Jahr 2011 bei 11,6 Prozent \cite{bundesbank_2014}. Es sind nur wenig Beispiele bekannt, "in denen das symbolische Kapital in so außerordentlichem Maße zu ökonomischen Kapital verdinglicht worden ist", wie bei dem Geschäftsmodell des wissenschaftlichen Publizierens \cite{hagner_2015_sache_buches}.

Als Ausweg aus dieser Krise wird immer wieder die Digitalisierung und Öffnung der wissenschaftlichen Kommunikation durch Konzepte wie Open Access und Open Science genannt \cite{lewis_2015_future}. Beide Begriffe umfassen eine Vielzahl von Annahmen über die Zukunft der Wissensbildung und Wissensverbreitung. Sie funktionieren als Sammelbegriffe für unterschiedliche Aufassungen wie und wie weit Wissenschaft offener werden kann. Sie sind Bestandteil eines lebendige Diskurses in der wissenschaftlichen Gemeinschaft \cite{schulze_2013_open}. Kleinster gemeinsamer Nenner in dem Diskurs um die unterschiedlichen Konzepte ist, "dass wissenschaftliche Forschung sich irgendwie mehr öffnen muss" \cite{cite:9}.

Ziel der Literaturanalyse ist es existierenden Erkenntnisse über die Begrifflichkeiten und deren Entwicklung darzulegen sowie aufzuzeigen in welchen Bereichen weitere Forschung angestrebt werden sollte \cite{webster2002analyzing}. Für die Analyse wurden eine Vielzahl von Quellen mit thematischen Bezug zur Öffnung von Wissenschaft und Forschung ausgewählt und analysiert. Ziel dieses Kapitels ist es, die Debatten rund um die Begriffen und Forschungsfragen darzustellen um zu einer ausgewogenen Basis für die Betrachtung der Begrifflichkeiten und zur Beantwortung der vorab definierten Forschungsfragen zu gelangen. Dafür werden zu allererst die chronologische Entwicklung wissenschaftlicher Kommunikation sowie Herausforderungen im bestehenden System wissenschaftlicher Kommunikation dargestellt. In einem weiteren Schritt werden Anknüpfungspunkte zur Forderung nach Öffnung der wissenschaftlichen Kommunikation identifiziert und abschließend die Ergebnisse zusammengefasst und Anknüpfungspunkten für die empirische Untersuchung abgeleitet.

Folgenden detailierte Fragestellungen sollen mit Hilfe dieser Literaturstudie beantwortet werden:
\begin{itemize}
\item Welche historischen Entwicklungen haben die Entwicklung der wissenschaftlichen Kommunikation und der Forderung nach Öffnung?
\item Welche Anknüpfungspunkte ergeben sich daraus für Forderung nach Öffnung beeinflusst?
\item Welche Argumente spielen in der Debatte für und gegen die Öffnung wissenschaftlicher Kommunikation im Rahmen der Digitalisierung eine Rolle?
\item Welche Indikatoren für Reputationsverteilung im wissenschaftlichen Kommunikationsystem werden in der Literatur genannt?
\item Welche Herausforderungen bestehen im System wissenschaftlicher Kommunikation?
\item Welche Ableitungen können aus der Studie für die empirische Untersuchung gemacht werden?
\end{itemize}

Für die Auswertung der Literatur werden aus einem Korpus von ausgewählten Texten die Entwicklungen, Definitionen und Debatten rund um den Themenkomplex der Öffnung von Wissenschaft und Forschung extrahiert und zusammengefasst. Ziel ist es, weitere wissenschaftliche Fragestellungen für die Befragung von publizierender Wissenschaftler und Wissenschaftlerinnen verschiedener Fachbereiche zu entwickeln. Abschließend werden in diesem Kapitel aus den Texten die Treiber und Bremser für die Öffnung von Wissenschaft identifiziert, für die Befragung extrapoliert und in der Gesamtbetrachtung der Arbeit zusammengeführt und strukturiert ausgewertet.

\section{Chronologie der Entwicklung wissenschaftlichen Kommunikation}

Für das erweitertes Verständnis der Prozesse, die zu der Öffnung von Wissenschaft und Forschung führen, sowie für die Darstellung der Beziehung neuer digitaler Kommunikationsysteme zu ihren analogen Vorläufern, ist eine historische Betrachtung der Entwicklung wissenschaftlicher Kommunikation sowie der Forderung nach Offenheit in Wissenschaft und Forschung unabdingbar. Diese stellt zum einen die Grundlagen für die Analyse von Offenheit dar und ebnet die weitere Grundlage für die Dartstellung  des "Forschungsgegenstands" \cite{cite:10}. Diese historische Darstellung bietet einen ersten Ansatzpunkt für die Erforschung der Definitionen und Debatten um Open Access und Science \cite{Scheliga_2014}, da diese historischen Übergange bisher noch immer nur unzureichend dargestellt wurden \cite{CREATe_2014}.

Angelehnt an die Arbeiten des kanadischen Philosophen McLuhan und des Germanisten Wenzel können dabei drei bedeutende Umbrüche der Medienentwicklung im Rahmen der Kommunikation von Wissen genannt werden \cite{wunderlich_2008_buchdruck} \cite{wenzel_mediengeschichte_2007}:
\begin{enumerate}
\item der Übergang vom Körpergedächtnis (brain memory) zum Schriftgedächtnis (script memory)
\item der Übergang von der Handschriftenkultur zur Druckkultur (print memory)
\item und der Übergang vom Buch zum Bildschirm (electronic memory)
\end{enumerate}

---- TODD: weiter Ausarbeiten ----

\subsection{Wissenschaft und wissenschaftliche Kommunikation in pre-modernen Zivilisationen}

Die pre-modernen Zivilisationen bezog sich "Wissenschaft" unmittelbar auf die täglichen Bedürfnissen und das Wissen und Informationen sind als nicht besitzbare Ware angesehen worden\cite{cite:18} \cite{steiner_1998_autorenhonorar}. Im Vergleich zu den heutigen Möglichkeiten war in den vormodernen Zivilisationen der Wissensaustausch stark beschränkt \cite{cite:17c} und es gab keine "scharfe Grenze zwischen dem vorhandenen und dem aktuell benutzten Wissen" \cite{Luhmann1998}. Die Produktion von Literatur beschränkte sich in den vorwissenschaftlichen Gesellschaften vornehmlich auf "auf die Überlieferung und Kommentierung des althergebrachten Wissens, insbesondere des theologischen" Wissens \cite{steiner_1998_autorenhonorar}. Was die Gelehrte "zu sagen und zu schreiben hatten, war nicht als Beitrag zum Fortschritt von Wissenschaft als einem kollektiven Unternehmen zu verstehen, sondern eher als Dokumentation ihrer persönlichen Erkenntnisfortschritte" \cite{graefen2007_wissenschaftliche_artikel}, somit stellten sich vor allem der Aufgabe "das Wissen zu verbessern und vor allem zu erhalten und zu tradieren" \cite{Luhmann1998}. "Eine Textart, die dem wissenschaftlichen Artikel entspricht oder mit ihm vergleichbar ist, existierte im Mittelalter nicht. Noch im 15. und 16. Jahrhundert sind nur wenige Texte "fachinterner Kommunikation" also schriftlicher Kommunikation unter Vertretern eines Faches über fachliche Inhalte, nachgewiesen" \cite{graefen2007_wissenschaftliche_artikel}. Texte die wir heute als wissenschaftlich bezeichnen würden, wurden im Mittelalter nur dann akzeptiert, wenn sie den Namen eines (anderen) Autors trugen \cite{foucault_2000_autor}.

Die Sprachwissenschaftlerin Graefen hat exemplarisch die Entwicklung hin zum wissenschaftlichen Text wie folgt zusammengefasst: "Erst wenn ein gesamtgesellschaftlicher Bedarf an Wissen und an ständiger Wissenserweiterung allgemein erkennbar wird und entsprechende Leistungen von Individuen auch persönliche Vorteile versprechen, findet eine Umorientierung von sporadischer individueller wissenschaftlicher Betätigung hin zu gesellschaftlich anerkannter und zur Kenntnis genommener, kollektiv bzw. arbeitsteilig betriebener Wissenschaft statt" \cite{graefen2007_wissenschaftliche_artikel}.

\subsection{Einführung des Buchdrucks als Grundlagen der modernen Wissenschaft}

Die Geschichte des gedruckten Buchs beginnt maßgeblich mit Johannes Gensfleischs, auch Gutenberg genannt, Beiträgen zur Buchdruckerkunst \cite{wittmann_1999_geschichte} Mitte des 15. Jahrhunderts \cite{suchen}. Gutenberg führte um 1460 die Druckerpresse ein, "die er von den Weinpressen der rheinischen Winzer abgeschaut und dann verbessert haben dürfte" \cite{stober_2014_pressegeschichte}. Die Einführung des Buchdrucks führte nicht nur zu neuen Möglichkeiten der Kommunikation, sondern zu einer Veränderung der generellen Aufgabe von Wissenschaft, insbesondere ihrer der Orientierung auf den täglichen Bedarf \cite{Luhmann1998}. Durch die neue Möglichkeiten der Vervielfältigung und Massenverbreitung hat das Selbstverständnis der europäischen Kultur in bis dahin unbekannter \cite{giesecke_1991_buchdruck} und revolutionärer Weise verändert \cite{wunderlich_2008_buchdruck} \cite{stober_2014_pressegeschichte}. Der Buchdruck stellte die "Grundlagen und Meilensteine sowohl für die Kommunikation der Menschheit insgesamt als auch für den wissenschaftlichen Gedankenaustausch im Besonderen dar" \cite{schirmbacher_2009_wisspub}, er war ein "Bestandteil des Übergangs vom Mittelalter in die frühe Neuzeit" \cite{lange2008medienwettbewerb} und die Druckerpresse, nahm die "entscheidende Schwelle für das Entstehen der neuzeitlichen Wissenschaften" \cite{luhmann_1997_gesellschaft}.

Diese neue Technologie führte zu einem bis dahin unbekannten, explodierenden Informationsangebot. Infolgedessen sich eine neue Denkstruktur entwickelte \cite{eisenstein_1997_druckerpresse}, bei der das "mittelalterliche Denken in Bildern und Metaphern" von der "wissenschaftlich-systematischen Methodik" abgelöst wurde \cite{wunderlich_2008_buchdruck}. Sie führte zur Befreiung des Autors aus der weitgehenden Anonymität mittelalterlicher Manuskriptkultur und zur Entkopplung der "Herstellung und Verbreitung vom singulären Interesse eines Autors, Kopisten oder Auftraggebers"\cite{wunderlich_2008_buchdruck} \cite{schirmbacher_2009_wisspub}.

Mit der Entwicklung der Buchdrucktechnologie folgte im 16. Jahrhundert die Verbreitung eines "freien Marktes als Vertriebsnetz für typographische Informationen"\cite{giesecke_1991_buchdruck} und die "Kapitalisierung der Buchproduktion" \cite{steiner_1998_autorenhonorar}. Das gedruckte Wort führte somit zu einem "Verlust an Macht und Herrschaft über das geschriebene Wort" \cite{wunderlich_2008_buchdruck}. Anfangs handelte es sich bei der Technologie nur um ein "elitäres und teures Medium für die gebildete Klasse" \cite{hartmann_2008_medien}, Bücher waren "Luxusgegenstände" und die Gewinnspannen der Buchdrucker und -händler waren "enorm" \cite{stober_2014_pressegeschichte}. Sie führte weder von Beginn an zum zeitlich unmittelbaren Zugang zu Wissen noch war sie sofort für die Allgemeinheit zugänglich \cite{hartmann_2008_medien}. Die wissenschaftliche Elite der damaligen Zeit forderte deshalb, dass Werke ohne Rücksicht auf Profitgier und "Geiz" \cite{luther_1876} erscheinen sollten und appellierte an eine "obrigkeitliche Lenkung", damit der Buchhandel "seiner Aufgabe der Verbreitung von nützlichem Wissen gerecht würde" \cite{wittmann_1999_geschichte}. Gutenbergs Druckinnovation sollte als sogenannte "Schlüsseltechnologie" \cite{jager_1993_theoretische} eine neue Dimension der Informations- und Wissensverbreitung für die Gesamtgesellschaft ermöglichen.

In der Übergangszeit von der primären Kommunikation zwischen den Gelehrten anhand von Briefen und der Verbreitung des Buchdrucks kam es zu einer Vielzahl sogenannter Prioritätsstreits in der Wissenschaft und Forschung \cite{schirmbacher_2009_wisspub}. Denn die meisten wissenschaftliche Erkenntnisse waren zwar im direkten Briefwechsel, aber noch nicht öffentlich verbreitet worden und deshalb konnte zu dieser Zeit selten ein für alle nachvollziehbarer Bezug zum jeweiligen Entdecker hergestellt werden. Als Beispielhaft für einen solchen Prioritätsstreit kann die Auseinandersetzung zwischen Isaac Newton und Gottfried Wilhelm Leibniz um eine Veröffentlichung zur Fluxionsrechnung im 17. Jahrhundert genannt werden. Leibniz rezensierte eine von Newton verfasste Veröffentlichung anonym und stellte sich selbst namentlich als Erfinder dieser dar \cite{2013_leibniz}, ohne auf eine öffentliche Publikation seiner deutlich länger vorhandenen Erkenntnisse hinweisen zu können \cite{schirmbacher_2009_wisspub}. Auf Grund des fehlenden öffentlichen Nachweises wurde Leibniz infolgedessen durch die Royal Society, einer der ersten Gelehrtenvereinigungen, des Plagiats für schuldig befunden und Entdeckung Newton zugesprochen. Doch selbst wenn Newton seine Fluxionsrechnung "früher entwickelt hat, geht die algorithmische Eleganz von Differentialen und Integralen doch auf Leibniz zurück" \cite{kittler_faz_1996}.

Der Buchdruck, wie auch die ersten wissenschaftlichen Zeitungen, wurden für die wissenschaftlichen Autoren somit nicht nur zu einem neuen "Kommunikationsinstrument", einem Instrument zur "Erlangung von Reputation" oder zu einem Instrument "zur Generierung finanzieller Erträge" sondern auch zu einem "Nachweisinstrument" \cite{wunderlich_2008_buchdruck} \cite{schirmbacher_2009_wisspub} für die Vermeidung solcher Prioritätskonflikte. Darüber hinaus "waren gedruckte Meinungen schwerer zu widerrufen oder umzuinterpretieren als nur mündlich geäußerte oder nur wenigen zugängliche (etwa Briefe)" \cite{luhmann_1997_gesellschaft}.

Die Verbreitung des Buchdrucks fand aber nicht ungebremst und nicht ohne umfassende Kritik in der damaligen Gesellschaft statt. Vor allem kirchliche Instanzen waren über eine "wachsende theologische Begriffsverwirrung" und die Verbreitung der Schriften in Volkssprachen besorgt \cite{giesecke_1991_buchdruck}. Sie stellten die größten Gruppe an Kritikern des Buchdrucks dar und versuchten die neue "Bücherflut" zu unterbinden\cite{giesecke_1991_buchdruck}. Zwischenzeitlich führte die Einführung des Buchdrucks zu einer neuen Bedeutung der Zensur, als "prohibitives Instrument für die Überwachung der Lektüren" und als "Kampfmittel" \cite{sprachgeschichte_1998_besch} gegen zu viel Wissen \cite{suchen} und "unerwünschte Literatur" \cite{suchen}. Beispielhaft für diese Art der Zensur, zitiert der Kommunikations- und Medientheoretiker Michael Giesecke aus einem Gutachten dieser Zeit: "In den Anfängen muß man Widerstand (gegen das Übel des Drucks von Büchern, die aus den heiligen Schriften in die Volkssprache übersetzt sind), damit nicht durch die Vermehrung der deutschsprachigen Bücher der Funke des Irrtums endlich sich zu einem großen Feuer entwickle" \cite{giesecke_1991_buchdruck}.

Zusammenfassend nennt Giesecke vor allem folgende grundlegenden Einwände gegen den Buchdruck als unregulierte, "freie" Kunst \cite{giesecke_1991_buchdruck} für die Verbreitung von Wissen und Informationen:
\begin{itemize}
\item Die Einführung des Buchdrucks wurden von vielen Warnungen vor Missbrauch der Technologie begleitet \cite{lange2008medienwettbewerb}. Im Mittelpunkt der Warnungen standen der antireligöser Missbrauch durch die Verbreitung gefährlichen Gedankenguts \cite{kruse2003multimedia}, die bewusste Falschinformation und Verfälschung von Inhalten \cite{sprachgeschichte_1998_besch}, die willkürliche Informationsverbreitung über Bücher, ohne Zustimmung der geistlichen und weltlichen Regenten \cite{rother_2002_siebenbuergen} und die Angst der Traditionalisten, die ihre Herrschaft durch das Monopol auf die Interpretation der Bibel gefährdet sahen \cite{lange2008medienwettbewerb}.
\item Ein weitere Einwand adressierte die Befürchtung, dass die Qualität und Reinhaltung der besten Texterzeugnisse beim Buchdruck nicht sichergestellt werden kann \cite{giesecke_1991_buchdruck}.
\item Auch die Nachlässigkeit und Unachtsamkeit von Buchdruckern und Setzern wurde früh kritisiert. Sie spielten im Buchdruckprozess eine entscheidende Rolle, da sie großen Einfluss auf die Qualität der Nachdrucke hatten. Nachlässigkeit oder ungenaues Arbeiten führten zu erheblichen strukturellen und inhaltlichen Qualitätsverlusten, was von Autoren wie Martin Luther schon früh beklagt wurde \cite{sprachgeschichte_1998_besch} \cite{stober_2014_pressegeschichte} \cite{luther_1876}.
\item Die Multiplikation von Fehlern, da in den gedruckten Exemplaren auch die Fehler völlig übereinstimmen und nicht behoben werden können, schließt an die Kritik der Qualität der gedruckten Bücher an. Die Befürchtung begründete auf der Irreversibilität der Verbreitung fehlerhafter Inhalte beim Buchdruck, die bei der geringeren Anzahl handschriftlichen Kopien bisher weniger Einfluss hatte \cite{kittler_2004}.
\item Die staatlichen und geistigen Obrigkeiten befürchteten durch die Demokratisierung der Vervielfältigung und Verbreitung von Wissen die Verwirrung der "Laien" (der Glaubensgemeinschaft) und damit einen Kontrollverlust für die bestehende gesellschaftliche Ordnung. \cite{giesecke_1991_buchdruck}.
\item Demzufolge befürchtete die Obrigkeit, die Auflösung der ständischen Ordnung da der "Zugang zu den Speichern des Wissens nicht länger bestimmten Schichten vorbehalten bleibt" und das "Schreiben und Lesen wird von einer ständischen zu einer gemeinen Tätigkeit". Aus heutiger Sicht mag diese Sicht auf Grund der sehr geringen Alphabetisierungsrate und der noch immer sehr geringen Anzahl an Büchern Ende des 15. Jahrhunderts als unbegründet erscheinen, dennoch wurde die soziale Umwälzung durch den Buchdruck beschleunigt und unumkehrbar gemacht. \cite{giesecke_1991_buchdruck}
\item Auflösung des "Amts" des Bücherschreibers als eigenes Handwerk
\item Die Angst vor dem Überfluss an Büchern und Wissen stellte einen weiteren Einwand dar. Die Kritiker der Buchdrucktechnologie befürchteten  durch die massenhafte Verbreitung ein Chaos an Informationen \cite{giesecke_1991_buchdruck}.
\item Sogar physische Konsequenzen wurden befürchtet: "Augen schmerzen, vom Lesen, unsere Finger vom Blättern" \cite{giesecke_1991_buchdruck}
\item Auch "psychische Bedenken" wurden eingebracht, so gab es im 15. Jahrhundert bei den Menschen die Angst vor dem Anhäufen von Informationen. Sie galt im Mittelalter als "gefährliches und verwirrendes Unterfangen" und führte zu Annahmen wie "je gelehrter, je verkehrter". \cite{giesecke_1991_buchdruck}
\end{itemize}

Die genannten Einwände fußten alle auf Ängsten oder Befürchtungen vor den Veränderungen der etablierten Machtstrukturen, die die Informationsverbreitung bis Ende des Mittelalters beeinflusst hatten. Vor der Einführung des Buchdrucks wurde vorab entschieden, was veröffentlicht und verbreitet wurde und es gab klare Instanzen, die die Weitergabe von Wissen (meist Auftragsarbeiten) organisierten. Der Buchdruck kehrte dieses System um, da nun Texte erstmals verbreitet wurden und man es dem "Markt und dem nachträglichen Meinungsstreit überließ, welche Information zum Gemeingut wurden" \cite{giesecke_1991_buchdruck}. Niklas Luhmann fasste diese Veränderung hin zu einem Prozess wie folgt zusammen: "Wer für den Druck schreibt, gibt die Situationskontrolle auf" und "produziert für das Gedächtnis des Systems" bei dem weder "Kommunikationsvorgang" noch der "Wissenszuwachs" abgeschlossen sind \cite{Luhmann1998}.

Die Etablierung und schnelle Verbreitung \cite{stober_2014_pressegeschichte} des Drucks führte, zunächst "unbemerkt und naturwüchsig", zu einer Veränderung der Sozialisierung von Informationen, der Veröffentlichung \cite{giesecke_1991_buchdruck}. Das Medium der Schrift wurde demnach unter den Buchdruckbedingungen als eine "Verbreitungstechnologie" für Informationen genutzt, die zwar die unmittelbare Interaktion zwischen Sender und Empfänger (weiterhin) ausschloss, aber mittelbar nur mit Hilfe von Empfängern zu Wissen werden konnte \cite{Luhmann1998}.

Die Einführung des Buchdrucks stellte somit einen Bestandteil des "Übergangs vom Mittelalter in die frühe Neuzeit dar"\cite{lange2008medienwettbewerb}, da zwischen Buchdruck und demokratischen Freiheiten "sowohl faktisch als auch ideologisch" \cite{suchen} ein Zusammenhang hergestellt werden kann. Dieser Zusammenhang wird darin deutlich, dass im Gegensatz zum Mittelalter, in dem jede breitere Sozialisierung und Verbreitung privater Gedanken "legitimationsbedürftig" war, nun jeder Eingriff in die "Freiheit, Meinungen oder Informationen" zu drucken einer politischen Legitimation \cite{giesecke_1991_buchdruck}. Der Buchdruck kann demnach im Rahmen der "fundamentalen Umbrüche in Politik und Verwaltung, Ökonomie und Handel, Religion, Bildung und nicht zuletzt in den Prozessen der kognitiven Welterkenntnis" \cite{pscheida_2010_wikipedia} als "Katalysator des kulturellen Wandels"\cite{giesecke_1991_buchdruck} verstanden werden.

Um den Arbeitsaufwand der Drucker zu honorieren und die verlegerische Leistung zu würdigen\cite{szilagyi_2011_leistungsschutzrecht}, wurden mit der Entstehung des Druckerwesens auch erste Privilegien vergeben \cite{gieseke_1995_privileg}, die es den Druckern erlaubte, die Buchdruckkunst für einen bestimmten Zeitraum allein oder in einem bestimmten Gebiet auszuüben \cite{martin2008publizistische} \cite{koller_1995_Urheberrecht}. Diese Privilegien ermöglichen dem Begünstigen Sonderberechtigungen oder Rechte gegenüber den allgemeinen Rechtsregeln \cite{jaenich_2002_geistiges}. Im Zuge der Verbreitung der Drucktechnologie und des steigenden Wettbewerbs kam es auch zu den ersten Privilegien für Urheber, die bereits im 15. Jahrhundert damit begannen ihre Manuskripte zu verkaufen \cite{hesse2002rise}, und Erstverleger, die sich damit versuchten gegen das Nachdrucken und Raubdrucken zu erwehren. Die erfolgreiche Einforderung dieser Privilegien führte schon früh zu einer Art Monopolstellung bestimmter Druckereien und zu einem generellen Nachdruckverbot für bestimmte Werke in einem bestimmten Gebiet oder für einen bestimmten Zeitraum \cite{szilagyi_2011_leistungsschutzrecht} \cite{hesse2002rise}. Später wurden auch erste Autorenprivilegien gewährt.

\subsection{Wissenschaftliche Journale als Medium der wissenschaftlichen Kommunikation}

Noch zu Beginn des 17. Jahrhunderts stellten das Schreiben von Briefen oder Büchern die häufigsten Formen des wissenschaftlichen Austauschs dar \cite{porter_1964_scientific}. Der Brief, als besonders exklusive Form der Kommunikation stand dem Buch als sehr zeitaufwändige Form gegenüber \cite{fecher_hiig_2014}.

Erst die "drucktechnische Möglichkeit der schnellen Produktion, Vervielfältigungund Verbreitung von Texten" und "die Loslösung der Wissenschaft(en) von Religion und schöner Literatur" machten eine "Umorientierung von sporadischer individueller wissenschaftlicher Betätigung hin zu gesellschaftlich anerkannter und zur Kenntnis genommener, kollektiv bzw. arbeitsteilig betriebener Wissenschaft" möglich \cite{graefen2007_wissenschaftliche_artikel}. Die Gründung von Akademien als einer Art von nationalen Gelehrtengesellschaften im 17. und 18. Jahrhundert führte das zu Veränderungen der wissenschaftlichen Literatur \cite{graefen2007_wissenschaftliche_artikel} und Verschiebung der Darstellung wissenschaftlicher Praxis in separater Experimentierräume \cite{weingart_2005_wissenschaft}. Die Akademien fungierten als Vereingung einzelner Gelehrter und "durch sie fand eine Konzentration vereinzelter wissenschaftlicher Anstrengungen und Leistungen statt" \cite{graefen2007_wissenschaftliche_artikel}. Die "Einführung von Präzisionsmessungen als Teil der experimentellen Praxis", sowie "die Einrichtung separater Experimentierräume, um der Sensibilität der Präzisionsinstrumente gerecht zu werden" ging mit einer "Veränderung der Umgangsformen in der Akademie einher" und damit verlagerte sich "das Problem, andere zu überzeugen, von der unmittelbaren Demonstration von Evidenz auf die mittelbare Darstellung in Texten" \cite{weingart_2005_wissenschaft}.

Mitte des 17. Jahrhunderts kam es in Folge der Gründung der "Royal Society", als eine Akademie zur Förderung naturwissenschaftlicher Experimente, zu einer wissenschaftlichen Diskussion über die Etablierung einer neuen "Philosophie für die Förderung von Wissen". Die Mitglieder der Royal Society hegten den Wunsch nach einer Verbesserung bei der Verbreitung wissenschaftlicher Erkenntnisse und eine "wissenschaftlichen Revolution" mit Hilfe der Drucktechnologie voranzutreiben \cite{Dear_1985}. Als ein Ergebnis der 1660 gegründeten Akademie erschienen 1662 die ersten beiden Bücher, John Evelyn's "Sylva" und "Micrographia" von Robert Hooke \cite{hall_1992_library_rsol}. Am 6. März 1665 mit "Philosophical Transactions" eine der ersten wissenschaftliche Fachzeitschriften \cite{Peters_2014}, "die bis ins 20. Jahrhundert hinein eine der angesehensten Fachzeitschriften blieb" \cite{graefen2007_wissenschaftliche_artikel}. Im gleichen Jahr erschien auch das "Journal des sçavans" in Frankreich, das zu Beginn über aktuelle Entdeckungen berichtete \cite{epaa_Weiner_2001}. Bis zum 17. Jahrhunderts folgten circa 30 weitere Journalgründungen diesem Beispielen. Die Journals unterschieden sich in ihrer Struktur stark von den heutigen und wiesen bis Ende des 18. Jahrhunderts kaum eine fachliche Spezialisierung auf und beinhalteten "auch anwendungs- und praxisbezogene Beiträge" \cite{graefen2007_wissenschaftliche_artikel}. Sie enthielten im Vergleich zu den heutigen Fachzeitschriften jeweils eine nur sehr geringe Anzahl von Beiträgen \cite{suchen} und waren an wissenschaftlichen Briefe (meist in der Ich-Form) angelehnt, die Wissenschaftler vor der Entwicklung der Journale noch direkt aneinander verschickt hatten \cite{epaa_Weiner_2001}. "Oft handelte es sich gar nicht um Originalbeiträge, sondern die Herausgeber teilten der gelehrten und gebil- deten Menschheit mit, was sie aus ihren Briefwechseln mit Gelehrten Interessantes entnahmen" \cite{graefen2007_wissenschaftliche_artikel}.

Mit dieser Veränderung änderte sich auch die Rolle des Autors und es wurden, im Gegesatz zum Mittelalter, auch solche Texte als wissenschaftliche Texte akzeptiert, deren "Garantie" in der Zugehörigkeit zu einem systematischen Ganzen bestand und nicht mehr in dem Verweis auf das Individuum (Autoren) \cite{foucault_2000_autor}.

Die wissenschaftliche Fachzeitschrift oder das wissenschaftliche Journal, wie wir es heute kennen, geht strukturell auf das 19. Jahrhundert zurück, als die forscherischen Aktivitäten und das öffentliche Interesse an der Wissenschaft generell anstieg. In dieser Zeit kam es zu den meisten Gründungen der großen Fachzeitschriften von heute \cite{porter_1964_scientific}. Bis zur Etablierung des Peer-Review-Verfahrens als Qualitätssicherungsverfahren in der zweiten Hälfte des 20. Jahrhunderts gab es sehr unterschiedliche oder keine Verfahren zur Sicherung der Qualität von Inhalten in den Journalen. Im 20. Jahrhundert folgte auf die weltweite Intensivierung wissenschaftlicher Aktivitäten ein weiterer rasanter Anstieg der wissenschaftlichen Journale \cite[:23]{haustein_2012_multidimensional}. Im Jahr 1961 wurde die erste quantitative Studie an Hand der Anzahl von wissenschaftlichen Journalen durchgeführt. Im Rahmen dieser Erhebung wurde von 50.000 wissenschaftliche Zeitschriften und von einer Verdopplung der Anzahl aller wissenschaftlichen Journale alle 15 Jahre ausgegangen \cite{de_1982_little}.

\subsection{Rolle der Verlage und die Publikationskrise}

Noch bis in das 19. Jahrhundert wurden Bücher unter dem Eindruck einer "Unsterblichkeitsnorm geschrieben, die darauf baute, dass erst die Nachwelt das eigentliche Anliegen eines Buches verstehen würde" \cite{hagner_2015_sache_buches}. Darüber hinaus wurden Entdeckungen manchmal in Form eines Anagramms veröffentlicht, so etwa Galileis Entdeckungen der Jupitermonde \cite{miner2007discovery} und Hookes Elastizitätsgesetz \cite{szabo_2013_geschichte}. Auf diese Weise konnten Prioritätsrechte gesichert werden, ohne dass die Entdeckung selbst veröffentlicht werden mussten \cite{miner2007discovery} und Geheimnisse vor Diebstahl geschützt und religiösen Verfolgung vermeiden werden \cite{resnik_2005_ethics}. Erst ab Mitte des 19. Jahrhunderts "verlagerte sich die Produktion immer mehr auf das Hier und Jetzt" \cite{hagner_2015_sache_buches}.

Ursprünglich wurde Wissen an Universiäten gespeichert, übertragen, verarbeitet, aufgezeichnet und später in wissenschaftlichen Journalen und Büchern gedruckt \cite{kittler_2004}. Dieses Wissen wurde in gleicher Weise verbreitet \cite{hagner_2015_sache_buches} und war Eigentum derer, die dafür schrieben oder es lasen \cite{epaa_Weiner_2001}. Sie wurden durch die wissenschaftlichen Akademien oder akademischen Fachgesellschaften, die auch die inhaltliche Ausrichtung verantworteten sowie die finanzielle Trägerschaft übernahmen \cite{suchen}, als Kommunikationsmedium organsiert. Erst im 20. Jahrhundert kam es zu einer unterschiedlichen Verbreitung der Veröffentlichungsformate in und zwischen den unterschiedlichen Disziplinen und im späten 20. Jahrhundert kam der Sammelband als neue Form dazu \cite{hagner_2015_sache_buches}.

Mit dem weltweiten Anstieg der wissenschaftlichen Forschung Mitte des 20. Jahrhunderts und der stetig steigenden Anzahl an wissenschaftlichen Publikationen nach dem zweiten Weltkrieg stieß das universitätseigene Journalsystem an seine Grenzen und es entwickelte sich zu einem "Flaschenhals" \cite{epaa_Weiner_2001} im Kommunikationssystem. Dem Anstieg an wissenschaftlicher Forschung und dem zunehmenden Publikationsdruck der Wissenschaftler konnte das System nicht mehr gerecht werden. Kommerzielle Verlage entdeckten diese Lücke und begannen den Markt mit Unterstützung der überforderten Universitäten zu absorbieren \cite{Hirschi_2015_buch_oa}.

Nachdem die Kommerzialisierung des System anfangs gut funktionierte, kam es zunehmend zu einem Bruch, als die Anforderungen des Marktes nicht mehr denen der akademischen Gemeinschaft entsprachen \cite{epaa_Weiner_2001}. Dennoch verharrte die wissenschaftliche Gemeinschaft in einem "weltfremden" Zustand, in dem der Druck zu veröffentlichen, dazu führte, dass sie ein System unterstützten, das sie ausnutzte \cite{epaa_Weiner_2001}. Sie sahen sprachlos mit an, wie die "Zeitschriften immer größere Anteile der Bibliotheksetats verschlangen" \cite{hagner_2015_sache_buches}. Auch in Deutschland nahmen Anfang der 1990er Jahre die wissenschaftlichen Verlage eine marktbeherrschende Stellung ein und agierten exklusiver Distributor bei der Veröffentlichung wissenschaftlicher Informationen \cite{schloegl_2005} \cite{offhaus_2012_institutionelle_repos}.

Diese Entwicklung basiert auf dem in der Welt des geistigen Eigentums ungewöhnlichen Umstand, dass seit dem Beginn des wissenschaftlichen Journals im Jahr 1665, wissenschaftliche Autoren nicht vordergründig finanzieller Belohnung profitierten, sondern maßgeblich durch die weite Verbreitung und Hinweise auf ihre Arbeit, sowie die wissenschaftlichen Erkenntnisse ihrer Forschung \cite{albert_2006_open_implications}. Darüber hinaus ist es eine Besonderheit des Systems, dass Wissenschaftler sowohl Produzenten als auch Konsumenten der Wissenschaftskommunikation sind und damit Ihre eigene Zielgruppe darstellen \cite{Hess_2006}. Die kommerziellen Verlage haben sich dieses System zu nutze gemacht.

Im Laufe der Zeit erlangten die Verlage eine Vormachtstellung im wissenschaftlichen Publikations- und Distributionssystem. Diese stützt sich bis heute auf drei Säulen \cite{offhaus_2012_institutionelle_repos} \cite{bargheer_2006_open}:
\begin{enumerate}
\item "Urheberrecht, wonach Verlage [...] weitgehende Ansprüche an dem veröffentlichten Werk erwerben“;
\item "redaktionelle Themenbündelung (bundling)“;
\item Organisation der "Qualitätssicherung durch Begutachtung (Peer Review)"
\end{enumerate}

Die marktbeherrschende Stellung der Verlage führte zu einer Situation, in der die Verlage die Preise für wissenschaftliche Publikationen weitgehend diktieren und Preiserhöhungen unlimitiert durchgesetzt werden konnten. Als Folge der ungebremsten Ausnutzung dieser Marktmacht kam es kurz vor der Jahrtausendwende zur sogenannten "Zeitschriftenkrise"  \cite{Martin_2013} \cite{muller_2010_open} \cite{schirmbacher_2009_wisspub} \cite{Parks_2002_acadamic_faust}. Die Zeitschriftenkrise, "die richtigerweise Zeitschriftenpreiskrise oder Zeitschriftenpreisexplosion genannt werden müsste" \cite {Brintzinger_2010}, kam als Begriff das erste Mal in den 1990er Jahren auf \cite{Boni_2010}. Diese Krise war das Ergebnis folgender Entwicklungen auf der Angebots- und Nachfragenseite \cite{Brintzinger_2010}: Auf der Angebotsseite wurden durch einen "Konzentrationsprozess" "innerhalb von etwas mehr als einem Jahrzehnt im Bereich der Zeitschriften mittelständische Verlage nahezu vollkommen durch internationale Kapitalgesellschaften substituiert". \cite{Brintzinger_2010} Unterstützt von der Nachfrageseite resultierte daraus eine "monopolistische Preispolitik" der Verlage \cite{Brintzinger_2010}. Ein zeitgleicher Anstieg der Titelvielfalt, bei der aus "einer mehr generalistischen Zeitschrift drei oder vier Spezialzeitschriften" entstanden, "die dann allesamt wieder von Bibliotheken abonniert werden mussten" \cite{Brintzinger_2010}, verschärfte das Problem. Eine weitere Ursache für die krisenhafte Zuspitzung der Situation besteht in der institutionellen Organisation der Literaturbeschaffung an den Hochschulen und wissenschaftlichen Einrichtungen. Bei der Arbeitsteilung von Bibliothekaren und Wissenschaftlern war und ist es für das Ansehen des einzelnen Faches durchaus rational, mit einem möglichst hohen Anteil am Gesamtetat der Literaturbeschaffung zu partizipieren. Es gibt für individuelle Einsparungen von allen Seiten nur wenig Anlass, da beide Systeme unabhängig voneinander funktionieren. \cite{Brintzinger_2010}.

Die Preisexplosion konnte auch durch die Bildung von Bibliothekskonsortien, "deren Aufgabe es war, für Bibliotheken kostengünstige Rahmenbedingungen auszuhandeln", nicht gebändigt werden \cite{Fladung_2003} \cite{Brintzinger_2010}. Gleichzeitig standen die Wissenschaftler unter einem starken Publikationszwang, der mit "Publish or Perish" \cite{CLAPHAM_2005} beziehungsweise "impact factor fever" \cite{Cherubini_2008} und "impact factor race" \cite{Brischoux_2009} beschrieben wurde \cite{offhaus_2012_institutionelle_repos}. "Publish or Perish" beschreibt das Problem, dass im Rahmen der "wachsenden Konkurrenz um Forschungsförderung und akademische Positionen (...) kombiniert mit dem zunehmenden Einsatz bibliometrischer Parameter für Evaluation" \cite{Fanelli_2010} junge Akademiker viel und vornehmlich mit positiven wissenschaftlichen Ergebnissen publizieren müssen um Anerkennung und gegebenenfalls eine Anstellung im Wissenschaftsbetrieb zu erreichen \cite{pscheida_2010_wikipedia} \cite{Beasley_2005} \cite{hamilton_1990_publishing}. Das führte zu einer "beinahe explosionsartige Entwicklung der Anzahl wissenschaftlicher Publikationen" \cite{bortz_Doering_2006_fragestellung} und zu der damit einhergehenden Vemutung von viel "nutzlose Forschung und Artikel"\cite{smith1990killing}, einem „leeren Größenwachstum" \cite{bbaw_publizieren_2015} und viele wissenschaftliche Arbeiten mit "vernachlässigbare Beiträge zum Wissen" \cite{hamilton_1990_publishing}. Inwieweit es diese Entwicklungen allein zu einer "Lawine von niedriger Qualität der Forschung" \cite{Bauerlein_2010} in dem beschriebenen Umfang geführt hat oder ob die neuen (digitalen) Möglichkeiten die schon immer bestehenden Qualitätsunterschiede wissenschaftlicher Publikationen einfacher aufgedeckt haben, ist umstritten \cite{rekdal_2014_academic}.

Die genannten Entwicklungen machten dennoch mehrere problematischen Effekten im Publikationsystem sichtbar: Erstens, die vermehrte Einreichung von Manuskripten bei begutachteten Publikationsmedien führte zu einer "schädlichen und vermeidbaren zusätzlichen Belastung der Begutachtung", zweitens erhöhen das "Größenwachstum" auf Seiten des Lesers "den Aufwand für Auswahl, Beschaffung und Lektüre von Publikationen" und drittens "steigen (...) die Kosten für das Publikationssystem insgesamt" \cite{bbaw_publizieren_2015}.

\subsection{Computer und Internet als neue Medien wissenschaftlicher Kommunikation}

Der Begriff Medien ist ein Sammelbegriff, der grundsätzlich in der älteren Medientheorie entwender als neutrale technische Infrastrukturen oder als Kommunikations, Wahrnehmungs- oder kulturdeterminierende Techniken betrachtet wurde \cite{beck2005_Kommunikation}. Bei genauerer Betrachtung des Begriffs in den unterschiedlichen Disziplinen, die sich mit Medien beschäftigen, "sind die Gebrauchsweisen und Bestimmungen des Begriffs Medium äußerst hetrogen" und "es hat den Anschein, als könnte die Frage, was Medien sind, zu keiner befriedigenden Antwort führen" \cite{Burkhardt_2015}.

Mit dem Begriff digitale Medien hat das Denken über Medien nachhaltig beeinflusst  \cite{Burkhardt_2015}. Digitale Medien können als Medientechnologien bezeichnet werden, die durch Computer verarbeitet werden \cite{nunning_2013_metzler}. Durch die zunehmende Verbreitung des Computers und des Internets Ende der 1980er Jahre wurde dem Medienbegriff eine weitere Unbekannte hinzugefügt \cite{Burkhardt_2015}. Das Internet gilt dennoch als Paradigma für digitale Medien, da hier unterschiedlche Medien mehrfach venetzt werden: Zum einen werden miteinander vernetzte Computer lokal und global über Telekomunikationskanäle miteinander verbunden, zum Anderen konvergieren in diesem globalen Netz Schriftlichkeit, Bild und Ton \cite{nunning_2013_metzler}.

Der Bestand an Rechenkapazitäten an Universiäten hat sich seit den 1989 konstant weiter verdichtet \cite{Rutenfranz_1997}. Ende des letzten Jahrtausends eröffnete das Internet "neue Nutzungsmöglichkeiten, durch welche die Schrift als ein Medium einsetzbar wird, das den permanenten Wechsel zwischen Sender- und Empfängerposition ähnlich flexibel zu gestalten erlaubt, wie es im gesprochenen Gespräch der Fall ist" \cite{sandbothe_2000_pragmatische}. Die Vernetzung schaffte auch in der Wissenschaft eine mediale Schnittstelle zwischen Autoren und Rezipienten, die keiner meschlichen Vermittlung durch Dritte (wie z.b. Verlage) mehr bedarf \cite{naeder_2010_open}. Mit der Etablierung eines globalen Kommunikationsnetzes ging auch die Vermutung einher, "dass im Internet als einem frei zugänglichen Medium mit geringen Zugangsbarrieren (...) Zugang zur Öffentlichkeit erhalten können, der ihnen bei den alten Medien verwehrt bleibt" \cite{Gerhards_2007}. Auch wenn sich im Internet bisher keine direkte demokratischere Kommunikation finden lässt \cite{Gerhards_2007}, so herrscht weiterhin große Euphorie bezügliches der verminderten Zugangsbarieren, der umfassenden Möglichkeiten für die Vermittlung von Inhalten und die Transformation klassischer Kommunikationsmedien und Kanäle \cite{suchen}.

Digitale Souveränität und die Nutzung des Internets wird in Deutschland strukturell durch das Bildungsniveau und die erworbene Medienkompetenz bestimmt. Bei einer repräsentativen Befragung gaben 2014 92 Prozent der Teilnehmer mit abgeschloßenem Hochschulstudium an "Online" \cite{nonliner_2014}. Ganz pragmatisch ausgedrückt, gehören zum Einsatz digitaler Medien in den Geisteswissenschaften "die Nutzung von Textverarbeitungssoftware genauso wie die Recherche im Bibliothekskatalog mittels OPAC und die Informationsbeschaffung und Kommunikation mittels World Wide Web und E-Mail" \cite{naeder_2010_open}.

Im Zusammenhang mit der zunehmenden Verbreitung von Computer und Internet unter den Wissenschaftlerinnen und Wissenschaftlern, kann der Webbrowser dabie als Kreuzung zwischen Buch und Fernseher verstanden werden, bei dem das multimediale Dokument von der Buchkultur als zentrales Wahrnehmungsobjekt übernommen wird, zugleich aber darüber hinaus greift \cite{Warnke_2011}. Als weiteres Veränderung in Abgrenzung zur Technologie Buchdrucks revidierte das Internet "die Vorstellung von einem geschlossenen Sinngehalt" \cite{sandbothe_2000_pragmatische} mit einem Anfang und Ende wie zum Beispiel in einem Buch.

Insgesamt folgt diese Entwicklung der Annahme, dass die Buchkultur zwar von einer Dialogkultur abgelöst, aber nicht vollständig verdrängt wird. Das Gedruckte kommt demnach als eine Art Rückzugs- oder Entlastungsmedium zum Einsatz \cite{hagner_2015_sache_buches}. Dabei sind "wechselseitige Steigerungen, funktionale Kopplungen und vielfältige Kombinationen" zu erwarten und der damit einhergehende Medienwandel verändert vor allem "die bereits verbreiteten Medien und damit die medialen Verhältnisse einer Gesellschaft" \cite{Koenen_1997}, er verdrängt sie aber nicht zwangsläufig.

Die technologische Entwicklung des Computers und des Internets waren dabei bisher eng mit der Annahme verbunden, dass sie "Freiheit" sichert, bietet, verbessert, oder verstärkt. Der Computers, als Zugangsgerät zu digitalen Informationen, ermöglichte eine neue Form der Zusammenarbeit unterschiedlicher wissenschaftlicher Richtungen an einer gemeinsamen Arbeitstation, eröffnete die Perspektive einer methodischen Integration von unterschiedlichen wissenschaftlicher Betrachtungen und bietet die Chance der Vereinigung von bisher getrennten Notationssystemen von Alphabeten und mathematische Symbolen \cite{kittler_2004}. Doch auch nach 25 Jahren kann nicht abschließend evaluiert werden, inwieweit die Freiheit diesen Technologien innewohnt - und mit "free" nicht nur der Preis gemeint ist \cite{stallman2002} - und wie sie gestaltet werden kann \cite{kelty_2014_freedom}.

\subsection{Erste Experimente mit offenem Zugang zu wissenschaftlichen Publikationen}

Die Zeitschriftenkrise und der gestiegene Publikationsdruck stellen zwei fundamentale Gründe für das Aufkommen der Forderungen nach "Open Access" dar \cite{Brintzinger_2010} \cite{wein_2010_erwerbung}. Als Reaktion auf die Herausforderungen und auf Basis der Digitalisierung gründete Anfang der 1990er der Physiker Paul Ginsparg mit arXiv den ersten wissenschaftliche Preprint-Dienst des Internets \cite{cite:5} \cite{bjork_2004_open}, der es Wissenschaftlern ermöglichen sollte, Ideen vor der gedrukten Veröffentlichung zu teilen.

Ein Ausgangspunkt dafür waren die ersten Experimente mit offenem Zugang und freien Lizenzen für Publikationen in der Wissenschaft aus den 1960er Jahren und somit schon vor der Zeit der Erfindung des Internets \cite{cite:18b}. Noch bevor die digitalen Nutzungsmöglichkeiten verfügbar waren und bevor an das "globalen Dorf" \cite{mcluhan_1962_gutenberg} zu denken war, wurde vor allem in den Technik- und Naturwissenschaften eine “pre-print Kultur” entwickelt bei der die Autoren ihre zur Begutachtung eingereichten Artikel zeitgleich oder bevor diese veröffentlicht wurden unter Kollegen über den Postweg zirkulieren ließen, um den Kommunikationsprozess zu beschleunigen \cite{suchen-Hoffmann-Zugang-undVerwertung-oeffentlicher-Informationen}. Darüber hinaus gab und gibt es "informellen Wege des Zugangs" zu wissenschaftlichen Publikationen: zum Beispiel durch Kollegen an Institutionen die auf die Publikation zugreifen können oder durch die direkte Anfrage einer Kopie beim Autor \cite{davis_2011_open}.

Mitte der 1990er forderte Steven Harnad die wissenschaftliche Community dazu auf, sofort mit der digitalen Selbstarchivierung und öffentlichen Zurverfügungstellung ihrer Beiträge zu beginnen \cite{albert_2006_open_implications}, um "den Barrieren, die zwischen ihrer Arbeit und ihrer (kleinen) Leserschaft aufgestellt werden, zu entkommen" \cite{harnad_1995_subversive_proposal}.

Durch die zunehmende Verbreitung und Nutzung der dieser digitalen Pre-Print Dienste, gründete sich im Oktober 1999 im Rahmen der "Santa Fe Convention" die "Open Archives Initiative", die sich maßgeblich mit den technischen und organisatorischen Aspekten der Transformation der wissenschaftlichen Kommunikation beschäftigte \cite{van_2000_santa_fe}.

2001 wurde der europäische Ableger von der Scholarly Publishing and Academic Resources Coalition (SPARC) einer der späteren "major player" der Open Access Bewegung \cite{russell2008business} \cite{Herb_2012} gegründet. Als Konsequenz aus der Zeitschriftenkrise sollte diese 1998 in den USA gegründete Allianz zwischen Universitäten und wissenschaftlichen Bibliotheken dafür sorge tragen, dass die Kosten für wissenschaftliche Publikationen reduziert werden und durch die Bereitstellung kostengünstiger oder freier, nicht-kommerzieller, Peer-Review-Fachzeitschriften zu ersetzen sind. Durch Weiterbildung, politische Arbeit und die Förderung alternativer Geschäftsmodelle, war es Ziel von SPARC, Initiativen für offenes wissenschaftliches Publizieren zu stimulieren \cite{suchen}.

\subsection{Die Manifestierung der Forderung nach offenem Zugang}

In 2001 erschien Open Access erstmals im wissenschaftlichen Diskurs als öffentlichkeitswirksames Thema \cite{cite:19}. Die Public Library of Science (PLoS), gegründet im Oktober 2000, forderte die gesamte wissenschaftliche Gemeinschaft in einem offenen Brief im Mai 2001 dazu auf, ab September 2001 nur noch in Zeitschriften zu veröffentlichen, nur noch die Zeitschriften zu reviewen, zu editieren und zu abonnieren, deren Beiträge spätestens sechs Monate nach ihrer Erstveröffentlichung für jedermann im Internet kostenlos und unentgeltlich einsehbar sind \cite{cite:20}. Schon nach kurzer Zeit unterzeichneten nach eigenen Angaben \cite{cite:19a} rund 38.000 Wissenschaftler und Wissenschaftlerinnen aus 180 Nationen das Schreiben. Auf diesen Brief folgte ein 20-monatige sehr aktive und öffentlichkeitswirksame Phase der Forderung nach Öffnung der wissenschaftlichen Kommunikation. In diesen 20 Monaten wird neben PLoS der britische Verlag Biomed Central als weiterer "Wegbereiter in der von OA" \cite{suchen-Hoffmann-Zugang-undVerwertung-oeffentlicher-Informationen} gegründet und es entstehen drei der bis heute wichtigsten Erklärungen im Bereich der Öffnung des Zugangs zu wissenschaftlicher Kommunikation \cite{CREATe_2014}:
\begin{enumerate}

\item Erklärung der Budapest Open Access Initiative (Dezember 2002 und 2012)

Im gleichen Jahr wie der PLoS-Brief, wurden im Rahmen einer Konferenz des Open Society Institutes in Budapest, mit der “Budapest Open Access Initiative" (BOAI)\cite{boai_2002} erstmals die Bemühungen um Open Access in einer eigenen Erklärung zusammengefasst \cite{yiotis_2013_open} \cite{garcia_2010_open} \cite{cite:21a}. Im Fokus dieser Erklärung steht die Forderung nach freiem Zugang (ausschließlich) zu wissenschaftlichen Zeitschriftenpublikationen, "die zuvor einen Peer-Review-Prozess durchlaufen haben und anschließend, parallel zur Veröffentlichung in der Zeitschrift, im Netz frei zur Verfügung gestellt werden sollten" \cite{Schirmbacher_oa_2007}. In der BOAI wird erstmals manifestiert, dass wissenschaftliche Peer-Review-Fachliteratur "kostenfrei und öffentlich im Internet zugänglich sein sollte, so dass Interessenten die Volltexte lesen, herunterladen, kopieren, verteilen, drucken, in ihnen suchen, auf die Volltexte verweisen, sie indexieren, sie als Daten weiterverabreiten und sie auch sonst auf jede denkbare legale Weise benutzen können, ohne finanzielle, gesetzliche oder technische Barrieren jenseits von denen, die mit dem Internet-Zugang selbst verbunden sind". \cite{boai_2002} Die Erklärung manifestiert auch, dass in "allen Fragen des Wiederabdrucks und der Verteilung und in allen Fragen des Copyrights überhaupt, sollte die einzige Einschränkung darin bestehen, den Autoren Kontrolle über ihre Arbeit zu belassen und deren Recht zu sichern, dass ihre Arbeit angemessen anerkannt und zitiert wird." \cite{boai_2002}

Die Erklärung manifestierte erstmals ein Bild davon, was eine Open Access Publikation von einer Veröffentlichung in einer herkömmlichen Fachzeitschrift und von einer kostenlosen, aber nur sehr eingeschränkt nutzbaren Digitalversion eines Artikels unterscheidet und eigent sich demnach als Anknüpfungspunkt für die Open-Access-Bewegung \cite{naeder_2010_open}. Sie bezog sich dabei explizit erstmal nur auf wissenschaftliche Zeitschriftenliteratur \cite{boai_2002}.

Anlässlich des zehnten Jahrestages der BOAI, wurde von der Open Society Foundation mit der BOAI 10 (2012) die ursprüngliche Erklärung um weitere Richtlinien und Empfehlungen für die Entwicklungen und Herausforderungen bei der Öffnung wissenschaftlicher Kommunikation ergänzt. Die Initiatoren kommen unverändert zu dem Schluss, dass "noch immer Zugangsbeschränkungen zu Peer-Review-Forschungsliteratur, meist eher zugunsten der Verlage, als zugunsten der Autoren, Reviewer oder Redakteure und damit auch auf Kosten der Forschung, Forscher und Forschungseinrichtungen" \cite{boai_2012} bestehen. Dazu heißt es in der überarbeiteten Erklärung: "Nichts aus den letzten zehn Jahren lässt darauf schließen, dass das ursprüngliche Ziel von Open Access weniger sinnvoll oder erstrebenswert erscheint. Im Gegenteil, die Notwendigkeit, dass Wissen für jeden, der es nutzen, anwenden oder darauf aufbauen kann, offen verfügbar sein sollte, ist dringlicher als je zuvor" \cite{boai_2012}. Darüber hinaus erfolgte auch eine Adaption der weiterführenden Aspekte der Stellungnahme von Bethesda und der Berliner Erklärung.

\item Die Bethesda Stellungnahme (Juni 2003)

Ein Jahre nach Veröffentlichung der initialen Version der BOAI-Erklärung, im Juni 2003, verabschiedete eine Gruppe von Forschungsförderern, wissenschaftlicher Gesellschaften, Verlegern, Bibliothekaren, Forschungseinrichtungen und einzelner Wissenschaftler im US-Bundesstaat Maryland das "Bethesda Statement on Open Access Publishing" \cite{suber_2003_bethesda}. Ziel der Erklärung war die Stimulation der Diskussion in der biomedizinischen Forschung, "wie man schnellstmöglich den offenen Zugang zu der primären wissenschaftlichen Literatur in der Biomedizin erreichen könnte" \cite{suber_2003_bethesda}. Ähnlich wie in der BOAI benannten die Autoren des "Bethesda Statements on Open Access Publishing" die Bedingungen für den offenen Zugang zu wissenschaftlichen Publikationen \cite{suber_2003_bethesda}:

Erstens werden Autor(en) und Urheberrechts-Inhaber aufgefordert, für alle Benutzer ein freies, unwiderrufliches, weltweites und unbefristetes Recht auf den Zugang zu genehmigen, sowie eine Lizenz zu verwenden, die das Kopieren, Nutzen, Verbreiten, Übertragen und öffentliches Darstellen der Publikation ermöglicht. Darüber hinaus soll es erlaubt sein, abgeleitete Werke zu verteilen und in jedem digitalen Medium für jeden Zweck zu veröffentlichen, vorbehaltlich einer angemessenen Zuordnung der Urheberschaft. Das beinhaltet auch das Recht auf eine kleine Anzahl gedruckter Kopien für den persönlichen Gebrauch.

Zweitens, muss eine vollständige Version der Arbeit und aller ergänzender Materialien, einschließlich einer Kopie der Genehmigung, wie oben erwähnt, in einem geeigneten elektronischen Standardformat unmittelbar bei der ersten Veröffentlichung in mindestens einem Online-Repositorium, das von einer wissenschaftlichen Einrichtung unterstützt wird, hinterlegt werden. Dieses Repositorium muss von einer wissenschaftlichen Gesellschaft, Regierungsbehörde oder einer anderen etablierten Organisation akzeptiert sein. Diese muss sich für einen offenen Zugang, uneingeschränkte Verbreitung sowie Interoperabilität und Langzeitarchivierung (für die biomedizinischen Wissenschaften, PubMed Central ist ein solches Repository) verpflichtend einsetzen.

Die Bethesda Stellungnahme ist in einigen Punkten präziser als die Budapester Erklärung, öffnet aber ihren Wirkungsraum auch auf Monografien und nicht-wissenschaftliche Publikationen. So enthält das die Stellungnahme und damit einhergehende Definition Erweiterungen, die später in der Berliner Erklärung ebenfalls aufgegriffen werden, addressiert die Zugänglichkeit von im Rahmen der Publikationen erarbeiteten Zusatzmaterialien wie Mess- und statistische Daten, fordert das "Recht zur Erstellung und Publikation abgeleiteter Werke" (Derivate), "bindet Open Access unmittelbaran digitale Medien", schreibt sofort nach der Erstveröffentlichung die frei zugänglichen Veröffentlichung vor und rückt Open Access in die Nähe offener und freier Inhalte im weiteren Sinne \cite{naeder_2010_open}.

\item Die Berliner Erklärung (Oktober 2003)

Einen weiteren Meilenstein für die Verbreitung der Idee von Open Access auf dem europäischen Kontinent stellten die "Berlin Konferenzen" \cite{CREATe_2014} dar. Die erste Tagung wurde 2003 von der Max-Planck-Gesellschaft und dem Projekt European Cultural Heritage Online (ECHO) organisiert, um über "Zugangsmöglichkeiten zu Forschungsergebnissen" zu diskutieren. In diesem Rahmen entstand 2003 auch die "Berliner Erklärung über den offenen Zugang zu wissenschaftlichem Wissen" \cite{berliner_erklaerung_2003}, in der die Verfasser über die Budapester und die Bethesda Erklärung hinaus gehen und neben dem kostenlosen und freien Zugang zu wissenschaftlichen Endergebnissen in Form von Publikationen auch den freien und offenen Zugang zu wissenschaftlichen Daten fordern. „Open Access-Veröffentlichungen umfassen originäre wissenschaftliche Forschungsergebnisse ebenso wie Ursprungsdaten, Metadaten, Quellenmaterial, digitale Darstellungen von Bild- und Graphik-Material und wissenschaftliches Material in multimedialer Form.“ \cite{berliner_erklaerung_2003}

Mit dieser Ausweitung der Erklärung auf die Daten hinter den Publikationen, formiert sich erstmals ein klares erweitertes Verständnis von Open Access. Damit entsteht auch die erste Grundlage für ein erste Ansatzpunkte zur Definition von Open Science, da hier der offene Zugang als eine "umfassende Quelle menschlichen Wissens und kulturellen Erbes, die von der Wissensgemeinschaft bestätigt wurden" \cite{berliner_erklaerung_2003} verstanden wird. Die Erklärung schließt damit jegliche wissenschaftlichen und nicht-wissenschaftliche Arbeiten ein, "unabhängig von Disziplin und Art der Publikation" und "jedweder Herkunft" \cite{naeder_2010_open}. Die Diskussion um die Berliner Konferenzen konzentrieren sich in diesem Stadium aber dennoch hauptsächlich auf den bereits abgeschlossenen wissenschaftlichen Prozess.

Die Autoren der Berliner Erklärung erahnten die Bedeutung und möglichen Konsequenzen ihrer umfassenden Forderungen, sowie den Herausforderungen bei der Umsetzeung. Nur so erklärt sich die "Diskrepanz zwischen der kompromisslosen Proklamation der Prinzipien und der durch vorsichtige Wortwahl geprägten "Unterstützung des Übergangs zum ‚Prinzip des offenen Zugangs’"" in der Praxis" \cite{Lossau_oa_2007}.
\end{enumerate}

Alle drei Erklärungen, auch die "three B's" \cite{suber_2004_praising_oa} genannt, gelten als die anerkanntesten Erklärungen von Open Access und stimmen in den wesentlichsten Merkmalen überein \cite{albert_2006_open_implications}, divergieren aber in Detailfragen \cite{naeder_2010_open}. Sie alle eint vor allem die Kernforderung nach der Beseitigung der preislichen und in Teilen der rechtlichen Barrieren bezüglich des freien Zugangs zu den wissenschaftlichen Publikationen. Sie alle haben keine zwar rechtlich bindende Inverventionen und keine Saktionsmechanismen, nutzen aber Anreizelemente für die Durchsetzung der definierten Forderungen. Weiterhin eint sie auch, dass alle drei Erklärungen ihre Ursprünge in den STM-Fächern haben und vornehmlich auf den Erfahrungen mit der Zeitschriftenkrise in diesen Fächern basieren \cite{naeder_2010_open}. Trotz der Unterschiede im Detail ähneln sich die Definitionen auch bei geforderten Beseitigung der Barrieren für die kommerzielle Nutzung und die Erstellung von Derivaten \cite{CREATe_2014}. Die drei Erklärungen wurden darüber hinaus "von unterschiedlicher Seite vielfach präzisiert, interpretiert, eingeschränkt und erweitert" \cite{naeder_2010_open}, woraufhin sich eine eine "BBB-
Definition (Budapest-Bethesda-Berlin) von Open Access etabliert hat" \cite{Schirmbacher_oa_2007}.

Schon ein Jahr vor der ersten Open Access Erklärung, in 2001, folgte die Entwicklung und 2002 die Veröffentlichung der ersten Creative Commons Lizenzen \cite{garcia_2010_open}. Diese Lizenzen waren inspiriert von den Lizenzen der freien Softwarebewegung und wurden kostenlos zur Verfügung gestellt \cite{Minjeong_2007}. Sie ermöglichten das freie Lizenzieren von Werke für bestimmte Verwendungen, unter bestimmten Bedingungen; oder ermöglichten die gemeinfreie Nutzung ohne Einschränkungen. Die Creative Commons Lizenzen bilden bis heute die urheberrechtliche Grundlage für eine Vielzahl der Open Access Publikationen weltweit \cite{suchen}. Im Oktober 2004 waren 5 Millionen Werke unter einer CC-Lizenz verfügbar /cite{Suchen Forbes: Movement Seeks Copyright Alternatives}. Nach eigenen Angaben von Creative Commons (https://stateof.creativecommons.org/) stieg die Anzahl der unter CC-lizensierten Werke auf 50 Millionen im Jahr 2006, 400 Millionen in 2010 und 882 Millionen in 2014. Seit 2010 ist auch ein Shift hin zu offenen Lizenzmodellen innerhalb der CC-Lizenzen ersichtlich. Waren 2010 noch 60 Prozent der 400 Millionen Werke unter den restriktiven CC-Lizenzen veröffentlicht, sank der Anteil in 2014 auf 44 Prozent. Die modularen Lizenzen sind im Kontext von Open Access besonders wichtig, "um (Nach-)Nutzungsmöglichkeiten für Texte, Daten und andere wissenschaftliche Erzeugnisse festlegen zu können" \cite{suchen-Hoffmann-Zugang-undVerwertung-oeffentlicher-Informationen}.

\subsection{Weitere Etablierung von Offenheit}

Im Jahr 2003 entstand das Portal Directory of Open Access Journals (DOAJ), das bis zum Jahr 2013 von der schwedischen Universität Lund betrieben wurde \cite{doaj_2015_about}. Das Portal stellt eine zentrale Anlaufstelle für Open Access-Journale dar \cite{suber_2015} und "zielt darauf ab, Ausgangspunkt für qualitative und peer-reviewte open access Meterialien zu sein" \cite{doaj_2015_about}. 2012 folgte dem DOAJ-Modell mit dem Directory of Open Access Books (DOAB) ein Portal für qualitätsgeprüfte Open Access Bücher und Monografien \cite{adema_2013_political}.

Die Deutsche Forschungsgemeinschaft (DFG) reagierte Anfang 2006 und verabschiedete eine Richtlinie nach der sie zwar nicht voraussetzt, aber "erwartet", dass Publikationen aus DFG-geförderten Projekten "möglichst" als Open Access veröffentlicht werden \cite{suchen:dfg-richtlinie}. Eine ähnliche Erklärung verabschiedete auch die größte amerikanische Förderinstitution National Institutes of Health (NIH) und "stellte mit PubMed Central (PMC) eine entsprechende Plattform bereit \cite{muller_2010_open}. Anfangs wurde die offene Veröffentlichung der Publikation auf Grund eines Aufschreis der Verlage nur "empfholen". Die Verlage sahen in der Richtlinie einen Untergang der wissenschaftlichen Qualitätssicherungsprozesse vorher \cite{Baggs_2006}. In 2008 wurde die Veröffentlichung NIH-geförderter Publikationen nach einer Embargozeit dennoch verpflichtend \cite{Hanekop_2014}. Aktuell gibt es in Deutschland ist keine zentrale Plattform wie PMC und die Veröffentlichug der geförderten Ergbnisse als Open Access ist weiterhin nicht bindend.

In der Debatte über die Zukunft des wissenschaftlichen Publizierens und Kommunizierens besteht die Tendenz, Konzepte der offenen Wissenschaft als einen bisher beispiellosen und noch nie dagewesenen Wandel darzustellen \cite{cite:17a} \cite{cite:17b}. Diese Haltung basiert auf "verschiedenen Gründungsmythen", die auf "unterschiedliche Zielsetzungen und Lösungspfade" verweisen \cite{suchen-Hoffmann-Zugang-undVerwertung-oeffentlicher-Informationen}. Die Geschichte von Open Access ist eine Entwicklung, die eng mit der Digitalisierung von Kommunikationsprozessen auf der einen, sowie mit der Zeitschriftenkrise auf der anderen Seite verknüpft ist \cite{suchen-Hoffmann-Zugang-undVerwertung-oeffentlicher-Informationen} \cite{yiotis_2013_open} \cite{wein_2010_erwerbung}. Open Access ist kein Selbstzweck \cite{cite:17d}, sondern ein Attribut tiefergehender Prozesse, die mit der wachsenden Bedeutung der Digitalisierung in unserer Zivilisation und dem damit einhergehenden Wandlungsprozessen im Machtgefüge zusammenhängen \cite{cite:17e}. Es bleibt jedoch herauszuheben, dass es trotz vereinzelter Versuche, wissenschaftliche Informationen und Publikationen offen und frei zu kommunizieren, Open Access im Printzeitalter physisch und ökonomisch über lokale Grenzen hinaus schwer möglich war \cite{cite:18a}.

Die Forderung nach der Öffnung von Wissenschaft und Forschung ist in diesem Zusammenhang nicht nur eine "politische Reaktion" oder "technische Alternative", sondern eine "alternative Formatierungen einer wissenschaftlichen Infrastruktur im technischen, rechtlichen und zeitlichen Sinne" \cite{kelty_2004}. Sie betrifft "Wissenschaftler, politische Entscheidungsträger und die Öffentlichkeit" \cite{Scheliga_2014}.

Auf die Entwicklungen folgten weitere Deklarationen:
\begin{itemize}
\item 2007 die "Kronberg Declaration on the Future of Knowledge Acquisition and Sharing"
\item 2012 das "The Cost of Knowledge Manifesto"
\end{itemize}

---- TODO: weiter ausarbeiten ----

\subsection{Von Open Access zu Open Science}

Die zunehmenden Verbreitung des Internets, der zunehmenden Digitalisirung wissenschaftlicher Abläufe und den Möglichkeiten des kollaborativen Arbeits über digitale Infrastrukturen haben die "praktischen und wirtschaftlichen Bedingungen für die Verbreitung von wissenschaftlichem Wissen und kulturellem Erbe grundlegend verändert" \cite{berliner_erklaerung_2003}. Diese Veränderungen ermöglichen erstmals nicht nur die Öffnung der wissenschaftlichen Erkenntnisse in Form von Publikationen, sondern auch die Daten und Informationen der gesamten wissenschaftlicher Kommunikation. In Ergänzung zu dem Konzept von offenem Zugang (Open Access) zu wissenschaftlichen Publikationen, erstreckt sich das Konzept der offenen Wissenschaft (Open Science) denach auf sämtliche Prozesse des wissenschaftlichen Erkenntnisprozesses.

Die 2010 veröffentlichten Panton Principles greifen einen Teil dieser Erweiterung des offenen Zugangs zu wissenschaftlichen Publikationen auf und ergänzen diesen um den offenen Zugang zu Daten für die jeweilige Publikation. Sie folgen unter anderem der Annahme, dass andere Wissenschaftler und Wissenschaftlerinnen sowie die Gesamtgesellschaft nur dann vollumfänglich von wissenschaftlicher Forschung profitieren kann, wenn auch der Kern der Forschung, die Daten auf der sie basiert, unter den Kriterien von der Open Definition \cite{open_definition} zur Verfügung stehen.

Addressierte Open Access die Öffnung von wissenschaftlichen Publikationen, mit maximal geringfügiger Änderung des Kommunikationssystems, wird Open Science als Möglichkeit gesehen Wissenschaft grundlegender zu transformieren. Die Europäische Kommission sieht diesen Transformationsprozess vor allem in Hinblick auf die Demokratisierung von Forschung, neue Disziplinen und Forschungsthemen, die Symbiose aus Wissenschaft, Gesellschaft und Leitlinien und transparenter, reproduzierbarer Forschung.

--- TODO: Bild Europäische Kommission: Digital Agenda for Europe - Open Science Dreieck einbauen ----

Im Jahr 2013 konsultierte die Europäische Kommission über 130 Vertreterinnen und Vertreter aus Forschung, Industrie, Forschungsförderung, Bibliotheken, Verlagen und Anbietern von Forschungsinfrastrukturen, um die Implikationen aus diesem raschen technologischen Wandel zusammenzufassen sowie Grundlagen für kommende europäische Forschungsförderung zu definieren. Dabei stehen vorallem Forschungsdaten im Vordergrund. Aus der Sicht der Forscher, umfassen Forschungsdaten alle Daten aus einem Experiment, Analyse oder Messung, einschließlich Metadaten und Details über die Verarbeitung der Daten \cite{eu_consult_data_2013}. Für Verlage, handelt es sich um Daten, die mit der  Publikationen verknüpft sind \cite{eu_consult_data_2013}.

Soziale Medien, die technologischen (Weiter)Entwicklungen im letzten Jahrzehnt in Bezug auf Geschwindigkeit der Verbreitung von Informationen und Speicherkapazität für Daten ermöglichen erstmals die digitale Bereitstellung sämtlicher Erkenntnisse und Informationen die in der Wissenschaft gewonnen werden. Die Berliner Eklärung (siehe Kapitel xx) nimmt diese Gedanken schon 2003 auf und ergänzt die Forderung nach offenem Zugang zu originäre wissenschaftliche Forschungsergebnisse um "Ursprungsdaten, Metadaten, Quellenmaterial, digitale Darstellungen von Bild- und Graphik-Material und wissenschaftliches Material in multimedialer Form" \cite{berliner_erklaerung_2003}.

Im April 2012 wurde die Erklärung "Open Science for the 21st century" vom Zusammenschluss der Europäischen Akademien (ALLEA) verabschiedet \cite{ALLEA_2012}. Sie war nur eine von mehreren Erklärungen und Positionspapieren für die Öffnung von Wissenschaft durch international angesehenen Einrichtungen, durch die verdeutlich wurde, dass die Forderung nach offenem Umgang mit Wissen und Information im wissenschaftlichen Bereich zunehmend an Relevanz gewinnt \cite{schulze_2013_open}.

2013 folgte mit der "San Francisco Declarationon Research Assessment" (DORA) \cite{DORA_2013} ein öffentlicher Aufruf, nicht länger auf journal-basierte Metriken als Maß für die Meßung der Qualität einzelner Forschungsartikel, die Beiträge eines einzelnen Wissenschaftlers, oder bei der Einstellung, Beförderung, oder Forschungsförderungsentscheidungen zu setzen. Die Erklärung fordert zudem Forschungsförderer auf, die gesamte Forschungsleistung und die Wirkung von Wissenschaftlern zu berücksichtigen. Dazu gehören neben der Publikation, auch Datensätze und Software.

Beide Erklärungen addressieren auf unterschiedliche Art und Weise eindrücklich die Notwendigkeit für die Öffnung des wissenschaftlichen Erkenntnisprozesses weit über den reinen Zugang zu wissenschaftlichen Publikationen hinaus. Nur durch eine Öffnung des gesamten Prozesses wissenschaftlicher Forschung, so die Annahme, kann Wissenschaft dem gesellschaftlichen Auftrag des Wissenschaftssystems vollumfänglich gerecht werden und die Herausforderungen an das wissenschafliche Kommunikationssystem gelöst werden.

\section{Herausforderungen im bestehenden System wissenschaftlicher Kommunikation}

Über die Wirksamkeit und Zweckmäßigkeit des wissenschaftlichen Kommunikationssystems existieren seit Jahrzehnten Debatten in der Literatur \cite{suchen}. Die Herausforderungen im bestehenden System formeller wissenschaftlicher Kommunikation beziehen sich dabei vor allem auf zehn Bereiche:
\begin{enumerate}
\item Leistungsbewertung und Qualitätssicherung
\item Geschwindigkeit
\item Freiheit von Wissenschaft und Forschung
\item Kosten und Effizienz
\item Fehlerresistenz
\item Verbreitung und Zugänglichkeit
\item Digitalisierung
\item Reliabilität und Validität
\item Objektivität und Unabhängigkeit
\item Missbrauch
\end{enumerate}

Diese Bereiche dienen zur Einordung der Herausforderungen und zur Eingrenzung der Debatten um das aktuelle System der wissenschaftlichen Kommunikation. Als eine Möglichkeit dem folgenden Herausforderungen im wissenschaftlichen Kommunikations- und Reputationssystem zu begegnen, wird ein größtmögliches Maß an Transparenz und Offenheit betrachtet.

\subsection{Leistungsbewertung und Qualität in der Wissenschaft}

Die Verlage haben in den letzten Dekaden mit den wissenschaftlichen Journalen ein zentrales Steuerungs- und Bewertungssystem in der Wissenschaft etablieren können. In diesem System werden die Grundprinzipien der Wissenschaft für die verlegerischen Verwertungsinteressen (aus)genutzt und das, obwohl diese "wissenschaftlichen Grundprinzipien und Normen eigentlich ökonomischen Verwertungsinteressen zu widersprechen scheinen" \cite{hanekop_2006}. Darüber agieren die Forscherinnen und Forscher in einem Umfeld, in dem sie in vielen Fällen wenig oder keine Verantwortung für den Einkauf der wissenschaftlichen Informationen haben, die er oder sie im Rahmen der Veröffentlichung "verschenkt" \cite{steele_2006}.

Die Einführung der quantitativer Bewertungsindikatoren wie das Zitationsregister und die Impact Faktoren, sowie die Definition der Kernzeitschriften, führte zu einer weitgehenden Erstarrung des wissenschaftliche Zeitschriftenmarktes und gleichzeitig zu einem Anstieg der Kapazität der kommerziellen Verlagen, sowie deren Gewinnmargen \cite{CREATe_2014}. Die Steuerungsmechanismen werden über die Messbarkeit mittels Methoden direkt oder indirekt ausgeübt. Dabei stehen insbesondere die Methoden, die auf der quantitativen Grundlage der Zitationsraten wissenschaftlicher Publikationen gemessen werden in der Kritik \cite{Brembs_2013} \cite{Dong_2005} und auch andere Indikatoren für die Messung von Forschungsleistungen sind hoch umstritten \cite{Hornbostel_1997} \cite{Hicks_1996} \cite{Havemann_2002} \cite{Warnke_2012}. Die Verfahren, um die Wirkung von Wissenschaft und damit auch die Reputation von Wissenschaftlern zu messen, sind kein eigentliches wissenschaftliches Produkt \cite{suchen} und erfassen zum Beispiel die Tätigkeit einzelner Forschergruppen zu stark \cite{schmoch_2009}. Darüber hinaus sind "weder importance noch impact noch quality direkt messbar" und man kann sich ihnen nur "nähern" \cite{Hornbostel_1997}. Das führt unter anderem dazu, dass der aus der "Zahl der Zitationen auch die Beiträge einer Zeitschrift ermittelte" \cite{weishaupt_2009_goldenOA} Impact Factor nicht als perfektes Werkzeug betrachtet werden kann, um die Qualität der Artikel zu messen \cite{garfield_1999} und "selbst die grundlegendsten wissenschaftlichen Standards verletzt" \cite{Brembs_20013}. Trotzdem wird er zur Bewertung von Wissenschaft genutzt, denn “es gibt nichts Besseres" und er hat den Vorteil, dass er allein durch seine lange Existenz "eine gute Technik für die wissenschaftliche Bewertung” darstellt \cite{garfield_1999} \cite{weishaupt_2009_goldenOA}.

Die Kritik am Impact Faktor lässt sich laut der Bibliotheks- und Informationswissenschaftlerin Dr. Karin Weishaupt, am Beispiel des "Thomson Reuters Journal Citation Factors" in sechs Punkten zusammenfassen \cite{weishaupt_2009_goldenOA}:
\begin{enumerate}
\item Der Impact Factor bezieht sich immer auf die gesamte Zeitschrift und hat somit keine Aussagekraft über die "Rezeption oder Qualität des einzelnen Artikels" \cite{weishaupt_2009_goldenOA}.
\item Der Impact Factor berücksichtigt nur die Zeitschriften, die im eigenen Index gelistet sind und enthält weder Monographien, Tagungsbeiträge, sonstige Beiträge oder Internetquellen.
\item Durch Selbstzitierungen sind Manipulationen möglich.
\item Es werden nur Zitate aus den letzten beiden Jahren berücksichtigt und je nach Fachgebiet ist es von Vorteil wenn im eigenen Gebiet die Verwertungszyklen kürzer sind.
\item Publikationen, die nicht in englischer Sprache verfasst sind, weisen meist eine geringere Sichtbarkeit und Popularität auf, da englische Journale überproportional vertreten sind
\item Spezialisierte Zeitschriften sind ebenfalls systematisch benachteiligt gegenüber Journalen großer Fach-Communities oder Journalen mit Übersichtsartikeln.
\end{enumerate}

Es bleibt festzuhalten, dass die im wissenschaftlichen System genutzten Indikatoren die komplexe Realität der Leistungsbewertung in der Wissenschaft nicht abbilden können und sie eine eigene Realität konstruieren \cite{Hornbostel_1997}. Versteht man Wissenschaft als soziales System, so stellen Reputation und nicht die Wahrheit der Beobachtungen und Erklärungen "nicht selten auch eingestandenes vorrangiges Ziel wissenschaftlicher Tätigkeit" dar \cite{luhmann_1970_selbststeuerung}. Wie gering der Wirkungsgrad und die Methoden zur Messung “zur Reproduktion des traditionellen wissenschaftlichen Diskurses ausfallen, wird von dem Moment an klar, an dem ein neues und offenes Kommunikationsmedium wie das Internet als alternativer Publikations- und Verbreitungskanal für Wissenschaft zur Verfügung steht" \cite{Rost_1998}.

\subsubsection{Indikatoren für Reputationsverteilung im wissenschaftlichen Kommunikationsystem}

Die wissenschaftliche Publikation ist wesentlich für die Funktion der Reputationsverteilung in dem wissenschaftlichen Kommunikationsystem \cite{hirschauer2004peer}. Folgende Indikatoren werden für die Verteilung von Reputation in der Literatur aufgeführt. Die Kategorisierung ist angelehnt an Heidemarie Hanekop \cite{hanekop_2008} und die Befragung durch das SOFI 2007 \cite{Hanekop_Wittke_2007_Fragebogen} :
\begin{enumerate}
\item \textbf{Anzahl der wissenschaftlichen Aufsätze / Beiträge}: Die Anzahl der Texte die Wissenschaftler im Rahmen ihrer Tätigkeit publizieren ist ein wesentlicher Faktor der Bewertung wissenschaftlicher Reputation \cite{Warnke_2012} \cite{CLAPHAM_2005} \cite{luhmann_1970_selbststeuerung}. Zum Beispiel erhöht die Anzahl an Texten die Chance durch die wissenschaftliche Community zitiert zu werden und damit die Möglichkeit auf die Erlangung von Reputation. Durch den zunehmenden Wettbewerb in der Wissenschaft muss sich der einzelne Wissenschaftler entscheiden, "zu publizieren oder im wissenschaftlichen System zu scheitern" \cite{Suess_2006}. Dadurch entsteht im wissenschaftlichen Kommunikationssystem ein konstanter Publikationsdruck, bei dem die Relevanz der publizierten Ergebnisse nicht immer im Vordergrund steht \cite{hamilton_1990_publishing}. Die Anzahl der veröffentlichten Artikel hat einen Einfluss auf die Vergabe von Ressourcen und finanziellen Mittel für weitere Forschung an Institutionen und Individuen \cite{Warnke_2012} \cite{hamilton_1990_publishing}.
\item \textbf{Relevanz der publizierten Ergebnisse}: Die Relevanz der publizierten Ergebnisse ist für das Wissenschaftssystem ein wesentlicher Treiber für den Prozess der Wissensgewinnung. Relevante Erkenntnisse sind die Grundlage für die Produktion von neuem Wissen und damit Grundlage für den gesellschaftlichen Auftrag des Wissenschaftssystems \cite{hanekop_2008}. Das wissenschaftliche System beruht auf der Annahme, dass die Relevanz der publizierten Ergebnisse einen direkten Einfluss auf die wissenschaftliche Reputation hat.
\item \textbf{Anzahl Monografien}: Die Anzahl der veröffentlichten Monographien ist ein wesentlicher Reputationsfaktor. Das gilt für die Disziplinen, in denen diese Publikationsform wichtig ist, wie den Geistes- und Sozialwissenschaften. In den anderen wissenschaftlichen Fachrichtungen spielt die Anzahl der Veröffentlichungen von Artikeln in wissenschaftlichen Journalen eine wichtige Rolle.
\item \textbf{Drittmittelprojekte}: Drittmittel sind, so der deutsche Wissenschaftsrat, "solche Mittel, die zur Förderung der Forschung und Entwicklung sowie des wissenschaftlichen Nachwuchses und der Lehre zusätzlich zum regulären Hochschulhaushalt (Grundausstattung) von öffentlichen oder privaten Stellen eingeworben werden" \cite{wr_2014}. Die Drittmitteleinwerbung hat sich in Deutschland als "meist gebrauchter Maßstab der Messung von Forschungsqualität durchgesetzt" \cite{M_nch_2006}. Diese Entwicklung geht mit einer zunehmenden Finanzierung der Forschung über Drittmittel einher \cite{Neidhardt_2010} \cite{Jansen_2007} \cite{simon_2009_wissenschaft_governance}. Durch die zunehmende Knappheit öffentlicher Ressourcen für Wissenschaft und Forschung, ist die Akquise von Drittmitteln zu einem kritisch zu betrachtenden Kernziel geworden \cite{Jansen_2007}. Das führt, dass zunehmend direkte finanzielle und administrative Kontrolle der Forschung eine Rolle spielen \cite{Barl_sius_2008}. Dabei spielt die Frage eine Rolle, ob die Publikationen, die im Rahmen der Drittmittelfinanzierung als wissenschaftliche Erkenntnisse veröffentlicht werden und ob der Antrag um Drittmitteleinwerbung selbst, "zum Erkenntnisfortschritt in der wissenschaftlichen Gemeinschaft beiträgt" \cite{M_nch_2006}. Die wissenschaftliche Community befürchtet durch die zunehmende Relevanz der Anzahl von Drittmittelprojekten bei der Erlangung von wissenschaftlicher Reputation eine Einschränkung der Freiheit von Wissenschaft und Forschung.
\item \textbf{Patente}: "Unter einem Patent versteht man das vom Staat verliehene Schutzrecht für eine technische Erfindung, welches dem Patentinhaber für eine bestimmte Zeit die ausschließliche wirtschaftliche Nutzung der Erfindung vorbehält." \cite{greif_2003_patente} Die Anzahl dieser Schutzrechte im Hochschulbereich nimmt seit den 1970er konstant zu. \cite{schmoch_2003_hochschulforschung} \cite{Fabrizio_2008}. Vor allem in den technischen Fachdisziplinen wird eine Patentschrift "als funktionales Äquivalent zur wissenschaftlichen Publikation begriffen" und bewertet \cite{mersch_2014_patente}. Die deutsche Hochschulrektorenkonferenz hält fasst die Rolle des Patentwesen an den Hochschulen wie folgt zusammen: "Patente leisten einen Beitrag zur Förderung der Wissenschaft, die Grundlagen des Patentwesens sind daher dem wissenschaftlichen Nachwuchs über entsprechende Lehrangebote zu vermitteln." \cite{suchen-Position-HRK} Die Befürchtung, dass Patente einen negativen Effekt auf die Erstellung und Veröffentlichung von fundamentaler Forschungsergebnisse hat, konnte nicht abschließend bestätigt werden \cite{Fabrizio_2008}.
\item \textbf{Vorträge}: Vorträge dienen der Verbreitung der Forschungserkenntnisse, sowie Zwischenständen und ermöglichen das Vermitteln des Wissens an andere \cite{rassenhoevel_2010_performancemessung}. Vorträge stellen eine informelle und schnelle Form für die Verbreitung neuer wissenschaftlicher Erkenntnisse und Ergebnisse dar. Die in einem Vortrag vermittelten Inhalten müssen meist nicht genauer belegt werden und die kommunizierten Inhalte lassen gegebenenfalls später schriftlich konkretisieren oder korrigieren \cite{haberle_2002_jahrbuch}. Vorträge bieten die Möglichkeit bereits vor der eigentlichen Publikation von wissenschaftlichen Erkenntnissen Anregungen und Reaktion einzuholen.
\item\textbf{Anwendungsrelevanz bzw. Verwertbarkeit}: Ein vergleichsweise neuer Indikator die Reputation von Hochschulen und außeruniversitäre Forschungsinstitute ist die Anwendungsrelevanz von Wissenschaft und Forschung \cite{simon_2009_wissenschaft_governance}. Sie bezieht sich auf einen Outputfaktor, der sich primär auf den Einsatz der gewonnenen wissenschaftlichen Erkenntnisse und auf die Verwertbarkeit für wirtschaftliche Produkte oder Patente als auf die eigentliche wissenschaftliche Veröffentlichung abzielt \cite{suchen}.
\item \textbf{Netzwerke und Kontakte}: Netzwerke beschreiben formelle und informelle Verbundsysteme zwischen Wissenschaftlern. Sie erlauben den schnellen Austausch und können Grundlage für Aktivitäten zur Steigerung der wissenschaftlichen Reputation darstellen. Diese Aktivitäten umfassen zum Beispiel gemeinsame Publikationsvorhaben und den Austausch wissenschaftlicher Erkenntnisse. Kontakte und Netzwerke schaffen soziale Beziehungen, die für eine erfolgreiche Integration an der Hochschule und der Fachcommunity sorgen, Zugang zu wissenschaftlicher Kommunikation ermöglichen und somit einen Einfluss auf die Anerkennung eines Wissenschaftler oder einer Wissenschaftlerin haben können.
\item \textbf{Öffentliche Aufmerksamkeit}: Die öffentliche Aufmerksamkeit stellt zum einen eine Möglichkeit des Wissenstransfers außerhalb der wissenschaftlichen (Fach-)Community dar, zum Anderen ermöglicht sie die Einflussnahme auf die politische Relevanz wissenschaftlicher Forschungsthemen. Die Veröffentlichung von wissenschaftlichen Informationen zu einem bestimmten Thema des öffentlichen Interesses stellt eine Möglichkeit dar, dieses Thema öffentlichkeitswirksam zu katalysieren. Öffentliche Aufmerksamkeit im Rahmen von wissenschaftlicher Tätigkeit stellen eine kritisch zu hinterfragende Möglichkeit für die alternative Ressourcengewinnung dar. \cite{suche}
\item \textbf{Politische Relevanz}: Die wissenschaftliche Tätigkeit mit politischer Relevanz stellt eine weitere Möglichkeit dar, wissenschaftliche Inhalte ausserhalb der Wissenschaft anwendbar zu machen und führt zu Anerkennung der wissenschaftlichen Arbeit. Daraus ergeben sich allerdings grundsätzliche "Verständigungsprobleme und Interessenkonflikte", da  "Wissenschaft und Politik aufgrund unterschiedlicher Rationalitäten handeln, einander aber zugleich brauchen" \cite{Mayntz_1996}. Während es im Wissenschaftssystem "um Erwerb und Erhalt von Wissen" geht, zielt die Politik auf "Erwerb und Erhalt von Macht" \cite{Mayntz_1996} ab. Die daraus resultierenden Interessenkonflikte können die Legitimität der Wissenschaft beeinträchtigen \cite{weingart_2005_wissenschaft} führen zu "gegenseitigen Enttäuschungen", vor allem in der "forschungspolitischen Beziehung" \cite{Mayntz_1996}.
\item \textbf{Renommee der Forschungseinrichtung}: Das Renommee einer Forschungseinrichtung ist die Wahrnehmung der Einrichtung innerhalb und außerhalb der wissenschaftlichen (Fach-)Community. Sie hat für Wissenschaftler und die Wissenschaftlerin eine besondere Bedeutung \cite{mayntz_2008_wissensproduktion}. Sie basiert auf dem Konzept der "Ansteckung" \cite{luhmann_1970_selbststeuerung}. Diese Ansteckung führt zum Beispiel dazu, dass renommierte Professoren den Ruf einer Fakultät und eine renommierte Fakultät auch den Ruf von Professoren aufbessern können. Übertragen auf das wissenschaftliche Publizieren profitiert ein Autor oder eine Autorin bei der "Ansteckung" von dem Renommee einer Einrichtung, wenn er durch die Publikationsorgane der renommierten Institution veröffentlicht \cite{lutz_2012_zugang}.
\item \textbf{Renommee von Herausgebern oder Mitautoren} Der Herausgeber organisiert den Begutachtungsprozess und sichert bestimmte Qualitätskriterien mit seiner Reputation und seinem Namen \cite{mueller_2009_peerreview}. Auch hier kommt es im Rahmen des symbolischen wissenschaftlichen Kapitals zu einer Übertragung der Reputation der Herausgeber oder Mitautoren auf die anderen veröffentlichenden Autoren.
\item \textbf{personelle und materielle Ausstattung}: Die materielle Ausstattung beschreibt die Rahmenbedingungen, in der ein Wissenschaftler arbeitet. Diese Rahmenbedingungen haben eine herausragende Bedeutung bei der Entscheidung über einen Wirkungsort von Wissenschaftlern \cite{mayntz_2008_wissensproduktion}. Insbesondere die materielle und personelle Ausstattung sind bei traditionellen Berufungsverfahren deutscher Professorinnen und Professoren von besonderem Belang \cite{himpele_2011_job}, da sie die Arbeitsfähigkeit und die Anerkennung direkt beeinflussen \cite{suche}. Wie die materielle Ausstattung gilt auch die personelle Ausstattung als ein reputationstiftendes Merkmal für Wissenschaftler und die Institution, an denen sie arbeiten \cite{mayntz_2008_wissensproduktion}. Bei der Ausstattung handelt es sich um einen bilateralen Indikator, der zum einen aus der Bewertung der wissenschaftlichen Arbeit (im Rahmen der Forschungsförderung) resultiert \cite{Herb_vermessung_2008} und  zum anderen Reputation innerhalb der Community schafft \cite{mayntz_2008_wissensproduktion}.
\item \textbf{Gutachtertätigkeit und Herausgeberschaft}: Gutachter werden zum Beispiel in Peer-Review-Verfahren Autoren des entsprechenden Fachgebietes zugeordnet und entscheiden über die Veröffentlichung des Textes \cite{Frey_2005}. Bei manchen Publikationen wird ein Text mehrmals abgelehnt und eine Überarbeitung durch den Autoren eingefordert, bevor der Artikel final akzeptiert und daraufhin publiziert wird \cite{Frey_2005}. In diesem Zusammenhang wirkt sich die Reputation der mit diesem Verfahren betrauten Gutachter auch auf das Image des Verlages aus und umgekehrt. Die Gutachtertätigkeit ist aber nicht nur Kernbestandteil des wissenschaftlichen Qualitätssicherungs- und interdependenten Reputationssystems, sondern stellt auch einen informellen Weg der Kommunikation dar. Er ermöglicht den Gutachtern die Vorabsichtung neuster wissenschaftlicher Informationen und Erkenntnisse. Ähnlich wie die Gutachtertätigkeit ist auch die Herausgeberschaft fester Bestandteil des interdependenten wissenschaftlichen Reputationssystems \cite{Frey_2005}: Herausgeber profitieren von den publizierten Inhalten und Erkenntnissen der Autoren, Autoren von der Reputation Herausgebern und der Verlag von beiden \cite{suchen}.
\item \textbf{Funktion}: Die jeweilige Funktion oder die (universitäre) Stellenbezeichnung ist ein weiter Faktor für wissenschaftliche Reputation. Zum wissenschaftlichen Personal zählen Professoren, Juniorprofessoren, wissenschaftliche und künstlerische Mitarbeiter, sowie Lehrkräfte \cite{erhardt_2011_hochschulen}. Eine Weiterentwicklung und der "Aufstieg" in der wissenschaftlichen Hierarchie zielt auf das akademische Streben nach einer Professur \cite{Klecha_2008}.
\item \textbf{Awards und Preise}: Preise sind ein weitere Indikator für das wissenschaftliche Belohnungs- und Bewertungssystem. "Die Praxis der Award-Verleihung beruht auf dem Konzept, dass Ressourcen von unabhängigen Dritten auf Qualität geprüft und (...) zertifiziert werden" \cite{bargheer_2002_qualitatskriterien}. Wissenschaftler und Wissenschaftlerinnen, die Preise oder Awards gewinnen, erfahren Anerkennung. Diese Anerkennung können jedoch nicht automatisch als "Garant für wissenschaftsrelevante Qualität"\cite{bargheer_2002_qualitatskriterien} verstanden werden. Die Ehrung mit einem Preis weckt große Erwartungen und führt zu dem Anspruch eines stetigen Nachschubs an Anerkennung für den Wissenschftler oder die Wissenschaftlerin \cite{suchen}.
\end{enumerate}

\subsection{Geschwindigkeit}

Einen weiterer Aspekt der Debatte betrifft die Kritik an der zeitliche Geschwindigkeit zwischen der Fertigstellung einer wissenschaftlichen Arbeit durch den Autoren und der finalen Veröffentlichung der Ergebnisse. Trotz der Beschleunigung der Prozesse bei der Qualitätssicherung und Bewertung von wissenschaftlichen Arbeiten durch die Digitalisierung der Kommunikation zwischen Wissenschaftlern, Gutachtern und Verlagen kann es bis zu mehrere Jahre dauern, bevor ein Text veröffentlicht wird \cite{suchen}. Diese Verzögerung beruht unter anderem auf folgenden Umständen:

\begin{enumerate}
\item Gutachter/innen können aufgrund der Ausführung dieser Funktion als Nebentätigkeit meist Termine nicht einhalten \cite{bar_2009_wissenschaftliche}.
\item Es gibt weder Anreiz- noch Sanktionsmöglichkeiten für Gutacher und Gutachterinnen.
\item Die wissenschaftlichen Zeitschriften erscheinen größtenteils noch immer als Periodika und wissenschaftlichen Bücher orientieren sich am Druck. Sie sind damit für einen bestimmten Zeitraum der Veröffentlichung terminiert.
\end{enumerate}

Eine Möglichkeit die wissenschaftlichen Inhalte schneller zugängnlich zu machen, ohne den sehr zeitaufwändigen Begutachtungsprozess strukturell oder inhaltlich zu verändern, ist die Veröffentlichung der wissenschaftlichen Arbeit als digitalen Pre-Print. Ergänzend stellt die offene Begutachtung (Open Peer Commentary) eine Möglichkeit dar, bei der ein Text anonymisiert (vorab) veröffentlicht und kommuniziert, sowie von der wissenschaftlichen Gemeinschaft kollaborativ bewertet wird \cite{mueller_2009_peerreview}. Dabei darf der Wunsch nach einer erhöhten Geschwindigkeit nicht über dem Wunsch nach einem ausgewogenen Qualitätssicherungsprozess gestellt werden \cite{Beall_2012}.

\subsection{Wahrung der Freiheit von Wissenschaft und Forschung}

---- TODO: Ausarbeiten und von oben übernehmen ----

\subsection{Kosten und Effizienz}

An dem Kosten-Nutzen-Verhältnis des aktuellen wissenschaftlichen Kommunikationssystems gibt es seit Jahren detaillierte und grundsätzliche Zweifel. Für die Veröffentlichung einzelner Texte ergeben sich je nach Schätzungen bis zu xxxx Dollar pro veröffentlichten Text und xxxx Dollar pro veröffentlichtem Buch.

Mit Beginn der Verbreitung elektronischer Publikationen kam es zu einer Umkehr des Bring- zum Holprinzip bei der Verbreitung wissenschaftlicher Publikationen und die Erwartungen an die neuen Kanäle richten sich vor allem darauf, mit elektronischen Publikationen die  Publikationszyklen kostengünstiger und effizienter zu machen \cite{Brueggemann-Klein_1995}. Die Vermutung Ende der 1990er Jahre: "Einsparungen in Zeit, Raum und Kosten werden erheblich sein, wenn zunehmend Schreib- und Publikationstätigkeiten in den elektronischen Raum verlegt werden" \cite{roberts_1999_scholarly}.

Daten und Informationen hinter Publikationen stehen zur selten offen zur Verfügung, dabei wird eine Effizienzsteigerung durch die Zweitnutzung und Weiterverwendung von Daten die während des wissenschaftlichen Erkenntnissprozesses vermutet, die sofort eintritt, sobald ein Datensatz wiederverwendet wird \cite{RIN_2010_open_research}.

Der restriktiven Umgang mit Daten im aktuelle System verhindert somit die Möglichkeit der kollaborativen und zusätzlichen Forschungsergebnissen, bessere Bildung, neue Möglichkeiten und Nutzungsszenarien und eine umfassendere Aufzeichnung, Evaluation und Dartstellung von Wissenschaft.

\subsection{Fehlerresistenz}

Damit der Erkenntnisfortschritt im Kommunikationsprozess gelingt braucht es Verlässlichkeit bei der Vermeidung von Fehlern im wissenschaftlichen Erkenntnisprozess \cite{Bargheer_2015}. Trotz des aufwändigen wissenschaftlichen Qualitätssicherungssystems kommt es immer wieder zu Fehlern und falschen Aussagen bei der Veröffentlichung wissenschaftlicher Erkenntnisse und Ergebnisse \cite{brembs2015open} \cite{Luescher_2014}. Die Gründe für diese Fehler sind vielfältig und erstrecken sich von  Nachlässigkeit über Fahrlässigkeit bis hin zu Vorsatz.

In der Literatur werden insbesondere folgende Herausforderungen bei der Absicherung der Fehlerressistenz genannt:
\begin{enumerate}
\item Geschlossene Begutachtungsverfahren ermöglichen nur eine kleinen Anzahl an Gutachtern wissenschaftliche Inhalte auf Fehler zu prüfen \cite{suchen}
\item Nichtverfügbare Methoden und Daten hinter den Publikationen behindern die Qualitätssicherung und Reproduzierbarkeit von Wissen \cite{suchen}
\item Nicht dokumentierte und veröffentlichte Kommunikation während des wissenschaftlichen Wertschöpfungsprozesses, macht es unmöglich Fehler bereits bei der Erstellung der Publikation sichtbar und transparent nachvollziehbar zu machen \cite{suchen}
\end{enumerate}

\subsection{Verbreitung und Zugänglichkeit}

Ebenso wie die Frage nach der optimalen Geschwindigkeit des aktuellen wissenschaftlichen Kommunikationssystems, stellt sich auch die Frage nach der optimalen Verbreitung und Zugänglichkeit von wissenschaftlichen Informationen. Während die Geschwindigkeit auf die zeitliche Komponente von der Herstellung bis zum Vertrieb des Wissens abzielt, geht es bei der Frage nach Verbreitung um die Verfügbarkeit des Wissens für eine möglichst große Rezipientengruppe. Es gibt erhebliche Zweifel daran, ob es sich bei dem aktuellen System um ein System mit optimalen Voraussetzungen für eine möglichst hohe Verbreitung von Wissen an die Gesamtgesellschaft handelt \cite{suchen}.

Analoge Publikationen und Verbreitungswege sowie Nutzungsmöglichkeiten nutzen nicht die Möglichkeiten und die Vorteile des grenzüberschreitenden Austauschs über digitalen Infrastrukturen. Die Verbreitung der Informationen wird zudem durch Bezahlschranken gehemmt und die Zirkulation von Wissen eingeschränkt.

\subsection{Digitalisierung}

Die Digitalisierung der wissenschaftlichen Kommunikation beschränkt sich bisher in vielen Fällen noch immer darauf, dass die analog gedruckten und bewährten Journale, sowie andere Publikationsformen der großen wissenschaftlichen Verlage mit nahezu unverändertem Geschäftsmodell digital verbreitet werden \cite{Hanekop_2014} \cite[:179]{Fehling_2014}. Die digitale Distribution wird in diesem Zusammenhang als weiterer Kanal nach dem Drucken der Informationen verstanden. Die Möglichkeiten, die die Digitalisierung bietet, sind damit bei weitem nicht ausgeschöpft.

Während Wissenschaftler und Wissenschaftlerinnen schon seit Ende des letzten Jahrhunderts überwiegend mit Hilfe von Textsystemen schreiben \cite{Brueggemann-Klein_1995} \cite{bjork_2004_open} haben Verlage erst mit großer Verzögerung auf die elektronische Produktion von Wissen reagiert.

Die neuen Möglichkeiten einer umfassenderen Kommunikation von wissenschaftlichen Erkenntnissen werden bei der Veröffentlichung nur selten genutzt. Darüber hinaus arbeitet das derzeitige System der wissenschaftlichen Veröffentlichungen noch immer gegen maximale Verbreitung der wissenschaftlichen Daten hinter den eigentlichen Publikationen \cite{Molloy_2011}.

---- TODO: weiter anhand von Literatur ausarbeiten ----

\subsection{Reliabilität und Validität}

Die Zuverlässigkeit des Kommunikationssystems kann anhand dessen geprüft werden, ob die Einreichung einer Arbeit über unterschiedliche Wege den selben Erfolg hat beziehungsweise, wie stark Zufallsfaktoren den Erfolg der Einreichung beeinflussen.

In dem aktuellen System wird die Replizierbarkeit und Zuverlässigkeit von Ergebnissen stark kritisiert und bezweifelt \cite{Luescher_2014}. Das liegt zum einen an den Herausforderungen im Zusammenhang mit der Dissemination von Daten, zum anderen an der Verwendung von geschlossenen Systemen und Formaten.

Als weitere Faktoren werden unter anderem Lücken im Qualitässicherungsprozess \cite{bar_2009_wissenschaftliche} und der zunehmende zeitliche Druck im Rahmen der Qualitätssicherung genannt \cite{Luescher_2014}.

\subsubsection{Objektivität und Unabhängigkeit}

Objektivität in der Wissenschaft gilt für die Sammlung, Aufzeichnung, Analyse, Interpretation, gemeinsame Nutzung und Speicherung von Daten, sowie andere wichtige Verfahren in der Wissenschaft, wie zum Beispiel die Veröffentlichungspraxis und Peer-Review \cite{resnik_2005_ethics}.

Die Kenntnis von Eigenschaften der Autoren durch die Gutachter stellt eine der größten Herausforderungen für die Wahrung der Objektivität und Unabhängigkeit im wissenschaftlichen Qualitätssicherungsprozess dar. Aber auch bei anderen Formen der wissenschaftlichen Bewertung können Unabhängigkeit und Objektivität nicht gewährleistet werden. In der Literatur finden sich Beiträge, die mehrheitlich zu dem Ergebnis kommen, dass die Objektivität und Unabhängigkeit im bestehenden System nur schwer bis nicht gesichert werden können \cite{binswanger_2014_excellence}.

Resnik beschreibt die Herausforderungen an die Objektivität und an das selbst korrigierende System der Wissenschaft  in folgende Kategorien \cite{resnik_2005_ethics}.:
\begin{enumerate}
\item Präzision der wissenschaftlichen Arbeit
\item Erhlichkeit bei der Datenerhebung und Darstellung der Ergebnisse
\item Vermeidung von Fehlverhalten
\item Vermeidung von Fehlern und Selbsttäuschung
\item Offenlegung Interessenskonflikte
\item Offenheit bezüglich Daten, Ideen, Theorien und  Ergebnissen
\item bewusstes Datenmanagement und Dokumentation
\end{enumerate}

\subsection{Missbrauch und wissenschaftliches Fehlverhalten}

Die ethischen Grundsätze stellen in der wissenschaftlichen Debatte von Beginn an eine Besonderheit dar. Vertrauen, das Interesse aller Akteure an optimaler Kommunikation zwischen den Wissenschaftlern, Ehrlichkeit und der Ausschluss von Interessenskonflikten sind Grundpfeiler im wissenschaftlichen Wertschöpfungs- und Kommunikationsprozess \cite{Bargheer_2015}. "Betrug ist dabei zwingend an die Absicht zu täuschen gebunden" \cite{Luescher_2014}.

Es muss das Anliegen jedes Forschers sein, "die Wahrheit und nichts als die Wahrheit zu suchen und zu berichten" \cite{Luescher_2014}. "Ohne Vertrauen in die Erhlichkeit von Forschern gäbe es keine Wissenschaft mehr" \cite{hagner_2015_sache_buches}. Vertrauen und Redlichkeit bilden die Grundlage der Wissenschaft \cite{Bargheer_2015} auch wenn diese auf einer "delikaten Struktur weitgehend ungeschriebenen Regeln" \cite{grand_2012_open} beruhen.

Diesem wissenschaftlichem Ethos stehen die Beispiele gegenüber, bei denen bewusster Missbrauch durch Akteure des Kommunikationssystems zu Verwirklichung partikularer Interessen oder konkreten Einfluss auf wirtschaftliche Aspekte geführt haben \cite{Luescher_2014}  \cite{binswanger_2014_excellence} \cite{Beall_2012}.

Margo Bargheer und Birgit Schmidt klassifizieren wissenschaftliches Fehlverhalten wie folgt \cite{Bargheer_2015}:
\begin{enumerate}
\item Unlauterer Umgang mit Ergebnissen (z.B. erfundene Ergebnisse)
\item Unlauteres Forschungsverhalten (z.B. Unzulässige Forschungsmethoden)
\item Fehlverhalten im Datenmanagement (z.B. Zurückhalten von Daten wider besseres Wissen )
\item Fehlverhalten im Publikationsprozess (z.B. Unangemessene Partitionierung von Ergebnissen „Salamitaktik“  \cite{binswanger_2014_excellence})
\item Soziales Fehlverhalten (z.B. Sabotage oder Behinderung der Arbeit Anderer )
\item Administratives Fehlverhalten (z.B. Verstoß gegen Verwendungsrichtlinien)
\end{enumerate}

Gegen ein solches Fehlverhalten im Rahmen der wissenschaftlichen Kommunikation wurden die internationalen Leitlinien "Principles of Transparency and Best Practice in Scholarly Publishing" \cite{oaspa_principles_2013} veröffentlicht, "sie sollen die Qualitätsstandards im Publikationswesen und zugleich die Filterfunktion der initiierenden Mitgliedsorganisationen stärken" \cite{Bargheer_2015}.

Auch wenn die noch nie zuvor über Betrug in der Wissenschaft so intensiv berichtet worden ist \cite{brembs2015open} wie in den letzten Jahren, ist es "keineswegs ausgemacht, dass die Intensität der Berichterstattung allein auf die tatsächlich gestiegene Inzidenz von Betrug" \cite{weingart_2005_wissenschaft}, sondern eher auf den Anstieg medialer Beobachtung zurückzuführen ist.

\section{Anknüpfungspunkte zur Forderung nach Öffnung der wissenschaftlichen Kommunikation}

Die etablierten Prozesse wissenschaftlicher Kommunikation stehen also vor umfangreichen Herausforderungen. Die Zeitschriften- und Monographienkrise, der zunehmende finanzielle Druck im Rahmen der öffentlichen Finanzierung von Wissenschaft, die Veränderungen im wissenschaftlichen Kommunikationsprozess durch neue Arten und Möglichkeiten der Distribution, die steigenden Beschaffungskosten für wissenschaftliche Literatur \cite{cite:17} \cite{muller_2010_open}, sowie die Veränderungen in der Rezeption von Inhalten \cite{holub_2013_reception} zwingen zum Umdenken in der wissenschaftlichen Kommunikationspraxis \cite{suchen}. Die anhaltende Forderung nach mehr Offenheit im wissenschaftlichen Kommunikationsprozess entwickelte sich zu einem konkreten Lösungsansatz für die Herausforderungen an das etablierte System. Nachfolgend wird die Debatte um das Modell des Offenen Zugangs zu Wissenschaft analysiert.

Dieser Zugang beruht auf der Annahme, dass die Öffnung der wissenschaftlichen Kommunikation eine große Chance für Veränderungen im wissenschaftlichen Qualitäts- und Reputationssystem darstellt. Diese Chancen beziehen sich auf die Aktivität der Wissenschaftler und die Qualität der Forschungsergebnisse. Die wissenschaftlichen Erkenntnisse werden bisher häufig erst nach langen intransparenten Verfahren bewertet, publiziert und nur an einen beschränkten Kreis von Rezipienten vermittelt. Diese intransparente Praxis hat einen signifikant-negativen Einfluss für Allokation von Ressourcen durch die öffentliche Hand und die Kosten die im Rahmen öffentlich-finanzierter Forschung entstehen \cite{suchen}.

Als Auslöser für die Entwicklung von Open Access werden auch die infrastrukturellen Veränderungen angeführt, die "seit spätestens Mitte der 1990er-Jahre entscheidend Einfluss auch auf die Wissenschaftskommunikation und das wissenschaftliche Arbeiten genommen haben" \cite{schulze_2013_open}. Open Access entwickelte sich vorerst primär in den STM-Fächern, in denen viel Aufmerksamkeit auf der Selbstarchivierung der Wissenschaftler und Wissenschaftlerinnen vor der finalen Publikation (Pre-Print) in privaten, zentralen oder institutionellen Repositorien lag \cite{adema_2013_political} und bei denen die Auswirkungen der Zeitschriftenkrise am stärksten zum Tragen kam \cite{naeder_2010_open}. Wissenschaftliche Informationen werden seither nicht nur in "digitaler Form konsumiert, sondern auch kollaborativ und kooperativ, zeitlich versetzt, durch teilweise räumlich weit verstreute Arbeitsgruppen und Forschungsverbünde, genutzt und weiterverarbeitet" \cite{schulze_2013_open}. Die Verbreitung und Akzeptanz von Open Access variiert dabei zwischen den einzelnen wissenschaftlichen Disziplinen erheblich \cite{cite:21a}.

Bei der Betrachtung der Fordernung nach Öffnung wissenschaftlicher Kommunikation muss allerdings zwischen den Konzepten von Open Access und Open Science unterschieden werden. Bei Open Access geht es um einen möglichst uneingeschränkte Zugang zu finalen wissenschaftlichen Ergebnispublikationen für die Gesamtgesellschaft. Open Science beschreibt hingegen den umfassenden Zugriff auf den gesamten wissenschaftlichen Wertschöpfungsprozess inklusive aller Daten und Informationen, die bereits bei der Erstellung, Bewertung und Kommunikation der wissenschaftlichen Erkenntnisse entstanden sind.

\subsection{Offener Zugang zur wissenschaftlichen Kommunikation: Open Access}

Open Access wird von Peter Suber "zugespitzt" \cite{naeder_2010_open} als "digital, online, kostenlos, und frei von den meisten Urheber- und Lizenzbeschränkungen" \cite{suber_2012_open} definiert \cite{Adema_2014_open_access}. Open Access bedeutet den "Verzicht auf die finanzielle, technische und rechtliche Hindernisse, die dazu bestimmt sind, den Zugang zu wissenschaftlichen Forschungsartikel für zahlende Kunden zu begrenzen" und dass, "im Interesse der Beschleunigung der Forschung und den Austausch von Wissen, Verlage ihre Kosten aus anderen Quellen schöpfen" \cite{Suber_2002}. Die meisten programmatischen Erklärungsversuche sehen Open Access demnach "als adäquate Selbsthilfe wissenschaftlicher Autoren und Institutionen gegen die diskurshemmenden Auswirkungen der 'Zeitschriftenkrise'” \cite{naeder_2010_open}. In der Literatur herrschen unterschiedliche Auffassungen über die Definition Open Access, wie es erreicht werden kann und welchen genauen Bezugsrahmen das Attribut "Open" aufweist \cite{Adema_2014_open_access} \cite{cite:17}. Dies ist darauf zurückzuführen, dass es keine formelle Struktur, keine offizielle Organisation und keinen ernannter Führer innerhalb der Open Access Bewegung gibt \cite{poynder_2011_suber}. Darüber hinaus sind die existierenden Definitionen meist interessengeleitet und neigen dazu, "Kriterien, Methoden, Ziele und Fogenabschätzungen ineinander zu verflechten" \cite{naeder_2010_open}.

Einigkeit besteht allerdings darin, dass es der Forderung nach Open Access nicht um die Abschaffung oder die Entwertung materiellen geistigen Eigentums geht. Kaum jemand bestreitet, dass Open Access mit dem Urheberrecht, mit dem Peer-Review System, mit Einnahmen (auch Gewinn), dem Drucken, der Erhaltung, Reputation, Qualität, wissenschaftlichen Karriere-Fortschritt, der Indexierung, und andere Merkmale und unterstützende Aspekte die mit dem herkömmlichen wissenschaftlichen Publikationssystems assoziiert werden kann \cite{suber_2015}. Die Unterstützer dieser Idee vereint das gemeinsame Ziel, "die Bedingungen zu verbessern, unter denen wissenschaftliche Arbeiten zirkulieren können"\cite{Adema_2014_open_access}. Die Propagierung der Öffnung der wissenschaftlichen Ergebnisse erstreckt sich dabei vor allem "Publikationen, die nicht darauf angelegt sind, Einnahmen aus Verkaufserlösen für ihre Urheber zu generieren" \cite{muller_2010_open}.

Exemplarisch gelten folgende Definitionen als Ansatzpunkt für das Verständnis von Open Access in dieser Arbeit. Im Rahmen der Inhaltsanalyse wurden sie auf Grundlage hoher Zitationsraten als die gängigen Einordnungen für Open Access identifiziert:
\begin{tabular}{lcccc}
\textbf{Autor (Jahr)} & \textbf{Typ der Publikation} & \textbf{Titel} & \textbf{Inhalt} & \textbf{BBB-Bezug} \\
Peter Suber (2004) & Webseite & A Very Brief Introduction to Open Access & "Open-access (OA) literature is digital, online, free of charge, and free of most copyright and licensing restrictions. What makes it possible is the internet and the consent of the author or copyright-holder." & ja \\
Gunther Eysenbach (2006)  & Artikel  & Citation Advantage of Open Access Articles & "Open access (OA) to the scientific literature means the removal of barriers (including price barriers) from accessing scholarly work." &  ja \\
Willinsky, J. (2006) & Artikel & The access principle: The case for open access to research and scholarship. & increasing access and improving access to the journal literature, largely through the use of the Internet &  ja \\
Harnad, Stevan, et al. (2008)  & Buch  & The access/impact problem and the green and gold roads to open access: An update. & "full texts are accessible online toll-free—let us call that “Open Access” (OA), in line with the definition provided in 2001 by the Budapest Open Access Initiative" &  ja \\
Uwe Müller (2010) & Buchkapitel & Open Access. Eine Bestandsaufnahme. & "Mit Open-Access wir der freie, unmittelbare und uneingechränkte Zugang zu wissenschaftlichen Publikationen und Forschungsergebnissen in elektronischer Form bezeichnet." &  ja \\
Laakso M., et al. (2011) & Artikel & The Development of Open Access Journal Publishing from 1993 to 2009 & "Open Access (OA), in the context of scholarly publishing, is a term widely used to refer to unrestricted online access to articles published in scholarly journals." &  ja \\
Herb, Ulrich (2012) & Buchkapitel & Offenheit und wissenschaftliche Werke: Open Access, Open Review, Open Metrics, Open Science und Open Knowledge & "Open Access bezeichnet demnach die Möglichkeit, wissenschaftliche Dokumente entgeltfrei nutzen zu können" & ja  \\
\end{tabular}

---- TODO: Tabelle weiter ausfüllen - Auflistung Open Access Definitionen in der Literatur und Zusammenfassung der Definition ---

Das Attribut "Open" definiert den Bezugsrahmen für den offenen Zugang zu wissenschaftlichen Publikationen. Eine der Definitionen der Bedingungen von "Open" ist die Open Definition der Open Knowledge Foundation. Sie hat den Anspruch die Prinzipien und Bedingungen für die Offenheit von Daten und Inhalten zu definieren. Diese Definition von Offenheit setzt voraus, dass Daten und Publikationen als ganzes und für nicht mehr als angemessene Wiederherstellungskosten (vorzugsweise als Download) und in einer bequemen und modifizierbaren Form verfügbar sein sollten \cite{Molloy_2011}.

Gemäß der Open Definition gilt der Inhalt als "Open", der "für jeden Zweck von jedem kostenlos genutzt, modifiziert und geteilt werden" \cite{open_definition} kann. Ziel dieser Definition ist es, "die Bedeutung von offen in Bezug auf Wissen" zu präzisieren. Wissen erstreckt sich in diesem Zusammenhang auf Inhalte wie Musik, Filme, Bücher, jegliche Art von Daten, ob wissenschaftlicher, historischer, geographischer oder anderer Art und Regierungs- und andere Verwaltungsinformationen \cite{open_definition}.

Die Open Definition wurde von der Open Scource Definition abgeleitet und ist als synonym für "frei" oder "libre" im Rahmen der Definition für "freie kulturelle Werke" zu verstehen \cite{suchen}. Ein Werk oder Inhalt gilt nach dieser Definition als "offen", wenn es bei der Verbreitung folgenden Kriterien erfüllt:
\begin{enumerate}
\item Einhaltung der Prinzipien von Zugang, Verteilung, Wiederverwendung und dem Fernbleiben von technologischen Restriktionen
\item Attribuierung, Integrität als maximale Einschränkung
\item Unterbindung der Diskriminierung von Personen, Gruppen oder bestimmten Bereichen/Gebieten
\item Einhaltung genannten Kriterien  im Rahmen der Lizensierung
\end{enumerate}

\subsubsection{Möglichkeiten des Open Access Publizierens}

In der Literatur wird Open Access in unterschiedliche Formen unterteilt \cite{CREATe_2014} \cite{albert_2006_open_implications} und es bestehen unterschiedliche Auffassungen über die verschiedenen  von Open Access \cite{CREATe_2014} \cite{cite:22b} \cite{lewis_2012_inevitability}. Sowohl die Definitionen, als auch die Modelle orientieren sich an den "three Bs", den derzeit geltenden Definitionen von Open Access \cite{Adema_2014_open_access}. Am Beispiel der Budapest Open Access Initiative werden zwei Wege für Open Access dargestellt \cite{albert_2006_open_implications}:
\begin{enumerate}
\item Die Etablierung "einer neuen Generation von Fachzeitschriften", die einen kostenfreien und unmittelbaren Zugang zu den Beiträgen ermöglichen ("goldener" Weg)
\item Die öffentlich zugängliche (Selbst-)Archivierung durch die Urheber ("grüner" Weg) \cite{adema_2013_political} \cite{hall_2008_digitize}
\end{enumerate}

Der "grüne Weg" beschreibt ein Modell, bei dem der Autor im Rahmen einer (Selbst-)Archivierung von Beiträgen in Repositorien (teilweise öffentlichen Dokumentenservern) die Verfügbarkeit seiner Publikation anstrebt \cite{brembs2015open} \cite{muller_2010_open} \cite{grand_2012_open}. Das vom Autor initial eingereichte Dokument (Manuskriptfassung) steht dabei als Pre-Print oder Post-Print-Version auf institutionellen oder disziplinären Dokumentenservern \cite{suchen} oder privaten Homepages \cite{suchen} jedem zur Verfügung. Im Unterschied zu Post-Prints, hat bei Pre-Print keine Peer Review stattgefunden \cite{suchen} und der Beitrag hat somit keine externe wissenschaftliche Qualitätssicherungsmaßnahme durchlaufen. Beim "grünen Weg" hat der publizierende Verlag darüber hinaus die Möglichkeit innerhalb einer Speerfrist von üblicherweise 6-12 Monaten \cite{suchen} den lektorierten und fertig-publizierten Beitrag unter einer eigenen Lizenz zu verkaufen \cite{suchen}. Erst nach Ablauf dieser Frist wird die finale und lektorierte Fassung des Beitrags frei und offen zur Verfügung gestellt. Es existieren je nach Verlag und Publikationsform verschiedenen Möglichkeiten der Ausgestaltung dieses Publikationsweges \cite{suchen}. Sie alle einigt die Möglichkeit für den Autor seinen eingereichten Beitrag unmittelbar, frei und kostenlos zu veröffentlichen und die freie und kostenlos Veröffentlichung der finalen Publikation durch den Verlag nach einer Speerfrist \cite{dorschel_2006_open}. Die vertragsrechtliche Ausgestaltung des grünen Weges ist vielfältig und reicht von einer tatsächlichen Beschränkung der Rechtseinräumung auf das für den Vertragszweck erforderliche Maß bis zu einer für Autoren und Archivaren ungünstigen "vollständigen Übertragung, gepaart mit einer schuldrechtlichen Gestattung einzelner Nutzungshandlungen nach Ablauf einer gewissen Schutzfrist" \cite{dorschel_2006_open}. Der grüne Weg ist demnach als Kompromiss für ein Open Access auf Grundlage der Interessen der Verlage anzusehen \cite{Mussell_2013}.

Beim "goldene Weg" stellt der Autor unmittelbar nach der Fertigstellung die finale und lektorierte Publikation über einen Verlag frei und offen zur Verfügung. Auch die Verlagsversion muss ohne Sperrfrist in einem Repositorium unmittelbar zur Verfügung gestellt werden. Der Verlag hat allerdings zusätzlich die Möglichkeit, den Beitrag kommerziell zu vertreiben und zu verkaufen, muss jedoch parallel eine freie und offene Version der Publikation zur Verfügung stellen.

Alternativ ermöglicht es der verzögerte goldene Open Access-Weg dem Verlag, zeitverzögert für die Öffentlichkeit die finale Version der Publikation unter einer offenen Lizenz zur Verfügung zu stellen \cite{lewis_2012_inevitability}. Der Verlag hat bei diesem verzögerten Modell den Vorteil, einen bestimmten Zeitraum die Publikation vertreiben zu können, ohne zeitgleich eine offene und freie Version anbieten zu müssen. Der Autor hat im Gegensatz zum "grünen Modell" aber dennoch die Möglichkeit diese finale Publikation vollumfassend sofort kostenfrei anzubieten.

Im Rahmen anderer Modelle, meist gemischter Modelle, wird den Autoren im Nachhinein die Möglichkeit eingeräumt, durch zusätzliche Zahlung, die Publikation offen und frei zur Verfügung zu stellen\cite{lewis_2012_inevitability}. Das hat für den Autor den Nutzen, dass er von den Vorteilen bei der offenen Verbreitung von Publikationen unter den Bedingungen von Open Access profitiert. Macht der Autor davon erst nach einem gewissen Zeitraum gebrauch, generiert der Verlag neben den initialen Verkaufserlösen über diesen Weg zusätzliche Einnahmen. Diese alternativen Modelle ermöglichen es, das trotz des Open Access Publizierens parallel zu den kostenlosen und offenen elektronischen Veröffentlichungen weitere kostenpflichtige Publikation in gedruckter oder digitaler Form erfolgen können. Eine Grundvoraussetzung dafür ist, dass neben der kostenpflichtigen Version, auch eine kostenfreie Version der Publikation unter den in der Open Definition erklärten Bedingungen exisitiert.

Darüber hinaus findet in der Literatur die Segmentierung in gratis und libre Open Access statt \cite{Martin_2013} \cite{naeder_2010_open} \cite{Mounce_2015}. Mit gratis Open Access wird dabei die Möglichkeit bezeichnet, den Zugang zu Publikationen und Forschungsergebnisse zu erleichtern und die Kostenpflichtigkeit zu beenden. Libre Open Access bedeutet, dass weitere Barrieren, wie Urheber- und Lizenzbeschränkungen aufgehoben werden. \cite{Adema_2014_open_access} Diese Unterteilung wird von einigen Autoren kritisiert, da durch das Hinzufügen eines weiteren Attributs die eigentlich scharfe Abgrenzung von "Close" und "Open" geschwächt wird, was sich auch auf andere Bereiche der Open-Bewegung (Open Data, Open Government, Open Spending uvm.) auswirken könnte \cite{suchen}. Diese Kritik kann auch auf die Modelle von Green und Golden Open Access ausgeweitet werden und die Differenzierung der Begriffe steht unter dem Verdacht grundsätzlich wenig oder falsch verstanden zu werden \cite{Mounce_2015}.

Neben den dargestellten Modellen existieren weitere Veröffentlichungsmodelle für Open Access Publikationen. Die Einteilung in hybride, radikale und sonstige Formen von Open Access stellt dabei eine weitere entwertete Ebene der Unterteilung dar \cite{Mounce_2015}. Weitere, aber im Vergleich wenig genutzte Modelle sind hybride Modelle. Als hybrid werden diese deshalb bezeichnet, weil der Autor wählen kann, ob er den Verlag für den kostenlosen Zugriff auf seine Publikation finanziert oder der Leser über Subskriptionsmodell zahlt \cite{muller_2010_open}. Dieses Modell steht aber in der Kritik, da die rechtlichen Bedingungen nur selten eine Nachnutzung oder Weiterverbreitung erlauben und die Verlage nur selten auf das exklusive Verwertungsrecht verzichten \cite{muller_2010_open}. Diese Publikationsformen werden als Open Access bezeichnet, genügen aber nicht den gängigen Deklarationen \cite{boai_2012} oder verstoßen gegen die Open Definition. Entspricht eine Veröffentlichung nicht der Open Definition wird aber vom Verlag oder der herausgebenden Institution als "Open" bezeichnet, so wird auch von "Open Washing" gesprochen \cite{suchen}. Von einer weiteren Unterteilung der Open Access Modellen wird deshalb und aufgrund ihrer geringen Verbreitung und Praktikabilität in dieser Arbeit abgesehen.

Der verzögerte goldene Weg und grüne Weg beeinträchtigen das klassische Geschäftsmodell der Verlage vorerst nicht direkt. Publikationen werden wie bisher angeboten und erst nach einer bestimmten Zeit auch kostenlos zur Verfügung gestellt. Im Gegensatz dazu kommt der goldene Weg, auf Grundlage unmittelbarer, freier und offener Veröffentlichungspflicht, ohne das tradierte Geschäftsmodell der Verlage aus \cite{lewis_2012_inevitability}.

Allerdings werden für Publikationen, die unter den Bedingungen von Open Access veröffentlicht werden, durch die Verlage vorab Veröffentlichungsgebühren von den Autoren erhoben \cite{suchen}. Diese sogenannten article processing charges (APC) werden damit gerechtfertigt, dass bei dieser Publikationsform weder auf den Peer-Review-Prozess, noch auf die Möglichkeit Umsatz zu generieren, Urheber zu schützen oder andere Stärken der traditionellen Publikationsformen verzichtet wird \cite{albert_2006_open_implications} \cite{Open_Access_net_2009}.

Somit ändert das Open Access Geschäftsmodell die Erlösstruktur der Verlage von nachgelagerten, verkaufsorientierten Einnahmen hin zu Vorabeinnahmen für die Erstellung und den Vertrieb der Publikation. Strukturell steht Open Access für Verlage damit vorerst in keinem Widerspruch zur Bewahrung der wissenschaftlichen Qualität oder den Vorteile des klassischen Publikationssystems \cite{Suber_2002}. Verlage nutzen zwar Open Access-Optionen, wollen damit aber die etablierten Verhältnisse möglichst fortschreiben und halten am Subskriptionsmodell weiter fest \cite{schmidt_2007_goldenen}.

---- Todo: Tabelle - Auflistung Open Access  Modelle und Formen in der Literatur
Gegenstand / Zeitraum / Referenz
Zusammenfassung der Definition ----

\subsubsection{Open Access Kanäle und Publikationsformate}

In diesem Abschnitt wird auf unterschiedliche Open Access Kanäle und Publikationsformate eingegangen. Es wird unterschieden in: Open Access Aggregatoren, Open Access Repositorien, Open Access Journals, Open Access Bücher und Monografien. Diese Kanäle und Formate adressieren die unterschiedlichen Publikationsformen der wissenschaftlichen Kommunikation oder konkrete Herausforderungen in Bezug auf die Distribution und Archivierung im Rahmen der neuen Möglichkeiten von offenem und freien Publizierens.

Da es eine enge Verknüpfung zwischen Repositorien und der Entwicklung der Open-Access-Bewegung gibt \cite{adema_2013_political} \cite{offhaus_2012_institutionelle_repos}, soll in diesem Kapitel auf die Rolle der Repositorien als spezifischen Kanal für die Verbreitung von Publikationen eingegangen werden. Repositorien sind verwaltete Orte zur Aufbewahrung geordneter Dokumente. Institutionelle Repositorien gelten als ein Instrument für wissenschaftliche Einrichtungen oder eine Gruppe von Einrichtungen, um Publikationen für einen institutionell in einem meist abgegrenzten Bereich frei zugänglich zu machen \cite{dobratz_2007_open} \cite{Baggs_2006}. Über die Hälfte der forschungsorientierten deutschen Universitäten betreiben ein solches institutionelles Repositorium \cite{Schmidt_2009}.

Institutionelle Repositorien haben erhebliche Vorteile für die Institutionen, wenn sie in die ganzheitlichen Rahmenbedingungen der Universität integriert sind \cite{steele_2006}. Repositorien können neben der Kernaufgabe der Archivierung und Verbreitung von Publikationen auf für die Lernumgebungen, den Forschungsservice und die Marketingaktivitäten einer Universität eine wichtige Rolle spielen. Sie ermöglichen zum Beispiel die Dokumentation des universitären Outputs und verbessern den institutionellen Austausch \cite{steele_2006}. Ökonomisch rentieren sie sich vor allem dann, wenn Skaleneffekte eintreten und Forschungseinrichtungen in Verbünden agieren \cite{blythe_2005value}. Neben den institutionellen sind auch fachliche oder andere Arten von Repositorien eng mit der Open Access Bewegung verknüpft. Repositorien stehen für die digitale Speicherung von Dokumenten und zunehmend auch Daten zur Verfügung. Somit entwickeln sich von "bloßen Repositorien für Literatur in Richtung digitaler Forschungsportale und -umgebungen", die "verschiedenste Materialien integrieren und damit nutzbar und zitierfähig machen" \cite{Schmidt_2009}. Digitle Repositorien bieten Mehrwertdienste, insbesondere die Erhebung von Nutzungsstatistiken, Zitionsanalysen und webometrischer Daten \cite{jahn_2011_personliche} \cite{mayr_2005_webometrie}. Über diese Repositorien wird der Zugang zu den unterschiedlichen Modellen und Publikationswegen von Open Access Publikationen ermöglicht \cite{suber_2015}.

--- TODO: Tabelle - Auflistung Open Access  Modelle und Formen in der Literatur Gegenstand / Zeitraum / Referenz Zusammenfassung der Definition ---

\subsubsection{Kritische Betrachtungen von Open Access}

Open Access ist nicht unumstritten. Kritik an Open Access kommt vor allem von den "etablierten Wissenschaftsverlagen, aber auch von Autoren, die um Einnahmen aus Autorenverträgen fürchten" \cite{Schirmbacher_oa_2007}. Neben der Kritik am ökonomischen Modell, sowie der Angst vor der Einschränkung von Freiheit in Forschung, Lehre und Forschungsheterogenität wird gar befürchtet \cite{Szczesny_2014}, dass Open Access es "tatsächlich in den Händen" hat ganze Publikationsformen, wie "dem geisteswissenschaftliche Buch ein Ende zu bereiten" \cite{Hirschi_2015_buch_oa}.

Aus der Perspektive der Leser gibt es wenig Kritik am Konzept von Open Access \cite{wein_2010_erwerbung} \cite{weishaupt_2009_goldenOA}. Sie bezieht sich wenn überhaupt zumeist auf die Befürchtungen aus den Konsequenzen der Öffnung für die Wissenschaft und Forschung. Dabei werden vor allem die sinkende Forschungsheterogenität \cite{Hirschi_2015_buch_oa}, die eventuell steigende Einflussnahme durch die "Steuerzahler", die Gefahr der Medialisierung der Wissenschaft sowie den Konsequenzen einer Unterwanderung der Steuerungsmechanismen von Wissenschaft und Forschung genannt. Als theoretische Gefahr wird diesbezüglich beispielhaft die Gefahr der Au­ßer­kraft­set­zung des Wahrheitsmonopols der Wissenschaft durch das Aufmerksamkeitsmonopol der Medien genannt \cite{weingart_2005_wissenschaft}.

Von Seiten der Autoren herrscht eine geringe Akzeptanz sowie wenig Interesse an Open Access Publikationen und es existieren "viele Vorbehalte und Missverständnisse" \cite{Suber_2002}. Diese fehlende Akzeptanz für Open Access in der wissenschaftlichen Gemeinschaft stellt noch immer eine der größten Herausforderungen für die Etablierung offener Kommunikation in der Wissenschaft und Forschung dar \cite{weishaupt_2009_goldenOA}. Die Vorurteile betreffen insbesondere die Verschiebung des Reader/Library-Pay Systems zum Author-Pay System \cite{EuropeanCommission_sciencepub_2006} \cite{Chibnik_2015} zur Refinanzierung der Publikation, bei dem Autoren für die Veröffentlichung der Texte selbst zahlen müssen, damit die Texte frei zugänglich sind \cite{Mussell_2013} und das "obwohl auch bei konventionellen (nicht Open Access) Veröffentlichungen oft genug die Druckkosten selbst aufgebracht werden mussten" \cite{weishaupt_2009_goldenOA}. Im Rahmen der Verschiebung des Erlösmodells von der Bibliothek hin zum Autor, entwickeln betrügerische Verleger zunehmend falsche Open Access Journale, die eine ernsthafte Bedrohung für die Zukunft der Wissenschaftskommunikation \cite{Beall_2012}.

Eine weitere Hürde für die Akzeptanz stellen Probleme bei der Sicherung der "Authentizität und Integrität der Texte" dar \cite{weishaupt_2009_goldenOA} \cite[:191]{Fehling_2014}. Darüber hinaus gibt es Herausforderungen bei der Langzeitarchivierung \cite{hagner_2015_sache_buches} \cite{Martin_2013} und der Einbettung offener Kommunikation in das wissenschaftliche Reputationssystem \cite{weishaupt_2009_goldenOA} \cite{Suber_2002} \cite{Adema_2014_open_access}. Darüber hinaus gerät das bisher von den Verlagen organisierte Bewertungssystem ins Wanken, wenn Wissenschaftler und Wissenschaftlerinnen anfangen einfach ihre Publikationen frei und offen im Internet veröffentlichen und "die Auszeichnung, eine Veröffentlichung in einem so genannten renommierten wissenschaftlichen Journal zu platzieren, nichts mehr gelten soll" \cite{Schirmbacher_oa_2007}.

Die Vertriebsarten, die auf einem Modell beruhen, bei denen der Autor die Kosten für die Publikation trägt kann auch als "Sozialismus für die Reichen" \cite{cope2014future} bezeichnet werden. Denn bei dem Modell tritt zum einen beim tradierten Publikationssystem kritisierte Matthäus-Effekt wieder in Kraft und nur gut ausgestattete und damit meist bereits renommierte Universitäten, Institutionen oder Lehrsrühle werden die Ressourcen aufbringen können um Publikationen zu veröffentlichen. Der Effekt verstärkt sich in sozial Schwächeren Umgebungen und verhindert bei der Wissensproduktion die erhoffte Schaffung gleicher Bedingungne im wissenschaftlichen Kommunikationssystem \cite{suchen}.

---- TODO: Tabelle - Auflistung Open Access Kritik in der Literatur
Gegenstand / Zeitraum / Referenz Zusammenfassung der Kritik / Anzahl der Zitationen ----

Im Folgenden werden exemplarisch zwei Bereiche der Kritik weden genauer dargestellt um einen tieferen Einblick in die Themen und Akteure der Debatten um Open Access zu ermöglichen: Die Kritik am ökonomischen Modell und die Kritik an der Einschränkung von Freiheit in Forschung, Lehre und Forschungsdiversität

\textbf{1. Kritik am ökonomischen Modell}

Ein Kritikpunkt an dem Open Access Modell bezieht sich auf das Kostenargument und die ursprüngliche Hoffnung, dass die technologischen Treiber gesteuert und organisiert von der Forschungscommunity selbst, anstatt durch Fachverlage, die durchschnittlichen Kosten für einen publizierten Artikel signifikant senken könnten. So stellte sich die Frage, ob "aus der Sicht des individuellen Nutzenkalküls von Wissenschaftlern, Verlagen und weiteren Einrichtungen wie Bibliotheken als auch aus Sicht gesamtwirtschaftlicher Wohlfahrtsüberlegungen (...) der Markt der Wissenschaftskommunikation nicht effizienter organisiert werden könnte."\cite{Hess_2006} In einigen Beiträgen wurden schon sehr früh Kostensenkungen von bis zu 90 Prozent \cite{hilf_2004} \cite{suchen} prognostiziert.

Folgende Punkte schürten darüber hinaus die Hoffnung, das System leistungsfähiger zu machen und "von seinen durch den Papierdruck auferlegten Fesseln" zu befreien \cite{hilf_2004}:
\begin{itemize}
\item langer Zeitverzug vom Einreichen eines Manuskriptes bis zum finalen Bereitstellung des Wissens,
\item komplizierter Vertriebsweg vom Verlag über Grossisten zu Bibliotheken,
\item hohe Kosten (ca. 3.000,- Euro für die gesamte Verlagsarbeit je Artikel) mit den daraus folgenden horrenden Zeitschriftenpreisen,
\item und daraus folgend wenige, sowie ungleich in der Welt verteilte Leser (digital divide),
\item unvollständige Information (aus Platzmangel), was Nachnutzungen und das Nachprüfen erschwert und somit auch Fälschungen erleichtert,
\item nur anonymes Referieren vor der Veröffentlichung, was den Missbrauch erleichtert.
\end{itemize}

Verlage die Open Access publizieren stehen unter einer besonderen und neuen Herausforderung mit diesem Modell nachhaltig zu operieren und passen deshalb ihre Preise von Zeit zu Zeit an. "Auffällig ist jedoch, dass gerade die großen erfolgreichen Projekte wie BioMed Central und Public Library of Science nach ihrer Einführung am Markt deutlichen Gebrauch von Preissteigerungen gemacht haben"\cite{schmidt_2007_goldenen}. Diese Entwicklung hält, wenn auch verlangsamt, weiter an\cite{suchen}. Unter diesem Kostenaspekt unterscheiden sich subskriptionsbasierte und Open Access-Verlage nicht fundamental \cite{schmidt_2007_goldenen}.

Neben der Refinanzierung über Modelle bei den Autoren vorab die Kosten für die Veröffentlichung übernehmen, werden in der Literatur auch andere Möglichkeiten genannt. Die Refinanzierung über Werbung eignet sich nur für einige Disziplinen \cite{bjork_2004_open}. Die Finanzierung über hybride Modelle, bei denen Open Access Text mit Texten nach dem klassischen Erlösmpdell gemixt werden und die Autoren gegen Zahlung Text freikaufen können, sind bisher ebenfalls nur mäßig erfolgreich \cite{bjork_2012_hybrid}. Nach einer ersten Dekade von Experimenten rund um die Refinanzierung von Open Access bleibt die Kritik an der Nachhaltigkeit von Open Access in Bezug auf das ökonomische Modell. Somit bleibt die Frage nach der Refinanzierung von zentraler Bedeutung für die weitere Verbreitung von Open Access.

\textbf{2. Gefahr der Einschränkung von Freiheit in Forschung, Lehre und Forschungsdiversität}

Würden Forschungsförderer eine Erstveröffentlichung als Open-Access (golder Weg) verlangen, so wäre zweifelsohne der Schutzbereich der positiven Publikationsfreiheit und damit ein integraler Bestandteil der Wissenschaftsfreiheit berührt \cite[:191]{Fehling_2014}. Eine Veröffenlichungspflicht unter den Kriterien von Open Access würde allerdings auch klar die negative Publikationsfreiheit im  Sinne der Freiheit, Forschungsergebnisse nicht zu publizieren wiedersprechen \cite[:192]{Fehling_2014}. Wobei der Schutz dieser gelegentlich bestritten wird \cite[:192]{Fehling_2014}.

Darüber hinaus wird vermutet, dass die umfassende Öffnung der wissenschaftlichen Kommunikation weitreichende Implikationen, auf das "wie" geforscht wird und auch "was geforscht" wird \cite{suchen} hätte. Dabei soll die Vermischung von Interessen bei der Finanzierung von Forschung in Deutschland soll durch die Unabhängigkeit der Deutschen Forschungsgemeinschaft verhindert werden. Ziel dieser Trennung ist die unabhängige Verteilung der Mittel, völlig frei von politischer Couleur \cite{suchen}. Dennoch kann, so die Befürchtung einiger Autoren \cite{suchen}, nicht sichergestellt werden, dass eine die umfassende Einbeziehung und Information der Gesamtöffentlichkeit nicht doch mittelbar Einfluss auf die Mittelvergabe haben könnte \cite{weingart_2005_wissenschaft}. Ein Großteil der Wissenschaft wird durch Steuergelder finanziert, was nicht ausschließt, dass politische Interessen, die Steuerungsmechanismen von Wissenschaft und Forschungsförderung trotz unabhängiger Forschungsförderungsstrukturen beeinflussen können.

Der Mediziner und Wissenschaftshistoriker Michael Hagner formuliert seine Befürchtung in einem Beitrag für die Frankfurter Allgemeine Zeitung wie folgt: "Open Access als Traum der Verwaltungen". Er und andere beschreiben die Gefahr, dass die Wissenschaft bei der Verpflichtung zur elektronischen Veröffentlichung von Forschungsergebnissen für Wissenschaftler durch Universitäten auf eine vollends verwaltete Forschung hinaus laufen würde \cite{hagner_faz_2009}. Andere antizipieren eine weitere Gefährdung von Wissenschaft und Forschung, weil Grundlagenforschung, sowie andere komplexe oder explorative Forschungsbereiche in Zukunft weniger Berücksichtigung finden würden, wenn die Öffnung der wissenschaftlichen Forschungsprozesse unter rein kommerziellen Aspekten weiter vorangetrieben wird \cite{suchen}.

Obwohl die Normen von Offenheit schon immer eine entscheidende Rolle bei der Aufrechterhaltung der systemischen Wirksamkeit der modernen wissenschaftlichen Forschung gespielt haben, sind sie sehr anfällig für die Legitimation des Rückzugs der staatlichen Schirmherrschaft und den öffentlichen Schutz zur Sicherung der Rahmenbedingungn für die eigentliche Aufgabe von Wissenschaft \cite{david1998_common}.

Um diese Aspekte und Prognosen über die Implikationen von Open Access zu evaluieren, wird in diesem Teil der Arbeit auf Grundlage von Textbeispielen die Kritik an der Öffnung von Wissenschaft und der (forschungs-)politischen, rechtlichen und freiheitlichen Entwicklungen dargestellt.

Als ein konkretes Beispiel für die "Kontroversen um die Zukunft des Buches, um Autorenschaft und geistiges Eigentum, die Rolle von Velagen und die für Leser kostenlose Bereitstellung aller wissenschaftlichen Literatur" \cite{hagner_2015_sache_buches}, die Einschränkung der Wissenschafts- und Publikationsfreiheit soll der "Heidelberger Appell" für Publikationsfreiheit und die Wahrung von Urheberrechten dienen. Am 22. März 2009 wurde auf der Webseite der „Frankfurter Allgemeinen Zeitung“ der Artikel "Geistiges Eigentum: Autor darf Freiheit über sein Werk nicht verlieren" \cite{faz_heidelberger_apell_2009} veröffentlicht. Vorangegangen war eine öffentlich ausgetragene Diskussion zwischen dem Literaturwissenschaftler Prof. Dr. Roland Reuß und weiteren Wissenschaftlern in einem Spezial der Onlineausgabe der Frankfurter Allgemeinen Zeitung: "Die Debatte über Open Access". Im Anhang zu diesem Artikel fand sich ein öffentlicher Aufruf, auch "Heidelberger Appell" genannt.

Der Appell richtete sich vor allem an "die Bundesregierung und die Regierungen der Länder, das bestehende Urheberrecht, die Publikationsfreiheit und die Freiheit von Forschung und Lehre entschlossen und mit allen zu Gebote stehenden Mitteln zu verteidigen" \cite{ITK_2009}. Die Autoren forderten, unter anderem in Bezug auf die Google Buchsuche (Google Books), die Politik, Öffentlichkeit und Kreative auf, sich für die "Wahrung der Urheberrechte" und "gegen eine angebliche „Enteignung“ der Autoren durch das Vorgehen von Google einerseits sowie durch das Publikationsmodell Open Access andererseits" \cite{WD_bundestag_2009} zu engagieren.

Die Autoren des Appells unterscheiden zwei Ebenen: \textit{International} kritisieren sie "die nach deutschem Recht illegale Veröffentlichung urheberrechtlich geschützter Werke geistigen Eigentums auf Plattformen wie GoogleBooks und YouTube", sowie die Entwendung dieser "ohne strafrechtliche Konsequenzen". Im \textit{nationalen Rahmen}, so prangern die Autoren weiter an, werden diese "Eingriffe in die Presse- und Publikationsfreiheit, deren Folgen grundgesetzwidrig wären" durch die "Allianz der deutschen Wissenschaftsorganisationen (Mitglieder: Wissenschaftsrat, Deutsche Forschungsgemeinschaft, Leibniz-Gesellschaft, Max Planck-Institute u.a.)" sogar unterstützt \cite{ITK_2009}.

Die Kritik der Autoren des Heidelberger Apells an Open Access bezieht sich, laut einer Untersuchung des Wissenschaftlichen Diensts des Bundestags im wesentlichen auf die folgenden Aspekte \cite{WD_bundestag_2009}:
\begin{enumerate}
\item Erzwungene Vertriebswege
"Eine Forschung, der man diktieren könnte, wo ihre Ergebnisse publiziert werden sollen, sei nicht mehr frei." Die Verpflichtung auf "bestimmte Publikationsform (...) dient nicht der Verbesserung der wissenschaftlichen Information" \cite{ITK_2009}.
\item Subventionierung von Vertriebswegen oder der Gefährdung von Fachzeitschriftenverlagen
\end{enumerate}

Der Appell "hat eine außergewöhnlich heftige Diskussion über die urheberrechtliche Problematik im Hinblick auf die aktuellen Entwicklungen im Internet ausgelöst. Viele Parlamentarier und Politiker sind für das Thema sensibilisiert" worden \cite{WD_bundestag_2009}. In Bezug auf Open Access widerlegt der Wissenschaftliche Dienst die Befürchtungen der Autoren des Heidelberger Apells. Dem Kritikpunkt der "Erzwungenen Vertriebswege" widerspricht der Wissenschaftliche Dienst mit einem Verweis auf Gudrun Gersmann, weil "auch (Anmerkung: unter Open Access) eine Veröffentlichung bei einem Verlag mit einfachem Nutzungsrecht weiterhin möglich sei". In Bezug auf das Modell und das Abhängigkeitsverhältnis halten die wissenschaftlichen Autoren des Bundestags Reuß entgegen, dass es im bisherigen System "zwischen Autor und Fachzeitschriftverlag oft ein einseitiges Abhängigkeitsverhältnis zu Lasten des Autors gibt" und Wissenschaftler "oftmals alle Rechte an ihren Beiträgen abtreten" \cite{WD_bundestag_2009} müssen. "Der Befürchtung im Heidelberger Appell, das Publikationsmodell Open Access gefährde Fachzeitschriftenverlage wird entgegengehalten, dass die digitale Plattform auf lange Sicht auch ein Ausweg aus der Zeitschriftenkrise sein könnte" \cite{WD_bundestag_2009}. Abschließend konstatiert der Wissenschaftliche Dienst des Bundestags, dass die "Kritik an Open Access kaum nachvollzogen werden" kann und "die hier gemachten Vorwürfe" "eher auf die traditionellen Vertriebswege zu treffen, als auf das neue Publikationsmodell" \cite{WD_bundestag_2009}.

Obwohl der Heidleberger Apell unter dem Verdacht stand, eine "eine an Informationsdefiziten und Fehlinterpretationen reiche Kampagne" \cite{Schmidt_2009} darzustellen, scheint ein Teil der Kritik  mindestens zwei Punkten berechtigt zu sein. Erstens, dass man seitens der Forschungsförderer nicht besonders bemüht war \cite{suchen}, sich "ein genaues Bild von den Nebenwirkungen (Anmerkung: von Open Access)" \cite{Reuss_2009} zu verschaffen und zweitens wurde die Sicherung von Freiheit von Forschung und Lehre sowie die Anpassung der Steuerungsmechanismen bei den Bestrebungen zur Öffnung von Wissenschaft und Forschung nur ungenügend berücksichtigt \cite{hagner_2015_sache_buches}.

Die Kritik am urheberrechtlichem Aspekt der Google Buchsuche soll in dieser Arbeit nicht berücksichtigt werden, da es sich dabei zwar um einen Aspekt der Digitalisierung von Büchern, nicht aber um die Öffnung von wissenschaftlicher Kommunikation nach den Kriterien der in dieser Arbeit gewählten Deklarationen handelt, sowie die Google Buchsuche als Dienst keinen Bezug zur Open Access-Bewegung aufweist \cite{hagner_2015_sache_buches}. Dennoch sei auf den Umstand verwiesen, dass die Fixierung auf das Urheberrecht einem idealisierten Verständnis der wissenschaftlichen Verlagswesens entspringt und von den wirklichen gefahren für die Buchkultur ablenkt \cite{Hirschi_2015_buch_oa}.

\subsection{Offener Zugriff auf wissenschaftliche Kommunikation: Open Science}

Hinter dem Begriff Open Science oder Offene Wissenschaft verbirgt sich die Forderung, dass wissenschaftliche Erkenntnisse aller Art im Rahmen des wissenschaftlichen Erkenntnisprozesses schnellstmöglich offen verbreitet und nutzbar gemacht werden sollen. \cite{stafford_2010_science}. Open Science beschränkt sich dabei nicht nur auf den Zugang zur wissenschaftlichen Publikation am Ende des wissenschaftlichen Erkenntnisprozesses (Open Access) und auf die daraus resultierenden Veränderungen wissenschaftlicher Kommunikationsprozessen im Rahmen von Publikationen, sondern auf sämtliche Daten und Informationen die während des Prozesses anfallen. Aus technischer Sicht ist damit jeder Aspekt der wissenschaftlichen Arbeit gemeint, der digital auf einem Desktop-Computer stattfindet und somit auch öffentlich über das Web potenziell verfügbar gemacht werden kann \cite{mietchen2012wissenschaft}.

Offene Wissenschaft kann als Sammelbegriff für eine Vielzahl an Aktivitäten und Mechanismen der der kumulativen Wissensproduktion verstanden werden \cite{Mukherjee_2009}. Sie alle tragen dazu bei, dass die sämtliche Inhalte der Kommunikation während und nach der Wissensproduktion durch andere innerhalb und außerhalb der wissenschaftlichen Gemeinschaft weiterverwendet werden können. Open Science resultiert auch aus der zunehmenden Anwendung von Diensten und Applikationen des sozialen Webs auf die Arbeit von Wissenschaftlern und umfasst die "Zugänglichkeit des gesamten For- schungsprozesses, vom Sammeln der Daten an, über die Begutachtung hin zur fertigen Publikation" \cite{brembs2015open}.

Open Science basiert zudem auf der in der ureigenen wissenschaftlichen Anforderung, dass die Ausübung von wissenschaftlichen Tätigkeiten auf eine Art und Weise erfolgt, die es anderen ermöglicht zu den Forschungsbemühungen beizutragen, zusammenzuarbeiten und auf alle Daten, Ergebnisse und Protokolle in allen Phasen des Forschungsprozesses frei zuzugreifen \cite{RIN_2010_open_research}. Der gesamte Forschungsprozess sollte demnach so transparent und so zugänglich wie möglich gestaltet werden \cite{Scheliga_2014}.

Anhand der folgenden Einteilung werden die Charakteristika des wissenschaftlichen Erkenntnisprozesses erläutert und dargestellt, um zu verdeutlichen, was die Öffnung von Wissenschaft im Sinne von Open Science beinhaltet. Zur Verdeutlichung des Prozesses der Wissensschaffung wird in der vorliegenden Arbeit eine Einteilung extrapoliert in vier Phasen vorgenommen:
\begin{enumerate}
\item Fragestellung und Planung
\item Ausführung
\item Verarbeitung und Analyse
\item Auswertungsverfahren
\item Verwendung und Kommunikation der Ergebnisse
\end{enumerate}

---- TODO: Grafik aus http://de.slideshare.net/petermurrayrust/osbrazil Slide 14 bauen und an Phasen anpassen ----

Die Forderung nach Öffnung des gesamten Prozesses der Wissensschaffung begründet sich dabei nicht (nur) durch Unzulänglichkeiten am bestehenden wissenschaftlichen Kommunikationssystem, sondern basiert auf den folgenden weiterführenden Annahmen:
\begin{enumerate}
\item Der offene Zugang zum gesamten Wissenschaftsprozess erhöht die Möglichkeiten der Validierung und Reproduzierbarkeit der gesamten Forschung(-skette) \cite{Aleksic_2014} \cite{Krumholz_2014} \cite{hey_2015_open} und die Entwicklung neuer Qualitätskriterien. (enhanced Validation/Reputation-Argument)
\item Im Rahmen des Teilens (z.B. von Rohdaten) erhöht sich die Effizienz und Verwendbarkeit durch in der Forschung und Wissenschaft entstandenen Informationen \cite{Fecher_2015}. (Shared-Science-Argument)
\item Im klassischen wissenschaftlichen Kommunikationssystem gibt es keine Anreize negative, widerlegende oder unerfolgreiche wissenschaftliche Ergebnisse zu veröffentlichen. Eine vollumfängliche Öffnung des wissenschaftlichen Erkenntnisprozesses könnte dazu beitragen, dass Wissenschaft ihrem Anspruch an Falsifizierbarkeit gerecht wird. (negative-science/falsifiability-Argument)
\item In Ergänzung zu den bestehenden Mechanismen unter denen Vertrauen unter Wissenschaftlern und von der Öffentlichkeit in Wissenschaft besteht \cite{weingart_2005_wissenschaft}, bietet die vollständige Veröffentlichung der Informationen, die von offenen Wissenschaft ausgeht, als Ersatz für oder Ergänzung zu älteren Vertrauenssystem betrachtet werden, zum Nutzen der wissenschaftlichen Gemeinschaft und der Gesamtgesellschaft. \cite{grand_2012_open}.  (Trust-Technology-Argument)
\end{enumerate}

Fast die Hälfte der klinischen Forschungstudien werden nie veröffentlicht und die selektive Veröffentlichung verzerrt medizinische Erkenntnisse und hemmt den Fluss von Informationen, die wichtig sind, um die Entscheidungsfindung durch die Patienten und ihre Ärzte zu unterstützen \cite{Ross_2013}. Werden Ergebnisse nicht veröffentlicht beeinträchtigt das auch andere Forschung, da auch negative Ergebnisse einen Beitrag zum Falsifikationsprozess liefern. Die Replikation ein wesentlicher Teil der wissenschaftlichen Arbeit, Qualitätskontrolle und Methode.

Die Kultur der Forschung braucht diesen Grad an Offenheit in der wissenschaftlichen Kommunikation, sowie den Zugang zu Daten anderer Wissenschaftler und Wissenschaftlerinnen um im wissenschaftlichen Erkenntnisprozess erfolgreich zu sein \cite{Fecher_2015} \cite{Krumholz_2014} \cite{patlak_2010_open}.

Bestrebung der Öffnung des wissenschaftlichen Arbeitsprozesses sollte es demnach sein, erfolgreiche Wege zu finden, um Daten und benutzten Programmcode unter Berücksichtigung der Interessen aller Beteiligten möglichst umfangreich im besten Interesse der Gesellschaft zu teilen \cite{naeder_2010_open} \cite{Ross_2013} \cite{hey_2015_open}. Offene Wissenschaft hat das Potenzial durch Transparenz und die Möglichkeit des Eröffnung des Zugriffs auf wissenschaftliche Informationen und Daten einen notwendigen Beitrag zu dem Vertrauen der Menschen in Wissenschaft und das Vertrauen von Wissenschaft in Menschen zu leisten \cite{grand_2012_open}.

---- TODO: Tabelle - Auflistung Open Science Definitionen in der Literatur Gegenstand / Zeitraum / Referenz Zusammenfassung der Definitionen ----

\subsubsection{Open Science: Modelle, Formate und Kanäle}

Open Science vereint als Sammelbegriff viele Modelle, Formate und Kanäle. Ein maßgeblicher Treiber für Open Science war die technologische Entwicklung und die neue Möglichkeiten für Wissenschaft aus methodischdologischer Sicht und für die Dissemination von Forschungsinformationen \cite{garcia_2010_open}. Ermöglichte das Internet zunächst die einfache Darstellung von Inhalten in einem globalen Netzwerk, führte die Entwicklung des sozialen Webs zu der Möglichkeit eines umfassenden Austauschs in nahezu Echtzeit und die offenen Kommunikation in Wissenschaft und Forschung. Weder die einfache Darstellung von Inhalten in einem globalen Netzwerk, noch die Möglichkeiten zum direkten und offenen Austausch zwischen Wissenschaftlern waren im analogen Zeitalter rein technisch nicht möglich und die Frage eines freien und umfassenden Zugriffs auf wissenschaftliche Information stellte sich erst garnicht \cite{Schirmbacher_oa_2007}.

---- TODO: Grafik e-Science -> Open Access -> Science 2.0 -> Open Science ----

Open Science umfasst alle Charakteristika des wissenschaftlichen Erkenntnisprozesses. Exemplarisch werden hier Möglichkeiten der Öffnung des Prozesses dargestellt:
\begin{itemize}
\item Veröffentlichung in Datenrepositorien - Diese Repositorien rmöglichen die Ablage und die Verbreitung von wissenschaftlichen Daten, die im Rahmen des wissenschaftlichen Erkenntnisprozesses anfallen. Dabei können hier nicht nur die Daten abgelegt werde, die im Rahmen der endgültigen Publikation genutzt wurden, sondern auch die, die im Vorfeld erhoben wurden oder negative Ergebnisse enthalten. Grundsätzlich können diese Repositorien in begutachtete und nicht-begutachtete Repositorien unterteilt werden.
\item Offene Erstellung von Forschungsanträgen - Forschungsförderung ermöglicht die Einwerbung und Allokation von Ressourcen für ein wissenschaftliches Vorhaben. Die öffentliche Erstellung eines solchen Antrags bietet zwar die Gefahr der Kopie durch andere, ermöglicht aber auch die Einbeziehung externen Wissens und somit die Möglichkeit eines besseren Antrags. Eventuelle Herausforderungen können so früh erkannt werden und die Möglichkeit der positiven Begutachtung steigt. Zudem schafft diese Art der Beantragung mehr Transparenz bei der Mittelvergabe und (fach-)öffentliches Interesse an dem Prokjekt.
\item Arbeit mit offenen Laborbüchern - Stellen eine Möglichkeit für die offene Ablage von Informationen und die Dokumentation rund um die wissenschaftliche Arbeit dar. Ziel ist es, ein möglichst umfassendes Bild von der Materie und den eingesetzten Methoden und Applikationen frühstmöglich und so umfangreich wie ausführbar zu dokumentieren. Das verbessert die Vorraussetzungen für die Replizierbarkeit zu und ermöglicht gegebenefalls Fehler früh zu erkennen.
\item Erweitertes offenes Publizieren - Neben den offenen Publizieren von feritgen Texten (Open Access), ist es grundsätzlich realisierbar die digitalen Publikationen auch mit den Daten anzureichern. Demnach hätten Leserinnen und Leser von Literatur nicht nur die Möglichkeit eines Zugangs zum wissenschaftlichen Text, sondern könnten beim Lesen auch auf die Daten auf der die Ergebnisse beruhen zugreifen.
\end{itemize}

---- TODO: Überarbeiten und weiter ausarbeiten ----

---- TODO: Grafik bauen ----

\subsubsection{Kritik an Open Science}

Während viele Wissenschaftler und Wissenschaftlerinnen Offenheit in der Forschung als wertvoll erachten, sind nur wenige tatsächlich bereit, die zusätzliche Zeit und Mühe dafür zu investieren und potenzielle nicht abgrenzbare Risiken einzugehen, Forschung offen und uneingeschränkt zugänglich zu machen \cite{Scheliga_2014} \cite{Tenopir_2011} \cite{Procter_2010}. In vorhergenden Studien waren es vor allem jüngere Forscherinnen und Forscher, die ein spezielles Interesse hatten, ihre Daten nicht zu ohne Einschränkungen veröffentlichen \cite{Tenopir_2011}. Forscherinnen und Forscher, die offene Wissenschaft praktizieren wollen, werden mit einer Reihe von Hindernissen konfrontiert \cite{Scheliga_2014}:
\begin{enumerate}
\item individuelle Hindernisse: Angst vor Trittbrettfahrern, gefürchteter Mehraufwand an Zeit und Mühe, Herausforderungen bei der Nutzung der digitaler Dienste, fehlender Anstoß, Angst negative Ergebnisse zu veröffentlichen, Herausforderung den Datenschutz sicherzustellen, Abneigung den Code zu teilen
\item systemische Hindernisse: Evaluationskriterien behindern Offenheit, kulturelle und institutionelle Einschränkungen, ineffektive (politische) Richtlinien, Mangel an Standards für das Teilen von Forschungsmaterialien, Mangel an rechtlicher Klarheit, finanzielle Aspekte der Offenheit
\end{enumerate}

Ergänzend ide

Betrachtet man wie Scheliga und Friesike das Phänomen Open Science anhand des Konzepts des sozialen Dilemmas, wird deutlich, dass das was im kollektives Interesse der wissenschaftlichen Gemeinschaft ist, nicht unbedingt im Interesse des einzelnen Wissenschaftlers steht. In der wissenschaftlichen Gemeinschaft besteht dabei ein Spannungsverhältnis zwischen dem tun was das Beste für die Gemeinschaft ist, gegenüber dem, was am besten für den einzelnen Wissenschaftler ist \cite{Ekins_2014} \cite{patlak_2010_open} \cite{wein_2010_erwerbung}. "Wenn alle Wissenschaftler ihr Wissen nur in den Situationen teilen, in denen sie erwarten, dass sie selbst davon profitieren, ist der gemeinsame Wissenspool fragmentiert und alle Wissenschaftler stehen schlechter dar" \cite{Scheliga_2014}.

Kritisch wird auch angemerkt, dass Wissenschaftler, die vorläufige Ergebnisse veröffentlichen, ein unkalkulierbares Risiko eingehen, dass andere die Arbeit kopieren und die Anerkennung dafür erlangen, oder die Ergebnisse sogar patentieren lassen \cite{Peters_2014}.

--- TODO: institutionelles Dilemma der Balance zwischen offener und eingeschränkter Wissenschaft - Tabelle - Auflistung Open Science Kritik in der Literatur Gegenstand / Zeitraum / Referenz Zusammenfassung der Definition ---

\section{Zusammenfassung und Ableitungen für die empirische Untersuchung}

Viele der Erklärungsansätze die Forderung nach einem Wandel der wissenschaftlichen Kommunikation hin zur Öffnung der Wissenschaft basieren auf Annahmen, bei denen ein direkter Zusammenhang von technischen Entwicklungen auf (wissenschafts-)politische und kulturelle Bewegungen geschlossen wird. Diese Perspektive beschränkt sich dabei bisher auf den Zugang zu Ergebnissen von Wissenschaft und weniger auf die Öffnung des gesamten Prozesses.

Die theoretische Auseinandersetzung mit der Geschlossenheit des wissenschaftlichen Diskurses auf der einen und den Treibern und Bremsern im realen wissenschaftlichen Prozess auf der anderen Seite werden in der Literatur bisher nur ungenügend berücksichtigt. Insbesondere wird die Verbindung zwischen wissenschaftlicher Reputation, die Motivation das etablierte System zu unterstützen und die Geschlossenheit des Wissensproduktionsprozesses nur selten erörtert. Als weiteres Manko kann angeführt werden, "dass die Deliberation und die Verbreitung von Wissen ein stabiles Set von Infrastrukturen braucht" \cite{kelty_2004}, nach denen man heute noch vergeblich sucht. Das Potenzial bei der Verwendung von digitalen Technologien um Wissenschaft offen zu teilen, ist nicht annährend ausgeschöpft und es "besteht eine erhebliche Diskrepanz zwischen der Idee der offenen Wissenschaft und wissenschaftliche Realität" \cite{Scheliga_2014}. Demgegenüber ist die "(geistes-)wissenschaftliche Alltagspraxis längst von digitalen Recherche- und Kommunikationsformen durchsetzt" \cite{hagner_2015_sache_buches}.

Openness kann als "schwimmender Signifikant (...) ohne eindeutige Definition, adaptierbar von unterschiedlichen politischen Ideologien" verstanden werden \cite{Adema_2014_open_access}. Der Begriff Open Access wird in der neoliberalen Rethorik als effizientes Wettbewerbsmodell, verbunden mit den Ideen von Transparenz und Effizienz von Unternehmen und Regierung, eingesetzt \cite{tkacz_2012_open}. Über diesen Ansatz kann mittels Openness der wissenschaftlichen Prozess outputorientierter und seine Ergebnisse effektiver gestaltet, überwacht und gesteuert werden \cite{adema_2010_oaoverview}.

Diese Entwicklung betrifft auch das System der Universität als Produzent, Archivar und bei der Dissemination von Wissen. Die Öffnung von Wissenschaft und Forschung ermöglicht es der Universiät wieder ein Ort der Wissensproduktion, -speicherung und -vermittlung zu werden, der er mal gewesen ist \cite{kittler_2004}.

\subsection{Treiber für die Öffnung der wissenschaftlichen Kommunikation}

In den analysierten wissenschaftlichen Beiträgen zu Open Access und Open Science werden meist die positiven Auswirkungen der Forderungen auf das wissenschaftliche Kommunikationssystem dargestellt. Dafür war die in Kapitel xxx ---- TODO: welches Kapitel ----  durchgeführte Erarbeitung der Unzulänglichkeiten am bestehenden wissenschaftlichen Kommunikationssystem notwendig \cite{cite:17}.

Bei der Etablierung von Offenheit in Wissenschaft und Forschung wird unter zwei Herangehensweisen unterschieden \cite{schulze_2013_open}:
\begin{enumerate}
\item "Top-down durch Förderstrategien, Vorgaben und Empfehlungen": Hierbei können durch die Bereitstellung zusätzlicher Mittel im Rahmen der Forschungsförderung konkrete Anreize für die offene Veröffentlichung und die Publikation von Forschungsergebnissen geschaffen werden \cite{suchen}. Eine weitere Möglichkeit der "Top-Down"-Etablierung von Offenheit in Wissenschaft und Forschung stellen Empfehlungen dar, bei denen Institutionen, Organisationen oder Gruppen nicht bindende Empfehlungen aussprechen, anhand derer Wissenschaftler und Wissenschaftlerinnen überzeugt werden sollen, ihre wissenschaftlichen Ergebnisse offen zu veröffentlichen \cite{suchen}. Sind weder Anreize, noch Empfehlungen als Top-Down-Ansatz erfolgreich, können bindende Vorgabe etabliert werden um eine Verhaltensänderungen der Wissenschaftler und Wissenschaftlerinnen zu erzwingen \cite{suchen}.
\item "Bottom-up durch Graswurzelprojekte und den Einsatz von Evangelisten":
Im Gegensatz zur Strategie von "oben" gibt es auch Bestrebungen, die von einzelnen WissenschaftlerInnen oder Gruppen initiiert sind. Sie sind überwiegend informell und zielen die Verbreitung von Verhaltensänderungen oder die Etablierung von Richtlinien ab \cite{suchen}. Bottum-up-Projekte kommen aus dem wissenschaftlichen Alltag und erfahren meist keine politische oder monetäre Incentivierung für die Öffnung von Wissenschaft und Forschung. Der Einsatz von Evangelisten basiert auf der Idee einer konkreten Stelle oder Position um eine Änderung zu begleiten \cite{suchen} oder einen Multiplikator innerhalb und außerhalb von Institutionen oder Organisationen zu etablieren, der das gewünschte Ziel pro aktiv kommuniziert und verbreitet \cite{suchen}. Evangelisten können helfen die Befindlichkeiten und Vorbehalte auszutarieren und die teils diffusen, teils realen Ängste bezüglich der Entwicklung von Offenheit und Transparenz der Wissenschaft innerhalb und ausserhalb der wissenschaftlichen Gemeinschaft zu beseitigen \cite{schulze_2013_open}.
\end{enumerate}

Ergänzend dazu sehen die Rechtwissenschaftler Götting und Lauber-Rönsberg vier konkrete, rechtliche und faktische Maßnahmen zur Förderung von Open Access \cite{Goetting_2015}:
\begin{enumerate}
\item Verplichtungen durch das Hochschulrecht - z.B. eine rechtliche Verplichtung steuerinanzierte wissenschatliche Werke als Open Access zu veröffentlichen
\item Maßnahmen der Hochschulen - z.B. durch institutionelle Selbstverplichtungen oder finanzielle und andere faktische Anreizsysteme
\item Maßnahmen der öffentlichen Forschungsförderung - z.B. Verpflichtung im Rahmen der Drittmittelfinanzierung von Forschungsvorhaben oder direkte Förderungsinstrumente für den Aufbau oder die Refinanzierung von Open Access Publikation
\item Urheberrechtliche Maßnahmen - z.B.  Vorhaben steuerfinanzierte wissenschatliche Werke vom urheberrechtlichen Schutz auszunehmen oder Schrankenregelungen bzw. Zwangslizenzen für öffentlich-finanzierte Werke einzuführen
\end{enumerate}

Grundsätzlich steht und fällt der Erfolg bei der Etablierung von Verhaltensänderungen damit, ob sich die jeweiligen Zielgruppe ein unmittelbarer Mehrwert und Nutzen erschließen wird \cite{schulze_2013_open}. Bisher scheint dieser eher gering, denn rechtlich steht es bereits nach der heutigen Rechtslage Wissenschatlerinnen und Wissenschaftlern frei, "sich für eine Erstveröfentlichung ihrer Werke im Wege des Open Access zu entscheiden" \cite[:146]{Goetting_2015}, auch wenn "Möglichkeiten und Grenzen von Open-Access-Publikationsverpflichtungen wesentlich durch die urheberrechtlichen Rahmenbedingungen beeinflusst werden" \cite[:211]{Fehling_2014}.

Für die weitere Gruppierung der Argumente für die Öffnung von Wissen wurde die folgende Kategorisierung vorgenommen. Auf sie folgt die Beschreibung der grundlegende Treiber und Argumente für die Öffnung des wissenschaftlichen Kommunikationssystems:
\begin{enumerate}
\item \textbf{Transition-Argument} - Die Nutzung der neuen Möglichkeiten für eine offene Wissensverbreitung neben den konventionellen Wegen der nicht-elektronischen Publikationen \cite{hall_2008_digitize} \cite{berliner_erklaerung_2003}. Voraussetzung für die Aufbereitung des Wissens als strukturierte Daten zur Wissensweiterverwendung und -verarbeitung über alle Kanäle.
\item \textbf{Speed and Circulation-Argument} - Offene Publikationsverfahren bieten die Chance wissenschaftliche Inhalte schneller und umfassender der wissenschaftlichen Community zur Verfügung zu stellen \cite{muller_2010_open}\cite{RIN_2010_open_research} \cite{hall_2008_digitize} \cite{EuropeanCommission_sciencepub_2006}. Wenn das Wissen schneller zur Verfügung steht, kann es auch schneller zirkulieren und effizienter genutzt werden \cite{Woelfle_2011}. In den tradierten Verfahren wird die Wissensverbreitung künstlich durch Embargos und ineffiziente Validierungs- und Qualitätssicherungssysteme zurückgehalten. Die Digitalisierung und Verbreitung über elektronische Kanäle stellt einen Vorteil für die Wissensverbreitung und -verwertung dar. Eine offene Veröffentlichung erreicht potentiell eine größere Leserschaft als bei Subskriptionsmodellen \cite{cope2014future}.
\item \textbf{Higher Impact and Citation-Argument} - Die uneingeschränkte und globale Verfügbarkeit der offenen wissenschaftlichen Informationen führt zu einem wesentlich höheren Verbreitungsgrad und Einfluss von Wissenschaft \cite{davis_2011_open} \cite{muller_2010_open} \cite{Baggs_2006} \cite{cite:5} \cite{Kurtz2005_oa_citation}. Der Verbreitungsgrad kann einen positiven Einfluss auf die Zitierhäufigkeit haben \cite{muller_2010_open} \cite{EuropeanCommission_sciencepub_2006} \cite{Hajjem_2005}. Die Zitationsrate wissenschaftlicher Publikationen, die nach den Kriterien von Offenheit veröffentlicht werden ist damit potenziell höher \cite{cite:21a}. Diese Kausalität wird "access-citation effect"\cite{davis_2011_open} genannt und ist durch bedeutsame Untersuchungen bestätigt worden \cite{Lawrence_2001} \cite{Jeffrey_2008} \cite{Hajjem_2005} \cite{Eysenbach_2006} \cite{Antelman_2004}. Dennoch gibt Gründe diesen Effekt genau zu hinterfragen und im Detail mögliche Abschwächungseffekte zu berücksichtigen \cite{davis_2011_open} \cite{davis_2008_open}.
\item \textbf{Tax-Payer-Argument} - Die Kosten des traditionellen Publikationsverfahrens werden im Wesentlichen durch die öffentliche Hand getragen \cite{muller_2010_open}. Dem Steuerzahler ist die konventionelle wissenschaftliche Kommunikation jedoch nur selten unentgeltlich zugänglich, obwohl er defacto im Rahmen öffentlich geförderter Forschungsprogramme die Forschung bereits (mit-)finanziert hat \cite{suber_2003_taxpayer} \cite{resnik_2005_ethics} \cite{Baggs_2006} \cite{Woelfle_2011} \cite{Beverungen_2012} \cite{Adema_2014_open_access}. Da die Mittel nach intransparenten Kriterien verteilt werden ist im aktuellen Kommunikationssystem unklar, ob wissenschaftliche Kommunikation nach dem bestmöglichen Einsatz der monetären Ressourcen für Wissenschaft und Forschung abläuft \cite{Glasziou_2014} \cite{altman_1994_scandal}. Die Europäische Union und die Organisation für wirtschaftliche Zusammenarbeit und Entwicklung (OECD) kommen in diesem Zusammenhang zu dem Ergebnis, dass der volkswirtschaftliche Nutzen von Open Access die Kosten signifikant übersteigt \cite{WD_bundestag_2009}.
\item \textbf{Economic Promotion Argument} - Bisher profitieren wirtschaftliche Unternehmungen nur unzureichend von staatlich finanzierter wissenschaftlicher Kommunikation. Eine schnellere, kommerziell verwertbare und umfassendere Bereitstellung wissenschaftlicher Inhalte kann einen Beitrag zur non-monetären Wirtschaftsförderung und Innovation leisten \cite{heise_2012} \cite{suchen OECD EU}. Im Rahmen der offenen und schnelleren Verbreitung wissenschaftlicher Informationen sind darüberhinaus auch neue Geschäftsmodelle denkbar \cite{suchen}.
\item \textbf{Digital Divide Argument} - Der offene Zugang zu Wissenschaft ermöglicht neue Chancen für die Überwindung sozialer, nationaler und globaler Wissenskluften \cite{suchen} zwischen bildungsferneren und -affineren Bevölkerungsteilen und -schichten der Welt \cite{boai_2012}. Darüber hinaus ist der Mehrwert und die Chance von wissenschaftlichen Informationen für die schulische Bildung und für die Bewegung der offenen Bildungsmaterialien bisher ebenfalls noch nicht vollumfänglich ausgeschöpft \cite{heise_lernen_2013}.
\item \textbf{Validation, Quality and Reputation-Argument} - Offenheit in Wissenschaft und Forschung ermöglicht die Entwicklung neuer Verfahren, die die Aktivität und Qualität eines Forschers umfassender, transparenter und demokratischer messbar und kommunizierbar machen, als das es im bestehenden Reputations- und Förderungssystem möglich ist \cite{grand_2012_open}. \cite{chalmers_2009_avoidable_waste}. Es wird vermutet, dass Wissenschaftsevaluation dadurch effizienter wird, da Wissenschaft "per Definition die Bemühung um integre Information ist" \cite{umstatter_2007_qualitatssicherung}. Die Falsifikation ist nur dann umfassend und einfach möglich, wenn der Aufwand für die Falsifikation gering beziehungsweise der Zugriff auf die wissenschaftlichen Informationen überhaupt gegeben \cite{umstatter_2007_qualitatssicherung} und offen ist \cite{Peters_2014}. "Offenheit verhindert, dass Wissenschaft dogmatisch, unkritisch und voreingenommen wird" \cite{resnik_2005_ethics}.
\item \textbf{Paradoxon of Information Argument} - Überwindung des bestehenden Informationsparadoxons bei der Verbreitung und Vermarktung wissenschaftlicher Inhalte. Hierbei handelt es sich um die Herausforderung im Rahmen kommerziell Be- und Verwertung von wissenschaftlichen Informationen ohne zu viel über Inhalt und Qualität auszusagen. Eine im Rahmen von Offenheit angestrebte Entkommerzialisierung des Zugangs zu Wissen würde dieses Informationsparadoxon aufheben.
\item \textbf{Science communication Crisis-Argument} - Durch die Öffnung wissenschaftlicher Kommunikations- und Reputationsprozesse entsteht die Möglichkeit, der vorherrschenden Zeitschriften- und Monographienkrise durch neue Geschäftsmodelle zu begegnen \cite{muller_2010_open} \cite{naeder_2010_open}.
\item \textbf{Interdiscipline and International Exchange/Collaboration Argument} - Die Globalisierung führt auch in der Wissenschaft zunehmend zu internationalem Austausch und zur transnationalen Zusammenarbeit von Wissenschaftlern \cite{Waltman_2011}. Das gilt nicht nur für die grenzüberschreitende Zusammenarbeit in Bezug auf die lokale Verortung, sondern auch für die Interdisziplinarität der Forschungsvorhaben. Die Öffnung der Wissenschaft ermöglicht auch fachfremden Wissenschaftlern Zugriff auf Publikationen und damit auf Wissensressourcen für die eigenen Arbeiten \cite{suchen}.
\item \textbf{Sustainable Access and Archiving Argument} - Nur Offenheit im Sinne von Verwertbarkeit ermöglicht es, in dezentralen Strukturen wie der des Internets alle Informationen nachhaltig und unabhängig voneinander zu speichern. Im Falle von Natur- oder anderen Katastrophen ermöglicht die digitale Ablage auf mehreren Kontinenten eine Präservierung von Wissen unabhängig von lokalen Gegebenheiten oder Bedingungen.
\item \textbf{Dataquality-Argument} - Die Veröffentlichung der Daten hinter den wissenschaftlichen Publikationen kann zu einer insgesamten Erhöhung der Datenqualität wissenschaftlichen Arbeitens führen. Ähnliche Erfahrungen wurden bereits im Bereich der Veröffentlichung von Daten der Verwaltung und bei der Entwicklungszusammenarbeit gemacht \cite{heise_2014_bundestag}.
\end{enumerate}

\subsection{Bremser für die Öffnung der wissenschaftlichen Kommunikation}

Neben den Gründen und Treibern die in den letzten Dekaden für die Öffnung wissenschaftlicher Kommunikation verantwortlich waren, gibt es Prozesse, die entweder zu einer Verlangsamung der Entwicklung geführt haben, oder sie in einigen Teilbereichen zum Erliegen gebracht haben. Dazu gehören unter anderem:
\begin{itemize}
\item Fehlende Richtlinien auf regionaler, nationaler und internationaler Ebene
\item Führungslosigkeit der Open Access Bewegung
\item Mangelhafte Infrastruktur und Applikationen
\end{itemize}

Für eine ausgewogene Betrachtung darf nicht nur auf die Vorteile der Öffnung von Wissenschaft und Forschung betrachtet werden, auch wenn es sich "lohnt das Augenmerk auf "diejenigen Vorteile zu legen, von denen Wissenschaftler selbst profitieren können" \cite{muller_2010_open}, so müssen auch die Argumente adressiert werden, die Argumente gegen die Öffnung der wissenschaftlichen Prozesse und Publikationen sprechen:
\begin{enumerate}
\item \textbf{Quality-Argument} - Das erste Argument umschreibt die Befürchtung, dass die Qualität von offener wissenschaftlicher Kommunikation aufgrund von schlechten oder nicht vorhandenen wissenschaftlichen Überprüfungsmechanismen leidet \cite{Chibnik_2015} \cite{Beall_2012}. Dabei wird argumentiert, dass ein durch ein Autorengebühren finanziertes Publikationsmodell keinen klaren Anreiz für Ablehnung bietet.
\item \textbf{Renommee-Argument} - Die Möglichkeit zur Erlangung von wissenschaftlicher Reputation ist ein grundlegender Motivationsfaktor für Wissenschaftler und Wissenschaftlerinnen die Ergebnisse ihrer Arbeit zu veröffentlichen. Eine Veröffentlichung zahlt nur dann auf die Reputation ein, wenn sie im Rahmen von renommierten Publikationskanälen stattfindet. Offene Publikationsplattformen und Journale können aufgrund des kurzen Zeitraums ihres Bestehens und aufgrund von Vorbehalten dieses Renomee nur selten vorweisen. Die Renommeefrage stellt eine der größten Hürden für die offene wissenschaftliche Kommunikation dar \cite{weishaupt_2009_goldenOA} \cite{Woelfle_2011}.
\item \textbf{Archiving- and Sustainability-Argument} - Den grundsätzlichen Vorteilen des elektronischen Publizierens stehen Probleme und Zweifel an der langfristen Verfügbarkeit und Langzeitarchivierung \cite{weishaupt_2009_goldenOA} gegenüber. Einige Autoren kritisieren, dass die Sicherstellung der Langzeitarchivierung und die langfristige Auffindbarkeit, sowie Bereitstellung der Dokumente bisher nicht vollumfänglich durch digitale Strukturen gewährleistet werden kann \cite{umstatter_2007_qualitatssicherung} \cite{Gersmann_2007}.
\item \textbf{Authenticity- or Integrity-Argument} - Ein weiteres Problem stellt die Sicherung der Authentizität der offen publizierten wissenschaftlichen Informationen dar \cite{umstatter_2007_qualitatssicherung} \cite{weishaupt_2009_goldenOA} \cite{grand_2012_open}. Weil elektronische Dokumente oft innerhalb weniger Tage oder Wochen in mehreren Versionen zugänglich sind wird befürchtet, dass Texte und Arbeiten, im Zeitablauf inhaltlich nicht mehr unverändert ihrem Autor zuordenbar sind. Das gilt "solange sie nicht in Digitalen Bibliotheken mit gesicherter Authentizität abgeliefert" werden \cite{umstatter_2007_qualitatssicherung}.
\item \textbf{Rightsmanagement-Argument} - Die Verpflichtung für Mitarbeiter staatlich finanzierter Forschungsinstitutionen, alle Texte und Daten elektronisch frei und offen zu publizieren, wird von einigen Autoren als kritisch bewertet \cite{suchen}. In dem 2009 veröffentlichten "Heidelberger Appell" \cite{faz_heidelberger_apell_2009} kritisieren zahlreiche Autoren, Wissenschaftler, Verleger und Publizisten, dass das “verfassungsmäßig verbürgte Grundrecht von Urhebern auf freie und selbstbestimmte Publikation” … “derzeit massiven Angriffen ausgesetzt und nachhaltig bedroht” ist. Weiter sehen die Unterzeichner „weitreichende Eingriffe in die Presse- und Publikationsfreiheit, deren Folgen grundgesetzwidrig wären“ \cite{ITK_2009}. Rechtliche Bedenken und die Befürchtung vor kostspieligen juristischen Fehltritten stellen einen weiteren Vorbehalt gegen die offene Veröffentlichung von Forschung- und Forschungsergebnissen dar \cite{weishaupt_2009_goldenOA}.
\item \textbf{(Re-)Financing-Argument} - Die unklare Refinanzierung der Kosten, die im Rahmen der offenen wissenschaftlichen Kommunikation vermutet werden, werden als weiteres Kernargumente gegen das offene Publizieren von Arbeiten und Daten angeführt \cite{Chibnik_2015}. Die Befürchtung ist, dass die umfassende Öffnung des wissenschaftlichen Systems überhaupt nicht finanziert werden kann, konnte bisher nicht vollumfänglich ausgeräumt werden \cite{weishaupt_2009_goldenOA}.
\item \textbf{Ressource-Allocation-Argument} - Dieses Argument befasst sich mit der Annahme, dass der Vergabe von Fördermitteln und bei den reputationsbildenen Maßnahmen durch offene System nicht Rechnung getragen werden kann. Das Argument ruht auf dem Verdacht, dass die Öffnung des wissenschaftlichen Prozesses Einfluss auf Mittelvergabe hat \cite{grand_2012_open} und ausschließlich zugunsten populärer Forschung stattfindet und es zu einer Aushöhlung der wissenschaftlichen Fächer- und Facettenvielfalt kommt.
\item \textbf{Open-Caring-Argument} - Wissenschaftlerinnen und Wissenschaftler befürchten durch den Zwang zu umfassenden Bereitstellung ihrer Publikationen und gegebenenfalls sogar der Quelldaten sowie des genutzten Softwarecodes einen nicht unwesentlichen zeitlichen und finanziellen Mehraufwand \cite{bbaw_publizieren_2015} \cite{mennes_2013_making_os} \cite{grand_2012_open}. Der nötige Aufwand den die umfassende Öffnung der wissenschaftlichen im Alltag des Wissenschaftlers mit sich bringen würde, ist bisher kaum evaluiert \cite{osterloh2008anreize}.
\item \textbf{Scientific-Freedom/Loss of Idea-Diversity-Argument}
Dieses Argument betrifft zwei Ebenen: Die Sorge dass durch Offenheit und Transparenz Forschungsförderung und Öffentlichkeit die bestehenden Steuerungsmechanismen der Wissenschaft ausgehebelt werden und infolgedessen nur die wissenschaftlichen Projekte gefördert und unterstützt werden, die vom Souverän verstanden werden. Diese Befürchtung ruht auf der Annahme, dass die Gewinnung von Wissen zum Beispiel in der Grundlagenforschung ein "öffentliches Gut" darstellt, "dessen Wert von der Öffentlichkeit nur schwer beurteilt werden kann"\cite{osterloh2008anreize}. Darüber hinaus wird in der Literatur die Befürchtung geäußert, dass durch die Öffnung die Freiheit von Forschung und Lehre im Sinne der Publikations- und Veröffentlichungsfreiheit gefährdet wird \cite{Jochum_2009}. Damit ist die Wahl des Publikationsmediums gemeint, die bei den Wissenschaftlerinnen und Wissenschaftlern liegen sollten \cite{bbaw_publizieren_2015} . Infolgedessen wird an vielen Stellen die Befürchtung geäußert, dass im Rahmen von zunehmender Kollaboration über digitale Kanäle, sowie durch die Effizienz der elektronischen Suche die Diversität von wissenschaftlichen Meinungen und Projekten zu einem gleichen oder ähnlichem Thema eingeschränkt werden könnte \cite{Evans_2008}. Diese Betrachtung ist allerdings nicht unumstritten \cite{lariviere2009decline}.
\item \textbf{Interpretations-Argument} - Eine weitere Sorge, die den Öffnungsprozess bremst, ist die Angst der wissenschaftlichen Community vor  Fehlinterpretationen \cite{grand_2012_open}, sowie der Verlust der Kontrolle über die Informationssteuerung \cite{gibbons_1994}. Dabei steht vor allem die Befürchtung im Vordergrund, dass die frei verfügbaren veröffentlichten Arbeiten genutzt werden, um die Arbeit der Wissenschaft zu diskreditieren oder sie gezielt zur Falschinformation der Öffentlichkeit genutzt werden.
\item \textbf{Transparent-Research-Intentions-Argument} - Die Forderung nach Offenlegung des gesamten Forschungsprozesses beinhaltet auch die Forderung nach "Transparenz der Interaktion zwischen Sponsoren (insbesondere kommerzielle Förderer wie die Pharma- und Medizinprodukteindustrie) und Auftragnehmern" \cite{Stengel_2013}
\end{enumerate}

---- TODO: Ausarbeiten: Irritationspotenziale durch ausweitung des Zugangs und Kritik an digitaler Medien  ----
