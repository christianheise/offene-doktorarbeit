\chapter{Zusammenfassung und Ausblick}

In dieser Arbeit wurden die Herausforderungen der wissenschaftlichen Kommunikation im Rahmen der Digitalisierung und die Forderung nach Öffnung dieser umfassend diskutiert. Dabei wurden die einzelnen Definitionen von Open Access und Open Science, sowie Implikationen und Rahmenbedingungnen näher betrachtet, analysiert und spezifiziert.

Ungeachtet der viefältigen Kritik am aktuellen Kommunikationssystem, scheint dieses jedoch auch nach zwei Jahrzehnten noch immer weitestgehend stabil.

\section{Definition von Open Access und Open Science in der Literatur}
\subsection{Open Access}
\subsection{Open Science}
\section{Die Öffnung von Wissenschaft auf Grundlage der Befragung}
\subsection{Treiber und Bremser}
\subsection{Differenzierung zwischen den wissenschaftlichen Disziplinen auf Grundlage der Befragung}
\section{Feldbericht und Beobachtungen im Rahmen des Experiments}
\subsection{Handlungsempfehlungen zum Verfassen offener Dissertationen}


Sollten die Zeiten des "stürmischen Wachstums der Wissenschaft endgültig vorüber" \cite{K_lbel_2002} sein, lässt sich ein Grund dafür in der Geschlossenheit des wissenschaftlichen Kommunikationssystems vermuten.