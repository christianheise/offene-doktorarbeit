\chapter{Zusammenfassung und Ausblick}

\begin{quote}
\textbf{"Denn ich habs umsonst empfangen, umsonst hab ich's gegeben und begehre auch dafür nichts"}
\end{quote} \cite{luther_1876}

In dieser Arbeit wurden die Herausforderungen der wissenschaftlichen Kommunikation im Rahmen der Digitalisierung und die Forderung nach Öffnung dieser umfassend analysiert. Die Entwicklungen im Bereich der Öffnung wissenschaftlicher Kommunikation aus geistes- und kulturwissenschaftlicher Perspektive wurden genauer untersucht und den Erkenntnissen über die Öffnung wissenschaftlicher Kommunikation aus der Literatur empirisch erhobene Daten gegenübergestellt, sowie das Ergebnis dieser Gegenüberstellung abschließend diskutiert. Im folgenden werden die Ergebnisse der Arbeit zusammengefasst und weitere Anknüpfungspunkte für weitere Forschungsbemühungen dargestellt.

Nach einer Einführung in die Thematik des wissenschaftlichen Kommunizierens und der Ausführung des Vorgehens sowie der Darstellung die eigenen Position und der Beweggründe, wurde die methodischen Herangehensweisen beschrieben. Im weiteren Verlauf wurden die theoretischen Grundlagen, Debatten um Veränderungen im wissenschaftlichen Kommunikationssystem dargestellt und weitere Annahmen, Katalysatoren und Hindernisse identifiziert. Im darauffolgenden Abschnitt wurde schließlich mithilfe der quantitativen Methode der Online-Befragung unter 1.112 Wissenschaftlern und Wissenschaftlerinnen analysiert, welche Auffassungen un Annahmen in Bezug auf den postulierten Wandel wissenschaftlicher Kommunikation vorherrschen und inwiefern diese mit anderen Eigenschaften korrelieren. Daraufhin wurden die erhobenen Daten in den Kontext bisheriger Untersuchungen in diesem Feld gesetzt und diskutiert. Im fünften Kapitel wurden die bisher gewonnen Erkenntnisse experimentell anhand der eigenen Öffnung des wissenschaftlichen Erkenntnisprozesses autoethnografisch analysiert und erneut diskutiert.

---- TODO: weiter ausarbeiten ----

\section{Katalysatoren und Hindernisse für die Etablierung der Öffnung wissenschaftlicher Kommunikation}

Nach der Einführung in den Themenbereich der wissenschaftlichen Kommunikation und des wissenschaftlichen Publizierens wurden anhand aktueller Literatur die Entwicklung und die Debatte um die Forderung nach Öffnung der wissenschaftlichen Kommunikation dargestellt. Aus den dargestellten Debatten wurden Treiber und Bremser für die Entwicklungen identifiziert und herausgearbeitet sowie Open Science von Open Access abgegrenzt.

Für die weitere Identifikation der Treiber und Bremser für einen Wandel in der wissenschaftlichen Kommunikation wurden darauf hin mithilfe einer Online-Befragung unter 1112 deutschsprachigen Wissenschaftlerinnen und Wissenschaftlern genauer abgefragt. Dabei wurde analysiert, inwiefern ein Verständnis von Offenheit im Rahmen der wissenschaftlichen Kommunikation vorherrscht und inwieweit das Interesse an Offenheit in den unterschiedlichen wissenschaftlichen Fachdisziplinen verbreitet ist und praktiziert wird. Die Ergebnisse wurden mit einer Studie des Soziologischen Forschungsinstituts Göttingen aus dem Jahr 2007 verglichen um Trends und Entwicklungen zu identifizieren.

Die Ergebnisse der Studie können insofern als repräsentativ gelten, als sie auf einer relativ großen Stichprobe beruhen. In der Auswertung wurde vor allem deutlich, dass die Verbreitung offener wissenschaftlicher Kommunikationsverfahren wesentlich vom Fachgebiet abhängt. Ungeachtet der vielfältigen Kritik am aktuellen Publikations- und Kommunikationssystem scheint dieses jedoch auch nach zwei Jahrzehnten noch immer weitestgehend stabil. In der Studie wurde außerdem eine große Diskrepanz zwischen dem Interesse nach und dem Verständnis für Offenheit und der tatsächlich praktizierten offenen Kommunikations- und Arbeitsweise festgestellt. Das deckt sich auch mit bisherigen Analysen \cite{yiotis_2013_open} \cite{Bartling_2013} \cite{hagner_2015_sache_buches} \cite{Fecher_2015}.

Ein Ergebnis dieser Arbeit ist, dass es maßgeblich von den publizierenden Wissenschaftlern selbst abhängen wird eine treibende Kraft und aktive Position bei der Gestaltung des Wandels hin zur Öffnung der wissenschaftlichen Kommunikation zu übernehmen. Die Frage nach den Konsequenzen der Öffnung wissenschaftlicher Kommunikation ist dabei eng mit der Frage nach der zukünftigen Rolle der Universität und dem Hochschulwesen verbunden. "Eine Systemanalyse (...) des akademischen Alltagslebens  könnte  ein wenig Klarheit und damit vielleicht auch Rückgewinnung von Gestaltungsraum geben" \cite{Warnke_2012}. Die Wissenschaftler und Wissenschaftlerinnen sollten proaktiv die Möglichkeiten des Gestaltungsspielraums nutzen um ihre Rolle als Produzent, Archivar und bei der Verbreitung von Wissen zurückzugewinnen.

Das ist allerdings aus diversen Gründen eine Herausforderung: Erstens sind sie gefragt den Wandel so zu gestalten, dass die Wissenschafts- und Publikationsfreiheit größtmöglich gewahrt wird und zweitens, dass der Wettbewerb um die Autorengebühren und Publikationsgeschwindigkeit nicht zu einer Bedrohung für die Zukunft der Wissenschaftskommunikation wird \cite{Beall_2012} \cite{Lossau_oa_2007}. Wissenschaftler und Wissenschaftlerinnen haben sich bisher "erstaunlich wenig für das Thema interessiert" \cite{hagner_2015_sache_buches}.

Sucht man nach Gründen für das Desinteresse, wird schnell deutlich, dass unter anderem Unwissen über die wirtschaftlichen Aspekte der wissenschaftlichen Informationsversorgung dafür eine Rolle spielen. Dabei muss vor allem die komfortable Situation der Wissenschaftler in einem System genannt werden, in dem für die meisten Mitglieder der wissenschaftlichen Gemeinschaft kein oder nur geringer unmittelbarer Anreiz besteht, sich aktiv mit dem Publikationssystem zu beschäftigen, weil sie weder die Kosten des Publikationssystems tragen \cite{Sietmann_oa_2007}, noch sich mit den finanziellen Aspekten auseinander setzen müssen \cite{herb_2010}. Darüber hinaus scheint der Bedarf an Veränderung gering, sich mit der Entscheidung wo und wie veröffentlicht wird und mit den Konsequenzen dieser Entscheidung auseinander zu setzen, da diese Veränderungen im wissenschaftlichen Reputationssystem nur unzureichend honoriert werden.

Als weitere Gründe für das geringe Interesse an der proaktiven Öffnung von Wissenschaft und Forschung seitens der wissenschaftlichen Gemeinschaft müssen die Unterschiede und fachspezifischen Eigenheiten der wissenschaftlichen Kommunikation, wie zum Beispiel die unterschiedlichen Publikationsformen, genannt werden. Demnach ist ein gemeinsames Handeln der wissenschaftlichen Gemeinschaft nur schwer möglich. So erfolgten die ersten offenen Publikationsvorhaben und Entwicklungen hin zu Open Access in erster Linie aus den STM-Fächern, die von Zeitschriftenkrise viel stärker und früher betroffen waren. Die daraus resultierten Erklärungen und Bestrebungen führten zu erheblichen "Vorbehalten hinsichtlich der Sinnhaftigkeit und Durchführbarkeit von Open Access, die wiederum zu Desinteresse oder Polarisierung bei vielen Vertretern dieser Disziplinen führen" \cite{naeder_2010_open}.

Christopher Kelty sieht darüber hinaus für das Desinteresse an Offenheit im wissenschaftlichen Kommunikationssystem folgende zwei Aspekte: Erstens, ist Open Access ziemlich langweilig und zweitens ist das Thema bei genauerer Betrachtung sehr komplex. Ergänzend zu Keltys Einschätzung kann als Ergebnis dieser Arbeit aber auch noch ein dritter Grund genannt werden: Durch die Beschäftigung mit dem wissenschaftlichen Publikationsmarkt, würden "Praktiken auf dem Spiel stehen, die dem Tun vieler Geisteswissenschaftler (...)" und auch Wissenschaftler anderer Fächer, "Sinn und Legitimität zu verleihen scheinen" \cite{Hirschi_2015_buch_oa}. Die Ergebnisse der Befragung nähren diese Befürchtung, dass sie auch in naher Zukunft keine hervorgehobene Rolle bei der Gestaltung der Transformation einnehmen werden.

---- TODO: weiter ausarbeiten ----

\section{Erkenntnisse aus dem offenen Verfassen der Arbeit}

Ein weiteres Hemmnis für die Umsetzung der Konzepte um die offen praktizierte Wissenschaft sind die fehlenden technischen Möglichkeiten. Die zur Verfügung stehenden Plattformen und Applikationen sind noch nicht ausgereift und bequem genug um im Alltag offene Wissenschaft zu praktizieren. Die Arbeit ist seit Dekaden auf die geschlossene Publikation oder den nicht-öffentlichen Publikationsprozess ausgelegt und bisher nutzt nur eine Minderheit Webplattformen für die wissenschaftliche Textarbeit \cite{Perkel_2014}.

Die offene Erstellung dieser Arbeit wäre ohne programmiertechnische Vorkenntnisse schwer bis nicht möglich gewesen. So muss der "Open Scientist" entweder in der Lage sein, selbst programmieren zu können oder es müssen die Rahmenbedingungen geschaffen werden, dass er ohne solche Kenntnisse den gesamten wissenschaftlichen Prozess offen und transparent abbilden kann.

Im Gegensatz zum Träger- und Speichermedium Papier wird wissenschaftliches Wissen im Rahmen der Digitalisierung als Code gespeichert. Und hier liegt die "Quelle des revolutionären Selbstverständnisses, das zumindest Teile der Open-Access-Bewegung trägt" und die Konsequenz aus der Digitalisierung des wissenschaftlichen Publikationswesens: Neben dem gedruckten Wort besteht der Kern von Wissen im digitalen Zeitalter nicht mehr aus dem gedruckten Wort sondern aus Code. Will man demnach die Rohform von Wissen lesen, verstehen, interpretieren, oder verändern muss man diesen Code lesen können. Die Vorteile vom Teilen und Verbreiten von Wissen erfüllen sich demnach bisher nur für den, der für die Migration das nötige Know-How hat.

Eine weitere Herausforderung besteht darin, dass viele Förderorganisationen im Rahmen von Forschungsförderung noch immer fast ausschließlich auf klassische Publikationen abzielen und nur langsam die Software-Entwicklung unterstützen \cite{hey_2015_open}. Förderorganisationen müssen diese Verantwortung ernst nehmen, indem sie die zusätzlichen Ressourcen, die für die Schaffung der strukturellen Grundlagen die mit der Öffnung von Wissenschaft und Forschung verbunden sind, zur Verfügung stellen \cite{mennes_2013_making_os} \cite{patlak_2010_open}. Um diese Bemühungen voranzubringen, müssen sich Förderorganisationen mehr denn je ihrer Rolle als "einflussreiche  Akteuren  im  komplexen und  sich  wandelnden  Markt  für  wissenschaftliche  Publikationen" \cite{wein_2010_erwerbung} bewusst werden und entscheiden, ob die Umsetzung der gemeinsamen Nutzung von Daten durch Incentivierung Mandate fördern \cite{mennes_2013_making_os}. Bisher ist diese Nutzung von wissenschaftlichen Daten nur sehr gering verbreitet und auch wenn die Erwähnung der Weiternutzung von wissenschaftlichen Daten ansteigt, bleiben bis zu 86 Prozent der veröffentlichen Daten bisher ungenutzt beziehungsweise unzitiert \cite{peters_2015_research}.

Der zunehmende Grad an Digitalisierung im Arbeitsalltag der Wissenschaftler und Wissenschaftlerinnen stellt die Notwendigkeit dar, sich mit den produzierten Daten auseinanderzusetzen. Dabei ist die Veränderungen der Arbeitsweise von analogen Methoden und Tools auf digitale Formate für die Gewinnung von Wissen unausweichlich. Dieser Wechsel von einem analogen Speicher und Arbeitsmedium in den digitalen Raum ist weder ausweichlich, noch verhandelbar. Die Konsequenzen und Möglichkeiten für die Verbreitung der digital abgelegten Informationen ermöglicht dennoch eine neue Form der wissenschaftlichen Kommunikation. Aus der reinen Digitalisierung wissenschaftlicher Arbeitsprozesse und Anreicherung um die Möglichkeiten des digitalen Austauschs (Science 2.0) sowie die freie und offene Publikation finaler Forschungsergebnisse (Open Access) kann zu einer umfassenden Öffnung dieser Kommunikation für die Gesamtgesellschaft (Open Science) führen.

Die Abgrenzung von Open Access zu Open Science im Rahmen wissenschaftlicher Kommunikation wurde in dieser Arbeit auf Grundlage der Unterscheidung von "Zugang zu Wissen" (Open Access) und "Zugriff auf Wissen" (Open Science) durchgeführt. Das Konzept von Open Access und der damit verbundenen Verfügbarkeit von wissenschaftlichen Publikationen als Ergebnis von wissenschaftlicher Forschung im bestehenden wissenschaftlichen System betrifft nur ein Teil der grundlegenden Neuordnung wissenschaftlicher Kommunikation. Open Science als Sammelbegriff betrifft nicht nur die Digitalisierung und den Zugang zu wissenschaftlichen Publikationen, sondern fordert die Transformation und die Möglichkeit des Zugriffs auf den gesamten wissenschaftlichen Prozess.

Rainer Kuhlen definierte diesbezüglich schon 2002 drei Szenarien wie der Zugriff auf Wissen in Zukunft organisiert sein wird \cite{Kuhlen_2002_universalaccess}:
\begin{enumerate}
\item Elektronische Informationen sind frei zugänglich und die Konzepte der individuellen Autorenschaft und des geistigen Eigentums werden zu Relikten aus bürgerlichen Vorinformationsgesellschaften
\item Wissen und Informationen sind kontrolliert und dem Markt ohne politische Gegensteuerung überlassen: Die Kommerzialisierung und Zonierung von Wissen und Information wird umfassend sein und den Alltag bestimmen.
\item Wissen und Information werden über koexistente oder Paralleluniversen organisiert: Das Wissens als Produkt ist frei, öffentlich zugänglich und nutzbar. Es bleibt aber genug Spielraum bei der Adaption, Beratung, Veredlung oder anderen Mehrwertleistungen einer kommerziellen Informations- und Wissenswirtschaft
\end{enumerate}

---- TODO: weiter ausarbeiten und Einfluss auf die Enwicklung der eigenen Position darstellen ----

\section{Herausforderungen an die wissenschaftliche Gemeinschaft und an das System Universität}

\begin{quote}
\textbf{"Die Freiheit von Fremdbestimmung verpflichtet die wissenschaftliche Gemeinschaft und ihre Mitglieder zu verantwortlicher Selbstbestimmung."}
\end{quote} \cite{Oezmen_2015}

Sollten die Zeiten des "stürmischen Wachstums der Wissenschaft endgültig vorüber" sein \cite{K_lbel_2002}, lässt sich ein Grund dafür in der Geschlossenheit des wissenschaftlichen Kommunikationssystems finden. Mit der Verpflichtung auf Offenheit darf aber der Wissenschaft nicht die Unabhängigkeit und auch nicht "die Fähigkeit genommen werden, "Nein" zu sagen" \cite{suchen_Hornbostel_2006}. Unter Berücksichtigung dieser Faktoren ist zu vermuten, dass der neue Wachstum an Wissen durch den digitalen Wandel und die Öffnung wissenschaftlicher Kommunikation wirklich zu besseren Bedingungen für Schaffung neuen Wissens und die Bewahrung alten Wissens führen wird.

Läutete der Buchdruck die Moderne ein und legte den Grundstein für die wissenschaftliche Kommunikation wie wir sie heute kennen, könnte im Rahmen der Digitalisierung eine erneute Revolution des wissenschaftlichen Systems bevorstehen. Die unmittelbare und umfassende Bereitstellung der wissenschaftlichen Kommunikation im Rahmen der alltäglichen wissenschaftlichen Arbeit unter den Bedingungen von größtmöglicher, stellt das wissenschaftlichen System aber innerhalb und außerhalb vor neue Herausforderungen.

---- TODO: Diskussion um unterschiedle Konzepte/Verstaendnis von positive und negative liberty \cite{kelty_2014_freedom} auf Openess in der wissenschaftlichen Kommunikation ueberttragen & Wir werden die Übermittlung von Wissen bei der wissenschaftlichen Kommunikation nur dann verstehen können, wenn die Übermittlungswege und Formen transparent und offen gestaltet sind. \cite{davis_2011_open} ----

Diese Entwicklung kann und muss von der wissenschaftlichen Community gestaltet werden, wenn sie nicht machtlos den Kräften ausgeliefert sein will, "die von außen auf die Publikation und Rezeption ihrer Schriften einwirken" \cite{Hirschi_2015_buch_oa}. Die Frage ist, ob sie entscheidet die Digitalisierung als Gefahr für den Fortbestand der Wissenschaft einfach nur zu negieren, als rein digitales Abbild der analogen Realität der wissenschaftlichen Kommunikation zu verstehen und das aktuelle System mit all seinen Vor- und Nachteilen zu bewahren oder wagen es die Wissenschaftler und Wissenschaftlerinnen eine zweite wissenschaftliche Revolution einzuläuten, die zu einer umfassenden Wissensverbreitung an die Gesamtgesellschaft und zu einer grundlegenden Veränderung des aktuellen wissenschaftlichen Systems führen könnte?

Entzieht sich die wissenschaftliche Community dieser Auseinandersetzung, ist davon auszugehen, dass langfristig die Freiheit von Wissenschaft und Forschung darunter leidet und zunehmend rein politische und wirtschaftliche Interessen darüber entscheiden \cite{Warnke_2012}, wie, wann, wo und wozu Wissenschaftler und Wissenschaftlerinnen in Zukunft kommunizieren werden. Wenn Wissenschaftler und Wissenschaftlerinnen sich in der Auseinandersetzung mit der Forderung nach Öffnung von Kommunikation vornehmlich mit den Herausforderungen des Karrieredrangs und wirtschaftlichen Eigeninteressen befassen \cite{resnik_2005_ethics} besteht die Gefahr einer weiteren Verschließung wissenschaftlicher Kommunikation.

Die Auseinandersetzung bezieht sich auch auf die Anpassungen bei der Qualitätssicherung und auf Veränderungen im Rahmen des wissenschaftlichen Reputationssystems. Diese Anpassungen der Methoden um gute wissenschaftliche Praxis bei der Digitalisierung wissenschaftlicher Arbeit hin zu einem guten Standard in der Wissenschaft, obliegt ebenfalls der Wissenschaftsgemeinschaft selbst. Vor allem weil diese Kriterien einen direkten und unmittelbaren Einfluss auf die Bewertung und Steuerung der Wissenschaft selbst hat, muss die Wissenschaftsgemeinschaft Verantwortung übernehmen und diesen Prozess aktiv gestalten.

Wissenschaftler den unvermeidlichen Wandel im Rahmen der Digitalisierung gestalten und über Interessensvertretungen  Verlage wie Elsevier und andere wirken seit Dekaden auf die forschungspolitischen Agenda ein und versuchen ihre wirtschaftlichen Interessen durchzusetzen. Die Entwicklungen zeigen, dass auch sie diesen Wandel erkannt haben, und versuchen ihre kompfortable Situation in dem System der wissenschaftlichen Kommunikation zu sichern, wenn nicht sogar auszubauen.

Die Öffnung wissenschaftlicher Kommunikation kann somit als "vielleicht kostbarste Geschenk des Internet an die Wissensgesellschaft" betrachtet werden, wenn sie nicht ausschließlich durch die "ökonomischen Interessen des Informationskapitalismus" \cite{hagner_2015_sache_buches} gelenkt wird, sondern sich (weiterhin) maßgeblich auf die Aufgabe konzentriert dem gesamtgesellschaftlichen Auftrag des Wissenschaftssystems gerecht zu werden. Es bleibt auch eine Notwendigkeit, dass jeder Autor und jede Autorin selbst, aber unter Kenntnisstand der Konsequenzen für die Zugänglichkeit durch andere, entscheidet ob und wie seine oder ihre Forschungsergebnisse erscheinen.

Zusammenfassend kann konstatiert werden, dass es bisher weder gelungen ist, Anreize für die einzelnen Wissenschaftler so zu setzen, dass deren Eigeninteresse mit dem Wohl der Wissenschaft und dem der Öffentlichkeit harmonieren, noch führten staatlichen Interventionen zu einer fundamentalen Veränderung im Publikationsverhalten. Die Verlage und ihre Forderung, dass "Autoren in einem gesunden, unverzerrten freien Markt frei wählen sollten, wo sie publizieren" \cite{Brussels_Declaration_2007} kann demnach als ein Faktor für die schleichende Entwicklung vermutet werden.

25 Jahre nach den ersten Versuchen den Open Access mit Hilfe digitaler Netze umzusetzen gibt es kaum noch Zweifel, dass sich diese Entwicklung weltweit durchsetzen wird. Die Zweifel bestehen allerdings bei der Ausgestaltung der Herausforderungen und "bei der Frage wie sich Vor- und Nachteile zueinander verhalten" \cite{hagner_2015_sache_buches}. "Eine Weise Politikgestaltung in diesem sensiblen Bereich muss besonders acht auf die komplexen Geschichte der Organisationen der öffentlichen Wissenschaft geben und die potenzielle Fragilität der eigentümlichen institutionellen Matrix respektieren in der sich die modernen Forschung entwickelt hat und aufgeblüht."\cite{david1998_common}. Darüber hinaus müssen die Rahmenbedingungen ausgehandelt werden unter denen die Digitalisierung der wissenschaftlichen Kommunikation und die Öffnung von Wissenschaft und Forschung stattfindet \cite{mennes_2013_making_os}.

Die Forderung nach Offenheit von Wissenschaft und Forschung darf nicht nur als "Strategie" gegen die unterschiedlichen Krisen im wissenschaftlichen Kommunikationssystem verstanden werden, die maßgeblich STM-Forschern gefördert wird und im Ergebnis zu einer weiterhin polarisierende Abwehr- und Gegenreaktion führen würde, die zu einer weiteren Verwässerung und Fehlleitung der ursprünglichen Ansätze zur Folge hätte \cite{naeder_2010_open}. Sind wissenschaftlichen Aktivitäten nicht offen und zugänglich, steigt die Gefahr, dass die öffentliche Unterstützung für die Wissenschaft erodiert und die Menschen vertrauen in sie verlieren \cite{resnik_2005_ethics}.

Sicher ist demnach auch, dass es nicht allein bei der Forderung nach dem Zugang zu wissenschaftlichen Publikationen bleiben wird. Die Umsetzung von Open Access wird früher oder später auch in einer weiteren Öffnung des wissenschaftlichen Erkenntnisprozesses münden. Wie bei Open Access ist auch hier die wissenschaftliche Gemeinschaft gefragt die Ausgestaltung aktiv und konstruktiv-kritisch zu beeinflussen.

Wir befinden uns bei diesem Aushandlungsprozess erst am Anfang: Mehr als 500 Jahre Buchdruck stehen nur 25 Jahre Internet gegenüber. Vor 350 Jahren waren es Wissenschaftler, die sich zusammengetan haben um eine neue Philosophie für die Förderung von Wissen zu etablieren und das erste wissenschaftliche Journal zu gründen. Die Transformation des wissenschaftlichen Kommunikationssystems von der Gutenberg-Galaxie in den Turing-Galaxis verlangt eine Neugestaltung der Rahmenbedingungen für die wissenschaftliche Kommunikation und allen Beteiligten eine Neudefinition ihrer Rolle in diesem System ab. Die neuen Möglichkeiten unterschiedlicher Formen der Darstellung von wissenschaftlichen Informationen kann dabei als neue Chance für eine aktive Verbesserung, Gestaltung und Modifikation wissenschaftlicher Kommunikation verstanden und genutzt werden, wenn die Beteiligten ihre Rolle als aktive Gestalter und Gestalterinnen wahrnehmen.

---- TODO: weiter ausarbeiten ----

\section{Ausblick und Anknüpfungspunkte für weitere Forschungsbemühungen}

Als wichtige Felder für zukünftige Evaluationen sind die Themen Datenschutz und sowie die Gefahr, dass nützliche Forschungsergebnisse auch zu schädlichen, etwa kriminellen oder terroristischen Zwecken missbraucht werden könnten \cite{Fritsch_2015} zu benennen. Den Schutz der Privatsphäre gegen den immensen Wert von Open-Access-(Daten)nutzung auszugleichen und auszuhandeln stellt eine wichtige zukünftige Herausforderung dar. Dabei sollten nicht nur die Vorteile für alle Beteiligten, sondern auch nicht sofort überschaubare Auswirkungen berücksichtigt werden. Dafür benötigt es einen Aushandlungsprozess zwischen der wissenschaftlichen Gemeinschaft, der Politik und der Gesellschaft.

Auch die Herausforderung, das als Resultat der Öffnung der wissenschaftlichen Kommunikation und einer daraus möglicherweise resultieren Konkurrenz der wissenschaftlichen und medialen Kommunikation miteinander, die das Wahrheitsmonopol der Wissenschaft durch das Aufmerksamkeitsmonopol der Medien außer Kraft beeinflussen könnte \cite{weingart_2005_wissenschaft}, stellt einen Ansatzpunkt für weitere Analysen dar.

---- TODO: weiter ausarbeiten und überarbeiten, ist bisher zu pathetisch und moegliche weitere Forschungsfelder aufzeigen & aufgreifen: "We have reached a period in science somewhat simiular to that encountered by our colleagues of 300 years ago. Creative and inventive minds must now discover new methods for coping with the scienctific literature" \cite{porter_1964_scientific}; Dispositiv; "Wir werden der technologischen Zukunft nicht dadurch Herr, dass wir uns von ihr abwenden. Die Verantwortung besteht darin, ihre Logik zu verstehen, um möglichst viele ihrer Auswirkungen vorwegzunehmen. Ein Diskurs über die Zwecke und Werte, der sich nicht auf einen präzisen Zustand der zur Verfügung stehenden Mittel stützt, ist ein leerer Diskurs. Aber ein Diskurs über die Innovation, der diese nicht im Lichte der Erinnerung genau untersucht, ist ein Diskurs." \cite{naeder_2010_open}; Weitere Umstrukturierung der ökonomischen und juristischen Rahmenbedingungen des wissenschaftlichen Publizierens sind nötig \cite{naeder_2010_open};  Neugestaltung der Qualitätssicherungsmaßnahmen und Selektionsmechanismen notwendig; Gefahr Predatory Journals \cite{Beall_2012}; Offenheit in Wissenschaft und Forschung addressiert den Kern der Produktion von Wissen und das betrifft uns alle. \cite{Mussell_2013}; Offene Schreibweise gewohnte Arbeitsweise gegenüberstellen. Z.B: Wer gewohnt ist mit Literaturverwaltungsprogramme wie Citavi oder Zotero zu arbeiten, wird auf Grund die fehlenden Implementierung dieser Programme  durch mehr manuelle Arbeit kompensieren müssen; Die Arbeit auf dem eigenen Rechner in einem geschlossenen Umfeld ist noch immer viel einfacher als das öffentliche Verfassen der Arbeit und wird demnach noch immer von Wissenschaftlern und Wissenschaftlrinnen bevorzugt. ; Popper einarbeiten; Wertschöpfungsprozess vs. Erkenntnisprozess <- siehe Einleitung; Anstatt den aussichtslosen Weg der gemeinsamen Definition von Öffnung wissenschaftlicher Kommunikations zu gehen, könnte man versuchen die Geschlossenheit des Systems zu definieren und sagen, welchen Zustand "man" definitiv nicht mehr haben will. Das würde den unterschiedlichen Wegen gerecht werden und man könnte diese daraufhin abprüfen - quasi aus den Erfahrungen der letzten Jahre "Öffnung wissenschaftlicher Kommunikation" eine Closed Definition ableiten <- vielleicht hier einbauen? ----
