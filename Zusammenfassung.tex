\chapter{Zusammenfassung und Ausblick}

\begin{quote}
\textbf{"When [...] the profession can no longer evade anomalies that subvert the existing tradition of scientific practice — then begin the extraordinary investigations that lead the profession at last to a new set of commitments, a new basis for the practice of science. The extraordinary episodes in which that shift of professional commitments occurs are the ones known in this essay as scientific revolutions."}
\end{quote} \cite[:6]{Kuhn_2012}

Wissenschaft und Forschung sind eng mit den Normen der schnellen Weitergabe von Forschungsergebnissen, einer Umgebung des Wissensaustauschs, Co-Autorenschaft und dem kumulativen Lernen sowie Innovationen verbunden \cite{Partha_1994}. Folglich scheint eine möglichst unbeschränkte und offene wissenschaftliche Kommunikation für das Wissenschaftssystem theoretisch unverzichtbar. In der wissenschaftlichen Realität basiert die wissenschaftliche Arbeit jedoch weitgehend auf einem von der Gesamtgesellschaft abgeschlossenen System und beruht noch immer auf der Annahme, dass das "was nicht gedruckt wird, kaum Chancen hat, die Entwicklung des Faches zu beeinflussen" \cite{Luhmann_1997}.

In dieser Arbeit wurden die Herausforderungen an das System der wissenschaftlichen Kommunikation im Rahmen der Digitalisierung und die Forderung nach Öffnung dieser Kommunikation umfassend dargestellt und analysiert. Die Entwicklungen im Bereich der Öffnung wissenschaftlicher Kommunikation wurden aus geistes- und kulturwissenschaftlicher Perspektive genauer untersucht und den bisherigen Erkenntnissen über die Öffnung wissenschaftlicher Kommunikation gegenübergestellt sowie das Ergebnis dieser Gegenüberstellung diskutiert.

Mithilfe der quantitativen Methode einer Online-Befragung wurde unter 1.112 Wissenschaftlern und Wissenschaftlerinnen analysiert, welche Auffassungen und Annahmen in Bezug auf den postulierten Wandel wissenschaftlicher Kommunikation im Rahmen von Offenheit und Digitalisierung vorherrschen und inwiefern diese mit anderen Aspekten des wissenschaftlichen Kommunikationssystems korrelieren. Die gewonnenen Erkenntnisse wurden in den Kontext bisheriger Untersuchungen und der autoethnographisch erarbeiteten offenen Kommunikation im eigenen wissenschaftlichen Erkenntnisprozess bei der Anfertigung dieser Arbeit gestellt.

Zusammenfassend kann als Ergebnis der durchgeführten Untersuchungen festgestellt werden, dass es bisher weder gelungen ist, die nötigen Anreize für die einzelnen Wissenschaftler und Wissenschaftlerinnen so zu setzen, dass deren Eigeninteresse in Bezug auf die Verbreitung von Erkenntnissen mit dem Wohl der Wissenschaft und dem der Öffentlichkeit gleichermaßen harmonieren, noch gab es bisher staatliche Interventionen, die zu einer fundamentalen Veränderung im Publikationsverhalten geführt haben. Das wissenschaftliche Kommunikationssystem bleibt bisher stabil, die Kommunikationsformate wie Monographie und Journal behalten noch immer ihren hohen Stellenwert und die zunehmende Nutzung digitaler Werkzeuge führt bisher zu keiner strukturellen Veränderung von Wissenschaft. Bisher ebenfalls unbeantwortet bleibt die Frage, inwiefern und in welchem Umfang die Öffnung des gesamten wissenschaftlichen Erkenntnisprozesses einen wünschenswerten Schritt darstellen, welche möglichen Nebenfolgen durch eine offene Wissensproduktion entstehen und ob es sich bei den postulierten Veränderungen um eine wissenschaftliche Revolution oder um kleinere Anpassungen an die bestehenden Paradigmen und traditionsgebundenen Aktivitäten der Wissenschaft handelt.

Sicher hingegen ist, dass es sich bei den aktuellen Entwicklungen um die Vorläufer eines umfassenden Medienwandels handelt, der neue Möglichkeiten und Chancen aber auch Herausforderungen für die Wissenschaft eröffnet. Diese Entwicklungen bieten neue Möglichkeiten für die aktive Veröffentlichung von Supplementen und (Roh-)Daten, sie unterstützen die Bereitschaft der Forschenden, positive wie negative Daten zu teilen, zurückgezogene Artikel sichtbar zu machen und den wissenschaftlichen Erkenntnisprozess dahingehend zu öffnen, so dass notwendige effektive Mechanismen zur Verfolgung wissenschaftlichen Fehlverhaltens installiert und die bestehenden Mechanismen zur Selbstkorrektur gestärkt werden können. In der notwendigen Auseinandersetzung muss jedoch auch behandelt werden, welche Aspekte zusätzlich oder anstatt der traditionellen wissenschaftlichen Reputation für Wissenschaftler und Wissenschaftlerinnen an Relevanz gewinnen, wie vernetzte Computer und Algorithmen eingesetzt werden, um der ansteigenden Verfügbarkeit an Informationen als Folge der Überwindung der erzwungenen Datenreduktion analoger Medien gerecht zu werden, wie sich in diesem Zusammenhang die Wahrheitskriterien ändern, welche Möglichkeiten sich wie weit für eine Neuordnung der wissenschaftlichen Kommunikation anbieten und wie die wissenschaftliche Gemeinschaft diese Technologien und Geschäftsmodelle – vielleicht auch zusammen mit Bewegungen wie der Open-Source-Bewegung – prägt. Dabei muss neu verhandelt werden, welche Werte für die wissenschaftliche Praxis durch neue Medientechnologien definiert oder von der Bewegung für das freie und offene Netz übernommen werden können.

Neben der Erkenntnis, dass die Hindernisse für die Veränderungen nicht ausschließlich technischer oder finanzieller, sondern auch sozialer Natur sind \cite{Nosek_2012}, muss weiterhin die klare rechtliche Klärung für die Zweitverwertung von Inhalten als ein wichtiger Katalysator für die weitere Entwicklung gesehen werden. Darüber hinaus fehlen noch immer etablierte Reputationsmechanismen, die die Öffnung wissenschaftlicher Kommunikation befördern. Diese Mechanismen können aber nur dann erfolgreich sein, wenn eine Diskussion über die Gestaltung der Zukunft wissenschaftlicher Kommunikation innerhalb der wissenschaftlichen Gemeinschaft unterstützt und die wissenschaftlichen Institutionen für diese Diskussion den notwendigen Raum und die Anreize schaffen.

Wird dieser Raum nicht geschaffen und die Öffnung der wissenschaftlichen Kommunikation weiterhin maßgeblich über externe und politisch motivierte Maßnahmen angestrebt, ist zu vermuten, dass rein kommerzielle, forschungs- und steuerungspolitische Interessen sowie das Interesse an der Förderung von Wissenstransfer und wirtschaftlicher Verwertung von wissenschaftlichen Inhalten negative Konsequenzen auf das Wahrheitsmonopol und die Unabhängigkeit der Wissenschaft haben werden. Verlage wie Elsevier und andere wirken in diesem Zusammenhang seit Dekaden auf die forschungspolitische Agenda ein und versuchen ihre wirtschaftlichen Interessen im Rahmen des Wandels durchzusetzen \cite[:15]{Hirschi_2015} \cite{Elsevier_2012}. Die wissenschaftliche Gemeinschaft muss durch das Hinterfragen der bestehenden Kriterien wissenschaftlicher Arbeit, das Experimentieren mit den neuen Möglichkeiten der Kommunikation, die Förderung der Katalysatoren für die Öffnung und die Beseitigung der Hindernisse den unvermeidlichen Wandel im Rahmen der Digitalisierung eigeninitiativ gestalten. Wenn in diesem Prozess schon Verlage fordern, dass "Autoren in einem gesunden, unverzerrten freien Markt frei wählen sollten, wo sie publizieren" \cite{Brussels_Declaration_2007}, muss sich die wissenschaftliche Gemeinschaft fragen lassen, ob sie oder Verlage die Ausgestaltung des wissenschaftlichen Kommunikationssystems übernehmen sollten und ob sie in der Vergangenheit die Publikationsfreiheit ausreichend und selbstbestimmt mit dem Ziel der möglichst umfassenden Verbreitung von Wissen genutzt haben.

\section{Die Öffnung wissenschaftlicher Kommunikation und der wissenschaftliche Alltag}

Die Ergebnisse der im Rahmen dieser Arbeit durchgeführten Befragung zeigen eine mehrheitliche Zustimmung und ein überwiegend großes Interesse an der Öffnung wissenschaftlicher Kommunikation. Im wissenschaftlichen Alltag haben dieses Interesse und die Zustimmung zu digitalen und offenen Verfahren der Kommunikation bisher jedoch noch immer nicht zu einer fundamentalen Veränderung des Publikations- und Veröffentlichungsverhaltens geführt. Dem theoretischen und ideellen Interesse an der Öffnung steht somit ein praktisches Alltagsdesinteresse an der Auseinandersetzung mit dem Thema gegenüber.

Ein Grund dafür liegt in der Verortung der Entwicklung der Forderungen um die Öffnung wissenschaftlicher Kommunikation. Die ersten offenen Publikationsvorhaben und Entwicklungen zu Open Access erfolgten aus den STM-Fächern, die von der Zeitschriftenkrise viel stärker und früher betroffen waren als andere Fächer. Die aus dieser Entwicklung resultierenden Erklärungen und Bestrebungen führten bisher allerdings zu erheblichen "Vorbehalten hinsichtlich der Sinnhaftigkeit und Durchführbarkeit von Open Access, die wiederum zu Desinteresse oder Polarisierung bei vielen Vertretern dieser Disziplinen" \cite{Naeder_2010} führte. Diese Polarisierung stellt weiterhin eine große Herausforderung für die Etablierung und Verbreitung der Konzepte für die Öffnung wissenschaftlicher Kommunikation dar. Derzeit ist eine gemeinsame Darstellung des Interesses an der Öffnung wissenschaftlicher Kommunikation über alle wissenschaftlichen Fachbereiche hinaus und ein gemeinsames Handeln der wissenschaftlichen Gemeinschaft nur schwer vorstellbar.

Sucht man nach weiteren Gründen für das Desinteresse an Offenheit bei der praktischen Umsetzung im wissenschaftlichen Alltag, wird deutlich, dass unter anderem unvollständiges Wissen über die wirtschaftlichen Aspekte wissenschaftlicher Informationsversorgung für die Diskrepanz zwischen dem Interesse an Offenheit und der tatsächlichen Publikationspraxis eine Rolle spielt. In der untersuchten Literatur wird diese Diskrepanz mit der komfortablen Situation der Wissenschaftler und Wissenschaftlerinnen in einem System begründet, in dem für die meisten Mitglieder der wissenschaftlichen Gemeinschaft kein oder nur ein geringer unmittelbarer Anreiz besteht, sich aktiv mit dem Publikationssystem und möglichen Veränderungen zu beschäftigen, weil sie weder die Kosten des Publikationssystems tragen \cite{Sietmann_2007}, noch eine Auseinandersetzung mit den finanziellen Aspekten als notwendig erachtet wird \cite{Herb_2010}. Darüber hinaus kommt es im Zuge der zunehmend quantitativen Bewertung von Forschungsleistungen und der vornehmlichen Berücksichtigung quantitativer Indikatoren (deren Qualitätsmessung nach wissenschaftlichen Methoden kritisiert werden kann) bei der Bewertung für die Mittelzuweisung zu einer Entwicklung von Fehlanreizen, bei denen die reine Anzahl und nicht die Qualität der Veröffentlichung in Publikationen etablierter Verlage bei wissenschaftlicher Arbeit im Vordergrund stehen.

Der offenbar geringe Bedarf an Veränderung im Alltag der Wissenschaftler und Wissenschaftlerinnen kann als weiterer Grund für die schleppende Umsetzung genannt werden. Es gibt kaum positive Anreize, sich mit der Entscheidung, wo und wie veröffentlicht wird oder mit den Konsequenzen dieser Entscheidung auseinanderzusetzen, da diese im wissenschaftlichen Reputationssystem bisher nicht oder nur unzureichend abgebildet und honoriert werden können oder anders bewertet werden. Neben den begrenzten finanziellen und zeitlichen Ressourcen, sich mit den umfangreichen Aspekten der Informationsversorgung auseinanderzusetzen, gibt es zudem weiterhin rechtliche Unsicherheiten, die die Wissenschaftler und Wissenschaftlerinnen davon abhalten, ihre Kommunikation zu öffnen.

Für die betuliche Umsetzung des Konzepts von Offenheit in der wissenschaftlichen Kommunikation allein eine Kombination aus Beharrungsvermögen der wissenschaftlichen Akteure, Unsicherheit über alternative Publikationsmodelle und die rechtliche Situation sowie die Existenz von Gruppen, die in die Bewahrung der Ineffizienz des gegenwärtigen Systems investiert haben, auszumachen, ist unzureichend. Die eingangs genannten Unterschiede und fachspezifischen Eigenheiten der wissenschaftlichen Kommunikation, wie zum Beispiel die unterschiedlichen Publikationsformen, müssen als unterschätzte Herausforderung betrachtet werden. Möglichkeiten und Debatten bleiben bisher bewusst überwiegend auf die jeweilige Disziplin beschränkt und berücksichtigen selten andere Forschungsrichtungen.

Christopher Kelty macht für das alltägliche Desinteresse an der Praktizierung von Offenheit im wissenschaftlichen Kommunikationssystem und Alltag folgende zwei Aspekte aus: Erstens ist die Auseinandersetzung mit Offenheit außerordentlich komplex und zweitens ziemlich langweilig \cite[:203]{Kelty_2014a}. Darüber hinaus würden durch die umfassende Auseinandersetzung mit dem wissenschaftlichen Publikationsmarkt "Praktiken auf dem Spiel stehen, die dem Tun vieler Geisteswissenschaftler (...)" und auch Wissenschaftlern anderer Fächer, "Sinn und Legitimität zu verleihen scheinen" \cite[:6]{Hirschi_2015}, die lieber nicht hinterfragt werden.

Die Ergebnisse der durchgeführten Befragung nähren diese Befürchtungen und lassen vermuten, dass das Gros der Wissenschaftler und Wissenschaftlerinnen auch in naher Zukunft keine hervorgehobene Rolle bei der Gestaltung der Transformation der Kommunikation einnehmen wird und trotz der Möglichkeiten zur Veränderung eher der Status quo bewahrt wird \cite{Nosek_2012}. Diese Befürchtung ist zusätzlich im Spannungsfeld zwischen Wissenschaft und Politik zu sehen, in dem Wissenschaftler und Wissenschaftlerinnen für sich in Anspruch nehmen, als Strategen im politischen Kampf um Glaubwürdigkeit für die wissenschaftliche Arbeit zu agieren \cite{Latour_2013}. Diese Rolle nehmen sie aktuell allerdings nur unzureichend wahr. Eine tiefgreifende Debatte innerhalb der wissenschaftlichen Gemeinschaft über die historischen und aktuellen Entwicklungen der Medien der wissenschaftlichen Kommunikation sowie deren Konsequenzen auf den wissenschaftlichen Alltag findet bisher nicht oder nur im begrenzten Maße statt. Anstatt einer umfassenden Diskussion und der aktiven Gestaltung der Entwicklungen aus dem wissenschaftseigenen Wunsch nach Verbesserungen bei der Verbreitung wissenschaftlicher Erkenntnisse heraus, wie mithilfe der Drucktechnologie im 17. Jahrhundert, beschränkt sich die Debatte im 21. Jahrhundert trotz ähnlicher technologischer Fortschritte noch immer primär auf (Verlust-)Ängste.

Die Auseinandersetzung mit den Katalysatoren und Hindernissen bezieht sich ferner auf die Anpassungen bei der Qualitätssicherung und auf Veränderungen des wissenschaftlichen Reputationssystems sowie deren Konsequenzen. Die Öffnung wissenschaftlicher Publikationen kann nur als erster Schritt in Richtung einer Öffnung des gesamten wissenschaftlichen Erkenntnisprozesses verstanden werden. Die dabei entstehenden notwendigen Anpassungen der Methoden, um gute wissenschaftliche Praxis bei der Digitalisierung wissenschaftlicher Arbeit zu einem Standard in der Wissenschaft zu machen, obliegt ebenfalls der Wissenschaftsgemeinschaft selbst. Vor allem in Anbetracht der Tatsache, dass diese neu zu verhandelnden Kriterien für die wissenschaftliche Praxis unter dem Einsatz neuer Medientechnologien sowie den Bedingungen von Offenheit einen direkten und unmittelbaren Einfluss auf die Bewertung und (Selbst-)Steuerung der Wissenschaft haben, muss die Wissenschaftsgemeinschaft Verantwortung übernehmen und diesen Prozess aktiv gestalten. Diese Verantwortung sollte nicht durch die Wahl "gar nicht mehr auf der Universitätsseite zu erscheinen oder keine jährlichen Geldmittel für ihre Publikationsergebnisse zu erhalten" \cite{Warnke_2012} entschieden werden, sondern durch eine aktive Ausgestaltung der Rahmenbedingungen der wissenschaftlichen Kommunikation und damit auch der Steuerung des wissenschaftlichen Systems.

Die Öffnung der wissenschaftlichen Kommunikation ist mit vielen Anstrengungen für die wissenschaftliche Gemeinschaft verbunden, kann aber als "vielleicht kostbarstes Geschenk des Internets an die Wissensgesellschaft" betrachtet werden, wenn sie nicht ausschließlich durch die "ökonomischen Interessen des Informationskapitalismus" \cite[:65]{Hagner_2015} gelenkt wird, sondern sich (weiterhin) maßgeblich auf die Aufgabe konzentriert, dem gesamtgesellschaftlichen Auftrag des Wissenschaftssystems gerecht zu werden. Jeder Autor und jede Autorin entscheidet selbst, allerdings unter dem Kenntnisstand der Folgen und Konsequenzen für die Zugänglichkeit durch andere, ob und wie er seine oder sie ihre Forschungsergebnisse verbreitet. Die Erfüllung akademischer Erwartungen für die Veröffentlichung von Erkenntnissen unter den Regeln der jeweiligen Disziplin, unter der Verwendung der gegebenen intellektuellen Rahmenbedingungen und dem Anspruch, mit Kollegen und Kolleginnen zurechtzukommen, macht es allerdings sehr einfach, den sozialen Idealismus und die mehrheitliche Bekenntnis zu Offenheit \cite[:66]{Hagner_2015} auf den akademischen Syllabus zu beschränken und sich nicht an (aktivistischen) Veränderungsprozessen zu beteiligen \cite[:25]{Flood_2013}. Diese Diskrepanz zwischen dem Bekenntnis der Wissenschaftler und Wissenschaftlerinnen zu Offenheit bei der wissenschaftlichen Kommunikation und der tatsächlich praktizierten offenen Kommunikations- und Arbeitsweise konnte erneut durch die Ergebnisse der Befragung im Rahmen dieser Arbeit belegt werden (siehe Kapitel 5).

\section{Katalysatoren und Hindernisse für die Etablierung der Öffnung wissenschaftlicher Kommunikation}

Nach der Einführung in den Themenbereich der wissenschaftlichen Kommunikation und des wissenschaftlichen Publizierens wurden anhand aktueller Literatur die Entwicklung und die Debatte um die Forderung nach Öffnung der wissenschaftlichen Kommunikation dargestellt. In den geschilderten Debatten wurden Treiber und Bremser für die Entwicklungen identifiziert und herausgearbeitet. Diese wurden daraufhin mithilfe einer Online-Befragung unter 1.112 deutschsprachigen Wissenschaftlern und Wissenschaftlerinnen detailliert evaluiert. Dabei wurde analysiert, inwiefern ein Verständnis von Offenheit im Rahmen der wissenschaftlichen Kommunikation besteht und inwieweit das Interesse an Offenheit in den unterschiedlichen wissenschaftlichen Fachdisziplinen verbreitet ist und praktiziert wird. Die Ergebnisse wurden mit einer Studie des Soziologischen Forschungsinstituts Göttingen aus dem Jahr 2007 verglichen, um Trends und Entwicklungen zu identifizieren.

In den Augen der befragten Wissenschaftler und Wissenschaftlerinnen stehen die Beschleunigung der Wissensverbreitung (64 Prozent) und die offene Verfügbarkeit bereits finanzierter Forschung für alle (55 Prozent) vor allem dem Fehlen von etablierten Reputationskriterien für die Bewertung von offener Wissenschaft (43 Prozent), der Gefahr der Fehlinterpretation und Falschinformation (40 Prozent) sowie einem erhöhten zeitlichen Mehraufwand für die Bereitstellung der wissenschaftlichen Publikationen und/oder Forschungsdaten (34 Prozent) gegenüber. Die 1.112 Befragten gaben außerdem mehrheitlich an, dass sie rechtliche Bedenken (39 Prozent) und Unwissenheit über die Erlaubnis (29 Prozent) davon abhält, wissenschaftliche Inhalte ohne finanzielle, rechtliche oder technische Barrieren öffentlich zur Verfügung zu stellen.

Die Ergebnisse der Studie können insofern als repräsentativ gelten, als dass sie auf einer großen Stichprobe beruhen und an die Ergebnisse von vergleichbaren Studien anknüpfen (siehe zum Beispiel \cite{Hanekop_2007}). In der Auswertung der Daten wurde vor allem deutlich, dass die Hindernisse eine größere Verteilung aufwiesen als die Katalysatoren und dass die Verbreitung offener wissenschaftlicher Kommunikationsverfahren nicht unwesentlich mit der Fachrichtung der jeweiligen Autoren und Autorinnen korreliert. Darüber hinaus scheint ungeachtet der vielfältigen Kritik am aktuellen Publikations- und Kommunikationssystem dieses auch nach zwei Jahrzehnten noch immer weitestgehend stabil zu sein. Die Ergebnisse der Befragung belegen die in der Literatur immer wieder beschriebene Diskrepanz zwischen dem Interesse nach und dem Verständnis für Offenheit sowie der tatsächlich praktizierten offenen Kommunikations- und Arbeitsweise \cite{Yiotis_2005} \cite{Bartling_2013} \cite{Hagner_2015} \cite{Fecher_2015}.

Viele Programme der Forschungsförderungsorganisationen zielen noch immer darauf ab, fast ausschließlich klassisch zu publizieren, und nur sehr langsam kommt es zur Unterstützung von digitaler Forschungsinfrastruktur oder Softwareentwicklungsprogrammen \cite{Hey_2015}. Sie behindern damit die nötigen Veränderungsprozesse im Rahmen der Digitalisierung des wissenschaftlichen Alltags und befördern das Beharrungsvermögen, das aktuelle wissenschaftliche Kommunikationssystem zu unterstützen. Förderorganisationen müssen jedoch die Veränderungsprozesse in der wissenschaftlichen Praxis durch die neuen Medientechnologien akzeptieren und ihre Verantwortung wahrnehmen, indem sie die zusätzlichen Ressourcen, die für die Schaffung der strukturellen Grundlagen, die mit der Öffnung von Wissenschaft und Forschung verbunden sind, zur Verfügung stellen \cite{Mennes_2013} \cite{Patlak_2010}. Um diese Bemühungen voranzubringen, müssen sich Förderorganisationen mehr denn je ihrer Rolle als "einflussreiche Akteure im komplexen und sich wandelnden Markt für wissenschaftliche Publikationen" \cite[:287]{Wein_2010} bewusst werden und entscheiden, ob sie die Umsetzung der gemeinsamen Nutzung von Daten durch Incentivierung fördern \cite{Mennes_2013}. Bisher ist diese Nutzung wissenschaftlicher Daten nur sehr gering verbreitet und auch wenn ihre Erwähnung im Rahmen der Weiternutzung ansteigt, bleiben noch immer bis zu 86 Prozent der veröffentlichten Daten bisher ungenutzt beziehungsweise unzitiert \cite{Peters_2015}.

Die Entwicklungen der letzten Jahre zeigen auch, dass Verlage aktiv daran arbeiten den digitalen Wandel zu nutzen, um ihre komfortable Situation im System der wissenschaftlichen Kommunikation zu sichern, wenn nicht sogar auszubauen. Dabei stellen die technischen und finanziellen Defizite und Unzulänglichkeiten für die Umsetzung der ursprünglichen Ideale der Öffnung der wissenschaftlichen Kommunikation für die Gesamtgesellschaft im Rahmen der Digitalisierung nur einen Teil der Gründe für die noch immer vorherrschende Beständigkeit des aktuellen Systems und der starken Rolle von Verlagen bei der Neugestaltung der wissenschaftlichen Kommunikation dar. Würden die wissenschaftliche Gemeinschaft und die Fachgesellschaften wie auch die Bibliotheken stärker den Wandel weg von abonnementbasierten Modellen unterstützen, wäre die Entwicklung womöglich bereits weiter fortgeschritten \cite{Nosek_2012}.

25 Jahre nach den ersten Versuchen, einen offenen Zugang zu wissenschaftlicher Kommunikation mithilfe digitaler Netze umzusetzen und nach einer Vielzahl weicher Erklärungen und Bekenntnissen für die Öffnung wissenschaftlicher Kommunikation, gibt es kaum noch Zweifel, dass sich das System verändern wird. Dennoch bestehen weiterhin grundsätzliche Bedenken und Unklarheiten bezüglich der genauen Ausgestaltung von Offenheit wissenschaftlicher Kommunikation, der Frage, ob diese Öffnung noch dem ursprünglichen Anspruch auf Verbesserung des Systems gerecht werden kann, und insbesondere bezüglich der Frage, wie sich die Vor- und Nachteile dieser Entwicklung schließlich zueinander verhalten werden \cite{Hagner_2015}. Bisher hat sich ein Großteil der wissenschaftlichen Gemeinschaft eher verhalten aktiv an der Gestaltung des Veränderungsprozesses beteiligt. Sollte es durch das weitere Ausbleiben der Gestaltung aus der wissenschaftlichen Gemeinschaft heraus zu einem Eingriff der Politik in diesen "sensiblen Bereich" kommen, "muss besonders acht auf die komplexe Geschichte der Organisationen der öffentlichen Wissenschaft gegeben und die potenzielle Fragilität der eigentümlichen institutionellen Matrix respektiert werden, in der sich die moderne Forschung entwickelt hat und aufgeblüht ist" \cite{David_1998}. Bloßes teilnahmsloses Abwarten seitens der wissenschaftlichen Gemeinschaft birgt die Gefahr, dass die Organisationen und die Selbstständigkeit der öffentlichen Wissenschaft negativ beeinflusst werden. Darüber hinaus müssen die Rahmenbedingungen von allen Seiten ausgehandelt werden, unter denen die Digitalisierung der wissenschaftlichen Kommunikation im Einklang mit der Öffnung wissenschaftlicher Kommunikation stattfinden kann \cite{Mennes_2013}.

\section{Erkenntnisse aus dem offenen Verfassen der Arbeit}

Neben den theoretischen Hemmnissen bei der Etablierung der Konzepte um die offene Wissenschaft behindern ganz praktische Aspekte die möglichst umfassende und für jeden frei verfügbare Veröffentlichung der Informationen im Rahmen wissenschaftlicher Erkenntnisprozesse. Die zur Verfügung stehenden Plattformen und Applikationen sind im Jahr 2015 noch nicht ausgereift, etabliert und zweckdienlich genug, um ohne großen Mehraufwand offene Wissenschaft im Alltag zu praktizieren. Dabei gilt es zu berücksichtigen, dass die wissenschaftliche Arbeit trotz zunehmender Digitalisierung seit Dekaden auf die geschlossene Publikation und den nicht-öffentlichen Publikationsprozess ausgelegt ist sowie dem Druck der Etablierung des Marktmodus als dominante Governance-Form von Wissenschaft ausgesetzt ist.

Die Arbeit orientierte sich an der Forderung, dass der möglichst umfassende Zugriff auf den gesamten wissenschaftlichen Erkenntnisprozess inklusive aller Daten und Informationen, die bereits bei der Erstellung, Bewertung und Kommunikation der wissenschaftlichen Erkenntnisse entstanden sind und zur Reproduzierbarkeit der Ergebnisse beitragen, jederzeit gegeben sein soll. Das bedeutet jedoch nicht, dass jedes Protokoll oder jeder Ansatz veröffentlicht wurde. Dabei handelt es sich nur um die offene Kommunikation jeder Aktion im Rahmen der Anfertigung der Promotion, die zur Nachvollziehbarkeit der wissenschaftlichen Qualität und Erkenntnisse wichtig sind sowie zur Möglichkeit der Wiederholung des Erkenntnisprozesses beitragen. Als Erkenntnis des Experiments der offenen Anfertigung dieser Arbeit kann festgestellt werden, dass der offene wissenschaftliche Erkenntnisprozess nach den Forderung von Open Science und der Open-Definition zwar grundsätzlich möglich ist, die Möglichkeiten für die Anfertigung von offenen wissenschaftlichen (Qualifikations-)Arbeiten (bisher) aber dennoch als unzureichend zu bezeichnen sind. Darüber hinaus birgt diese Art der Anfertigung einer Qualifikationsarbeit noch immer die Gefahr, von der wissenschaftlichen Gemeinschaft oder der wissenschaftlichen Institution nicht anerkannt oder nicht akzeptiert zu werden.

Stellt man die gewohnte wissenschaftliche Arbeitsweise dem offenen Erstellungsprozess dieser Arbeit gegenüber, so muss die Arbeit auf dem lokalen Computer (selbst bei der Verwendung internetbasierter Dienste) in einem geschlossenen Umfeld noch immer als um vieles einfacher als das öffentliche Verfassen einer Arbeit bewertet werden. Das hat zum einen mit den gewohnten und etablierten strukturellen, technischen sowie rechtlichen Umgebungen wissenschaftlicher Arbeit zu tun, die überwiegend inkompatibel mit der offenen Darstellung und Verbreitung von Inhalten sind. Zum anderen müssen die fehlenden Möglichkeiten und Funktionen für die offene Arbeitsweise sowie die daraus resultierenden Einschränkungen bei der Bedienbarkeit und den Abläufen durch mehr Aufwand und manuelle Arbeit seitens der Forscher und Forscherinnen kompensiert werden.

So erscheint es fast verständlich, dass bisher nur eine Minderheit offene Webplattformen für die wissenschaftliche Kommunikation \cite{Perkel_2014} nutzt und die Mehrheit der im Rahmen dieser Arbeit befragten Wissenschaftler und Wissenschaftlerinnen bei der Öffnung von Forschungsdaten einen Mehraufwand befürchtet, wenngleich die Daten durch den zunehmenden Einsatz computerunterstützter wissenschaftlicher Verfahren bereits digital vorliegen. Ergänzend zeigen die Ergebnisse des Selbstexperiments, dass die Offenlegung des gesamten Erkenntnisprozesses bei der Erstellung dieser Arbeit ohne programmiertechnische Vorkenntnisse schwer bis nicht möglich gewesen wäre. So musste bei der Erstellung der Arbeit eigens Software programmiert werden, um dem Anspruch der permanenten und umfassenden Verfügbarkeit der Arbeit sowie der generierten Daten gerecht zu werden \cite{Heise_2015c}. Fehlende Standards und technische Hürden stellen noch immer große Herausforderungen bei der Auswertung, Erstellung und Darstellung der wissenschaftlichen Inhalte dar. Darüber hinaus können die spezifischen Anforderungen für die möglichst umfassende Öffnung des wissenschaftlichen Erkenntnisprozesses von den gängigen Lösungen bisher nicht erfüllt werden. Die notwendige (Weiter-)Entwicklung der Plattformen wird jedoch nur dann stattfinden, wenn die Nachfrage nach solchen Lösungen steigt. Die wissenschaftliche Gemeinschaft ist auch hier gefragt, diese Nachfrage (zum Beispiel durch Experimente mit offener wissenschaftlicher Kommunikation) zu erzeugen und bei der Entwicklung solcher Lösungen eine aktive und gestaltende Rolle einzunehmen.

Allerdings bedeutet das bisher für den "Open Scientist" entweder selbst befähigt zu sein, zu programmieren beziehungsweise bestehende Software den eigenen spezifischen Bedürfnissen anpassen zu können oder Bibliotheken, Rechenzentren oder andere wissenschaftliche Institutionen müssen die infrastrukturellen Rahmenbedingungen schaffen, so dass auch durch Wissenschaftler und Wissenschaftlerinnen ohne solche Kenntnisse der gesamte wissenschaftliche Prozess offen und transparent abgebildet werden kann. Im Rahmen des laufenden strukturierten Promotionsverfahrens war kein solches Angebot verfügbar, das dieses Wissen über Daten und Codes vermittelt oder das Vorhaben aktiv technisch unterstützt hätte.

Abgesehen von der Erstellung eigener Arbeiten ist für den Wissenschaftler oder die Wissenschaftlerin diese Expertise in Zukunft auch deshalb wichtig, weil im Gegensatz zum Träger- und Speichermedium Papier das wissenschaftliche Wissen im Rahmen der Digitalisierung zunehmend als Code gespeichert wird. Die Übermittlung von Wissen bei der wissenschaftlichen Kommunikation kann aber von den beteiligten Akteuren nur profund verstanden werden, wenn auf technisches Wissen zurückgegriffen werden kann und die Übermittlungswege und Formen transparent und offen gestaltet sind \cite{Davis_2011}. Die wissenschaftliche Gemeinschaft darf dieser Auseinandersetzung mit den technologischen Arbeitsmitteln und dem digitalen Wandel nicht aus dem Weg gehen, sondern ist gefordert, ihre Logik zu verstehen. Johannes Näder zitiert in diesem Zusammenhang den französischen Philosophen Régis Debray, laut dem "ein Diskurs über die Zwecke und Werte, der sich nicht auf einen präzisen Zustand der zur Verfügung stehenden Mittel stützt, (...) ein leerer Diskurs (ist). Aber ein Diskurs über die Innovation, der diese nicht im Lichte der Erinnerung genau untersucht, ist ein gefährlicher Diskurs" \cite[:117]{Naeder_2010} \cite[:246]{Debray_2003}.

Genau in diesem Diskurs liegt auch eine Quelle des revolutionären Selbstverständnisses, das zumindest Teile der Open-Bewegung beinhaltet und das Konsequenzen für das gesamte wissenschaftliche System hat: Im digitalen Zeitalter besteht der Kern kommunizierbaren Wissens nicht mehr aus dem gedruckten Wort, sondern aus Code und Daten. Will man demzufolge die Rohform von Wissen lesen, verstehen, interpretieren oder verändern – alles Grundvoraussetzungen für die Erstellung wissenschaftlicher (Qualifikations-)Arbeiten – muss man diesen Code lesen, verstehen und schreiben können. Die Vorteile von digitalem Teilen und Verbreiten von Wissen erfüllen sich folglich bisher nur für den, der für die Migration das nötige Know-how hat. Der zunehmende Grad an Digitalisierung im Arbeitsalltag der Wissenschaftler und Wissenschaftlerinnen stellt die Notwendigkeit dar, sich mit allen produzierten Daten auseinanderzusetzen und experimentell den Umgang mit ihnen zu erforschen. Dabei ist die Veränderung der Arbeitsweise analoger Methoden, von Speicher- und Arbeitsmedien sowie Tools auf digitale Formate für die Gewinnung von Wissen als unausweichlich zu betrachten. Diese Notwendigkeit für eine ausgewogenen Betrachtung findet bei der Ausbildung von Nachwuchswissenschaftlern und Nachwuchswissenschaftlerinnen bisher viel zu wenig Berücksichtigung.

Neben den technischen und strukturellen Herausforderungen bei der Erstellung der Arbeit sowie den daraus resultierenden Diskursen, birgt diese Art der offnen Anfertigung einer wissenschaftlichen Qualifikationsarbeit allerdings auch die Gefahr, weder von der wissenschaftlichen Gemeinschaft noch von der wissenschaftlichen Institution anerkannt oder akzeptiert zu werden. Im Falle dieser Arbeit wurde nach schriftlicher Anfrage bei der zum Zeitpunkt der Zulassung amtierenden Promotionskomission der direkten und unmittelbaren Veröffentlichung des Schreibprozesses der Dissertation stattgegeben, vorbehaltlich der gleichbleibenden Interpretation der Promotionsordnung durch die amtierende Promotionskommission bei Abgabe der Arbeit.

\section{Chancen für und Herausforderungen an die wissenschaftliche Gemeinschaft}

\begin{quote}
\textbf{"Die Freiheit von Fremdbestimmung verpflichtet die wissenschaftliche Gemeinschaft und ihre Mitglieder zu verantwortlicher Selbstbestimmung."}
\end{quote} \cite[:69]{Oezmen_2015}

Die Öffnung wissenschaftlicher Kommunikation und der digitale Wandel haben unbestreitbar Konsequenzen für die Möglichkeiten der Verbreitung, Erstellung und Speicherung wissenschaftlicher Informationen \cite[:233]{Gould_2009}. Sie erlauben eine grundlegende Neujustierung der Produktion von Wissen, die Neuordnung von wissenschaftlichen Werten und Praxen sowie eine neue Form der wissenschaftlichen Kommunikation. Diese Situation kann als einmalige Chance und Möglichkeit für die notwendige Neugestaltung wissenschaftlicher Kommunikation unter Berücksichtigung der Herausforderungen im aktuellen System betrachtet werden \cite{Naeder_2010}. Offenheit in Wissenschaft und Forschung spricht dabei den Kern der Produktion von Wissen an und betrifft folglich nicht nur die Wissenschaft, sondern auch die Gesamtgesellschaft \cite{Mussell_2013}. Die Digitalisierung wissenschaftlicher Arbeitsprozesse und die Nutzung der neuen Möglichkeiten des kollaborativen Austauschs (Science 2.0) sowie die freie und offene Publikation finaler Forschungsergebnisse (Open Access) können zu einer umfassenden Öffnung wissenschaftlicher Kommunikation für die Gesamtgesellschaft (Open Science) führen.

Die Abgrenzung von Open Access zu Open Science im Rahmen wissenschaftlicher Kommunikation wurde in dieser Arbeit auf Grundlage der Unterscheidung von "Zugang zu Wissen" (Open Access) und "Zugriff auf Wissen" (Open Science) durchgeführt (siehe Kapitel 2). Das Konzept von Open Access und der damit verbundenen Verfügbarkeit von wissenschaftlichen Publikationen als Ergebnis von wissenschaftlicher Forschung im bestehenden wissenschaftlichen System betrifft demnach nur einen Teil der grundlegenden Neuordnung wissenschaftlicher Kommunikation. Open Science als Sammelbegriff betrifft nicht nur den digitalen Zugang zu bereits veröffentlichten wissenschaftlichen Publikationen, sondern fordert die Transformation des gesamten wissenschaftlichen Prozesses sowie die Möglichkeit des umfassenden Zugriffs auf diesen.

Für das wissenschaftliche System bedeutet diese Transformation auch, dass nach der Aushandlung und Anpassung der genannten Rahmenbedingungen des wissenschaftlichen Publizierens sowie nach der weitreichenden Etablierung von Open Access, eine Neujustierung der Qualitätssicherungsmaßnahmen und Selektionsmechanismen von Wissenschaft im Rahmen der weitestgehenden Öffnung des wissenschaftlichen Erkenntnisprozesses (Open Science) notwendig ist. Diese Neujustierung ist dabei eng mit der Frage nach der zukünftigen Rolle der Universität und des Hochschulwesens verbunden. Dabei ist die Ablösung des gedruckten Buchs "als Leitmedium der Universität" durch andere Kommunikationsmittel "lediglich als ein Epiphänomen einzustufen" \cite{Warnke_2012}, denn die Digitalisierung und zunehmende Verbreitung von Softwaresystemen verwässert auch die bisherigen Kriterien der Wissenschaft.

Eine "Systemanalyse (...) des akademischen Alltagslebens könnte ein wenig Klarheit und damit vielleicht auch Rückgewinnung von Gestaltungsraum geben" \cite{Warnke_2012}. Das gilt auch für die Produktion und den Vertrieb von Wissen. Durch eine solche Analyse können die Wandlungsprozesse rund um die Digitalisierung und die Öffnung wissenschaftlicher Kommunikation zu einer neuen Chance werden, die Erstellung und Verbreitung des Wissens wieder stärker an die Universität zu binden und zu integrieren.

Läutete der Buchdruck die Moderne ein und legte den Grundstein für die wissenschaftliche Kommunikation, wie wir sie heute kennen, kann durch die Digitalisierung eine erneute Revolution des wissenschaftlichen Systems bevorstehen. Die unmittelbare und umfassende Bereitstellung der wissenschaftlichen Kommunikation unter dem Anspruch größtmöglicher Transparenz stellt das wissenschaftliche System aber innerhalb und außerhalb der wissenschaftlichen Gemeinschaft vor neue Herausforderungen. Das Aufbrechen der strikten Unterscheidung wissenschaftlicher Kommunikation in formelle und informelle sowie interne und externe Kommunikation und die Konsequenzen für die Wahrheitskriterien, Bewertung und Einordnung dieser stellt dabei nur wenige von vielen Aufgaben dar.

Unter Berücksichtigung dieser Herausforderungen und der Rückgewinnung von Gestaltungsraum ist zu hoffen, dass der digitale Wandel und die Öffnung wissenschaftlicher Kommunikation zu besseren Bedingungen für die Schaffung und die Bewahrung von Wissen führen wird. Sollten die Zeiten des "stürmischen Wachstums der Wissenschaft" \cite{Koelbel_2002} dennoch vorüber sein, lässt sich ein Grund dafür in dem Festhalten an der Geschlossenheit des wissenschaftlichen Kommunikationssystems finden. Ein Wandel hin zur Offenheit wissenschaftlicher Kommunikation darf jedoch nicht mit Einschränkungen der Unabhängigkeit und Freiheit von Wissenschaft einhergehen und den wissenschaftlichen Akteuren und der Universität nicht "die Fähigkeit genommen werden, "Nein" zu sagen" \cite[:12]{Neidhardt_2006}. Das ist allerdings aus diversen Gründen eine Herausforderung: Erstens entzieht sich die wissenschaftliche Gemeinschaft bisher der Verantwortung, den Wandel so zu gestalten, dass die Wissenschafts- und Publikationsfreiheit größtmöglich gewahrt wird, und zweitens muss gewährleistet werden, dass der Wettbewerb um die Autorengebühren und die Publikationsgeschwindigkeit nicht zu einer Bedrohung für die Zukunft der Wissenschaftskommunikation wird \cite{Beall_2012} \cite{Lossau_2007}.

Aus der Forderung nach "unbeschränktem Zugang zur gesamten wissenschaftlichen Zeitschriftenliteratur" \cite{BOAI_2012} ist ein gesamtgesellschaftliches und umfassendes Modernisierungsvorhaben der Wissenschaft geworden, das neben dem Aspekt der Zugänglichkeit zu Wissen und Wissenschaft eine Vielzahl an weiteren Unzulänglichkeiten beinhaltet, die den Fortbestand öffentlicher Forschung insgesamt beeinflussen \cite{Brembs_2015}. Mit Blick auf die Umsetzung und Etablierung der Konzepte von Offenheit werden demnach, anders als in ursprünglichen Forderungen nach Offenheit im wissenschaftlichen Kommunikationssystem intendiert, auch Einschränkungen der akademischen Freiheit befürchtet \cite{Hagner_2015}.

Die Entwicklungen im Rahmen der Digitalisierung und Forderungen nach Öffnung müssen von der wissenschaftlichen Community gestaltet werden, wenn sie nicht machtlos den Kräften ausgeliefert sein will, "die von außen auf die Publikation und Rezeption ihrer Schriften einwirken" \cite[:6]{Hirschi_2015}. Die Frage ist, ob sie entscheidet, die Konsequenzen der Digitalisierung als Gefahr für den Fortbestand der Wissenschaft einfach nur zu ignorieren, als reines digitales Abbild der analogen Realität der wissenschaftlichen Kommunikation zu verstehen und das aktuelle System mit all seinen Vor- und Nachteilen zu bewahren oder ob sie es wagt eine zweite wissenschaftliche Revolution einzuläuten, die zu einer umfassenden Wissensverbreitung an die Gesamtgesellschaft und dadurch zu einer grundlegenden Veränderung des aktuellen wissenschaftlichen Systems führen könnte.

Die Forderung nach Offenheit von Wissenschaft und Forschung muss demnach nicht nur als "Strategie" gegen die unterschiedlichen Krisen im wissenschaftlichen Kommunikationssystem und für die Gestaltung des digitalen Wandels verstanden werden. Diese Gestaltung der Strategie ausschließlich den STM-Forschern zu überlassen wird im Ergebnis zu einer weiterhin polarisierenden Abwehr- und Gegenreaktion anderer Disziplinen führen und hätte eine weitere Verwässerung und Fehlleitung der ursprünglichen Ansätze zur Folge \cite{Naeder_2010}. Sie muss dabei auch als Ansatz für eine zukünftige Sicherung der Freiheit von Wissenschaft verstanden werden. Sind wissenschaftliche Aktivitäten nicht offen zugänglich, steigt die Gefahr, dass die öffentliche Unterstützung für die Wissenschaft erodiert und die Menschen Vertrauen in ein System verlieren, das sie nicht unmittelbar verstehen können \cite{Resnik_2005}.

Bisher haben sich die Wissenschaftler und Wissenschaftlerinnen für die Entwicklungen der Forderung nach Öffnung und Digitalisierung der wissenschaftlichen Kommunikation "erstaunlich wenig (...) interessiert" \cite[:67]{Hagner_2015}. Als Folge dieses Desinteresses fehlt es noch immer an einem konkreten Aushandlungsprozess wie dieser Forderung nach Öffnung und Digitalisierung der wissenschaftlichen Kommunikation begegnet werden soll, wie diese praktiziert werden kann, welche Grenzen der Öffnung notwendig sind und wie die Neugestaltung der Kriterien für die wissenschaftliche Arbeit sowie die Qualitäts- und Leistungsbemessung ausgestaltet werden sollen. Darüber hinaus müssen in diesem Zusammenhang auch stärker Themen wie Privatheit in der Wissenschaft und die Notwendigkeit der Freiheit von wissenschaftlichen Tätigkeiten im abgeschlossenen Raum betrachtet werden. Entzieht sich die wissenschaftliche Community diesen Auseinandersetzungen weiterhin, ist zu befürchten, dass langfristig die Freiheit von Wissenschaft und Forschung eingeschränkt und zunehmend rein politische und wirtschaftliche Interessen darüber entscheiden \cite{Warnke_2012}, wie, wann, wo und wozu Wissenschaftler und Wissenschaftlerinnen in Zukunft kommunizieren werden.

Wenn Wissenschaftler und Wissenschaftlerinnen sich in der Auseinandersetzung mit der Forderung nach Öffnung von Kommunikation weiterhin vornehmlich mit den Herausforderungen der Karrierenotwendigkeit und wirtschaftlichen Eigeninteressen befassen \cite{Resnik_2005}, besteht außerdem die Gefahr einer weiteren Verschließung wissenschaftlicher Kommunikation beziehungsweise einer (Aus-)Nutzung der Bewegung hin zur Öffnung durch den Drang zur Etablierung privatwirtschaftlicher Marktmechanismen zur Verwertung und Steuerung von Wissenschaft. Jegliche Abweichung, Einschränkung und Verwässerung von Openness begünstigt demnach diese negative (Weiter-)Entwicklung zu einer rein privatwirtschaftlichen oder von politischen Interessen gesteuerten Wissenschaft. Die Vermutung liegt nahe, dass in diesem Fall die Allokation und Nutzung von Ressourcen für den wissenschaftlichen Erkenntnisprozess ausschließlich auf Grundlage von Marktmechanismen die Heterogenität und auch die Effizienz der Produktion von neuem Wissen langfristig negativ beeinflussen wird.

Es gilt diesbezüglich weiterhin zu betonen, dass wir uns bei diesem Aushandlungsprozess erst am Anfang befinden. Mehr als 500 Jahre Buchdruck und 350 Jahre wissenschaftliches Journal stehen nur 25 Jahren Internet gegenüber. Vor 350 Jahren waren es Wissenschaftler, die sich zusammengetan haben, um eine neue Philosophie für die Förderung von Wissen zu etablieren und das erste wissenschaftliche Journal zu gründen. Sicher ist auch, dass es nicht allein bei der Forderung nach dem Zugang zu wissenschaftlichen Publikationen bleiben wird. Die Umsetzung von Open Access wird früher oder später auch in einer Forderung nach Öffnung des wissenschaftlichen Erkenntnisprozesses münden. Wie bei Open Access ist auch hier die wissenschaftliche Gemeinschaft gefragt, die Ausgestaltung aktiv und konstruktiv-kritisch zu beeinflussen und die Chancen für die Universität nutzbar zu machen.

\section{Ausblick und Anknüpfungspunkte für weitere Forschungsbemühungen}

\begin{quote}
\textbf{We have reached a period in science somewhat similar to that encountered by our colleagues of 300 years ago. Creative and inventive minds must now discover new methods for coping with the scientific literature.}
\end{quote} \cite[:229]{Porter_1964}

Die Transformation des wissenschaftlichen Kommunikationssystems von der Gutenberg-Galaxis in die Turing-Galaxis verlangt eine Neugestaltung der Rahmenbedingungen für die wissenschaftliche Kommunikation und eine Neudefinition der Rolle aller Beteiligten in diesem System. Die neuen Möglichkeiten unterschiedlicher Formen der Darstellung wissenschaftlicher Informationen sollten dabei als Chance für eine aktive Verbesserung, Gestaltung und Modifikation wissenschaftlicher Kommunikation verstanden und genutzt werden. Diese Neugestaltung unter Wahrung der Freiheiten des wissenschaftlichen Systems funktioniert jedoch nur, wenn die Beteiligten ihre Rolle als aktive Gestalter und Gestalterinnen wahrnehmen. Sie müssen dabei in angemessener Form agieren und unbedingt vermeiden, dass der Öffnungs- und Digitalisierungsprozess das wissenschaftliche System technologisch oder ökonomisch rückständiger macht als das bisherige. Diese Arbeit hat den Anspruch durch ihre offene Ausarbeitung und die Erarbeitung von Handlungsempfehlungen für das offene Verfassen wissenschaftlicher (Qualifikations-)Arbeiten einen ersten praktischen Beitrag zu der Debatte über Neugestaltung wissenschaftlicher Kommunikation zu leisten und Anknüpfungspunkte für weitere Forschungsbemühungen zu bieten.

Als wichtige Felder für weitere zukünftige Evaluationen sind die Themen Datenschutz und der Missbrauch von Forschung \cite{Fritsch_2015} zu nennen. Den Schutz der Privatsphäre gegen den immensen Wert von Open-Access-(Daten-)Nutzung auszugleichen und auszuhandeln stellt dabei eine wichtige zukünftige Herausforderung dar. Dabei sollten in einer Debatte nicht nur die nicht sofort überschaubaren Auswirkungen und Konsequenzen berücksichtigt, sondern auch die Vorteile gewissenhaft abgewogen werden. In diesem Zusammenhang bedarf es auch eines Aushandlungsprozesses zwischen der wissenschaftlichen Gemeinschaft, der Politik und der Gesellschaft.

Ein weiter Anknüpfungspunkt für Forschungsbemühungen ergibt sich aus der Umstellung des Publikationssystems vom Verkauf der Inhalte auf eine Voraberstattung der Kosten für die Publikation von wissenschaftlichen Erkenntnissen durch die öffentliche Hand. Im Rahmen der damit einhergehenden Transformation der Geschäftsmodelle von Verlagen hin zur freien Verfügbarkeit der veröffentlichten wissenschaftlichen Inhalte für die Gesamtgesellschaft sollte untersucht werden, wie verhindert werden kann, dass die Erstattung von Autorengebühren (APCs) für die offene Publikationen nicht zu falschen Entwicklungen und Fehlanreizen führen, die einen kommerziellen Open-Access-Markt befeuern, der zu einer ungerechtfertigten Verteuerung von APCs und zu einer weiteren Konzentration im Publikationsmarkt führen könnte. Die Fragestellungen sind, wie in dieser Arbeit dargestellt, eng mit den Publikationsentscheidungen der wissenschaftlichen Autoren und Autorinnen sowie mit der Erlangung von symbolischem, wissenschaftlichem Kapital verbunden.

Die mögliche Konkurrenz zwischen der wissenschaftlichen und medialen Kommunikation im Rahmen der Forderung nach mehr Öffnung stellt einen weiteren Ansatz für Untersuchungen dar. Als Konsequenz der Öffnung der gesamten wissenschaftlichen Kommunikation muss hinterfragt werden, ob und inwieweit das Wahrheitsmonopol der Wissenschaft durch das Aufmerksamkeitsmonopol der Medien im Rahmen der Möglichkeiten des offenen Zugangs zu Wissenschaft und des Zugriffs auf Wissen negativ beeinflusst werden könnte \cite{Weingart_2005}. Die Herausforderungen müssen dabei offensiv den Möglichkeiten und Chancen gegenübergestellt werden. Hierbei sind auch die jüngsten Entwicklungen um das Thema Bürgerwissenschaft (Citizen Science) genauer zu berücksichtigen. Diese sollten dabei im Kontext anderer ähnlicher Entwicklungen, wie zum Beispiel im Bereich des Bürgerjournalismus (Citizen Journalism), betrachtet werden.

Weitere Fragestellungen ergeben sich im Rahmen der neuen Möglichkeiten der Kontrolle, Überwachung und Quantifizierung individueller wissenschaftlicher Tätigkeiten, zum Beispiel bei der revisionsgetriebenen offenen wissenschaftlichen Publikation im digitalen Raum sowie deren Konsequenzen auf die (Selbst-)Steuerung von Wissenschaft. Als eine Folge der offenen Anfertigung dieser Arbeit wurde jeder Schritt umfassend mit Zeitstempel und weiteren Metainformationen dokumentiert und somit nachvollziehbar gemacht. Diese Informationen können genutzt werden um den Arbeitsprozess einzelner Wissenschaftler und Wissenschaftlerinnen zu überwachen und auf den Erstellungsprozess einzuwirken. Als Erkenntnis aus diesem offenen Erstellungsprozess kann vermutet werden, dass diese neuen Kontroll- und Überwachungsmöglichkeiten der wissenschaftlichen Arbeit auch eine neue Herausforderung für die Freiheit und den Datenschutz von Wissenschaftlern und Wissenschaftlerinnen sowie ihrer Tätigkeit darstellen können.

Die vorliegende Arbeit befindet sich an vielen Stellen im Spannungsfeld der Forderungen nach Öffnung von wissenschaftlicher Kommunikation, technologischen Entwicklungen und ihren sozialen Konsequenzen sowie der zunehmenden Digitalisierung wissenschaftlicher Arbeit. Trotz der teilweise hitzigen Debatten in der Literatur und den Auseinandersetzungen innerhalb und außerhalb der wissenschaftlichen Gemeinschaft in diesen Themenfeldern fehlt es bisher noch immer an einem konkreten Aushandlungsprozess, wie diese Entwicklungen aus Sicht der wissenschaftlichen Gemeinschaft in Zukunft konkret (mit-)gestaltet werden können und welche Auswirkungen sie auf die Wissenschaft haben werden. Als Ergebnis dieser Arbeit können für das Fehlen dieser Aushandlung unter anderem fehlende Anreizsysteme für eine nachhaltige Auseinandersetzung mit den Themenfeldern und ein geringes Interesse an Experimenten mit den neuen Möglichkeiten des wissenschaftlichen Kommunizierens genannt werden. Diese Faktoren gilt es weiter zu untersuchen, um Strategien zu entwickeln, wie die notwendige Aushandlung innerhalb der wissenschaftlichen Gemeinschaft und eine neue Experimentierfreudigkeit mit wissenschaftlicher Kommunikation helfen können die zukünftigen Kriterien von Wissenschaft zu gestalten.
