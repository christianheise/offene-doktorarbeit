\chapter{Zusammenfassung und Ausblick}

In dieser Arbeit wurden die Herausforderungen der formellen wissenschaftlichen Kommunikation im Rahmen der Digitalisierung und die Forderung nach Öffnung dieser umfassend analysiert und diskutiert. Dabei wurden die einzelnen Definitionen von Open Access und Open Science, sowie Implikationen und Rahmenbedingungen näher betrachtet, analysiert und spezifiziert.

Nach einer Einführung in den Themenbereich der wissenschaftlichen Kommunikation und des wissenschaftlichen Publizierens wurde anhand aktueller Literatur zum Thema die Debatte um die Forderung nach Öffnung der wissenschaftlichen Kommunikation dargestellt. Dabei wurden Treiber und Bremser für die Entwicklungen identifiziert.

Diese Treiber und Bremser wurden darauf hin mithilfe einer repräsentativen Umfrage unter Wissenschaftlern genauer abgefragt. Dabei wurde analysiert, inwiefern ein Verständnis von Offenheit im Rahmen der wissenschaftlichen Kommunikation vorherrscht und inwieweit Offenheit in den unterschiedlichen wissenschaftlichen Fachdisziplinen verbreitet ist und praktiziert wird.

Die Ergebnisse der Studie können insofern als repräsentativ gelten, als sie auf einer relativ großen Stichprobe beruht. In der Auswertung wurde vor allem deutlich, dass die Verbreitung offener Kommunikationsverfahren wesentlich vom Fachgebiet abhängt. Ungeachtet der vielfältigen Kritik am aktuellen Publikations- und Kommunikationssystem scheint dieses jedoch auch nach zwei Jahrzehnten noch immer weitestgehend stabil. In der Studie wurde eine große Diskrepanz zwischen dem Interesse nach und dem Verständnis für Offenheit und der tatsächlich praktizierten offenen Kommunikations- und Arbeitsweise festgestellt. Das deckt sich auch mit bisherigen Analysen \cite{Bartling_2013} \cite{hagner_2015_sache_buches}.

Rainer Kuhlen definierte 2002 drei Szenarien wie der Zugriff auf Wissen in mittlerer Perspektive organisiert sein wird \cite{Kuhlen_2002_universalaccess}:
\begin{enumerate}
\item Elektronische Informationen sind frei zugänglich und die Konzepte der individuellen Autorenschaft und des geistigen Eigentums werden zu Relikten aus bürgerlichen Vorinformationsgesellschaften
\item Wissen und Informationen sind kontrolliert und dem Markt ohne politische Gegensteuerung überlassen: Die Kommerzialisierung und Zonierung von Wissen und Information wird umfassend sein und den Alltag bestimmen.
\item Wissen und Information werden über koexistente oder Paralleluniversen organisiert: Das Wissens als Produkt ist frei, öffentlich zugänglich und nutzbar. Es bleibt aber genug Spielraum bei der Adaption, Beratung, Veredlung oder anderen Mehrwertleistungen einer kommerziellen Informations- und Wissenswirtschaft
\end{enumerate}

Ein Ergebnis dieser Arbeit ist, dass es maßgeblich von den publizierenden Wissenschaftlern selbst abhängen wird eine treibende Kraft und aktive Position bei der Gestaltung des Wandels zu übernehmen. Bisher haben sich die Wissenschaftler und Wissenschaftlerinnen, vor allem der Geistes- und Kulturwissenschaft sowie der Wissenschaftsforschung "erstaunlich wenig für das Thema interessiert" \cite{hagner_2015_sache_buches}. Die Ergebnisse der Befragung nähren die Befürchtung, das sie auch in naher Zulunft versäumen werden eine gestaltende Position einzunehmen.

Christopher Kelty sieht für das Desinteresse an Offenheit im wissenschaftlichen Kommunikationssystem vorallem zwei Gründe: Erstens, ist Open Access ziehmlich langweilig und zweitens ist das Thema bei genauerer Betrachtung sehr komplex. Ergänzend zu Kelthys Einschätzung kann als Ergebnis dieser Arbeit aber auch noch ein dritter Grund genannt werden: Die fehlenden technischen Möglichkeiten. Die zur verfügungstehenden Plattformen und Applikationen sind noch nicht ausgereift genug um im Alltag offene Wissenschaft zu praktizieren. Die offene Erstellung dieser Arbeit wäre ohne programiertechnische Vorkenntnisse schwer bis nicht möglich gewesen. So muss der "Open Scientists" entweder in der Lage sein, selbst programieren zu können oder es müssen die Rahmenbedinungen geschaffen werden, dass er ohne solche Kenntnisse den gesamten wissenschaftlichen Prozess offen und transparent abbilden kann.

Sollten die Zeiten des "stürmischen Wachstums der Wissenschaft endgültig vorüber" sein \cite{K_lbel_2002}, lässt sich ein Grund dafür in der Geschlossenheit des wissenschaftlichen Kommunikationssystems finden. Mit der Verpflichtung auf Offenheit darf aber der Wissenschaft nicht die Unabhängigkeit und auch nicht "die Fähigkeit genommen werden, "Nein" zu sagen" \cite{suchen_Hornbostel_2006}. Unter Berücksichtigung dieser Faktoren ist zu vermuten, dass der neue Wachstum an Wissen durch den digitalen Wandel im Rahmen und die Öffnung wissenschaftlicher Kommunikation wirklich zu besseren Bedingungen für Schaffung neuen Wissens und die Bewahrung alten Wissens führen wird.

Läutete der Buchdruck die Moderne ein und legte den Grundstein für die wissenschaftliche Kommunikation wie wir sie heute kennen, könnte im Rahmen der Digitalisierung eine erneute Revolution des wissenschaftlichen Systems bevorstehen. Die unmittelbare und umfassende Bereitstellung der wissenschaftlichen Kommunikation im Rahmen der alltäglichen wissenschaftlichen Arbeit unter den Bedingungen von Openness, stellt das wissenschaftlichen System innerhalb und außerhalb vor neue Herausforderungen.
Diese Entwicklung kann und muss von der wissenschaftlichen Community entschieden werden. Die Frage ist, ob sie entscheidet die Digitalisierung als rein digitales Abbild der analogen Realität der wissenschaftlichen Kommunikation zu verstehen und das aktuelle System mit all seinen Vor- und Nachteilen zu bewahren oder wagen es die Wissenschaftler und Wissenschaftlerinnen eine zweite wissenschaftliche Revolution einzuläuten, die zu einer umfassenden Wissensverbreitung an die Gesamtgesellschaft und zu einer grundlegenden Veränderung des aktuellen wissenschaftlichen Systems führen könnte?

Entzieht sich die wissenschaftliche Community dieser Entscheidung, werden die Freiheit von Wissenschaft und Forschung darunter leiden und rein wirtschaftliche Interessen darüber entscheiden, wie und wozu Wissenschaftler und Wissenschaftlerinnen in Zukunft arbeiten werden.

---- TODO: 350 Jahre Journals und gerade mal 25 Jahre Digitalisierung ----
