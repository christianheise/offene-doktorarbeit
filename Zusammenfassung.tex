\chapter{Zusammenfassung und Ausblick}

\begin{quote}
\textbf{"Denn ich habs umsonst empfangen, umsonst hab ich's gegeben und begehre auch dafür nichts"}
\end{quote} \cite{luther_1876}

In dieser Arbeit wurden die Herausforderungen an das System der wissenschaftlichen Kommunikation im Rahmen der Digitalisierung und die Forderung nach Öffnung dieser umfassend dargestellt und analysiert. Die Entwicklungen im Bereich der Öffnung wissenschaftlicher Kommunikation wurden aus geistes- und kulturwissenschaftlicher Perspektive genauer untersucht und den Erkenntnissen über die Öffnung wissenschaftlicher Kommunikation gegenübergestellt, sowie das Ergebnis dieser Gegenüberstellung abschließend diskutiert.

Nach einer Einführung in die Thematik des wissenschaftlichen Kommunizierens, der Ausführung des Vorgehens und der Darstellung die eigenen Position und der Beweggründe, wurden die methodischen Herangehensweisen beschrieben. Im weiteren Verlauf wurden theoretischen Grundlagen dargestellt, Debatten um Veränderungen im wissenschaftlichen Kommunikationssystem erläutert und weitere Annahmen, sowie die Katalysatoren und Hindernisse für die Entwicklungen identifiziert. Mit Hilfe der quantitativen Methode einer Online-Befragung wurde unter 1.112 Wissenschaftlern und Wissenschaftlerinnen analysiert, welche Auffassungen und Annahmen in Bezug auf den postulierten Wandel wissenschaftlicher Kommunikation im Rahmen von Offenheit und Digitalisierung vorherrschen und inwiefern diese mit anderen Aspekten des wissenschaftlichen Kommunikationssystems korrelieren. Diese Daten wurden in den Kontext bisheriger Untersuchungen gestellt und analysiert. Im fünften Kapitel wurden die gewonnen Erkenntnisse experimentell anhand der eigenen Öffnung des wissenschaftlichen Erkenntnisprozesses im Rahmen der Anfertigung dieser Arbeit autoethnografisch erarbeitet und dargestellt. Die Ergebnisse dieser Arbeit wurden im sechsten Kapitel diskutiert und werden im Folgenden zusammengefasst sowie mögliche Anknüpfungspunkte für weitere Forschungsbemühungen genannt.

Zusammenfassend kann konstatiert werden, dass es bisher weder gelungen ist, die nötigen Anreize für die einzelnen Wissenschaftler so zu setzen, dass deren Eigeninteresse mit dem Wohl der Wissenschaft und dem der Öffentlichkeit harmonieren, noch führten die staatlichen Interventionen zu einer fundamentalen Veränderung im Publikationsverhalten. Neben der Erkenntnis, dass die kritischen Hindernisse für Veränderungen nicht technische oder finanzielle, sondern sozialer Natur sind \cite{nosek_2012_scientific}, kann die klare rechtliche Klärung für die Zweitverwertung von Inhalten als ein wichtiger Katalysator für die Entwicklung gesehen werden. Darüber hinaus sollten Reputationsmechanismen etabliert werden, die diese Öffnung befördern. Diese Maßnahmen können aber nur erfolgreich sein, wenn eine Diskussion innerhalb der wissenschaftlichen Gemeinschaft unterstützt und subventioniert wird. Dafür sollten die wissenschaftlichen Institutionen Raum schaffen.

Wird dem nicht Rechnung getragen und die Öffnung der wissenschaftlichen Kommunikation weiter ausschließlich durch externe und politisch motivierte Maßnahmen angestrebt, werden rein kommerzielle, forschungspolitische und steuerungspolitische Interessen den Prozess weiter vorantreiben. Diese Entwicklung gilt es zu vermeiden, da so die einstigen Ideale und Gründe für eine Öffnung der wissenschaftlichen Kommunikation eben nicht der Gesamtgesellschaft zu Gute kommen, sondern nur partikularen Interessen oder zu einer Medialisierung der Wissenschaft mit negativen Konsequenzen auf das Wahrheitsmonopol der Wissenschaft führt. Die wissenschaftliche Gemeinschaft muss sich die Förderung der Katalysatoren und Beseitigng der Hindernisse den unvermeidlichen Wandel im Rahmen der Digitalisierung selbst aktiv gestalten. Verlage wie Elsevier und andere wirken seit Dekaden auf die forschungspolitischen Agenda ein und versuchen ihre wirtschaftlichen Interessen im Rahmen des Wandels durchzusetzen. Wenn in diesem Prozess Verlage fordern, dass "Autoren in einem gesunden, unverzerrten freien Markt frei wählen sollten, wo sie publizieren" \cite{Brussels_Declaration_2007}, muss sich die wissenschaftliche Gemeinschaft fragen lassen, ob sie oder Verlage die Ausgestaltung des wissenschaftlichen Kommunikationssystems gestaltet haben und ob sie in der Vergangenheit die Publikationsfreiheit für die möglichst umfassende Verbreitung von Wissen genutzt haben.

\section{Interesse an und Verbreitung von Öffnung wissenschaftlicher Kommunikation und der wissenschaftliche Alltag}

Die Ergebnisse der im Rahmen dieser Arbeit durchgeführten Befragung zeigen eine überraschend mehrheitlich große Zustimmung und ein überwiegend großes Interesse an der Öffnung wissenschaftlicher Kommunikation. Im wissenschaftlichen Alltag hat dieses Interesse und die Zustimmung zu digitalen und offenen Verfahren der Kommunikation bisher jedoch noch nicht zu einer fundamentalen Veränderung des Publikations- und Veröffentlichungsverhaltens geführt. Hier muss die bisherige Rolle der wissenschaftlichen Gemeinschaft und ihrer Auseinandersetzung mit dem Prozess der Digitalisierung und Öffnung wissenschaftlicher Kommunikation genauer betrachtet werden.

Sucht man nach genauen Gründen für das Desinteresse bei der praktischen Umsetzung von Offenheit im wissenschaftlichen Alltag, wird deutlich, dass unter anderem Unwissen über die wirtschaftlichen Aspekte der wissenschaftlichen Informationsversorgung für die Diskrepanz zwischen dem Interesse und der tatsächlichen Praxis eine Rolle spielen. In der Literatur wird diese mit der komfortable Situation der Wissenschaftler in einem System genannt, in dem für die meisten Mitglieder der wissenschaftlichen Gemeinschaft kein oder nur geringer unmittelbarer Anreiz besteht, sich aktiv mit dem Publikationssystem und möglichen Veränderungen zu beschäftigen, weil sie weder die Kosten des Publikationssystems tragen \cite{Sietmann_oa_2007}, noch sich mit den finanziellen Aspekten auseinander setzen müssen \cite{herb_2010}. Darüber hinaus kam es im Zuge der quantitativen Bewertung von Forschungsleistung und der vonehmlichen Berücksichtigung quantitativer Indikatoren, deren Qualitätsmessung nach wissenschaftlichen Methoden kritisiert werden kann, bei der Bewertung für die Mittelzuweisung zu einer Entwicklung von Fehlanreizen nach denen die reine Anzahl und nicht die Qualität der Veröffentlichung in Publikationen etablierter Verlage bei wissenschaftlicher Arbeit im Vordergrund steht.

Ergänzend muss für die schleppende Umsetzung auch der scheinbar geringe Bedarf an Veränderung im Alltag der Wissenschaftlerinnen und Wissenschaftler genannt werden. Es gibt kaum positive Anreize sich mit der Entscheidung wo und wie veröffentlicht wird sowie mit den Konsequenzen dieser Entscheidung auseinander zu setzen, da diese im wissenschaftlichen Reputationssystem bisher nicht oder nur unzureichend abgebildert werden können und honoriert werden. Neben den wirtschaftlichen den begrenzten zeitlichen Ressourcen sich mit den umfangreichen Aspekten der Informationsversorgung auseinanderzusetzen, gibt es zudem weiterhin viele rechtliche Unsicherheiten, die die Wissenschaftlerinnen und Wissenschaftler davon abhalten ihre Kommunikation zu öffnen.

Als Hindernisse für die notwendigen Veränderungen allein eine Kombination aus Trägheit der wissenschaftlichen Akteure, Unsicherheit über alternative Publikationsmodelle, und die Existenz von Gruppen die in die Ineffizienz des gegenwärtigen Systems investiert haben auszumachen \cite{nosek_2012_scientific} wird muss dennoch als unzureichend betrachtet werden. Als weitere Gründe für das geringe Interesse an der proaktiven Öffnung von Wissenschaft und Forschung seitens der wissenschaftlichen Gemeinschaft müssen auch Unterschiede und fachspezifischen Eigenheiten der wissenschaftlichen Kommunikation, wie zum Beispiel die unterschiedlichen Publikationsformen, genannt werden. Möglichkeiten und Debatten sind bewusst meist auf die jeweilige Disziplin beschränkt und berücksichtigen selten andere Forschungsrichtungen.

Die ersten offenen Publikationsvorhaben und Entwicklungen hin zu Open Access in erfolgten erster Linie aus den STM-Fächern, die von vor allem von der Zeitschriftenkrise viel stärker und früher betroffen waren als andere Fächer. Die daraus resultierten Erklärungen und Bestrebungen führten allerdings zu erheblichen "Vorbehalten hinsichtlich der Sinnhaftigkeit und Durchführbarkeit von Open Access, die wiederum zu Desinteresse oder Polarisierung bei vielen Vertretern dieser Disziplinen" \cite{naeder_2010_open} zur Folge hatte. Diese Polarisierung stellt weiterhin eine große Herausforderung für die Etablierung und Verbreitung der Konzepte für die Öffnung wissenschaftlicher Kommunikation dar. Demnach ist eine Darstellung des Interesses an der Öffnung wissenschaftlicher Kommunikation über alle wissenschaftliche Fachbereiche und ein gemeinsames Handeln der wissenschaftlichen Gemeinschaft auch zukünftig nur schwer vorstellbar.

Christopher Kelty sieht darüber hinaus für das Desinteresse an Offenheit im wissenschaftlichen Kommunikationssystem und Alltag folgende zwei Aspekte: Erstens, ist die Auseinandersetzung mit Offenheit für die meisten ziemlich langweilig und zweitens ist das Thema bei genauerer Betrachtung eben sehr komplex. Darüber hinaus würden durch die umfassende Auseinandersetzung mit dem wissenschaftlichen Publikationsmarkt, "Praktiken auf dem Spiel stehen, die dem Tun vieler Geisteswissenschaftler (...)" und auch Wissenschaftler anderer Fächer, "Sinn und Legitimität zu verleihen scheinen" \cite{Hirschi_2015_buch_oa}, die lieber nicht hinterfragt werden.

Die Ergebnisse der durchgeführten Befragung nähren all diese Befürchtungen und lassen vermuten, dass das Gros der Wissenschaftler und Wissenschaftlerinnen auch in naher Zukunft keine hervorgehobene Rolle bei der Gestaltung der Transformation der Kommunikation einnehmen und sich trotz der Möglichkeiten zur Veränderung somit eher der Status quo bewahren wird \cite{nosek_2012_scientific}. Diese Befürchtungen ist darüber hinaus gesondert im Spannungsfeld zwischen Wissenschaft und Politik zu sehen, in dem Wissenschaftler eigentlich als Strategen im politischen Kampf um Glaubwürdigkeit für die wissenschaftliche Arbeit aggieren \cite{latour_2013_laboratory}. Diese beiden Rollen nehmen sie allerdings nur unzureichend war und eine tiefgreifende Debatte innerhalb der wissenschaftlichen Community findet nicht statt.

Das theoretische und ideelle Interesse steht somit einem praktischen Alltags-Desintresse an der Umsetzung und den Herausforderungen im aktuellen Publikationssystem gegenüber, von denen die Forschrinnen und Forscher bisher nur begrenzt betroffen sind. Das ist eigentlich keine schlechte Situation, wenn man darüber hinaus davon ausgeht, dass das Wissen über das Kommunikationssystem, die Öffnung wissenschaftlicher Kommunikation und die Verbreitung digitaler Arbeitsmittel und -methoden noch immer sehr begrenzt ist aber unaufhaltsam ansteigt. Dennoch muss weiter auf die Stärkung der in dieser Arbeit identifizierten Katalysatoren und die Beseitigung der genannten Hindernisse für die Etablierung der Öffnung wissenschaftlicher Kommunikation hingearbeitet werden.

Diese Auseinandersetzung mit den Katalysatoren und Hindernissen beziehen sich auch auf die Anpassungen bei der Qualitätssicherung und auf Veränderungen im Rahmen des wissenschaftlichen Reputationssystems sowie deren Konsequenzen. Die Öffnung wissenschaftlicher Publikationen ist dabei nur der erste Schritt hin zur Öffnung des gesamten wissenschaftlichen Erkenntnisprozesses. Diese Anpassungen der Methoden um gute wissenschaftliche Praxis bei der Digitalisierung wissenschaftlicher Arbeit hin zu einem guten Standard in der Wissenschaft zu machen, obliegt ebenfalls der Wissenschaftsgemeinschaft selbst. Vor allem weil diese Kriterien einen direkten und unmittelbaren Einfluss auf die Bewertung und (Selbst-)Steuerung der Wissenschaft hat, muss die Wissenschaftsgemeinschaft Verantwortung übernehmen und diesen Prozess aktiv gestalten. Diese Verantwortung sollte nicht durch die Wahl "gar nicht mehr auf der Universitätsseite zu erscheinen oder keine jährlichen Geldmittel für ihre Publikationsergebnisse zu erhalten" \cite{Warnke_2012} entschieden werden, sondern durch eine aktive Ausgestaltung der Rahmenbedinungnen der wissenschaftlichen Kommunikation und damit auch der Steuerung des wissenschaftlichen Systems.

Die Öffnung wissenschaftlicher Kommunikation kann als "vielleicht kostbarste Geschenk des Internet an die Wissensgesellschaft" betrachtet werden, wenn sie nicht ausschließlich durch die "ökonomischen Interessen des Informationskapitalismus" \cite{hagner_2015_sache_buches} gelenkt wird, sondern sich (weiterhin) maßgeblich auf die Aufgabe konzentriert dem gesamtgesellschaftlichen Auftrag des Wissenschaftssystems gerecht zu werden. Es bleibt auch eine Notwendigkeit, dass jeder Autor und jede Autorin selbst, aber unter Kenntnisstand der Folgen und Konsequenzen für die Zugänglichkeit durch andere, entscheidet ob und wie seine oder ihre Forschungsergebnisse verbreitet werden. Der Prozess der Erfüllung akademischen Erwartungen für die Veröffentlichung, den Einbau in einer Disziplin, unter Verwendung von Standard geistigen Rahmenbedingungen und dem Zurechtkommen mit Kollegen macht es allerding sehr einfach, den soziale Idealismus auf den akademischen Syllabus zu beschränken und sich nicht an aktivistischen Veränderungsprozessen zu beteiligen \cite[:25]{flood_2013_combining}.

\section{Katalysatoren und Hindernisse für die Etablierung der Öffnung wissenschaftlicher Kommunikation}

Nach der Einführung in den Themenbereich der wissenschaftlichen Kommunikation und des wissenschaftlichen Publizierens wurden anhand aktueller Literatur die Entwicklung und die Debatte um die Forderung nach Öffnung der wissenschaftlichen Kommunikation dargestellt. Aus den dargestellten Debatten wurden Treiber und Bremser für die Entwicklungen identifiziert und herausgearbeitet. Diese wurden darauf hin mithilfe einer Online-Befragung unter 1.112 deutschsprachigen Wissenschaftlerinnen und Wissenschaftlern detailiert abgefragt. Dabei wurde analysiert, inwiefern ein Verständnis von Offenheit im Rahmen der wissenschaftlichen Kommunikation vorherrscht und inwieweit das Interesse an Offenheit in den unterschiedlichen wissenschaftlichen Fachdisziplinen verbreitet ist und praktiziert wird. Die Ergebnisse wurden mit einer Studie des Soziologischen Forschungsinstituts Göttingen aus dem Jahr 2007 verglichen um Trends und Entwicklungen zu identifizieren.

In den Augen der befragten Wissenschaftlerinnen und Wissenschaftler stehen die Beschleunigung der Wissensverbreitung (65 Prozent), die Eröffnung neuer Möglichkeiten für Wissensverbreitung (64 Prozent) und die offene Verfügbarkeit bereits finanzierte Forschung für alle (55 Prozent) vor allem den Herausforderungen nach etablierten Reputationskriterien für die Bewertung von offener Wissenschaft (43 Prozent), der Gefahr der Fehlinterpretation und Falschinformation (40 Prozent) sowie einem erhöhten zeitlichen Mehraufwand für die Bereitstellung der wissenschaftlichen Publikationen und/oder Forschungsdaten (34 Prozent) gegenüber. Die 1.112 Befragten gaben außerdem mehrheitlich an, dass sie rechtliche Bedenken (39 Prozent) und Unwissenheit über die Erlaubnis (29 Prozent) davon abhält, wissenschaftliche Inhalte ohne finanzielle, rechtliche oder technische Barrieren öffentlich zur Verfügung zu stellen.

Die Ergebnisse der Studie können insofern als repräsentativ gelten, als dass sie auf einer relativ großen Stichprobe beruhen. In der Auswertung der Daten wurde vor allem deutlich, dass die Hindernisse eine größere Verteilung aufwiesen als die Katalysatoren und dass die Verbreitung offener wissenschaftlicher Kommunikationsverfahren im wesentlichen vom Fachgebiet abhängt. Darüber hinaus scheint ungeachtet der vielfältigen Kritik am aktuellen Publikations- und Kommunikationssystem, dieses auch nach zwei Jahrzehnten noch immer weitestgehend stabil. Die Ergebnisse belegen auch eine große Diskrepanz zwischen dem Interesse nach und dem Verständnis für Offenheit und der tatsächlich praktizierten offenen Kommunikations- und Arbeitsweise vorherrscht. Das deckt sich auch mit bisherigen Analysen in der Literatur \cite{yiotis_2013_open} \cite{Bartling_2013} \cite{hagner_2015_sache_buches} \cite{Fecher_2015}.

25 Jahre nach den ersten Versuchen einen Offenen Zugang zu wissenschaftlicher Kommunikation mit Hilfe digitaler Netze umzusetzen, Unmengen von weichen Erklärungen und (teilweise leeren) Bekenntnissen für die Öffnung von wissenschaftlicher Kommunikation gibt zwar es kaum noch Zweifel, dass sich das System verändern wird, dennoch bestehen Zweifel bei der genauen Ausgestaltung von Offenheit wissenschaftlicher Kommunikation, der Frage ob die letztendlich erzielte Öffnung noch den ursprünglichen Anspruch an Verbesserung des Systems gerecht werden kann und bei der Frage wie sich die Vor- und Nachteile dieser Entwicklung letztendlich zueinander verhalten werden \cite{hagner_2015_sache_buches}. Bisher hat die wissenschaftliche Gemeinschaft sich eher Verhalten an dem Veränderungsprozess beteiligt. Sollte es durch das weitere Ausbleiben der Gestaltung aus der wissenschaftlichen Gemeinschaft zu einem Eingriff der Politik in diesem "sensiblen Bereich" kommen, "muss besonders acht auf die komplexen Geschichte der Organisationen der öffentlichen Wissenschaft gegeben und die potenzielle Fragilität der eigentümlichen institutionellen Matrix respektiert werden, in der sich die modernen Forschung entwickelt hat und aufgeblüht ist" \cite{david1998_common}. Darüber hinaus müssen die Rahmenbedingungen ausgehandelt werden unter denen die Digitalisierung der wissenschaftlichen Kommunikation und die Öffnung von Wissenschaft und Forschung stattfindet \cite{mennes_2013_making_os}.

Eine weitere Herausforderung besteht darin, dass viele Förderorganisationen im Rahmen von Forschungsförderung noch immer fast ausschließlich auf klassische Publikationen abzielen und nur langsam die Software-Entwicklung unterstützen \cite{hey_2015_open}. Förderorganisationen müssen diese Verantwortung ernst nehmen, indem sie die zusätzlichen Ressourcen, die für die Schaffung der strukturellen Grundlagen die mit der Öffnung von Wissenschaft und Forschung verbunden sind, zur Verfügung stellen \cite{mennes_2013_making_os} \cite{patlak_2010_open}. Um diese Bemühungen voranzubringen, müssen sich Förderorganisationen mehr denn je ihrer Rolle als "einflussreiche  Akteuren  im  komplexen und  sich  wandelnden  Markt  für  wissenschaftliche  Publikationen" \cite{wein_2010_erwerbung} bewusst werden und entscheiden, ob die Umsetzung der gemeinsamen Nutzung von Daten durch Incentivierung Mandate fördern \cite{mennes_2013_making_os}. Bisher ist diese Nutzung von wissenschaftlichen Daten nur sehr gering verbreitet und auch wenn die Erwähnung der Weiternutzung von wissenschaftlichen Daten ansteigt, bleiben bis zu 86 Prozent der veröffentlichen Daten bisher ungenutzt beziehungsweise unzitiert \cite{peters_2015_research}.

Die Entwicklungen der letzten Jahre zeigen auch, dass Verlage aktiv daran arbeiten den digitalen Wandel zu nutzen um ihre kompfortable Situation in dem System der wissenschaftlichen Kommunikation zu sichern, wenn nicht sogar auszubauen. Dabei stellen die technischen und finanziellen Herausforderungen für die ursprünglichen Ideale hin zu der Öffnung der wissenschaftlichen Kommunikation für die Gesamtgesellschaft im Rahmen der Digitalisierung nur einen Teil der Gründe für noch immer vorherschende Beständigkeit des aktuellen Systems und der starken Rolle von Verlagen bei der Neugestaltung der wissenschaftlichen Kommunikation dar. Würde die wissenschaftliche Gemeinschaft und die Gesellschaften die Bibliotheken stärker bei dem Wandel weg von abonnementen-basierten Modellen unterstützen, wäre die Entwicklung womöglich fortgeschrittener \cite{nosek_2012_scientific}.

\section{Erkenntnisse aus dem offenen Verfassen der Arbeit}

Ein Hemmnis für die Umsetzung der Konzepte um die offen praktizierte Wissenschaft sind die fehlenden rechtlichen Rahmenbedingungen und ökonomischen Vorrausetzungen aber auch die gering verbreiteten technischen Möglichkeiten und Standards. Die zur Verfügung stehenden Plattformen und Applikationen sind noch nicht ausgereift, etabliert und bequem genug um im Alltag offene Wissenschaft zu praktizieren. Dabei gilt es zu berücksichtigen, dass die wissenschaftliche Arbeit, trotz zunehmender Digitalisierung, seit Dekaden auf die geschlossene Publikation und den nicht-öffentlichen Publikationsprozess ausgelegt ist sowie dem Druck der Etablierung des Martkmodus als dominante Governanceform von Wissenschaft ausgesetzt ist.

Stellt man die gewohnte wissenschaftliche Arbeitsweise dem offenen Erstellungsprozess dieser Arbeit gegenüber, so muss die Arbeit auf dem Rechner in einem geschlossenen Umfeld noch immer als um vieles einfacher als das öffentliche Verfassen einer Arbeit beschrieben werden. Das hat zum einen mit den gewohnten Softwareumgebungen, die meist kein gleiches Abbild unter den Applikationen für offene und  strukturierte Texterstellung notwendig sind und zum Anderen müssen die fehlenden Implementierung der Funktionen aus gängigen wissenschaftlichen Umgebungen sowie Beschränkungen bei Usability meist durch mehr manuelle Arbeit kompensiert werden.

Deshalb ist es verständlich, dass bisher nur eine Minderheit offene Webplattformen für die wissenschaftliche Textarbeit \cite{Perkel_2014} nutzt und die Mehrheit der im Rahmen dieser Arbeit befragen wissenschaftler bei der Öffnung von Forschungsdaten einen großen Aufwand befürchtet, obwohl die Daten durch den zunehmende Einsatz computerunterstützter wissenschaftlicher Verfahren bereits digital vorliegen. Ergänzend zeigen die Ergebnisse des Selbstexperiments, dass die Offenlegung des gesamten Erkenntnisprozessess bei der Erstellung dieser Arbeit ohne programmiertechnische Vorkenntnisse schwer bis nicht möglich gewesen wäre. Auch hier stellten fehlende Standards und technische Hürden große Herausforderungen bei der Auswertung, Erstellung und Darstellung der Inhalte dar. Darüber hinaus konnten spezifische Anforderungen von den gängigen Lösungen bisher nicht abgebildet werden.

Bisher muss der "Open Scientist" entweder in der Lage sein, selbst programmieren beziehungsweise bestehenden Code nach seinen Bedürfnissen anpassen zu können oder es müssen die Rahmenbedingungen geschaffen werden, dass er ohne solche Kenntnisse den gesamten wissenschaftlichen Prozess offen und transparent abbilden kann. Im Rahmen des durchlaufenen strukturierten Promotionsverfahrens, gab es kein Angebot, was dieses Wissen über Daten und Code vermittelt hätte und die Vermutung liegt nahe, dass das auch an anderen Universitäten nicht möglich ist.

Das ist auch deshalb wichtig, weil im Gegensatz zum Träger- und Speichermedium Papier wird wissenschaftliches Wissen im Rahmen der Digitalisierung als Code gespeichert. Wir werden die Übermittlung von Wissen bei der wissenschaftlichen Kommunikation aber nur dann verstehen können, wenn die Übermittlungswege und Formen transparent und offen gestaltet sind \cite{davis_2011_open}. Die wissenschaftliche Gemeinschaft darf sich nicht von der Auseinandersetzung mit der technologischen Zukunft abwenden, sonder muss versuchen ihre Logik zu verstehen. Johannes Näder zitiert in diesem Zusammenhang den französischen Philosophen Régis Debray laut dem "ein Diskurs über die Zwecke und Werte, der sich nicht auf einen präzisen Zustand der zur Verfügung stehenden Mittel stützt, (...) ein leerer Diskurs (ist). Aber ein Diskurs über die Innovation, der diese nicht im Lichte der Erinnerung genau untersucht, ist ein Diskurs." \cite[:117]{naeder_2010_open} \cite[:246]{debray2003einfuhrung}.

Hier liegt eine "Quelle des revolutionären Selbstverständnisses, das zumindest Teile der Open-Access-Bewegung trägt" und die Konsequenz aus der Digitalisierung des wissenschaftlichen Publikationswesens: Neben dem gedruckten Wort besteht der Kern von Wissen im digitalen Zeitalter eben nicht mehr aus dem gedruckten Wort sondern aus Code. Will man demnach die Rohform von Wissen lesen, verstehen, interpretieren, oder verändern - alles Grundvorraussetzungen für die Erstellung von wissenschaftlichen (Qualifikations-)Arbeiten - muss man diesen Code lesen, verstehen und schreiben können. Die Vorteile vom Teilen und Verbreiten von Wissen erfüllen sich demnach bisher nur für den, der für die Migration das nötige Know-How hat.

Der zunehmende Grad an Digitalisierung im Arbeitsalltag der Wissenschaftler und Wissenschaftlerinnen stellt die Notwendigkeit dar, sich mit den produzierten Daten auseinanderzusetzen. Dabei ist die Veränderungen der Arbeitsweise von analogen Methoden, Spreicher- und Arbeitsmedien sowie Tools auf digitale Formate für die Gewinnung von Wissen als unausweichlich zu betrachten. Diese Herausforderungen werden bei der Ausbildung von Nachwuchswissenschaftlern und Nachwuchswissenschaftlerinnen zu wenig berücksichtigt.

Dabei hat dieser Wandel Konsequenzen auf die Möglichkeiten für die Verbreitung, Erstellung und Speicherung der digital abgelegten Informationen und erlaubt eine Neujustierung der Produktion von Wissen und eine neue Form der wissenschaftlichen Kommunikation. Demnach addressiert Offenheit in Wissenschaft und Forschung den Kern der Produktion von Wissen und das betrifft uns alle. \cite{Mussell_2013} Die reinen Digitalisierung wissenschaftlicher Arbeitsprozesse und Anreicherung um die Möglichkeiten des digitalen Austauschs (Science 2.0) sowie die freie und offene Publikation finaler Forschungsergebnisse (Open Access) können nur durch eine Neugestaltung der Rahmenbedingungen zu einer umfassenden Öffnung dieser Kommunikation für die Gesamtgesellschaft (Open Science) führen.

Eine weitere Umstrukturierung dieser Rahmenbedingungen des wissenschaftlichen Publizierens ist nötig \cite{naeder_2010_open} und muss durch die wissenschaftliche Gemeinschaft beziehungsweise deren legitimen Vertreter und Vertreterinnen beeinflusst werden.

\section{Chancen für und Herausforderungen an die wissenschaftliche Gemeinschaft und an das System Universität}

\begin{quote}
\textbf{"Die Freiheit von Fremdbestimmung verpflichtet die wissenschaftliche Gemeinschaft und ihre Mitglieder zu verantwortlicher Selbstbestimmung."}
\end{quote} \cite{Oezmen_2015}

Die Abgrenzung von Open Access zu Open Science im Rahmen wissenschaftlicher Kommunikation wurde in dieser Arbeit auf Grundlage der Unterscheidung von "Zugang zu Wissen" (Open Access) und "Zugriff auf Wissen" (Open Science) durchgeführt. Das Konzept von Open Access und der damit verbundenen Verfügbarkeit von wissenschaftlichen Publikationen als Ergebnis von wissenschaftlicher Forschung im bestehenden wissenschaftlichen System betrifft demnach nur einen Teil der grundlegenden Neuordnung wissenschaftlicher Kommunikation. Open Science als Sammelbegriff adressiert auch weitere Teile und nicht nur die Digitalisierung der Aspekte rund um den Zugang zu fertigen wissenschaftlichen Publikationen, sondern fordert die Transformation und die Möglichkeit des umfassenden Zugriffs auf den gesamten wissenschaftlichen Prozess.

Rainer Kuhlen definierte diesbezüglich schon 2002 drei Szenarien wie der Zugriff auf Wissen in Zukunft organisiert sein könnte \cite{Kuhlen_2002_universalaccess}:
\begin{enumerate}
\item Elektronische Informationen sind frei zugänglich und die Konzepte der individuellen Autorenschaft und des geistigen Eigentums werden zu Relikten aus bürgerlichen Vorinformationsgesellschaften
\item Wissen und Informationen sind kontrolliert und dem Markt ohne politische Gegensteuerung überlassen: Die Kommerzialisierung und Zonierung von Wissen und Information wird umfassend sein und den Alltag bestimmen.
\item Wissen und Information werden über koexistente oder Paralleluniversen organisiert: Das Wissens als Produkt ist frei, öffentlich zugänglich und nutzbar. Es bleibt aber genug Spielraum bei der Adaption, Beratung, Veredlung oder anderen Mehrwertleistungen einer kommerziellen Informations- und Wissenswirtschaft
\end{enumerate}

Zusammenfassend muss für die Etablierung der Öffnung wissenschaftlicher Kommunikation festgehalten werden, dass es maßgeblich von den publizierenden Wissenschaftlern selbst abhängen wird eine treibende Kraft und aktive Position bei der Gestaltung des Wandels im Rahmen der wissenschaftlichen Kommunikation zu übernehmen. Denn spätestens wenn die genannten Rahmenbedingungen des wissenschaftlichen Publizierens angepasst sind und Open Access etabliert ist, braucht es auch eine Neujustierung der Qualitätssicherungsmaßnahmen und Selektionsmechanismen im Rahmen der weitestgehenden Öffnung des wissenschaftlichen Erkenntnisprozesses (Open Science). Meint man es mit der Öffnung ernst und beachtet die Herausforderungen ebenso, wie die Möglichkeiten muss das Ziel sein den Standard in dem wissenschaftlichen Kommunikationsprozess auf "offen" zu setzen und Ausnahmen dafür zu begründen.

Die Frage nach den Konsequenzen einer solch umfassenden Öffnung wissenschaftlicher Kommunikation ist dabei eng mit der Frage nach der zukünftigen Rolle der Universität und dem Hochschulwesen verbunden. Dabei ist die Ablösung des gedruckten Buchs "als Leitmedium der Universität" durch anderen Kommunikationsmitteln "lediglich als ein Epiphänomen einzustufen" \cite{Warnke_2012}, denn die Digitalisierung und zunehmende Verbreitung von Softwaresystemen, Unkenntnis ihrer Relevanz und Bedienung durch die wissenschaftlichen Akteure im wissenschaftlichen System bedroht zunehmend auch die Freiheit der Wissenschaft im Kern. "Eine Systemanalyse (...) des akademischen Alltagslebens  könnte ein wenig Klarheit und damit vielleicht auch Rückgewinnung von Gestaltungsraum geben" \cite{Warnke_2012}.

Sollten die Zeiten des "stürmischen Wachstums der Wissenschaft endgültig vorüber" sein \cite{K_lbel_2002}, lässt sich ein Grund dafür in der Geschlossenheit des wissenschaftlichen Kommunikationssystems finden. Mit der Verpflichtung auf Offenheit darf aber der Wissenschaft nicht die Unabhängigkeit und auch nicht "die Fähigkeit genommen werden, "Nein" zu sagen" \cite{suchen_Hornbostel_2006}. Unter Berücksichtigung dieser Faktoren ist zu hoffen, dass der neue Wachstum an Wissen durch den digitalen Wandel und die Öffnung wissenschaftlicher Kommunikation wirklich zu besseren Bedingungen für Schaffung neuen Wissens und die Bewahrung alten Wissens führen wird. Die Institutionen sowie Wissenschaftler und Wissenschaftlerinnen können im Rahmen des Prozesses um die Öffnung der Kommunikation demnach die Möglichkeiten eines neuen Gestaltungsspielraums nutzen um ihre Rolle als Produzent, Archivar und bei der Verbreitung von Wissen zurückzugewinnen.

Das ist allerdings aus diversen Gründen eine Herausforderung: Denn erstens ist wissenschaftliche Gemeinschaft gefragt den Wandel so zu gestalten, dass die Wissenschafts- und Publikationsfreiheit größtmöglich gewahrt wird und zweitens, dass der Wettbewerb um die Autorengebühren und Publikationsgeschwindigkeit nicht zu einer Bedrohung für die Zukunft der Wissenschaftskommunikation wird \cite{Beall_2012} \cite{Lossau_oa_2007}. Wissenschaftler und Wissenschaftlerinnen haben sich bisher "erstaunlich wenig für das Thema interessiert" \cite{hagner_2015_sache_buches}.

Läutete der Buchdruck die Moderne ein und legte den Grundstein für die wissenschaftliche Kommunikation wie wir sie heute kennen, wird im Rahmen der Digitalisierung eine erneute Revolution des wissenschaftlichen Systems bevorstehen. Die unmittelbare und umfassende Bereitstellung der wissenschaftlichen Kommunikation im Rahmen der alltäglichen wissenschaftlichen Arbeit unter den Bedingungen von größtmöglicher, stellt das wissenschaftlichen System aber innerhalb und außerhalb vor neue Herausforderungen. Das Aufbrechen der strikten Unterscheidung wissenschaftlicher Kommunikation in formelle und informelle sowie interne und externe Kommunikation und die Konsequenzen für die Bewertung und Einordnung dieser stellt dabei nur eine von vielen Aufgaben da.

Diese Entwicklung kann und muss von der wissenschaftlichen Community gestaltet werden, wenn sie nicht machtlos den Kräften ausgeliefert sein will, "die von außen auf die Publikation und Rezeption ihrer Schriften einwirken" \cite{Hirschi_2015_buch_oa}. Die Frage ist, ob sie entscheidet die Digitalisierung als Gefahr für den Fortbestand der Wissenschaft einfach nur zu negieren, als rein digitales Abbild der analogen Realität der wissenschaftlichen Kommunikation zu verstehen und das aktuelle System mit all seinen Vor- und Nachteilen zu bewahren oder wagen es die Wissenschaftler und Wissenschaftlerinnen eine zweite wissenschaftliche Revolution einzuläuten, die zu einer umfassenden Wissensverbreitung an die Gesamtgesellschaft und zu einer grundlegenden Veränderung des aktuellen wissenschaftlichen Systems führen könnte?

Aus der Forderung nach "unbeschränkten Zugang zur gesamten wissenschaftlichen Zeitschriftenliteratur" \cite{boai_2012} ein gesamtgesellschaftliches und umfassendes Modernisierungsvorhaben der Wissenschaft geworden, dass neben den Aspekten bezüglich der Zugänglichkeit zu Wissen und Wissenschaft eine Vielszal an weiteren Unzulänglichkeiten adressiert die den Fortbestand öffentlicher Forschung insgesamt beeinflussen \cite{brembs2015open}. Mit Blick auf die Umsetzung werden in der Ausseinandersetzung, anders als in es ursprünglich bei Open Access intendiert war, Einschränkungen der akademischen Freiheit befürchtet \cite{hagner_2015_sache_buches}.

Entzieht sich die wissenschaftliche Community dieser Auseinandersetzung, ist davon auszugehen, dass langfristig die Freiheit von Wissenschaft und Forschung darunter leidet und zunehmend rein politische und wirtschaftliche Interessen darüber entscheiden \cite{Warnke_2012}, wie, wann, wo und wozu Wissenschaftler und Wissenschaftlerinnen in Zukunft kommunizieren werden. Wenn Wissenschaftler und Wissenschaftlerinnen sich in der Auseinandersetzung mit der Forderung nach Öffnung von Kommunikation vornehmlich mit den Herausforderungen des Karrieredrangs und wirtschaftlichen Eigeninteressen befassen \cite{resnik_2005_ethics} besteht die Gefahr einer weiteren Verschließung wissenschaftlicher Kommunikation beziehungsweise eine (Aus-)Nutzung der Bewegung hin zur Öffnung durch den Drang zur Etablierung privatwirtschaftlicher Marktmechanismen zur Verwertung und Steuerung von Wissenschaft. Jegliche Abweichung, Einschränkung und Verwässerung von Openness begünstigt demnach die negative (Weiter-)Entwicklung hin zu einer rein privatwirtschaftlich oder aus politischen Interessen gesteuerten Wissenschaft. Die Vermutung liegt demnach nahe, dass die Allokation und Nutzung von Ressourcen für den wissenschaftlichen Erkenntnisprozess ausschließlich auf Grundlage von Marktmeachnismen die Hetrogenität und auch die Effizienz der Produktion von neuem Wissen langfristig negativ beeinflussen wird.

Die Forderung nach Offenheit von Wissenschaft und Forschung muss demnach nicht nur als "Strategie" gegen die unterschiedlichen Krisen im wissenschaftlichen Kommunikationssystem verstanden werden, die maßgeblich durch STM-Forscher gefördert wird und im Ergebnis zu einer weiterhin polarisierende Abwehr- und Gegenreaktion führen würde und eine weitere Verwässerung und Fehlleitung der ursprünglichen Ansätze zur Folge hätte \cite{naeder_2010_open}, sondern muss auch als Ansatz zur zukünftigen Sicherung der Freiheit von Wissenschaft verstanden werden. Sind wissenschaftlichen Aktivitäten nicht offen und zugänglich, steigt darüberhinaus die Gefahr, dass die öffentliche Unterstützung für die Wissenschaft erodiert und die Menschen vertrauen in ein System verlieren, dass sie nicht immer unmittelbar verstehen können \cite{resnik_2005_ethics}.

Wir befinden uns bei diesem Aushandlungsprozess erst am Anfang: Mehr als 500 Jahre Buchdruck stehen nur 25 Jahre Internet gegenüber. Vor 350 Jahren waren es Wissenschaftler, die sich zusammengetan haben um eine neue Philosophie für die Förderung von Wissen zu etablieren und das erste wissenschaftliche Journal zu gründen. Sicher ist auch, dass es nicht allein bei der Forderung nach dem Zugang zu wissenschaftlichen Publikationen bleiben wird. Die Umsetzung von Open Access wird früher oder später auch in einer Forderung nach Öffnung des wissenschaftlichen Erkenntnisprozesses münden. Wie bei Open Access ist auch hier die wissenschaftliche Gemeinschaft gefragt die Ausgestaltung aktiv und konstruktiv-kritisch zu beeinflussen.

Die Transformation des wissenschaftlichen Kommunikationssystems von der Gutenberg-Galaxie in den Turing-Galaxis verlangt eine Neugestaltung der Rahmenbedingungen für die wissenschaftliche Kommunikation und von allen Beteiligten eine Neudefinition ihrer Rolle in diesem System. Die neuen Möglichkeiten unterschiedlicher Formen der Darstellung von wissenschaftlichen Informationen sollte dabei als neue Chance für eine aktive Verbesserung, Gestaltung und Modifikation wissenschaftlicher Kommunikation verstanden und genutzt werden. Diese Neugestaltung unter Wahrung der Freiheiten des wissenschaftlichen Systems funktioniert jedoch nur, wenn die Beteiligten ihre Rolle als aktive Gestalter und Gestalterinnen wahrnehmen. Sie müssen dabei in angemessener Form aggieren und unbedingt vermeiden, dass der Öffnungs- und Digitalisierungsprozess das wissenschaftliche System technologisch oder ökonomisch rückständiger macht als das bisherige.

\section{Ausblick und Anknüpfungspunkte für weitere Forschungsbemühungen}

\begin{quote}
\textbf{We have reached a period in science somewhat simiular to that encountered by our colleagues of 300 years ago. Creative and inventive minds must now discover new methods for coping with the scienctific literature.}
\end{quote} \cite{porter_1964_scientific}

Als wichtige Felder für zukünftige Evaluationen sind die Themen Datenschutz und der Missbrauch von Forschung \cite{Fritsch_2015} zu benennen. Den Schutz der Privatsphäre gegen den immensen Wert von Open-Access-(Daten)nutzung auszugleichen und auszuhandeln stellt eine wichtige zukünftige Herausforderung dar. Dabei sollten in einer Debatte nicht nur die Vorteile für alle Beteiligten, sondern auch nicht sofort überschaubare Auswirkungen und Konsequenzen berücksichtigt werden. In diesem Zusammenhang benötigt es einen Aushandlungsprozess zwischen der wissenschaftlichen Gemeinschaft, der Politik und der Gesellschaft.

Eine weitere zu untersuchende Herausforderung stellt die mögliche Konkurrenz der wissenschaftlichen und medialen Kommunikation miteinander dar. Als Konsequenz der Öffnung der gesamten Wissenschaftlichen Kommunikation muss hinterfragt werden, ob Wahrheitsmonopol der Wissenschaft durch das Aufmerksamkeitsmonopol der Medien negativ beeinflusst werden könnte \cite{weingart_2005_wissenschaft}.

Die neuen Möglichkeiten der Quantifizierung von wissenschaflichen Tätigkeiten im Rahmen der revisionsgetrieben offenen wissenschaftlichen Publikation im digitalen Raum sowie deren Konsequenzen auf die Selbststeuerung von Wissenschaft bieten einen weiteren Ansatz für wissenschaftliche Analysen. Die Überwachungsmöglichkeiten der wissenschaftlichen Arbeit durch die Öffnung wissenschaftlicher Kommunikation stellt dabei eine Herausforderung für die Freiheit und den Datenschutz von Wissenschaftlern und Wissenschaftlerinnen dar.

---- TODO: weiter ausarbeiten und überarbeiten, ist bisher zu pathetisch und moegliche weitere Forschungsfelder aufzeigen & aufgreifen: Dispositiv; Gefahr Predatory Journals \cite{Beall_2012}; Popper einarbeiten; Wertschöpfungsprozess vs. Erkenntnisprozess <- siehe Einleitung; Anstatt den aussichtslosen Weg der gemeinsamen Definition von Öffnung wissenschaftlicher Kommunikations zu gehen, könnte man versuchen die Geschlossenheit des Systems zu definieren und sagen, welchen Zustand "man" definitiv nicht mehr haben will. Das würde den unterschiedlichen Wegen gerecht werden und man könnte diese daraufhin abprüfen - quasi aus den Erfahrungen der letzten Jahre "Öffnung wissenschaftlicher Kommunikation" eine Closed Definition ableiten <- vielleicht hier einbauen?; Wenn WissenschaftlerInnen nicht Coden können und nicht hinter den Computerbildschirm und hinter Software gucken, überlassen Sie die Codierung des Wissens anderen im schlimmsten denen, die damit unfug treiben - siehe auch https://www.academia.edu/6104267/Computerscreen_und_Tafelbild, um wirklich die Hohheit über Wissen zu behalten hilft es nicht da Wissen einzusperren (vor allem nicht hinter kommerziellen Paywalls) sondern mman muss den code verstehen, mit dem es geschrieben/gespeichert wird; Software und Informatik stukturieren Arbeitsprozesse \cite{Warnke_2012}  ----
