\chapter{Zusammenfassung und Ausblick}

In dieser Arbeit wurden die Herausforderungen der wissenschaftlichen Kommunikation im Rahmen der Digitalisierung und die Forderung nach Öffnung dieser umfassend diskutiert. Dabei wurden die einzelnen Definitionen von Open Access und Open Science, sowie Implikationen und Rahmenbedingungnen näher betrachtet, analysiert und spezifiziert. Ungeachtet der viefältigen Kritik am aktuellen Kommunikationssystem, scheint dieses jedoch auch nach zwei Jahrzehnten noch immer weitestgehend stabil.

Nach einer Einführung in den Themenbereich der wissenschaftlichen Kommunikation und des wissenschafltichen Publizierens wurde mit anhand aktueller Literatur zum Thema die Debatte um die Forderung nach Öffnung der wissenschaftlichen Kommunikation dargestellt. Dabei wurden Treiber und Bremser für die Entwicklungen identifiziert 

Diese Treiber und Bremser wurden darauf hin mithilfe einer repräsentativen Umfrage unter Wissenschaftlern genauer abgefragt. Dabei wurde analysiert, inwiefern ein Verständnis von Offenheit im Rahmen der wissenschaftlichen Kommunikation vorherrscht und inwieweit Offenheit in den unterschiedlichen wissenschaftlichen Fachdisziplinen verbreitet ist und praktiziert wird.

\section{Definition von Open Access und Open Science in der Literatur}
\subsection{Open Access}
\subsection{Open Science}
\section{Die Öffnung von Wissenschaft auf Grundlage der Befragung}
\subsection{Treiber und Bremser}
\subsection{Differenzierung zwischen den wissenschaftlichen Disziplinen auf Grundlage der Befragung}
\section{Feldbericht und Beobachtungen im Rahmen des Experiments}
\subsection{Handlungsempfehlungen zum Verfassen offener Dissertationen}


Sollten die Zeiten des "stürmischen Wachstums der Wissenschaft endgültig vorüber" \cite{K_lbel_2002} sein, lässt sich ein Grund dafür in der Geschlossenheit des wissenschaftlichen Kommunikationssystems vermuten.