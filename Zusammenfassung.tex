\chapter{Zusammenfassung und Ausblick}

\begin{quote}
\textbf{"Denn ich habs umsonst empfangen, umsonst hab ich's gegeben und begehre auch dafür nichts"}
\end{quote} \cite{luther_1876}

Die unbeschränkte und offene wissenschaftliche Kommunikation scheint für das Wissenschaftssystem theoretisch unverzichtbar. Wissenschaft und Forschung sind folglich eng mit den Normen der schnellen Weitergabe von Forschungsergebnissen, einer Umgebung des Wissensaustauschs, Co-Autorenschaft und dem kumulativen Lernen sowie Innovationen verbunden \cite{Partha_1994_economics_science}. In der wissenschaftlichen Realität basiert die wissenschaftliche Arbeit jedoch auf einem in sich geschlossenen System und noch immer auf der Annahme, das "was nicht gedruckt wird, kaum Chancen hat, die Entwicklung des Faches zu beeinflussen" \cite{luhmann_1997_gesellschaft}.

In dieser Arbeit wurden die Herausforderungen an das System der wissenschaftlichen Kommunikation im Rahmen der Digitalisierung und die Forderung nach Öffnung dieser umfassend dargestellt und analysiert. Die Entwicklungen im Bereich der Öffnung wissenschaftlicher Kommunikation wurden aus geistes- und kulturwissenschaftlicher Perspektive genauer untersucht und den bisherigen Erkenntnissen über die Öffnung wissenschaftlicher Kommunikation gegenübergestellt, sowie das Ergebnis dieser Gegenüberstellung diskutiert.

Nach einer Einführung in die Thematik wissenschaftlichen Kommunizierens, der Ausführung des Vorgehens und der Darstellung der Beweggründe und der eigenen Position, wurden die methodischen Herangehensweisen beschrieben. Im weiteren Verlauf wurden die theoretischen Grundlagen dargestellt, Debatten um Veränderungen im wissenschaftlichen Kommunikationssystem erläutert und weitere Annahmen, sowie die Katalysatoren und Hindernisse für die Entwicklungen herausgearbeitet. Mit Hilfe der quantitativen Methode einer Online-Befragung wurde unter 1.112 Wissenschaftlern und Wissenschaftlerinnen analysiert, welche Auffassungen und Annahmen in Bezug auf den postulierten Wandel wissenschaftlicher Kommunikation im Rahmen von Offenheit und Digitalisierung vorherrschen und inwiefern diese mit anderen Aspekten des wissenschaftlichen Kommunikationssystems korrelieren. Diese Daten wurden in den Kontext bisheriger Untersuchungen gestellt und analysiert. Im sechsten Kapitel wurden die gewonnen Erkenntnisse aus den Analysen der Debatten und der Befragung experimentell anhand der eigenen Öffnung des wissenschaftlichen Erkenntnisprozesses im Rahmen der Anfertigung dieser Arbeit autoethnografisch erarbeitet und dargestellt. Die Ergebnisse dieser Arbeit wurden im siebten Kapitel diskutiert und werden im Folgenden zusammengefasst sowie mögliche Anknüpfungspunkte für weitere Forschungsbemühungen genannt.

Zusammenfassend kann festgestellt werden, dass es bisher weder gelungen ist, die nötigen Anreize für die einzelnen Wissenschaftler so zu setzen, dass deren Eigeninteresse im Rahmen der Verbreitung von Erkenntnissen mit dem Wohl der Wissenschaft und dem der Öffentlichkeit gleichermaßen harmonieren, noch gab es bisher staatliche Interventionen, die zu einer fundamentalen Veränderung im Publikationsverhalten geführt haben.

Bei den aktuellen Entwicklungen handelt es sich um einen umfassenden Medienwandel, der neue Möglichkeiten eröffnet, wobei es auch um die Frage geht, wie wir den vernetzten Computer einsetzen, welche Möglichkeiten sich für eine Neuordnung der wissenschaftlichen Kommunikation anbieten und wie die wissenschaftliche Gemeinschaft diese Technologien und Geschäftsmodelle - vielleicht auch zusammen mit Bewegungen wie der Open Source Bewegung - prägen. Dabei muss auch neu verhandelt werden, welche Werte für die wissenschaftliche Praxis durch neue Medientechnologien definiert oder von der Bewegung für das freie und offene Netz übernommen werden können.

Neben der Erkenntnis, dass die Hindernisse für die nötigen Veränderungen nicht ausschließlich technische oder finanzielle, sondern sozialer Natur sind \cite{nosek_2012_scientific}, muss weiterhin die klare rechtliche Klärung für die Zweitverwertung von Inhalten als ein wichtiger Katalysator für die Entwicklung gesehen werden. Darüber hinaus fehlen noch immer etablierte Reputationsmechanismen, die die Öffnung wissenschaftlicher Kommunikation befördern. Diese Maßnahmen können aber nur dann erfolgreich sein, wenn eine Diskussion für die Gestaltung der Zukunft wissenschaftlicher Kommunikation innerhalb der wissenschaftlichen Gemeinschaft unterstützt und die wissenschaftlichen Institutionen für diese Diskussion den Raum schaffen.

Wird dieser Raum nicht geschaffen und die Öffnung der wissenschaftlichen Kommunikation weiter ausschließlich durch externe und politisch motivierte Maßnahmen angestrebt, werden rein kommerzielle, forschungs- und steuerungspolitische Interessen den Öffnungsprozess weiter vorantreiben. Diese Entwicklung gilt es zu vermeiden, da so die einstigen Ideale und der ursprünglich proklamierte Nutzen einer Öffnung der wissenschaftlichen Kommunikation nicht der Gesamtgesellschaft zu Gute kommen, sondern nur partikularen Interessen oder zu einer Medialisierung der Wissenschaft mit negativen Konsequenzen auf das Wahrheitsmonopol und die Unabhängigkeit der Wissenschaft führt. Verlage wie Elsevier und andere wirken in diesem Zusammenhang seit Dekaden auf die forschungspolitische Agenda ein und versuchen ihre wirtschaftlichen Interessen im Rahmen des Wandels durchzusetzen. Die wissenschaftliche Gemeinschaft muss endlich durch die Förderung der Katalysatoren und die Beseitigung der Hindernisse den unvermeidlichen Wandel im Rahmen der Digitalisierung eigeninitiativ gestalten. Wenn in diesem Prozess schon Verlage fordern, dass "Autoren in einem gesunden, unverzerrten freien Markt frei wählen sollten, wo sie publizieren" \cite{Brussels_Declaration_2007}, muss sich die wissenschaftliche Gemeinschaft fragen lassen, ob sie oder Verlage die Ausgestaltung des wissenschaftlichen Kommunikationssystems gestalten und ob sie in der Vergangenheit die Publikationsfreiheit ausreichend und selbstbestimmt mit dem Ziel der möglichst umfassenden Verbreitung von Wissen genutzt haben.

\section{Interesse an und Verbreitung von Öffnung wissenschaftlicher Kommunikation und der wissenschaftliche Alltag}

Die Ergebnisse der im Rahmen dieser Arbeit durchgeführten Befragung zeigen eine überraschend mehrheitlich große Zustimmung und ein überwiegend großes Interesse an der Öffnung wissenschaftlicher Kommunikation. Im wissenschaftlichen Alltag hat dieses Interesse und die Zustimmung zu digitalen und offenen Verfahren der Kommunikation bisher jedoch noch nicht zu einer fundamentalen Veränderung des Publikations- und Veröffentlichungsverhaltens geführt. Hier muss die bisherige Rolle der wissenschaftlichen Gemeinschaft und ihrer Auseinandersetzung mit dem Prozess der Digitalisierung und Öffnung wissenschaftlicher Kommunikation genauer betrachtet werden.

Sucht man nach Gründen für das Desinteresse bei der praktischen Umsetzung des durchaus vorhandenen Interesses an Offenheit im wissenschaftlichen Alltag, wird deutlich, dass unter anderem unvollständiges Wissen über die wirtschaftlichen Aspekte wissenschaftlicher Informationsversorgung für die Diskrepanz zwischen dem Interesse an Offenheit und der tatsächlichen Publikationspraxis eine Rolle spielen. In der untersuchten Literatur wird diese mit der komfortablen Situation der Wissenschaftler in einem System beschrieben, in dem für die meisten Mitglieder der wissenschaftlichen Gemeinschaft kein oder nur geringer unmittelbarer Anreiz besteht, sich aktiv mit dem Publikationssystem und möglichen Veränderungen zu beschäftigen, weil sie weder die Kosten des Publikationssystems tragen \cite{Sietmann_oa_2007}, noch sich mit den finanziellen Aspekten auseinander setzen müssen \cite{herb_2010}. Darüber hinaus kommt es im Zuge der zunehmend quantitativen Bewertung von Forschungsleistung und der vornehmlichen Berücksichtigung quantitativer Indikatoren (deren Qualitätsmessung nach wissenschaftlichen Methoden kritisiert werden kann) bei der Bewertung für die Mittelzuweisung zu einer Entwicklung von Fehlanreizen nach denen die reine Anzahl und nicht die Qualität der Veröffentlichung in Publikationen etablierter Verlage bei wissenschaftlicher Arbeit im Vordergrund steht.

Ergänzend muss für die schleppende Umsetzung der offenbar geringe Bedarf an Veränderung im Alltag der Wissenschaftlerinnen und Wissenschaftler genannt werden. Es gibt kaum positive Anreize sich mit der Entscheidung wo und wie veröffentlicht wird sowie mit den Konsequenzen dieser Entscheidung auseinander zu setzen, da diese im wissenschaftlichen Reputationssystem bisher nicht oder nur unzureichend abgebildet und honoriert werden. Neben den wirtschaftlichen, und den begrenzten zeitlichen Ressourcen, sich mit den umfangreichen Aspekten der Informationsversorgung auseinanderzusetzen, gibt es zudem weiterhin viele rechtliche Unsicherheiten, die die Wissenschaftlerinnen und Wissenschaftler davon abhalten ihre Kommunikation zu öffnen.

Als Gründe für die die schleppende Umsetzung des Konzepts von Offenheit in der wissenschaftlichen Kommunikation aber allein eine Kombination aus Beharrungsvermögen der wissenschaftlichen Akteure, Unsicherheit über alternative Publikationsmodelle und die rechtliche Situation sowie die Existenz von Gruppen die in die Bewahrung der Ineffizienz des gegenwärtigen Systems investiert haben auszumachen \cite{nosek_2012_scientific}, muss dennoch als unzureichender Erklärungsversuch betrachtet werden. Als weitere Gründe für die geringe Verbreitung von der proaktiven Öffnung von Wissenschaft und Forschung seitens der wissenschaftlichen Gemeinschaft müssen auch Unterschiede und fachspezifischen Eigenheiten der wissenschaftlichen Kommunikation, wie zum Beispiel die unterschiedlichen Publikationsformen, genannt werden. Möglichkeiten und Debatten sind bewusst überwiegend auf die jeweilige Disziplin beschränkt und berücksichtigen selten andere Forschungsrichtungen.

Die ersten offenen Publikationsvorhaben und Entwicklungen zu Open Access erfolgten in erster Linie aus den STM-Fächern, die von von der Zeitschriftenkrise viel stärker und früher betroffen waren als andere Fächer. Die daraus resultierten Erklärungen und Bestrebungen führten bisher zu erheblichen "Vorbehalten hinsichtlich der Sinnhaftigkeit und Durchführbarkeit von Open Access, die wiederum zu Desinteresse oder Polarisierung bei vielen Vertretern dieser Disziplinen" \cite{naeder_2010_open} zur Folge hatte. Diese Polarisierung stellt weiterhin eine große Herausforderung für die Etablierung und Verbreitung der Konzepte für die Öffnung wissenschaftlicher Kommunikation dar. Derzeit ist eine Darstellung des Interesses an der Öffnung wissenschaftlicher Kommunikation über alle wissenschaftlichen Fachbereiche hinaus und ein gemeinsames Handeln der wissenschaftlichen Gemeinschaft bisher nur schwer vorstellbar.

Christopher Kelty sieht darüber hinaus für das Desinteresse an der Praktizierung von Offenheit im wissenschaftlichen Kommunikationssystem und Alltag folgende zwei Aspekte: Erstens, ist die Auseinandersetzung mit Offenheit für die meisten ziemlich langweilig und zweitens ist das Thema bei genauerer Betrachtung eben sehr komplex. Darüber hinaus würden durch die umfassende Auseinandersetzung mit dem wissenschaftlichen Publikationsmarkt, "Praktiken auf dem Spiel stehen, die dem Tun vieler Geisteswissenschaftler (...)" und auch Wissenschaftler anderer Fächer, "Sinn und Legitimität zu verleihen scheinen" \cite{Hirschi_2015_buch_oa}, die lieber nicht hinterfragt werden.

Die Ergebnisse der durchgeführten Befragung nähren all diese Befürchtungen und lassen vermuten, dass das Gros der Wissenschaftler und Wissenschaftlerinnen auch in naher Zukunft keine hervorgehobene Rolle bei der Gestaltung der Transformation der Kommunikation einnehmen und sich trotz der Möglichkeiten zur Veränderung, eher der Status quo bewahren wird \cite{nosek_2012_scientific}. Diese Befürchtung ist darüber hinaus gesondert im Spannungsfeld zwischen Wissenschaft und Politik zu sehen, in dem Wissenschaftler für sich in Anspruch nehmen als Strategen im politischen Kampf um Glaubwürdigkeit für die wissenschaftliche Arbeit zu agieren \cite{latour_2013_laboratory}. Diese beiden Rollen nehmen sie aktuell allerdings nur unzureichend war. Eine tiefgreifende Debatte innerhalb der wissenschaftlichen Community findet nicht statt.

Das theoretische und ideelle Interesse steht somit einem praktischen Alltags-Desintresse an der Umsetzung sowie den Herausforderungen im aktuellen Publikationssystem gegenüber, von denen die Forscherinnen und Forscher bisher nur begrenzt direkt betroffen sind. Das ist theoretisch keine schlechte Situation, wenn man darüber hinaus davon ausgeht, dass das Wissen über das Kommunikationssystem, die Öffnung wissenschaftlicher Kommunikation und die Verbreitung digitaler Arbeitsmittel und -methoden noch immer sehr begrenzt ist, aber unaufhaltsam ansteigt. Dennoch muss weiter auf die Stärkung der in dieser Arbeit identifizierten Katalysatoren neben der Beseitigung der genannten Hindernisse für die Etablierung der Öffnung wissenschaftlicher Kommunikation hingearbeitet werden.

Diese Auseinandersetzung mit den Katalysatoren und Hindernissen beziehen sich ferner auf die Anpassungen bei der Qualitätssicherung und auf Veränderungen im Rahmen des wissenschaftlichen Reputationssystems sowie deren Konsequenzen. Die Öffnung wissenschaftlicher Publikationen ist dabei nur der erste Schritt in Richtung einer Öffnung des gesamten wissenschaftlichen Erkenntnisprozesses. Die dabei entstehenden notwendigen Anpassungen der Methoden, um gute wissenschaftliche Praxis bei der Digitalisierung wissenschaftlicher Arbeit zu einem guten Standard in der Wissenschaft zu machen, obliegt ebenfalls der Wissenschaftsgemeinschaft selbst. Vor allem in Anbetracht der Tatsache, dass diese neu zu verhandelnen Kriterien für die wissenschaftliche Praxis unter dem Einsatz neuer Medientechnologien sowie den Bedingungen von Offenheit einen direkten und unmittelbaren Einfluss auf die Bewertung und (Selbst-)Steuerung der Wissenschaft haben, muss die Wissenschaftsgemeinschaft Verantwortung übernehmen und diesen Prozess aktiv gestalten. Diese Verantwortung sollte nicht durch die Wahl "gar nicht mehr auf der Universitätsseite zu erscheinen oder keine jährlichen Geldmittel für ihre Publikationsergebnisse zu erhalten" \cite{Warnke_2012} entschieden werden, sondern durch eine aktive Ausgestaltung der Rahmenbedinungen der wissenschaftlichen Kommunikation und damit auch der Steuerung des wissenschaftlichen Systems.

Die Öffnung der wissenschaftlichen Kommunikation ist auf der einen Seite mit vielen Anstrengungen für die wissenschaftliche Gemeinschaft verbunden, kann aber auch als "vielleicht kostbarste Geschenk des Internets an die Wissensgesellschaft" betrachtet werden, wenn sie nicht ausschließlich durch die "ökonomischen Interessen des Informationskapitalismus" \cite{hagner_2015_sache_buches} gelenkt wird, sondern sich (weiterhin) maßgeblich auf die Aufgabe konzentriert dem gesamtgesellschaftlichen Auftrag des Wissenschaftssystems gerecht zu werden. Jeder Autor und jede Autorin entscheiden selbst, allerdings unter dem Kenntnisstand der Folgen und Konsequenzen für die Zugänglichkeit durch andere, ob und wie er seine oder sie ihre Forschungsergebnisse verbreitet. Der Prozess der Erfüllung akademischer Erwartungen für die Veröffentlichung von Erkenntnissen unter den Regeln der jeweiligen Disziplin, unter der Verwendung der gegebenen intellektuellen Rahmenbedingungen und dem Anspruch  mit Kollegen zurechtzukommen macht es allerding sehr einfach, den sozialen Idealismus auf den akademischen Syllabus zu beschränken und sich nicht an aktivistischen Veränderungsprozessen zu beteiligen \cite[:25]{flood_2013_combining}.

\section{Katalysatoren und Hindernisse für die Etablierung der Öffnung wissenschaftlicher Kommunikation}

Nach der Einführung in den Themenbereich der wissenschaftlichen Kommunikation und des wissenschaftlichen Publizierens wurden anhand aktueller Literatur die Entwicklung und die Debatte um die Forderung nach Öffnung der wissenschaftlichen Kommunikation dargestellt. Aus den dargestellten Debatten wurden Treiber und Bremser für die Entwicklungen identifiziert und herausgearbeitet. Diese wurden daraufhin mithilfe einer Online-Befragung unter 1.112 deutschsprachigen Wissenschaftlerinnen und Wissenschaftlern detailiert evaluiert. Dabei wurde analysiert, inwiefern ein Verständnis von Offenheit im Rahmen der wissenschaftlichen Kommunikation besteht und inwieweit das Interesse an Offenheit in den unterschiedlichen wissenschaftlichen Fachdisziplinen verbreitet ist und praktiziert wird. Die Ergebnisse wurden mit einer Studie des Soziologischen Forschungsinstituts Göttingen aus dem Jahr 2007 verglichen um Trends und Entwicklungen zu identifizieren.

In den Augen der befragten Wissenschaftlerinnen und Wissenschaftler stehen die Beschleunigung der Wissensverbreitung (65 Prozent), die Eröffnung neuer Möglichkeiten für Wissensverbreitung (64 Prozent) und die offene Verfügbarkeit bereits finanzierter Forschung für alle (55 Prozent) vor allem den Herausforderungen nach etablierten Reputationskriterien für die Bewertung von offener Wissenschaft (43 Prozent), der Gefahr der Fehlinterpretation und Falschinformation (40 Prozent) sowie einem erhöhten zeitlichen Mehraufwand für die Bereitstellung der wissenschaftlichen Publikationen und/oder Forschungsdaten (34 Prozent) gegenüber. Die 1.112 Befragten gaben außerdem mehrheitlich an, dass sie rechtliche Bedenken (39 Prozent) und Unwissenheit über die Erlaubnis (29 Prozent) davon abhält, wissenschaftliche Inhalte ohne finanzielle, rechtliche oder technische Barrieren öffentlich zur Verfügung zu stellen.

Die Ergebnisse der Studie können insofern als repräsentativ gelten, als dass sie auf einer relativ großen Stichprobe beruhen. In der Auswertung der Daten wurde vor allem deutlich, dass die Hindernisse eine größere Verteilung aufwiesen als die Katalysatoren und dass die Verbreitung offener wissenschaftlicher Kommunikationsverfahren nicht unwesentlich mit der Fachrichtung der jeweiligen Autoren korreliert. Darüber hinaus scheint ungeachtet der vielfältigen Kritik am aktuellen Publikations- und Kommunikationssystem, dieses auch nach zwei Jahrzehnten noch immer weitestgehend stabil zu sein. Die Ergebnisse der Befragung belegen die in der Literatur immer wieder beschriebene Diskrepanz zwischen dem Interesse nach, dem Verständnis für Offenheit und der tatsächlich praktizierten offenen Kommunikations- und Arbeitsweise \cite{yiotis_2013_open} \cite{Bartling_2013} \cite{hagner_2015_sache_buches} \cite{Fecher_2015}.

Viele Programme der Forschungsförderungsorganisationen zielen noch immer darauf ab, fast ausschließlich klassisch zu publizieren und nur sehr langsam kommt es zur Unterstützung von digitalen Forschungsinfrastruktur oder Softwareentwicklungsprogrammen \cite{hey_2015_open}. Sie behindern damit die nötigen Veränderungsprozesse im Rahmen der Digitalisierung des wissenschaftlichen Alltags und befördern damit das Beharrungsvermögen das aktuelle wissenschaftliche  Kommunikationssystem zu unterstützen. Förderorganisationen müssen jedoch die Veränderungsprozesse in der wissenschaftliche Praxis durch die neue Medientechnologien akzeptieren und ihre Verantwortung wahrnehmen, indem sie die zusätzlichen Ressourcen, die für die Schaffung der strukturellen Grundlagen die mit der Öffnung von Wissenschaft und Forschung verbunden sind, zur Verfügung stellen \cite{mennes_2013_making_os} \cite{patlak_2010_open}. Um diese Bemühungen voranzubringen, müssen sich Förderorganisationen mehr denn je ihrer Rolle als "einflussreiche  Akteuren  im  komplexen und  sich  wandelnden  Markt  für  wissenschaftliche  Publikationen" \cite{wein_2010_erwerbung} bewusst werden und entscheiden, ob sie die Umsetzung der gemeinsamen Nutzung von Daten durch Incentivierung fördern \cite{mennes_2013_making_os}. Bisher ist diese Nutzung wissenschaftlicher Daten nur sehr gering verbreitet und auch wenn die Erwähnung der Weiternutzung von wissenschaftlichen Daten ansteigt, bleiben bis zu 86 Prozent der veröffentlichen Daten bisher ungenutzt beziehungsweise unzitiert \cite{peters_2015_research}.

Die Entwicklungen der letzten Jahre zeigen auch, dass Verlage aktiv daran arbeiten den digitalen Wandel zu nutzen, um ihre komfortable Situation in dem System der wissenschaftlichen Kommunikation zu sichern, wenn nicht sogar auszubauen. Dabei stellen die technischen und finanziellen Defizite und Unzulänglichkeiten für die Umsetzung der ursprünglichen Ideale der Öffnung der wissenschaftlichen Kommunikation für die Gesamtgesellschaft im Rahmen der Digitalisierung nur einen Teil der Gründe für die noch immer vorherschende Beständigkeit des aktuellen Systems und der starken Rolle von Verlagen bei der Neugestaltung der wissenschaftlichen Kommunikation dar. Würden die wissenschaftliche Gemeinschaft und die Fachgesellschaften, wie auch die Bibliotheken stärker den Wandel weg von Abonnement-basierten Modellen unterstützen, wäre die Entwicklung womöglich bereits weiter fortgeschritten \cite{nosek_2012_scientific}.

25 Jahre nach den ersten Versuchen einen offenen Zugang zu wissenschaftlicher Kommunikation mit Hilfe digitaler Netze umzusetzen, einer Vielzahl weicher Erklärungen und (teilweise leeren) Bekenntnissen für die Öffnung wissenschaftlicher Kommunikation gibt es kaum noch Zweifel, dass sich das System verändern wird. Dennoch bestehen Bedenken bezüglich der genauen Ausgestaltung von Offenheit wissenschaftlicher Kommunikation, der Frage ob die letztendlich erzielte Öffnung noch den ursprünglichen Anspruch an Verbesserung des Systems gerecht werden kann und insbesondere bezüglich der Frage wie sich die Vor- und Nachteile dieser Entwicklung letztendlich zueinander verhalten werden \cite{hagner_2015_sache_buches}. Bisher hat sich die wissenschaftliche Gemeinschaft eher verhalten an dem Veränderungsprozess beteiligt. Sollte es durch das weitere Ausbleiben der Gestaltung aus der wissenschaftlichen Gemeinschaft heraus zu einem Eingriff der Politik in diesem "sensiblen Bereich" führen, "muss besonders acht auf die komplexe Geschichte der Organisationen der öffentlichen Wissenschaft gegeben und die potenzielle Fragilität der eigentümlichen institutionellen Matrix respektiert werden, in der sich die modernen Forschung entwickelt hat und aufgeblüht ist" \cite{david1998_common}. Darüber hinaus müssen die Rahmenbedingungen ausgehandelt werden, unter denen die Digitalisierung der wissenschaftlichen Kommunikation und die Öffnung von Wissenschaft und Forschung stattfindet \cite{mennes_2013_making_os}.

\section{Erkenntnisse aus dem offenen Verfassen der Arbeit}

Hemmnisse für die Umsetzung und Etablierung der Konzepte um die offen praktizierte Wissenschaft sind die genannten fehlenden rechtlichen Rahmenbedingungen und ökonomischen Vorrausetzungen wie auch die gering verbreiteten technischen Möglichkeiten und Standards. Die zur Verfügung stehenden Plattformen und Applikationen sind auch 2015 noch nicht ausgereift, etabliert und zweckdienlich genug um im Alltag ohne viel Mehraufwände offene Wissenschaft zu praktizieren. Dabei gilt es zu berücksichtigen, dass die wissenschaftliche Arbeit, trotz zunehmender Digitalisierung, seit Dekaden auf die geschlossene Publikation und den nicht-öffentlichen Publikationsprozess ausgelegt ist sowie dem Druck der Etablierung des Marktmodus als dominante Governanceform von Wissenschaft ausgesetzt ist.

Als Erkenntnis des Experiments der offenen Anfertigung dieser Arbeit kann festgestellt werden, dass der offene wissenschaftliche Erkenntnisprozess grundsätzlich möglich ist, die Möglichkeiten für die Anfertigung von offenen wissenschaftlichen (Qualifikations-)Arbeiten aber dennoch unzureichend sind. Stellt man die gewohnte wissenschaftliche Arbeitsweise dem offenen Erstellungsprozess dieser Arbeit gegenüber, so muss die Arbeit auf dem lokalen Computer (selbst bei der Verwendung internetbasierter Dienste) in einem geschlossenen Umfeld noch immer als um vieles einfacher als das öffentliche Verfassen einer Arbeit bewertet werden. Das hat zum einen mit den gewohnten und etablierten strukturellen, technischen sowie rechtlichen Umgebungen wissenschaftlicher Arbeit zu tun, die überwiegend inkompatibel zu der offenen Darstellung und Verbreitung von Inhalten sind. Zum Anderen müssen diese fehlende Möglichkeiten und Funktionen für die offene Arbeitsweise sowie die daraus resultiererenden Einschränkungen bei Bedienbarkeit und den Abläufen durch mehr Aufwand und manuelle Arbeit seitens der Forscherinnen und Forscher kompensiert werden.

So erscheint es fast verständlich, dass bisher nur eine Minderheit offene Webplattformen für die wissenschaftliche Textarbeit \cite{Perkel_2014} nutzt und die Mehrheit der im Rahmen dieser Arbeit befragen Wissenschaftler und Wissenschaftlerinnen bei der Öffnung von Forschungsdaten einen großen Aufwand befürchtet, wenngleich die Daten durch den zunehmende Einsatz computerunterstützter wissenschaftlicher Verfahren bereits digital vorliegen. Ergänzend zeigen die Ergebnisse des Selbstexperiments, dass die Offenlegung des gesamten Erkenntnisprozessess bei der Erstellung dieser Arbeit ohne programmiertechnische Vorkenntnisse schwer bis nicht möglich gewesen wäre. Auch hier stellten fehlende Standards und technische Hürden große Herausforderungen bei der Auswertung, Erstellung und Darstellung der Inhalte dar. Darüber hinaus konnten spezifische Anforderungen von den gängigen Lösungen, wie zum Beispiel die Un­weg­sam­keiten bei der offenen Bereitstellung dieser Arbeit gezeigt haben, bisher nicht bedient werden. Dennoch wird die notwendige (Weiter-)Entwicklung der Plattformen nur dann stattfinden, wenn die Nachfrage danach steigt. Die wissenschaftliche Gemeinschaft ist auch hier gefragt, diese Nachfrage zu erzeugen und bei der Entwicklung solcher Lösungen eine aktive Rolle einzunehmen. Das Beispiel der kollaborativen wissenschaftlichen Textverarbeitungsplattform Authorea zeigt, wie ehemalige Wissenschaftler die eigene Situation nutzen können um die Entwicklung solcher Plattformen aktiv voranzutreiben.

Allerdings bedeutet das bisher für den "Open Scientist" entweder selbst befähigt zu sein, zu programmieren, beziehungsweise bestehenden Code nach den eigenen spezifischen Bedürfnissen anpassen zu können, oder Bibliotheken, Rechenzentren oder andere Einrichtungen der wissenschaftlichen Institutionen müssen die infrastrukturellen Rahmenbedingungen schaffen, so dass auch durch Wissenschaftler und Wissenschaftlerinnen ohne solche Kenntnisse der gesamte wissenschaftliche Prozess offen und transparent abgebildet werden kann. Im Rahmen des durchlaufenen strukturierten Promotionsverfahrens, war kein solches Angebot verfügbar, was dieses Wissen über Daten und Code vermittelt oder das Vorhaben aktiv technisch unterstützt hätte.

Abgesehen von der Erstellung eigener Arbeiten ist für die Wissenschaftlerin oder den Wissenschaftler diese Expertise in Zukunft auch deshalb wichtig, weil im Gegensatz zum Träger- und Speichermedium Papier das wissenschaftliches Wissen im Rahmen der Digitalisierung zunehmen als Code gespeichert wird. Die Übermittlung von Wissen bei der wissenschaftlichen Kommunikation kann aber von den beteiligten Akteuren nur profund verstanden werden, wenn auf auf technisches Wissen zurückgegriffen werden kann und die Übermittlungswege und Formen transparent und offen gestaltet sind \cite{davis_2011_open}. Die wissenschaftliche Gemeinschaft darf sich vor dieser Auseinandersetzung mit den technologischen Arbeitsmitteln und dem digitalen Wandel nicht abwenden, sondern ist gefordert ihre Logik zu verstehen. Johannes Näder zitiert in diesem Zusammenhang den französischen Philosophen Régis Debray laut dem "ein Diskurs über die Zwecke und Werte, der sich nicht auf einen präzisen Zustand der zur Verfügung stehenden Mittel stützt, (...) ein leerer Diskurs (ist). Aber ein Diskurs über die Innovation, der diese nicht im Lichte der Erinnerung genau untersucht, ist ein Diskurs." \cite[:117]{naeder_2010_open} \cite[:246]{debray2003einfuhrung}.

Genau bei diesem Diskurs liegt auch eine Quelle des revolutionären Selbstverständnisses, das zumindest Teile der Open-Access-Bewegung und das Konsequenzen für das gesamte wissenschaftliche System hat: Neben dem gedruckten Wort besteht der Kern von Wissen im digitalen Zeitalter eben nicht mehr aus dem gedruckten Wort sondern aus Code und Daten. Will man demzufolge die Rohform von Wissen lesen, verstehen, interpretieren, oder verändern - alles Grundvorraussetzungen für die Erstellung wissenschaftlicher (Qualifikations-)Arbeiten - muss man diesen Code lesen, verstehen und schreiben können. Die Vorteile von digitalem Teilen und Verbreiten von Wissen erfüllen sich folglich bisher nur für den, der für die Migration das nötige Know-How hat. Der zunehmende Grad an Digitalisierung im Arbeitsalltag der Wissenschaftler und Wissenschaftlerinnen stellt die Notwendigkeit dar, sich mit den produzierten Daten auseinanderzusetzen. Dabei ist die Veränderungen der Arbeitsweise analoger Methoden, von Speicher- und Arbeitsmedien sowie Tools auf digitale Formate für die Gewinnung von Wissen als unausweichlich zu betrachten. Diese Herausforderungen finden bei der Ausbildung von Nachwuchswissenschaftlern und Nachwuchswissenschaftlerinnen bisher viel zu wenig Berücksichtigung.

\section{Chancen für und Herausforderungen an die wissenschaftliche Gemeinschaft und an das System Universität}

\begin{quote}
\textbf{"Die Freiheit von Fremdbestimmung verpflichtet die wissenschaftliche Gemeinschaft und ihre Mitglieder zu verantwortlicher Selbstbestimmung."}
\end{quote} \cite{Oezmen_2015}

Die Öffnung wissenschaftlicher Kommunikation und der digitale Wandel haben Konsequenzen auf die Möglichkeiten für die Verbreitung, Erstellung und Speicherung der wissenschaftlichen Informationen. Sie erlauben eine grundlegende Neujustierung der Produktion von Wissen, die Neuordnung von wissenschaftlichen Werten und Praxen sowie eine neue Form der wissenschaftlichen Kommunikation. Dieser Situation kann als einmalige Chance und Möglichkeit für die notwendigen Neugestaltung wissenschaftlicher Kommunikation unter Berücksichtigung der Herausforderungen im aktuellen System betrachtet werden \cite{naeder_2010_open}. Offenheit in Wissenschaft und Forschung adressiert dabei den Kern der Produktion von Wissen und betrifft folglich nicht nur die Wissenschaft, sondern auch die Gesamtgesellschaft \cite{Mussell_2013}. Die reine Digitalisierung wissenschaftlicher Arbeitsprozesse und Anreicherung um die Möglichkeiten des digitalen Austauschs (Science 2.0) sowie die freie und offene Publikation finaler Forschungsergebnisse (Open Access) können durch in eine aktive, selbstbestimmte Neugestaltung der wissenschaftlichen Gemeinschaft zu einer umfassenden Öffnung der Kommunikation für die Gesamtgesellschaft (Open Science) im Sinne der Wissenschaft führen.

Die Abgrenzung von Open Access zu Open Science im Rahmen wissenschaftlicher Kommunikation wurde in dieser Arbeit auf Grundlage der Unterscheidung von "Zugang zu Wissen" (Open Access) und "Zugriff auf Wissen" (Open Science) durchgeführt. Das Konzept von Open Access und der damit verbundenen Verfügbarkeit von wissenschaftlichen Publikationen als Ergebnis von wissenschaftlicher Forschung im bestehenden wissenschaftlichen System betrifft demnach nur einen Teil der grundlegenden Neuordnung wissenschaftlicher Kommunikation. Open Science als Sammelbegriff adressiert auch weitere Teile und nicht nur die Digitalisierung der Aspekte rund um den Zugang zu fertigen wissenschaftlichen Publikationen, sondern fordert die Transformation und die Möglichkeit des umfassenden Zugriffs auf den gesamten wissenschaftlichen Prozess.

Für das wissenschaftliche System bedeuten diese Transformation auch, dass spätestens wenn die genannten Rahmenbedingungen des wissenschaftlichen Publizierens angepasst sind und Open Access weitreichend etabliert ist, es auch eine Neujustierung der Qualitätssicherungsmaßnahmen und Selektionsmechanismen im Rahmen der weitestgehenden Öffnung des wissenschaftlichen Erkenntnisprozesses (Open Science) braucht. Meint man es mit der Öffnung ernst und beachtet die Umkehr des Bring- zum Holprinzip, muss es Ziel sein, den Standard im wissenschaftlichen Kommunikationsprozess auf "offen" zu setzen und die Ausnahmen davon zu begründen.

Die Frage nach den Konsequenzen einer solch umfassenden Öffnung wissenschaftlicher Kommunikation ist dabei eng mit der Frage nach der zukünftigen Rolle der Universität und dem Hochschulwesen verbunden. Dabei ist die Ablösung des gedruckten Buchs "als Leitmedium der Universität" durch anderen Kommunikationsmitteln "lediglich als ein Epiphänomen einzustufen" \cite{Warnke_2012}, denn die Digitalisierung und zunehmende Verbreitung von Softwaresystemen, die bisherige Unkenntnis über ihre Relevanz und Bedienung durch die wissenschaftlichen Akteure im wissenschaftlichen System bedroht zunehmend auch die Freiheit der Wissenschaft im Kern. "Eine Systemanalyse (...) des akademischen Alltagslebens  könnte ein wenig Klarheit und damit vielleicht auch Rückgewinnung von Gestaltungsraum geben" \cite{Warnke_2012}. Das gilt obendrein für die Produktion und den Vertrieb von Wissen. Die Wandlungsprozesse rund um die Digitalisierung und die Öffnung wissenschaftlicher Kommunikation bieten eine einmalige Chance die Erstellung und Verbreitung des Wissens wieder stärker an die Universität zu binden und zu integrieren.

Läutete der Buchdruck die Moderne ein und legte den Grundstein für die wissenschaftliche Kommunikation, wie wir sie heute kennen, wird im Rahmen der Digitalisierung eine erneute Revolution des wissenschaftlichen Systems bevorstehen. Die unmittelbare und umfassende Bereitstellung der wissenschaftlichen Kommunikation im Rahmen der alltäglichen wissenschaftlichen Arbeit unter den Bedingungen größtmöglicher Transparenz stellt das wissenschaftlichen System aber innerhalb und außerhalb vor neue Herausforderungen. Das Aufbrechen der strikten Unterscheidung wissenschaftlicher Kommunikation in formelle und informelle, sowie interne und externe Kommunikation und die Konsequenzen für die Bewertung und Einordnung dieser stellt dabei nur eine von vielen Aufgaben dar.

Unter Berücksichtigung dieser Herausforderungen ist zu hoffen, dass ein neues Wachstum an Wissen durch den digitalen Wandel und die Öffnung wissenschaftlicher Kommunikation zu besseren Bedingungen für Schaffung neuen Wissens und die Bewahrung alten Wissens führen wird. Die Institutionen sowie Wissenschaftler und Wissenschaftlerinnen können im Rahmen des Prozesses um die Öffnung der Kommunikation die Möglichkeiten eines neuen Gestaltungsspielraums nutzen um ihre Rolle als Produzent, Archivar und bei der Verbreitung von Wissen zurückzugewinnen.

Sollten die Zeiten des "stürmischen Wachstums der Wissenschaft" \cite{K_lbel_2002} dennoch vorüber sein, lässt sich ein Grund dafür in dem Festhalten an der Geschlossenheit des wissenschaftlichen Kommunikationssystems finden. Die Verpflichtung zu Offenheit wissenschaftlicher Kommunikation darf jedoch nicht mit Enschränkungen der Unabhängigkeit von Wissenschaft einhergehen und den wissenschaftlichen Akteuren und der Universiät darf nicht "die Fähigkeit genommen werden, "Nein" zu sagen" \cite{suchen_Hornbostel_2006}. Das ist allerdings aus diversen Gründen eine Herausforderung: Erstens ist die wissenschaftliche Gemeinschaft gefragt den Wandel so zu gestalten, dass die Wissenschafts- und Publikationsfreiheit größtmöglich gewahrt wird und zweitens, dass der Wettbewerb um die Autorengebühren und Publikationsgeschwindigkeit nicht zu einer Bedrohung für die Zukunft der Wissenschaftskommunikation wird \cite{Beall_2012} \cite{Lossau_oa_2007}.

Aus der Forderung nach "unbeschränkten Zugang zur gesamten wissenschaftlichen Zeitschriftenliteratur" \cite{boai_2012} ist ein gesamtgesellschaftliches und umfassendes Modernisierungsvorhaben der Wissenschaft geworden, das neben den Aspekten der Zugänglichkeit zu Wissen und Wissenschaft eine Vielzahl an weiteren Unzulänglichkeiten adressiert, die den Fortbestand öffentlicher Forschung insgesamt beeinflussen \cite{brembs2015open}. Mit Blick auf die Umsetzung und Etablierung der Konzepte von Offenheit werden demnach, anders als in ursprünglichen Forderungen nach Offenheit im wissenschaftlichen Kommunikationssystem intendiert, auch Einschränkungen der akademischen Freiheit befürchtet \cite{hagner_2015_sache_buches}.

Die Entwicklungen im Rahmen der Digitalisierung und Forderungen nach Öffnung müssen von der wissenschaftlichen Community gestaltet werden, wenn sie nicht machtlos den Kräften ausgeliefert sein will, "die von außen auf die Publikation und Rezeption ihrer Schriften einwirken" \cite{Hirschi_2015_buch_oa}. Die Frage ist, ob sie entscheidet die Digitalisierung als Gefahr für den Fortbestand der Wissenschaft einfach nur zu negieren, als rein digitales Abbild der analogen Realität der wissenschaftlichen Kommunikation zu verstehen und das aktuelle System mit all seinen Vor- und Nachteilen zu bewahren oder ob sie es wagen eine zweite wissenschaftliche Revolution einzuläuten, die zu einer umfassenden Wissensverbreitung an die Gesamtgesellschaft und dadurch zu einer grundlegenden Veränderung des aktuellen wissenschaftlichen Systems führen könnte.

Die Forderung nach Offenheit von Wissenschaft und Forschung muss demnach nicht nur als "Strategie" gegen die unterschiedlichen Krisen im wissenschaftlichen Kommunikationssystem und für die Gestaltung des digitalen Wandels verstanden werden, die maßgeblich durch STM-Forscher gefördert werden und im Ergebnis zu einer weiterhin polarisierenden Abwehr- und Gegenreaktion führen würden sowie eine weitere Verwässerung und Fehlleitung der ursprünglichen Ansätze zur Folge hätte \cite{naeder_2010_open}. Sie muss auch als Ansatz für eine zukünftigen Sicherung der Freiheit von Wissenschaft verstanden werden. Sind wissenschaftlichen Aktivitäten nicht offen und zugänglich, steigt darüberhinaus die Gefahr, dass die öffentliche Unterstützung für die Wissenschaft erodiert und die Menschen Vertrauen in ein System verlieren, dass sie nicht unmittelbar verstehen können \cite{resnik_2005_ethics}.

Bisher haben sich die Wissenschaftler und Wissenschaftlerinnen für die Entwicklungen um die Forderung nach Öffnung und die Digitalisierung der wissenschaftlichen Kommunikation "erstaunlich wenig interessiert" \cite{hagner_2015_sache_buches}. Entzieht sich die wissenschaftliche Community diesen Auseinandersetzungen weiterhin, ist zu befürchten, dass langfristig die Freiheit von Wissenschaft und Forschung darunter leidet und zunehmend rein politische und wirtschaftliche Interessen darüber entscheiden \cite{Warnke_2012}, wie, wann, wo und wozu Wissenschaftler und Wissenschaftlerinnen in Zukunft kommunizieren werden.

Wenn Wissenschaftler und Wissenschaftlerinnen sich in der Auseinandersetzung mit der Forderung nach Öffnung von Kommunikation vornehmlich mit den Herausforderungen des Karrieredrangs und wirtschaftlichen Eigeninteressen befassen \cite{resnik_2005_ethics} besteht die Gefahr einer weiteren Verschließung wissenschaftlicher Kommunikation beziehungsweise eine (Aus-)Nutzung der Bewegung hin zur Öffnung durch den Drang zur Etablierung privatwirtschaftlicher Marktmechanismen zur Verwertung und Steuerung von Wissenschaft. Jegliche Abweichung, Einschränkung und Verwässerung von Openness begünstigt demnach die negative (Weiter-)Entwicklung zu einer rein privatwirtschaftlich oder aus politischen Interessen gesteuerten Wissenschaft. Die Vermutung liegt nahe, dass die Allokation und Nutzung von Ressourcen für den wissenschaftlichen Erkenntnisprozess ausschließlich auf Grundlage von Marktmeachnismen die Hetrogenität und auch die Effizienz der Produktion von neuem Wissen langfristig negativ beeinflussen wird.

Es gilt diesbezüglich weiterhin zu betonen, dass wir uns bei diesem Aushandlungsprozess erst am Anfang befinden. Mehr als 500 Jahre Buchdruck und 350 Jahre wissenschaftliches Journal stehen "nur" 25 Jahre Internet gegenüber. Vor 350 Jahren waren es Wissenschaftler, die sich zusammengetan haben, um eine neue Philosophie für die Förderung von Wissen zu etablieren und das erste wissenschaftliche Journal zu gründen. Sicher ist auch, dass es nicht allein bei der Forderung nach dem Zugang zu wissenschaftlichen Publikationen bleiben wird. Die Umsetzung von Open Access wird früher oder später auch in einer Forderung nach Öffnung des wissenschaftlichen Erkenntnisprozesses münden. Wie bei Open Access ist auch hier die wissenschaftliche Gemeinschaft gefragt, die Ausgestaltung aktiv und konstruktiv-kritisch zu beeinflussen und die Chancen für die Universtät nutzbar zu machen.

\section{Ausblick und Anknüpfungspunkte für weitere Forschungsbemühungen}

\begin{quote}
\textbf{We have reached a period in science somewhat simiular to that encountered by our colleagues of 300 years ago. Creative and inventive minds must now discover new methods for coping with the scienctific literature.}
\end{quote} \cite{porter_1964_scientific}

Die Transformation des wissenschaftlichen Kommunikationssystems von der Gutenberg-Galaxie in den Turing-Galaxis verlangt eine Neugestaltung der Rahmenbedingungen für die wissenschaftliche Kommunikation und eine Neudefinition der Rolle von allen Beteiligten in diesem System. Die neuen Möglichkeiten unterschiedlicher Formen der Darstellung wissenschaftlicher Informationen sollte dabei als neue Chance für eine aktive Verbesserung, Gestaltung und Modifikation wissenschaftlicher Kommunikation verstanden und genutzt werden. Diese Neugestaltung unter Wahrung der Freiheiten des wissenschaftlichen Systems funktioniert jedoch nur, wenn die Beteiligten ihre Rolle als aktive Gestalter und Gestalterinnen wahrnehmen. Sie müssen dabei in angemessener Form agieren und unbedingt vermeiden, dass der Öffnungs- und Digitalisierungsprozess das wissenschaftliche System technologisch oder ökonomisch rückständiger macht als das bisherige.

Als wichtige Felder für zukünftige Evaluationen sind die Themen Datenschutz und der Missbrauch von Forschung \cite{Fritsch_2015} zu nennen. Den Schutz der Privatsphäre gegen den immensen Wert von Open-Access-(Daten)nutzung auszugleichen und auszuhandeln stellt dabei eine wichtige zukünftige Herausforderung dar. Dabei sollten in einer Debatte nicht nur die nicht sofort überschaubaren Auswirkungen und Konsequenzen berücksichtigt, sondern auch die Vorteile gewissenhaft abgewägt werden. In diesem Zusammenhang benötigt es auch eines Aushandlungsprozesses zwischen der wissenschaftlichen Gemeinschaft, der Politik und der Gesellschaft.

Ein weiter Anknüpfungspunkt für Forschungsbemühungen ergibt sich aus der Umstellung des Publikationssystems vom Verkauf der Inhalte auf eine Voraberstattung der Kosten für die Publikation von wissenschaftlichen Erkenntnissen durch die öffentliche Hand. Im Rahmen der damit einhergehenden Transformation der Geschäftsmodelle von Verlagen hin zur freien Verfügbarkeit der veröffentlichten wissenschaftlichen Inhalte für die Gesamtgesellschaft, sollte untersucht werden, wie verhindert werden kann, dass die Erstattung von Autorengebühren (APCs) für die offene Publikationen nicht zu falschen Entwicklungen führen die einen kommerziellen Open-Access-Markt befeuern der zu einer ungerechtfertigten Verteuerung von APCs und zu einer weiteren Konzentration im Publikationsmarkt führen könnte. Die Fragestellungen sind, wie in dieser Arbeit dargestellt, eng mit den Publikationsentscheidungen der wisenschaftlichen Autoren und Autorinnen sowie mit der Erlangung von symbolischen, wissenschaftlichen Kapital verbunden.

Die mögliche Konkurrenz zwischen der wissenschaftlichen und medialen Kommunikation im Rahmen der Forderung nach mehr Öffnung stellt einen weiteren Ansatz für Untersuchungen dar. Als Konsequenz der Öffnung der gesamten wissenschaftlichen Kommunikation muss hinterfragt werden, ob und inwieweit das Wahrheitsmonopol der Wissenschaft durch das Aufmerksamkeitsmonopol der Medien im Rahmen der Möglichkeiten des offenen Zugangs zu Wissenschaft und des Zugriffs auf Wissen negativ beeinflusst werden könnte \cite{weingart_2005_wissenschaft}. Die Herausfordungen müssen dabei offensiv den Möglichkeiten und Chancen gegenübergestellt werden. Hierbei sind auch die jungsten Entwicklungen um das Thema Bürgerwissenschaft (Citizen Science) genauer zu berücksichtigen. Diese sollten dabei im Kontext anderer ähnlicher Entwicklungen, wie zum Beispiel den Bürger Journalismus (Citizen journalism) betrachtet werden.

Weitere Fragestellungen ergeben sich im Rahmen der neuen Möglichkeiten der Quantifizierung wissenschaflicher Tätigkeiten, zum Beispiel bei der revisionsgetrieben offenen wissenschaftlichen Publikation im digitalen Raum, sowie deren Konsequenzen auf die (Selbst-)Steuerung von Wissenschaft. Es wird vermutet, das die neuen Überwachungsmöglichkeiten der wissenschaftlichen Arbeit durch die Öffnung wissenschaftlicher Kommunikation eine zentrale Herausforderung für die Freiheit und den Datenschutz von Wissenschaftlern und Wissenschaftlerinnen darstellen können.
