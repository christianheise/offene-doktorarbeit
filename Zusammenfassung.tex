\chapter{Zusammenfassung und Ausblick}

In dieser Arbeit wurden die Herausforderungen der wissenschaftlichen Kommunikation im Rahmen der Digitalisierung und die Forderung nach Öffnung dieser umfassend diskutiert. Dabei wurden die einzelnen Definitionen von Open Access und Open Science, sowie Implikationen und Rahmenbedingungnen näher betrachtet, analysiert und spezifiziert. Ungeachtet der viefältigen Kritik am aktuellen Kommunikationssystem, scheint dieses jedoch auch nach zwei Jahrzehnten noch immer weitestgehend stabil.

Nach einer Einführung in den Themenbereich der wissenschaftlichen Kommunikation und des wissenschafltichen Publizierens wurde mit anhand aktueller Literatur zum Thema die Debatte um die Forderung nach Öffnung der wissenschaftlichen Kommunikation dargestellt. Dabei wurden Treiber und Bremser für die Entwicklungen identifiziert 

Diese Treiber und Bremser wurden darauf hin mithilfe einer repräsentativen Umfrage unter Wissenschaftlern genauer abgefragt. Dabei wurde analysiert, inwiefern ein Verständnis von Offenheit im Rahmen der wissenschaftlichen Kommunikation vorherrscht und inwieweit Offenheit in den unterschiedlichen wissenschaftlichen Fachdisziplinen verbreitet ist und praktiziert wird.

Die Ergebnisse der Studie können insofern als repräsentati geltenv, als sie auf einer in Bezug auf eine relativ große Stichprobe beruht. In der Auswertung wurde vor allem deutlich, dass die Verbreitung offener Kommunikationsverfahren wesentlich vom Fachgebiet abhängt. Darüber hinaus konnten in der Studie eine große Diskrepanz zwischen dem Interesse nach und dem Verständnis für Offeneheit und der tatsächlich praktizierten offenen Arbeitsweise festgestellt werden.

Sollten die Zeiten des "stürmischen Wachstums der Wissenschaft endgültig vorüber" \cite{K_lbel_2002} sein, lässt sich vielleicht ein Grund dafür in der Geschlossenheit des wissenschaftlichen Kommunikationssystems finden. Aber wird ein neuer Wachstum durch den digitalen Wandel im Rahmen und die Öffnung wissenschaftlicher Kommunikation wirklich zu besseren Bedingungen für Schaffung neuen Wissens und die Bewahrung alten Wissens führen?