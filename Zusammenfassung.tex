\chapter{Zusammenfassung und Ausblick}

\begin{quote}
\textbf{"Denn ich habs umsonst empfangen, umsonst hab ich's gegeben und begehre auch dafür nichts"}
\end{quote} \cite{luther_1876}

In dieser Arbeit wurden die Herausforderungen der formellen wissenschaftlichen Kommunikation im Rahmen der Digitalisierung und die Forderung nach Öffnung dieser umfassend analysiert und diskutiert. Dabei wurden die einzelnen Definitionen von Open Access und Open Science, sowie Implikationen und Rahmenbedingungen näher betrachtet, analysiert und spezifiziert.

Nach einer Einführung in den Themenbereich der wissenschaftlichen Kommunikation und des wissenschaftlichen Publizierens wurde anhand aktueller Literatur zum Thema die Debatte um die Forderung nach Öffnung der wissenschaftlichen Kommunikation dargestellt. Dabei wurden Treiber und Bremser für die Entwicklungen identifiziert.

Diese Treiber und Bremser wurden darauf hin mithilfe einer repräsentativen Umfrage unter Wissenschaftlern genauer abgefragt. Dabei wurde analysiert, inwiefern ein Verständnis von Offenheit im Rahmen der wissenschaftlichen Kommunikation vorherrscht und inwieweit Offenheit in den unterschiedlichen wissenschaftlichen Fachdisziplinen verbreitet ist und praktiziert wird.

Die Ergebnisse der Studie können insofern als repräsentativ gelten, als sie auf einer relativ großen Stichprobe beruhen. In der Auswertung wurde vor allem deutlich, dass die Verbreitung offener wissenschaftlicher Kommunikationsverfahren wesentlich vom Fachgebiet abhängt. Ungeachtet der vielfältigen Kritik am aktuellen Publikations- und Kommunikationssystem scheint dieses jedoch auch nach zwei Jahrzehnten noch immer weitestgehend stabil. In der Studie wurde ausserdem eine große Diskrepanz zwischen dem Interesse nach und dem Verständnis für Offenheit und der tatsächlich praktizierten offenen Kommunikations- und Arbeitsweise festgestellt. Das deckt sich auch mit bisherigen Analysen \cite{Bartling_2013} \cite{hagner_2015_sache_buches}.

Rainer Kuhlen definierte 2002 drei Szenarien wie der Zugriff auf Wissen in mittlerer Perspektive organisiert sein wird \cite{Kuhlen_2002_universalaccess}:
\begin{enumerate}
\item Elektronische Informationen sind frei zugänglich und die Konzepte der individuellen Autorenschaft und des geistigen Eigentums werden zu Relikten aus bürgerlichen Vorinformationsgesellschaften
\item Wissen und Informationen sind kontrolliert und dem Markt ohne politische Gegensteuerung überlassen: Die Kommerzialisierung und Zonierung von Wissen und Information wird umfassend sein und den Alltag bestimmen.
\item Wissen und Information werden über koexistente oder Paralleluniversen organisiert: Das Wissens als Produkt ist frei, öffentlich zugänglich und nutzbar. Es bleibt aber genug Spielraum bei der Adaption, Beratung, Veredlung oder anderen Mehrwertleistungen einer kommerziellen Informations- und Wissenswirtschaft
\end{enumerate}

Ein Ergebnis dieser Arbeit ist, dass es maßgeblich von den publizierenden Wissenschaftlern selbst abhängen wird eine treibende Kraft und aktive Position bei der Gestaltung des Wandels hin zur Öffnung der wissenschaftlichen Kommunikation zu übernehmen. Das ist unter anderem aus zwei Gründen eine Herausforderung: Erstens sind sie gefragt den Wandel so zu gestalten, dass die Wissenschafts- und Publikationsfreiheit größtmöglich gewahrt wird und zweitens, dass der Wettbewerb um die Autorengebühren und Publikationsgeschwindigkeit nicht zu einer Bedrohung für die Zukunft der Wissenschaftskommunikation wird \cite{Beall_2012}. Bisher haben sich die Wissenschaftler und Wissenschaftlerinnen, zum Beispiel der Geistes- und Kulturwissenschaft sowie der Wissenschaftsforschung "erstaunlich wenig für das Thema interessiert" \cite{hagner_2015_sache_buches}. Die Ergebnisse der Befragung nähren diese Befürchtung, dass sie auch in naher Zukunft keine hervogehobene Rolle  bei der Gestaltung der Transformation einnehmen werden.

Christopher Kelty sieht für das Desinteresse an Offenheit im wissenschaftlichen Kommunikationssystem vorallem zwei Gründe: Erstens, ist Open Access ziehmlich langweilig und zweitens ist das Thema bei genauerer Betrachtung sehr komplex. Ergänzend zu Kelthys Einschätzung kann als Ergebnis dieser Arbeit aber auch noch ein dritter Grund genannt werden: Durch die Beschäftigung mit dem wissenschaftlichen Publikationsmarkt, würden "Praktiken auf dem Spiel stehen, die dem Tun vieler Geisteswissenschafter (...)" und auch Wissenschaftler anderer Fächer, "Sinn und Legitimität zu verleihen scheinen" \cite{Hirschi_2015_buch_oa}.

Ein weiteres Hemnisse für die Umsetzung der Konzepte um die offen praktizierte Wissenschaft sind die fehlenden technischen Möglichkeiten. Die zur verfügungstehenden Plattformen und Applikationen sind noch nicht ausgereift und bequem genug um im Alltag offene Wissenschaft zu praktizieren. Die offene Erstellung dieser Arbeit wäre ohne programiertechnische Vorkenntnisse schwer bis nicht möglich gewesen. So muss der "Open Scientist" entweder in der Lage sein, selbst programieren zu können oder es müssen die Rahmenbedinungen geschaffen werden, dass er ohne solche Kenntnisse den gesamten wissenschaftlichen Prozess offen und transparent abbilden kann.

Im Gegensatz zum Träger- und Speichermedium Papier wird wissenschaftlichen Wissen im Rahmen der Digitalisierung als Code gespeichert. Will man die Rohform der Information lesen, interpretieren, oder verändern muss man Code lesen können. Wissenschaftler arbeiten mit unterschiedlichen Datensätzen \cite{patlak_2010_open}, die Erhofften Vorteile vom Teilen der Daten erfüllen sich demnach auch hier nur dann, wenn für die Migration das nötige Know-How vorhanden ist.

Eine weitere Herausforderung besteht darin, dass viele Förderorganisationen im Rahmen von Forschungsföderung ausschließlich klassische Publikationen abzielen und nur langsam die Software-Entwicklung unterstützen \cite{hey_2015_open}. Förderorganisationen müssen diese Verantwortung ernst nehmen, indem sie die zusätzlichen Ressourcen, die für die Schaffung der strukturellen Grundlagen die mit der Öffnung von Wissenschaft und Forschung verbunden sind, zur Verfügung stellen \cite{mennes_2013_making_os} \cite{patlak_2010_open}. Um diese Bemühungen voranzubringen, müssen sich  Förderorganisationen auch entscheiden, ob die Umsetzung der gemeinsamen Nutzung von Daten durch Incentivierung Mandate fördern \cite{mennes_2013_making_os}.


Sollten die Zeiten des "stürmischen Wachstums der Wissenschaft endgültig vorüber" sein \cite{K_lbel_2002}, lässt sich ein Grund dafür in der Geschlossenheit des wissenschaftlichen Kommunikationssystems finden. Mit der Verpflichtung auf Offenheit darf aber der Wissenschaft nicht die Unabhängigkeit und auch nicht "die Fähigkeit genommen werden, "Nein" zu sagen" \cite{suchen_Hornbostel_2006}. Unter Berücksichtigung dieser Faktoren ist zu vermuten, dass der neue Wachstum an Wissen durch den digitalen Wandel im Rahmen und die Öffnung wissenschaftlicher Kommunikation wirklich zu besseren Bedingungen für Schaffung neuen Wissens und die Bewahrung alten Wissens führen wird.

Läutete der Buchdruck die Moderne ein und legte den Grundstein für die wissenschaftliche Kommunikation wie wir sie heute kennen, könnte im Rahmen der Digitalisierung eine erneute Revolution des wissenschaftlichen Systems bevorstehen. Die unmittelbare und umfassende Bereitstellung der wissenschaftlichen Kommunikation im Rahmen der alltäglichen wissenschaftlichen Arbeit unter den Bedingungen von größtmöglicher, stellt das wissenschaftlichen System aber innerhalb und außerhalb vor neue Herausforderungen.

---- TODO: Diskussion um unterschiedle Konzepte/Verstaendnis von positive und negative liberty \cite{kelty_2014_freedom} auf Openess in der wissenschaftlichen Kommunikation ueberttragen ----


---- TODO: Wir werden die Übermittlung von Wissen bei der wissenschaftlichen Kommunikation nur dann verstehen können, wenn die Übermittlungswege und Formen transparent und offen gestaltet sind. \cite{davis_2011_open} ----

Diese Entwicklung kann und muss von der wissenschaftlichen Community gestaltet werden, wenn sie nicht machtlos den Kräften ausgeliefert sein will, "die von außen auf die Publikation und Rezeption ihrer Schriften einwirken" \cite{Hirschi_2015_buch_oa}. Die Frage ist, ob sie entscheidet die Digitalisierung als Gefahr für den Fortbestand der Wissenschaft einfach nur zu negieren, als rein digitales Abbild der analogen Realität der wissenschaftlichen Kommunikation zu verstehen und das aktuelle System mit all seinen Vor- und Nachteilen zu bewahren oder wagen es die Wissenschaftler und Wissenschaftlerinnen eine zweite wissenschaftliche Revolution einzuläuten, die zu einer umfassenden Wissensverbreitung an die Gesamtgesellschaft und zu einer grundlegenden Veränderung des aktuellen wissenschaftlichen Systems führen könnte?

Entzieht sich die wissenschaftliche Community dieser Auseinandersetzung, ist davon auszugehen, dass die Freiheit von Wissenschaft und Forschung darunter leiden und rein politische und wirtschaftliche Interessen darüber entscheiden, wie, wann, wo und wozu Wissenschaftler und Wissenschaftlerinnen in Zukunft arbeiten werden. Wissenschaft könnte sehr leicht wieder ein sehr geschlossenes System werden, wenn Wissenschaftler nicht mit allen Facetten der Offenheit befassen und
die Bedrohungen des Karrieredrangs und wirtschaftliche Eigeninteressen zu stark werden \cite{resnik_2005_ethics}.

Die Öffnung wissenschaftlicher Kommunikation kann als "vielleicht kostbarste Geschenk des Internet an die Wissensgesellschaft" betrachtet werden, wenn sie nicht ausschließlich durch die "ökonomischen Interessen des Informationskapitalismus" \cite{hagner_2015_sache_buches} gelenkt wird, sondern sich (weiterhin) maßgeblich auf die Aufgabe konzentriert dem gesamtgesellschaftlichen Auftrag des Wissenschaftssystems gerecht zu werden. Es bleibt auch eine Notwendigkeit, dass jeder Autor und jede Autorin selbst, aber unter Kenntnisstand der Konsequenzen für die Zugänglichkeit durch andere, entscheidet ob und wie seine oder ihre Forschungsergebnisse erscheinen.

25 Jahre nach den ersten Versuchen den Open Access mit Hilfe digitaler Netze umzusetzen gibt es kaum noch Zweifel, dass sich diese Entwicklung weltweit durchsetzen wird. Die Zweifel bestehen allerdings bei der Ausgestaltung der Herausforderungen und "bei der Frage wie sich Vor- und Nachteile zueinander verhalten" \cite{hagner_2015_sache_buches}. "Eine Weise Politikgestaltung in diesem sensiblen Bereich muss besonders acht auf die komplexen Geschichte der Organisationen der öffentlichen Wissenschaft geben und die potenzielle Fragilität der eigentümlichen institutionellen Matrix respektieren in der sich die modernen Forschung entwickelt hat und aufgeblüht."\cite{david1998_common}. Darüber hinaus müssen die Rahmenbedingungn ausgehandelt werden unter denen die Digitalisierung der wissenschaftlichen Kommunikation und die Öffnung von Wissenschaft und Forschung stattfindet \cite{mennes_2013_making_os}.

Die Forderung nach Offenheit von Wissenschaft und Forschung darf nicht nur als "Strategie" gegen die unterschiedlichen Krisen im wissenschaftlichen Kommunikationssystem verstanden werden, die maßgeblich STM-Forschern gefördert wird und im Ergebnis zu einer weiterhin polarisierende Abwehr- und Gegenreaktion führen würde, die zu einer weiteren Verwässerung und Fehlleitung der ursprünglichen Ansätze zur Folge hätte \cite{naeder_2010_open}. Sind wissenschaftlichen Aktivitäten nicht offen und zugänglich, steigt die Gefahr, dass die öffentliche Unterstützung für die Wissenschaft erodiert und die Menschen vertrauen in sie verlieren \cite{resnik_2005_ethics}.

Sicher ist demnach auch, dass es nicht allein bei der Forderung nach dem Zugang zu wissenschaftlichen Publikationen bleiben wird. Die Umsetzung von Open Access wird früher oder später auch in einer weiteren Öffnung des wissenschaftlichen Erkenntnisprozesses münden. Wie bei Open Access ist auch hier die wissenschaftliche Gemeinschaft gefragt die Ausgestaltung aktiv und konstruktiv-kritisch zu beeinflussen.

Wir befinden uns bei diesem Aushandlungsprozess erst am Anfang: Mehr als 500 Jahre Buchdruck stehen gerade mal 25 Jahre Internet gegenüber. Vor genau 350 Jahren waren es Wissenschaftler, die sich zusammengetan haben um eine "Philosophie für die Förderung von Wissen" zu etablieren. Die neuen Möglichkeiten unterschiedlicher Formen der Darstellung von wissenschaftlichen Informationen kann als neue Chance für eine aktive Verbesserung, Gestaltung und Modifikation wissenschaftlicher Kommunikation verstanden und genutzt werden.

Ein wichtiges Feld zukünftiger Evaluationen wird das Thema Datenschutz, um den Schutz der Privatsphäre gegen die immense Wert von Open-Access-Datennutzung auszugleichen und auszuhandeln. Dafür benötigt es einen Aushandlungsprozess zwischen der wissenschaftlichen Gemeinschaft, der Politik und der Gesellschaft.

---- TODO: Aufgreifen "We have reached a period in science somewhat simiular to that encountered by our colleagues of 300 years ago. Creative and inventive minds must now discover new methods for coping with the scienctific literature" \cite{porter_1964_scientific}; Dispositiv; "Wir werden der technologischen Zukunft nicht dadurch Herr, dass wir uns von ihr abwenden. Die Verantwortung besteht darin, ihre Logik zu verstehen, um möglichst viele ihrer Auswirkungen vorwegzunehmen. Ein Diskurs über die Zwecke und Werte, der sich nicht auf einen präzisen Zustand der zur Verfügung stehenden Mittel stützt, ist ein leerer Diskurs. Aber ein Diskurs über die Innovation, der diese nicht im Lichte der Erinnerung genau untersucht, ist ein Diskurs." \cite{naeder_2010_open}; Weitere Umstrukturierung der ökonomischen und juristischen Rahmenbedingungen des wissenschaftlichen Publizierens sind nötig \cite{naeder_2010_open};  Neugestaltung der Qualitätssicherungsmaßnahmen und Selektionsmechanismen notwendig; Gefahr Predatory Journals \cite{Beall_2012} ----

---- TODO: weiter ausarbeiten und überarbeiten, ist bisher zu pathetisch und moegliche weitere Forschungsfelder aufzeigen ----
