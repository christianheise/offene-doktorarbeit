\section{Methode der Onlinebefragung}
Um der Entwicklung der Öffnungvon Wissenschaft sowie deren Treiber und Bremser nachgehen zu können, soll eine Onlinebefragung bei den Stakeholdern des akademischen Publizierens explorativ durchgeführt werden, nicht zuletzt weil theoretische Vorannahmen bestehen.

\subsection{Forschungsfragen}
Folgende Forschungsfragen sollen bei der Befragung genauer adressiert werden:
\begin{itemize}
\item Wie verändert die Digitalisierung, wie wir auf wissenschaftliche Daten und Informationen zugreifen?
\item In welchem Umfang herrst unter den Wissenschaftlern_innen Wissen über die Öffnung von Wissenschaft vor? 
\item Welche Faktoren und Argumente begünstigen die Öffnung von Wissenschaft in einer wissenschaftlichen Disziplin? 
\item Welche Faktoren und Argumente sprechen gegen die Öffnung von Wissenschaft in einer wissenschaftlichen Disziplin? 
\item Wie wird der geschätzte Aufwand für die Öffnung von Wissenschaft in einer wissenschaftlichen Disziplin eingeschätzt?
\item Welche weiteren extrinsischen Faktoren unterstützen die Verbreitung von Openness in Wissenschaft und Forschung? 
\item In welchem Umfang wird bereits heute im wissenschaftlichem Umfeld offen kommuniziert?
\item Welche Veränderungen im Zusammenhang mit dem Publizieren sind seit 2007 zu erkennen?
\end{itemize}

\subsection{Erhebungsmethode und Messinstrumente}
\subsection{Technische Realisation des Fragebogens}