Im Prozess meiner Auseinandersetzung mit dem Wandel digitaler Kulturen und mit der Offenheit in der wissenschaftlichen Kommunikation entstand relativ schnell die Frage: Wie offen und transparent kann eine wissenschaftliche Arbeit erstellt werden? Ich wollte wissen, ob es möglich sein würde, alle Informationen zu einem solchen Vorhaben und den Arbeitsprozess selbst möglichst umfassend direkt und unmittelbar während der Erstellung für jeden, jederzeit frei zugänglich im Internet unter einer offenen und freien Lizenz (CC-BY-SA) einsehbar und verfolgbar zu machen.

Von Beginn an stellten die Beantwortung dieser Frage, der Selbstversuch sowie der damit einhergehende Anspruch an Offenheit mehrere Herausforderungen dar. Zum einen an mich selbst, zum anderen aber auch an die Institution Universität und den rechtlichen Anforderungen der aktuell geltenden Prüfungsordnung. Diese beruht noch immer auf der Veröffentlichungsform eines individuell und während des Erstellungsprozesses nicht öffentlich einsehbaren abgeschlossenen Buchs. Eine Doktorarbeit entstand so bisher individuell am Schreibtisch des Promovenden und ohne die Möglichkeit der öffentlichen Einsicht in den Erstellungsprozess sowie in damit verbundene Dokumente und Daten. Diese Einsichtnahme in die gedruckte Arbeit war bisher nur nach Abschluss des gesamten Promotionsverfahrens vorgesehen und möglich. Daten und weitere Dokumente sind nur selten überhaupt einsehbar.

Mit einem Schreiben an die Promotionskommission versuchte ich dieser rechtlichen Unsicherheit zu begegnen. Ich erklärte mein Vorhaben, erfragte die Bedingungen für diese Art der offenen Erstellung der Arbeit und legte eine mögliche Begründung für die Vereinbarkeit mit der Promotionsordnung dar. Nach der fast einjährigen rechtlichen Prüfung durch die Kommission und das Justiziariat der Universität entsprach die Promotionskommission am 12. Dezember 2013 mehrheitlich dem Gesuch, die Arbeit unter den genannten Bedingungen "offen" verfassen zu dürfen. Dabei hat die Kommission hier jedoch nur ein Meinungsbild gegeben, denn die finale Annahme oder Ablehnung einer Dissertation geschieht erst, wenn sie eingereicht wird. Somit war nicht ausgeschlossen, dass die Kommission die Arbeit bei Einreichung doch noch ablehnt.

Nach drei Jahren entstand unter diesen Voraussetzungen und den daraus resultierenden technischen, rechtlichen und strukturellen Herausforderungen bei der Erstellung ein aus wissenschaftlicher Perspektive zweifach reflexives Projekt: Es handelt sich dabei um die wissenschaftliche Analyse von Wissenschaft und auch um ein Vorhaben, dass sich in seiner experimentellen und offenen Herstellung selbst noch einmal spiegelt.

Die Forschung fand dabei aus drei Perspektiven statt: Erstens aus intellektuell kontemplativer Perspektive durch die Abarbeitung der definitorischen Fragen sowie die historische Einordnung und die Erarbeitung theoretischer Motive und Beweggründe für Wissenschaftler und Wissenschaftlerinnen für wissenschaftliches Kommunizieren. Zweitens aus der Perspektive des Beobachters, bei der die aktuellen Rahmenbedingungen wissenschaftlicher Kommunikation durch empirische Forschung erhoben, historisch verglichen und mit den theoretisch erarbeiteten Annahmen abgeglichen wurden. Aus der dritten Perspektive, der des aktiven Teilnehmers, kann dieses Vorhaben als praktischer Beitrag zu der Debatte zur Neugestaltung wissenschaftlicher Kommunikation betrachtet werden.

Mit diesem Vorspann und rund ein Jahr nach Fertigstellung der eigentlichen Dissertation möchte ich den Selbstversuch, ein Promotionsvorhaben so offen wie möglich zu gestalten, zum Abschluss führen, denn ein Promotionsvorhaben endet nicht mit der Einreichung der Arbeit, sondern erst nach der Begutachtung, der Verteidigung und der Veröffentlichung dieser.

\section{Die Gutachten und die Benotung}

Nach der Einreichung der Arbeit im Juni 2016 wurden die Gutachten durch die drei Gutachter erstellt, die Note ermittelt und ein Termin für die Disputation festgelegt. In der Zwischenzeit hatte der praktische Teil, das offene Erstellungsverfahren, das mediale Interesse verstärkt. Im Interview für die Helmholtz Gemeinschaft, für das Merton Magazin der Stiftung Mercator sowie für netzpolitik.org, einem der reichweitenstärksten deutschsprachigen Blog zu digitalen Freiheitsrechten und netzpolitischen Themen, habe ich Fragen zu meiner offenen Promotion beantwortet.

Ende November 2016 trafen alle Gutachten ein und die erstellte Arbeit wurde zusammen mit den Gutachten vom 25. November 2016 bis zum 23. Dezember 2016 hochschulöffentlich an der Leuphana Universität ausgelegt. Welche Herausforderung diese neue Arbeitsweise an die rechtlichen Rahmenbedingungen stellt, zeigt sich allein daran, dass es bislang in den Regularien nicht vorgesehen ist, die Gutachten zu einer Promotion zu veröffentlichen. Auf die Anfrage an die Promotionskommission, ob die Gutachten zu meiner Doktorarbeit online zum Download zur Verfügung gestellt werden dürfen, wurde mit Verweis auf die Promotionsordnung  eine solche Veröffentlichung nicht gestattet.

Inhaltlich bewerteten die beiden Gutachter und die Gutachterin die Arbeit durchweg positiv und bewegten sie alle in einem ähnlichen Rahmen der Benotung. Die Kritik beschränkte sich vornehmlich auf strukturelle Schwierigkeiten in der Arbeit. So wurde angemerkt, dass es nicht immer gelungen sei, die Komplexität des Themas zu bändigen und dass es im Verlauf der Arbeit oft zu sprachlichen und inhaltlichen Wiederholungen kommt. Diese wurden in dieser finalen Druckversion weitestgehend korregiert.

\section{Die Verteidigung}

Nach Auslage der Gutachten Ende 2016 und der Annahme der Dissertation durch die Promotionskommission Anfang 2017 erfolgte im Februar 2017 die mündliche Verteidigung (Disputation). Die Disputation wird als hochschulöffentliches, wissenschaftliches Streitgespräch beziehungsweise als mündliche Doktorprüfung abgehalten, bei der Fragen durch Gutachter und Gutachterinnen sowie die erarbeiteten Argumente abgewägt und diskutiert werden.

In der Vorbereitung zur Disputation hatte ich die Gelegenheit den gesamten Prozess mit einer zeitlichen Distanz erneut zu reflektieren und daraus entstandene zusätzliche Aspekte in meinen Vortrag einzubauen. In der Verteidigung habe ich mein Vorhaben dargestellt und herausgestellt, dass es bisher kaum Versuche gab, die theoretischen Erkenntnisse praktisch-experimentell zu überprüfen und die Diskussion und kritischen Ausblicke bisher wenig differenziert sind, wenn es um weitere Entwicklungen bei Anwendung, Neuinterpretation und Verarbeitung von Forschungsergebnissen geht. Ich habe meine Hypothesen, Herangehensweise, die Methode und die erarbeiteten Ergebnisse dargestellt und erläutert. In der darauffolgenden Diskussion wurden die Punkte der Gutachten besprochen und über Möglichkeiten der Veränderungen bei wissenschaftlichen Qualifikationsarbeiten diskutiert.

\section{Die Veröffentlichung}

Seit der Einreichung der Arbeit im Juni 2016 hat sich viel verändert. Technologisch haben sich die Möglichkeiten zur offenen Publikation wissenschaftlicher Arbeiten weiterentwickelt. Politisch sind klare Entwicklungen für eine Novellierung der rechtlichen Rahmenbedingungen wissenschaftlicher Kommunikation erkennbar und gesamtgesellschaftlich werden Forderungen nach Öffnung von Wissenschaft und der wissenschaftlichen Organisationen lauter. Laut der Ausgabe 16/2017 der Wochenzeitung DIE ZEIT sind die "Chancen für einen Aufbruch (...) wohl noch nie so gut wie heute". Und trotzdem bleibt es primär bei theoretischen Forderungen, Demonstrationen oder zaghaften politischen Reaktionen.

Im Gegensatz zu der rein theoretischen Auseinandersetzung mit dem Wandel digitaler Kulturen und mit der Offenheit in der wissenschaftlichen Kommunikation hoffe ich mit dieser Arbeit einen konkreten praktischen Beitrag zur Debatte um Offenheit in Wissenschaft und Forschung geleistet zu haben. Das wohl eindeutigste Ergebnis meines Vorhabens ist, dass mehr Offenheit in Wissenschaft und Forschung sowie bei wissenschaftlichen Qualifikationsarbeiten schon heute grundsätzlich möglich und umsetzbar sind.

Bevor sich wissenschaftliche Institutionen öffnen und offene wissenschaftliche Kommunikation im Forschungsalltag eine ernstzunehmende Rolle spielen wird, braucht es mehr konkrete Experimente zu Offenheit und den Grenzen der Offenheit in der wissenschaftlichen Gemeinschaft. Wissenschaftlerinnen und Wissenschaftler sollten nicht nur Forderungen stellen, sondern in Experimenten versuchen, konkrete Veränderungen zu erzielen. Wir befinden uns bei der Erforschung von Offenheit noch immer erst am Anfang.

\section{Die Danksagung: Die letzten Worte eines Unpromovierten}

Ich versichere, dass ich die eingereichte Dissertation selbstständig und ohne unerlaubte Hilfsmittel verfasst habe, dennoch wäre sie ohne die Unterstützung vieler Menschen niemals zustande gekommen. Um dem Rechnung zu tragen, möchte ich mich an dieser Stelle des Vorspanns bei allen Personen bedanken, die mich über die Jahre der Erstellung dieser Arbeit vielfältig unterstützt haben.

An erster Stelle gilt mein Dank meinem Doktorvater und Erstgutachter Herr Prof. Dr. Martin Warnke der mir neben seiner wissenschaftlichen und methodischen Unterstützung während der gesamten Bearbeitungsphase meiner Dissertation auch viel Freiraum bei der Ausarbeitung ließ. Dabei muss auch hervorgehoben werden, dass ohne seinen Einsatz das experimentelle Vorgehen in dieser Arbeit nicht möglich gewesen wäre.

Außerdem danke ich Prof. Dr. Götz Bachmann - der auch als Zweitgutachter fungierte - und Dr. Armin Beverungen die mich in zahlreichen und unermüdlichen Gesprächen in den unterschiedlichsten Lebenslagen mit vielen Ratschläge und zielführenden Anmerkungen während der gesamten Erstellung der Arbeit stets begleitet und vielseitig unterstützt haben. Dieser fachliche und persönliche Beistand bewahrte mich vor so manchen Fehltritt. Für die Übernahme des Drittgutachtens geht mein Dank an Prof. Dr. Isabella Peters.

Ein besonderer Dank gilt auch meiner Mutter, die mich auf meinem Weg durch das Promotionsstudium und bei der Erstellung dieser Arbeit immer sehr motivierend begleitet hat und mich mit unzähligen hilfreichen Ratschlägen und Korrekturen unterstützt hat.

Der wichtigste Dank richtet sich an meine Frau Rhea Eckwolf, ohne deren Unterstützung diese Arbeit niemals beended worde wäre. Sie hat mich mit ihrem schier grenzenlosen Rückhalt, ihrem unermüdlichen Verständnis und ihrer Liebe während der gesamten Erarbeitung dieser Dissertation unterstützt. Diese Arbeit möchte ich deshalb auch unserem ersten gemeinsamen Kind widmen.

Mit dieser Danksagung beende ich den Selbstversuch, ein Promotionsvorhaben so offen wie möglich zu gestalten und danke allen Beiträge zu dieser Arbeit.
