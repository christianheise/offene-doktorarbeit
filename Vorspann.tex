Als ich begann die vorliegende Arbeit zu schreiben, stand für mich neben der inhaltlichen Auseinandersetzung mit Offenheit im wissenschaftlichen Kommunikationsprozess sowie der Herleitung von Offenheit in Wissenschaft und Forschung die Frage im Vordergrund, ob und wie offen und transparent eine Doktorarbeit erstellt werden kann. Ich wollte wissen, ob es möglich sein würde, alle Informationen zu einem solchen Vorhaben und die Arbeit direkt und unmittelbar während der Erstellung für jeden, jederzeit frei zugänglich im Internet unter einer offenen und freien Lizenz (CC-BY-SA) einsehbar und verfolgbar zu machen.

Schon von Beginn an, stellte dieser Selbstversuch und der Anspruch an Offenheit bei der Erstellung eine große Herausforderung dar. Zum einen an mich selbst, aber auch an die Institution Universität. Um den Anforderungen der aktuell geltenden Prüfungsordnung der Leuphana Universität zu entsprechen, wurden 2013 in einem Schreiben an die Promotionskommission die Bedingungen für die offene Erstellung der Arbeit abgefragt und eine mögliche Begründung für die Vereinbarkeit mit der Promotionsordnung dargelegt. Nach einer rechtlichen Prüfung durch das Justiziariat der Universität entsprach die Promotionskommission am 12. Dezember 2013 mehrheitlich dem Gesuch, die Arbeit "offen" verfassen zu dürfen.

Da die Aufmerksamkeit das offene Verfahren und die offene Schreibweise der Doktorarbeit in meinem direkten Umfeld groß war, war ich mir nicht immer sicher den richtigen Weg gewählt zu haben. Zum einen war ich mir selbst unsicher, ob dieses Vorhaben technisch nach meinen eigenen Ansprüchen und der vorausgesetzen Form umsetzbar ist oder ob ich der Herausforderung der Promotion sowie den Anspruch an die offenen Kommunikation und der eventuellen Mehrarbeit standhaft bleiben würde.

Dennoch entstand im Folgenden unter diesen Voraussetzungen und den daraus resultierenden technischen, rechtlichen und strukturellen Herausforderungen bei der Erstellung ein aus wissenschaftlicher Perspektive zweifach reflexives Projekt. Es handelt sich dabei nicht nur um die wissenschaftliche Analyse von Wissenschaft, sondern auch um ein Vorhaben, dass sich in seiner experimentellen und offenen Herstellung selbst noch einmal spiegelt.

Diese drei Jährige Erforschung fand aus drei Perspektiven statt. Erstens, aus intellektuell kontemplativen Perspektive. Im Vordergrund stand dabei die Abarbeitung der definitorischen Fragen sowie die historische Einordnung und die Erarbeitung theoretischer Motive und Beweggründe für Wissenschaftler und Wissenschaftlerinnen für wissenschaftliches Kommunizieren. Zweitens, aus der Perspektive des Beobachters, bei der die aktuellen Rahmenbedingungne wissenschaftlicher Kommunikation durch empirischen Forschung erhoben, historisch verglichen und mit den theoretisch erarbeiteten Annahmen abgeglichen wurden. Aus der dritten Perspektive, der des aktiven Teilnehmers kann dieses Vorhaben als praktischer Beitrag zur der Debatte zur Neugestalltung wissenschaftlicher Kommunikation betrachtet werden.

Mit diesem Vorspann und fast ein Jahr nach Fertigstellung der eigentlichen Dissertation, möchte ich diesen Selbstversuch ein Promotionsvorhaben so offen wie möglich zu gestalten zu ende bringen, denn ein Promotionsvorhaben endet nicht mit der Einreichung der Arbeit: Es fehlen noch die Begutachtung, die Verteidigung und die Veröffentlichung.

\section{Die Gutachten und die Benotung}

Nach der Einreichung der Arbeit im Juni 2016 folgte eine lange Zeit des Wartens. In den Folgemonaten wurden nun Gutachten durch die drei Gutachter erstellt, die Note ermittelt und ein Termin für die Disputation festgelegt. In der Zwischenzeit hatte der praktische Teil, das offene Verfahren, auch das mediale Interesse verstärkt. In einem Interview für die Helmholtz Gemeinschaft, für das Merton Magazin der Stiftung Mercator sowie für netzpolitik.org, dem reichweitenstärkstem deutschsprachigen Blog zu digitalen Freiheitsrechten und netzpolitischen Themen habe ich Fragen zu meiner offenen Promotion beantwortet.

Ende November 2016 traffen alle drei Gutachten ein und die erstellte Arbeit wurde zusammen mit den Gutachten vom 25. November 2016 bis zum 23. Dezember 2016 hochschulöffentlich an der Leuphana Universität ausgelegt. Als weiteres Beispiel institutioneller Herausforderungen muss an dieser Stelle genannt werden, dass mir nicht gestattet wurde, diese Gutachten zu veröffentlichen. Zwar habe ich die Promotionskommission angefragt, ob die Gutachten zu meiner Doktorarbeit online zum Download zu Verfügung gestellt werden dürfen. Mit Verweis auf die Promotionsordnung wurde eine solche Veröffentlichung jedoch nicht gestattet.

Inhaltlich bewerteten die beiden Gutachter und die Gutachterin die Arbeit durchweg positiv und bewegten sie alle im ähnlichen Rahmen der Benotung. Die Kritik beschränkte sich vornehmlich auf strukturelle Schwierigkeiten in der Arbeit. Zum Beispiel wurde angemerkt, dass es mir nicht immer gelungen ist, die Komplexität zu bändigen und das es im Verlauf der Arbeit oft zu sprachliche Wiederholungen kommt. ---- TODO: Weiter ausarbeiten ----

\section{Die Verteidigung}

Nach Auslage der Gutachten Ende 2016 und der Annahme der Dissertation durch die Promotionskommission Anfang 2017 erfolgte im Feburar 2017 die mündliche Verteidigung (Disputation). In der Vorbereitung zur Disputation konnte ich den gesamten Prozess mit einer zeitlichen Distanz noch mal reflektieren und diese in die Betrachtung einbauen. In der Verteidigung habe ich mein Vorhaben dargestellt und herausgestellt, dass es bisher kaum Versuche die theoretischen Erkenntnisse praktisch- experimentell zu überprüfen gab und die Diskussion und kritische Ausblicke für weitere Entwicklungen im Sinne der Verarbeitung von Forschungsergebnissen sowie der Anwendung und Neuinterpretation von Ergebnissen bisher wenig differenziert stattfindet. Ich habe meine Hypothesen erläutert, die Herangehensweise erklärt das Vorgehen und die Methode dargestellt und bin zu guter letzt auf die Ergebnisse eingegangen. In der darauffolgenden Diskussion wurden die Punkte der Gutachten besprochen und über Möglichkeiten der Veränderungen bei wissenschaftlichen Qualifikationsarbeiten diskutiert. ---- TODO: Weiter ausarbeiten ----

\section{Die Veröffentlichung}

Nach erfolgreicher Disputation steht zwisschen dem Promovenden und dem Abschluss der Promotion nur die Druckerlaubnis (Imprimatur). Sie stellt den letzten Schritt vor der Veröffentlichung einer Dissertationsschrift dar und dient dazu, nach der Bewertung der eigentlichen Doktorarbeit und der Durchführung der mündlichen Prüfung, bestehende Mängel vor dem finalen Druck auszuräumen. Die Freigabe wird nach der Überarbeitung und der Beseitigung der in den Gutachten und während der Disputation durch die Gutachter und Gutachterin genannten Mängel durch den Doktorvater erteilt. Erst nach der Freigabe und der Ablieferung des Buchdrucks wird die Promotionsurkunde ausgehändigt und erst dann hat der Doktorand das Recht den Doktorgrad zu tragen.

Seit der Einreichung der Arbeit im Juni 2016 hat sich viel verändert. Technologisch haben sich die Möglichkeiten zur offenen Publikation wissenschaftlicher Arbeiten weiterentwickelt. Politisch sind klare Entwicklungen für eine Novellierung der rechtlichen Rahmenbedingungen wissenschaftlicher Kommunikation erkennbar und gesamtgesellschaftlich werden Forderungen nach Öffnung von Wissenschaft und der wissenschaftlichen Organisationen lauter. Laut der Ausgabe 16/2017 der Wochenzeitung DIE ZEIT sind die "Chancen für einen Aufbruch (...) wohl noch nie so gut wie heute". Und trotzdem bleibt es primär bei theroetischen Forderungen, Demonstrationen oder zaghaften politischen Reaktionen.

Im Gegensatz zu diesser theoretischen Auseinandersetzung hoffe ich mit dieser Arbeit einen konkreten praktischen Beitrag zur Debatte um Offenheit in Wissenschaft und Forschung geleistet zu haben. Das wohl eindeutigste Ergebnis meines Vorhabens ist, dass Offenheit in Wissenschaft und Forschung sowie bei wissenschaftlichen Qualifikationsarbeiten schon heute grundsätzlich möglich und umsetzbar sind.

Ich glaube auch, dass bevor sich Universitäten öffnen und offene wissenschaftliche Kommunikation im Forschungsalltag eine ernstzunehmende Rolle spielen braucht es mehr konkrete Experimente zu Offenheit und den Grenzen der Offenheit aus den Reihen der Wissenschaftlerinnen und Wissenschaftler selbst. Sie sind unausweichlich nicht nur Forderungen zu stellen, sondern konkrete Veränderungen zu erzielen. Wir befinden uns bei der Erforschung von Offenheit noch immer erst am Anfang. Gestalten Sie mit!

---- TODO: Weiter ausarbeiten ----
