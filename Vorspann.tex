\begin{vorspann}
Als ich begann die vorliegende Arbeit zu schreiben, stand für mich neben der inhaltlichen Auseinandersetzung mit Offenheit im wissenschaftlichen Kommunikationsprozess sowie der Herleitung von Offenheit in Wissenschaft und Forschung die Frage im Vordergrund, wie offen und transparent eine Doktorarbeit erstellt werden kann. Ich wollte wissen, ob  es möglich ist, alle Informationen zu einem solchen Vorhaben und die Arbeit ansich direkt und unmittelbar bei der Erstellung für jeden, jederzeit frei zugänglich im Internet unter einer offenen und freien Lizenz (CC-BY-SA) einsehbar und verfolgbar zu machen. Mit diesem Vorspann, der fast ein Jahr nach Fertigstellung der eigentlichen Dissertation geschrieben wurde, möchte ich diesen Selbstversuch ein Promotionsvorhaben so offen wie möglich zu gestalten zu ende zu bringen.

Schon von Beginn an, stellte dieser Selbstversuch und der Anspruch an Offenheit bei der Erstellung eine große Herausforderung dar. Zum einen an mich selbst, aber auch an die Institution Universität. Um den Anforderungen der aktuell geltenden Prüfungsordnung der Leuphana Universität zu entsprechen, wurden 2013 in einem Schreiben an die Promotionskommission die Bedingungen für die offene Erstellung der Arbeit angeboten und die eine Begründung für die Vereinbarkeit mit der Promotionsordnung dargelegt. Nach einer rechtlichen Prüfung durch das Justiziariat der Universität entsprach die Promotionskommission am 12. Dezember 2013 mehrheitlich dem Gesuch, die Arbeit "offen" verfassen zu dürfen.

Da die Aufmerksamkeit das offene Verfahren und die offene Schreibweise der Doktorarbeit in meinem direkten Umfeld groß war, war ich mir nicht immer sicher den richtigen Weg gewählt zu haben. Zum einen war ich mir selbst nicht immer sicher, ob dieses Vorhaben technisch in der Form umsetzbar ist oder ob ich der Herausforderung der Promotion sowie den Anspruch an die offenen Kommunikation und der eventuellen Mehrarbeit standhaft bleiben würde.

Dennoch entstand in den folgenden drei Jahren unter diesen Voraussetzungen und den daraus resultierenden technischen, rechtlichen und strukturellen Herausforderungen bei der Erstellung die vorliegende Arbeit. Sie kann als zweifach reflexives Projekt betrachtet werden, da es sich dabei nicht nur um die wissenschaftliche Analyse von Wissenschaft handelt, sondern in seiner experimentellen und offenen Herstellung sich selbst noch einmal spiegelt.

Nach der Einreichung der Arbeit im Juni 2016 erfolgte eine lange Zeit des Wartens. In den Folgemonaten sollten nun Gutachten durch die drei Gutachter erstellt, die Note ermittelt und ein Termin für die Disputation festgelegt. In der Zwischenzeit hatte das offene Verfahren auch das Interesse anderer geweckt. In einem Interview für die Helmholzgemeinschaft, für das Merton Magazin sowie für netzpolitik.org, dem reichweitenstärkstem deutschsprachigen Blog zu digitalen Freiheitsrechten und netzpolitischen Themen habe ich Fragen zu meiner offenen Promotion beantwortet.

Ende November 2016 war es dann soweit und alle drei Gutachten sind eingetroffen und meine Doktorarbeit wurde zusammen mit den Gutachten vom 25. November 2016 bis zum 23. Dezember 2016 hochschulöffentlich an der Leuphana Universität ausgelegt. Leider dürften ich die Gutachten nicht veröffentlichen. Zwar habe ich die Promotionskommission angefragt, ob die Gutachten zu meiner Doktorarbeit online veröffentlicht werden dürfen. Mit Verweis auf die Promotionsordnung wurde eine solche Veröffentlichung jedoch nicht gestattet.

Die Gutachten ware durchweg positiv und bewegten sie alle im ähnlichen Rahmen der Benotung. Die Kritik beschränkte sich vornehmlich auf Schwierigkeiten, die Komplexität zu bändigen, sprachliche Wiederholungen --- TBD ---

Nächster Schritt nach der Auslage der Gutachten und der Arbeit im Fall der Annahme der Dissertation die mündliche Verteidigung (Disputation). Diese sollte in der Regel innerhalb weiterer 4 Wochen später stattfinden. Die Arbeit wurde Anfang 2017 angenommen und die Disputation auf Anfang Februar 2017 festgelegt.

Im Februar 2017 erfolgte dann die Disputation. In der Vorbereitung konnte ich den gesamten Prozess noch mal reflektieren. In der Verteidigung bin ich vor allem auf die wenige Bemühungen die bisherigen Entwicklungen im Bereich der Forderung nach Öffnung wissenschaftlicher Kommunikation aus geistes- und kulturwissenschaftlicher Perspektive genauer zu untersuchen sowie die wenigen Erkenntnisse über die Öffnung wissenschaftlicher Kommunikation und deren Gegenüberstellung empirisch erhobener Daten eingegangen. Ich habe herausgestellt, dass es bisher kaum Versuche die theoretischen Erkenntnisse praktisch- experimentell zu überprüfen gab und die Diskussion und kritische Ausblicke für weitere Entwicklungen im Sinne der Verarbeitung von Forschungsergebnissen sowie der Anwendung und Neuinterpretation von Ergebnissen wenig differenziert stattfindet. Ich habe meine Hypothesen erläutert, die Herangehensweise erklärt das Vorgehen und die Methode dargestellt und bin zu guter letzt auf die Ergebnisse eingegangen.

In der Diskussion wurden die Punkte der Gutachten besprochen und über Möglichkeiten der Veränderungen bei wissenschaftlichen Qualifikationsarbeiten diskutiert.

--- TODO: Weiter ausarbeiten ---
\end{vorspann}
