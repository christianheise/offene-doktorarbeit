Als ich begann die vorliegende Arbeit zu schreiben, stand für mich neben der inhaltlichen Auseinandersetzung mit Offenheit im wissenschaftlichen Kommunikationsprozess sowie der Herleitung von Offenheit in Wissenschaft und Forschung eine Frage im Vordergrund: Wie offen und transparent kann eine Doktorarbeit erstellt werden? Ich wollte wissen, ob es möglich sein würde, alle Informationen zu einem solchen Vorhaben und den  Arbeitsprozess selbst direkt und unmittelbar während der Erstellung für jeden, jederzeit frei zugänglich im Internet unter einer offenen und freien Lizenz (CC-BY-SA) einsehbar und verfolgbar zu machen.

Von Beginn an stellten die Beantwortung dieser Frage, der Selbstversuch sowie der damit einhergehende Anspruch an Offenheit eine große Herausforderung dar. Zum einen an mich selbst, zum anderen aber auch an die Institution Universität und den Anforderungen der aktuell geltenden Prüfungsordnung. Zur Klärung schrieb ich zu Beginn der Arbeit an die Promotionskommission und erfragte die Bedingungen für die offene Erstellung der Arbeit. m selben Schreiben legte ich eine mögliche Begründung für die Vereinbarkeit mit der Promotionsordnung dar. Nach der fast einjährigen rechtlichen Prüfung durch das Justiziariat der Universität entsprach die Promotionskommission am 12. Dezember 2013 mehrheitlich dem Gesuch, die Arbeit unter den genannten Bedingungen "offen" verfassen zu dürfen.

In meinem direkten Umfeld war die Aufmerksamkeit für das offene Verfahren und die offene Schreibweise der Doktorarbeit groß.  Dennochwar ich mir nicht immer sicher den richtigen Weg gewählt zu haben. Zum einen konnte ich selbst nicht abschließend einschätzen, ob dieses Vorhaben technisch nach meinen eigenen Ansprüchen und der vorausgesetzten Form umsetzbar sein würde und ob ich den Anspruch an die offene Kommunikation erfüllen werden könnte. Zudem war mir das potenzielle Ausmaß der Mehrarbeit, die mein Vorgehen verursachen könnte, vorher nicht bewusst.

Nach drei Jahren entstand unter diesen Voraussetzungen und den daraus resultierenden technischen, rechtlichen und strukturellen Herausforderungen bei der Erstellung ein aus wissenschaftlicher Perspektive zweifach reflexives Projekt. Es handelt sich dabei nicht nur um die wissenschaftliche Analyse von Wissenschaft, sondern auch um ein Vorhaben, dass sich in seiner experimentellen und offenen Herstellung selbst noch einmal spiegelt.

Die Forschung fand aus drei Perspektiven statt: Erstens aus intellektuell kontemplativer Perspektive durch die Abarbeitung der definitorischen Fragen sowie die historische Einordnung und die Erarbeitung theoretischer Motive und Beweggründe für Wissenschaftler und Wissenschaftlerinnen für wissenschaftliches Kommunizieren. Zweitens aus der Perspektive des Beobachters, bei der die aktuellen Rahmenbedingungen wissenschaftlicher Kommunikation durch empirische Forschung erhoben, historisch verglichen und mit den theoretisch erarbeiteten Annahmen abgeglichen wurden. Aus der dritten Perspektive, der des aktiven Teilnehmers, kann dieses Vorhaben als praktischer Beitrag zu der Debatte zur Neugestaltung wissenschaftlicher Kommunikation betrachtet werden.

Mit diesem Vorspann und ein Jahr nach Fertigstellung der eigentlichen Dissertation möchte ich den Selbstversuch, ein Promotionsvorhaben so offen wie möglich zu gestalten, zum Abschluss führen, denn ein Promotionsvorhaben endet nicht mit der Einreichung der Arbeit, sondern erst nach der Begutachtung, der Verteidigung und der Veröffentlichung dieser.

\section{Die Gutachten und die Benotung}

Nach der Einreichung der Arbeit im Juni 2016 folgte eine lange Zeit des Wartens. In den Folgemonaten wurden die Gutachten durch die drei Gutachter erstellt, die Note ermittelt und ein Termin für die Disputation festgelegt. In der Zwischenzeit hatte der praktische Teil, das offene Verfahren, auch das mediale Interesse verstärkt. Im Interview für die Helmholtz Gemeinschaft, für das Merton Magazin der Stiftung Mercator sowie für netzpolitik.org, dem reichweitenstärkstem deutschsprachigen Blog zu digitalen Freiheitsrechten und netzpolitischen Themen, habe ich Fragen zu meiner offenen Promotion beantwortet.

Ende November 2016 trafen alle drei Gutachten ein und die erstellte Arbeit wurde zusammen mit den Gutachten vom 25. November 2016 bis zum 23. Dezember 2016 hochschulöffentlich an der Leuphana Universität ausgelegt. Als weiteres Beispiel institutioneller Herausforderungen muss an dieser Stelle genannt werden, dass mir nicht gestattet wurde, diese Gutachten zu veröffentlichen. Auf die Anfrage an die Promotionskommission, ob die Gutachten zu meiner Doktorarbeit online zum Download zur Verfügung gestellt werden dürfen, wurde mit Verweis auf die Promotionsordnung  eine solche Veröffentlichung nicht gestattet.

Inhaltlich bewerteten die beiden Gutachter und die Gutachterin die Arbeit durchweg positiv und bewegten sie alle in einem ähnlichen Rahmen der Benotung. Die Kritik beschränkte sich vornehmlich auf strukturelle Schwierigkeiten in der Arbeit. So wurde angemerkt, dass es nicht immer gelungen sei, die Komplexität des Themas zu bändigen und dass es im Verlauf der Arbeit oft zu sprachlichen und inhaltlichen Wiederholungen kommt. ---- TODO: Weiter ausarbeiten ----

\section{Die Verteidigung}

Nach Auslage der Gutachten Ende 2016 und der Annahme der Dissertation durch die Promotionskommission Anfang 2017 erfolgte im Februar 2017 die mündliche Verteidigung (Disputation). In der Vorbereitung zur Disputation konnte ich den gesamten Prozess mit einer zeitlichen Distanz noch einmal reflektieren und diese in die Betrachtung einbauen. In der Verteidigung habe ich mein Vorhaben dargestellt und herausgestellt, dass es bisher kaum Versuche gab, die theoretischen Erkenntnisse praktisch-experimentell zu überprüfen und die Diskussion und kritischen Ausblicke bisher wenig differenziert sind, wenn es um weitere Entwicklungen bei Anwendung, Neuinterpretation und Verarbeitung von Forschungsergebnissen geht. Ich habe meine Hypothesen erläutert, die Herangehensweise erklärt, das Vorgehen und die Methode dargestellt und bin zu guter letzt auf die erarbeiteten Ergebnisse eingegangen. In der darauffolgenden Diskussion wurden die Punkte der Gutachten besprochen und über Möglichkeiten der Veränderungen bei wissenschaftlichen Qualifikationsarbeiten diskutiert. ---- TODO: Weiter ausarbeiten ----

\section{Die Veröffentlichung}

Nach erfolgreicher Disputation steht zwischen dem Promovenden und dem Abschluss der Promotion nur die Druckerlaubnis (Imprimatur). Sie stellt den letzten Schritt vor der Veröffentlichung einer Dissertationsschrift dar und dient dazu, nach der Bewertung der eigentlichen Doktorarbeit und der Durchführung der mündlichen Prüfung bestehende Mängel vor dem finalen Druck auszuräumen. Die Freigabe wird nach der Überarbeitung und der Beseitigung der in den Gutachten und während der Disputation durch die Gutachter und Gutachterin genannten Mängel durch den Doktorvater erteilt. Erst nach der Freigabe und der Ablieferung des Buchdrucks wird die Promotionsurkunde ausgehändigt und erst dann hat der Doktorand das Recht den Doktorgrad zu tragen.

Seit der Einreichung der Arbeit im Juni 2016 hat sich viel verändert. Technologisch haben sich die Möglichkeiten zur offenen Publikation wissenschaftlicher Arbeiten weiterentwickelt. Politisch sind klare Entwicklungen für eine Novellierung der rechtlichen Rahmenbedingungen wissenschaftlicher Kommunikation erkennbar und gesamtgesellschaftlich werden Forderungen nach Öffnung von Wissenschaft und der wissenschaftlichen Organisationen lauter. Laut der Ausgabe 16/2017 der Wochenzeitung DIE ZEIT sind die "Chancen für einen Aufbruch (...) wohl noch nie so gut wie heute". Und trotzdem bleibt es primär bei theoretischen Forderungen, Demonstrationen oder zaghaften politischen Reaktionen.

Im Gegensatz zu der rein theoretischen Auseinandersetzung hoffe ich mit dieser Arbeit einen konkreten praktischen Beitrag zur Debatte um Offenheit in Wissenschaft und Forschung geleistet zu haben. Das wohl eindeutigste Ergebnis meines Vorhabens ist, dass mehr Offenheit in Wissenschaft und Forschung sowie bei wissenschaftlichen Qualifikationsarbeiten schon heute grundsätzlich möglich und umsetzbar sind.

Bevor sich Universitäten öffnen und offene wissenschaftliche Kommunikation im Forschungsalltag eine ernstzunehmende Rolle spielen wird, braucht es mehr konkrete Experimente zu Offenheit und den Grenzen der Offenheit in der wissenschaftlichen Gemeinschaft. Wissenschaftlerinnen und Wissenschaftler sollten nicht nur Forderungen stellen, sondern in Experimenten versuchen, konkrete Veränderungen zu erzielen. Wir befinden uns bei der Erforschung von Offenheit noch immer erst am Anfang.

---- TODO: Weiter ausarbeiten ----
