\subsubsection{Die Ökonomie des wissenschaftlichen Kommunizierens}
Die klassische Ökonomie der wissenschaftlichen Kommunikation beruht auf der Durchsetzung von Urheberrechten, die den Zugriff und die Wiederverwendung von geschützten Inhalten beschränken und die Zahlung einer Gebühr durch den Leser verlangen um Zugang zu der Veröffentlichung zu erhalten.  Dieses Modell baisert auf einer sozial ineffizientem Ebene.  Diese ungewöhnlichen Ökonomie der Wissenschaftsverlage ist nicht neu und hat sich im Laufe der Zeit entwickelt, die starke Wahrnehmung der Ungerechtigkeit des Systems, vorallem an den Preismodellen für  wissenschaftliche Publikationen , findet aber erst seit kurzem statt.  
