\section{Forschungsstand}
Das in Kapitel 1 beschriebene “ungewöhnliche Geschäftsmodell” hinter der wissenschaftlichen Kommunikation ermöglichte den Verlegern hohe Betriebsgewinnmargen von über 35 Prozent  und hohe jährliche Wachstumsraten . Sucht man nach Gründen für die Unterstützung des bisherigen Modells durch die Wissenschaftsgemeinschaft, wird deutlich, dass vor allem die in Kapitel 2.3 beschriebene wissenschaftliche Reputation einen zentralen, extrinsischen Motivationsfaktor für Wissenschaftler darstellt . Die akademische Reputation „ist [dabei] die zentrale Kommunikationsform, die das Wissenschaftssystem charakterisiert“ . Die Ergebnisse aus wissenschaftlicher Forschung werden dabei als Publikationen vor allen Mitgliedern der Wissenschaft präsentiert, „um diese intern von der Wissenschaftsgemeinde als wissenschaftlich bzw. unwissenschaftlich zertifizieren zu lassen" . In diesem Kapitel soll der aktuelle Forschungsstand zur Öffnung von Wissenschaft, zu den Treibern und Bremsern dieser Entwicklung in diesem Kontext und dem damiteinhergehenden Paradigmenwechsel mit Fokus auf den Themenbereich der wissenschaftlichen Reputation wiedergegeben werden. Grundlage der Darstellung sind ausgewählte relevante und aktuelle Werke der Fachdiskussion die sich mit dem Phänomen Öffnung von Wissenschaft beschäftigen. Ziel des Kapitels ist die Entwicklung einer geeigneten wissenschaftlichen Fragestellung, die weitere empirische und theoretische Diskussion bedarf.