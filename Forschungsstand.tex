\section{Forschungsstand}
Das in Kapitel 1 beschriebene “ungewöhnliche Geschäftsmodell” hinter der wissenschaftlichen Kommunikation ermöglichte den Verlegern hohe Betriebsgewinnmargen von über 35 Prozent  und hohe jährliche Wachstumsraten . Sucht man nach Gründen für die Unterstützung des bisherigen Modells durch die Wissenschaftsgemeinschaft, wird deutlich, dass vor allem die in Kapitel 2.3 beschriebene wissenschaftliche Reputation einen zentralen, extrinsischen Motivationsfaktor für Wissenschaftler darstellt . Die akademische Reputation „ist [dabei] die zentrale Kommunikationsform, die das Wissenschaftssystem charakterisiert“ . Die Ergebnisse aus wissenschaftlicher Forschung werden dabei als Publikationen vor allen Mitgliedern der Wissenschaft präsentiert, „um diese intern von der Wissenschaftsgemeinde als wissenschaftlich bzw. unwissenschaftlich zertifizieren zu lassen" . In diesem Kapitel soll der aktuelle Forschungsstand zur Öffnung von Wissenschaft, zu den Treibern und Bremsern dieser Entwicklung in diesem Kontext und dem damiteinhergehenden Paradigmenwechsel mit Fokus auf den Themenbereich der wissenschaftlichen Reputation wiedergegeben werden. Grundlage der Darstellung sind ausgewählte relevante und aktuelle Werke der Fachdiskussion die sich mit dem Phänomen Öffnung von Wissenschaft beschäftigen. Ziel des Kapitels ist die Entwicklung einer geeigneten wissenschaftlichen Fragestellung, die weitere empirische und theoretische Diskussion bedarf.
\subsection{Beschreibung des Forschungsstands}
In diesem Teil der Arbeit soll der Forschungsstand umfassend analysiert und beschrieben werden. Es soll mit Hilfe einer ausgewogenen und umfassenden Literaturanalyse dargestellt werden, welche Argumentationen es für und gegen sowie welche Möglichkeiten und Grenzen es für die Öffnung der Wissenschaft im Rahmen von Veröffentlichung angeführt werden, soweit das der derzeitige sehr lückenhafte Forschungsstand erlaubt. Eine kritische Analyse soll dabei Pro- und Kontraargumente zusammenfassen und einen Überblick über die aktuelle Debatte um Open Science und Open Access ermöglichen. Diese Analyse wird auf der Annahme durchgeführt, dass sich Open Access in einer Übergangsphasen von der reinen offenen Bereitstellung wissenschaftlicher Publikationen und dem damit verbundenen offenen Zugang zur Wissenschaft zur umfassenden und offenen Wissensverteilung und dem damit einherdehnenden Zugriff auf Wissenschaft an die Gesamtgesellschaft (Open Science) befindet. Darüber hinaus sollen medienkulturwissenschaftlich Open Science und Open Access in ihren technischen als auch in ihren gesellschaftlichen und politischen Aspekten sowie die kulturellen Auswirkungen der Medienbrüchen im Rahmen von hybridem Publizieren auf hohem theoretischem Niveau reflektiert werden. Abschließend sollen Treiber und Bremser für die Öffnung von Wissenschaft empirisch erhoben werden und in der Gesamtbetrachtung der Arbeit zusammengeführt werden.
\subsection{Defizite}
Viele der Erklärungsansätze für den Paradigmenwechsel hin zur Öffnung der Wissenschaft basieren auf Annahmen, in denen ein direkter Zusammenhang von technischen Entwicklungen unmittelbar auf (wissenschafts-)politische und kulturelle Bewegungen geschlossen werden. Darüber hinaus beschränkt sich die Perspektive primär auf den Zugang zum Ergebnis von Wissenschaft und weniger auf die Öffnung des gesamten Prozesses. Die theoretische Auseinandersetzung mit der Geschlossenheit des wissenschaftlichen Diskurses  auf der Einen und mit den Treibern und Bremsern im realen wissenschaftlichen Prozess werden in der gängigen Literatur auf der anderen Seite, wird nicht genügend berücksichtigt. Hier wird vor allem die Verbindung zwischen wissenschaftlicher Reputation und Geschlossenheit des Wissensproduktionsprozesses nur selten erörtert.
\subsection{Entwicklung der Fragestellung}
Nach der theoretischen Diskussion sollen in diesem Teil die offenen inhaltliche Fragestellungen dargestellt werden.  Die Fragestellung dieser Arbeit soll dabei drei Kriterien entsprechen :
\begin{enumerate}
\item aus ihrer Formulierung soll klar hervorgehen wie sie verstanden werden kann
\item sie soll im Kontext der wissenschaftlichen Disziplin einen klaren definierten Ort haben 
\item ihr Gegenstand muss eindeutig sein.
\end{enumerate}
Die vorläufige forschungsleitende Hypothese in dieser Arbeit ist (siehe auch 3.1), dass Open Access sich in der Übergangsphase zu Open Science befindet. Die daraus ableitende Fragestellung umfasst dabei zum einen die theoretische Bedeutung von Offenheit im Rahmen des wissenschaftlichen Diskurs- und Machtbegriffs (Kapitel 4.2) aber auch die empirische Frage nach den Motiven und Beweggründen für Wissenschaftler, Verlagen und Universitäten diese Offenheit auf Wissenschaft in den unterschiedlichen Disziplinen zu ermöglichen oder zu verhindern (Kapitel 4.1). Abschließend soll erörtert werden, welche mögliche Auswirkungen auf Selbstverständnis von Wissenschaft, auf das wissenschaftliche Kapital sowie auf die unterschiedlichen Disziplinen durch diesen Prozess der Öffnung zu erwarten sind (Kapitel 5).
