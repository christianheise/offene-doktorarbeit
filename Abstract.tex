\begin{abstract}Eine praxistheoretische Studie zum Verständnis der Konzepte von Open Access und Open Science im Rahmen einer Differenzierung zwischen den verschiedenen wissenschaftlichen Disziplinen und vor dem Hintergrund wissenschaftlicher Reputation.

Ziel der Arbeit ist die Betrachtung der definitorischen Fragen um die Begriffe Open Access und Open Science in ihren unterschiedlichen Ausprägungen, sowie die Darstellung des aktuellen Forschungsstandes und der Debatte um die Öffnung von Wissenschaft und Forschung. Im weiteren Fokus steht die Identifikation der Treiber und Bremser für die Öffnung von wissenschaftlicher Kommunikation und Prozessen. Durch eine Befragung und die Dokumentation des eigenen offene Promotionsvorhabens (Arbeitsperspektive) wird dem wissenschaftlichen Anspruch Rechnung getragen.
\end{abstract}

\textbf{HINWEIS: KOMMENTARE SIND AUFGRUND DER PROMOTIONSORDNUNG NICHT ERLAUBT UND DEAKTIVERT! DIESE ARBEIT IST IN BEARBEITUNG (WORKING DRAFT) UND WURDE IM AUGUST 2014 AUS ANDEREN DOKUMENTEN ZUSAMMEGESETZT.  Weitere Informationen: \href{http://offene-doktorarbeit.de}{http://offene-doktorarbeit.de}}

\begin{quote}
\textbf{Scientia Donum Dei Est, Unde Vendi Non Potest}
\end{quote}
