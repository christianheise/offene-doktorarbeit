\begin{abstract}Eine praxistheoretische Studie zum Verständnis der Konzepte vor dem Hintergrund der Differenzierung zwischen verschiedenen wissenschaftlichen Disziplinen und dem möglichen Paradigmenwechsel bei der Betrachtung der wissenschaftlichen Kommunikation und Reputation, sowie der experimentelle Versuch der Anfertigung einer Dissertation als Open Science-Projekt.

Ziel der Arbeit ist die Abarbeitung der definitorischen Fragen um die Begriffe Open Access und Open Science in seinen unterschiedlichen Ausprägungen vor dem Hintergrund von wissenschaftlicher Reputation und über die unterschiedlichen Fachdisziplinen, sowie die Identifikation der Treiber und Bremser für die Öffnung von wissenschaftlicher Informationen und Prozesse durch eine Befragung und die Dokumentation des eigenen offene Promotionsvorhabens (Arbeitsperspektive).
\end{abstract}

\textbf{HINWEIS: KOMMENTARE SIND AUFGRUND DER PROMOTIONSORDNUNG NICHT ERLAUBT UND DEAKTIVERT! DIESE ARBEIT IST IN BEARBEITUNG (ROUGH DRAFT) UND WURDE IM AUGUST 2014 AUS ANDEREN DOKUMENTEN ZUSAMMEGESETZT. Todos: Zitate und Referenzen übernehmen; Dokumente zusammenführen}
