
\begin{abstract}Eine praxistheoretische Studie zum Verständnis der Konzepte von Open Access und Open Science im Rahmen einer der Differenzierung zwischen verschiedenen wissenschaftlichen Disziplinen und vor dem Hintergrund wissenschaftlicher Reputation.

Ziel der Arbeit ist die Betrachtung der definitorischen Fragen um die Begriffe Open Access und Open Science in seinen unterschiedlichen Ausprägungen, sowie die Darstellung des aktuellen Forschungsstandes, die Identifikation der Treiber und Bremser für die Öffnung von wissenschaftlicher Informationen und Prozesse durch eine Befragung und die Dokumentation des eigenen offene Promotionsvorhabens (Arbeitsperspektive).
\end{abstract}

\textbf{HINWEIS: KOMMENTARE SIND AUFGRUND DER PROMOTIONSORDNUNG NICHT ERLAUBT UND DEAKTIVERT! DIESE ARBEIT IST IN BEARBEITUNG (WORKING DRAFT) UND WURDE IM AUGUST 2014 AUS ANDEREN DOKUMENTEN ZUSAMMEGESETZT.  Weitere Informationen: \href{http://offene-doktorarbeit.de}{http://offene-doktorarbeit.de}}

\begin{quote}
\textbf{scientia donum dei est unde vendi non potest}
\end{quote}
