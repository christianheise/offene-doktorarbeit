\begin{abstract}

\textbf{Deutsch}

Bei dieser Arbeit handelt es sich um eine explorative Studie zum Verständnis der Konzepte von Open Access und Open Science im Rahmen der Digitalisierung und der Differenzierung zwischen den verschiedenen wissenschaftlichen Disziplinen und vor dem Hintergrund wissenschaftlicher Reputation.
Ziel der Arbeit ist die Darstellung, Analyse und Verhandlung der Annahmen rund um die Etablierung sowie die Durchführung von offenen und digitalen wissenschaftlichen Erkenntnisprozessen. Die forschungsleitende Hypothese dieser Arbeit ist, dass sich die Öffnung des Zugangs zu wissenschaftlichen Erkenntnissen für die Gesamtgesellschaft (Open Access) in einer Übergangsphase zur Öffnung des Zugriffs auf den gesamten wissenschaftlichen Erkenntnisprozess (Open Science) befindet.

Im Verlauf der Arbeit werden die Vorannahmen zum Interesse an der Öffnung wissenschaftlicher Kommunikation und der Verbreitung dieser den praktischen Gegebenheiten im wissenschaftlichen Alltag in einer Befragung gegenübergestellt. Dabei wird die Thematik in Bezug zu den Herausforderungen an die wissenschaftliche Gemeinschaft und das wissenschaftliche System gesetzt sowie in einen historischen Kontext gestellt. In diesem Zusammenhang werden insbesondere die Diskrepanz zwischen der Idee der Öffnung von wissenschaftlicher Kommunikation und der wissenschaftlichen Realität adressiert, sowie Katalysatoren und Hindernisse für die Umsetzung der Konzepte rund um die Öffnung von Wissenschaft identifiziert und empirisch überprüft.

Die Erfahrungen und Meinungen der befragten Wissenschaftler und Wissenschaftlerinnen werden den Erfahrungen aus einem Selbstversuch des jederzeit öffentlich einsehbaren Erstellungsprozesses dieser Arbeit gegenübergestellt, die Unterschiede zwischen den Disziplinen herausgearbeitet und Handlungsempfehlungen für das offene Bearbeiten wissenschaftlicher Fragestellungen abgeleitet.
Abschließend werden die Ergebnisse zusammengefasst, bewertet und in einem Ausblick Anknüpfungspunkte für weitere Forschungsbemühungen dargestellt.

Die gesamte Arbeit wurde direkt und unmittelbar bei der Erstellung für jeden, jederzeit frei zugänglich im Internet auf live.offene-doktorarbeit.de unter einer offenen und freien Lizenz (opendefinition.org) maschinenlesbar veröffentlicht.

\textbf{Englisch}

Although open science is currently having a lot of attention and opening up of research is considered as an appropriate and trend-setting model for a scientific communication system of the future, it is still quite difficult to actually show examples or put the principles of open science into practise.

This doctoral thesis with the title „Open Access to Open Science: About the transition of digital cultures relating to scientific communication“ is an explorative study of the concepts of Open Access and Open Science in the context of digitization and the differentiation between different scientific disciplines. The aim of this work is the theoretical and practical presentation, analysis and negotiation of the assumptions concerning the establishment as well as the implementation of open digital scientific knowledge processes.

It is the first approach to conduct a doctoral research thesis as open as possible. The goal of this open approach: Publish everything related to the doctoral study and research process as soon as possible, as comprehensive as possible and under an open license (opendefinition.org), machine readable, online available at all time for everyone. It was written in german language, online and always open readable and accessible at live.offene-doktorarbeit.de.

\end{abstract}

\textbf{HINWEIS: KOMMENTARE SIND AUFGRUND DER PROMOTIONSORDNUNG NICHT ERLAUBT UND DEAKTIVERT! DIESE ARBEIT IST IN BEARBEITUNG UND WURDE IM AUGUST 2014 AUS ANDEREN DOKUMENTEN ZUSAMMEGESETZT. Diese Version der Arbeit wurde im Juni 2016 eingereicht.  Weitere Informationen: \href{http://offene-doktorarbeit.de}{http://offene-doktorarbeit.de}}
