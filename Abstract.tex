\begin{abstract}Eine Studie zum Verständnis der Konzepte von Open Access und Open Science im Rahmen der Digitalisierung und der Differenzierung zwischen den verschiedenen wissenschaftlichen Disziplinen und vor dem Hintergrund wissenschaftlicher Reputation.

Ziel der Arbeit ist die Darstellung, Analyse und Verhandlung der Annahmen und Definitionen rund um die Etablierung sowie die Durchführung von offenen wissenschaftlichen Erkenntnisprozessen. Im Verlauf der Arbeit werden diese Annahmen den praktischen Gegebenheiten im wissenschaftlichen Alltag gegenübergestellt. Dabei wird die Thematik in Bezug zu den Herausforderungen an die wissenschaftliche Gemeinschaft und das System Universität gesetzt sowie in einen historischen Kontext gestellt. In diesem Zusammenhang werden insbesondere die Diskrepanz zwischen der Idee der Öffnung von wissenschaftlicher Kommunikation und der wissenschaftliche Realität adressiert, sowie Katalysatoren und Hindernisse für die Umsetzung der Konzepte rund um die Öffnung von Wissenschaft identifiziert und empirisch überprüft. Die Erfahrungen und Meinungen der Wissenschaftler und Wissenschaftlerinnen werden dabei den Erfahrungen aus einem Selbstversuch gegenübergestellt, die Unterschiede zwischen den Disziplinen herausgearbeitet und abschließend Handlungsempfehlungen für das offene Bearbeiten wissenschaftlicher Fragestellungen abgeleitet. \end{abstract}

\textbf{HINWEIS: KOMMENTARE SIND AUFGRUND DER PROMOTIONSORDNUNG NICHT ERLAUBT UND DEAKTIVERT! DIESE ARBEIT IST IN BEARBEITUNG (WORKING DRAFT) UND WURDE IM AUGUST 2014 AUS ANDEREN DOKUMENTEN ZUSAMMEGESETZT.  Weitere Informationen: \href{http://offene-doktorarbeit.de}{http://offene-doktorarbeit.de}}

\begin{quote}
\textbf{Scientia Donum Dei Est, Unde Vendi Non Potest}
\end{quote}
