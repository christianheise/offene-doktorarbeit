\begin{abstract}

Eine Studie zum Verständnis der Konzepte von Open Access und Open Science im Rahmen der Digitalisierung und der Differenzierung zwischen den verschiedenen wissenschaftlichen Disziplinen und vor dem Hintergrund wissenschaftlicher Reputation.

Ziel der Arbeit ist die Darstellung, Analyse und Verhandlung der Annahmen und Definitionen rund um die Etablierung sowie die Durchführung von offenen wissenschaftlichen Erkenntnisprozessen. Die forschungsleitende Hypothese dieser Arbeit ist, dass sich die Öffnung des Zugangs zu wissenschaftlichen Erkenntnissen für die Gesamtgesellschaft (Open Access) in einer andauernden Übergangsphase zur Öffnung des Zugriffs auf den gesamten wissenschaftlichen Erkenntnisprozess (Open Science) befindet.

Im Verlauf der Arbeit werden die Vorannahmen den praktischen Gegebenheiten im wissenschaftlichen Alltag in einer Befragung gegenübergestellt. Dabei wird die Thematik in Bezug zu den Herausforderungen an die wissenschaftliche Gemeinschaft und das System Universität gesetzt sowie in einen historischen Kontext gestellt. In diesem Zusammenhang werden insbesondere die Diskrepanz zwischen der Idee der Öffnung von wissenschaftlicher Kommunikation und der wissenschaftlichen Realität adressiert, sowie Katalysatoren und Hindernisse für die Umsetzung der Konzepte rund um die Öffnung von Wissenschaft identifiziert und empirisch überprüft.

Die Erfahrungen und Meinungen der befragten Wissenschaftler und Wissenschaftlerinnen werden den Erfahrungen aus dem Selbstversuch des jederzeit öffentlich einsehbaren Erstellungsprozesses dieser Arbeit gegenübergestellt, die Unterschiede zwischen den Disziplinen herausgearbeitet und Handlungsempfehlungen für das offene Bearbeiten wissenschaftlicher Fragestellungen abgeleitet.

Abschließend werden die Ergebnisse zusammengefasst und in einem Ausblick Anknüpfungspunkte für weitere Forschungsbemühungen dargestellt.
\end{abstract}

\textbf{HINWEIS: KOMMENTARE SIND AUFGRUND DER PROMOTIONSORDNUNG NICHT ERLAUBT UND DEAKTIVERT! DIESE ARBEIT IST IN BEARBEITUNG (WORKING DRAFT) UND WURDE IM AUGUST 2014 AUS ANDEREN DOKUMENTEN ZUSAMMEGESETZT.  Weitere Informationen: \href{http://offene-doktorarbeit.de}{http://offene-doktorarbeit.de}}
