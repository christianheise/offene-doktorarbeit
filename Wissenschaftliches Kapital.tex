\subsubsection{Wissenschaftliches Kapital}
Im Rahmen der Betrachtung von Steuerungs- und Reputationsmethoden für Wissenschaft ist der Begriff wissenschaftliches Kapital als zentral zu sehen. Wissenschaftliches Kapital kann dabei als eine Ausprägung des kulturellen Kapitals und als symbolisches Kapital  verstanden werden. “Scientia donum dei est, unde vendi non potest", daß die Wissenschaft nicht verkauft werden kann, weil Wissen eine Gabe Gottes ist, sollen die Gelehrten des Mittelalters gerufen haben. Bereits damals war aber abzusehen, daß dieses Verständnis nicht lange halten würde. 
So basiert die “Gewährung wissenschaftlichen Kapitals” im wissenschaftlichen System heute auf einer engen Verbindung zwischen den Verlagen und den publizierenden Wissenschaftlern . Dabei steht die Wissenschaft in einer klaren Abhängigkeit zu den Verlagen. Ulrich Herb definiert in diesem Zusammenhang wissenschaftliches Kapital mit Hilfe Pierre Bourdieus als das “Ergebnis einer Investition (...), die sich auszahlen muss” ist. “Diejenigen, die diese Berechtigungsscheine in der Hand halten, verteidigen ihr 'Kapital' und ihre 'Profite', in dem sie diejenigen Institutionen verteidigen, die ihnen dieses 'Kapital' garantieren.”  Wissenschaftler sind also von einem Ressourcenzufluss abhängig um Ihre Aufgaben dauerhaft wahrzunehmen . Herb kommt zu dem Schluss, dass die Öffnung von Wissenschaft dabei bisher nicht wissenschaftlicher Logik folgt, “sondern einer feldunabhängigen Logik der Akkumulation von Kapital.” 
Hinzu kommt, dass, zum Beispiel bei dem Performanzindikator Drittmittel , in der Wissenschaft neben der Sicherung der Qualität von Forschung und Lehre zunehmend direkte finanzielle und administrative Kontrolle eine Rolle spielen . Daraus resultiert die Gefahr, dass nicht nur die Erwartungen an die Bewertung von Wissenschaft zu hoch gegriffen sind, sondern auch, dass sich Wissenschaft zu sehr an diesen Erwartungen orientiert und die Legitimität öffentlicher Ausgaben über den Zweck gestellt werden. Vor allem die Verknüpfung von  wissenschaftlicher Reputation und der damit einhergehenden Verteilung von Mittel und Stellen stellt eine Herausforderung an das Wissenschaftsystem, “dessen Währung [ursprünglich] nicht Geld ist” , dar. In diesem Teil der Arbeit soll das Konzept des wissenschaftlichen Kapitals vor dem Hintergrund des Widerspruchs gegenüber den Grundprinzipien der Wissenschaft erläutert und dargestellt werden, welche Möglichkeiten die Öffnung von Wissenschaft für die Legitimität öffentlicher Ausgaben darstellt, ohne das diese über den eigentlichen Zweck gestellt werden.
