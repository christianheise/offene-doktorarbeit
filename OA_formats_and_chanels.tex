\subsubsection{Open Access Kanäle und Formate}
In diesem Teil der Arbeit soll nach den unterschiedlichen Modellen in Bezug auf den Weg der Veröffentlichung von wissenschaftlichen Inhalten als Open Access Publikationen auch auf die unterschiedlichen Open Access Kanäle und Publikationsformaten eingegangen werden.

Dabei soll in folgende unterschiedliche unterschieden: Open Access Aggregatoren, Open Access Repositorien, Open Access Jounrals, Open Access Bücher. Sie alle beschäftigen sich entweder mit bestimmten Publikationsformen der wissenschaftlichen Kommunikation oder mit den Herausforderungen die im Rahmen der Distribution und Archivierung im Umfeld der neuen Möglichkeiten von Open Access entstanden sind. 

Da es eine enge Verknüpfung zwischen der Entwicklung von Repositorien und der Open-Access-Bewegung gibt\cite{offhaus_2012_institutionelle_repos}, soll hier auf die Rolle der Repositorien als Kanal für die Verbreitung von Publiaktionen eingegangen werden. Institutionelle Repositorien sind ein Instrument für wissenschaftliche Einrichtungen wie etwa Universitäten, um ihre Publikationen frei zugänglich zu machen\cite{dobratz_2007_open}.

Institutionelle Repositorien haben potenziell erhebliche Vorteile für die Institutionen, wenn sie in die Universität ganzheitlichen Rahmenbedingungen integriert sind\cite{steele_2006}. Diese Repositorien können auch für die Lernumgebungen und die Marketingaktivitäten einer Universität einen wichtige Rolle spielen, so können sie eine den Universitäten Output dokumentieren und den Zugang zu institutionellen Austausch verbessern\cite{steele_2006}. Ökonomisch rentieren sie sich vor allem dann, wenn skaleneffekte eintreten und in Verbünden agiert wird.\cite{blythe_2005value} Neben den institutionellen sind auch fachliche oder andere Arten von Repositorien eng mit der Open Access Bewegung verknüpft, sie werden in diesem Kapitel aber nicht weiter unterschieden.