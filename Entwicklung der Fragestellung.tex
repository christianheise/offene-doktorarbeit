\subsection{Entwicklung der Fragestellung}
Nach der theoretischen Diskussion sollen in diesem Teil die offenen inhaltliche Fragestellungen dargestellt werden.  Die Fragestellung dieser Arbeit soll dabei drei Kriterien entsprechen :
\begin{enumerate}
\item aus ihrer Formulierung soll klar hervorgehen wie sie verstanden werden kann
\item sie soll im Kontext der wissenschaftlichen Disziplin einen klaren definierten Ort haben 
\item ihr Gegenstand muss eindeutig sein.
\end{enumerate}
Die vorläufige forschungsleitende Hypothese in dieser Arbeit ist (siehe auch 3.1), dass Open Access sich in der Übergangsphase zu Open Science befindet. Die daraus ableitende Fragestellung umfasst dabei zum einen die theoretische Bedeutung von Offenheit im Rahmen des wissenschaftlichen Diskurs- und Machtbegriffs (Kapitel 4.2) aber auch die empirische Frage nach den Motiven und Beweggründen für Wissenschaftler, Verlagen und Universitäten diese Offenheit auf Wissenschaft in den unterschiedlichen Disziplinen zu ermöglichen oder zu verhindern (Kapitel 4.1). Abschließend soll erörtert werden, welche mögliche Auswirkungen auf Selbstverständnis von Wissenschaft, auf das wissenschaftliche Kapital sowie auf die unterschiedlichen Disziplinen durch diesen Prozess der Öffnung zu erwarten sind (Kapitel 5).
