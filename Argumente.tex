\subsection{Argumente für und gegen die Öffnung von Wissenschaft}

Die \textbf{Forderung nach Öffnung} adressiert mehrere Unzulänglichkeiten am bestehenden wissenschaftlichen Kommunikationssystem : 
1.	Transition-Argument
Die Nutzung der neuen Möglichkeiten für eine offene Wissensverbreitung neben den konventionellen Wegen der nicht-elektronischen Publikationen . Dabei gilt die Grundvoraussetzung der Aufbereitung des Wissens als strukturierte Daten zur Wissensweiterverwendung und -verarbeitung über alle Kanäle.
2.	Speed & Circulation-Argument
Wissensverbreitung wird künstlich durch Embargos und ineffiziente Validationssysteme zurückgehalten. Die Digitalsierung und Verbreitung über elektronische Kanäle stellt einen Vorteil für Wissensverbreitung und -verwertung dar. Wenn das Wissen schneller zur Verfügungsteht wird es schneller zirkulieren und effizienter genutzt werden können.
3.	Tax-Payer-Agrument
Durch Steuergelder finanzierte Forschung ist dem Steuerzahler im Rahmen konventioneller wissenschaftlicher Kommunikation nicht immer unentgeldlich zugänglich, obwohl er im Rahmen öffentlich-geförderter Forschungsprogramme die Forschung dahinter bereits finanziert  hat .
4.	Economic Promotion Argument
Bisher profitieren wirtschaftliche Unternehmungen nur unzureichend von staatlich-finanzierter wissenschaftlicher Kommunikation, dabei könnte eine schnellere, kommerziell verwertbare und umfassendere Bereitstellung der wissenschaftlichen Inhalte einen eklatanten Beitrag zur non-monetären Wirtschaftsförderung darstellen. Im Rahmen der offenen und schnelleren Verbreitung von wissenschaftlichen Informationen können neue Geschäftmodelle  entstehen.
5.	Digital Divide Argument
Der offene Zugang zu Publikationen ermöglicht neue Möglichkeiten für die Überwindung der sozialen, nationalen und globalen Wissenskluften  zwischen bildungsfernereren und -affineren Bevölkerungsteilen und -schichten der Welt .
6.	Validation & Reputation-Argument
Die Entwicklung neuer Verfahren, die die Aktivität und Qualität eines Forschers umfassender, transparenter und demokratischer messbar und kommunizierber machen, als im bestehenden Reputations- und Förderungssystem. Wissenschaftsevaluation wird durch Offenheit effizienter.
7.	Paradoxon of Information Argument
Überwindung des bestehenden Informationsparadoxons bei der Verbreitung und Vermarktung von wissenschaftlichen Inhalten. Hierbei handelt es sich um das Problem, dass es schwer ist eine Information kommerziell zu verwerten ohne zu viel über Inhalt und Qualität auszusagen. Eine Entkommerzialisierung des Vertriebs von Wissen  würde das Informationsparadoxon aufheben.
8.	Science communication Crisis-Argument
Durch die Öffnung von der wissenschaftlichen Kommunikations- und Reputationsprozesse besteht die Möglichkeit der vorherrschende Zeitschriften- und Monographienkrise durch neue Geschäftsmodelle zu begegnen.
9.	Interdicipline & International Exchange/Collaboration Argument
Die Globalisierung in der Wissenschaft führt immerstärker zu internationalem Austausch und zur internationalen Zusammenarbeit von Wissenschaftlern . Doch das gilt nicht nur für die grenzenüberschreitende Zusammenarbeit in Bezug auf die lokale Verortung sondern auch für die Interdisziplinarität der Forschungsvorhaben. Die Öffnung von Wissenschaft ermöglicht also auch Fächerfremden Wissenschaftlern Zugruff auf Publikationen und damit auf Wissensressourcen für die eigene Arbeit .
10.	Sustainable Access & Archiving Argument
Nur Offenheit im Sinne von Verwertbarkeit ermöglicht es in dezentralen Strukturen wie der des Internets alle Informationen nachhaltig und unabhängig voneinander zu speichern. Im Falle von Natur- oder anderen Katastrophen ermöglicht die digitale Ablage auf mehreren Kontinenten eine präservierung von Wissen undabhängig von lokalen Gegebenheiten oder Bedingungen.

\textbf{Demgegenüber} stehen aber auch Argumente gegen die Öffnung der wissenschaftlichen Prozesse und Publikationen:
1.	Quality-Argument
Die Befürchutung, das die Qualität auf Grund von schlechten oder nicht vorhandenen wissenschaftlichen Überprüfungsmechanismen leidet. Hauptargument ist das durch ein Autorengebühren finanziertes Publikationsmodell keinen klaren Anreiz für Ablehnung bietet.
2.	Archiving-Argument
Die Sicherstellung der Langzeitarchivierung und die Garantierung der langfristigen Auffindbarkeit sowie Bereitstellung der Dokumente kann im Auge der Kritiker von Offenheit in Wissenschaft und Forschung nicht durch alternative digitale Strukturen gewährleistet werden. 
3.	Authenticity-Argument
Forscherinnen und Forscher befürchten durch die dezentrale und offene Handhabung ihrer Texte und Arbeiten, dass diese im Zeitablauf inhaltlich nicht mehr unverändert zuordnenbar ihrem Autor sind.
4.	Rightsmanagement-Argument
Hierbei handelt es sich um die Verpflichtung für Mitarbeiter staatlich finanzierter Forschungsinsitutionen alle Texte elektronisch frei und offen zu publizieren. In dem 2009 veröffentlichten Heidelberger Appell kritisieren zahlreiche Autoren, Wissenschaftler, Verleger und Publizisten, dass das “verfassungsmäßig verbürgte Grundrecht von Urhebern auf freie und selbstbestimmte Publikation” … “derzeit massiven Angriffen ausgesetzt und nachhaltig bedroht” ist. Weiter sehen die Unterzeichner „weitreichende Eingriffe in die Presse- und Publikationsfreiheit, deren Folgen grundgesetzwidrig wären“  und die Befürchtung, dass die Freiheit von Forschung und Lehre gefährdet ist . 
5.	(Re-)Financing-Argument
Die unklare Refinanzierung der Öffnung von Wissenschaft ist eines der Kernargumente gegen das offene Publizieren von Arbeiten und Daten. Die Befürchtung ist, das ein solches System überhaupt nicht finanziert werden kann, konnte bisher nicht ausgeräumt werden.
6.	Sustainability-Argument
7.	Ressource-Allocation-Argument
Die Befürchtung, dass die Vergabe von Fördermittel und für die Karriere wichtige Aspekte der Reputationsbildung durch offenen System nicht Rechnung getragen wern kann ist eine weiteres Argument der Kritiker der Öffnung von Wissenschaft und Forschung. Eine Mittelvergabe zu gunsten populärer Forschung und damit eine Aushöhlung des wissenschaftlichen Systems in Ihrer Fächer und Facettenvielfalt wäre eine unmittelbare Folge dessen.
8.	Open-Caring-Argument
Wissenschaftlerinnen und Wissenschaftler fürchten durch den Zwang zu umfassenderen Bereitstellung von Publikationen und gegebenenfalls soagar Daten einen nicht unwesentlichen zeitlichen Mehraufwand für die Öffnung ihrer Arbeiten. Sie möchten aber möglichst wenig Zeit für die Veröffentlichung, Bereithaltung und Verbreitungung ihrer wissenschaftlichen Arbeiten aufbringen.
	Aufwand für Offenheit im Alltag des Wissenschaftlers
9.	Scientific-Freedom/Loss of Idea-Diversity-Argument
Angst dass durch Offenheit und Transparenz Forschungsförderung und Öffentlichkeit nur die wissenschaftlichen Projekte Fördern und unterstützen, die von der Öffentlichkeit verstanden werden. Darüber hinaus herrscht die Annahme, dass im Rahmen von zunehmender Kollaboration die Diversität von Projekten zu einem gleichen oder Ähnlichem Thema eingeschränkt wird.
10.	Interpretations-Argument
Eine der weiteren Ängste der wissenschaftlichen Community ist die Angst vor der Fehlinterpretation ihres kommunizierten Wissens sowie der Verlust der Kontrolle über die Informationen. Dabei steht vor allem die Befürchtung im Vordergrund, dass die offen veröffentlichten Arbeiten genutzt werden um die Arbeit zu miskreditieren oder gezielt zur Falschinfromation der Öffentlichkeit zu nutzen.

