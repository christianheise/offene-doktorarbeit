\subsection{Methode der Inhaltsanalyse}
Die unterschiedliche Verwendung der Begriffe Open Science und Open Access in der wissenschaftlichen Auseinandersetzung machen es notwendig die Begriffsbestimmungen für Open Science und Open Access im Rahmen dieser Arbeit vorzunehmen und zu konkretisieren. In Ergänzug zu der Literaturanalyse von Benedikt Fecher und Sascha Friesike für den Begriff "Open Science"\cite{cite:9} sowie der Litaraturanalyse von Giancarlo Frosio und Estelle Derclaye "Open Access Publishing" \cite{CREATe_2014} soll für diesen Zweck auch eine systematischen Literaturanalyse für die Begriffe "Open Access" und "Open Science" inklusive der Treiber und Bremser der Öffnung von Wissenschaft im Kontext des Begriffs "wissenschaftliche Reputation" durchgeführt werden. Neben der Berücksichtigung von Arbeiten aus den Medienwissenschaften im engeren Sinn sollen auch Arbeiten aus den Wirtschaftswissenschaften und den Kulturwissenschaften berücksichtigt werden.
\subsubsection{Forschungsfragen} 
Folgende Forschungsfragen sollen bei der Inhalsanalyse genauer analysiert werden:
\begin{itemize}
\item Warum kommt es zu der Bestrebungen hin zur Öffnung von Wissenschaft? 
\item Wie werden Open Science und Open Access definiert und voneinander abegrenzt? 
\item Welche Pro- und Contraargumente gibt es für die Öffnung von Wissenschaft - ist Offenheit in der Wissenschaft gut oder schlecht? 
\item Wo sind die Grenzen der Öffnung? 
\item Warum ist die Öffnung von Wissen in den verschiedenen wissenschaftlichen Disziplinen unterschiedlich weit verbreitet? 
\item Was bedeutet Offenheit und freier Zugang im Rahmen des wissenschaftlichen Diskurs-, Reputations- und Machtbegriffs?
\end{itemize}	

\subsubsection{Erhebungsmethode und Umfang} 
tbd

\subsubsection{Analyse der Definitionen von Open Access} 
tbd