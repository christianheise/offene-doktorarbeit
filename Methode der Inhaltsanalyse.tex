\subsection{Methode der Inhaltsanalyse}
Die unterschiedliche Verwendung der Begriffe Open Science und Open Access in der wissenschaftlichen Auseinandersetzung machen es notwendig die Begriffsbestimmungen für Open Science und Open Access im Rahmen dieser Arbeit vorzunehmen und zu konkretisieren. Dazu soll eine systematischen Literaturanalyse durchgeführt werden in der neben Arbeiten aus den Medienwissenschaften im engeren Sinn auch Arbeiten aus den Wirtschaftswissenschaften und den Kulturwissenschaften  berücksichtigt werden.
Es wurde jeweils eine xxx Literaturanalyse hinsichtlich der Begriffsbestimmung von Open Science und Open Access mit dem Fokus auf die Differenzierung zwischen den verschiedenen wissenschaftlichen Disziplinen und dem anstehenden Paradigmenwechsel bei der Betrachtung der wissenschaftlichen Reputation durchgeführt. Eine dritte Literaturanalyse befasste sich mit dem Stand der Forschung bezüglich den Treibern und Bremsern der Öffnung von Wissenschaft.
\subsubsection{Forschungsfragen} 
Folgende Forschungsfragen sollen bei der Inhalsanalyse genauer analysiert werden:
\begin{itemize}
\item Warum kommt es zu der Bestrebungen hin zur Öffnung von Wissenschaft? 
\item Wie werden Open Science und Open Access definiert und voneinander abegrenzt? 
\item Welche Pro- und Contraargumente gibt es für die Öffnung von Wissenschaft - ist Offenheit in der Wissenschaft gut oder schlecht? 
\item Wo sind die Grenzen der Öffnung? 
\item Warum ist die Öffnung von Wissen in den verschiedenen wissenschaftlichen Disziplinen unterschiedlich weit verbreitet? 
\item Was bedeutet Offenheit und freier Zugang im Rahmen des wissenschaftlichen Diskurs-, Reputations- und Machtbegriffs?
\end{itemize}	

\subsubsection{Erhebungsmethode und Umfang} 
tbd
\subsubsection{Analyse der Definitionen } 
tbd