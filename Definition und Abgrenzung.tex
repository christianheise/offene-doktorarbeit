\chapter{Theoretischer Bezugsrahmen, Definition und Abgrenzung } 

Der theoretische Bezugsrahmen auf Grundlage von wissenschaftlich gesicherten Modellen, Theorien und Ansätzen ermöglicht es Erklärungen und Handlungsempfehlungen abzuleiten\cite{martin_2007_wissenschaftstheorie}. Er trägt dazu bei, die Fragestellungen in einen Zusammenhang zu stellen, legitimiert die Erforschung dieser Fragen und bildet den Rahmen für die Auswertung der gesammelten Erkenntisse \cite{suchen}. Dabei stellen die theoretischen Vorannahmen die Ausgangslage für die experimentellen und empirischen Betrachtungen in dieser Arbeit dar und werden im Rahmen einer Umfrage, der Literaturanalyse sowie der eigenen experimentellen Arbeit an der wissenschaftlichen Praxis geprüft. Dabei sollen die Konzepte von Open Access und Open Science den empirische Beobachtungen zugeordnet werden können. Ziel ist es, letztlich zu einem vertieften theoretischen Verständnis der gesammelten empirischen und experimentellen Ergebnisse zu gelangen. 

Open Science und Open Access im Kontext von wissenschaftlicher Reputation werden in dieser Arbeit in technischen als auch in ihren gesellschaftlichen und politischen Aspekten sowie die kulturellen Auswirkungen der Medienbrüchen im Rahmen von wissenschaftlichen Publizieren auf theoretischem Niveau reflektiert. Die Analysen in dieser Arbeit werden dabei aus den Perspektive des Produzenten (Wissenschaftler als Autoren) und auch aus der damit nicht immer harmonisierenden Perspektive des Rezipienten beziehungsweise Medienkonsumenten (Wissenschaftler als Leser) stattfinden. In diesem Zusammenhang wird adressiert inwiefern Macht, regulierende Prinzipien wie die Verknappung sowie die Ein- und Ausgrenzung im Rahmen von wissenschaftlichen Diskursen, nach dem Diskurs- und Machtbegriff von Michel Foucault, mit den Modellen der Open Access, Open Science und wissenschaftliche Reputation in der wissenschaftlichen Kommunikation vereinbar sind oder diesen diametral dem gegenüberstehen. 

Wenn es im aktuellen öffentlichen Diskurs um wissenschaftliche Informationen, Infrastruktur und Arbeiten geht, werden immer öfter Schlagworte mit dem Attribut „Open“, wie Open Access, Open Research und Open Science, verwendet \cite{bunz_2014} \cite{schulze_2013_open}. "Offen" bezieht sich dabei üblicherweise auf zwei Kernaspekte: Zum einen die Offenheit des Zugangs zu Daten, Code oder Ergebnissen und zum anderen auf das Gebot der Transparenz, also die Offenlegung beziehungsweise der Zugriff auf Verfahren, Methoden und Ziele \cite{schulze_2013_open}.

In der Literatur sind die Begriffe „Open Access“ und „Open Science“ nicht allgemeingültig definiert \cite{suchen}. Sie finden in der wissenschaftlichen Auseinandersetzung auf unterschiedlichste Art und Weise Verwendung \cite{cite:9}. In diesem Kapitel wird der historische und gesellschaftliche Kontext ihrer Anwendung dargestellt und mittels der Analyse wissenschaftlicher Literatur abgegrenzt. Es wird angestrebt, zu erläutern, welche Bedeutung sie in Forschung, Gesellschaft und Politik haben. Die Begriffe werden in ihrer Entwicklung dargestellt. Um ein möglichst umfassendes Bild zu erhalten, wird "Entwicklung" hier in den drei Dimensionen erfasst: erstens, als "analytische Kategorie", zweitens als "Forschungsgegenstand" und drittens als "politische Praxis in der moralischen Auseinandersetzung über die Wünschbarkeit von Zuständen". \cite{cite:10} Ziel ist es hierbei nicht, gemeingültige Definitionen zu erarbeiten, sondern die Darstellung dieser Begriffflichkeiten in verschiedenen Diskursen. 

Im Rahmen dieses Kapitels wird darüber hinaus auch adressiert inwiefern Macht, regulierende Prinzipien wie Verknappung oder Ein- und Ausgrenzung in wissenschaftliche Diskurse, nach dem Diskurs- und Machtbegriff von Michel Foucault, mit den Modellen der offenen Initativen in der wissenschaftlichen Kommunikation vereinbar sind oder ob sie denen gegenüberstehen.

Die Unterscheidung "Zugang" und "Zugriff" ist wesentlich in dieser Arbeit und stellt eine wesentliche Grundlage für die Definition und Abgrenzung der Begriffe "Open Access" und "Open Science" dar. "Zugang" bezieht sich in diesem Zusammenhang auf einen unbeschränkten Zugang zur finalen wissenschaftliche Publikation. "Unbeschränkt" meint hier: lesen\cite{cite:9a}, Verarbeitung und Weiternutzung. "Zugriff" soll als erweiterte Nutzung der jeweiligen Wissensressourcen verstanden werden. Eingeschlossen sind neben dem "Zugang" zur Publikation sämtliche Informationen und Daten sowie die komplette Kommunikation hinter der finalen Veröffentlichung \cite{cite:9b}. "Zugriff" beschränkt sich hier also nicht nur auf den reinen Zugang zu wissenschaftlicher Information im Rahmen des Publikationsprozesses, sondern schließt auch den Zugriff auf sämtliche Forschungsdaten, Methoden und alle weiteren Informationen, die während der wissenschaftliche Arbeit auf dem Weg zur finalen Publikation entstehen \cite{cite:9c}, ein.

Die Themenbereiche Social Media in Wissenschaft und Forschung, Citizen Science und aktuelle Diskurse zu Tools und Diensten werden in dieser Arbeit bewusst ausgelassen und nur am Rande, beziehungsweise nur wenn sie die Beantwortung der Forschungsfragen tangieren, eingeschlossen.

\section{Open Access} 

\begin{quote}
„Open Access“ meint, dass [= Peer-Review-Fachliteratur] kostenfrei und öffentlich im Internet zugänglich sein sollte, sodass Interessenten die Volltexte lesen, herunterladen, kopieren, verteilen, drucken, in ihnen suchen, auf sie verweisen und sie auch sonst auf jede denkbare legale Weise benutzen können, ohne finanzielle, gesetzliche oder technische Barrieren jenseits von denen, die mit dem Internet-Zugang selbst verbunden sind. In allen Fragen des Wiederabdrucks und der Verteilung und in allen Fragen des Copyrights überhaupt sollte die einzige Einschränkung darin bestehen, den Autoren Kontrolle über ihre Arbeit zu belassen und deren Recht zu sichern, dass ihre Arbeit angemessen anerkannt und zitiert wird.
\cite{boai_2012}
\end{quote}
Der Fortschritt der Wissenschaft hängt auch, und maßgeblich von dem freien Austausch und der Verbreitung von Informationen ab \cite{cite:11}. Das System der wissenschaftlichen Kommunikation, das so seit mehreren hundert Jahren besteht, basierte auf Forschung, der Begutachtung, dem Druck sowie der Kommunikation der Egebnisse in wissenschaftlichen Publikationen, der Verbreitung sowie dem Verkauf an Bibliotheken und andere wissenschaftliche Institutionen gegen Kosten \cite{cite:11a} und dem anschließenden Diskurs in der wissenschaftlichen Fachöffentlichkeit \cite{suchen}. Der offene Zugang zu wissenschaftlicher Kommunikation ist seit der Entwicklung des gedruckten Wortes aber auch eng mit der Frage nach Urheberrechten für wissenschaftliche Informationen verknüpft \cite{Case_2000}. Open Access beschreibt in ein wissenschaftliches Kommunikationssystem, in dem der Zugriff auf die unterschiedlichsten Formen wissenschaftlicher Publikationen, im Gegensatz zum bestehenden System, unter freien, kostenlosen Bedinungen und ohne finanzielle, gesetzliche oder technische Barrieren (Online) möglich ist \cite{WD_bundestag_2009}, das aber auch ein "alternatives Geschäftsmodell"\cite{lewis_2012_inevitability} für wissenschaftliche Publikationen ermöglicht. Das beruht auf der Maßgabe, dass der Autor die "Eigentumsrechte an den Artikeln, die bisher für die Publikation in wissenschaftlichen Journals an die jeweiligen Fachverlage abgetreten wurden, (...) nun bei den Autoren der Artikel selbst verbleiben" \cite{Hess_2006}. 

Durch den weltweit steigenden Haushaltsdruck an Bibliotheken und wissenschaftlichen Insitutionen, dem “ungewöhnlichen Geschäftsmodell” \cite{cite:12} der Wissenschaftsverlage mit immer höheren Margen \cite{albert_2006_open_implications} und dem Umstand, dass private Wissenschaftsverlage durch das wissenschaftlichen Reputationssystem über öffentlich finanzierte Wissenschaftlerkarrieren entscheiden \cite{heise_2012}, befindet sich das System in einer Krise\cite{cite:14}. Open Access beschäftigt sich in diesem Rahmen mit der Öffnung (Open) und dem freien Zugang (Access) zu den wissenschaftlichen Publikationen. "Geringere Kostenbarrieren und damit eine einfachere Verbreitung ihrer eigenen Werke" \cite{WD_bundestag_2009} stellen dabei die Wünsche der wissenschaftlichen Autoren und Urheber an Open Access dar und der Einsatz (offener) Lizenzen ist dafür einer der Haupteinflussfaktoren \cite{cite:16}. Es ist derzeit üblich, Open Access in drei Modelle einzuteilen \cite{suchen}: Green Open Access, Golden Open Access und andere (Misch-)Formen.

\subsection{Offener Zugang zu wissenschaftlicher Kommunikation}

Der bisherige Prozess wissenschaftlicher Kommunikation steht vor großen Herausforderungen. Die Zeitschriften- und Monographienkrise, der zunehmende finanzielle Druck sowie die Veränderungen im wissenschaftlichen Kommunikationsprozess durch neue Arten und Möglichkeiten der Distribution, die steigenden Beschaffungskosten für wissenschaftliche Literatur \cite{cite:17}, sowie die Veränderungen in der Rezeption von Inhalten \cite{holub_2013_reception}, zwingen zum Umdenken in der wissenschaftlichen Kommuinkationspraxis \cite{suchen}. Nachfolgend wird das Modell des Offenen Zugangs zu Wissenschaft erläutert, analysiert und abgerenzt.

Der Schwerpunkt beruht dabei auf den Themenbereichen wissenschaftliche Reputation und (Effizienz der) Kommunikation. Dieser Zugang beruht auf der Annahme, dass Offenheit eine große Chance für dringend notwendige Veränderungen im wissenschaftlichen Qualitäts- und Reputationssystem (siehe Kapitel 2.3) darstellt. Das betrifft vor allem die Aktivität der Wissenschaftler und die Qualtiät der Forschungergebnisse, deren Erkenntnisse bisher häufig erst nach langen intransparenten Verfahren bewertet und publiziert, sowie nur an einen beschränkten Kreis von Rezipienten vermittelt werden. Das hat auch einen signifikanten Einfluss auf Kosten die im Rahmen öffentlich-finanzierter Forschung entstehen \cite{suchen}. Ein besonderer Fokus liegt auf dem generellen Zugang zu wissenschaftlichen Informationen im Rahmen des "klassischen" Kommuniktations- und Publikationsprozess. Im Rahmen von Open Access ist dabei nicht zwingend der Zugriff auf Informationen oder Daten, die bei Erstellung der Publikation entstehen, eingeschlossen. 

Als Grundlage für diese Entwicklung werden vor allem die infrastrukturellen Veränderungen angeführt, die "seit spätestens Mitte der 1990er-Jahre entscheidend Einfluss auch auf die Wissenschaftskommunikation und das wissenschaftliche Arbeiten genommen haben" \cite{schulze_2013_open}. Wissenschaftliche Informationen werden seither nicht nur in "digitaler Form konsumiert, sondern auch kollaborativ und kooperativ, zeitlich versetzt, durch teilweise räumlich weit verstreute Arbeitsgruppen und Forschungsverbünde, genutzt und weiterverarbeitet" \cite{schulze_2013_open}. Die Verbreitung und Akzeptanz von Open Access variiert zwischen den einzelnen wissenschaftlichen Disziplinen erheblich \cite{cite:21a} .

\subsection{Chronologie der Bewegung}
Um Open Access einordnen zu können, ist eine historische Betrachtung der Entwicklung wissenschaftlicher Kommunikation, aber auch der Forderung nach Offenheit in eben dieser unabdingbar. 

Schon im antiken Griechenland, und in vielen anderen pre-modernen Zivilisationen, wurden Wissen und Informationen als nicht besitzbare Ware angesehen\cite{cite:18}. Dennoch war der Austausch im Vergleich zu den heutigen Möglichkeiten stark beschränkt \cite{cite:17c}.

- Gutenberg Auswirkung des Buchdrucks auf die Rolle der Univerisäten + Der Prioritätenstreit der Differentialrechnung zwischen Newton und Leibniz -

In Deutschland nahmen bis Anfang der 1990er Jahre die wissenschaftlichen Verlage eine marktbeherrschende Stellung ein und waren exklusiver Dienstleister bei der Veröffentlichung wissenschaftlicher Informationen \cite{schloegl_2005} \cite{offhaus_2012_institutionelle_repos}. Die Entwicklung basiert auf dem in der Welt des geistigen Eigentums ungewöhnlichen Umstand, dass seit dem Beginn des wissenschaftlichen Journals im Jahr 1665, wissenschaftliche Autoren nicht finanzielle Belohnung profitierten sondern durch die weite Verbreitung und Hinweise auf ihre Arbeit sowie die Infromationen dahinter \cite{albert_2006_open_implications}. Darüber hinaus beruht das System auf der Eigenheit, dass Wissenschaftler sowohl Produzenten als auch Konsumenten der Wissenschaftskommunikation sind und damit Ihre eigene Zielgruppe darstellen \cite{Hess_2006}.

Die Vormachtstellung der Verlage im wissenschaftlichen Publikationssystem stüzt sich auf drei Säulen \cite{offhaus_2012_institutionelle_repos} \cite{bargheer_2006_open}: 
\begin{enumerate}
\item "Urheberrecht, wonach Verlage [...] weitgehende Ansprüche an dem veröffentlichten Werk erwerben“;
\item "redaktionelle Themenbündelung (bundling)“;
\item "Qualitätssicherung durch Begutachtung (Peer Review)"
\end{enumerate}

Durch diese marktbeherschenden Stellung, waren die Verlage in der Lage drastische Preiserhöhungen durchzusetzen und es kam kurz vor der Jahrtausendwende zur sogenannten "Zeitschriftenkrise" \cite{suchen}. 
- Zeitschriftenkrise beschreiben -

Gleichzeitig standen die Wissenschaftler unter einem starken Publikationszwang, der mit "Publish or Perish" \cite{CLAPHAM_2005} beziehungsweise "impact factor fever" \cite{Cherubini_2008} und "impact factor race" \cite{Brischoux_2009} beschrieben wurde \cite{offhaus_2012_institutionelle_repos}. 

Bereits Anfang der 1990er gründete der Physiker Paul Ginsparg mit arXiv den ersten wissenschaftliche Preprint-Dienst des Internets \cite{suchen}, der es Wissenschaftlern ermöglichen sollte Ideen vor der gedrukten Veröffentlichung zu teilen. Vier Jahre später forderte Steven Harnad die wissenschaftliche Community dazu auf sofort mit der digitalen Selbstarchivierung und öffentlichen Zurverfügungstellung ihrer Beiträge zu beginnen \cite{albert_2006_open_implications}, um "den Barrieren, die zwischen ihrer Arbeit und ihrer (kleinen) Leserschaft aufgestellt werden, zu entkommen" \cite{harnad_1995_subversive_proposal}. 1998 wurde mit der Scholarly Publishing and Academic Resources Coalition (SPARC) einer der späteren "major player" der Open Access Bewegung\cite{russell2008business} gegründet. Als Konsequenz aus der Zeitschriftenkrise sollte diese Allianz zwischen Universitäten und wissenschaftlichen Bibliotheken dafür sorgen, dass die Kosten für wissenschaftliche Zeitschriften reduziert oder durch die Bereitstellung kostengünstiger oder freier, nicht-kommerzieller, Peer-Review-Fachzeitschriften ersetzt werden. Durch Weiterbildung, politische Arbeit und der Förderung alternativer Geschäftsmodelle wurde angestrebt, Initiativen des offenen wissenschaftlichen Publizierens zu stimulieren \cite{suchen}.

In 2001 erschien Open Access erstmals als eigenes und öffentlichkeitswirksames Thema im wissenschaftlichen Diskurs \cite{cite:19}. Die Public Library of Science (PLoS), gegründet im Oktober 2000, foderte Wissenschaftler in einem offenen Brief im Mai 2001 dazu auf, ab September 2001 nur noch in den Zeitschriften zu veröffentlichen, beziehungsweise nur noch die Zeitschriften zu reviewen, zu editieren und zu abonnieren, deren Beiträge spätestens sechs Monate nach ihrer Erstveröffentlichung für jedermann im Internet kostenlos und unentgeltlich einsehbar sind \cite{cite:20}. Schon nach kurzer Zeit unterzeichneten (nach eigenen Angaben \cite{cite:19a}) rund 38.000 Wissenschaftler aus 180 Nationen das Schreiben. Dieser Brief kann als Auftakt zu einem 20-monatigen theoretischen Schub gesehen werden. In dem Zeitraum entstanden drei der bis heute wichtigsten Erklärungen im Bereich der Öffnung des Zugangs zu wissenschaftlicher Kommunikation \cite{CREATe_2014}: 

\begin{enumerate}
\item Erklärung der Budapest Open Access Initiative (Dezember 2001)

Im gleichen Jahr wie der PLoS-Brief, wurden im Rahmen einer Konferenz des Open Society Institutes in Budapest, mit der “Budapest Open Access Initative” (BOAI)\cite{boai_2012} erstmals die Bemühungen um Open Access in einer eigenen Erklärung zusammengefasst\cite{cite:21a}. Im Fokus dieser steht die Forderung nach dem freien Zugang zu wissenschaftlichen Publikationen. In der BOAI wird erstmals manifestiert, dass wissenschaftliche Peer-Review-Fachliteratur “kostenfrei und öffentlich im Internet zugänglich sein sollte, so dass Interessenten die Volltexte lesen, herunterladen, kopieren, verteilen, drucken, in ihnen suchen, auf sie verweisen und sie auch sonst auf jede denkbare legale Weise benutzen können, ohne finanzielle, gesetzliche oder technische Barrieren jenseits von denen, die mit dem Internet-Zugang selbst verbunden sind. In allen Fragen des Wiederabdrucks und der Verteilung und in allen Fragen des Copyrights überhaupt sollte die einzige Einschränkung darin bestehen, den Autoren Kontrolle über ihre Arbeit zu belassen und deren Recht zu sichern, dass ihre Arbeit angemessen anerkannt und zitiert wird."\cite{boai_2012} 

Anlässlich des zehnten Jahrestages der BOAI, wurde von der Open Society Foundation mit der BOAI 10 die usrprüngliche Erklärung bestärkt und anhand von weitere Richtlinien und Empfehlungen die Entwicklungen und Herausforderungen in seiner zehnjährigem Bestehen adressiert. Die Initiatoren kommen noch nach diesem Zeitraum unverändert zu dem Schluss, dass "noch immer Zugangsbeschränkungen zu Peer-Review-Forschungsliteratur, meist eher zugunsten der Verlage, als zugunsten der Autoren, Reviewer oder Redakteure und damit auch auf Kosten der Forschung, Forscher und Forschungseinrichtungen" \cite{boai_2012} bestehen. "Nichts aus den letzten zehn Jahren" lässt "darauf schließen, dass das ursprüngliche Ziel von OA weniger sinnvoll oder erstrebenswert erscheint. Im Gegenteil, die Notwendigkeit, dass Wissen für jeden, der es nutzen, anwenden oder darauf aufbauen kann, offen verfügbar sein sollte, ist dringlicher als je zuvor" \cite{boai_2012}.

\item Die Bethesda Erklärung (Juni 2003)

Zwei Jahre nach Veröffentlichung der initalen Version der BOAI-Erklärung, im Juni 2003, verabschiedete im US-Bundesstaat Maryland eine Gruppe von Forschungsförderern, wissenschaftlicher Gesellschaften, Verleger, Bibliothekare, Forschungseinrichtungen und einzelnen Wissenschaftler das "Bethesda Statement on Open Access Publishing".\cite{suchen} Ziel der Erklärung war die Stimulation der Diskussion in der biomedizinischen Forschung, "wie man schnellstmöglich den offenen Zugang zu der primären wissenschaftlichen Literatur in der Biomedizin erreichen könnte"\cite{suchen}. Wie bereits in der BOAI erklärten die Autoren des "Bethesda Statement on Open Access Publishing" Bedingungen an den offenen Zugang zu wissenschaftlichen Publikationen \cite{suchen}: Erstens werden Autor(en) und Urheberrechts-Inhaber aufgefordert für alle Benutzer eine freies, unwiderrufliches, weltweites und unbefristetes Recht auf den Zugang zulassen, sowie eine Lizenz zu verwenden, die das Kopieren, Nutzen, Verbreiten, Übertragen und öffentliche darstellen der Publikation ermöglichen. Darüber hinaus muss es erlaubt sein abgeleitete Werke zu verteilen, in jedem digitalen Medium für jeden Zweck zu veröffentlichen, vorbehaltlich einer angemessenen Zuordnung der Urheberschaft. Das beinhaltet auch das das Recht auf eine kleine Anzahl von gedruckten Kopien für den persönlichen Gebrauch zu machen. Zweitens, muss eine vollständige Version der Arbeit und allen ergänzenden Materialien, einschließlich einer Kopie der Genehmigung, wie oben erwähnt, in einem geeigneten elektronischen Standardformat sofort bei der ersten Veröffentlichung in mindestens einem Online-Repositorium, das von einer wissenschaftlichen Einrichtung unterstützt wird hinterlegt werden. Dieses Repositorium muss von einer wissenschaftlichen Gesellschaft, Regierungsbehörde oder einer anderen etablierten Organisation akzeptiert und die sich zu einen offenen Zugang, uneingeschränkte Verbreitung sowie Interoperabilität und Langzeitarchivierung (für die biomedizinischen Wissenschaften, PubMed Central ist ein solches Repository) einsetzen.

\item \textbf{Die Berliner Erklärung} (Oktober 2003)

Ein weiterer Meilenstein für die Verbreitung von Open Access auf dem europäischen Kontinent waren die "Berlin Konferenzen"\cite{CREATe_2014}. Die erste Tagung wurde 2003 von der Max-Planck-Gesellschaft und dem Projekt European Cultural Heritage Online (ECHO) organisiert, um über "Zugangsmöglichkeiten zu Forschungsergebnissen" zu diskutieren. In diesem Rahmen entstand auch die "Berliner Erklärung über den offenen Zugang zu wissenschaftlichem Wissen"\cite{berliner_erklaerung_2003}, in der die Verfasser über die Budapester Erklärung hinaus gehen und neben dem kostenlosen und freien Zugang zu wissenschaftlichem Wissen in Form von Publikationen auch den freien und offenen Zugang zu den Daten fordern: „Open Access-Veröffentlichungen umfassen originäre wissenschaftliche Forschungsergebnisse ebenso wie Ursprungsdaten, Metadaten, Quellenmaterial, digitale Darstellungen von Bild- und Graphik-Material und wissenschaftliches Material in multimedialer Form.“\cite{berliner_erklaerung_2003} Sie symbolisiert damit auch ein erweitertes Verständnis von Open Access und bildet die Grundlage für ein erstes Ansatzpunkt zur Definition von Open Science, konzentriert sich aber dennoch ausschließlich auf den abgeschlossenen wissenschaftlichen Prozess\cite{suchen}.

\end{enumerate}

Alle drei Erklärungen, auch die "three B's"\cite{suber_2004_praising_oa}genannt, gelten als die angesehensten Definitionen von Open Access und sind in wesentlichen Merkmalen in sich stimmig\cite{albert_2006_open_implications}.

Die Debatte über die Zukunft des wissenschaftliche Publizierens und Kommunizierens neigt dazu, Konzepte um offene Wissenschaft als einen bisher beispiellosen und noch nie dagewesenen Paradigmenwechsel dar zu stellen \cite{cite:17a} \cite{cite:17b}. Die Geschichte von Open Access ist eine Geschichte, die eng mit der Digitalisierung von Kommunikationsprozessen verknüpft ist \cite{albert_2006_open_implications}. Open Access ist dabei kein Selbstzweck\cite{cite:17d}, sondern ein Symptom für tiefergehende Prozesse die mit der wachsenden Bedeutung der Digitalisierung in unserer Zivilisation sowie die damit einhergehenden Möglichkeiten für tiefgreifende Veränderungen im Machtgefüge zusammenhängen\cite{cite:17e}. Denn obwohl es vorher schon vereinzelte Versuche in der Wissenschaft gab komplett Informationen und Publikationen offen und frei zu kommunizieren war Open Access im Printzeitalter physisch und ökonomisch über lokale Grenzen hinaus schier unmöglich \cite{cite:18a}. Trotz dieser Grenzen gehen die ersten Experimente mit offenem Zugang und freien Lizenzen für Publikationen in der Wissenschaft bis in die 60er Jahre des vorherigen Jahrhunderts und somit schon vor der Zeit der Erfindung des Internets, zurück \cite{cite:18b}. 

\subsection{Open Access Modelle}

In der einschlägigen Literatur wird in viele unterschiedliche Formen von Open Access unterteilt und es existieren mehrere Definitionen\cite{CREATe_2014}\cite{albert_2006_open_implications}, sowie unterschiedliche Auffassungen über die verschiedenen Modelle von Open Access\cite{CREATe_2014}\cite{cite:22b}\cite{lewis_2012_inevitability}. Als Grundlage für diese generelle Unterteilung gelten die "three Bs" (siehe Kapitel X), die als angesehensten Definitionen von Open Access gelten. Am Beispiel der Budapest Open Access Initiative werden zwei Wege für Open Access artikuliert\cite{albert_2006_open_implications}: 
\begin{enumerate}
\item Einrichtung von "einer neuen Generation von Fachzeitschriften," die einen kostenfreien und unmittelbaren Zugang zu den Beiträgen ermöglichen (als "goldener" Weg bekannt)
\item öffentlich zugängliche (Selbst-)Archivierung durch den Urheber (als "grüner" Weg bekannt)
\end{enumerate}

Eine zweite Ebene der Unterteilung in hybride, radikale und sonstige Formen von Open Access soll allen weiteren, in der Literatur aufkommenden Formen, gerecht werden. Abschliessend werden auch die Formen genannt, die zwar häufig als Open Access bezeichnet werden, aber den gängigen Deklarationen \cite{boai_2012} und Definitionen nicht gerecht werden. Open Access-Publikationen erheben in der Regel vom Autor Veröffentlichungsgebühr und verzichten nicht auf Peer-Review, um die akademische Reputation zu bewahren \cite{albert_2006_open_implications} \cite{Open_Access_net_2009}.

Der Umstand, dass eine eindeutige Klassifizierung dennoch schwer möglich ist, kann damit begründet werden, dass es "keine formelle Struktur, keine offizelle Organisation und kein ernannter Führer" gibt, der die Open Access Bewegung antreibt\cite{poynder_2011_suber}.

Der grüne Weg beschreibt das Modell, in dem der Autor

Der goldene Weg hingegen stellt die Publikation unmittelbar nach Fertigstellung zur Verfügung. Hierbei gibt es auch die Unterscheidung, dass einige Verlage die Publikation mit Verzögerung zur Verfügung stellen, in der Literatur wird in diesem Zusammenhang von verzögertem goldenen Open Access gesprochen\cite{lewis_2012_inevitability}. Im Rahmen anderer Modelle, vornehmlich bei der Publikation in Zeitschriften, wird dem Autor die Möglichkeit eingeräumt durch zusätzliche Zahlung die Publikation offen und frei zur Verfügung zu stellen\cite{lewis_2012_inevitability}.

Als Kernunterschied zwischen den beiden Modellen kann hervorgehoben werden, dass die grüne und hybride Form sowie das verzögerte Open Access, das klassische Geschäftsmodell der Verlage nicht beeinträchtigt, währen der goldene Weg auch ohne das bisherige Geschäftsmodell des Verlags auskommen kann\cite{lewis_2012_inevitability}.

Weitere, aber wenig genutzte Modelle sind

Hybride Modelle

Open Choice \cite{Hess_2006} 

\subsection{Open Access Kanäle und Formate}
In diesem Teil der Arbeit soll nach den unterschiedlichen Modellen in Bezug auf den Weg der Veröffentlichung von wissenschaftlichen Inhalten als Open Access Publikationen auch auf die unterschiedlichen Open Access Kanäle und Publikationsformaten eingegangen werden.

Dabei soll in folgende unterschiedliche unterschieden: Open Access Aggregatoren, Open Access Repositorien, Open Access Jounrals, Open Access Bücher. Sie alle beschäftigen sich entweder mit bestimmten Publikationsformen der wissenschaftlichen Kommunikation oder mit den Herausforderungen die im Rahmen der Distribution und Archivierung im Umfeld der neuen Möglichkeiten von Open Access entstanden sind. 

Da es eine enge Verknüpfung zwischen der Entwicklung von Repositorien und der Open-Access-Bewegung gibt\cite{offhaus_2012_institutionelle_repos}, soll hier auf die Rolle der Repositorien als Kanal für die Verbreitung von Publiaktionen eingegangen werden. Institutionelle Repositorien sind ein Instrument für wissenschaftliche Einrichtungen wie etwa Universitäten, um ihre Publikationen frei zugänglich zu machen\cite{dobratz_2007_open}.

Institutionelle Repositorien haben potenziell erhebliche Vorteile für die Institutionen, wenn sie in die Universität ganzheitlichen Rahmenbedingungen integriert sind\cite{steele_2006}. Diese Repositorien können auch für die Lernumgebungen und die Marketingaktivitäten einer Universität einen wichtige Rolle spielen, so können sie eine den Universitäten Output dokumentieren und den Zugang zu institutionellen Austausch verbessern\cite{steele_2006}. Ökonomisch rentieren sie sich vor allem dann, wenn skaleneffekte eintreten und in Verbünden agiert wird.\cite{blythe_2005value} Neben den institutionellen sind auch fachliche oder andere Arten von Repositorien eng mit der Open Access Bewegung verknüpft, sie werden in diesem Kapitel aber nicht weiter unterschieden.

\section{Open Science}
Der Sammelbegriff Open Science geht über die Idee vom offenen Zugang (Open Access) zur Publikationen von wissenschaftlicher Forschung hinaus und beschäftigt sich mit dem offenen Zugriff auf die gesamte wissenschaftliche Wertschöpfungskette inklusive dem ganzen wissenschaftlichen Prozess. Open Science ist als Evolutionsschritt von Open Access zu verstehen. 

Ähnlich wie bei dem Konzept von Open Access erhoffen sich die Befürworter dabei erstmal grundsätzlich einen einfacheren und breiteren Weg Wissenschaft zu kommunizieren. Grundlegend lässt sich das Konzept von Open Science als die Forderung nach dem offenen Zugriff auf die öffentlich finanzierter Forschung verstehen. Das Konzepz folgt damit der Annahme, "dass sich die Bedeutung von Forschungsergebnissen zukünftig nicht mehr auf sog. klassische wissenschaftliche Publikationen (im Format von Einleitung –Methoden – Ergebnisse – Diskussion), sondern die globale Echtzeitpublikation von Originaldaten stützen wird"\cite{Stengel_2013}.

Wie bei den Entwicklungen rund um Open Access, kann grundsätzlich in zwei Strategien für die Etablierung von Offenheit unterscheiden werden \cite{schulze_2013_open}: 
\begin{enumerate}
\item "Top-down durch Förderstrategien, Vorgaben und Empfehlungen"
\item "Bottom-up durch Graswurzelprojekte und den Einsatz von Evangelists"
\end{enumerate} In allen Fällen steht und fällt der Erfolg damit, ob sich der jeweiligen Zielgruppe ein unmittelbarer Mehrwert und Nutzen erschließt \cite{schulze_2013_open}.


In diesem Kapitel soll Open Science (medien)kulturwissenschaftlich in ihren technischen als auch in ihren gesellschaftlichen und politischen Aspekten sowie die kulturellen Auswirkungen der Medienbrüchen im Rahmen von hybridem Publizieren evaluiert und reflektiert werden.
\subsection{Offener Zugriff auf wissenschaftliche Kommunikation}
In Ergänzung zu 2.1.1. geht es bei Open Science dabei eben nicht nur um den offenen Zugang zu Wissenschaft und den daraus resultierenden Veränderungen von Kommunikationsprozessen im Rahmen von Publikationen, sondern auch um den generellen und offenen Zugriff auf den gesamten Prozess der Wissensschaffung. Dieser Ansatz folgt dabei der Annahme, dass aus technischer Sicht praktisch jeder Aspekt der Wissenschaftskommunikation, der digital auf einem Desktop-Computer stattfindet, auch öffentlich über das Web erfolgen kann\cite{mietchen2012wissenschaft}. Dieser Wissenschafts-Prozess wird in dieser Arbeit grob in vier Phasen gegliedert:
\begin{enumerate}
\item Planung
\item Ausführung
\item Verarbeitung
\item Auswertung
\end{enumerate}
In diesem Kapitel sollen die Charakteristika des Wissenschafts-Prozesss erläutert werden und dargestellt werden, was eine Öffnung im Sinne des Zugriffs bedeutet. In diesem Zusammenhang soll Open Science nicht nur als Sammelbegriff, sondern auch als weiteren Evolutionsschritt nach Open Access verstanden werden. Die Forderung nach Open Science begründet sich dabei nicht nur durch die in 2.1.1 genannten Unzulänglichkeiten am wissenschaftlichen Kommunikationssystem sondern basiert auf folgenden weiteren Annahmen:
\begin{enumerate}
\item Der offene Zugang zum gesamten Wissenschaftsprozess erhöht die Möglichkeiten der Validierung und Reproduzierbarkeit der gesamten Forschung(skette) und die Entwicklung neuer Qualitätskriterien. (enhanced Validation/Reputation-Argument)
\item Im Rahmen des Teilens (z.B. von Rohdaten) erhöht sich die Effizienz und Verwendbarkeit von Forschung und im Rahmen von Wissenschaft entstandenen Informationen (Shared-Science-Argument)
\item im klassischen wissenschaftlichen Kommunikationssystem gibt es kaum Anreize negative, widerlegende oder unerfolgreiche wissenschaftliche Ergebnisse zu veröffentlichen, eine grundsätzliche Öffnung könnte dazu beitragen, dass Wissenschaft ihrem Anspruch an Falsifizierbarkeit gerecht wird z.B. in Pharmalogie (negative-science/falsifiability-argument)
\end{enumerate}

Unterschiedliche Geschwindigkeiten bei der Technologieaneignung stellen ein Problem dar \cite{schulze_2013_open}

\subsection{Wissenschaft als Open-Source-Prozeß}

Open Source ist ein Begriff aus der Softwareentwicklung der als Gegensatz zum “Verfahren der Wissenssicherung”\cite{stallman2002} zugunsten einer quelloffenen Handhabe von Softwarecode verstanden werden will. Der Ende der 90iger Jahre des letzten Jahrhunderts eingeführte Begriff beschreibt, auch wenn es im Detail Unterschiede im Konzept gibt , das gleiche wie der Begriff “freie Software“ . Besonders der Entwicklungsprozess von Open-Source, in Ergänzung zum reinen Zugang und damit mit Open Science konvergent , unterscheidet sich von den klassisch-traditionellen closed-source Prozessen. Dabei folgt Open Source der Maxime, dass die Kernsteuerungsinformationen und -befehle (Quelltext) von Software öffentlich einsehbar und zugänglich und je nach gewähltem Lizenzmodell modifiziert, kopiert oder weitergegeben werden müssen . Es gibt aber auch unterschiede, so betont Steven Weber den Unterschied zwischen Open-Source-Software und dem traditionellen Modell des geistigen Eigentums mit der Feststellung, dass Open-Source-Software macht das Prinzip der Exklusivität des geistigen Eigentums auf den Kopf, weil diese Software 'um das Recht auf Vertrieb konfiguriert, nicht auszuschließen. "
Als Maurer und Scotchmer angemerkt haben, Open-Source-Software-Entwicklung Rechtsmittel ein Defekt der Schutz des geistigen Eigentums, die nicht allgemein zu fördern hat die Offenlegung des Quellcodes. 

Ebenso wie die Open Definition, gibt es festgelegte Kriterien für die Klassifizierung von Open Source Produkten. So heißt es in der Open Source Definition :
\begin{enumerate}
\item Freie Weitergabe
\item Quellcode, das Programm muss den Quellcode beinhalten, bzw. muss dieser offen zur Verfügung gestellt werden
\item Verwendete Lizenz muss Derivate zulassen
\item Unversehrtheit des Quellcodes des Autors muss garantiert werden
\item Auschluss von Diskriminierung von Personen oder Gruppen
\item Keine Enschränkung des Einsatzfeldes
\item Lizenz muss weitergegeben werden könnne
\item Lizenz muss auf das Produktpaket angewandt werden
\item Lizenz darf die Weitergabe zusammen mit anderer Software nicht einschränken
\end{enumerate}

Im Vergleich zum klassischen wissenschaftlichen Entwicklungsprozess gelten dabei folgende charakteristische Merkmale :
\begin{enumerate}
\item “Anzahl der beteiligten Entwickler: Im Vergleich zu traditioneller Softwareentwicklung ist eine weitaus größere Anzahl von Entwicklern beteiligt. Zudem gibt es keine klare Grenze zwischen Entwicklern und Anwendern, da die Hürden für eine Partizipation im Entwicklungsprozess sehr gering sind. Auch wenn ein großer Teil der Entwicklungsarbeit von Freiwilligen übernommen wird, gibt es dennoch den Trend zum Einsatz bezahlter Entwickler.
\item Zuteilung der Arbeit: Im OSP wird die Entwicklungsarbeit nicht länger von einer definierten Instanz zugeteilt, sondern die Teilnehmer wählen ihre Arbeitspakete selbst aus.
\item Architektur: In der Regel orientierten sich die Teilnehmer eines OSP nicht an einer vorgegebenen System-Architektur. Die Gestaltung der Architektur geschieht dezentral und ist oftmals häufigen Richtungswechseln unterworfen.
\item Koordination: Es gibt wenig oder keine institutionalisierten traditionellen Koordinationsmechanismen, wie z.B. Projekt- und Zeitpläne, Lasten- und Pflichtenhefte u.ä.”
\end{enumerate}

Vergleicht man diese mit dem traditionellen Wissenschaftsprozess (siehe 2.2.1.), ergeben sich gewisse Parallelen. Adaptiert man also den Open-Source-Prozess auf Wissenschaft und versteht wissenschaftliche Publikationen als Quellcode, ist das Konzept übertragbar. Der deutsche Literaturwissenschaftler und Medientheoretiker Friedrich Kittler sieht den Gedanken hinter Open-Source fest verankert und äussert in seinem Beitrag “Wissenschaft als Open-Source-Prozeß” die Sorge, “daß mit der Freiheit von Quellcode auch die Freiheit der Wissenschaft steht und fällt” . Wie Wissenschaft als Open-Source-Prozess verstanden werden kann soll in diesem Kapitel genauer erläutert werden. 
\subsection{Entwicklung der Bewegung}
Wissenschaft und Offenheit sind seit jeher zwei stark verbundene Elemente. “Open Science” ist dabei ein Begriff der historisch sehr eng mit der Entwicklung von kollaborativen Arbeiten durch neue Kommunikationstechniken verbunden ist. Open Science ist im Rahmen der Open Movements als Evolution zur reinen Öffnung des Zugangs (Open Access) zu wissenschaftlichen Publikationen zu verstehen. Eine klare Definition für den Sammelbergiff steht jedoch noch aus. Dabei spielt insbesondere, die Entwicklung der Tradition für eine "offenen Wissenschaft" im siebzehnten Jahrhundert einen Ansatzpunkt, da dieser historische Übergang noch nicht erforscht ist.\cite{CREATe_2014} Dieser Strang der Forschung ist eine sinnvolle Analyse, um grundlegende Argumente für Open Science in wissenschaftliche Forschung zu untersuchen. 

Im April 2012 wurde die Erklärung "Open Science for the 21st century", vom Zusammenschluss der Europäischen Akademien (ALLEA) verabschiedet \cite{ALLEA_2012}. Sie war nur eine von mehreren Erklärungen und Positionspapiere für die Öffnung von Wissenschaft durch international angesehenen Einrichtungen, die deutlich machten, dass die Forderung nach offenem Umgang mit Wissen und Information im wissenschaftlichen Bereich zunehmend an Relevanz gewinnt \cite{schulze_2013_open}.

In diesem Kapitel werden die historischen Aspekte der Veröffentlichung und Verbreitung von wissenschaftlichen Informationen chronologisch unter der Berücksichtigung der Frage, wie Wissen der Allgemeinheit zur Verfügung gestellt wird und wurde, erfasst und erläutert. 

Die Verschlüsselungs- und Patentwut zur Wahrung eines möglichen kommerziellen Vorteils durch Wissenschaft im Rahmen öffentlich-finanzierter Forschung, geht dabei bis auf die xxxx Jahre zurück. ### Beispiel Galileo, Kepler, Newton ### Das Ergebnis dieser Wut war eine Debatte über die Verfügbarkeit der wissenschaftlichen Arbeit und die Entlohnung der “Erfinder“ im wissenschaftlichen System. 
\subsection{Open Science Modelle}
\subsection{Open Science Formate}
Data Repositorien, (offne) Forschungsanträge, offenes Publizieren (siehe OA), Laborbücher

\section{Wissenschaftliche Reputation}
Ein Grundprinzip des Wissenschaftssystems basiert auf der "gegenseitigen Beurteilung und Anerkennung der jeweils neuen Ergebnisse ihrer Fachkollegen (Peers) durch die Wissenschaftler selbst"\cite{Hanekop_2014}. In dem Peer-Review Prozess "werden eingereichte Beiträge von fachlich versierten Wissenschaftlern (...) beurteilt und gemäß den qualitativen Anforderungen der Forschungs-Community zur Veröffentlichung angenommen oder abgelehnt" \cite{Hess_2006}. "Peer-Review" beschränkt sich dabei nicht nur auf die Publikation von Texten, sondern deckt ein breites Spektrum von Aktivitäten ab: die Beobachtung der klinischen Praxis; Beurteilung des lehrenden Fähigkeiten der Kollegen; Bewertung durch Experten bei der Forschungsförderung und Stipendien bei Einreichung von Anträgen an staatliche und anderen Förderorganisationen; Begutachtung von Redakteuren und externen Gutachtern bei Artikeleinreichungen für wissenschaftlichen Zeitschriften; Bewertung von Papieren und Plakate für Konferenzen; Bewertung von Buchvorschlägen für Universitätverlagen oder andere Verlagenn; und Einschätzungen der Qualität, Anwendbarkeit und Interpretierbarkeit von Datensätzen und wissenschaftlichen Organisationen" \cite{Lee_2012}. Dennoch bilden "Publikationen im Hinblick auf die Funktion der Reputationsverteilung eine Art Telos wissenschaftlicher Kommunikation"\cite{hirschauer2004peer}. Im Rahmen von Reputation ist wissenschaftliche Arbeit besonders auf funktionierendes Peer-Review-System angewiesen\cite{suchen}. Dennoch haben qualitatives Peer Review-Systeme und quantitative bibliometrischen Verfahren viele Mängel\cite{osterloh2008anreize} \cite{Lee_2012}.

In diesem Kapitel soll wissenschaftliche Reputation genau definiert und ihre Abhängigkeit zum bestehenden wissenschaftlichen Kommunikationssystem herausgestellt werden. Wissenschaftliche Reputation soll hier als Währung verstanden werden, mittels derer “Status und Ressourcen verteilt werden”\cite{hanekop_2006}. Wissenschaftliche Reputation verteilt sich auf Einrichtungen und einzelne Personen, die wissenschaftlich tätig sind\cite{suchen}. 

Die Evaluation wissenschaftlicher Einrichtungen findet dabei über “Beobachtungen und Gespräche mit den Wissenschaftlern vor Ort sowie über den Austausch über die Eindrücke innerhalb der Begehungsgruppe und die gemeinsame Verständigung”\cite{Barl_sius_2008} statt. Diese wird in dieser Arbeit nur beiläufig betrachtet.

Die Reputation einzelner Wissenschaftler steht in dieser Arbeit im Vordergrund. Für sie sind Publikationen und die damit einhergehende Verbreitung von wissenschaftlichen Erkenntnissen sehr entscheidend \cite{Hess_2006}. Vereinfacht lässt sich das System der Wechselbeziehungen der Reputationsverteilung im Rahmen von Publikationen wie folgt darstellen\cite{cite:21a}: \href{http://www.eap-journal.com/archive/v39_i1_8_bernius.pdf}{Grafik aus Text von Bernius}

Bernius et al. unterscheiden in diesem Zusammenhang zwischen drei koordinierenden Marktmechanismen aufeinandertreffen: die Reputation und die Nutzung von wissenschaftlichen Publikationen, sowie der Preis für den Erwerb. Während die Reputation ein Aushandlungsmechanismus zwischen den Verlagen und wissenschaftlichen Autoren darstellt, findet die Preisdefinition zwischen den Bibliotheken und den Verlagen statt. Das Zusammenspiel zwischen Wissenschaftlern und Bibliotheken beschränkt sich dabei auf die Frage der Nutzung.\cite{cite:21a}

Doch nicht jede Publikation hat die gleiche Wertigkeit\cite{suchen} und damit den gleichen Einfluss auf die Reputation. Die neuen Möglichkeiten der Verbreitung von Informationen lassen deshalb einen vergleichbaren Veränderungsprozess der wissenschaftlichen Verbreitung und damit auch Anerkennung vermuten, die wie Entwicklung des Drucks.\cite{hanekop_2006} Neben dem Publizieren müssen auch folgende weitere Indikatoren für wissenschaftliche Reputation für wissenschaftliche Institutionen und Personen genannt werden\cite{hanekop_2008}:
\begin{enumerate}
\item Drittmittelprojekte
\item Patente
\item Vorträge
\item Anwendungsrelevanz bzw. Verwertbarkeit
\item Netzwerke
\item öffentliche Aufmerksamkeit sowie politische Relevanz 
\item Renommee der Forschungseinrichtung
\item materielle Ausstattung, Großgeräte etc.
\item personelle Ausstattung
\item Gutachtertätigkeit
\item Herausgeberschaft
\item Funktion
\end{enumerate}
In diesem Rahmen wurden durch den US-amerikanische Soziologe Robert K. Merton vier Grundprinzipien als normative Struktur der Wissenschaft beschrieben\cite{Merton_1985}, auf deren Grundlage in diesem Kaptitel wissenschaftliche Reputation und das damit einhergehende Anreizsystem, sowie deren Veränderungsprozess durch die Öffnung von Wissenschaft erläutert werden soll. 

\subsection{Wissenschaftliches Kapital}
Im Rahmen der Betrachtung von Steuerungs- und Reputationsmethoden für Wissenschaft ist der Begriff wissenschaftliches Kapital als zentral zu sehen\cite{suchen}. Wissenschaftliches Kapital kann dabei als eine Ausprägung des kulturellen Kapitals und als symbolisches Kapital\cite{irmer2011} verstanden werden. “Scientia donum dei est, unde vendi non potest", daß die Wissenschaft nicht verkauft werden kann, weil Wissen eine Gabe Gottes ist, sollen die Gelehrten des Mittelalters gerufen haben\cite{suchen}. Bereits damals war aber abzusehen, daß dieses Verständnis nicht lange halten würde.

Wie bereits in der Einleitung erwähnt, basiert die “Gewährung wissenschaftlichen Kapitals” im wissenschaftlichen System heute auf einer engen Verbindung zwischen den Verlagen und den publizierenden Wissenschaftlern\cite{herb_2006}. Dabei steht die Wissenschaft in einer klaren Abhängigkeit zu den Verlagen. Ulrich Herb definiert in diesem Zusammenhang wissenschaftliches Kapital mit Hilfe Pierre Bourdieus als das “Ergebnis einer Investition (...), die sich auszahlen muss” ist. “Diejenigen, die diese Berechtigungsscheine in der Hand halten, verteidigen ihr 'Kapital' und ihre 'Profite', in dem sie diejenigen Institutionen verteidigen, die ihnen dieses 'Kapital' garantieren.”\cite{Bourdieu_1992} Wissenschaftler sind also von einem Ressourcenzufluss abhängig um Ihre Aufgaben dauerhaft wahrzunehmen\cite{Suess_2006}. Herb kommt zu dem Schluss, dass die Öffnung von Wissenschaft dabei bisher nicht wissenschaftlicher Logik folgt, “sondern einer feldunabhängigen Logik der Akkumulation von Kapital”\cite{herb_2006}. Hinzu kommt, dass vor allem das deutsche Wissenschaftssystem durch durch die Einführung an Outputs orientierter Anreizsystem gekennzeichnt ist.\cite{osterloh2008anreize}

Als Beispiel kann der Performanzindikator "Drittmittel"\cite{Jansen_2007} dienen, in der Wissenschaft neben der Sicherung der Qualität von Forschung und Lehre zunehmend direkte finanzielle und administrative Kontrolle eine Rolle spielen\cite{Barl_sius_2008},. Daraus resultiert die Gefahr, dass nicht nur die Erwartungen an die Bewertung von Wissenschaft zu hoch gegriffen sind, sondern auch, dass sich Wissenschaft zu sehr an diesen Erwartungen orientiert und die Legitimität öffentlicher Ausgaben über den Zweck gestellt werden. Vor allem die Verknüpfung von wissenschaftlicher Reputation und der damit einhergehenden Verteilung von Mittel und Stellen stellt eine Herausforderung an das Wissenschaftsystem, “dessen Währung [ursprünglich] nicht Geld ist”\cite{hanekop_2006} dar. In diesem Teil der Arbeit soll das Konzept des wissenschaftlichen Kapitals vor dem Hintergrund des Widerspruchs gegenüber den Grundprinzipien der Wissenschaft erläutert und dargestellt werden, welche Möglichkeiten die Öffnung von Wissenschaft für die Legitimität öffentlicher Ausgaben darstellt, ohne das diese über den eigentlichen Zweck gestellt werden.
\subsection{Die Ökonomie des wissenschaftlichen Kommunizierens}
Die klassische Ökonomie der wissenschaftlichen Kommunikation beruht auf der Durchsetzung von Urheberrechten, die den Zugriff und die Wiederverwendung von geschützten Inhalten beschränken und die Zahlung einer Gebühr durch den Leser verlangen um Zugang zu der Veröffentlichung zu erhalten.\cite{CREATe_2014} Bislang werden dafür "in der Regel wissenschaftliche Arbeiten zwar mit öffentlichenMitteln finanziert, aber von privaten Verlagen in Fachzeitschriften herausgegeben"\cite{WD_bundestag_2009}. Dieses Modell baisert auf einer sozial ineffizientem Ebene\cite{mueller-langer_2010}. Diese ungewöhnlichen Ökonomie der Wissenschaftsverlage ist nicht neu und hat sich im Laufe der Zeit entwickelt, die starke Wahrnehmung der Ungerechtigkeit des Systems, vorallem an den Preismodellen für wissenschaftliche Publikationen\cite{King_2008}, findet aber erst seit kurzem statt\cite{CREATe_2014}.

Als weitere wesentliche Besonderheit der Wissenschaftskommunikation ist die Organisation des Marktes, die von spezifischen Akteuren und Prozessen geprägt ist \cite{Hess_2006}, zu nennen. Vereinfacht kann der klassische wissenschaftliche Kommunikationsprozess im Rahmen von Publikationen wie folgt unterteilt werden\cite{cite:11b} \cite{Hess_2006}:
\begin{enumerate}
\item Erstellung durch Wissenschaftler - Inhalte erzeugen: 
Nach der Entwicklung eines konkreten Forschungsvorhabens sowie einer wissenschaftlichen Fragestellung enstehen im Rahmen der wissenschaftlichen Forschung oder der jewiligen Untersuchung Informationen\cite{cite:11c}, die im Forschungsprozess gesammelt, analysiert, ausgewertet, aufbereitet und verschriftlicht wurden\cite{cite:11d}. Diese Infromationen werden strukturiert zusammengefasst und niedergeschrieben \cite{Hess_2006}.
\item Qualitätskontrolle durch Wissenschaftler - Inhalte bewerten: 
Die Qualitätskontrolle ist einer der wesentlichen Bestandteile der wissenschaftlichen Kommunikation. Sie sichert die gewonnen Erkenntnisse\cite{cite:11e} und stellt einen klaren Abrenzungsaspekt zu nicht-wissenschaftlichen Informationen und Erkenntnissen dar\cite{cite:11f}. Sie findet im Kommunikationsprozess an zwei Stellen statt. Hier ist die erste Stelle gemeint, in der vor der Produktion der Informationen in Form der Publikation, die Erkenntnisse von anderen Wissenschaftlern überprüft und gesichert werden (Peer-Review) \cite{Hess_2006}.
\item Bündlung durch Verlage - Inhalte auswählen:
\item Publikation durch Verlage - Inhalte distribuieren: 
Nach Erstellung und Erkenntnissicherung findet die für die Distribution notwendige Publikation der Informationen statt. Vor der digitalen Revolution bestand dieser Schritt ausschließlich in dem Druck auf Papier.\cite{cite:11h}
\item Distribution durch Verlage: 
Der Vertrieb und die Verbreitung der Inhalte ermöglicht den Zugriff auf die Information der Forschung durch andere Wissenschaftler. Der Schritt stellt einen wichtigen Teil zur Zirkulation des neu gewonnen Wissens dar\cite{cite:11i}. Er sichert die Verfügbarkeit und den Zugriff auf die Informationen und ist essentieller Teil des Selektionsprozesses für die Erschaffung neuen Wissens.\cite{cite:11l}
\item Suppoert und Archivierung durch Bibliotheken
\item Konsum beziehungsweise Rezeption durch Wissenschaftler: 
Der nächste Schritt im wissenschaftlichen Kommunikationsprozess, der wiederum den gesamten Prozess von vore beginnen lässt ist die Rezeption der veröffentlichten Inhalte. Hier geht es zum einen um die Rezeption der wissenschaftlichen Vorschung aus der "Erstellung", zum anderen kommt hier auch die zweite Stufe der Qualitätsicherung zum Tragen.\cite{cite:11j} Der Konsum von wissenschaftlicher Informationen ist dabei auch als Grundlage für die "Erstellung" neuen Wissens zu betrachten. Somit ist der Endpunkt des wissenschaftlichen Kommunikationsprozess auch gleichzeitig Ausgangspunkt für einen neuen Prozess\cite{cite:11k}.
\end{enumerate}

An diesem Prozess sind vor allem drei Gruppen beteiligt: erstens die Wissenschaftler, als Produzenten und Konsumenten der Informationen, zweitens die Verleger, die als Intermediäre wissenschaftliche Informationen sammeln, bündeln und verkaufen, sowie drittens die Bibliotheken, die die Informationen wieder den Wissenschaftlern zur Verfügung stellen \cite{Odlyzko_1997}. Aus diesem Prozess und den beteiligten Gruppen, werden folgende Problemfelder ersichtlich:


klassisches Geschäftsmodell/Wertschöpfungskette vs. Open Access Geschäftsmodell/Wertschöpfungskette \cite{Hess_2006}

\subsection{Messbarkeit wissenschaftlicher Qualität vs. Publikationsquantität}
Wissenschaft ist ein Prozess, bei dem aus “unterschiedlichen Inputfaktoren, mittels verschiedener Transformationen Beiträge zur Schaffung neuer wissenschaftlicher Erkenntnisse als Output entstehen”\cite{Jansen_2007}. Die Bewertungen des jeweiligen Outputs führt “zur Ausage über die Forschungsperformanz”. Neben den Indikatoren für den Output wissenschaftlicher Perfomanz, müssen aber auch intermediäre Aspekte berücksichtigt werden\cite{schmoch_2009}. Nach diesem Ansatz etablierten sich nach dem zweiten Weltkrieg die ersten Indikatoren für die Effizienzmessung wissenschaftlicher Wissensproduktion. Seit den 1960er Jahren wird diese Messung in der Gestalt von Indikatoren, die die Forschungsleistung quantifizieren sollen, durchgeführt. Seit den 1990er Jahren fällt diese Bewertung in Gestalt von Zahlen als unkontrollierte Nebenprodukte digitaler Wissenskommunikation an\cite{angermueller_2010}. Heute zählen in der Wissenschaft vor allem die wissenschaftliche Reputation und die als Impact bezeichnete Wirkung wissenschaftlicher Publikationen\cite{herb_open_2013}. Die Wirkung wird dabei anhand der Zitationen der jeweiligen Publikation gemessen. Eine häufige Zitation stellt dabei einen Indikator für einen große Wirkung der wissenschaftlichen Arbeit dar. In diesem Kapitel werden die gängigen Methoden und Möglichkeiten der Messbarkeit von wissenschaftlicher Reputation und Wirkung im Kontext der Arbeit dargestellt.
\subsubsection{Wissenschaftliche Diskurse, nach dem Diskurs- und Machtbegriff}
Nach Niklas Luhmann operiert der wissenschaftliche Diskurs funktional eigenständig und alles was durch Wissenschaft kommuniziert wird, ist “entweder wahr oder unwahr”\cite{Luhmann1998}. Der wissenschaftliche Diskurs gründet dabei aber nur zum Teil auf der Forschung und kann auch nicht nur als “Kontaktglied zwischen dem Denken und dem Sprechen”\cite{foucault_ordnung_2003} definiert werden. In der Foucault'schen Diskursanalyse wird der Diskurs deshalb als die Fähigkeit definiert, die “Beziehungen” zwischen “Institutionen, ökonomischen und gesellschaftlichen Prozessen, Verhaltensformen, Normsystemen, Techniken, Klassifikationstypen und Charakterisierungsweisen herzustellen”\cite{foucault_archaologie_1981}. Foucault beschäftigt sich in diesem Zusammenhang vor allem mit den Grenzen des Diskurses sowie dessen institutioneller und praktischer Verortung. In diesem Zusammenhang soll in dieser Arbeit auch adressiert werden inwiefern Macht, regulierende Prinzipien wie Verknappung sowie die Ein- und Ausgrenzung bezüglich des wissenschaftlichen Diskurses, nach dem Diskurs- und Machtbegriff von Michel Foucault, mit den Modellen der Open Initiatives in der wissenschaftlichen Kommunikation vereinbar sind oder dem gegenüberstehen. Im Gegensatz zu innerdiziplinären Betrachtung eignet sich Foucaults “Werkzeugkiste”\cite{Honneth_2003} dabei besonders um die transdisziplinäre Öffnung von wissenschaftlichen Prozessen und den damit einhergehende Öffnung des Diskurses theoretisch zu hinterfragen. 
In diesem Kapitel soll deshalb der Diskursbegriff in den Kontext der Thematik der Öffnung des Zugriff auf den wissenschaftlichen Prozess erläutert werden.
\subsection{Kritik}
Die Verlage haben mit Hilfe von wissenschaftlichen Journalen ein zentrales Steuerungs- und Bewertungssystem in der Wissenschaft etablieren können. Dabei werden die Grundprinzipien der Wissenschaft für die verlegerischen Verwertungsinteressen genutzt und das obwohl diese “wissenschaftlichen Grundprinzipien und Normen eigentlich ökonomischen Verwertungsinteressen zu widersprechen scheinen”\cite{hanekop_2006}. Darüber hinaus haben die Forscher in vielen Fällen wenig oder keine Verantwortung für den Einkauf von der wissenschaftlichen Informationen, die er oder sie "verschenkt"\cite{steele_2006}. Spätestens durch die Einführung von Zitationsregistern und Impact Faktoren sowie die Definition der Kernzeitschriften hat den wissenschaftliche Zeitschriftenmarkt extrem unelastisch gemacht und gleichzeitig die Kapazität der kommerziellen Verlagen sowie deren Gewinnmargen ansteigen lassen.\cite{CREATe_2014} Die Steuerungsmechanismen werden über die Messbarkeit an Hand der in 2.3.4 beschriebenen Methoden direkt oder indirekt ausgeübt. Dabei stehen insbesondere die Methoden, die auf der quantitativen Grundlage der Zitationsraten wissenschaftlicher Publikationen gemessen werden in der Kritik\cite{Dong_2005} und auch andere Indikatoren für die Messung von Forschungsleistungen sind hoch umstritten\cite{Hornbostel_1997}\cite{Hicks_1996}\cite{Havemann_2002}. Der Hauptkritikpunkt: Die Verfahren, um die Wirkung von Wissenschaft und damit auch die Reputation von Wissenschaftlern zu messen, sind kein eigentliches wissenschaftliches Produkt\cite{suchen} und erfassen zum Beispiel die Tätigkeit einzelner Forschergruppen zu stark\cite{schmoch_2009}. Das führt unter anderem dazu, dass der Impact Factor “kein perfektes Werkzeug (ist) um die Qualität der Artikel zu messen” und trotzdem wird er zur Bewertung von Wissenschaft genutzt, denn “(...) es gibt nichts Besseres, und er hat den Vorteil, dass er bereits lange existiert und ist daher eine gute Technik für die wissenschaftliche Bewertung”\cite{garfield_1999}. Wie “gering der Wirkungsgrad” und die Methoden zur Messung “zur Reproduktion des traditionellen wissenschaftlichen Diskurses ausfall(en), wird von dem Moment an klar, an dem ein neues und offenes Kommunikationsmedium wie das Internet als alternativer Publikations- und Verbreitungskanal für Wissenschaft zur Verfügung steht\cite{Rost_1998}. Im folgenden soll aufgezeigt werden, welche Kritik es an dem System der wissenschaftlichen Reputation sowie dem Steuerungsmodell durch die Verlage im Rahmen der Veränderung der wissenschaftlichen Publikationskanälen gibt.
