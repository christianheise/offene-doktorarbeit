\chapter{Definition und Abgrenzung} 
Wenn es im aktuellen öffentlichen Diskurs um wissenschaftliche Informationen, Infrastruktur und Arbeiten geht, werden immer öfter Schlagworte mit dem Attribut „Open“, wie Open Access, Open Research und Open Science, verwendet \cite{bunz_2014} \cite{schulze_2013_open}. "Offen" bezieht sich dabei üblicherweise auf zwei Kernaspekte: Zum einen die Offenheit des Zugangs zu Daten, Code oder Ergebnissen und zum anderen auf das Gebot der Transparenz, also die Offenlegung beziehungsweise der Zugriff auf Verfahren, Methoden und Ziele \cite{schulze_2013_open}.

In der Literatur sind die Begriffe „Open Access“, „Open Science“ und „wissenschaftliche Reputation“ nicht allgemeingültig definiert \cite{suchen}. Sie finden in der wissenschaftlichen Auseinandersetzung auf unterschiedlichste Art und Weise Verwendung \cite{cite:9}. In diesem Kapitel wird der historische und gesellschaftliche Kontext ihrer Anwendung dargestellt und mittels der Analyse wissenschaftlicher Literatur abgegrenzt. Es wird angestrebt, zu erläutern, welche Bedeutung sie in Forschung, Gesellschaft und Politik haben. Die Begriffe werden in ihrer Entwicklung dargestellt. Um ein möglichst umfassendes Bild zu erhalten, wird "Entwicklung" hier in den drei Dimensionen erfasst: erstens, als "analytische Kategorie", zweitens als "Forschungsgegenstand" und drittens als "politische Praxis in der moralischen Auseinandersetzung über die Wünschbarkeit von Zuständen". \cite{cite:10} Ziel ist es hierbei nicht, gemeingültige Definitionen zu erarbeiten, sondern die Darstellung dieser Begriffflichkeiten in verschiedenen Diskursen. 

Im Rahmen dieses Kapitels wird darüber hinaus auch adressiert inwiefern Macht, regulierende Prinzipien wie Verknappung oder Ein- und Ausgrenzung in wissenschaftliche Diskurse, nach dem Diskurs- und Machtbegriff von Michel Foucault, mit den Modellen der offenen Initativen in der wissenschaftlichen Kommunikation vereinbar sind oder ob sie denen gegenüberstehen.

Die Unterscheidung "Zugang" und "Zugriff" steht im Zentrum dieser Arbeit und stellt eine wesentliche Grundlage für die Definition und Abgrenzung der Begriffe "Open Access" und "Open Science" dar. "Zugang" bezieht sich in diesem Zusammenhang auf einen unbeschränkten Zugang zur finalen wissenschaftliche Publikation. "Unbeschränkt" meint hier: lesen\cite{cite:9a}, Verarbeitung und Weiternutzung. "Zugriff" soll in diesem Kontext als erweiterte Nutzung der jeweiligen Wissensressourcen verstanden werden. Eingeschlossen sind neben dem "Zugang" zur Publikation sämtliche Informationen und Daten sowie die komplette Kommunikation hinter der finalen Veröffentlichung \cite{cite:9b}. "Zugriff" beschränkt sich hier also nicht nur auf den reinen Zugang zu wissenschaftlicher Information im Rahmen des Publikationsprozesses, sondern schließt auch den Zugriff auf sämtliche Forschungsdaten, Methoden und alle weiteren Informationen, die während der wissenschaftliche Arbeit auf dem Weg zur finalen Publikation entstehen \cite{cite:9c}, ein.

Die Themenbereiche Social Media in Wissenschaft und Forschung, Citizen Science und aktuelle Diskurse zu Tools und Diensten werden in dieser Arbeit bewusst ausgelassen und nur am Rande, beziehungsweise nur wenn sie die Beantwortung der Forschungsfragen tangieren, eingeschlossen.