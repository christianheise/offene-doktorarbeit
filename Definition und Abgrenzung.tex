\chapter{Definition und Abgrenzung} 
In theoretischer Hinsicht existiert keine allgemeingültige Definition der Begriffe „Open Access“, „Open Science“ und „wissenschaftliche Reputation“, deshalb sollen in diesem Kapitel die Begriffe anhand wissenschaftlicher Literatur genau analysiert, zueinander abgegrenzt und zuletzt der Versuch gewagt werden, zu erläutern, wie die der Begriff Open Science in der Analyse, Forschung und Politik funktioniert und agiert. Dabei soll nicht eine gemeingültige Definition erarbeitet sondern dargestellt werden, wie die Begriffflichkeiten rund um Open Science in verschiedenen Diskursen verwendet werden und in welchen Relation sie zu anderen Begriffen stehen.

Darüber hinaus sollen die Begriffe in ihrer Entwicklung dargestellt werden. Entwicklung soll hier nach Bierschenk in den drei Dimensionen erfasst werden: erstens, als analytische Kategorie, zweitens als Forschungsgegenstand und drittens als politische Praxis im "moralischen Diskurs über die Wünschbarkeit von gesellschaftlichen Zustände"  betrachtet und erarbeitet werden um ein möglichst umfassendes Bild der Begriffe zu erhalten. 

Da die Begriffe „Open Access“, Open Science” und wissenschaftliche Reputation in der wissenschaftlichen Auseinandersetzung auf unterschiedlichste Art und Weise verwendet werden und nicht eindeutig definiert sind, werden in diesem Kapitel die Begriffsbestimmungen konkretisiert sowie in den historischen und gesellschaftlichen Kontext eingebettet.
Im Rahmen dieses Kapitels soll darüber hinaus auch adressiert werden inwiefern Macht, regulierende Prinzipien wie Verknappung sowie die Ein- und Ausgrenzung in den wissenschaftlichen Diskursen , nach dem Diskurs- und Machtbegriff  von Michel Foucault, mit den Modellen der Open Initiatives in der wissenschaftlichen Kommunikation vereinbar sind oder dem gegenüberstehen.
