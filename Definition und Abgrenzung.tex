\chapter{Definition und Abgrenzung} 

Der theoretische Bezugsrahmen wissenschaftlich gesicherter Modelle, Theorien und Ansätze ermöglicht es, Erklärungen und Handlungsempfehlungen abzuleiten\cite{martin_2007_wissenschaftstheorie}. Er trägt dazu bei, die Fragestellungen in einen Zusammenhang zu stellen, legitimiert die Erforschung dieser Fragen und bildet den Rahmen für die Auswertung gesammelter Erkenntisse \cite{suchen}. 
Diese theoretischen Grundlagen werden im Rahmen einer Literaturanalyse, einer Umfrage, sowie an der wissenschaftlichen Praxis im Rahmen der eigenen experimentellen Arbeit geprüft. Dafür werden die aus der theoretischen Betrachtung analysierten Konzepte Open Access und Open Science den eigenen empirischen und experimentellen Beobachtungen zugeordnet. In diesem Abschnitt der Arbeit wird eine grundlegende Einordnung und Abgrenzung von Open Access, Open Science und wissenschaftliche Reputation vorgenommen. Ziel ist es, die theoretischen Grundlagen für die im nächsten Abschnitt folgende Literaturanalyse und die empirischen und experimentellen Ergebnisse zu erarbeiten. 

Open Science und Open Access im Kontext wissenschaftlicher Reputation wird in dieser Arbeit in technischen, als auch in ihren gesellschaftlichen und politischen Aspekten beschrieben. Die Beschreibung wird auf die kulturellen Auswirkungen der Medienbrüche im Rahmen wissenschaftlichen Publizierens erweitert. Die Analysen in dieser Arbeit werden aus der Perspektive des Produzenten (Wissenschaftler als Autoren) und aus der, damit nicht immer harmonisierenden, Perspektive des Rezipienten, beziehungsweise Medienkonsumenten (Wissenschaftler als Leser) stattfinden. In diesem Zusammenhang wird adressiert, inwiefern Macht, regulierende Prinzipien wie die Verknappung, sowie die Ein- und Ausgrenzung im Rahmen wissenschaftlicher Diskurse (nach dem Diskurs- und Machtbegriff von Michel Foucault) mit den Modellen Open Access, Open Science und wissenschaftlicher Reputation in der Kommunikation vereinbar sind oder diesen gegenüberstehen. 

Wenn es im aktuellen öffentlichen Diskurs um wissenschaftliche Informationen, Infrastruktur und Arbeiten geht, werden immer öfter Schlagworte mit dem Attribut „Open“, wie Open Access, Open Research und Open Science, verwendet \cite{bunz_2014} \cite{schulze_2013_open}. "Offen" bezieht sich dabei üblicherweise auf zwei Kernaspekte: Zum einen die Offenheit des Zugangs zu Daten, Code oder Ergebnissen und zum anderen auf das Gebot der Transparenz, also die Offenlegung, beziehungsweise der Zugriff auf Verfahren, Methoden und Ziele \cite{schulze_2013_open}. Während "Openness" vielfach auf die Entwicklung rund um offene Software zurückzuführen sind die Anknüpfungspunkte von "Offenheit" als Begriff in der wissenschaftlichen Auseinandersetzung früher anzusetzen \cite{Tkacz_2014}. So sieht Christopher Kel­ty die ersten Anfänge bereits in den 1980ern \cite{kelty_2008_two_bits} und Andrew Russell sieht die ideologischen Ursprünge von "Offenheit" als Standard mit der Entwicklung des Telegraphs und weiteren Ingenieurleistungen ab 1860\cite{Russell_2014}.

In der gegenwärtigen Literatur sind die Begriffe „Open Access“ und „Open Science“ nicht allgemeingültig definiert \cite{suchen}. Sie finden in der wissenschaftlichen Auseinandersetzung auf unterschiedlichste Art und Weise Verwendung \cite{cite:9}. 

Der historische und gesellschaftliche Kontext ihrer Anwendung wird dargestellt und mittels der Analyse wissenschaftlicher Literatur abgegrenzt. Es wird erläutert, welche Bedeutung sie in Forschung, Gesellschaft und Politik haben. Die Begriffe werden in ihrer Entwicklung dargestellt. Um ein möglichst umfassendes Bild zu erhalten, wird "Entwicklung" hier in den drei Dimensionen erfasst: erstens, als "analytische Kategorie", zweitens als "Forschungsgegenstand" und drittens als "politische Praxis in der moralischen Auseinandersetzung über die Wünschbarkeit von Zuständen" \cite{cite:10}.

Die Unterscheidung "Zugang" und "Zugriff" ist wesentlich in dieser Arbeit und stellt eine zentrale Grundlage für die Definition und Abgrenzung der Begriffe "Open Access" und "Open Science" dar. "Zugang" bezieht sich in diesem Zusammenhang auf einen unbeschränkten Zugang zur finalen wissenschaftliche Publikation. "Unbeschränkt" meint hier: lesen\cite{cite:9a}, Verarbeitung und Weiternutzung. "Zugriff" soll als erweiterte Nutzung der jeweiligen Wissensressourcen verstanden werden. Eingeschlossen sind neben dem "Zugang" zur Publikation sämtliche Informationen und Daten, sowie die komplette Kommunikation hinter der finalen Veröffentlichung \cite{cite:9b}. "Zugriff" beschränkt sich hier also nicht nur auf den reinen Zugang zu wissenschaftlicher Information im Rahmen des Publikationsprozesses, sondern schließt auch den Zugriff auf sämtliche Forschungsdaten, Methoden und alle weiteren Informationen, die während der wissenschaftliche Arbeit auf dem Weg zur finalen Publikation entstehen \cite{cite:9c}, ein. 

Die Themenbereiche kollaboratives Arbeiten, Social Media in Wissenschaft und Forschung, Citizen Science und aktuelle Diskurse zu Tools und Diensten werden in dieser Arbeit bewusst ausgelassen und nur am Rande, beziehungsweise nur wenn sie die Beantwortung der Forschungsfragen tangieren, eingeschlossen.

Es ist derzeit üblich, Open Access in drei Modelle einzuteilen \cite{suchen}: Green Open Access, Golden Open Access und andere (Misch-)Formen. --- TODO: Grafik ----

\section{Open Access} 

\begin{quote}
„Open Access“ meint, dass [= Peer-Review-Fachliteratur] kostenfrei und öffentlich im Internet zugänglich sein sollte, sodass Interessenten die Volltexte lesen, herunterladen, kopieren, verteilen, drucken, in ihnen suchen, auf sie verweisen und sie auch sonst auf jede denkbare legale Weise benutzen können, ohne finanzielle, gesetzliche oder technische Barrieren jenseits von denen, die mit dem Internet-Zugang selbst verbunden sind. In allen Fragen des Wiederabdrucks und der Verteilung und in allen Fragen des Copyrights überhaupt sollte die einzige Einschränkung darin bestehen, den Autoren Kontrolle über ihre Arbeit zu belassen und deren Recht zu sichern, dass ihre Arbeit angemessen anerkannt und zitiert wird.
\cite{boai_2012}
\end{quote}
Der Fortschritt der Wissenschaft hängt auch, und maßgeblich von dem freien Austausch und der Verbreitung von Informationen ab \cite{cite:11}. Das System der wissenschaftlichen Kommunikation, das so seit mehreren hundert Jahren besteht, basierte auf Forschung, der Begutachtung, dem Druck sowie der Kommunikation der Egebnisse in wissenschaftlichen Publikationen, der Verbreitung sowie dem Verkauf an Bibliotheken und andere wissenschaftliche Institutionen gegen Kosten \cite{cite:11a} und dem anschließenden Diskurs in der wissenschaftlichen Fachöffentlichkeit \cite{suchen}. Der offene Zugang zu wissenschaftlicher Kommunikation ist seit der Entwicklung des gedruckten Wortes eng mit der Frage nach Urheberrechten für wissenschaftliche Informationen verknüpft \cite{Case_2000}. Open Access beschreibt in ein wissenschaftliches Kommunikationssystem, in dem der Zugriff auf die unterschiedlichsten Formen wissenschaftlicher Publikationen, im Gegensatz zum bestehenden System, unter freien, kostenlosen Bedinungen und ohne finanzielle, gesetzliche oder technische Barrieren (Online) möglich ist \cite{WD_bundestag_2009}. Dieses System ermöglicht darüber hinaus ein "alternatives Geschäftsmodell"\cite{lewis_2012_inevitability} für wissenschaftliche Publikationen. Was auf Maßgabe beruht, dass der Autor die "Eigentumsrechte an den Artikeln, die bisher für die Publikation in wissenschaftlichen Journals an die jeweiligen Fachverlage abgetreten wurden, (...) nun bei den Autoren der Artikel selbst verbleiben" \cite{Hess_2006}. 

Infolge des weltweit steigenden Haushaltsdrucks der Bibliotheken und wissenschaftlichen Insitutionen, des “ungewöhnlichen Geschäftsmodells” \cite{cite:12} der Wissenschaftsverlage mit immer höheren Margen \cite{albert_2006_open_implications} und des Umstandes, dass private Wissenschaftsverlage durch das wissenschaftlichen Reputationssystem über öffentlich finanzierte Wissenschaftlerkarrieren entscheiden \cite{heise_2012}, befindet sich das wissenschaftliche Kommunikationssystem in einer Krise \cite{cite:14}. 

Durch die digitalen Infrastrukturen ist Open Access die Antwort auf diese Krise und setzt bei der Öffnung (Open) und dem freien Zugang (Access) zu den wissenschaftlichen Publikationen an. "Geringere Kostenbarrieren und damit eine einfachere Verbreitung ihrer eigenen Werke" \cite{WD_bundestag_2009} stellen dabei die Wünsche der wissenschaftlichen Autoren und Urheber an Open Access dar. Der Einsatz (offener) Lizenzen ist dafür ein weiterer Haupteinflussfaktor \cite{cite:16}. 

\subsection{Chronologie der Bewegung}
Um Open Access einzuordnen, ist eine historische Betrachtung der Entwicklung wissenschaftlicher Kommunikation, aber auch der Forderung nach Offenheit in eben dieser unabdingbar. 

Im antiken Griechenland, und in vielen anderen pre-modernen Zivilisationen, wurden Wissen und Informationen als nicht besitzbare Ware angesehen\cite{cite:18}. Dennoch war der Austausch im Vergleich zu den heutigen Möglichkeiten stark beschränkt \cite{cite:17c}. "In vorwissenschaftlichen Gesellschaften gibt es keine scharfe Grenze zwischen dem vorhandenen und dem aktuell benutzten Wissen"\cite{Luhmann1998}. Die vormoderne Wissenschaft setzte bei den täglichen Bedürfnissen an und stellte sich die Aufgabe "das Wissen zu verbessern und vor allem zu erhalten und zu tradieren" \cite{Luhmann1998} Angeleht an die drei Etappen der Medienentwicklung von McLuhan \cite{wunderlich_2008_buchdruck} identifiziert der Germanist Wenzel diesbezüglich drei bedeutenden Umbrüchen \cite{wenzel_mediengeschichte_2007}: 
\begin{enumerate}
\item dem Übergang vom Körpergedächtnis (brain memory) zum Schriftgedächtnis(script memory)
\item dem Übergang von der Handschriftenkultur zur Druckkultur (print memory)
\item und dem Übergang vom Buch zum Bildschirm (electronic memory)
\end{enumerate}

Der Buchdruck änderte die Aufgabe von Wissenschaft und seine Orientierung an täglichem Bedarf \cite{Luhmann1998}. Die Geschichte des gedruckten Buchs beginnt mit Gutenberg \cite{wittmann_1999_geschichte} und er hat das Selbstverständnis der europäischen Kultur durch die neue Möglichkeiten der Vervielfältigbarkeit und Massenverbreitung verändert \cite{wunderlich_2008_buchdruck}, wie sie in der Geschichte (bisher) ohne Parallele ist \cite{giesecke_1991_buchdruck}. Die Entwicklung des Buchdrucks stellt "Grundlagen und Meilensteine sowohl für die Kommunikationder Menschheit insgesamt als auch fürden wissenschaftlichen Gedankenaustausch im Besonderen dar" \cite{schirmbacher_2009_wisspub} und leitete die Moderne ein \cite{luhmann_1997_gesellschaft}. 

Die Buchdrucktechnologie führte zu einem explodierenden Informationsangebot, zu einer Durchsetzung der "wissensschaftlich-systematische Methodik (...) gegen das mitteralterliche Denken in Bilder und Methaphern", sowie zu einem neuen Grad an und zur Befreihung des Autors aus der überwiegenden Anonymität mitteralterlicher Manuskriptkultur und zur Enkopplung von "Herstellung und Verbreitung vom singulären Interesse eines Autors, Kopisten oder Auftraggebers"\cite{wunderlich_2008_buchdruck} \cite{schirmbacher_2009_wisspub}. Der Prioritätsstreit zwischen Isaac Newton und Leibniz. In dem Streit geht es um eine Veröffentlichung zur Fluyionsrechnung von Newton, die Leibniz anonym rezensiert und sich selbst als Erfinder dieser darstellt.\cite{2013_leibniz} Dabei wurde Leibniz durch die Royal Society "nur deshalb des Plagiats für schuldig befunden, weil er seine bereits deutlich länger vorhandenen Erkenntnisse nicht veröffentlicht hatte" \cite{schirmbacher_2009_wisspub}. Der Buchdruck sowie die ersten wissenschaftlichen Zeitungen stellte für das wissenschaftliche Publizieren somit neben einem "Kommunikationsinsturment", einem Instrument zur "Erlangungvon Reputation" sowie einem Instrument "zur Generierungfinanzieller Erträge" auch ein "Nachweisinstrument" dar \cite{wunderlich_2008_buchdruck} \cite{schirmbacher_2009_wisspub}. 

Die Entwicklung eines Marktes für das gedruckte Wort führte zu einem "Verlust an Macht und Herrschaft über das geschriebene Wort" \cite{wunderlich_2008_buchdruck}. So bewirkte Gutenbergs Druckinnovation als sogenannte "Schlüsseltechnologie"\cite{jager_1993_theoretische} weder "unmittelbar Wissen für alle" noch handelte es sich dabei von Beginn an "um ein allgemeines, sondern um ein elitäres und teures Medium für die gebildete Klasse" \cite{hartmann_2008_medien}. Leibniz forderte in diesem Zusammenhang, dass Werke ohne Rücksicht auf Profitgier erscheinen und appelierte an eine "obrigkeitliche Lenkung", damit der Buchhandel "endlich wieder seiner Aufgabe der Verbreitung von nützlichem Wissen gerecht würde" \cite{wittmann_1999_geschichte}. Die Folge war die "Zensur als prohibitives Instrument für die Überwachung der Lektüren und zur Eindämung" von unerwünschter Literatur.

"Wer für den Druck schreibt, gibt die Situationskontrolle auf" und produziert für das Gedächtnis des Systems" durch den weder "Kommunikationsvorgang" noch der "Wissenszuwachs" abgeschlossen ist \cite{Luhmann1998}. Wurde "das Medium der Schrift" unter Buchdruckbedingungen "als eine Verbreitungstechnologie genutzt, welche die unmittelbare Interaktion zwischen Sender und Empfänger (weiterhin) ausschließt". Soll der Text "Wissen werden (...) muss er Leser finden" \cite{Luhmann1998}. Erst das Medium Internet eröffnete "Nutzungsmöglichkeiten, durch welche die Schrift als ein Medium einsetzbar wird, das den permanenten Wechsel zwischen Sender- und Empfängerposition ähnlich flexibel zu gestalten erlaubt, wie es im gesprochenen Gespräch der Fall ist" und "die Vorstellung von einem geschlossenen Sinngehalt (...) wird problematisch" \cite{sandbothe_2000_pragmatische}". 

Die ersten Experimente mit offenem Zugang und freien Lizenzen für Publikationen in der Wissenschaft gehen bis in die 60er Jahre des vorherigen Jahrhunderts und somit schon vor der Zeit der Erfindung des Internets, zurück \cite{cite:18b}. Noch bevor die digitalen Nutzungsmöglichkeiten verfügbar waren und bevor an das "globalen Dorf"\cite{mcluhan_1962_gutenberg} zu denken war, wurde vor allem in den Technik- und Naturwissenschaften eine “pre-print Kultur” entwickelt bei der die Autoren ihre zur Begutachtung eingereichten Artikel zeitgleich unter Kollegen zirkulieren ließen, um den Kommunikationsprozess zu beschleunigen \cite{suchen-Hoffmann-Zugang-undVerwertung-öffentlicher-Informationen}. 

In Deutschland nahmen bis Anfang der 1990er Jahre die wissenschaftlichen Verlage eine marktbeherrschende Stellung ein und waren exklusiver Dienstleister bei der Veröffentlichung wissenschaftlicher Informationen \cite{schloegl_2005} \cite{offhaus_2012_institutionelle_repos}. Die Entwicklung basiert auf dem in der Welt des geistigen Eigentums ungewöhnlichen Umstand, dass seit dem Beginn des wissenschaftlichen Journals im Jahr 1665, wissenschaftliche Autoren nicht finanzielle Belohnung profitierten sondern durch die weite Verbreitung und Hinweise auf ihre Arbeit sowie die Infromationen dahinter \cite{albert_2006_open_implications}. Darüber hinaus beruht das System auf der Eigenheit, dass Wissenschaftler sowohl Produzenten als auch Konsumenten der Wissenschaftskommunikation sind und damit Ihre eigene Zielgruppe darstellen \cite{Hess_2006}.

Die Vormachtstellung der Verlage im wissenschaftlichen Publikationssystem stüzt sich auf drei Säulen \cite{offhaus_2012_institutionelle_repos} \cite{bargheer_2006_open}: 
\begin{enumerate}
\item "Urheberrecht, wonach Verlage [...] weitgehende Ansprüche an dem veröffentlichten Werk erwerben“;
\item "redaktionelle Themenbündelung (bundling)“;
\item "Qualitätssicherung durch Begutachtung (Peer Review)"
\end{enumerate}

Durch diese marktbeherschende Stellung waren die Verlage in der Lage, drastische Preiserhöhungen durchzusetzen und es kam kurz vor der Jahrtausendwende zur sogenannten "Zeitschriftenkrise" \cite{schirmbacher_2009_wisspub}, die auch durch die Bildung von Bibliothekskonsortien, "deren Aufgabe es war  für Bibliotheken kostengünstige Rahmenbedingungen auszuhandeln" nicht gebändigt werden konnte. \cite{Fladung_2003} \cite{Brintzinger_2010}. Die Zeitschriftenkrise, "die richtigerweise Zeitschriftenpreiskrise oder Zeitschriftenpreisex-
plosion genannt werden müsste"\cite {Brintzinger_2010}, kam als Begriff das erste Mal in den 1990er Jahren auf \cite{Boni_2010}. Die Krise war das Ergbnis folgernder Herausforderungen auf der Angebots- und Nachfragenseite\ cite{Brintzinger_2010}:

\begin{enumerate}
\item auf der Angebotsseite wurden durch einen "Konzentrationsprozess" bei dem "innerhalb von etwas mehr als einem Jahrzehnt im Bereich der STM-Zeitschriften mittelständische Verlage nahezu vollkommen durch internationale Kapitalgesellschaften substituiert" \cite{Brintzinger_2010}
\item daruas resultierte eine "monopolistische Preispolitik" der Verlage \cite{schirmbacher_2009_wisspub}
\item die auf der Nachfrageseite durch Anstieg der Titelvielfalt, bei der aus einer mehr generalistischen Zeitschrift drei oder vier Spezialzeitschriften entstanden, gestützt
\item und von der "wenig beachteten" Ursache der institutionellen Organisation der Literaturbeschaffung in Hochschulen, im Rahmen Arbeitsteilung von Bibliothekaren und Wissenschaftlern - es für das Ansehen des einzelnen Faches ist es also durchaus rational, mit einem möglichst hohen Anteil an diesem Etat zu partizipieren; für individuelle Einsparungen gibt es keinen Anlass - begünstigt \cite{Brintzinger_2010} wurde.
\end{enumerate} 

Gleichzeitig standen die Wissenschaftler unter einem starken Publikationszwang, der mit "Publish or Perish" \cite{CLAPHAM_2005} beziehungsweise "impact factor fever" \cite{Cherubini_2008} und "impact factor race" \cite{Brischoux_2009} beschrieben wurde \cite{offhaus_2012_institutionelle_repos}. 

"Die Zeitschriftenkrise" und der Publikationszwang sind jedoch zwie von vielen Aspekten für die Entwicklung von Open Access \cite{Brintzinger_2010}: Auf Grundlage der ersten Früchte der Digitalisierung gründete Anfang der 1990er der Physiker Paul Ginsparg mit arXiv den ersten wissenschaftliche Preprint-Dienst des Internets \cite{suchen}, der es Wissenschaftlern ermöglichen sollte Ideen vor der gedrukten Veröffentlichung zu teilen. Vier Jahre später forderte Steven Harnad die wissenschaftliche Community dazu auf sofort mit der digitalen Selbstarchivierung und öffentlichen Zurverfügungstellung ihrer Beiträge zu beginnen \cite{albert_2006_open_implications}, um "den Barrieren, die zwischen ihrer Arbeit und ihrer (kleinen) Leserschaft aufgestellt werden, zu entkommen" \cite{harnad_1995_subversive_proposal}. 1998 wurde mit der Scholarly Publishing and Academic Resources Coalition (SPARC) einer der späteren "major player" der Open Access Bewegung\cite{russell2008business} \cite{Herb_2012} gegründet. Als Konsequenz aus der Zeitschriftenkrise sollte die Allianz zwischen Universitäten und wissenschaftlichen Bibliotheken dafür sorgen, dass die Kosten für wissenschaftliche Zeitschriften reduziert oder durch die Bereitstellung kostengünstiger oder freier, nicht-kommerzieller, Peer-Review-Fachzeitschriften ersetzt werden. Durch Weiterbildung, politische Arbeit und die Förderung alternativer Geschäftsmodelle wurde angestrebt, Initiativen für offenes wissenschaftliches Publizieren zu stimulieren \cite{suchen}.

In 2001 erschien Open Access erstmals als eigenes und öffentlichkeitswirksames Thema im wissenschaftlichen Diskurs \cite{cite:19}. Die Public Library of Science (PLoS), gegründet im Oktober 2000, forderte Wissenschaftler in einem offenen Brief im Mai 2001 dazu auf, ab September 2001 nur noch in den Zeitschriften zu veröffentlichen, beziehungsweise nur noch die Zeitschriften zu reviewen, zu editieren und zu abonnieren, deren Beiträge spätestens sechs Monate nach ihrer Erstveröffentlichung für jedermann im Internet kostenlos und unentgeltlich einsehbar sind \cite{cite:20}. Schon nach kurzer Zeit unterzeichneten (nach eigenen Angaben \cite{cite:19a}) rund 38.000 Wissenschaftler aus 180 Nationen das Schreiben. Dieser Brief kann als Auftakt zu einem 20-monatigen theoretischen Schub gesehen werden. Neben PLoS wird der im Jahr 2000 gegründete britische Verlag Biomed Central als "Wegbereiter in der von OA" \cite{suchen-Hoffmann-Zugang-undVerwertung-öffentlicher-Informationen}. In dem Zeitraum entstanden drei der bis heute wichtigsten Erklärungen im Bereich der Öffnung des Zugangs zu wissenschaftlicher Kommunikation \cite{CREATe_2014}: 

\begin{enumerate}
\item Erklärung der Budapest Open Access Initiative (Dezember 2001 und 2012)

Im gleichen Jahr wie der PLoS-Brief, wurden im Rahmen einer Konferenz des Open Society Institutes in Budapest, mit der “Budapest Open Access Initative” (BOAI)\cite{boai_2012} erstmals die Bemühungen um Open Access in einer eigenen Erklärung zusammengefasst\cite{cite:21a}. Im Fokus dieser steht die Forderung nach dem freien Zugang zu wissenschaftlichen Publikationen. In der BOAI wird erstmals manifestiert, dass wissenschaftliche Peer-Review-Fachliteratur “kostenfrei und öffentlich im Internet zugänglich sein sollte, so dass Interessenten die Volltexte lesen, herunterladen, kopieren, verteilen, drucken, in ihnen suchen, auf sie verweisen und sie auch sonst auf jede denkbare legale Weise benutzen können, ohne finanzielle, gesetzliche oder technische Barrieren jenseits von denen, die mit dem Internet-Zugang selbst verbunden sind. In allen Fragen des Wiederabdrucks und der Verteilung und in allen Fragen des Copyrights überhaupt sollte die einzige Einschränkung darin bestehen, den Autoren Kontrolle über ihre Arbeit zu belassen und deren Recht zu sichern, dass ihre Arbeit angemessen anerkannt und zitiert wird."\cite{boai_2012} 

Anlässlich des zehnten Jahrestages der BOAI, wurde von der Open Society Foundation mit der BOAI 10 (2012) die usrprüngliche Erklärung bestärkt und anhand von weitere Richtlinien und Empfehlungen die Entwicklungen und Herausforderungen in seiner zehnjährigem Bestehen adressiert. Die Initiatoren kommen unverändert zu dem Schluss, dass "noch immer Zugangsbeschränkungen zu Peer-Review-Forschungsliteratur, meist eher zugunsten der Verlage, als zugunsten der Autoren, Reviewer oder Redakteure und damit auch auf Kosten der Forschung, Forscher und Forschungseinrichtungen" \cite{boai_2012} bestehen. "Nichts aus den letzten zehn Jahren" lässt "darauf schließen, dass das ursprüngliche Ziel von OA weniger sinnvoll oder erstrebenswert erscheint. Im Gegenteil, die Notwendigkeit, dass Wissen für jeden, der es nutzen, anwenden oder darauf aufbauen kann, offen verfügbar sein sollte, ist dringlicher als je zuvor" \cite{boai_2012}.

\item Die Bethesda Erklärung (Juni 2003)

Zwei Jahre nach Veröffentlichung der initalen Version der BOAI-Erklärung, im Juni 2003, verabschiedete im US-Bundesstaat Maryland eine Gruppe von Forschungsförderern, wissenschaftlicher Gesellschaften, Verlegern, Bibliothekaren, Forschungseinrichtungen und einzelner Wissenschaftler das "Bethesda Statement on Open Access Publishing".\cite{suchen} Ziel der Erklärung war die Stimulation der Diskussion in der biomedizinischen Forschung, "wie man schnellstmöglich den offenen Zugang zu der primären wissenschaftlichen Literatur in der Biomedizin erreichen könnte"\cite{suchen}. Wie bereits in der BOAI erklärten die Autoren des "Bethesda Statement on Open Access Publishing" Bedingungen für den offenen Zugang zu wissenschaftlichen Publikationen \cite{suchen}: 

Erstens werden Autor(en) und Urheberrechts-Inhaber aufgefordert für alle Benutzer eine freies, unwiderrufliches, weltweites und unbefristetes Recht auf den Zugang zulassen, sowie eine Lizenz zu verwenden, die das Kopieren, Nutzen, Verbreiten, Übertragen und öffentliches Darstellen der Publikation ermöglichen. Darüber hinaus muss es erlaubt sein, abgeleitete Werke zu verteilen, in jedem digitalen Medium für jeden Zweck zu veröffentlichen, vorbehaltlich einer angemessenen Zuordnung der Urheberschaft. Das beinhaltet auch das das Recht auf eine kleine Anzahl gedruckter Kopien für den persönlichen Gebrauch. 

Zweitens, muss eine vollständige Version der Arbeit und aller ergänzender Materialien, einschließlich einer Kopie der Genehmigung, wie oben erwähnt, in einem geeigneten elektronischen Standardformat sofort bei der ersten Veröffentlichung in mindestens einem Online-Repositorium, das von einer wissenschaftlichen Einrichtung unterstützt wird hinterlegt werden. Dieses Repositorium muss von einer wissenschaftlichen Gesellschaft, Regierungsbehörde oder einer anderen etablierten Organisation akzeptiert sein. Diese muss sich für einen offenen Zugang, uneingeschränkte Verbreitung sowie Interoperabilität und Langzeitarchivierung (für die biomedizinischen Wissenschaften, PubMed Central ist ein solches Repository) verpflichtend einsetzen.

\item Die Berliner Erklärung (Oktober 2003)

Ein weiterer Meilenstein für die Verbreitung von Open Access auf dem europäischen Kontinent waren die "Berlin Konferenzen"\cite{CREATe_2014}. Die erste Tagung wurde 2003 von der Max-Planck-Gesellschaft und dem Projekt European Cultural Heritage Online (ECHO) organisiert, um über "Zugangsmöglichkeiten zu Forschungsergebnissen" zu diskutieren. In diesem Rahmen entstand 2003 auch die "Berliner Erklärung über den offenen Zugang zu wissenschaftlichem Wissen" \cite{berliner_erklaerung_2003}, in der die Verfasser über die Budapester Erklärung hinaus gehen und neben dem kostenlosen und freien Zugang zu wissenschaftlichem Wissen in Form von Publikationen auch den freien und offenen Zugang zu den Daten fordern. „Open Access-Veröffentlichungen umfassen originäre wissenschaftliche Forschungsergebnisse ebenso wie Ursprungsdaten, Metadaten, Quellenmaterial, digitale Darstellungen von Bild- und Graphik-Material und wissenschaftliches Material in multimedialer Form.“ \cite{berliner_erklaerung_2003} Es formiert sich ein erweitertes Verständnis von Open Access und es entsteht damit die Grundlage für ein erste Ansatzpunkte zur Definition von Open Science. Dennoch konzentriert sich die Diskussion in diesem Stadium noch ausschließlich auf den bereits abgeschlossenen wissenschaftlichen Prozess \cite{suchen}.

\end{enumerate}

Alle drei Erklärungen, auch die "three B's"\cite{suber_2004_praising_oa} genannt, gelten als die angesehensten Definitionen von Open Access und sind in wesentlichen Merkmalen in sich stimmig\cite{albert_2006_open_implications}.

Im Jahr 2003 entstand das Directory of Open Access Journals (DOAJ) und damit eine zentrale Anlaufstelle für OA-Journale \cite{suchen-Hoffmann-Zugang-undVerwertung-öffentlicher-Informationen}. Vorangegangen war im Jahr 2002 die Entwicklung und veröffentlichung der Creative Commons Initiative \cite{suchen-Hoffmann-Zugang-undVerwertung-öffentlicher-Informationen} und ihrer ersten Version der Lizenzen, die bis heute als rechtliche Grundlage für eine Vielzahl der Open Access Publikationen dienen\cite{suchen}. Die modularen Lizenzen sind im Kontext von Open Access wichtig, "um (Nach)nutzungsmöglichkeiten für Texte, Daten und andere wissenschaftliche Erzeugnisse festlegen zu können" \cite{suchen-Hoffmann-Zugang-undVerwertung-öffentlicher-Informationen}.

In der Debatte über die Zukunft des wissenschaftliche Publizierens und Kommunizierens neigt man dazu, Konzepte der offene Wissenschaft als einen bisher beispiellosen und noch nie dagewesenen Paradigmenwechsel darzustellen \cite{cite:17a} \cite{cite:17b}. Dabei basiert der Paradigmenwechsel auf "verschiedenen Gründungsmythen", die auf "unterschiedliche Zielsetzungen und Lösungspfade" verweisen \cite{suchen-Hoffmann-Zugang-undVerwertung-öffentlicher-Informationen}. Die Geschichte von Open Access ist also eine Geschichte, die mit der Digitalisierung von Kommunikationsprozessen \cite{albert_2006_open_implications} auf der einen und mit der Zeitschriftenkrise auf der anderen Seite verknüpft ist \cite{suchen-Hoffmann-Zugang-undVerwertung-öffentlicher-Informationen}. Open Access ist kein Selbstzweck\cite{cite:17d}, sondern ein Symptom für tiefergehende Prozesse die mit der wachsenden Bedeutung der Digitalisierung in unserer Zivilisation und dem damit einhergehenden Wandlungsprozessen im Machtgefüge zusammenhängen\cite{cite:17e}. Dennoch, obwohl es vorher schon vereinzelte Versuche in der Wissenschaft gab, komplett Informationen und Publikationen offen und frei zu kommunizieren, war Open Access im Printzeitalter physisch und ökonomisch über lokale Grenzen hinaus schwer möglich \cite{cite:18a}. 

Die Forderung nach Öffnung von Wissenschaft und Forschung ist in diesem Zusammenhang nicht nur eine "politische Reaktion" oder "technische Alternative", sondern muss "als alternative Formatierungen einer wissenschaftlichen Infrastruktur im technischen, rechtlichen und zeitlichen Sinne" \cite{kelty_2004} aufgefasst werden und betrifft "Wissenschaftler, politische Entscheidungsträger und die Öffentlichkeit" \cite{Scheliga_2014}.

\section{Open Science}
In diesem Kapitel soll Open Science (medien)kulturwissenschaftlich sowohl in technischen als auch in gesellschaftlichen und politischen Aspekten definiert und abgegrenzt werden.

Der Sammelbegriff Open Science erstreckt sich über die gesamte wissenschaftliche Wertschöpfungskette \cite{Scheliga_2014}: Vom offenen Zugang zur Publikationen wissenschaftlicher Forschung (Open Access), sowie den ganzheitlichen wissenschaftlichen Erkenntnisprozess. Unter diesem Gesichtspunkt kann Open Science als eine Weiterentwicklung von Open Access bezeichnet werden. Die diesbezügliche Evolution des Konzepts von Open Access führt zu einem direkten und breiten Weg, Wissenschaft an jedem Schritt der wissenschaftlichen Wertschöpfungskette zu kommunizieren und zu transferieren. Open Science ist die Reaktion auf die Forderung nach offenem Zugriff auf Wissenschaft und Forschung und kann dazu führen, "dass sich die Bedeutung von Forschungsergebnissen zukünftig nicht mehr auf sogenannte klassische wissenschaftliche Publikationen (im Format von Einleitung – Methoden – Ergebnisse – Diskussion), sondern die globale Echtzeitpublikation von Originaldaten stützen wird" \cite{Stengel_2013}.

\subsection{Chronologie der Bewegung}

Die Entwicklung von “Open Science” knüpft an die Entwicklung der Open Access-Bewegung an und kann als Folge der neuen Möglichkeiten für kollaboratives Arbeiten im Rahmen der Digitalisierung und neuer Kommunikationstechniken verstanden werden. Unter Berücksichtigung der Frage, wie der gesamte wissenschaftliche Wertschöpfungsprozess der Allgemeinheit zur Verfügung gestellt werden kann, wird Open Science abgegrenzt und definiert. 

Dennoch spielt insbesondere die Entwicklung der Tradition für eine "offenen Wissenschaft" im siebzehnten Jahrhundert einen Ansatzpunkt, da dieser historische Übergang noch nicht erforscht ist \cite{CREATe_2014}. Die Verschlüsselungs- und Patentwut zur Wahrung eines möglichen kommerziellen Vorteils durch Wissenschaft im Rahmen öffentlich-finanzierter Forschung, geht dabei bis auf die xxxx Jahre zurück. ---- TODO: sprung entfernen und Beispiel Galileo, Kepler, Newton ----  Das Ergebnis dieser Wut war eine Debatte über die Verfügbarkeit der wissenschaftlichen Arbeit und die Entlohnung der “Erfinder“ im wissenschaftlichen System. 

Im April 2012 wurde die Erklärung "Open Science for the 21st century", vom Zusammenschluss der Europäischen Akademien (ALLEA) verabschiedet \cite{ALLEA_2012}. Sie war nur eine von mehreren Erklärungen und Positionspapiere für die Öffnung von Wissenschaft durch international angesehenen Einrichtungen, durch die verdeutlich wurde, dass die Forderung nach offenem Umgang mit Wissen und Information im wissenschaftlichen Bereich zunehmend an Relevanz gewinnt \cite{schulze_2013_open}.

--- TODO: Weitere Entwicklung darstellen ----


\section{Wissenschaftliche Reputation}

Wissenschaftliche Reputation ist eine "Art von Kredit" \cite{luhmann_1970_selbststeuerung}. Ein wesentliches Prinzip des Wissenschaftssystems basiert auf der "gegenseitigen Beurteilung und Anerkennung der jeweils neuen Ergebnisse der Fachkollegen (Peers) durch die Wissenschaftler selbst"\cite{Hanekop_2014}, also teils auf der "Generalisierung von Einzelleistungen", auf "gegenseitiger Ansteckung" und teils "auf der bloßen Häufigkeit der Publikationen oder der Anwesenheit an renomierten Plätzen" \cite{luhmann_1970_selbststeuerung}. Dabei gesteht auch Luhmann die Existenz von "Nebencodes der Reputation" zu \cite{schmoch_2003_hochschulforschung}. Die Reputation steuert die wissenschaftliche Aufmerksamkeit und die Verteilung von motivierenden Effekten, die sich durch das reine Streben nach erkenntnis nicht erzeugen lassen \cite{suchen_luhmann}.

Als "guter akademischer Forscher" gilt nur der, "wer viel und in möglichst angesehenen Journalen" veröffentlicht \cite{Frey_2005}. Dabei spielt der Peer-Review-Prozess eine zentrale Rolle und ein Kernelement der Selbststeuerung von Wissenschaft \cite{Neidhardt_2010}. In dem Peer-Review Prozess "werden eingereichte Beiträge von fachlich versierten Wissenschaftlern (...) beurteilt und gemäß den qualitativen Anforderungen der Forschungs-Community zur Veröffentlichung angenommen oder abgelehnt" \cite{Hess_2006}. "Peer-Review" beschränkt sich dabei nicht nur auf den Prozess der Publikation von Texten, sondern deckt ein breites Spektrum von Aktivitäten über Fachdisziplinen ab: die Beobachtung der klinischen Praxis (z.B. in der Medizin); Beurteilung des Lehrenden, Fähigkeiten der Kollegen; Bewertung durch Experten bei der Forschungsförderung und Stipendien bei Einreichung von Anträgen an staatliche und anderen Förderorganisationen; Begutachtung von Redakteuren und externen Gutachtern bei Artikeleinreichungen für wissenschaftlichen Zeitschriften; Bewertung von Papieren und Plakate für Konferenzen; Bewertung von Buchvorschlägen für Universitätverlagen oder andere Verlagenn; und Einschätzungen der Qualität, Anwendbarkeit und Interpretierbarkeit von Datensätzen und wissenschaftlichen Organisationen" \cite{Lee_2012}. Dennoch bilden "Publikationen im Hinblick auf die Funktion der Reputationsverteilung eine Art Telos wissenschaftlicher Kommunikation "\cite{hirschauer2004peer}. Im Rahmen von Reputation ist wissenschaftliche Arbeit besonders auf funktionierendes Peer-Review-System angewiesen \cite{suchen}. Das Verfahren hat zwei Funktionen: 1. die Selektionsfunktion, in deren Rahmen die Auswahl von Personen, Projekten und Texten stattfindet und 2. eine Konstruktionsfunktion in der Gutachter "produktiv in den Wissenschaftsprozess eingreifen" und die eigenen Fachstandards durchzusetzen \cite{Neidhardt_2010}. Dennoch haben qualitatives Peer Review-Systeme und quantitative bibliometrischen Verfahren objektiv viele Mängel \cite{osterloh2008anreize} \cite{Lee_2012} \cite{Jansen_2007}.

Folgende weitere Indikatoren werden für wissenschaftliche Reputation für wissenschaftliche Institutionen und Personen genannt\cite{hanekop_2008}:
\begin{enumerate}
\item \textbf{Drittmittelprojekte} - Drittmittel, sind, so der Wissenschaftsrat bereits 1993, "solche Mittel, die zur Förderung der Forschung und Entwicklung sowie des wissenschaftlichen Nachwuchses und der Lehre zusätzlich zum regulären Hochschulhaushalt(Grundausstattung) von öffentlichen oder privaten Stellen eingeworben werden" \cite{wr_2014}. Die Drittmitteleinwerbung hat sich in Deutschland als "meist gebrauchter Maßstab der Messung von Forschungsqualtität durchgesetzt" \cite{M_nch_2006}. Das ging mit einer zunehmenden Finanzierung der Forschung über Drittmittel einher \cite{Neidhardt_2010} \cite{Jansen_2007} \cite{simon_2009_wissenschaft_governance}. Dabei spielt die Unterscheidung eine Rolle, ob die Publikationen, die im Rahmen der Drittmittelfinanzierung veröffentlicht werden, aber auch der Antrag um Drittmitteleinwerbung selbst ein "zum Erkenntnisfortschrit in der wissenschaftlichen Gemeinschaft beiträgt" \cite{M_nch_2006}. Kritisch ist dabei festzuhalten, dass der Indikator Drittmittel ein instrumentelles Ziel ist, das durch die Knappheit öffentlicher zu einem Kernziel geworden ist \cite{Jansen_2007} und dadurch zunehmend direkte finanzielle und administrative Kontrolle eine Rolle spielen \cite{Barl_sius_2008}.
\item \textbf{Patente} - "Unter einem Patent versteht man das vom Staat verliehene Schutzrecht für einetechnische Erfindung, welches dem Patentinhaber für eine bestimmte Zeit dieausschließliche wirtschaftliche Nutzung der Erfindung vorbehält." \cite{greif_2003_patente} Seit dem 1970er Jahen wird eine steigende Anzahl von Patente aus dem Hochschulbereich verzeichnet \cite{schmoch_2003_hochschulforschung}. Dabei wird die Patentschrift "als funktionale äquivalent zur wissenschaftlichen Publikation begriffen" und bewertet \cite{mersch_2014_patente}. "Patente leisten einen Beitrag zur Förderung der Wissenschaft, die Grundlagen des Patentwesens sind daher dem wissenschaftlichen Nachwuchs über entsprechende Lehrangebote zu vermitteln."
\item \textbf{Vorträge} - Vorträge dienen der Verbreitung der Forschungserkenntnisse und ermöglichen das Aufgreifen des Wissens durch andere \cite{rassenhoevel_2010_performancemessung}. Vorträge stellen eine schnelle Form für die Verbreitung neuer wissenschaftliche Erkenntnisse und Ergebnisse dar, ohne dass jeder Gedanke genauer belegt werden muss und die sich gegebenenfalls später wieder teilweise einfangen lassen \cite{haberle_2002_jahrbuch}. 
\item Anwendungsrelevanz bzw. Verwertbarkeit
Anwendungsrelevanz stellt eine neue relevante für Hochschulen und ausseruniversitäte Forschungsinstitute dar \cite{simon_2009_wissenschaft_governance}. Sie bezieht sich auf einen Outputfaktor, der sich primär auf den Einsatz der gewonnen wissenschaftlichen Erkenntnisse bezieht und eher auf eine Verwertbarkeit für Produkte oder Patente als auf wissenschaftliche Veröffentlichung abzielt \cite{suchen}.
\item Netzwerke - \textbf{Netzwerke} beschreiben meist informelle Verbünde zwischen Wissenschaftlern. Sie erlauben den schnellen Austausch und können Grundlage für Aktivitäten zur Steigerung der wissenschaftlichen Reputation darstellen (gemeinsame Publikationsvorhaben, Austausch von wissenschaftlichen Erkenntnissen usw.). Diese können einen Nutzen für den Wissenschaftsbereich darstellen .
\item \textbf{öffentliche Aufmerksamkeit sowie politische Relevanz} -
Die öffentliche Aufmerksamkeit stellt zum einen eine Möglichkeit des Wissenstransfers ausserhalb der wissenschaftlichen (Fach-)Community dar zum Anderen ermöglicht sie Einflussnahme auf die politische Relevanz von wissenschaftlichen Forschungsthemen. Die Veröffentlichung von wissenschaftlichen Informationen zu einem bestimmten Thema des öffentlichen Interesses, stellt eine Möglichkeit dar, das Thema zu besetzen. Politische Relevanz im Rahmen von Wissenschaft und Forschung ermöglicht stellt einen Faktor für die Ressourcengewinnung dar.
\item \textbf{Renommee der Forschungseinrichtung} - 
Das Renomee einer Forschungseinrichtung, damit ist die Wahrnehmung der Einrichtung innerhalb und außerhalb der wissenschaftlichen (Fach-)Community gemeint, hat für Wissenschaftler eine besondere Bedeutung\cite{mayntz_2008_wissensproduktion}. Es basiert auf "Ansteckung" in dem zum Beispiel rennomierte Professoren den Ruf einer Fakultät aufbessern oder umgekehrt \cite{luhmann_1970_selbststeuerung}. Am Beispiel von Publikationen, nimmt ein Autor an Renommee einer Einreichtung teil, wenn er durch ihre Publikationsorgane veröffentlicht \cite{lutz_2012_zugang}.
\item \textbf{materielle Ausstattung, Großgeräte etc. & personelle Ausstattung} - 
Die Materielle Ausstattung beschreibt die Rahmenbedinungne, in der ein Wissenschaftler arbeitet. Sie hat wie das Renomee eine wichtige Bedeutung bei der Überlegung von Wissenschaftlern einen Wechsel zu vollziehen \cite{mayntz_2008_wissensproduktion}. Die personelle Ausstattung stellt einen weiteren Indikator für wissenschaftliche Reputation beziehungsweise ihr Ansteckungspotenzial dar. Materielle und personelle Ausstattung gemeinsam sind vor allem bei traditionellen Berufungsverfahren deutscher Professorinnen und Professoren von besonderem Belang \cite{himpele_2011_job}, da sie die Arbeitsfähigkeit und die Annerkennung direkt beeinflussen \cite{suche}. 
\item \textbf{Gutachtertätigkeit} - Gutachter werden zum Beispiel in Peer-Review-Verfahren Autoren aus ihrem Fachgebiet zugeordnet und entscheiden über die Veröffentlichung des Textes oder weisen diesen zurück \cite{Frey_2005}. Dieses Verfahren wird bei manchen Publikationen mehr als drei mal durchlaufen, bevor der Artikel akzeptiert und publiziert wird \cite{Frey_2005}. Die Reputation der mit diesem Verfahren betrauten Gutachter wirkt sich auch auf das Image des Verlages aus und umgekehrt. Die Gutachtertätigkeit ist aber nicht nur Bestandteil des wissenschaftlichen Qualitätssicherungs- und interdependenten Reputationssystems \cite{Frey_2005}, sondern stellt auch einen informellen Weg der Kommunikation über eigene Publikationen mit Verlagen oder Herausgebern dar. Darüber hinaus ermöglichen sie die Vorabsichtung von neusten wissenschaftlichen Informationen und Erkenntnissen.
\item Herausgeberschaft - Ähnlich wie die Gutachtertätigkeit ist auch die Herausgeberschaft fester Bestandteil des interdependenten wissenschaftlichen Reputationssystems \cite{Frey_2005}. Herausgeber profitieren von den Autoren beziehungsweisen von deren publitzierten Ergebnissen in ihren Publikationen sowie von der Reputation dieser. Sie selber geben aber auch Reputation an den Verlag oder die Publikation ab.
\item Funktion
Die jeweilige Funktion
\end{enumerate}

Wissenschaftliche Reputation wird unter anderem als Währung bezeichnet, mittels derer “Status und Ressourcen verteilt werden” \cite{hanekop_2006}. Sie verteilt sich auf Einrichtungen und einzelne Personen, die wissenschaftlich tätig sind \cite{suchen}. Die Evaluation wissenschaftlicher Einrichtungen findet dabei über “Beobachtungen und Gespräche mit den Wissenschaftlern vor Ort sowie über den Austausch über die Eindrücke innerhalb der Begehungsgruppe und die gemeinsame Verständigung”\cite{Barl_sius_2008} statt. 

Die Reputation einzelner Wissenschaftler steht in enger Abhängigkeit zum bestehenden wissenschaftlichen Kommunikationssystem \cite{suchen}. Für die Wissenschafler sind Publikationen und die damit einhergehende Verbreitung wissenschaftlicher Erkenntnisse sehr entscheidend \cite{Hess_2006}. Vereinfacht lässt sich das System der Wechselbeziehungen der Reputationsverteilung im Rahmen von Publikationen wie folgt darstellen \cite{cite:21a}: 

Grafik aus Text von Bernius
http://www.eap-journal.com/archive/v39_i1_8_bernius.pdf

Bernius et al. unterscheiden drei aufeinandertreffende koordinierenden Marktmechanismen: die Reputation und die Nutzung wissenschaftlicher Publikationen, sowie der Preis für den Erwerb \cite{suchen}. Während die Reputation ein non-monetärer Aushandlungsmechanismus zwischen wissenschaftlichen Verlagen und wissenschaftlichen Autoren ist, findet die monetäre Preisdefinition zwischen Bibliotheken und Verlagen statt. Der monetäre Aushandlungsprozess zwischen Wissenschaftlern und Bibliotheken fokussiert sich auf die Bedeutung und Nutzung der jeweiligen Publikation \cite{cite:21a}. Nicht jede Publikation hat diesbezüglich die gleiche Wertigkeit \cite{suchen} und damit den gleichen Einfluss auf Reputation des Autors. 

Die neuen Möglichkeiten der Verbreitung von Informationen lassen deshalb einen vergleichbaren Veränderungsprozess der wissenschaftlichen Reputation und damit auch Anerkennung vermuten, wie sie durch die Entwicklung des Buchdrucks ausgelöst worden war.\cite{hanekop_2006} 

Der US-amerikanische Soziologe Robert K. Merton stellt vier Grundprinzipien als normative Struktur der wissenschaftlichen Reputation vor \cite{Merton_1985}:

---- TODO: Ausarbeiten ----

\subsection{Messbarkeit wissenschaftlicher Qualität vs. Publikationsquantität}
Wissenschaft ist ein Prozess, bei dem aus “unterschiedlichen Inputfaktoren, mittels verschiedener Transformationen Beiträge zur Schaffung neuer wissenschaftlicher Erkenntnisse als Output entstehen”\cite{Jansen_2007}. Die Bewertungen des jeweiligen Outputs führt “zur Ausage über die Forschungsperformanz”. Neben den Indikatoren für den Output wissenschaftlicher Perfomanz, müssen aber auch intermediäre Aspekte berücksichtigt werden\cite{schmoch_2009}. Nach dem zweiten Weltkrieg etablieren sich die ersten Indikatoren für die Effizienzmessung wissenschaftlicher Wissensproduktion und -verbreitung. Spätestens seit den 1960er Jahren werden diese Messungen in Gestalt von Indikatoren (--- TODO Was sind die Indikatoren? erklären! ----), die die Forschungsleistung quantifizieren sollen, flächendeckend durchgeführt \cite{suchen}. Seit den 1990er Jahren ist diese Bewertung in Gestalt von Zahlen als unkontrollierte Nebenprodukte digitaler Wissenskommunikation erweitert worden \cite{angermueller_2010}. Heute zählen in der Wissenschaft vor allem die wissenschaftliche Reputation und die als Impact bezeichnete Wirkung wissenschaftlicher Publikationen\cite{herb_open_2013}. Die Wirkung wird dabei anhand der Zitationen der jeweiligen Publikation gemessen \cite{suchen}. Eine häufige Zitation stellt dabei einen Indikator für einen große Wirkung der wissenschaftlichen Arbeit dar. 

---- TODO: Weiter ausarbeiten -----

\subsection{Wissenschaftliches Kapital}
Im Rahmen der Betrachtung von Steuerungs- und Reputationsmethoden für die Wissenschaft ist der Begriff "wissenschaftliches Kapital" von herausragender Bedeutung \cite{suchen}. Wissenschaftliches Kapital kann als eine Ausprägung des kulturellen Kapitals und als symbolisches beziehungsweise, non-monetäres Kapital \cite{irmer2011} verstanden werden. 

Die “Gewährung wissenschaftlichen Kapitals” im wissenschaftlichen System basiert heute auf einer engen Verbindung zwischen publizierenden Wissenschaftlern und Verlagen \cite{herb_2006}. Die Wissenschaftler stehen in einer klaren Abhängigkeit zu den Verlagen. Ulrich Herb definiert mit Hilfe Pierre Bourdieus, "wissenschaftliches Kapital" als “Ergebnis einer Investition (...), die sich auszahlen muss”. “Diejenigen, die diese Berechtigungsscheine in der Hand halten, verteidigen ihr 'Kapital' und ihre 'Profite', indem sie diejenigen Institutionen verteidigen, die ihnen dieses 'Kapital' garantieren.” \cite{Bourdieu_1992} Herb kommt zu dem Schluss, dass die Öffnung der Wissenschaft dabei bisher nicht wissenschaftlicher Logik folgt, “sondern einer feldunabhängigen Logik der Akkumulation von Kapital”\cite{herb_2006}. Hinzu kommt, dass vor allem das deutsche Wissenschaftssystem zunehmend von der Einführung an Output orientierter Anreizsysteme gekennzeichnet ist \cite{osterloh2008anreize}.

Als Beispiel für wissenschaftliches Kapital kann der Performanzindikator "Drittmittel" \cite{Jansen_2007} dienen, durch die in der Wissenschaft neben der Sicherung der Qualität von Forschung und Lehre zunehmend direkte finanzielle und administrative Kontrolle eine Rolle spielt \cite{Barl_sius_2008}. Daraus resultiert die Gefahr, dass nicht nur die Erwartungen an die Bewertung von Wissenschaft zu hoch gegriffen sind, sondern auch, dass sich die Wissenschaft zu sehr an diesen Erwartungen orientiert und die Interessen privater und öffentlicher Drittmittel-Auftraggeber in den Vordergrund rücken.  Vor allem die Verknüpfung von wissenschaftlicher Reputation und der damit einhergehenden Verteilung der Mittel und Stellen stellt eine Herausforderung an das Wissenschaftsystem dar, “dessen Währung [ursprünglich] nicht Geld ist” \cite{hanekop_2006}. 

--- TODO: Freiheit von Lehre und Forschung ----
Wahrung des in Artikel 5 Abatz. 3 GG garantiertes Grundrechts auf "Freiheit von Lehre und Forschung". In , das einerseits eine Garantie der Einrichtung wissenschaftlicher Hochschulen mit Anspruch auf Selbstverwaltung und Sicherung ihrer Arbeit durch den Staat beinhaltet und andererseits dem einzelnen Wissenschaftler ein subjektives Recht auf Nichteinmischung des Staates in seine wissenschaftliche Tätigkeit gibt. --- TODO: Freiheit von Lehre und Forschung ----

\subsection{Ökonomie der wissenschaftlichen Kommunikation}
Die klassische Ökonomie der wissenschaftlichen Kommunikation beruht auf der Durchsetzung von Urheberrechten, die den Zugriff auf und die Wiederverwendung von geschützten Inhalten beschränken sowie die Zahlung einer Gebühr durch den Leser verlangen, um Zugang zu der Veröffentlichung zu erhalten \cite{CREATe_2014}. Bislang werden dafür "in der Regel wissenschaftliche Arbeiten zwar mit öffentlichen Mitteln finanziert, aber von privaten Verlagen in Fachzeitschriften herausgegeben" \cite{WD_bundestag_2009}. Diese Ökonomie der Wissenschaftsverlage ist nicht neu und hat sich im Laufe der Zeit weiter ausdifferenziert. Dieses Modell baisert auf einer sozial ineffizientem Ebene\cite{mueller-langer_2010}. Die Wahrnehmung der Unverhältnismäßigkeit dieses Systems, insbesondere der Preisgestaltung für wissenschaftliche Publikationen \cite{King_2008} findet aber erst seit kurzem statt\cite{CREATe_2014}. 


Eine weitere wesentliche Besonderheit der Wissenschaftskommunikation ist die Organisation des Marktes, die von spezifischen Akteuren und Prozessen geprägt wird \cite{Hess_2006}. Beim wissenschaftlichen Publizieren kann von einem "einem gesellschaftlich bedingten Kreislauf" \cite{schirmbacher_2009_wisspub} gesprochen werden. Der klassische wissenschaftliche Kommunikationsprozess im Rahmen von Publikationen kann wie folgt unterteilt werden\cite{cite:11b} \cite{Hess_2006}:
\begin{enumerate}
\item Erstellung durch Wissenschaftler - Inhalte erzeugen: 
Der Kreislauf beginnt mit der "Darstellung des geistigen Werkes durch die Autoren"\cite{schirmbacher_2009_wisspub}. Nach der Entwicklung eines konkreten Forschungsvorhabens sowie einer wissenschaftlichen Fragestellung enstehen im Rahmen der wissenschaftlichen Forschung oder der jewiligen Untersuchung Informationen\cite{cite:11c}, die im Forschungsprozess gesammelt, analysiert, ausgewertet, aufbereitet und verschriftlicht werden\cite{cite:11d}. Diese Infromationen werden strukturiert zusammengefasst und niedergeschrieben \cite{Hess_2006}.
\item Qualitätskontrolle durch Wissenschaftler - Inhalte bewerten: 
Die Qualitätskontrolle ist einer der wesentlichen Bestandteile der wissenschaftlichen Kommunikation. Sie sichert die gewonnen Erkenntnisse\cite{cite:11e} und stellt einen klaren Abrenzungsaspekt zu nicht-wissenschaftlichen Informationen und Erkenntnissen dar\cite{cite:11f}. Sie findet im Kommunikationsprozess an zwei Stellen statt. Hier ist die Stelle gemeint, in der vor der Produktion der Informationen in Form einer Publikation, die Erkenntnisse von anderen Wissenschaftlern überprüft und gesichert (Peer-Review) \cite{Hess_2006} sowie vom Verlag organisiert werden \cite{schirmbacher_2009_wisspub}.
\item Bündelung durch Verlage - Inhalte auswählen:
Verlage bündeln in Zusammenarbeit mit Wissenschaftlern und kuratieren die wissenschaftlichen Inhalte für die letztendliche Publikation. 
\item Publikation durch Verlage - Inhalte distribuieren: 
Nach Erstellung und Erkenntnissicherung findet "eigentlichen Publikation" \cite{schirmbacher_2009_wisspub} der Informationen statt. Bis zur Digitalisierung bestand dieser Schritt ausschließlich in dem Druck der Inhalte auf Papier.\cite{cite:11h}
\item Distribution durch die Verlage: 
Der Vertrieb und die Verbreitung von Forschungsergebnissen ermöglicht den Zugriff auf die Information durch andere Wissenschaftler. Dieser Schritt stellt einen essenziellen Teil der Zirkulation und Kommunikation des neu gewonnen Wissens dar\cite{cite:11i}. Er sichert die Verfügbarkeit und die Möglichkeit des Zugriffs auf die Informationen und ist Teil des Selektionsprozesses für die Erschaffung neuen Wissens.\cite{cite:11l}
\item Support und Archivierung durch Bibliotheken
Dieser Schritt besteht aus der Erschließung, Aufbewahrung und Bereitstellung der Publikation durch Bibliotheken \cite{schirmbacher_2009_wisspub}.
\item Konsum beziehungsweise Rezeption durch Wissenschaftler: 
Die Rezeption der veröffentlichten Inhalte durch die wissenschaftliche Gemeinschaft \cite{schirmbacher_2009_wisspub} stellt den letzen Schritt des wissenschaftlichen Kommunikationsprozesses dar. In diesem Schritt entsteht durch den Vergleich von neuen Ergebnissen mit bereits publizierten wissenschaftliche Qualität \cite{umstatter_2007_qualitatssicherung}. Aus der Mitte der Rezipierenden wissenschaftliche Gemeinschaft schaffen wiederum Autorinnen und Autoren ein nächstes geistiges Werk \cite{cite:11k} \cite{schirmbacher_2009_wisspub} und der ommunikationsprozess beginnt von vorne.
\end{enumerate}

An diesem Prozess sind drei Gruppen beteiligt: erstens die Wissenschaftler, als Produzenten und Konsumenten der Informationen, zweitens die Verleger, die als Intermediäre wissenschaftliche Informationen sammeln, bündeln und verkaufen, sowie drittens die Bibliotheken, die die Informationen wieder den Wissenschaftlern zur Verfügung stellen \cite{Odlyzko_1997}. Aus diesem Prozess und den beteiligten Gruppen, werden folgende Problemfelder ersichtlich:

--- TODO: klassisches Geschäftsmodell/Wertschöpfungskette vs. Open Access Geschäftsmodell/Wertschöpfungskette \cite{Hess_2006} ----

\subsection{Wissenschaftlicher Diskurs nach dem Diskurs- und Machtbegriff}
Nach Niklas Luhmann operiert der wissenschaftliche Diskurs funktional eigenständig und alles, was durch Wissenschaft kommuniziert wird, ist “entweder wahr oder unwahr” \cite{Luhmann1998}. Der wissenschaftliche Diskurs gründet sich dabei aber nur zum Teil auf Forschung und kann auch nicht nur als “Kontaktglied zwischen dem Denken und dem Sprechen” \cite{foucault_ordnung_2003} definiert werden. In der Foucault'schen Diskursanalyse wird der Diskurs als die Fähigkeit definiert, die “Beziehungen” zwischen “Institutionen, ökonomischen und gesellschaftlichen Prozessen, Verhaltensformen, Normsystemen, Techniken, Klassifikationstypen und Charakterisierungsweisen herzustellen”\cite{foucault_archaologie_1981}. Foucault beschäftigt sich insbesondere mit den Grenzen des Diskurses, sowie dessen institutioneller und praktischer Verortung. Im Gegensatz zu innerdiziplinärer Betrachtung eignet sich Foucaults “Werkzeugkiste”\cite{Honneth_2003} dabei besonders in der transdisziplinären Öffnung wissenschaftlicher Prozesse sowie die damit einhergehende Öffnung des Diskurses theoretisch zu hinterfragen. 

---- TODO: Weiter ausarbeiten In diesem Kapitel soll deshalb der Diskursbegriff in den Kontext der Thematik der Öffnung des Zugriff auf den wissenschaftlichen Prozess erläutert werden -----
