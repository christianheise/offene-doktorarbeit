\chapter{Definition und Abgrenzung} 
Wenn es im aktuellen öffentlichen Diskurs um wissenschaftliche Informationen, wissenschaftliche Infrastruktur und wissenschaftliches Arbeiten geht, werden immer öfter Schlagworte mit dem Attribut „Open“, wie Open Access, Open Research und Open Science, verwendet \cite{bunz_2014} \cite{schulze_2013_open}. "Offen" bezieht sich dabei meist auf zwei Kernaspekte: Zum einen die Offenheit des Zugangs zu Daten, Code oder Ergebnissen und zum anderen auf das Gebot der Transparenz, also der Offenlegung beziehungsweise der Zugriff auf Verfahren, Methoden und Zielen \cite{schulze_2013_open}.

In theoretischer Hinsicht existiert keine allgemeingültige Definition der Begriffe „Open Access“, „Open Science“ und „wissenschaftliche Reputation“, deshalb sollen in diesem Kapitel die Begriffe anhand wissenschaftlicher Literatur genau analysiert, zueinander abgegrenzt und zuletzt der Versuch gewagt werden, zu erläutern, wie die diese Begriffe in der Analyse, Forschung und Politik funktionieren und agieren. Dabei sollen gemeingültige Definitionen erarbeitet werden sondern dargestellt werden, wie die Begriffflichkeiten rund um Open Science in verschiedenen Diskursen verwendet werden und in welchen Relation sie zu anderen Begriffen stehen.

Darüber hinaus sollen die Begriffe in ihrer Entwicklung dargestellt werden. Entwicklung soll hier in den drei Dimensionen erfasst werden: erstens, als analytische Kategorie, zweitens als Forschungsgegenstand und drittens als politische Praxis in der moralischen Auseinandersetzung über die Wünschbarkeit von Zuständen betrachtet und erarbeitet werden, um ein möglichst umfassendes Bild der Begriffe zu erhalten. \cite{cite:10}

Da die Begriffe „Open Access“, Open Science” und wissenschaftliche Reputation in der wissenschaftlichen Auseinandersetzung auf unterschiedlichste Art und Weise Verwendung finden \cite{cite:9}, werden in diesem Kapitel die Begriffsbestimmungen konkretisiert sowie in den historischen und gesellschaftlichen Kontext eingebettet. Im Rahmen dieses Kapitels soll darüber hinaus auch adressiert werden inwiefern Macht, regulierende Prinzipien wie Verknappung sowie die Ein- und Ausgrenzung in den wissenschaftlichen Diskursen, nach dem Diskurs- und Machtbegriff  von Michel Foucault, mit den Modellen der offenen Initativen in der wissenschaftlichen Kommunikation vereinbar sind oder dem gegenüberstehen.

Grundlage für die Definition und Abgrenzung der beiden Begriffe stellt auch die Unterscheidung von "Zugang" und "Zugriff" auf wissenschaftliche Kommunikation dar. "Zugang haben" bezieht sich in diesem Zusammenhang auf einen unbeschränkten Weg zu der Kommunikation beziehungsweise die finale Information unbeschränkt lesen\cite{cite:9a}, verarbeiten und weiternutzen zu können. "Zugriff haben" hingegen soll als sich auf die erweiterte Nutzung der jeweiligen Wissensressourcen und auch den Zugang zu den Informationen oder Daten hinter der finalen Publikation und deren Kommunikation beziehen\cite{cite:9b}. Zugriff bedeutet beschränkt sich hier also nicht nur auf den reinen Zugang zu wissenschaftlicher Information im Rahmen des Publikationsprozess, sondern schließt auch den Zugriff auf sämtliche Forschungsdaten und alle weiteren Informationen, die während der wissenschaftliche Arbeit auf dem Weg zur finalen Publikation anfallen\cite{cite:9c}.

Die Themenbereiche Social Media in Wissenschaft und Forschung, Citizen Science und die aktuellen Diskurse zu Tools und Diensten werden in dieser Arbeit bewusst ausgelassen und nur am Rande, beziehungsweise wenn sie sinnvoll für die Beantwortung der Forschungsfragen sind, behandelt.