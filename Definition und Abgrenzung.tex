\chapter{Definition und Abgrenzung} 

Der theoretische Bezugsrahmen auf Grundlage von wissenschaftlich gesicherten Modellen, Theorien und Ansätzen ermöglicht es Erklärungen und Handlungsempfehlungen abzuleiten\cite{martin_2007_wissenschaftstheorie}. Er trägt dazu bei, die Fragestellungen in einen Zusammenhang zu stellen, legitimiert die Erforschung dieser Fragen und bildet den Rahmen für die Auswertung der gesammelten Erkenntisse \cite{suchen}. Dabei stellen die theoretischen Vorannahmen die Ausgangslage für die experimentellen und empirischen Betrachtungen in dieser Arbeit dar und werden im Rahmen einer Umfrage, der Literaturanalyse sowie der eigenen experimentellen Arbeit an der wissenschaftlichen Praxis geprüft. Dabei sollen die Konzepte von Open Access und Open Science den empirische Beobachtungen zugeordnet werden können. Ziel ist es, letztlich zu einem vertieften theoretischen Verständnis der gesammelten empirischen und experimentellen Ergebnisse zu gelangen. 

Open Science und Open Access im Kontext von wissenschaftlicher Reputation werden in dieser Arbeit in technischen als auch in ihren gesellschaftlichen und politischen Aspekten sowie die kulturellen Auswirkungen der Medienbrüchen im Rahmen von wissenschaftlichen Publizieren auf theoretischem Niveau reflektiert. Die Analysen in dieser Arbeit werden dabei aus den Perspektive des Produzenten (Wissenschaftler als Autoren) und auch aus der damit nicht immer harmonisierenden Perspektive des Rezipienten beziehungsweise Medienkonsumenten (Wissenschaftler als Leser) stattfinden. In diesem Zusammenhang wird adressiert inwiefern Macht, regulierende Prinzipien wie die Verknappung sowie die Ein- und Ausgrenzung im Rahmen von wissenschaftlichen Diskursen, nach dem Diskurs- und Machtbegriff von Michel Foucault, mit den Modellen der Open Access, Open Science und wissenschaftliche Reputation in der wissenschaftlichen Kommunikation vereinbar sind oder diesen diametral dem gegenüberstehen. 

Wenn es im aktuellen öffentlichen Diskurs um wissenschaftliche Informationen, Infrastruktur und Arbeiten geht, werden immer öfter Schlagworte mit dem Attribut „Open“, wie Open Access, Open Research und Open Science, verwendet \cite{bunz_2014} \cite{schulze_2013_open}. "Offen" bezieht sich dabei üblicherweise auf zwei Kernaspekte: Zum einen die Offenheit des Zugangs zu Daten, Code oder Ergebnissen und zum anderen auf das Gebot der Transparenz, also die Offenlegung beziehungsweise der Zugriff auf Verfahren, Methoden und Ziele \cite{schulze_2013_open}. Während "Openness" vielfach auf die Entwicklung rund um offene Software zurückzuführen sind die Anknüpfungspunkte von "Offenheit" als Begriff in der wissenschaftlichen Auseinandersetzung früher anzusetzen \cite{Tkacz_2014}. So sieht Christopher Kel­ty die ersten Anfänge bereits in den 1980ern \cite{kelty_2008_two_bits} und Andrew Russell sieht die ideologischen Ursprünge von "Offenheit" als Standard mit der Entwicklung des Telegraphs und weiteren Ingenieurleistungen ab 1860\cite{Russell_2014}.

In der Literatur sind die Begriffe „Open Access“ und „Open Science“ nicht allgemeingültig definiert \cite{suchen}. Sie finden in der wissenschaftlichen Auseinandersetzung auf unterschiedlichste Art und Weise Verwendung \cite{cite:9}. In diesem Kapitel wird der historische und gesellschaftliche Kontext ihrer Anwendung dargestellt und mittels der Analyse wissenschaftlicher Literatur abgegrenzt. Es wird angestrebt, zu erläutern, welche Bedeutung sie in Forschung, Gesellschaft und Politik haben. Die Begriffe werden in ihrer Entwicklung dargestellt. Um ein möglichst umfassendes Bild zu erhalten, wird "Entwicklung" hier in den drei Dimensionen erfasst: erstens, als "analytische Kategorie", zweitens als "Forschungsgegenstand" und drittens als "politische Praxis in der moralischen Auseinandersetzung über die Wünschbarkeit von Zuständen". \cite{cite:10} Ziel ist es hierbei nicht, gemeingültige Definitionen zu erarbeiten, sondern die Darstellung dieser Begriffflichkeiten in verschiedenen Diskursen. 

Im Rahmen dieses Kapitels wird darüber hinaus auch adressiert inwiefern Macht, regulierende Prinzipien wie Verknappung oder Ein- und Ausgrenzung in wissenschaftliche Diskurse, nach dem Diskurs- und Machtbegriff von Michel Foucault, mit den Modellen der offenen Initativen in der wissenschaftlichen Kommunikation vereinbar sind oder ob sie denen gegenüberstehen.

Die Unterscheidung "Zugang" und "Zugriff" ist wesentlich in dieser Arbeit und stellt eine wesentliche Grundlage für die Definition und Abgrenzung der Begriffe "Open Access" und "Open Science" dar. "Zugang" bezieht sich in diesem Zusammenhang auf einen unbeschränkten Zugang zur finalen wissenschaftliche Publikation. "Unbeschränkt" meint hier: lesen\cite{cite:9a}, Verarbeitung und Weiternutzung. "Zugriff" soll als erweiterte Nutzung der jeweiligen Wissensressourcen verstanden werden. Eingeschlossen sind neben dem "Zugang" zur Publikation sämtliche Informationen und Daten sowie die komplette Kommunikation hinter der finalen Veröffentlichung \cite{cite:9b}. "Zugriff" beschränkt sich hier also nicht nur auf den reinen Zugang zu wissenschaftlicher Information im Rahmen des Publikationsprozesses, sondern schließt auch den Zugriff auf sämtliche Forschungsdaten, Methoden und alle weiteren Informationen, die während der wissenschaftliche Arbeit auf dem Weg zur finalen Publikation entstehen \cite{cite:9c}, ein. 

Die Öffnung von Wissenschaft und Forschung wird in diesem Zusammenhang nicht einfach nur als politische  Rekation oder technische Alternative verstanden, sondern "als alternative Formatierungen einer wissenschaftlichen Infrastruktur im technischen, rechtlichen und zeitlichen Sinne" \cite{kelty_2004} und ist ein Thema, "das Wissenschaftler, politische Entscheidungsträger und die Öffentlichkeit betrifft" \cite{Scheliga_2014}.

Die Themenbereiche Social Media in Wissenschaft und Forschung, Citizen Science und aktuelle Diskurse zu Tools und Diensten werden in dieser Arbeit bewusst ausgelassen und nur am Rande, beziehungsweise nur wenn sie die Beantwortung der Forschungsfragen tangieren, eingeschlossen.

\section{Open Access} 

\begin{quote}
„Open Access“ meint, dass [= Peer-Review-Fachliteratur] kostenfrei und öffentlich im Internet zugänglich sein sollte, sodass Interessenten die Volltexte lesen, herunterladen, kopieren, verteilen, drucken, in ihnen suchen, auf sie verweisen und sie auch sonst auf jede denkbare legale Weise benutzen können, ohne finanzielle, gesetzliche oder technische Barrieren jenseits von denen, die mit dem Internet-Zugang selbst verbunden sind. In allen Fragen des Wiederabdrucks und der Verteilung und in allen Fragen des Copyrights überhaupt sollte die einzige Einschränkung darin bestehen, den Autoren Kontrolle über ihre Arbeit zu belassen und deren Recht zu sichern, dass ihre Arbeit angemessen anerkannt und zitiert wird.
\cite{boai_2012}
\end{quote}
Der Fortschritt der Wissenschaft hängt auch, und maßgeblich von dem freien Austausch und der Verbreitung von Informationen ab \cite{cite:11}. Das System der wissenschaftlichen Kommunikation, das so seit mehreren hundert Jahren besteht, basierte auf Forschung, der Begutachtung, dem Druck sowie der Kommunikation der Egebnisse in wissenschaftlichen Publikationen, der Verbreitung sowie dem Verkauf an Bibliotheken und andere wissenschaftliche Institutionen gegen Kosten \cite{cite:11a} und dem anschließenden Diskurs in der wissenschaftlichen Fachöffentlichkeit \cite{suchen}. Der offene Zugang zu wissenschaftlicher Kommunikation ist seit der Entwicklung des gedruckten Wortes aber auch eng mit der Frage nach Urheberrechten für wissenschaftliche Informationen verknüpft \cite{Case_2000}. Open Access beschreibt in ein wissenschaftliches Kommunikationssystem, in dem der Zugriff auf die unterschiedlichsten Formen wissenschaftlicher Publikationen, im Gegensatz zum bestehenden System, unter freien, kostenlosen Bedinungen und ohne finanzielle, gesetzliche oder technische Barrieren (Online) möglich ist \cite{WD_bundestag_2009}, das aber auch ein "alternatives Geschäftsmodell"\cite{lewis_2012_inevitability} für wissenschaftliche Publikationen ermöglicht. Das beruht auf der Maßgabe, dass der Autor die "Eigentumsrechte an den Artikeln, die bisher für die Publikation in wissenschaftlichen Journals an die jeweiligen Fachverlage abgetreten wurden, (...) nun bei den Autoren der Artikel selbst verbleiben" \cite{Hess_2006}. 

Durch den weltweit steigenden Haushaltsdruck an Bibliotheken und wissenschaftlichen Insitutionen, dem “ungewöhnlichen Geschäftsmodell” \cite{cite:12} der Wissenschaftsverlage mit immer höheren Margen \cite{albert_2006_open_implications} und dem Umstand, dass private Wissenschaftsverlage durch das wissenschaftlichen Reputationssystem über öffentlich finanzierte Wissenschaftlerkarrieren entscheiden \cite{heise_2012}, befindet sich das System in einer Krise\cite{cite:14}. Open Access beschäftigt sich in diesem Rahmen mit der Öffnung (Open) und dem freien Zugang (Access) zu den wissenschaftlichen Publikationen. "Geringere Kostenbarrieren und damit eine einfachere Verbreitung ihrer eigenen Werke" \cite{WD_bundestag_2009} stellen dabei die Wünsche der wissenschaftlichen Autoren und Urheber an Open Access dar und der Einsatz (offener) Lizenzen ist dafür einer der Haupteinflussfaktoren \cite{cite:16}. Es ist derzeit üblich, Open Access in drei Modelle einzuteilen \cite{suchen}: Green Open Access, Golden Open Access und andere (Misch-)Formen.

\section{Open Science}
In diesem Kapitel soll Open Science (medien)kulturwissenschaftlich sowohl in technischen als auch in gesellschaftlichen und politischen Aspekten definiert und abgegrenzt werden.

Der Sammelbegriff Open Science erstreckt sich über die gesamte wissenschaftliche Wertschöpfungskette \cite{Scheliga_2014}: Vom offenen Zugang (Open Access) zur Publikationen wissenschaftlicher Forschung sowie den ganzheitlichen wissenschaftlichen Erkenntnisprozess. Unter diesem Gesichtspunkt, kann Open Science als eine Weiterentwicklung von Open Access bezeichnet werden. Die diesbezügliche Evolution des Konzepts von Open Access führt zu einem direkten und breiten Weg, Wissenschaft an jedem Schritt der wissenschaftlichen Wertschöpfungskette zu kommunizieren und zu transferieren. Open Science ist die Reaktion auf die Forderung nach offenem Zugriff auf Wissenschaft und Forschung und basiert auf Annahme, "dass sich die Bedeutung von Forschungsergebnissen zukünftig nicht mehr auf sogenannte klassische wissenschaftliche Publikationen (im Format von Einleitung – Methoden – Ergebnisse – Diskussion), sondern die globale Echtzeitpublikation von Originaldaten stützen wird" \cite{Stengel_2013}.

Bei der Verbreitung von Open Science, werden grundsätzlich zwei Strategien für die Etablierung von Offenheit in Wissenschaft und Forschung abgegrenzt \cite{schulze_2013_open}: 
\begin{enumerate}
\item "Top-down durch Förderstrategien, Vorgaben und Empfehlungen"
Hiermit sind Prozesse beziehungsweise gemeint, bei denen durch die direkte Incentivierung im Rahmen von Forschungsförderung Anreize für die Berücksichtigung von Offenheit in den geförderten Projekten geschaffen werden. Beispielsweise kann durch die Bereitstellung zusätzlicher Mittel für die offene Bereitstellung und Publikation von Forschungsergebnissen ein Anreiz geschaffen werden  \cite{suchen}. Neben der Incentivierung bietet die bindende Vorgabe eine weitere Möglichkeit zur Etablierung von Verhaltensänderungen \cite{suchen}. So kann durch Änderung der politischen und rechtlichen Vorgaben eine Öffnung von Wissenschaft und Forschung erzwungen werden \cite{suchen}. Eine weitere Möglichkeit der "Top-Down"-Etablierung von Offenheit und Forschung stellen Empfehlungen dar, bei denen Insitutionen, Organisationen oder Gruppen Empfehlungen aussprechen, anhand derer WissenschaftlerInnen über nicht bindende Hinweise überzeugt werden sollen, die Öffnung von Wissenschaft und Forschung zu etablieren. Alle diese Strategien haben einen formellen Charakter \cite{suchen}.
\item "Bottom-up durch Graswurzelprojekte und den Einsatz von Evangelists"
Im Gegenzug zur Strategie von "oben" gibt es auch Bestrebungen, die von einzelnen WissenschaftlerInnen oder Gruppen initiiert sind. Sie sind zumeist informell und zielen auf eine beispielhafte Herangehensweise für die Verbreitung von Verhaltensänderungen ab \cite{suchen}. Bottum-up-Projekte kommen aus dem wissenschaftlichen Alltag und erfahren keine politische, rechtliche oder monitäre Incentivierung zur Umsetzung der Tätigkeiten für die Öffnung von Wissenschaft und Forschung. Der Einsatz von Evangelisten baisert auf der Idee einer konkreten Stelle oder Position um eine Änderung zu Begleiten \cite{suchen} oder einen Mulitplikator innerhalb und außerhalb von Insitutionen oder Organisationen zu etablieren, der das gewünschte Ziel proaktiv kommuniziert und verbreitet \cite{suchen}. Evangelisten stellen einen wesentliche Maßnahme dar, um "die Befindlichkeiten" "auszutarieren" und um "teils diffuse, teils reale Ängste" bei "Offenheit und Transparenz der Wissenschaft "\cite{schulze_2013_open} zu beseitigen.
\end{enumerate} 

In beiden Fällen steht und fällt der Erfolg damit, ob sich der jeweiligen Zielgruppe ein unmittelbarer Mehrwert und Nutzen erschließen wird \cite{schulze_2013_open}.

\subsection{Offener Zugriff auf wissenschaftliche Kommunikation}
Open Science beinhaltet nicht nur um den offenen Zugang zu Wissenschaft und den daraus resultierenden Veränderungen wissenschaftlicher Kommunikationsprozessen im Rahmen von Publikationen, sondern auch den unmittelbaren und offenen Zugriff auf den gesamten Prozess der Wissensschaffung. Aus technischer Sicht ist jeder Aspekt der Wissenschaftskommunikation, der digital auf einem Desktop-Computer stattfindet, auch öffentlich über das Web potenziell verfügbar \cite{mietchen2012wissenschaft}. 

Zur Verdeutlichung des Prozess der Wissensschaffung wird in der vorliegenden Arbeit eine Einteilung extrapoliert in vier Phasen vorgenommen:
\begin{enumerate}
\item Fragestellung & Planung
Basis für den Prozess der Wissenschaffung ist eine Frage zur Erklärug einer speziefischen Beobachtung oder eine offene Frage\cite{suchen}. Für die wissenschaftliche Bearbeitung eines Themas ist es entscheidend, dass eine präzise Fragestellung im Zentrum steht \cite{suchen}. --- TODO: weiter beschreiben ---
\item Ausführung
Testen der Hypothese durch den Einsatz von geeigneten wissenschaftlichen Kontrollen und unter Minimierung der möglichen Fehler.
\item Verarbeitung und Analyse --- TODO: weiter beschreiben ---
Analyse der gewonnen Daten und Informationen im Hinblick auf die Verifikation und Falsifikation der Hypothese. --- TODO: weiter beschreiben ---
\item Auswertung
----TODO: Beschreiben-----
\end{enumerate}

Anhand der hier vorgenommenen Einteilung werden die Charakteristika des Wissenschafts-Prozesss erläutert und dargestellt, um zu verdeutlichen, was die Öffnung von Wissenschaft im Sinne von Open Science bedeutet. Die Forderung nach Öffnung des Prozess der Wissensschaffung begründet sich dabei nicht  durch Unzulänglichkeiten am bestehenden wissenschaftlichen Kommunikationssystem, sondern basiert auch auf weiteren Annahmen:

\begin{enumerate}
\item Der offene Zugang zum gesamten Wissenschaftsprozess erhöht die Möglichkeiten der Validierung und Reproduzierbarkeit der gesamten Forschung(skette) und die Entwicklung neuer Qualitätskriterien. (enhanced Validation/Reputation-Argument)
\item Im Rahmen des Teilens (z.B. von Rohdaten) erhöht sich die Effizienz und Verwendbarkeit von Forschung und im Rahmen von Wissenschaft entstandenen Informationen (Shared-Science-Argument)
\item im klassischen wissenschaftlichen Kommunikationssystem gibt es kaum Anreize negative, widerlegende oder unerfolgreiche wissenschaftliche Ergebnisse zu veröffentlichen, eine grundsätzliche Öffnung könnte dazu beitragen, dass Wissenschaft ihrem Anspruch an Falsifizierbarkeit gerecht wird z.B. in Pharmalogie (negative-science/falsifiability-argument)
\end{enumerate}

\subsection{Wissenschaft als Open-Source-Prozeß}

Open Source ist ein Begriff aus der Softwareentwicklung der als Gegensatz zum “Verfahren der Wissenssicherung” \cite{stallman2002} eine quelloffenen Handhabe von Softwarecode beschreibt. Der Ende der 90iger Jahre des letzten Jahrhunderts eingeführte Begriff wird, auch wenn es im Detail Unterschiede im Konzept gibt \cite{suchen}, mit “freier Software“ (nicht Freeware) gleichgesetzt \cite{suchen}. Dabei folgt die Open Source-Entwicklung der Maxime, dass die Kernsteuerungsinformationen und -befehle (Quelltext) von Software öffentlich einsehbar und zugänglich und je nach gewähltem Lizenzmodell modifiziert, kopiert oder weitergegeben werden müssen\cite{suchen}. 

Die Entwicklungsmethode unterscheidet, so Steven Weber, zwischen Open-Source-Software und dem traditionellen Modell des geistigen Eigentums mit der Feststellung, dass Open-Source-Software das Prinzip der Exklusivität des geistigen Eigentums auf den Kopf stellt, weil diese Software 'um das Recht auf Vertrieb konfiguriert, nicht auszuschließen ist" \cite{suchen}. 

--- Prüfen ---
Auch Maurer und Scotchmer merken an, dass Open-Source-Software-Entwicklung Rechtsmittel ein Defekt der Schutz des geistigen Eigentums, die nicht allgemein zu fördern hat die Offenlegung des Quellcodes. 
--- Prüfen ---

Die Open Source Definition beinhaltet festgelegte Kriterien für die Klassifizierung von Open Source Produkten \cite{suchen}:
\begin{enumerate}
\item Freie Weitergabe
----TODO: Beschreiben-----
\item Quellcode, das Programm muss den Quellcode beinhalten, bzw. muss den Code offen zur Verfügung stellen
----TODO: Beschreiben-----
\item Verwendete Lizenz muss Derivate zulassen
----TODO: Beschreiben-----
\item Unversehrtheit des Quellcodes des Autors muss garantiert werden
----TODO: Beschreiben-----
\item Auschluss von Diskriminierung von Personen oder Gruppen
----TODO: Beschreiben-----
\item Keine Enschränkung des Einsatzfeldes
----TODO: Beschreiben-----
\item Lizenz muss weitergegeben werden könnne
----TODO: Beschreiben-----
\item Lizenz muss auf das Produktpaket angewandt werden
----TODO: Beschreiben-----
\item Lizenz darf die Weitergabe zusammen mit anderer Software nicht einschränken
----TODO: Beschreiben-----
\end{enumerate}

Im Vergleich zum klassischen Softwareentwicklungsprozess gelten folgende charakteristische Merkmale \cite{suchen}:
\begin{enumerate}
\item “Anzahl der beteiligten Entwickler: Im Vergleich zu traditioneller Softwareentwicklung ist eine weitaus größere Anzahl von Entwicklern beteiligt. Zudem gibt es keine klare Grenze zwischen Entwicklern und Anwendern, da die Hürden für eine Partizipation im Entwicklungsprozess sehr gering sind. Auch wenn ein großer Teil der Entwicklungsarbeit von Freiwilligen übernommen wird, gibt es dennoch den Trend zum Einsatz bezahlter Entwickler.
\item Zuteilung der Arbeit: Im OSP wird die Entwicklungsarbeit nicht länger von einer definierten Instanz zugeteilt, sondern die Teilnehmer wählen ihre Arbeitspakete selbst aus.
\item Architektur: In der Regel orientierten sich die Teilnehmer eines OSP nicht an einer vorgegebenen System-Architektur. Die Gestaltung der Architektur geschieht dezentral und ist oftmals häufigen Richtungswechseln unterworfen.
\item Koordination: Es gibt wenig oder keine institutionalisierten traditionellen Koordinationsmechanismen, wie z.B. Projekt- und Zeitpläne, Lasten- und Pflichtenhefte u.ä.” \cite{suchen}
\end{enumerate}

Der Literaturwissenschaftler und Medientheoretiker Friedrich Kittler beschreibt die Entwicklungsmethode Open-Source als fest mit dem Wissenschaftsprozess verankert \cite{suchen}. Open Source Entwicklungsprozesse unterscheiden sich von den klassisch-traditionellen (closed-source) Softwareentwicklungsprozessen insbesondere dadurch, dass sie jederzeit öffentlich einsehbar und transparent nachvollziehbar sind. Open Source zeigt diesbezüglich mit Open Science konvergenzen, als dass es nicht nur den freien und offenen Zugang zu wissenschaftlichen Informationen betrifft, sondern auch den Zugriff auf den gesammten Prozess zur Erlangung der wissenschaftlichen Informationen sowie die Daten offenlegt und transparent nachvollziehbar macht \cite{kelty_2004}. Adaptiert man den Open-Source-Prozess auf wissenschaftliche Wertschöpfungsprozesse und definiert in diesem Zusammenhang wissenschaftliche Publikationen als Quellcode, ist das Konzept übertragbar \cite{Singh_2008} \cite{Bradley_2008} \cite{Bradley_2007}. Daraus folgt, dass Open Access aus technologisch-entwicklungsmethodischer Sicht mit kostenloser Software (Freeware) \cite{suchen} verglichen werden kann. Freeware und Open Access Publikationen sind zwar kostenlos verfügbar, ihr Erstellungprozess wird jedoch nicht offen und transparent kommuniziert \cite{suchen}. Dieser Exkurs in die Softwareentwicklung versicht die Abgrenzung von Open Access zu Open Science zu verdeutlichen und stellt Parallelen zu Open Source versus kostenloser Software (Freeware) her. Es gibt aber noch eine  weitere Gemeinsamkeit: "Free Software (im Sinne von Open Source), Open Access und Creative Commons sind alles Rechts- und Infrastrukturexperimente"\cite{kelty_2004}.

\subsection{Entwicklung der Bewegung}

Die Entwicklung von “Open Science” knüpft an die Entwicklung von Open Access-Bewegung an und kann als Folge der neuen Möglichkeiten für kollaboratives Arbeiten im Rahmen der Digitalisierung und neuer Kommunikationstechniken verstanden werden. Unter Berücksichtigung der Frage, wie der gesamte wissenschaftliche Wertschöpfungsprozess der Allgemeinheit zur Verfügung gestellt wird und wurde, soll die Definition und Abrenzung von Open Science angestrebtwerden. 

Dennoch spielt insbesondere die Entwicklung der Tradition für eine "offenen Wissenschaft" im siebzehnten Jahrhundert einen Ansatzpunkt, da dieser historische Übergang noch nicht erforscht ist \cite{CREATe_2014}. Die Verschlüsselungs- und Patentwut zur Wahrung eines möglichen kommerziellen Vorteils durch Wissenschaft im Rahmen öffentlich-finanzierter Forschung, geht dabei bis auf die xxxx Jahre zurück. ### Beispiel Galileo, Kepler, Newton ### Das Ergebnis dieser Wut war eine Debatte über die Verfügbarkeit der wissenschaftlichen Arbeit und die Entlohnung der “Erfinder“ im wissenschaftlichen System. 

Im April 2012 wurde die Erklärung "Open Science for the 21st century", vom Zusammenschluss der Europäischen Akademien (ALLEA) verabschiedet \cite{ALLEA_2012}. Sie war nur eine von mehreren Erklärungen und Positionspapiere für die Öffnung von Wissenschaft durch international angesehenen Einrichtungen, durch die verdeutlich wurde, dass die Forderung nach offenem Umgang mit Wissen und Information im wissenschaftlichen Bereich zunehmend an Relevanz gewinnt \cite{schulze_2013_open}.

--- TODO: Weitere Entwicklung darstellen ----

Das Potenzial bei der Verwendung von digitalen Technologien um Wissenschaft offen zu teilen, ist jedoch nicht nicht annährend ausgeschöpft und es "besteht eine erhebliche Diskrepanz zwischen der Idee der offenen Wissenschaft und wissenschaftliche Realität" \cite{Scheliga_2014}.

\subsection{Open Science Modelle}
--- TODO: definieren ----
\subsection{Open Science Formate}
Data Repositorien, (offne) Forschungsanträge, offenes Publizieren (siehe OA), Laborbücher

\section{Wissenschaftliche Reputation}
Ein Grundprinzip des Wissenschaftssystems basiert auf der "gegenseitigen Beurteilung und Anerkennung der jeweils neuen Ergebnisse ihrer Fachkollegen (Peers) durch die Wissenschaftler selbst"\cite{Hanekop_2014}. In dem Peer-Review Prozess "werden eingereichte Beiträge von fachlich versierten Wissenschaftlern (...) beurteilt und gemäß den qualitativen Anforderungen der Forschungs-Community zur Veröffentlichung angenommen oder abgelehnt" \cite{Hess_2006}. "Peer-Review" beschränkt sich dabei nicht nur auf die Publikation von Texten, sondern deckt ein breites Spektrum von Aktivitäten ab: die Beobachtung der klinischen Praxis; Beurteilung des Lehrenden, Fähigkeiten der Kollegen; Bewertung durch Experten bei der Forschungsförderung und Stipendien bei Einreichung von Anträgen an staatliche und anderen Förderorganisationen; Begutachtung von Redakteuren und externen Gutachtern bei Artikeleinreichungen für wissenschaftlichen Zeitschriften; Bewertung von Papieren und Plakate für Konferenzen; Bewertung von Buchvorschlägen für Universitätverlagen oder andere Verlagenn; und Einschätzungen der Qualität, Anwendbarkeit und Interpretierbarkeit von Datensätzen und wissenschaftlichen Organisationen" \cite{Lee_2012}. Dennoch bilden "Publikationen im Hinblick auf die Funktion der Reputationsverteilung eine Art Telos wissenschaftlicher Kommunikation "\cite{hirschauer2004peer}. Im Rahmen von Reputation ist wissenschaftliche Arbeit besonders auf funktionierendes Peer-Review-System angewiesen \cite{suchen}. Dennoch haben qualitatives Peer Review-Systeme und quantitative bibliometrischen Verfahren viele Mängel\cite{osterloh2008anreize} \cite{Lee_2012}.

Folgende weitere Indikatoren werden für wissenschaftliche Reputation für wissenschaftliche Institutionen und Personen genannt\cite{hanekop_2008}:
\begin{enumerate}
\item Drittmittelprojekte
----TODO: Beschreiben-----
\item Patente
----TODO: Beschreiben-----
\item Vorträge
----TODO: Beschreiben-----
\item Anwendungsrelevanz bzw. Verwertbarkeit
----TODO: Beschreiben-----
\item Netzwerke
----TODO: Beschreiben-----
\item öffentliche Aufmerksamkeit sowie politische Relevanz 
----TODO: Beschreiben-----
\item Renommee der Forschungseinrichtung
----TODO: Beschreiben-----
\item materielle Ausstattung, Großgeräte etc.
----TODO: Beschreiben-----
\item personelle Ausstattung
----TODO: Beschreiben-----
\item Gutachtertätigkeit
----TODO: Beschreiben-----
\item Herausgeberschaft
----TODO: Beschreiben-----
\item Funktion
----TODO: Beschreiben-----
\end{enumerate}

Wissenschaftliche Reputation wird als Währung verstanden, mittels derer “Status und Ressourcen verteilt werden” \cite{hanekop_2006}. Sie verteilt sich auf Einrichtungen und einzelne Personen, die wissenschaftlich tätig sind \cite{suchen}. Die Evaluation wissenschaftlicher Einrichtungen findet dabei über “Beobachtungen und Gespräche mit den Wissenschaftlern vor Ort sowie über den Austausch über die Eindrücke innerhalb der Begehungsgruppe und die gemeinsame Verständigung”\cite{Barl_sius_2008} statt.

Die Reputation einzelner Wissenschaftler steht in enger Abhängigkeit zum bestehenden wissenschaftlichen Kommunikationssystem \cite{suchen}. Für die Wissenschafler sind Publikationen und die damit einhergehende Verbreitung von wissenschaftlichen Erkenntnissen sehr entscheidend \cite{Hess_2006}. Vereinfacht lässt sich das System der Wechselbeziehungen der Reputationsverteilung im Rahmen von Publikationen wie folgt darstellen \cite{cite:21a}: 

Grafik aus Text von Bernius
http://www.eap-journal.com/archive/v39_i1_8_bernius.pdf

Bernius et al. unterscheiden drei aufeinandertreffende koordinierenden Marktmechanismen: die Reputation und die Nutzung wissenschaftlicher Publikationen, sowie der Preis für den Erwerb \cite{suchen}. Während die Reputation ein non-monetärer Aushandlungsmechanismus zwischen wissenschaftlichen Verlagen und wissenschaftlichen Autoren ist, findet die monetäre Preisdefinition zwischen Bibliotheken und Verlagen statt. Der monetäre Aushandlungsprozess zwischen Wissenschaftlern und Bibliotheken fokussiert sich auf die Frage der Bedeutung und Nutzung der jeweiligen Publikation \cite{cite:21a}. Nicht jede Publikation hat diesbezüglich die gleiche Wertigkeit \cite{suchen} und damit den gleichen Einfluss auf die Reputation. 

Die neuen Möglichkeiten der Verbreitung von Informationen lassen deshalb einen vergleichbaren Veränderungsprozess der wissenschaftlichen Reputation und damit auch Anerkennung vermuten, wie sie durch die Entwicklung des Buchdrucks ausgelöst worden war.\cite{hanekop_2006} 

In diesem Rahmen wurden durch den US-amerikanische Soziologe Robert K. Merton vier Grundprinzipien als normative Struktur der Wissenschaft beschrieben \cite{Merton_1985}:

---- TODO: Ausarbeiten ----

\subsection{Wissenschaftliches Kapital}
Im Rahmen der Betrachtung von Steuerungs- und Reputationsmethoden für die Wissenschaft ist der Begriff "wissenschaftliches Kapital" von herausragender Bedeutung \cite{suchen}. Wissenschaftliches Kapital kann als eine Ausprägung des kulturellen Kapitals und als symbolisches beziehungsweise non-monetäres Kapital \cite{irmer2011} verstanden werden. 

Die “Gewährung wissenschaftlichen Kapitals” im wissenschaftlichen System basiert heute auf einer engen Verbindung zwischen publizierenden Wissenschaftlern und Verlagen \cite{herb_2006}. Die Wissenschaft steht demnach in einer klaren Abhängigkeit zu den Verlagen. Ulrich Herb definiert mit Hilfe Pierre Bourdieus, "wissenschaftliches Kapital" als “Ergebnis einer Investition (...), die sich auszahlen muss”. “Diejenigen, die diese Berechtigungsscheine in der Hand halten, verteidigen ihr 'Kapital' und ihre 'Profite', indem sie diejenigen Institutionen verteidigen, die ihnen dieses 'Kapital' garantieren.” \cite{Bourdieu_1992} Herb kommt zu dem Schluss, dass die Öffnung von Wissenschaft dabei bisher nicht wissenschaftlicher Logik folgt, “sondern einer feldunabhängigen Logik der Akkumulation von Kapital”\cite{herb_2006}. Hinzu kommt, dass vor allem das deutsche Wissenschaftssystem durch durch die Einführung an Outputs orientierter Anreizsystem gekennzeichnt ist \cite{osterloh2008anreize}.

Als Beispiel weiteres Beispiel für wissenschaftliches Kapital kann der Performanzindikator "Drittmittel" \cite{Jansen_2007} dienen, durch die in der Wissenschaft neben der Sicherung der Qualität von Forschung und Lehre zunehmend direkte finanzielle und administrative Kontrolle eine Rolle spielt \cite{Barl_sius_2008}. Daraus resultiert die Gefahr, dass nicht nur die Erwartungen an die Bewertung von Wissenschaft zu hoch gegriffen sind, sondern auch, dass sich die Wissenschaft zu sehr an diesen Erwartungen orientiert und die Interessen privater und öffentlicher Drittmittel-Auftraggeber in den Vordergrund rücken.  Vor allem die Verknüpfung von wissenschaftlicher Reputation und der damit einhergehenden Verteilung der Mittel und Stellen stellt eine Herausforderung an das Wissenschaftsystem dar, “dessen Währung [ursprünglich] nicht Geld ist” \cite{hanekop_2006}. 

--- TODO: Freiheit von Lehre und Forschung ----
Wahrung des in Artikel 5 Abatz. 3 GG garantiertes Grundrechts auf "Freiheit von Lehre und Forschung". In , das einerseits eine Garantie der Einrichtung wissenschaftlicher Hochschulen mit Anspruch auf Selbstverwaltung und Sicherung ihrer Arbeit durch den Staat beinhaltet und andererseits dem einzelnen Wissenschaftler ein subjektives Recht auf Nichteinmischung des Staates in seine wissenschaftliche Tätigkeit gibt. --- TODO: Freiheit von Lehre und Forschung ----

\subsection{Ökonomie der wissenschaftlichen Kommunikation}
Die klassische Ökonomie der wissenschaftlichen Kommunikation beruht auf der Durchsetzung von Urheberrechten, die den Zugriff auf und die Wiederverwendung von geschützten Inhalten beschränken sowie die Zahlung einer Gebühr durch den Leser verlangen um Zugang zu der Veröffentlichung zu erhalten \cite{CREATe_2014}. Bislang werden dafür "in der Regel wissenschaftliche Arbeiten zwar mit öffentlichenMitteln finanziert, aber von privaten Verlagen in Fachzeitschriften herausgegeben" \cite{WD_bundestag_2009}. Diese ungewöhnlichen Ökonomie der Wissenschaftsverlage ist nicht neu und hat sich im Laufe der Zeit entwickelt, die starke Wahrnehmung der Ungerechtigkeit dieses Systems, insbesondere der Preismodellen für wissenschaftliche Publikationen\cite{King_2008} findet aber erst seit kurzem statt\cite{CREATe_2014}. Dieses Modell baisert auf einer sozial ineffizientem Ebene\cite{mueller-langer_2010}.

Eine weitere wesentliche Besonderheit der Wissenschaftskommunikation ist die Organisation des Marktes, die von spezifischen Akteuren und Prozessen geprägt ist \cite{Hess_2006}. Vereinfacht kann der klassische wissenschaftliche Kommunikationsprozess im Rahmen von Publikationen wie folgt unterteilt werden\cite{cite:11b} \cite{Hess_2006}:
\begin{enumerate}
\item Erstellung durch Wissenschaftler - Inhalte erzeugen: 
Nach der Entwicklung eines konkreten Forschungsvorhabens sowie einer wissenschaftlichen Fragestellung enstehen im Rahmen der wissenschaftlichen Forschung oder der jewiligen Untersuchung Informationen\cite{cite:11c}, die im Forschungsprozess gesammelt, analysiert, ausgewertet, aufbereitet und verschriftlicht wurden\cite{cite:11d}. Diese Infromationen werden strukturiert zusammengefasst und niedergeschrieben \cite{Hess_2006}.
\item Qualitätskontrolle durch Wissenschaftler - Inhalte bewerten: 
Die Qualitätskontrolle ist einer der wesentlichen Bestandteile der wissenschaftlichen Kommunikation. Sie sichert die gewonnen Erkenntnisse\cite{cite:11e} und stellt einen klaren Abrenzungsaspekt zu nicht-wissenschaftlichen Informationen und Erkenntnissen dar\cite{cite:11f}. Sie findet im Kommunikationsprozess an zwei Stellen statt. Hier ist die erste Stelle gemeint, in der vor der Produktion der Informationen in Form der Publikation, die Erkenntnisse von anderen Wissenschaftlern überprüft und gesichert werden (Peer-Review) \cite{Hess_2006}.
\item Bündelung durch Verlage - Inhalte auswählen:
Verlage bündeln und kuratieren die wissenschaftlichen Inhalte für die letztendliche Publikation. 
\item Publikation durch Verlage - Inhalte distribuieren: 
Nach Erstellung und Erkenntnissicherung findet die für die Distribution notwendige Publikation der Informationen statt. Bis zur Digitalisierung bestand dieser Schritt ausschließlich in dem Druck der Inhalte auf Papier.\cite{cite:11h}
\item Distribution durch die Verlage: 
Der Vertrieb und die Verbreitung von Forschungsergebnissen ermöglicht den Zugriff auf die Information durch andere Wissenschaftler. Dieser Schritt stellt einen essenziellen Teil der Zirkulation und Kommunikation des neu gewonnen Wissens dar\cite{cite:11i}. Er sichert die Verfügbarkeit und die Möglichkeit des Zugriffs auf die Informationen und ist Teil des Selektionsprozesses für die Erschaffung neuen Wissens.\cite{cite:11l}
\item Support und Archivierung durch Bibliotheken
---- TODO Ausformulieren ----
\item Konsum beziehungsweise Rezeption durch Wissenschaftler: 
Der nächste Schritt im wissenschaftlichen Kommunikationsprozess, von dem der gesamten Prozess von neuem initiiert wird, ist die Rezeption der veröffentlichten Inhalte. Zum einen die Rezeption der wissenschaftlichen Forschung aus der "Erstellung", zum anderen die zweite Stufe der Qualitätsicherung ( --- TODO: http://edoc.hu-berlin.de/miscellanies/wifo2007/PDF/wifo2007-9-49.pdf erklären!!! Qualitätssicherung bei Konsum ----) kommen hier zum Tragen.\cite{cite:11j} Der Konsum wissenschaftlicher Informationen wird dadurch Grundlage für die "Erstellung" neuen Wissens. Der Endpunkt des wissenschaftlichen Kommunikationsprozesses ist auch gleichzeitig Ausgangspunkt für einen neuen Prozess\cite{cite:11k}.
\end{enumerate}

An diesem Prozess sind vor allem drei Gruppen beteiligt: erstens die Wissenschaftler, als Produzenten und Konsumenten der Informationen, zweitens die Verleger, die als Intermediäre wissenschaftliche Informationen sammeln, bündeln und verkaufen, sowie drittens die Bibliotheken, die die Informationen wieder den Wissenschaftlern zur Verfügung stellen \cite{Odlyzko_1997}. Aus diesem Prozess und den beteiligten Gruppen, werden folgende Problemfelder ersichtlich:

--- TODO: klassisches Geschäftsmodell/Wertschöpfungskette vs. Open Access Geschäftsmodell/Wertschöpfungskette \cite{Hess_2006} ----

\subsection{Messbarkeit wissenschaftlicher Qualität vs. Publikationsquantität}
Wissenschaft ist ein Prozess, bei dem aus “unterschiedlichen Inputfaktoren, mittels verschiedener Transformationen Beiträge zur Schaffung neuer wissenschaftlicher Erkenntnisse als Output entstehen”\cite{Jansen_2007}. Die Bewertungen des jeweiligen Outputs führt “zur Ausage über die Forschungsperformanz”. Neben den Indikatoren für den Output wissenschaftlicher Perfomanz, müssen aber auch intermediäre Aspekte berücksichtigt werden\cite{schmoch_2009}. Nach dem zweiten Weltkrieg etablieren sich die ersten Indikatoren für die Effizienzmessung wissenschaftlicher Wissensproduktion und -verbreitung. Spätestens seit den 1960er Jahren werden diese Messungen in Gestalt von Indikatoren (--- TODO Was sind die Indikatoren? erklären! ----), die die Forschungsleistung quantifizieren sollen, flächendeckend durchgeführt \cite{suchen}. Seit den 1990er Jahren ist diese Bewertung in Gestalt von Zahlen als unkontrollierte Nebenprodukte digitaler Wissenskommunikation erweitert worden \cite{angermueller_2010}. Heute zählen in der Wissenschaft vor allem die wissenschaftliche Reputation und die als Impact bezeichnete Wirkung wissenschaftlicher Publikationen\cite{herb_open_2013}. Die Wirkung wird dabei anhand der Zitationen der jeweiligen Publikation gemessen \cite{suchen}. Eine häufige Zitation stellt dabei einen Indikator für einen große Wirkung der wissenschaftlichen Arbeit dar. 

---- TODO: Weiter ausarbeiten -----

\subsection{Wissenschaftliche Diskurse nach dem Diskurs- und Machtbegriff}
Nach Niklas Luhmann operiert der wissenschaftliche Diskurs funktional eigenständig und alles was durch Wissenschaft kommuniziert wird, ist “entweder wahr oder unwahr” \cite{Luhmann1998}. Der wissenschaftliche Diskurs gründet sich dabei aber nur zum Teil auf Forschung und kann auch nicht nur als “Kontaktglied zwischen dem Denken und dem Sprechen” \cite{foucault_ordnung_2003} definiert werden. In der Foucault'schen Diskursanalyse wird der Diskurs als die Fähigkeit definiert, die “Beziehungen” zwischen “Institutionen, ökonomischen und gesellschaftlichen Prozessen, Verhaltensformen, Normsystemen, Techniken, Klassifikationstypen und Charakterisierungsweisen herzustellen”\cite{foucault_archaologie_1981}. Foucault beschäftigt insbesondere mit den Grenzen des Diskurses, sowie dessen institutioneller und praktischer Verortung. In diesem Zusammenhang ist es von Interesse, inwiefern Macht, regulierende Prinzipien wie Verknappung sowie die Ein- und Ausgrenzung vom wissenschaftlichen Diskurs (nach dem Diskurs- und Machtbegriff von Michel Foucault) mit den Modellen der Open Initiatives in der wissenschaftlichen Kommunikation vereinbar sind oder diesen gegenüberstehen. Im Gegensatz zu innerdiziplinärer Betrachtung eignet sich Foucaults “Werkzeugkiste”\cite{Honneth_2003} dabei besonders in der transdisziplinären Öffnung wissenschaftlicher Prozesse sowie die damit einhergehende Öffnung des Diskurses theoretisch zu hinterfragen. 

---- TODO: Weiter ausarbeiten In diesem Kapitel soll deshalb der Diskursbegriff in den Kontext der Thematik der Öffnung des Zugriff auf den wissenschaftlichen Prozess erläutert werden -----
\subsection{Kritik}
Die Verlage haben mit Hilfe von wissenschaftlicher Journale ein zentrales Steuerungs- und Bewertungssystem in der Wissenschaft etablieren können. Dabei werden die Grundprinzipien der Wissenschaft für die verlegerischen Verwertungsinteressen genutzt und das, obwohl diese “wissenschaftlichen Grundprinzipien und Normen eigentlich ökonomischen Verwertungsinteressen zu widersprechen scheinen” \cite{hanekop_2006}. Darüber hinaus haben die Forscher in vielen Fällen wenig oder keine Verantwortung für den Einkauf der wissenschaftlichen Informationen, die er oder sie "verschenkt" \cite{steele_2006}. Die Einführung der Zitationsregister und Impact Faktoren, sowie die Definition der Kernzeitschriften, hat zur weitgehenden Erstarrung des wissenschaftliche Zeitschriftenmarktes geführt und gleichzeitig die Kapazität der kommerziellen Verlagen, sowie deren Gewinnmargen ansteigen lassen \cite{CREATe_2014}. Die Steuerungsmechanismen werden über die Messbarkeit mittels --- TODO siehe alt 2.3.4 ---- beschriebenen Methoden direkt oder indirekt ausgeübt. Dabei stehen insbesondere die Methoden, die auf der quantitativen Grundlage der Zitationsraten wissenschaftlicher Publikationen gemessen werden in der Kritik \cite{Dong_2005} und auch andere Indikatoren für die Messung von Forschungsleistungen sind hoch umstritten \cite{Hornbostel_1997} \cite{Hicks_1996} \cite{Havemann_2002}. Der Hauptkritikpunkt: Die Verfahren, um die Wirkung von Wissenschaft und damit auch die Reputation von Wissenschaftlern zu messen, sind kein eigentliches wissenschaftliches Produkt\cite{suchen} und erfassen zum Beispiel die Tätigkeit einzelner Forschergruppen zu stark \cite{schmoch_2009}. Das führt unter anderem dazu, dass der Impact Factor “kein perfektes Werkzeug (ist) um die Qualität der Artikel zu messen” und trotzdem wird er zur Bewertung von Wissenschaft genutzt, denn “(...) es gibt nichts Besseres, und er hat den Vorteil, dass er bereits lange existiert und ist daher eine gute Technik für die wissenschaftliche Bewertung”\cite{garfield_1999}. Wie “gering der Wirkungsgrad” und die Methoden zur Messung “zur Reproduktion des traditionellen wissenschaftlichen Diskurses ausfall(en), wird von dem Moment an klar, an dem ein neues und offenes Kommunikationsmedium wie das Internet als alternativer Publikations- und Verbreitungskanal für Wissenschaft zur Verfügung steht \cite{Rost_1998}. 
