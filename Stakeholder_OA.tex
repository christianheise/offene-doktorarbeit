\subsubsection{Meinungen der Interessengruppen über Open Access}
tbd

Die Hoffnung damals: Die technologischen Treiber gesteuert und organisiert von der Forschungs Community selbst, anstatt durch Fachverlage, könnten die durchschnittlichen Kosten für einen publizierten Artikel signifikant senken. Kostensenkungen von bis zu 90 Prozent\cite{hilf_2004} waren zwar schon damals unrealistisch, haben aber die Debatte weiter befeuert. "Insgesammt muss aus der Sicht des individuellen Nutzenkalküls von Wissenschaftlern, Verlagen und weiteren Einrichtungen wie Bibliotheken als auch aus Sicht gesamtwirtschaftlicher Wohlfahrtsüberlegungen die Frag gestellt werden, ob der Markt der Wissenschaftskommunikation nicht effizienter organisiert werden könnte."\cite{Hess_2006}

Grundlegendes Ziel war es das System leistungsfähiger machen und "von seinen durch den Papierdruck auferlegten Fesseln befreien. Zu den Fesseln zählten"\cite{hilf_2004}:
\begin{itemize}
\item - langer Zeitverzug vom Einreichen eines Manuskriptes bis zum Gelesen werden,
\item - komplizierter Vertriebsweg vom Verlag über Grossisten zu Bibliotheken,
\item - horrende Kosten (ca. 3.000,- Euro für die gesamte Verlagsarbeit je Artikel) mit den daraus folgenden horrenden Zeitschriftenpreisen,
\item -und daraus folgend wenige Leser, auch noch ungleich in der Welt verteilt (digital divide),
\item - unvollständige Information (aus Platzmangel), was Nachnutzungen und das Nachprüfen erschwert und somit auch Fälschungen erleichtert,
\item - nur anonymes Referieren vor der Veröffentlichung, was den Missbrauch erleichtert. 
\end{itemize}




