\subsubsection{Meinungen der Interessengruppen über Open Access}
tbd

Die Hoffnung damals: Die technologischen Treiber gesteuert und organisiert von der Forschungs Community selbst, anstatt durch Fachverlage, könnten die durchschnittlichen Kosten für einen publizierten Artikel signifikant senken. Kosteneinspaarungen von bis zu 90 Prozent\cite{hilf_2004} waren zwar schon damals unrealistisch, haben aber die Debatte weiter befeuert. "Insgesammt muss aus der Sicht des individuellen Nutzenkalküls von Wissenschaftlern, Verlagen und weiteren Einrichtungen wie Bibliotheken als auch aus Sicht gesamtwirtschaftlicher Wohlfahrtsüberlegungen die Frag gestellt werden, ob der Markt der Wissenschaftskommunikation nicht effizienter organisiert werden könnte."\cite{Hess_2006}