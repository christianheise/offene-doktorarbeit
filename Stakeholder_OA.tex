\subsubsection{Meinungen der Interessengruppen über Open Access}
tbd

Die Hoffnung damals: Die technologischen Treiber gesteuert und organisiert von der Forschungs Community selbst, anstatt durch Fachverlage, könnten die durchschnittlichen Kosten für einen publizierten Artikel signifikant senken. Kostensenkungen von bis zu 90 Prozent\cite{hilf_2004} waren zwar schon damals unrealistisch, haben aber die Debatte weiter befeuert. "Insgesammt muss aus der Sicht des individuellen Nutzenkalküls von Wissenschaftlern, Verlagen und weiteren Einrichtungen wie Bibliotheken als auch aus Sicht gesamtwirtschaftlicher Wohlfahrtsüberlegungen die Frag gestellt werden, ob der Markt der Wissenschaftskommunikation nicht effizienter organisiert werden könnte."\cite{Hess_2006}

Grundlegendes Ziel war es das System leistungsfähiger machen und "von seinen durch den Papierdruck auferlegten Fesseln befreien. Zu den Fesseln zählten" \cite{hilf_2004}:
\begin{itemize}
\item langer Zeitverzug vom Einreichen eines Manuskriptes bis zum Gelesen werden,
\item komplizierter Vertriebsweg vom Verlag über Grossisten zu Bibliotheken,
\item horrende Kosten (ca. 3.000,- Euro für die gesamte Verlagsarbeit je Artikel) mit den daraus folgenden horrenden Zeitschriftenpreisen,
\item und daraus folgend wenige Leser, auch noch ungleich in der Welt verteilt (digital divide),
\item unvollständige Information (aus Platzmangel), was Nachnutzungen und das Nachprüfen erschwert und somit auch Fälschungen erleichtert,
\item nur anonymes Referieren vor der Veröffentlichung, was den Missbrauch erleichtert. 
\end{itemize}

\textbf{Der "Heidelberger Apell"}

Am 22. März 2009 wurde auf der Webseite der „Frankfurter Allgemeinen Zeitung“ der Artikel "Geistiges Eigentum: Autor darf Freiheit über sein Werk nicht verlieren" \cite{faz_heidelberger_apell_2009} veröffentlicht. Dabei handelte es sich um einen Aufruf an vor allem an "die Bundesregierung und die Regierungen der Länder, das bestehende Urheberrecht, die Publikationsfreiheit und die Freiheit von Forschung und Lehre entschlossen und mit allen zu Gebote stehenden Mitteln zu verteidigen" \cite{ITK_2009}. Darüber hinaus sollten sich Politik, Öffentlichkeit und weitere Kreative für die "Wahrung der Urheberrechte", unter anderem in Bezug auf die Google Buchsuche "gegen eine angebliche „Enteignung“ der Autoren durch das Vorgehen von Google einerseits und durch das Publikationsmodell Open Access andererseits" \cite{WD_bundestag_2009} engagieren. Die Kritik am urheberrechtlichem Aspekt der Google Buchsuche soll in dieser Arbeit nicht berücksichtigt werden. Hier soll nur untersucht werden, inwiefern die Kritik am Publikationsmodell Open Access berechtigt ist.

Der Apell unterscheidet dabei in zwei Ebenen: \textit{International} kritisieren die Autoren "die nach deutschem Recht illegale Veröffentlichung urheberrechtlich geschützter Werke geistiges Eigentum auf Plattformen wie GoogleBooks und YouTube" und die Entwendung dieser "ohne strafrechtliche Konsequenzen". \textit{National} werden die "Eingriffe in die Presse- und Publikationsfreiheit, deren Folgen grundgesetzwidrig wären" durch die "»Allianz der deutschen Wissenschaftsorganisationen« (Mitglieder: Wissenschaftsrat, Deutsche Forschungsgemeinschaft, Leibniz-Gesellschaft, Max Planck-Institute u. a.)" angeprangert.\cite{ITK_2009}

Die Kritik der Autoren des Heidelberger Apells bezieht laut einer Untersuchung des Wissenschafltichen Diensts des Bundestags insbesondere auf drei Aspekte \cite{WD_bundestag_2009}:
\begin{enumerate}
\item Erzwungene Vertriebswege
"Eine Forschung, der man diktieren könnte, wo ihre Ergebnisse publiziert werden sollen, sei nicht mehr frei"
\item Abhängigkeitsverhältnis
\item Subventionierung von Vertriebswegen
\end{enumerate}
