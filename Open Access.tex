\subsection{Open Access} 
Der Prozess der Wissenschaft hängt von dem freien Austausch und der Verbreitung von Informationen ab\cite{cite:11}. Das System der wissenschaftlichen Kommunikation, das so seit mehreren hundert Jahren besteht, basierte auf Forschung, Kommunikation der Egebnisse in Büchern und anderen wissenschaftlichen Publikationen, der Begutachtung, dem Druck der Publikation und der Verbreitung sowie dem Verkauf an Bibliotheken und andere wissenschaftliche Institutionen gegen Kosten\cite{cite:11a}. Der offene Zugang zu wissenschaftlicher Kommunikation ist seit der Entwicklung des gedruckten Wortes aber auch eng verknüpft mit der Frage nach Lizenzen und Rechten für wissenschaftliche Informationen{cite:11aa}. Open Access beschreibt in diesem Zusammenhang eine Situation in der der Zugriff auf die unterschiedlichsten Formen wissenschaftlicher Publikationen unter freien, kostenlosen und unmittelbaren Bedingungen (Online) möglich ist{cite:11a2}. Das beruht allerdings auch auf der Maßgabe, dass der Autor das letztendliche Urheberrecht behält{cite:11ab}.

Vereinfacht kann der klassische wissenschaftliche Kommunikationsprozess im Rahmen von Publikationen wie folgt unterteilt werden\cite{cite:11b} :
\begin{enumerate}
\item Erstellung: 
Nach der Entwicklung eines konkreten Forschungsvorhabens sowie einer wissenschaftlichen Fragestellung enstehen im Rahmen der wissenschaftlichen Forschung oder der jewiligen Untersuchung Informationen\cite{cite:11c}, die im Forschungsprozess gesammelt, analysiert, ausgewertet, aufbereitet und verschriftlicht wurden\cite{cite:11d}. Diese Infromationen werden strukturiert zusammengefasst und niedergeschrieben.\cite{cite:11d}
\item Qualitätskontrolle: 
Die Qualitätskontrolle ist einer der wesentlichen Bestandteile der wissenschaftlichen Kommunikation. Sie sichert die gewonnen Erkenntnisse\cite{cite:11e} und stellt einen klaren Abrenzungsaspekt zu nicht-wissenschaftlichen Informationen und Erkenntnissen dar\cite{cite:11f}. Sie findet im Kommunikationsprozess an zwei Stellen statt. Hier ist die erste Stelle gemeint, in der vor der Produktion der Informationen in Form der Publikation, die Erkenntnisse von anderen Wissenschaftlern überprüft und gesichert werden.\cite{cite:11g}
\item Publikation: 
Nach Erstellung und Erkenntnissicherung findet die für die Distribution notwendige Publikation der Informationen statt. Vor der digitalen Revolution bestand dieser Schritt ausschließlich in dem Druck auf Papier.\cite{cite:11h}
\item Distribution: 
Der Vertrieb und die Verbreitung der Inhalte ermöglicht den Zugriff auf die Information der Forschung durch andere Wissenschaftler. Der Schritt stellt einen wichtigen Teil zur Zirkulation des neu gewonnen Wissens dar\cite{cite:11i}. Er sichert die Verfügbarkeit und den Zugriff auf die Informationen und ist essentieller Teil des Selektionsprozesses für die Erschaffung neuen Wissens.\cite{cite:11l}
\item Konsum: 
Der nächste Schritt im wissenschaftlichen Kommunikationsprozess, der wiederum den gesamten Prozess von vore beginnen lässt ist die Konsumierung. Hier geht es zum einen um die Rezeption der wissenschaftlichen Vorschung aus der "Erstellung", zum anderen kommt hier auch die zweite Stufe der Qualitätsicherung zum Tragen.\cite{cite:11j} Der Konsum von wissenschaftlicher Informationen ist dabei auch als Grundlage für die "Erstellung" neuen Wissens zu betrachten. Somit ist der Endpunkt des wissenschaftlichen Kommunikationsprozess auch gleichzeitig Ausgangspunkt für einen neuen Prozess\cite{cite:11k}.
\item Support
\end{enumerate}
Durch den weltweit steigenden Haushaltsdruck an Bibliotheken und wissenschaftlichen Insitutionen sowie dem “ungewöhnlichen Geschäftsmodell”\cite{cite:12} mit immer höheren Margen der Wissenschaftsverlage und dem Umstand, dass private Wissenschaftsverlage über öffentlich finanzierte Wissenschaftlerkarrieren entscheiden\cite{cite:13}, befindet sich das System in einer Krise\cite{cite:14}. Open Access beschäftigt sich in diesem Rahmen mit der Öffnung (Open) und dem freien Zugang (Access) zu den wissenschaftlichen Publikationen. Die größtmögliche Verbreitung wissenschaftlicher Informationen stellt dabei eine der grundlegenden Forderungen von Open Access dar\cite{cite:15} und der Einsatz von (offenen) Lizenzen ist dafür einer der Haupteinflussfaktoren\cite{cite:16}. Das Modell kann dabei in drei Modelle eingeteilt werden: Green Open Access, Golden Open Access, Gray Open Access und andere Mischformen.
\subsubsection{Offener Zugang zu wissenschaftlicher Kommunikation}
Wie in Kapitel 1. beschrieben steht der bisherige Prozess der wissenschaftlichen Kommunikation vor großen Herausforderungen. Die Zeitschriften- und Monographienkrise sowie zunehmender finanzieller Druck sowie die Veränderungen im wissenschaftlichen Kommunikationsprozess durch neue Arten und Möglichkeiten der Distribution, die steigenden Beschaffungskosten für wissenschaftliche Literatur\cite{cite:17} und die Veränderungen der Rezeption von Inhalten durch die digitale Revolution zwingen zum Umdenken oder zumindest zur Veränderung. In diesem Teil der Arbeit wird Modell des Offenen Zugangs zu Wissenschaft erläutert, analysiert und erklärt.
Die Schwerpunktsetzung beruht dabei auf den Themenbereichen wissenschaftliche Reputiation und Effizienz der Kommunikation. Zum einen beruht diese Herangehensweise auf der Annahme, dass Offenheit eine große Chance für dringend notwendige Veränderungen im wissenschaftlichen Qualitäts- und Reputationssystem (siehe Kapitel 2.3) darstellt - das betrifft die Aktivität und Qualität eines Forschers - bisher werden die Erkenntnisse der Forschung häufig erst nach langen intransparenten Verfahren bewertet und publiziert sowie nur an einen beschränkten Kreis von Rezipienten vermittelt. Dieser Teil der Arbeit beschäftigt sich mit dem Zugang zu wissenschaftlichen Informationen im Rahmen des "klassischen" Kommuniktationsprozess, nicht aber mit dem Zugriff auf Informationen oder Daten die bei Erstellung der Publikation entstehen.
\subsubsection{Chronologie der Bewegung}
Um Open Access zu verstehen und einordnen zu können ist eine historische Betrachtung der Entwicklung von wissenschaftlicher Kommunikation, aber auch von der Forderung nach Offenheit in der wissenschaftlichen Kommunikation unabdingbar. 

Die Debatte über die Zukunft des wissenschaftliche Publizierens und Kommunizierens neigt dazu, Konzepte um offene Wissenschaft als einen bisher beispiellosen und noch nie dagewesenen Paradigmenwechsel dar zu stellen\cite{cite:17a}\cite{cite:17b}. Allerdings wurden schon im antiken Griechenland, und in vielen anderen pre-modernen Zivilisationen, Wissen und Informationen als nicht besitzbare Ware angesehen\cite{cite:18} dennoch war der Austausch im Vergleich zu den heutigen Möglichkeiten heute stark beschränkt\cite{cite:17c}.

Die Geschichte von Open Access ist also auch eine Geschichte, die eng mit der Digitalisierung von Kommunikationsprozessen verknüpft ist. Open Access ist dabei kein Selbstzweck\cite{cite:17d}, sondern ein Symptom für tiefergehende Prozesse die mit der wachsenden Bedeutung der Digitalisierung in unserer Zivilisation sowie die damit einhergehenden Möglichkeiten für tiefgreifende Veränderungen im Machtgefüge zusammenhängen\cite{cite:17e}. Denn obwohl es vorher schon vereinzelte Versuche in der Wissenschaft gab komplett Informationen und Publikationen offen und frei zu kommunizieren war Open Access im Printzeitalter physisch und ökonomisch über lokale Grenzen hinaus schier unmöglich\cite{cite:18a}. Trotz dieser Grenzen gehen die ersten Experimente mit offenem Zugang und freien Lizenzen für Publikationen in der Wissenschaft bis in die 60er Jahre des vorherigen Jahrhunderts und somit schon vor der Zeit der Erfindung des Internets, zurück\cite{cite:18b}.

Spätestens jedoch im Jahr 2001 erschien Open Access als eigenes und öffentlichkeitswirksames Thema im wissenschaftlichen Diskurs\cite{cite:19}. Die Public Library of Science (PLoS) foderte in einem offenen Brief\cite{cite:20} dazu auf, die in Zeitschriften erscheinenden Forschungsberichte spätestens sechs Monate nach ihrer Erstveröffentlichung für jedermann offen und unentgeltlich einsehbar im Internet zur Verfügung zu stellen. Schon nach kurzer Zeit unterzeichneten (nach eigenen Angaben\cite{cite:19a}) rund 38.000 Wissenschaftler aus 180 Nationen das Schreiben.

Ebenfalls 2001 wurden mit der “Budapest Open Access Initative”\cite{cite:21} erstmals die Bemühungen um Open Access in einer Erklärung zusammengefasst\cite{cite:21a}. In dem Zentrum steht die Forderung nach dem freien Zugang zu wissenschaftlichen Publikationen. Sie manifestiert erstmals, dass wissenschaftliche Peer-Review-Fachliteratur “kostenfrei und öffentlich im Internet zugänglich sein sollte, so dass Interessenten die Volltexte lesen, herunterladen, kopieren, verteilen, drucken, in ihnen suchen, auf sie verweisen und sie auch sonst auf jede denkbare legale Weise benutzen können, ohne finanzielle, gesetzliche oder technische Barrieren jenseits von denen, die mit dem Internet-Zugang selbst verbunden sind. In allen Fragen des Wiederabdrucks und der Verteilung und in allen Fragen des Copyrights überhaupt sollte die einzige Einschränkung darin bestehen, den Autoren Kontrolle über ihre Arbeit zu belassen und deren Recht zu sichern, dass ihre Arbeit angemessen anerkannt und zitiert wird.” 

Die Verfassser der Berliner Erklärung gehen darüber hinaus und forderten den kostenlosen und freien Zugang nicht nur zu wissenschaftlichem Wissen in Form von Publikationen sondern auch zu den Daten: „Open Access-Veröffentlichungen umfassen originäre wissenschaftliche Forschungsergebnisse ebenso wie Ursprungsdaten, Metadaten, Quellenmaterial, digitale Darstellungen von Bild- und Graphik-Material und wissenschaftliches Material in multimedialer Form.“\cite{cite:22} Sie symbolisiert damit auch ein erweitertes Verständnis von Open Access und bildet die Grundlage für ein erstes Verständnis von Open Science. Dennoch konzentriert sie sich ausschließlich auf den geschlossenen wissenschaftlichen Prozess.

\subsubsection{Open Access Modelle}

In der einschlägigen Literatur wird in unterschiedliche Modelle von Open Access unterschieden\cite{cite:22a}. Die klassische Unterscheidung der Unterschiedlichen Modelle, wissenschaftliche Inhalte frei, offen und unmittelbar für jedermann zur Verfügung zu stellen sind der sogenannte goldene und grüne Weg.\cite{cite:22b} 

Der grüne Weg beschreibt das Modell, in dem der Autor

Der goldene Weg hingegen

Weitere, aber wenig genutzte Modelle sind

\subsubsection{Open Access Kanäle und Formate}
In diesem Teil der Arbeit soll nach den unterschiedlichen Modellen in Bezug auf den Weg der Veröffentlichung von wissenschaftlichen Inhalten als Open Access Publikationen auch auf die unterschiedlichen Open Access Kanäle und Publikationsformaten eingegangen werden.

Dabei soll in folgende unterschiedliche unterschieden: Open Access Aggregatoren, Open Access Repositorien, Open Access Jounrals, Open Access Bücher. Sie alle beschäftigen sich entweder mit bestimmten Publikationsformen der wissenschaftlichen Kommunikation oder mit den Herausforderungen die im Rahmen der Distribution und Archivierung im Umfeld der neuen Möglichkeiten von Open Access entstanden sind. 