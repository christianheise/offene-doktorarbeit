\subsubsection{Offener Zugang zu wissenschaftlicher Kommunikation}

Wie in Kapitel 1. beschrieben steht der bisherige Prozess der wissenschaftlichen Kommunikation vor großen Herausforderungen. Die Zeitschriften- und Monographienkrise, der zunehmende finanzielle Druck sowie die Veränderungen im wissenschaftlichen Kommunikationsprozess durch neue Arten und Möglichkeiten der Distribution, die steigenden Beschaffungskosten für wissenschaftliche Literatur \cite{cite:17} und die Veränderungen der Rezeption von Inhalten \cite{holub_2013_reception}, zwingen zum Umdenken oder zumindest zur Veränderung der wissenschaftlichen Kommuinkationspraxis. In diesem Teil der Arbeit wird das Modell des Offenen Zugangs zu Wissenschaft erläutert, analysiert und erklärt.

Die Schwerpunktsetzung beruht dabei auf den Themenbereichen wissenschaftliche Reputation und (Effizienz der) Kommunikation. Diese Herangehensweise beruht auf der Annahme, dass Offenheit eine große Chance für dringend notwendige Veränderungen im wissenschaftlichen Qualitäts- und Reputationssystem (siehe Kapitel 2.3) darstellt - das betrifft vor allem die Aktivität und Qualität der Wissenschaftler, deren Erkenntnisse bisher häufig erst nach langen intransparenten Verfahren bewertet und publiziert sowie nur an einen beschränkten Kreis von Rezipienten vermittelt werden aber auch die Kostenfrage. Dabei beschäftigt sich dieser Teil der Arbeit aber auch mit dem generellen Zugang zu wissenschaftlichen Informationen im Rahmen des "klassischen" Kommuniktations- und Publikationsprozess, nicht aber mit dem Zugriff auf Informationen oder Daten die bei Erstellung der Publikation entstehen. 

Als Grundlage für diese Entwicklung werden in dieser Arbeit vor allem die infrastrukturellen Veränderungen, die "seit spätestens Mitte der 1990er-Jahre entscheidend Einfluss auch auf die Wissenschaftskommunikation und das wissenschaftliche Arbeiten genommen haben" \cite{schulze_2013_open} angeführt. Dabei werden die wissenschaftlichen Informationen nicht nur in "digitaler Form konsumiert, sondern auch kollaborativ und kooperativ, zeitlich versetzt, durch teilweise räumlich weit verstreute Arbeitsgruppen und Forschungsverbünde, genutzt und weiterverarbeitet werden" \cite{schulze_2013_open}. Es muss and dieser Stelle aber darauf verwiesen werden, dass die Verbreitung und Akzeptanz von Open Access zwischen den wissenschaftlichen Disziplinen variiert \cite{cite:21a} .
