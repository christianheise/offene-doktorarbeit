\subsection{Open Access} 
\begin{quote}
„Open Access“ meint, dass [= Peer-Review-Fachliteratur] kostenfrei und öffentlich im Internet zugänglich sein sollte, sodass Interessenten die Volltexte lesen, herunterladen, kopieren, verteilen, drucken, in ihnen suchen, auf sie verweisen und sie auch sonst auf jede denkbare legale Weise benutzen können, ohne finanzielle, gesetzliche oder technische Barrieren jenseits von denen, die mit dem Internet-Zugang selbst verbunden sind. In allen Fragen des Wiederabdrucks und der Verteilung und in allen Fragen des Copyrights überhaupt sollte die einzige Einschränkung darin bestehen, den Autoren Kontrolle über ihre Arbeit zu belassen und deren Recht zu sichern, dass ihre Arbeit angemessen anerkannt und zitiert wird.
\cite{boai_2012}\end{quote}
Der Prozess der Wissenschaft hängt von dem freien Austausch und der Verbreitung von Informationen ab\cite{cite:11}. Das System der wissenschaftlichen Kommunikation, das so seit mehreren hundert Jahren besteht, basierte auf Forschung, Kommunikation der Egebnisse in Büchern und anderen wissenschaftlichen Publikationen, der Begutachtung, dem Druck der Publikation und der Verbreitung sowie dem Verkauf an Bibliotheken und andere wissenschaftliche Institutionen gegen Kosten\cite{cite:11a}. Der offene Zugang zu wissenschaftlicher Kommunikation ist seit der Entwicklung des gedruckten Wortes aber auch eng verknüpft mit der Frage nach Lizenzen und Rechten für wissenschaftliche Informationen\cite{boai_2012}. Open Access beschreibt in diesem Zusammenhang eine Situation in der der Zugriff auf die unterschiedlichsten Formen wissenschaftlicher Publikationen unter freien, kostenlosen und unmittelbaren Bedingungen (Online) möglich ist{cite:11a2} aber auch ein "alternatives Geschäftsmodell"\cite{lewis_2012_inevitability} für wissenschaftliche Publikationen. Das beruht allerdings auch auf der Maßgabe, dass der Autor das letztendliche Urheberrecht behält{cite:11ab}.

Vereinfacht kann der klassische wissenschaftliche Kommunikationsprozess im Rahmen von Publikationen wie folgt unterteilt werden\cite{cite:11b} :
\begin{enumerate}
\item Erstellung: 
Nach der Entwicklung eines konkreten Forschungsvorhabens sowie einer wissenschaftlichen Fragestellung enstehen im Rahmen der wissenschaftlichen Forschung oder der jewiligen Untersuchung Informationen\cite{cite:11c}, die im Forschungsprozess gesammelt, analysiert, ausgewertet, aufbereitet und verschriftlicht wurden\cite{cite:11d}. Diese Infromationen werden strukturiert zusammengefasst und niedergeschrieben.\cite{cite:11d}
\item Qualitätskontrolle: 
Die Qualitätskontrolle ist einer der wesentlichen Bestandteile der wissenschaftlichen Kommunikation. Sie sichert die gewonnen Erkenntnisse\cite{cite:11e} und stellt einen klaren Abrenzungsaspekt zu nicht-wissenschaftlichen Informationen und Erkenntnissen dar\cite{cite:11f}. Sie findet im Kommunikationsprozess an zwei Stellen statt. Hier ist die erste Stelle gemeint, in der vor der Produktion der Informationen in Form der Publikation, die Erkenntnisse von anderen Wissenschaftlern überprüft und gesichert werden.\cite{cite:11g}
\item Publikation: 
Nach Erstellung und Erkenntnissicherung findet die für die Distribution notwendige Publikation der Informationen statt. Vor der digitalen Revolution bestand dieser Schritt ausschließlich in dem Druck auf Papier.\cite{cite:11h}
\item Distribution: 
Der Vertrieb und die Verbreitung der Inhalte ermöglicht den Zugriff auf die Information der Forschung durch andere Wissenschaftler. Der Schritt stellt einen wichtigen Teil zur Zirkulation des neu gewonnen Wissens dar\cite{cite:11i}. Er sichert die Verfügbarkeit und den Zugriff auf die Informationen und ist essentieller Teil des Selektionsprozesses für die Erschaffung neuen Wissens.\cite{cite:11l}
\item Konsum beziehungsweise Rezeption: 
Der nächste Schritt im wissenschaftlichen Kommunikationsprozess, der wiederum den gesamten Prozess von vore beginnen lässt ist die Rezeption der veröffentlichten Inhalte. Hier geht es zum einen um die Rezeption der wissenschaftlichen Vorschung aus der "Erstellung", zum anderen kommt hier auch die zweite Stufe der Qualitätsicherung zum Tragen.\cite{cite:11j} Der Konsum von wissenschaftlicher Informationen ist dabei auch als Grundlage für die "Erstellung" neuen Wissens zu betrachten. Somit ist der Endpunkt des wissenschaftlichen Kommunikationsprozess auch gleichzeitig Ausgangspunkt für einen neuen Prozess\cite{cite:11k}.
\item Support
\end{enumerate}
An diesem Prozess sind vor allem drei Gruppen beteiligt: erstens die Wissenschaftler, als Produzenten und Konsumenten der Informationen, zweitens die Verleger, die als Intermediäre wissenschaftliche Informationen sammeln, bündeln und verkaufen, sowie drittens die Bibliotheken, die die Informationen wieder den Wissenschaftlern zur Verfügung stellen.\cite{Odlyzko_1997}

Durch den weltweit steigenden Haushaltsdruck an Bibliotheken und wissenschaftlichen Insitutionen sowie dem “ungewöhnlichen Geschäftsmodell”\cite{cite:12} mit immer höheren Margen der Wissenschaftsverlage und dem Umstand, dass private Wissenschaftsverlage über öffentlich finanzierte Wissenschaftlerkarrieren entscheiden\cite{heise_2012}, befindet sich das System in einer Krise\cite{cite:14}. Open Access beschäftigt sich in diesem Rahmen mit der Öffnung (Open) und dem freien Zugang (Access) zu den wissenschaftlichen Publikationen. Die größtmögliche Verbreitung wissenschaftlicher Informationen stellt dabei eine der grundlegenden Forderungen von Open Access dar\cite{cite:15} und der Einsatz von (offenen) Lizenzen ist dafür einer der Haupteinflussfaktoren\cite{cite:16}. Das Modell kann dabei in drei Modelle eingeteilt werden: Green Open Access, Golden Open Access, Gray Open Access und andere Mischformen.
\subsubsection{Offener Zugang zu wissenschaftlicher Kommunikation}
Wie in Kapitel 1. beschrieben steht der bisherige Prozess der wissenschaftlichen Kommunikation vor großen Herausforderungen. Die Zeitschriften- und Monographienkrise, der zunehmende finanzielle Druck sowie die Veränderungen im wissenschaftlichen Kommunikationsprozess durch neue Arten und Möglichkeiten der Distribution, die steigenden Beschaffungskosten für wissenschaftliche Literatur\cite{cite:17} und die Veränderungen der Rezeption von Inhalten durch die digitale Revolution zwingen zum Umdenken oder zumindest zur Veränderung der wissenschaftlichen Kommuinkationspraxis. In diesem Teil der Arbeit wird das Modell des Offenen Zugangs zu Wissenschaft erläutert, analysiert und erklärt.
Die Schwerpunktsetzung beruht dabei auf den Themenbereichen wissenschaftliche Reputiation und Effizienz der Kommunikation. Diese Herangehensweise beruht auf der Annahme, dass Offenheit eine große Chance für dringend notwendige Veränderungen im wissenschaftlichen Qualitäts- und Reputationssystem (siehe Kapitel 2.3) darstellt - das betrifft vor allem die Aktivität und Qualität der Wissenschaftler, deren Erkenntnisse bisher häufig erst nach langen intransparenten Verfahren bewertet und publiziert sowie nur an einen beschränkten Kreis von Rezipienten vermittelt werden. Dabei beschäftigt sich dieser Teil der Arbeit aber auch mit dem generellen Zugang zu wissenschaftlichen Informationen im Rahmen des "klassischen" Kommuniktations- und Publikationsprozess, nicht aber mit dem Zugriff auf Informationen oder Daten die bei Erstellung der Publikation entstehen. Darüber muss darauf verwiesen werden, dass die Verbreitung und Akzeptanz von Open Access zwischen den wissenschaftlichen Disziplinen variiert\cite{cite:21a} .
\subsubsection{Chronologie der Bewegung}
Um Open Access verstehen und einordnen zu können ist eine historische Betrachtung der Entwicklung von wissenschaftlicher Kommunikation, aber auch von der Forderung nach Offenheit in der wissenschaftlichen Kommunikation unabdingbar. 

Die Debatte über die Zukunft des wissenschaftliche Publizierens und Kommunizierens neigt dazu, Konzepte um offene Wissenschaft als einen bisher beispiellosen und noch nie dagewesenen Paradigmenwechsel dar zu stellen\cite{cite:17a}\cite{cite:17b}. Allerdings wurden schon im antiken Griechenland, und in vielen anderen pre-modernen Zivilisationen, Wissen und Informationen als nicht besitzbare Ware angesehen\cite{cite:18} dennoch war der Austausch im Vergleich zu den heutigen Möglichkeiten heute stark beschränkt\cite{cite:17c}.

Die Geschichte von Open Access ist also auch eine Geschichte, die eng mit der Digitalisierung von Kommunikationsprozessen verknüpft ist. Open Access ist dabei kein Selbstzweck\cite{cite:17d}, sondern ein Symptom für tiefergehende Prozesse die mit der wachsenden Bedeutung der Digitalisierung in unserer Zivilisation sowie die damit einhergehenden Möglichkeiten für tiefgreifende Veränderungen im Machtgefüge zusammenhängen\cite{cite:17e}. Denn obwohl es vorher schon vereinzelte Versuche in der Wissenschaft gab komplett Informationen und Publikationen offen und frei zu kommunizieren war Open Access im Printzeitalter physisch und ökonomisch über lokale Grenzen hinaus schier unmöglich\cite{cite:18a}. Trotz dieser Grenzen gehen die ersten Experimente mit offenem Zugang und freien Lizenzen für Publikationen in der Wissenschaft bis in die 60er Jahre des vorherigen Jahrhunderts und somit schon vor der Zeit der Erfindung des Internets, zurück\cite{cite:18b}. 

In Deutschland nahmen bis Anfang der 1990er Jahre die wissenschaftlichen Verlage in Deutschland eine marktbeherrschende Stellung ein und waren exklusiver Dienstleister bei der Veröffentlichung wissenschaftlicher Informationen \cite{schloegl_2005}\cite{offhaus_2012_institutionelle_repos}. Diese Vormachtstellung der Verlagen im wissenschaftlichen Publikationssystem war auf drei Säulen begründet\cite{offhaus_2012_institutionelle_repos}\cite{bargheer_2006_open}: 
\begin{enumerate}
\item  "Urheberrecht, wonach Verlage [...] weitgehende Ansprüche an dem veröffentlichten Werk erwerben“;
\item "redaktionelle Themenbündelung (bundling)“;
\item "Qualitätssicherung durch Begutachtung (Peer Review)"
\end{enumerate}

Infolgedessen kam es kurz vor der Jahrtausendwende zur sogenannten Zeitschriftenkrise\cite{suchen}, in der die Velage die Situation nutzten um teilweise drastischen Preiserhöhungen durchzusetzen. Gleichzeitig standen und stehen die Wissenschaftler unter einem starken Publikationszwang, der mit "Publish or Perish"\cite{CLAPHAM_2005} beziehungsweise "impact factor fever"\cite{Cherubini_2008} und "impact factor race"\cite{Brischoux_2009} beschrieben wurde\cite{offhaus_2012_institutionelle_repos}.

Spätestens jedoch im Jahr 2001 erschien Open Access als eigenes und öffentlichkeitswirksames Thema im wissenschaftlichen Diskurs\cite{cite:19}. Die Public Library of Science (PLoS) foderte in einem offenen Brief im Mai 2001\cite{cite:20} dazu auf, ab September 2001 nur noch in den Zeitschriften zu veröffentlichen beziehungsweise nur noch die Zeitschriften zu reviewen, zu editieren und zu abonnieren deren Beiträge spätestens sechs Monate nach ihrer Erstveröffentlichung für jedermann im Internet kostenlos und unentgeltlich einsehbar sind. Schon nach kurzer Zeit unterzeichneten (nach eigenen Angaben\cite{cite:19a}) rund 38.000 Wissenschaftler aus 180 Nationen das Schreiben. Diese Brief kann als Auftakt zu einem 20-monatigen theoretischen Schub gesehen werden, der in drei der bis heute wichtigsten Erklärungen im Bereich der Öffnung des Zugangs zu wissenschaftlicher Kommunikation gesehen werden: Die Erklärung der Budapest Open Access Initiative, die Berliner Erklärung und die Bethesda Erklärung.\cite{CREATe_2014}

Im gleichen Jahr wie der PLoS-Brief, wurden mit der “Budapest Open Access Initative” (BOAI)\cite{boai_2012} erstmals die Bemühungen um Open Access in einer eigenen Erklärung zusammengefasst\cite{cite:21a}, die Rahmen einer Konferenz des Open Society Institutes in Budapest entstand. In dem Zentrum stand die Forderung nach dem freien Zugang zu wissenschaftlichen Publikationen. Sie manifestiert erstmals, dass wissenschaftliche Peer-Review-Fachliteratur “kostenfrei und öffentlich im Internet zugänglich sein sollte, so dass Interessenten die Volltexte lesen, herunterladen, kopieren, verteilen, drucken, in ihnen suchen, auf sie verweisen und sie auch sonst auf jede denkbare legale Weise benutzen können, ohne finanzielle, gesetzliche oder technische Barrieren jenseits von denen, die mit dem Internet-Zugang selbst verbunden sind. In allen Fragen des Wiederabdrucks und der Verteilung und in allen Fragen des Copyrights überhaupt sollte die einzige Einschränkung darin bestehen, den Autoren Kontrolle über ihre Arbeit zu belassen und deren Recht zu sichern, dass ihre Arbeit angemessen anerkannt und zitiert wird.”\cite{boai_2012} 

Nicht mal zwei Jahre später, im Juni 2003, verabschiedeten im US-Bundesstaat Maryland eine Gruppe von Forschungsförderer, wissenschaftlichen Gesellschaften, Verleger, Bibliothekare, Forschungseinrichtungen und einzelnen Wissenschaftlern das "Bethesda Statement on Open Access Publishing".\cite{suchen} Ziel der Erklärung war die Stimulation der Diskussion in der biomedizinischen Forschung, "wie man schnellstmöglich den offenen Zugang zu der primären wissenschaftlichen Literatur in der Biomedizin erreichen könnte"\cite{suchen}. Wie bereits in der Erklärung der BOAI erklärten die Autoren Bedingungen an diese Art des offenen Zugangs:
1. 

Ein weiterer Meilenstein für die Verbreitung von Open Access auf dem europäischen Kontinent waren die "Berlin Konferenzen"\cite{CREATe_2014}. Die erste Tagung wurde 2003 von der Max-Planck-Gesellschaft und dem Projekt European Cultural Heritage Online (ECHO) organisiert, um über "Zugangsmöglichkeiten zu Forschungsergebnissen" zu diskutieren. In diesem Rahmen entstand auch die "Berliner Erklärung über den offenen Zugang zu wissenschaftlichem Wissen"\cite{berliner_erklaerung_2003}, in der die Verfasser über die Budapester Erklärung hinaus gehen und neben dem kostenlosen und freien Zugang zu wissenschaftlichem Wissen in Form von Publikationen auch den freien und offenen Zugang zu den Daten fordern: „Open Access-Veröffentlichungen umfassen originäre wissenschaftliche Forschungsergebnisse ebenso wie Ursprungsdaten, Metadaten, Quellenmaterial, digitale Darstellungen von Bild- und Graphik-Material und wissenschaftliches Material in multimedialer Form.“\cite{berliner_erklaerung_2003} Sie symbolisiert damit auch ein erweitertes Verständnis von Open Access und bildet die Grundlage für ein erstes Ansatzpunkt zur Definition von Open Science, konzentriert sich aber dennoch ausschließlich auf den abgeschlossenen wissenschaftlichen Prozess\cite{suchen}.

\subsubsection{Open Access Modelle}

In der einschlägigen Literatur wird in viele unterschiedliche Formen von Open Access unterteilt\cite{CREATe_2014} und es herschen unterschiedliche Auffassungen über die verschiedenen Modelle von Open Access\cite{CREATe_2014}\cite{cite:22b}\cite{lewis_2012_inevitability}. In dieser Arbeit soll grundsätzlichen der Unterscheidung in den goldene und den grünen Weg gefolgt werden. Eine zweite Ebene der Unterteilung in hybride, alternative und sonstige Formen soll alle weiteren, in der Literatur aufkommenden Formen gerecht werden. Abschliessend werden weitere auch die Formen genannt, die zwar häufig als Open Access bezeichnet werden, aber den gängigen Deklarationen\cite{boai_2012} und Definitionen nicht gerecht werden. Der Umstand, dass eine klare Klassifizierung schwer möglich ist, kann damit begründet werden, dass es "keine formelle Struktur, keine offizelle Organisation und kein ernannter Führer" gibt, der die Open Access Bewegung antreibt\cite{poynder_2011_suber}.

Der grüne Weg beschreibt das Modell, in dem der Autor

Der goldene Weg hingegen stellt die Publikation unmittelbar nach Fertigstellung zur Verfügung. Hierbei gibt es auch die Unterscheidung, dass einige Verlage die Publikation mit Verzögerung zur Verfügung stellen, in der Literatur wird in diesem Zusammenhang von verzögertem goldenen Open Access gesprochen\cite{lewis_2012_inevitability}. Im Rahmen anderer Modelle, vornehmlich bei der Publikation in Zeitschriften, wird dem Autor die Möglichkeit eingeräumt durch zusätzliche Zahlung die Publikation offen und frei zur Verfügung zu stellen\cite{lewis_2012_inevitability}.

Als Kernunterschied zwischen den beiden Modellen kann hervorgehoben werden, dass die grüne und hybride Form sowie das verzögerte Open Access, das klassische Geschäftsmodell der Verlage nicht beeinträchtigt, währen der goldene Weg auch ohne das bisherige Geschäftsmodell des Verlags auskommen kann\cite{lewis_2012_inevitability}.

Weitere, aber wenig genutzte Modelle sind

Hybride Modelle

\subsubsection{Open Access Kanäle und Formate}
In diesem Teil der Arbeit soll nach den unterschiedlichen Modellen in Bezug auf den Weg der Veröffentlichung von wissenschaftlichen Inhalten als Open Access Publikationen auch auf die unterschiedlichen Open Access Kanäle und Publikationsformaten eingegangen werden.

Dabei soll in folgende unterschiedliche unterschieden: Open Access Aggregatoren, Open Access Repositorien, Open Access Jounrals, Open Access Bücher. Sie alle beschäftigen sich entweder mit bestimmten Publikationsformen der wissenschaftlichen Kommunikation oder mit den Herausforderungen die im Rahmen der Distribution und Archivierung im Umfeld der neuen Möglichkeiten von Open Access entstanden sind. 

Da es eine enge Verknüpfung zwischen der Entwicklung von Repositorien und der Open-Access-Bewegung gibt\cite{offhaus_2012_institutionelle_repos}, soll hier auf die Rolle der Repositorien als Kanal für die Verbreitung von Publiaktionen eingegangen werden. Institutionelle Repositorien sind ein Instrument für wissenschaftliche Einrichtungen wie etwa Universitäten, um ihre Publikationen frei zugänglich zu machen\cite{dobratz_2007_open}.

Institutionelle Repositorien haben potenziell erhebliche Vorteile für die Institutionen, wenn sie in die Universität ganzheitlichen Rahmenbedingungen integriert sind\cite{steele_2006}. Diese Repositorien können auch für die Lernumgebungen und die Marketingaktivitäten einer Universität einen wichtige Rolle spielen, so können sie eine den Universitäten Output dokumentieren und den Zugang zu institutionellen Austausch verbessern\cite{steele_2006}. Ökonomisch rentieren sie sich vor allem dann, wenn skaleneffekte eintreten und in Verbünden agiert wird.\cite{blythe_2005value} Neben den institutionellen sind auch fachliche oder andere Arten von Repositorien eng mit der Open Access Bewegung verknüpft, sie werden in diesem Kapitel aber nicht weiter unterschieden.