\subchapter{Open Access} 
Der Prozess der Wissenschaft hängt von dem freien Austausch und der Verbreitung von Informationen ab . Das System der wissenschaftlichen Kommunikation, das so seit mehreren hundert Jahren besteht, basierte auf kostenloser Forschung und anderen wissenschaftlichen Publikationen, kostenloser Begutachtung, dem Druck und der Publikation gegen Kosten und dem Verkauf an Bibliotheken und andere wissenschaftliche Institutionen für die Weiterverbreitung. Das System basiert dabei auf den folgenden sechs Schritten:
\begin{enumerate}
\item Erstellung
\item Qualitätskontrolle
\item Produktion
\item Distribution
\item Konsumierung
\item upport
\end{enumerate}
Durch den weltweit steigenden Haushaltsdruck an Bibliotheken und wissenschaftlichen Insitutionen sowie dem “ungewöhnlichen Geschäftsmodell”  mit immer höheren Margen der Wissenschaftsverlage und dem Umstand, dass private Wissenschaftsverlage über öffentlich finanzierte Wissenschaftlerkarrieren entscheiden , befindet sich das System in einer Krise . Open Access beschäftigt sich in diesem Rahmen mit der Öffnung (Open) und dem freien Zugang (Access) zu den wissenschaftlichen Publikationen. Die größtmögliche Verbreitung wissenschaftlicher Informationen stellt dabei eine der grundlegenden Forderungen von Open Access dar  und der Einsatz von (offenen) Lizenzen ist dafür einer der Haupteinflussfaktoren . Das Modell kann dabei in drei Modelle eingeteilt werden: Green Open Access, Golden Open Access, Gray Open Access und andere Mischformen.
