\subsubsection{Entwicklung der Bewegung}
Wissenschaft und Offenheit sind seit jeher zwei stark verbundene Elemente. “Open Science” ist dabei ein Begriff der historisch sehr eng mit der Entwicklung von kollaborativen Arbeiten durch neue Kommunikationstechniken verbunden ist. Open Science ist im Rahmen der Open Movements als Evolution zur reinen Öffnung des Zugangs (Open Access) zu wissenschaftlichen Publikationen zu verstehen. Eine klare Definition für den Sammelbergiff steht jedoch noch aus. Dabei spielt insbesondere, die Entwicklung der Tradition für eine "offenen Wissenschaft" im siebzehnten Jahrhundert einen Ansatzpunkt, da dieser historische Übergang noch nicht erforscht ist.  Dieser Strang der Forschung ist eine sinnvolle Analyse, um grundlegende Argumente für Open Science in wissenschaftliche Forschung zu untersuchen. 

In diesem Kapitel werden die historischen Aspekte der Veröffentlichung und Verbreitung von wissenschaftlichen Informationen chronologisch unter der Berücksichtigung der Frage, wie Wissen der Allgemeinheit zur Verfügung gestellt wird und wurde, erfasst und erläutert. 

Die Verschlüsselungs- und Patentwut zur Wahrung eines möglichen kommerziellen Vorteils durch Wissenschaft im Rahmen öffentlich-finanzierter Forschung, geht dabei bis auf die xxxx Jahre zurück. ### Beispiel Galileo, Kepler, Newton ### Das Ergebnis dieser Wut war eine Debatte über die Verfügbarkeit der wissenschaftlichen Arbeit und die Entlohnung der “Erfinder“ im wissenschaftlichen System. 
