\chapter{Einleitung} 

\begin{quote}
scientia donum dei est unde vendi non potest
\end{quote}Eine unbeschränkte und offene Kommunikation sowie die Kenntnis des gegenwärtigen Wissensstandes ist für wissenschaftliche Forschung und deren Aufgabe neues Wissen zu produzieren und dem gesellschaftlichen Auftrag des Wissenschaftssystems gerecht zu werden unverzichtbar \cite{Hanekop_2014} \cite{glaeser2006} \cite{gibbons_1994} \cite{Luhmann1998}. Der aktuelle Zustand des wissenschaftlichen Kommunikationssystems wird diesem Auftrag nicht gerecht \cite{suchen}. Mit der Etablierung des Internets als neuen Kanal für die Kommunikation und den Austausch von Informationen wurden große Erwartungen für die völlig neuen Möglichkeiten des Wissenstransfers geweckt \cite{Hanekop_2014} \cite{schulze_2013_open} \cite{albert_2006_open_implications} \cite{Goodrum_2001} \cite{Lawrence_1999}. 

Im Rahmen dieses digitalen Wandels stehen das Universitätsystem, aber auch andere Bildungseinrichtungen und ihre Bibliotheken vor vielen großen Herausforderungen \cite{Harter2006} \cite{Gu_don_2004} \cite{osterloh2008anreize}. Eine Herausforderung ist die Wahrung der Freiheit von Wissenschaft und Forschung auf der einen,  sowie der Forderung nach besseren Steuerungs- und Leistungsprozessen in Forschung und Lehre \cite{Adler_2009} \cite{gibbons_1994} auf der anderen Seite. Von herausragender Bedeutung in diesem Zusammenhang stehen die “gestörten Gleichgewichte im wissenschaftlichen Publikationssystem” \cite{cite:0} und das "kaputten wissenschaftlichen Reputationssystem" \cite{suchen}. Friedrich Kittler hat diese Aspekte in einer Rede "Wissenschaft als Open-Source-Prozeß" im Jahr 1999 wie folgt zusammengefasst: "mit Freiheit von Quellcode steht und fällt auch die Freiheit der Wissenschaft". Mit anderen Worten, dass die "Verarbeitung des Wissens technisch reproduzierbar" \cite{cite:1} und transparent kontrollierbar \cite{suchen} bleiben muss. Vor allem genuine Universitäten laufen Gefahr, ihre Bedeutung als exklusiver Ort der Wissensproduktion \cite{suchen} und -evulation \cite{suchen} zu verlieren.

Als Gründe für diese Entwicklung werden in der Literatur unter anderem folgende Faktoren genannt \cite{suchen}: Mit der Privatisierung der Verarbeitung, Speicherung und Übertragung von Wissen hörten Universitäten auf selber Bücher zu verlegen \cite{cite:0}. In einem Erklärungsversuch für diesen Trend stellt Peter Weingart fest, dass „die Wirtschaft (zunehmend) eine öffentliche Finanzierung der Wissenschaft und der Wissensproduktion". Gleichzeitig, so Weingard weiter, werden "im Endeffekt aber (...) die privat(-wirtschaftliche) Aneignung und Nutzung des produzierten Wissens erwartet“ \cite{cite:2}. Dieses Grundprinzip der uneigennützigen, kollektiven Wissensproduktion, um unentgeltlich an wissenschaftliche Informationen von den wissenschaftlichen Autoren zu gelangen, wird von Verlagen genutzt. Neben dem entgeldlichen Vertrieb der wissenschaftlichen Informationen erbringen sie den Autoren als Gegenleistung (durch die Privatisierung des wissenschafltichen Reputationssystems \cite{suchen}) die Chance auf Anerkennung und Reputation \cite{cite:21a}. 

Das steht dem Grundsatz, dass es der Wissenschaft im Kern um Erkenntnisse und die uneingeschränkte Zurverfügungstellung dieser geht \cite{hanekop_2006}, diametral entgegen \cite{offhaus_2012_institutionelle_repos}. Dieser Widerspruch wird ebenfalls im aktuellen Steuerungssystem der Wissenschaft deutlich, in dem anhand der wissenschaftlichen Reputation von Wissenschaftlern und Insitutionen Mittel und Stellen verteilt werden \cite{cite:4}.

Diese Entwicklungen mündeten letztendlich in eine wissenschaftlichen Publikations- und Kommunikationskrise, geprägt durch steigenden Kostendruck, Preissteigerungen und die Einschränkung des Zugriffs auf wissenschaftliche Informationen \cite{Hess_2006}. Dem gegenüber stehen die neuen Möglichkeiten der Digitalisierung und Globalisierung für die Wissensverbreitung. Da jede Beschränkung des Zugangs zu Wissen aber aber auch die Erstellung von neuem Wissen beschränkt \cite{cite:5} \cite{cite:8} ist dieser Schwebezustand unhaltbar \cite{suchen}. 

Die Suche nach einem Ausweg aus der Publikations- und Kommunikationskrise zeichnet sich in den anhaltenden Forderungen nach der Öffnung von Wissen und in der Suche nach Alternativen für das geschlossene wissenschaftliche Publikations- und Kommunikationssystem ab. Das folgt auch der Annahme, dass offene Innovation und offene wissenschaftliche Kommunikation den privaten und stattlichen Forschungsbereich effizienter machen sowie den industriellen Fortschritt beschleunigen \cite{cite:7}. 

Im Rahmen dieser erstmals vor über 20 Jahren artikulierten Forderungen nach Öffnung der in den letzten 350 Jahren nur marginal veränderten wissenschaftlichen Kommunikation \cite{poynder_2011_suber bzw suchen}, befinden wir uns erst jetzt inmitten eines "radikalen Wandels" \cite{poynder_2011_suber} tradierter Kommunikationssysteme. Ungeachtet dessen, ist "das etablierte Publikationssystem der Verlage auch nach zwei Jahrzehnten weitgehend stabil" \cite{Hanekop_2014}. Die analog gedruckten und bewährten Journale sowie andere Publikationsformen der großen wissenschaftlichen Verlage werden mit nahezu unverändertem Geschäftsmodell digital verbreitet \cite{Hanekop_2014} \cite{boai_2012} und von der wissenschaftlichen Community rezipiert \cite{suchen}.

Im Rahmen der Arbeit soll im Zusammenhang mit den genannten Herausforderungen und Möglichkeiten untersucht werden, welche Auswirkungen der digitale Wandel und die Forderung nach Öffnung der Wissenschaft beziehungsweise der wissenschafltichen Kommunikation auf Universitäten, wissenschaftliche Einrichtungen aber auch Wissenschaftler haben. Von besonderem Interesse in diesem Zusammenhang sind die Unterschiede von reinem Zugang zu Wissen auf der einen und dem kompletten Zugriff auf Wissen auf der anderen Seite. Sowie die Fragen, ob es sich dabei tatsächlich um einen Paradigmenwechsel handelt. Die Herausforderungen der Netzkultur, das Wissen frei(er) zugänglich zu machen oder machen zu müssen und der Umstand, dass die meisten wissenschaftlichen Informationen der Allgemeinheit nicht zugänglich sind \cite{cite:6} steht um Zentrum der Betrachtungen. Diese Herausforderungen sind in Relation zu der Massifizierung und Neoliberalisierung der Universität zu setzen und in einen historischen Kontext zu stellen. Die Arbeit strebt an ausgewogen zu untersuchen, welche Argumente für und welche gegen die Öffnung wissenschaftlicher Kommunikation aus Sicht der am wissenschaftlichen Kommunikationssystem Beteiligten sprechen.

In diser Einleitung wird ergänzend der Aufbau der Arbeit und der theoretische Bezugsrahmen genauer erläutert. Abschließend wird die Relevanz des Themas dargestellt.