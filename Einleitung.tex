\chapter{Einleitung} 

\begin{quote}
scientia donum dei est unde vendi non potest
\end{quote}Eine unbeschränkte und offene Kommunikation sowie die Kenntnis des gegenwärtigen Wissensstandes ist für wissenschaftliche Forschung und deren Aufgabe neues Wissen zu produzieren und dem gesellschaftlichen Auftrag des Wissenschaftssystems gerecht zu werden unverzichtbar \cite{Hanekop_2014} \cite{glaeser2006} \cite{gibbons_1994} \cite{Luhmann1998}. Mit der Etablierung des Internets als neuen Kanal für die Kommunikation und den Austausch von Informationen wurden große Erwartungen für die völlig neuen Möglichkeiten des Wissenstransfers geweckt \cite{Hanekop_2014} \cite{albert_2006_open_implications} \cite{Goodrum_2001} \cite{Lawrence_1999}. 

Im Rahmen dieses digitalen Wandels stehen das Universitätsystem, aber auch andere Bildungseinrichtungen und ihre Bibliotheken vor vielen großen Herausforderungen \cite{Harter2006} \cite{Gu_don_2004} \cite{osterloh2008anreize}. Eine der wichtigsten ist die der Wahrung der Freiheit der Wissenschaft und Forschung in Zeiten der digitalen Revolution sowie der Forderung nach besseren Steuerungs- und Leistungsprozessen in Forschung und Lehre \cite{Adler_2009} \cite{gibbons_1994} aber auch der Umgang mit den “gestörten Gleichgewichten im wissenschaftlichen Publikationssystem” \cite{cite:0}. Friedrich Kittler hat das in einer Rede "Wissenschaft als Open-Source-Prozeß" im Jahr 1999 wie folgt zusammengefasst: "mit Freiheit von Quellcode steht und fällt auch die Freiheit der Wissenschaft". Damit ist gemeint, dass die "Verarbeitung des Wissens technisch reproduzierbar" \cite{cite:1} und kontrollierbarer bleiben muss. Vorallem die europäische Universität verliert allerdings in dieser Hinsicht im Rahmen der Privatisierung seit den 90er Jahren immer weiter ihre Bedeutung als exklusiver und freier Ort der Wissensproduktion \cite{suchen}.

Als Treiber für diese Entwicklung werden unter anderem follgende Faktoren genannt: Mit der Privatisierung der Verarbeitung, Speicherung und Übertragung von Wissen hörten Universitäten auf selber Bücher zu verlegen \cite{cite:0}. In einem Erklärungsversuch für das Verständnis seitens der Verlage stellt Peter Weingart diesbezüglich fest, dass „die Wirtschaft (zunehmend) eine öffentliche Finanzierung der Wissenschaft und der Wissensproduktion, im Endeffekt aber gleichzeitig die privat(-wirtschaftliche) Aneignung und Nutzung des produzierten Wissens erwartet“ \cite{cite:2}. Verlage nutzen dieses Grundprinzip der uneigennützigen, kollektiven Wissensproduktion, um unentgeltlich an wissenschaftliche Informationen von den wissenschaftlichen Autoren zu gelangen. Neben dem entgeldlichen Vertrieb der wissenschaftlichen Informationen erbringen sie den Autoren als Gegenleistung die Chance auf Anerkennung und Reputation \cite{cite:21a}. 

Das steht allerdings der Annahme, dass es der Wissenschaft im Kern aber um Erkenntnisse und diese der Gesellschaft, insbesondere aber den Wissenschaftlern als öffentliches Gut uneingeschränkt zugänglich sein sollten \cite{hanekop_2006}, diametral gegenüber \cite{offhaus_2012_institutionelle_repos} und manifestiert sich spätestens dadurch, dass im Steuerungssystem der Wissenschaft anhand der wissenschaftlicher Reputation Mittel und Stellen verteilt werden \cite{cite:4}.

Diese Entwicklungen mündeten letztendlich in eine wissenschaftlichen Publikations- und Kommunikationskrise, geprägt durch steigenden Kostendruck, Preissteigerungen und die Einschränkung des Zugriffs auf wissenschaftliche Informationen durch Wissenschaft und Forschung auf der einen und die Möglichkeiten der Digitalisierung und Globalisierung auf der anderen Seite. Da jede Beschränkung des Zugangs zu Wissen aber aber auch die Erstellung von neuem Wissen beschränkt \cite{cite:5} \cite{cite:8} ist dieser Schwebezustand aber unhaltbar. So resultierte die Krise in der Forderung nach der Öffnung von Wissen, in der Überzeugung, dass offene Innovation und offene wissenschaftliche Kommunikation den privaten und stattlichen Forschungsbereich effizienter machen sowie den industriellen Fortschritt beschleunigen \cite{cite:7} und der Suche nach Alternativen für das geschlossene wissenschaftliche Publikations- und Kommunikationssystem. Auch wenn es bemerkenswert ist, dass im Rahmen dieser erstmal vor über 20 Jahren artikulierten Forderungen nach Öffnung der wissenschaftlichen Kommunikation, wir uns in Mitten des "radikalen Wandels eines Publikationssystem befinden, dass sich in den letzten 350 Jahren nur sehr wenig verändert hat" \cite{poynder_2011_suber}, bleibt "das etablierte Publikationssystem der Verlage auch nach zwei Jahrzehnten weitgehend stabil"\cite{Hanekop_2014}. Die analog gedruckten und bewährten Journale sowie andere Publikationen der großen wissenschaftlichen Verlage werden jetzt mit dem gleichen Geschäftsmodell meistens einfach nur digital vertrieben und verbreitet \cite{Hanekop_2014} \cite{boai_2012}.

Im Rahmen der Arbeit soll untersucht werden, wie Universitäten, wissenschaftliche Einrichtungen aber auch Wissenschaftler auf den digitalen Wandel und die Forderung nach Öffnung der Wissenschaft beziehungsweise der wissenschaftlichen Kommunikation reagieren. Welche Unterschiede es beim reinen Zugang zu Wissen und beim Zugriff auf Wissen gibt sowie ob es sich dabei wirklich um einen Paradigmenwechsel handelt. Im Vordergrund stehen dabei auch die Herausforderungen der Netzkultur, das Wissen frei(er) zugänglich zu machen oder machen zu müssen und der Umstand, dass die meisten wissenschaftlichen Informationen der Allgemeinheit nicht zugänglich sind \cite{cite:6}. Diese Herausforderungen sind in Relation zu der Massifizierung und Neoliberalisierung der Universität zu setzen und in einen historischen Kontext zu stellen. Dabei soll ebenfalls ausgewogen untersucht werden, welche Argumente für und gegen die Öffnung wissenschaftlicher Kommunikation aus Sicht der am wissenschaftlichen Kommunikationssystem beteiligten Akteure sprechen.

In der Einleitung wird der Aufbau der Arbeit und der theoretische Bezugsrahmen genauer erläutert. Abschließend wird die Relevanz des Themas dargestellt.