\chapter{Einleitung} 
\begin{quote}
scientia donum dei est unde vendi non potest
\end{quote}Eine unbeschränkte und offene Kommunikation sowie die Kenntnis des gegenwärtigen Wissensstandes ist für wissenschaftliche Forschung und deren Aufgabe neues Wissen zu produzieren und dem gesellschaftlichen Auftrag des Wissenschaftssystems gerecht zu werden unverzichtbar\cite{Hanekop_2014}\cite{glaeser2006}\cite{gibbons_1994}\cite{Luhmann1998}. Mit der Etablierung des Internets als neuen Kanal für die Kommunikation und den Austausch von Informationen wurden große Erwartungen für die völlig neuen Möglichkeiten des Wissenstransfers geweckt\cite{Hanekop_2014}\cite{Goodrum_2001}\cite{Lawrence_1999}. 

Im Rahmen diese digitalen Wandels stehen das Universitätsystem, aber auch andere Bildungseinrichtungen und ihre Bibliotheken vor vielen großen Herausforderungen\cite{Harter2006}\cite{Gu_don_2004}\cite{osterloh2008anreize}. Eine der wichtigsten ist die der Wahrung der Freiheit der Wissenschaft und Forschung in Zeiten der digitalen Revolution sowie der Forderung nach besseren Steuerungs- und Leistungsprozessen in Forschung und Lehre\cite{Adler_2009}\cite{gibbons_1994} aber auch der Umgang mit den “gestörten Gleichgewichten im wissenschaftlichen Publikationssystem”\cite{cite:0}. Friedrich Kittler hat das in einer Rede "Wissenschaft als Open-Source-Prozeß" im Jahr 1999 wie folgt zusammengefasst: "mit Freiheit von Quellcode steht und fällt auch die Freiheit der Wissenschaft". Damit ist gemeint, dass die "Verarbeitung des Wissens (im Rahmen der digitalen Revolution) technisch reproduzierbar"\cite{cite:1} und kontrollierbarer wird. Vorallem die europäische Universität verliert so spätestens seit den 90er Jahren immer weiter ihre Bedeutung als exklusiver und freier Ort der Wissensproduktion.

Als Treiber für diese Entwicklung sind können unter anderem follgende Faktoren genannt werden: 
\begin{enumerate}
\item Auf der einen Seite, der Umgang mit dem System der Bücher und der Verlage - bis zur Erfindung des Buchdrucks ebenfalls eine der exklusiven Aufgaben der Universität. Mit der Privatisierung der Verarbeitung, Speicherung und Übertragung von Wissen hörten Universitäten auf selber Bücher zu verlegen. In einem Erklärungsversuch für das Verständnis seitens der Verlage stellt Peter Weingart diesbezüglich fest, dass „die Wirtschaft (zunehmend) eine öffentliche Finanzierung der Wissenschaft und der Wissensproduktion, im Endeffekt aber gleichzeitig die privat(-wirtschaftliche) Aneignung und Nutzung des produzierten Wissens erwartet“\cite{cite:2}. Verlage nutzen das Grundprinzip der uneigennützigen, kollektiven Wissensproduktion, um unentgeltlich an wissenschaftliche Informationen von den wissenschaftlichen Autoren zu gelangen. Neben dem entgeldlichen Vertrieb der wissenschaftlichen Informationen erbringen sie den Autoren als Gegenleistung die Chance auf Anerkennung und Reputation. Das steht der Annahme, dass es der Wissenschaft im Kern aber um Erkenntnisse und diese der Gesellschaft, insbesondere aber den Wissenschaftlern als öffentliches Gut uneingeschränkt zugänglich sein sollten\cite{hanekop_2006}, diametral entgegen. Das manifestiert sich spätestens dadurch, dass anhand der wissenschaftlicher Reputation Mittel und Stellen verteilt werden\cite{cite:4}.
\item Die Überzeugung, dass offene Innovation und offene wissenschaftliche Kommunikation den privaten und stattlichen Forschungsbereich effizienter macht sowie den industriellen Fortschritt beschleunigt\cite{cite:7}.
\item Die Feststellung, dass jede Beschränkung im Zugang zu Wissen auch die Erstellung von neuem Wissen verhindert\cite{cite:5}\cite{cite:8}.  Die meisten wissenschaftlichen Informationen sind der Allgemeinheit nicht zugänglich und kann nur in Universitäten und Forschungseinrichtungen durch wissenschaftliche Mitarbeiter, Studenten und Professoren abgerufen werden\cite{cite:6}. 
\end{enumerate}	

Im Rahmen der Arbeit soll untersucht werden, wie Universitäten, wissenschaftliche Einrichtungen aber auch Wissenschaftler auf den digitalen Wandel reagieren. Im Vordergrund sehen dabei die neuen Herausforderungen aus der Netzkultur, das Wissen frei(er) zugänglich zu machen oder machen zu müssen. Diese Herausforderung ist in Relation zu der Massifizierung und Neoliberalisierung der Universität zu setzen und in einen historischen Kontext zu stellen. Dabei soll ebenfalls untersucht werden, welche Argumente für und gegen die Öffnung wissenschaftlicher Kommunikation aus sicht der Akteure sprechen.

Trotz der lange prognostizierten Veränderungen in der Wissenschaftskommunikation muss an dieser Stelle ebenfalls hervorgehoben werden, dass sich "das etablierte Publikationssystem der Verlage auch nach zwei Jahrzehnten weitgehend stabil"\cite{Hanekop_2014} verhält. Die analog gedruckten und bewährten Journale und andere Publikationen der großen wissenschaftlichen Verlage werden mit dem gleichen Geschäftsmodell nun digital vertrieben und verbreitet\cite{Hanekop_2014}.

In der Einleitung wird der Aufbau der Arbeit und der theoretische Bezugsrahmen genauer erläutert. Abschließend wird die Relevanz des Themas dargestellt.
