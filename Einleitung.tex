\chapter{Einleitung}

Uneingeschränkte und offene Kommunikation sowie die Kenntnis des gegenwärtigen Wissensstandes werden als wichtige Voraussetzungen für jede wissenschaftliche Forschung betrachtet \cite{Glaeser_2006} \cite{Gibbons_1994}. Offenheit und Transparenz werden zudem als wesentliche Bestandteile einer Ethik der Wissenschaft bezeichnet \cite{Peters_2014} \cite{Resnik_2005} und sind Grundlage für den gesellschaftlichen Auftrag des Wissenschaftssystems \cite[:3]{Hanekop_2014}, neues überprüfbares Wissen zu produzieren und zu verbreiten \cite[:551]{Luescher_2014} \cite[:298]{Luhmann_1998} \cite[:100]{Graefen_2007}. In dieser Arbeit wird untersucht, welche Auswirkungen die Digitalisierung und die Forderungen nach Öffnung von Wissenschaft auf die wissenschaftliche Kommunikation von Universitäten, wissenschaftlichen Einrichtungen und von den einzelnen Wissenschaftlern und Wissenschaftlerinnen haben.

Die gestörten Gleichgewichte im aktuellen wissenschaftlichen Publikationssystem \cite{Jospeph_2006}, die Mängel in den wissenschaftlichen Anreizsystemen \cite{Osterloh_2008}, der steigende Publikationsdruck, die finanzielle und ideelle Notlage von Bibliotheken \cite{Russell_2008} \cite{Sietmann_2007}, Herausforderungen bei der Wahrung der Freiheit und Unabhängigkeit von Wissenschaft und Forschung \cite{Goetting_2015}, fehlende Transparenz, der Anstieg an Wissenschaftsskandalen \cite{Brembs_2015} und die zunehmende Ökonomisierung des Universitätsbetriebs \cite{Bauer_2006} führen dabei zu der Frage, ob das wissenschaftliche Kommunikationssystem der theoretischen Aufgabe von Wissenschaft uneingeschränkt gerecht werden kann \cite{Schekman_2013} oder jemals in vollem Umfang gerecht werden konnte.

Mit der zunehmenden Verbreitung und Etablierung des Internets als Kanal für die wissenschaftliche Kommunikation, für Forschungsaktivitäten und den Austausch von Informationen wurden neue Hoffnungen für die Verbesserung der "fatalen und unhaltbaren Situation" \cite[:155]{Brembs_2015} sowie für die Öffnung des Wissenstransfers \cite{Schulze_2013} \cite{Albert_2006} und des wissenschaftlichen Kommunikationsprozesses geweckt \cite{Hanekop_2014} \cite{European_Commission_2006} \cite{Goodrum_2001} \cite{Lawrence_1999}. Diese Erwartungen umfassen unter anderem den Wunsch nach "unbeschränktem Zugang zur gesamten wissenschaftlichen Zeitschriftenliteratur" \cite{BOAI_2002}, nach mehr Transparenz im wissenschaftlichen Erkenntnisprozess \cite{European_Commission_2015a}, nach Möglichkeiten der Steigerung von Effizienz und Effektivität von Wissenschaft \cite{Partha_1994} und "dass die alten Zugangs- und Nutzungsbeschränkungen sukzessive ausgeräumt werden" \cite{BOAI_2002} können. Grundlage dafür ist die Annahme, dass die Folgen der technologischen Entwicklungen "zwangsläufig zu erheblichen Veränderungen im Wesen des wissenschaftlichen Publizierens führen und einen Wandel der bestehenden Systeme wissenschaftlicher Qualitätssicherung einleiten" \cite{Berliner_Erklaerung_2003} würden.

Im Zuge dieser technologischen Entwicklungen, politischer Forderungen und gesellschaftlicher Annahmen gab und gibt es auf der einen Seite ein großes Interesse an der offenen Kommunikation und Unterstützung für den Wunsch nach freiem Zugriff auf wissenschaftliche Informationen. Auf der anderen Seite hat die Medien- und Technikgeschichte gezeigt, dass es bei Einführung eines neuen Mediums mit größerer Reichweite immer wieder zu Irritationen \cite{Naeder_2010} und Irrelevanz- oder gar Verlustängsten kommt \cite{Hagner_2015}. So zeigten die ersten Erfahrungen des Experimentierens mit dem Internet als neuem Kommunikationskanal für den wissenschaftlichen Austausch schnell, dass es sehr viel schwieriger sein würde, das wissenschaftliche Kommunikationssystem zu öffnen, und dass die Hürden für einen Wandel des Systems größer sind, als ursprünglich angenommen \cite{Bjoerk_2004}.

Somit bestehen trotz der zunehmenden Digitalisierung wissenschaftlicher Kommunikationssysteme und -prozesse weiterhin umfangreiche Barrieren beim Zugang zu wissenschaftlichen Informationen sowie bei den Möglichkeiten der (Weiter-)Verwendung dieser Informationen. Nur sehr langsam führen die ersten Modifikationen im System Wissenschaft zu Effekten bei der Verfügbarkeit von Wissen für die Gesamtgesellschaft. Auch rund 25 Jahre nach den ersten elektronischen Verfahren zum offenen Austausch wissenschaftlicher Publikationen und 350 Jahre nach dem Erscheinen der ersten wissenschaftlichen Fachzeitschrift muss das "alte" System demnach noch immer als weitestgehend stabil bezeichnet werden \cite{Brembs_2015} \cite{Hanekop_2014} \cite{Warnke_2012} und eine Veränderung der Tradition der wissenschaftlichen Praxis im Sinne einer "wissenschaftlichen Revolution" \cite{Kuhn_2012} scheint bisher (noch) nicht absehbar. Die Gründe und Einflussfaktoren für diese Entwicklung in Wissenschaft und Forschung werden im Folgenden dargestellt, empirisch und experimentell überprüft sowie abschließend diskutiert und zusammengefasst.

\section{Relevanz des Themas}

Im Rahmen des postulierten Wandels stehen nicht nur die Wissenschaftler und Wissenschaftlerinnen, sondern auch das ganze Universitätssystem sowie andere Bildungseinrichtungen und wissenschaftliche Bibliotheken vor bedeutenden Herausforderungen \cite{Mueller_2010} \cite{Harter_2006} \cite{Guedon_2004} \cite{Osterloh_2008} \cite{Beverungen_2014}. Die wissenschaftliche Kommunikation hat sich dabei in den letzten Jahrhunderten nur marginal verändert. "Nach innen" (wissenschaftsintern) bietet sie zwar einen gewissen Grad an Offenheit, aber nach außen ist sie geschlossen \cite{Kelty_2004}.

Die Institution Universität sowie wissenschaftliche Einrichtungen laufen im Kontext dieser Entwicklungen Gefahr, ihre Bedeutung als Ort der Wissensproduktion und -evaluation (weiter) zu verlieren \cite[:343]{Kruecken_2001}. Denn spätestens mit der Privatisierung der Verarbeitung, Speicherung und Übertragung von Wissen seit der zweiten Hälfte des 20. Jahrhunderts hörten Universitäten auf, selber Bücher zu verlegen \cite{Jospeph_2006}. Darüber hinaus fordert die Wirtschaft (zunehmend) eine öffentliche Finanzierung der Wissensproduktion und erwartet gleichzeitig die privatwirtschaftliche Aneignung und Nutzung des produzierten Wissens \cite{Weingart_2001}. Dieses Prinzip der kollektiven Wissensproduktion, bei dem die Wirtschaft unentgeltlich an wissenschaftliche Informationen gelangt, wird vor allem von Verlagen für ihre Wertschöpfung genutzt. Neben dem entgeltlichen Vertrieb der wissenschaftlichen Informationen ermöglichen diese Verlage den Autoren und Autorinnen, durch den "Rückgriff auf informal konstituierte Reputationen" \cite[:237]{Luhmann_1970} als Gegenleistung die Chance auf Anerkennung von der wissenschaftlichen Community und Reputation im wissenschaftlichen System \cite{Bernius_2009}.

Im Zusammenhang mit dem Wandel sind als besondere Herausforderung für die Wissenschaftler und Wissenschaftlerinnen, und damit abgeleitet auch für die Institutionen, die Wahrung der Freiheit von Wissenschaft und Forschung bei möglichst uneingeschränkter Verbreitung wissenschaftlicher Erkenntnisse \cite{Hagner_2015} \cite{BBAW_2015} \cite{Buss_2001} auf der einen sowie die Forderung nach besseren (Selbst-)Steuerungs- und Leistungsprozessen \cite{Adler_2009} \cite{Gibbons_1994} auf der anderen Seite zu nennen. Ebenfalls von besonderer Bedeutung sind auch die Auswirkungen des Wandels auf das Kommunikations- und Reputationssystem der Wissenschaft.

Das in den letzten Jahrzehnten zu verzeichnende Auseinanderdriften der Interessen zwischen der privatwirtschaftlichen Nutzung wissenschaftlicher Erkenntnisse und der ursprünglichen Aufgabe von Wissenschaft, neues überprüfbares Wissen zu produzieren und zu verbreiten, führten zu einer wissenschaftlichen Publikations- und Kommunikationskrise. Sie ist durch den wachsenden Kostendruck, Preissteigerungen \cite{Lewis_2015}, Publikations- \cite{Egger_1997} \cite{Fanelli_2012} \cite{Beverungen_2012} \cite{Brembs_20013} und Reportbias \cite{Chan_2008} \cite{Dickersin_2011}, Cargo Cult Science \cite{Feynman_1974} und die Einschränkung des Zugriffs auf wissenschaftliche Informationen \cite{Hess_2006} gekennzeichnet. Das aktuelle System mit den genannten Problemen steht dem Bestreben, dass es der Wissenschaft im Kern um Erkenntnisse und die uneingeschränkte Zurverfügungstellung dieser geht \cite{Hanekop_2006}, entgegen \cite{Offhaus_2012}.

Infolgedessen entstand unter den Wissenschaftlern und Wissenschaftlerinnen auch die Befürchtung, dass es durch Publikationsdruck und den Druck anwendungsorientierter zu forschen, wahrscheinlicher wird, dass viele der veröffentlichten Forschungsergebnisse eher falsch sind \cite{Ioannidis_2005}. Die genannten Entwicklungen befördern die Geschlossenheit des wissenschaftlichen Kommunikationssystems, erschweren nachhaltig den Zugang zu Wissen, beeinträchtigen die Erstellung von neuem Wissen \cite{Willinsky_2006} \cite{Feyerabend_1986} \cite{Luhmann_1998} und führen zu einem zunehmend unhaltbaren Zustand bei der wissenschaftlichen Kommunikation \cite{Schekman_2013}.

Sucht man nach Gründen für die Beibehaltung des bisherigen Modells durch die Wissenschaftsgemeinschaft, wird deutlich, dass vor allem Unwissen über die wirtschaftlichen Entwicklungen, rechtliche Bedenken und das etablierte wissenschaftliche Reputationssystem zentrale extrinsische Motivationsfaktoren für die Unterstützung des bisherigen Systems durch die wissenschaftliche Gemeinschaft darstellen \cite{Herb_2015}. Als weiterer Grund wird die komfortable Situation der Wissenschaftler und Wissenschaftlerinnen in dem System genannt, bei der diese selten auf den Zugang zu wissenschaftlichen Publikationen verzichten müssen, von der Auseinandersetzung um diemit den finanziellen Aspekten wissenschaftlicher Kommunikation weitestgehend befreit sind \cite{Sietmann_2007} \cite{Hanekop_2006} und den Wissenschaftlern und Wissenschaftlerinnen davon abgeraten wird, die vorherrschenden Paradigmen der wissenschaftlichen Praxis zu hinterfragen \cite{Siegfried_2013} \cite{Loeb_2013}. Dennoch tragen die Verschärfung der Krise und die langsam spürbaren Auswirkungen auf die wissenschaftliche Gemeinschaft dazu bei, dass die Forderung nach Veränderung des Systems zunehmende Unterstützung erfährt.

Die Suche nach einem Ausweg aus dieser Kommunikations- und Publikationskrise führte zu der anhaltenden Forderung nach der besseren öffentlichen Verfügbarkeit von Ergebnissen wissenschaftlicher Forschung und Arbeit und nach Alternativen für das geschlossene wissenschaftliche Publikations- und Kommunikationssystem. Ergänzend zu den erstmals artikulierten Forderungen nach der Öffnung dieser geschlossenen Form der Kommunikation in Wissenschaft und Forschung befinden wir uns infolge der neuen Möglichkeiten durch die Digitalisierung und Globalisierung inmitten eines "radikalen Wandels" \cite{Poynder_2011} tradierter wissenschaftlicher Kommunikationssysteme. Dieser Wandel bietet nicht nur die Chance für die Lösung der Herausforderungen im aktuellen wissenschaftlichen Kommunikationssystem, sondern ermöglicht auch eine umfassende "Beschleunigung des Wissensumschlages" \cite[:540]{Giesecke_1991} und führt potenziell dazu, dass offene Innovation und offene wissenschaftliche Kommunikation den privaten und staatlichen Forschungsbereich effizienter machen \cite{Chesbrough_2006} sowie den gesamtgesellschaftlichen Fortschritt in bisher unbekannter Weise beschleunigen \cite{Chesbrough_2003}.

Ungeachtet dieser Entwicklungen ist unübersehbar, dass das System der wissenschaftlichen Kommunikation noch immer "weitgehend stabil" \cite[:2]{Hanekop_2014} geblieben ist und im aktuellen Steuerungssystem der Wissenschaft weiterhin anhand der tradierten wissenschaftlichen Bewertungssysteme Reputation, Mittel und Stellen verteilt werden \cite{hollricher_wandel_2009} \cite{de_Vries_2001}. Die analog gedruckten und bewährten Journale sowie andere Publikationsformen der großen wissenschaftlichen Verlage werden bisher einfach nur mit nahezu unverändertem Geschäftsmodell zusätzlich digital verbreitet \cite{Hanekop_2014} \cite{BOAI_2012}.

Trotz umfangreicher Literatur liegen bisher nur wenige Untersuchungen und Experimente zur Öffnung wissenschaftlicher Kommunikation, vor allem aus den Geisteswissenschaften \cite{Naeder_2010}, vor. Daraus ergibt sich die Relevanz und Notwendigkeit, die bisherigen Entwicklungen im Bereich der Öffnung wissenschaftlicher Kommunikation aus geistes- und kulturwissenschaftlicher Perspektive genauer zu untersuchen, den Erkenntnissen über die Öffnung wissenschaftlicher Kommunikation aus der Literatur empirisch erhobenen Daten gegenüberzustellen, die Erkenntnisse praktisch-experimentell zu überprüfen und das Ergebnis zu diskutieren und einen Ausblick für die weitere Entwicklung zu wagen.

\section{Zielsetzung der Arbeit}

In dieser Arbeit werden die möglichen Auswirkungen der Digitalisierung und der Forderung nach Öffnung der Wissenschaft beziehungsweise der wissenschaftlichen Kommunikation auf Universitäten, wissenschaftliche Einrichtungen, aber auch auf den einzelnen Wissenschaftler und die einzelne Wissenschaftlerin genauer untersucht. Von besonderem Interesse sind in diesem Zusammenhang die Unterschiede zwischen dem reinen Zugang zu publiziertem Wissen auf der einen und dem kompletten Zugriff auf den gesamten wissenschaftlichen Erkenntnisprozess auf der anderen Seite sowie das Zusammenspiel unterschiedlicher Formen der Wissensverbreitung vor dem Hintergrund der geschichtlichen Entwicklung.

Es wird dabei betrachtet, inwieweit es sich bei der Öffnung von Wissenschaft im Rahmen von Open Access und Open Science tatsächlich um einen grundlegenden Wandel in der wissenschaftlichen Kommunikation handelt. Im weiteren Fokus der Untersuchung stehen der Umstand, dass die meisten wissenschaftlichen Informationen der Allgemeinheit bisher nicht zugänglich sind und welche Herausforderungen, das Wissen frei(er) zugänglich zu machen, daraus resultieren sowie welche Konsequenzen für das Wissenschaftssystem daraus zu erwarten sind.

Das Thema wird auch in einen historischen Kontext gestellt und es werden Argumente für und wider die Öffnung wissenschaftlicher Kommunikation aus Sicht der am wissenschaftlichen Kommunikationssystem beteiligten Akteure durch eine Befragung erhoben, um Gründe für die bisherigen Entwicklungen zu erarbeiten und zu einem vertieften Verständnis der unterschiedlichen Definitionen von Open Access und Open Science im Kontext wissenschaftlicher Kommunikation zu gelangen. Die wissenschaftliche, disziplinübergreifende Debatte um die Öffnung von Wissenschaft und Forschung wird dabei dargestellt, auf den deutschsprachigen Raum begrenzt sowie Katalysatoren und Hindernisse für die Etablierung der Forderung nach Öffnung bei den wissenschaftlichen Akteuren identifiziert und abgefragt.

Ziel ist eine aktuelle Verhandlung der theoretischen Annahmen und unterschiedlichen Definitionsversuche rund um die Etablierung und Praktizierung offener wissenschaftlicher Kommunikation mit den praktischen Gegebenheiten im wissenschaftlichen Alltag. In diesem Zusammenhang wird insbesondere die Diskrepanz zwischen der Idee der Öffnung von wissenschaftlicher Kommunikation und der wissenschaftlichen Realität \cite{Scheliga_2014} erörtert sowie die Gründe für die schleppende Umsetzung der Konzepte rund um die Öffnung von Wissenschaft erarbeitet. Die Erfahrungen und Meinungen der Wissenschaftler und Wissenschaftlerinnen werden dabei den Erfahrungen aus einem Selbstversuch gegenübergestellt und abschließend Handlungsempfehlungen für das offene Bearbeiten wissenschaftlicher Fragestellungen gegeben.

Die forschungsleitende Hypothese dieser Arbeit ist, dass sich die Öffnung des Zugangs zu wissenschaftlichen Erkenntnissen für die Gesamtgesellschaft (Open Access) in einer andauernden Übergangsphase zur Öffnung des Zugriffs auf den gesamten wissenschaftlichen Erkenntnisprozess (Open Science) befindet. Die sich daraus ableitenden Fragestellungen umfassen zum einen die theoretische Bedeutung und Historie von Offenheit im Rahmen der wissenschaftlichen Kommunikation, zum anderen die empirische Frage nach den Motiven und Beweggründen für Wissenschaftler und Wissenschaftlerinnen der unterschiedlichen Disziplinen, entweder das aktuelle wissenschaftliche Kommunikationssystem oder die Forderung hin zu Offenheit zu unterstützen. Anschließend wird durch die Dokumentation des offenen Verfassens dieser Arbeit in einem Selbstexperiment erarbeitet, welche Hürden, Grenzen und welcher Aufwand durch die Öffnung der formellen Kommunikation für Wissenschaftler und Wissenschaftlerinnen tatsächlich entstehen. "Offenes Verfassen" bedeutet in diesem Zusammenhang, dass diese Arbeit direkt und unmittelbar bei der Erstellung in den Jahren 2013, 2014 und 2015 für jede Person, jederzeit frei zugänglich im Internet unter einer offenen und freien Lizenz (CC-BY-SA) veröffentlicht wurde und die Arbeit jederzeit öffentlich nachvollziehbar ist.

Es wird erörtert, welche möglichen Auswirkungen durch diesen Prozess der Öffnung auf das Selbstverständnis der Wissenschaft und auf die wissenschaftliche Reputation in den unterschiedlichen Disziplinen zu erwarten sind. Dafür werden relevante Wege des Wissenstransfers ermittelt, Probleme und Hemmnisse bei der offenen Durchführung von wissenschaftlicher Arbeit herausgearbeitet und Handlungsmöglichkeiten am Beispiel der Erstellung einer Dissertation erschlossen.

\section{Aufbau der Arbeit}

Die Arbeit ist in acht Kapitel unterteilt. Nach der Einführung in die Thematik der Arbeit, in die Relevanz des Themas sowie die Beweggründe und Positionen des Autors werden im Kapitel "Grundlagen" die Chronologie, Begriffsbestimmungen und Debatten des Themenbereiches genauer betrachtet. Es wird zunächst die Entwicklung wissenschaftlicher Kommunikation chronologisch dargestellt, auf die Forderung der Öffnung wissenschaftlicher Kommunikation und auf Veränderungen durch die  Digitalisierung eingegangen und diese in den Kontext der wissenschaftlichen Reputation, des wissenschaftlichen Ethos und Diskurses gestellt.

Im dritten Kapitel werden Vorüberlegungen zur Methodenwahl angestellt und die Forschungsfragen ausformuliert sowie der zur Beantwortung der Forschungsfragen angewandte Methodenmix beschrieben, begründet und kritisch betrachtet.

Die anhand der Debatten in der Literatur ausgearbeiteten Herausforderungen werden im Kapitel "Analyse der Herausforderungen in der wissenschaftlichen Kommunikation" dargestellt sowie Anknüpfungspunkte für die empirische Untersuchung als Zwischenergebnis abgeleitet.

Im darauffolgenden Kapitel wird eine empirische Untersuchung zur Prüfung der identifizierten Katalysatoren und Hindernisse für die Öffnung von Wissenschaft und Forschung, die mittels einer Online-Befragung im Rahmen dieser Arbeit durchgeführt wurde, dokumentiert und ausgewertet. Dabei wird auf die Defizite und die aktuellen Debatten um die Begriffe "Open Access" und "Open Science" auf Grundlage der Erkenntnisse aus den vorhergehenden Kapiteln zurückgegriffen.

Der offene Erstellungsprozess im Sinne eines prospektiven Realexperimentes und die Dokumentation des Experiments im siebten Kapitel erweitern den empirischen Ansatz der Befragung im sechsten Kapitel um praktisch gewonne Erkenntnisse. Diese Herangehensweise ermöglicht es, einen primär verstehenden Zugang zu den Forschungsfragen und den Zielen der Arbeit zu erhalten und diesen in Form einer Selbstbeobachtung zu dokumentieren. Im Ergebnis werden Vorteile und Nachteile der offenen Anfertigung der Arbeit dargestellt, die Praxistauglichkeit überprüft, der Aufwand dokumentiert und Handlungsempfehlungen für das offene Verfassen wissenschaftlicher (Qualifikations-)Arbeiten abgeleitet.

In den letzten beiden Kapiteln werden die gewonnenen Ergebnisse und die Vorgehensweise sowie die Fragestellungen dieser Arbeit kritisch diskutiert sowie abschließend zusammengefasst. Auf Grundlage der Forschungsergebnisse und der eigenen Erfahrungen werden Empfehlungen zum Schreiben offener wissenschaftlicher Arbeiten sowie ergänzend ein Ausblick auf die weitere Entwicklung offener Strukturen im Rahmen wissenschaftlicher Tätigkeit formuliert.

Alle Teile der Arbeit folgen der forschungsleitenden Hypothese, dass sich Open Access in einer Übergangsphase befindet, die derzeit noch überwiegend durch die reine offene Bereitstellung wissenschaftlicher Publikationen geprägt wird, langfristig aber zur Öffnung weiterer Teile der wissenschaftlichen Kommunikation als wesentliche Grundlage für den Wissenszuwachs in der Gesamtgesellschaft (Open Science) führen wird. Ausgehend von den Fragestellungen wird dazu ein interdisziplinärer Zugang zur wissenschaftlichen Bearbeitung gewählt, der translational von den Kulturwissenschaften über die Politikwissenschaften und die Wirtschaftswissenschaften bis hin zu den Medienwissenschaften reicht, an die Wissenschafts- und Technikforschung angelehnt ist und sich empirischer, analytischer sowie auch experimenteller Methoden bedient.

\section{Beweggründe und eigene Position}

Die Beweggründe für die Erstellung der vorliegenden Arbeit sind die Folge der langjährigen Beschäftigung des Autors mit dem Konzept von "Offenheit" als wissenschaftlicher Mitarbeiter am Hybrid Publishing Lab der Leuphana Universität und als Vereinsvorstand der Open Knowledge Foundation Deutschland. Die hier angestrebte Auseinandersetzung mit den Konzepten rund um Offenheit im wissenschaftlichen Kommunikationssystem zielt somit auch auf die kritische Auseinandersetzung und das Hinterfragen der eigenen Positionen.

Die eigene Position zum Beginn des Erstellungsprozesses der Arbeit muss als klar befürwortend gegenüber den Forderungen nach Öffnung des wissenschaftlichen Kommunikations- und Erkenntnisprozesses bezeichnet werden. Sie fußt auf den Erfahrungen des beruflichen und ehrenamtlichen Engagements in der Förderung, Forderung und Ausgestaltung offener und transparenter Kommunikation in den gesellschaftlichen Teilbereichen Wissenschaft, Politik und Verwaltung. Im Rahmen der Auseinandersetzung mit dem Konzept von Offenheit bildet auch die differenzierte Betrachtung der eigenen Auffassung einen Schwerpunkt dieser Arbeit und wird im Zusammenhang mit den jeweiligen Betrachtungen immer wieder eine Rolle spielen.

Ziel dieser Auseinandersetzung ist es, die anhaltenden Forderungen nach Öffnung wissenschaftlicher Kommunikation unter dem Einsatz von Technik als soziales und kulturelles Phänomen besser zu verstehen und diese Entwicklung einer differenzierten und kritischen Analyse zu unterziehen. Die Wandlungsprozesse des wissenschaftlichen Kommunikationssystems sollen auf soziotechnische Grundlagen hin untersucht werden, trotzdem soll eine kritische Distanz gewahrt bleiben. In dieser Arbeit wird abschließend auch die eigene Position und die Praxistauglichkeit der Forderungen an das wissenschaftliche Kommunikationssystem im Rahmen der offenen Anfertigung dieser Arbeit geprüft und kritisch hinterfragt.

Die Auseinandersetzung mit den Fragestellungen folgt dem Ansatz der Science and Technology Studies (STS) beziehungsweise der Wissenschafts- und Technikforschung. Sie bezeichnet ein transdisziplinäres Forschungsfeld, das Ende der späten 1970er Jahre angetreten ist, um "Wissenschaft und Politik neu zu denken" \cite[:92]{Potthast_2010}. Die "empirische Untersuchung der vielfältigen Rollen von Wissen und Technologie in modernen Gesellschaften" \cite[:11]{Beck_2014} ist dabei vorrangiges Ziel der STS. Diese Herangehensweise beschäftigt sich mit den soziotechnischen Entwicklungen, den sozialen, kulturellen und politischen Dynamiken, die Wissenschaft und Technik formen, sowie der Frage, wie diese Dynamiken zukünftig Gesellschaft, Politik und Kultur beeinflussen \cite{Potthast_2010} \cite{Brown_2014}.

Dieser Forschungszusammenhang ermöglicht es, das Forschungsthema aus unterschiedlichen Blickwinkeln zu betrachten und die Betrachtungen unterschiedlichen Disziplinen zuzuordnen \cite{Beck_2014} \cite{Potthast_2010}. Wissenschaft wird demnach nicht mehr nur "als Ergebnis rein intellektueller kontemplativer Tätigkeit, sondern als Ergebnis praktischen Tuns und sozialen Handelns" \cite[:13]{Beck_2014} verstanden. In den STS sehen sich die Wissenschaftler und Wissenschaftlerinnen nicht als Entdecker, sondern als aktive Teilnehmer und Teilnehmerinnen des Dialogs um die Entwicklung \cite{MacKenzie_1999}. Sie streben danach, die empirischen Realitäten von Technologie und Medien zu verstehen \cite{Kelty_2014}, ohne immer zwangsläufig auf bereits bestehende Konzepte und Theorien zurückzugreifen \cite[:8]{Brown_2014} und folgen den Akteuren, anstatt Urteile im Voraus abzugeben \cite[:584]{Irwin_2008}.

Es ist unter anderem die Aufgabe von STS, die "Verschränkung von Wissenschaft, Technologie und Gesellschaft im Alltag zu untersuchen und damit auch die Rolle von Wissen und Technologie in gesellschaftlichen Ordnungsprozessen näher zu bestimmen" \cite[:9]{Beck_2014}. Die STS haben sich bei der Erforschung von Wissenschafts- und Technologiekultur zu einer etablierten Herangehensweise entwickelt, um darzustellen, inwiefern Technologien politisch sind \cite{Kelty_2014}.

Dieser Ansatz scheint gut geeignet, um die Beobachtungen im Rahmen dieser Forschungsarbeit wissenschaftlich zugänglich zu machen. Die Einordnung basiert unter anderem auf der Annahme, dass die Öffnung wissenschaftlicher Kommunikation in engem Zusammenhang mit der technologischen und politischen Entwicklung steht \cite{Weingart_2005}. Dieser Zusammenhang wird in der vorliegenden Arbeit empirisch im Alltag durch eine Befragung der wissenschaftlichen Akteure und das eigene offene wissenschaftliche Kommunizieren bei der Erstellung erforscht. Die Herangehensweise ermöglicht eine als Realexperiment konzipierte ethnographische Untersuchung des eigenen Forschungsalltags, bei der die teilnehmende Beobachtung des offenen Verfassens der Arbeit sowie die Online-Befragung die Grundlage für den Methodenmix einer ethnographischen Betrachtung darstellen \cite{Bachmann_2011}. Dieser Mix ermöglicht es, die Auswirkungen auf die Kommunikation von Wissenschaftlern und Wissenschaftlerinnen möglichst vollumfänglich zu beschreiben.

Durch die Betrachtung der Rolle von offenem Wissen und Technologie für wissenschaftliche Kommunikationsprozesse sowie durch die empirische Erforschung der Wissensproduktion und -verbreitung, seinen epistemologischen Voraussetzungen und den daraus resultierenden Konsequenzen \cite[:12]{Beck_2014} soll mit dieser Arbeit ein Beitrag zum Fortschritt für die Wissenschafts- und Technikforschung geliefert werden. Auch wenn die Kombination von akademischer Arbeit mit Aktivismus nicht einfach ist, da in der akademischen Welt ein gewisser Druck herrscht, die Arbeit von sozialem Engagement zu trennen, kann sie lohnend für die Sache sein \cite[:25]{Flood_2013}.
