\chapter{Einleitung} 

\begin{quote}
scientia donum dei est unde vendi non potest
\end{quote}

\section{Relevanz des Themas} 

Unbeschränkte und offene Kommunikation, sowie die Kenntnis des gegenwärtigen Wissensstandes, ist für wissenschaftliche Forschung, deren Aufgabe neues Wissen zu produzieren und dem gesellschaftlichen Auftrag des Wissenschaftssystems gerecht zu werden, unverzichtbar \cite{Hanekop_2014} \cite{glaeser2006} \cite{gibbons_1994} \cite{Luhmann1998}. Der aktuelle Zustand des wissenschaftlichen Kommunikationssystems wird diesem Auftrag nicht gerecht \cite{Schekman_2013}. Mit der Etablierung des Internets als neuen Kanal für die Kommunikation und den Austausch von Informationen wurden große Erwartungen für völlig neuen Möglichkeiten des Wissenstransfers und der wissenschaftlichen Kommunikation geweckt \cite{Hanekop_2014} \cite{schulze_2013_open} \cite{albert_2006_open_implications} \cite{Goodrum_2001} \cite{Lawrence_1999}.

Im Rahmen dieses Wandels stehen aber auch das Universitätsystem, andere Bildungseinrichtungen und Bibliotheken vor bedeutenden Herausforderungen \cite{Harter2006} \cite{Gu_don_2004} \cite{osterloh2008anreize}. Eine der Herausforderungen ist die Wahrung der Freiheit von Wissenschaft und Forschung auf der einen, sowie die Forderung nach besseren Steuerungs- und Leistungsprozessen \cite{Adler_2009} \cite{gibbons_1994} auf der anderen Seite. Die “gestörten Gleichgewichte im wissenschaftlichen Publikationssystem” \cite{cite:0} und das "kaputten wissenschaftlichen Reputationssystem" \cite{suchen} sind in diesem Zusammenhang von besonderer Bedeutung. In Anbetracht dieser Entwicklung läuft die Institution Universtät Gefahr, ihre Bedeutung als exklusiver Ort der Wissensproduktion \cite{suchen} und -evulation \cite{suchen} weiter zu verlieren.

Mit der Privatisierung der Verarbeitung, Speicherung und Übertragung von Wissen hörten Universitäten auf, selber Bücher zu verlegen \cite{cite:0}. Die Wirtschaft forderte (zunehmend) eine öffentliche Finanzierung der Wissenschaft und der Wissensproduktion und die privat(-wirtschaftliche) Aneignung und Nutzung des produzierten Wissens wurde erwartet \cite{cite:2}. Dieses Prinzip der kollektiven Wissensproduktion, bei dem die Wirtschaft unentgeltlich an wissenschaftliche Informationen gelangt, wird auch von Verlagen für ihre Wertschöpfung genutzt. Neben dem entgeldlichen Vertrieb der wissenschaftlichen Informationen erbringen sie den Autoren als Gegenleistung (durch den "Rückgriff auf informal konstituierte Reputationen" \cite{luhmann_1970_selbststeuerung} bzw. deswissenschafltichen Reputationssystems \cite{suchen}) die Chance auf Anerkennung von der wissenschaftlichen Community und Reputation im wissenschaftlichen System \cite{cite:21a}. Dieses System steht allerdings dem Bestreben, dass es der Wissenschaft im Kern um Erkenntnisse und die uneingeschränkte Zurverfügungstellung der Erkenntnisse geht \cite{hanekop_2006}, entgegen \cite{offhaus_2012_institutionelle_repos}. 

Diese Entwicklungen führten zu einer wissenschaftlichen Publikations- und Kommunikationskrise \cite{suchen}, geprägt durch steigenden Kostendruck, Preissteigerungen, Publikations- \cite{Egger_1997} \cite{Fanelli_2012} und Reportbias \cite{Chan_2008} \cite{Dickersin_2011} , Cargo Cult Science \cite{Feynman_1974} und die Einschränkung des Zugriffs auf wissenschaftliche Informationen \cite{Hess_2006}. Darüber hinaus wächst die Besorgnis, dass es zunehmend wahrscheinlicher ist, dass die meisten veröffentlichten Forschungsergebnisse "eher falsch als richtig sind" \cite{Ioannidis_2005}. Da jede Beschränkung des Zugangs zu Wissen auch die Erstellung neuen Wissen behindert \cite{cite:5} \cite{cite:8} \cite{Luhmann1998} ist dieser "Schwebezustand unhaltbar" \cite{suchen}. 

Die Suche nach einem Ausweg aus der Publikations- und Kommunikationskrise zeichnet sich in den anhaltenden Forderungen nach der Öffnung von Wissen und in der Suche nach Alternativen für das geschlossene wissenschaftliche Publikations- und Kommunikationssystem ab. Die wissenschaftliche Kommunikation, die zwar "nach innen" (wissenschaftsintern) einen gewissen Grad an Offenheit bot, aber "nach außen geschlossen ist" \cite{kelty_2004}, hat sich in den letzten 400 Jahren nur marginal verändert. Nach der erstmals vor über 20 Jahren artikulierten Forderungen nach Öffnung dieser Kommunikation befinden wir uns derzeit inmitten eines "radikalen Wandels" \cite{poynder_2011_suber} dieser tradierten Kommunikationssysteme. Infolge der neuen Möglichkeiten im Rahmen der Digitalisierung und Globalisierung \cite{mcluhan_1963_gutenberg} eröffnet sich erstmal die Chance für eine umfassende "Beschleunigung des Wissensumschlages" \cite{Wenzel_2003}.

Ungeachtet dessen muss bisher konstatiert werden, dass derzeit trotz des Wandels "das etablierte Publikationssystem der Verlage auch nach zwei Jahrzehnten weitgehend stabil" \cite{Hanekop_2014} geblieben ist. Die analog gedruckten und bewährten Journale, sowie andere Publikationsformen der großen wissenschaftlichen Verlage werden mit nahezu unverändertem Geschäftsmodell digital verbreitet \cite{Hanekop_2014} \cite{boai_2012}. Daraus ergibt sich die Relevanz und Notwendigkeit, den Erkenntnissen über die Öffnung der wissenschaftlichen Kommunikation aus der Literatur empirische Daten gegenüberzustellen und diese genauer zu untersuchen.

\section{Zielsetzung der Arbeit und Fragestellung} 

Diese Arbeit wird untersuchen, welche Auswirkungen der digitale Wandel und die Forderung nach Öffnung der Wissenschaft, beziehungsweise der wissenschafltichen Kommunikation, auf Universitäten, wissenschaftliche Einrichtungen aber auch Wissenschaftler hat. Von besonderem Interesse in diesem Zusammenhang sind die Unterschiede von reinem Zugang zu Wissen auf der einen und dem kompletten Zugriff auf Wissenschaft auf der anderen Seite bzw. das Zusammenspiel unterschiedlicher Prozesse der Wissensverbreitung vor dem Hintergrund der geschichtlichen Entwicklung. Es wird betrachtet, ob es sich dabei tatsächlich um einen Paradigmenwechsel handelt. Die Herausforderungen der Netzkultur, das Wissen frei(er) zugänglich zu machen und der Umstand, dass die meisten wissenschaftlichen Informationen der Allgemeinheit bisher nicht zugänglich sind \cite{cite:6} stehen ebenfalls im Fokus der Untersuchung. Diese Thematik wird in Relation zu der Massifizierung und Neoliberalisierung der Universität gesetzt und in einen historischen Kontext gestellt. Die Arbeit strebt an, Argumente für und gegen die Öffnung wissenschaftlicher Kommunikation aus Sicht der am wissenschaftlichen Kommunikationssystem Beteiligten zu evaluieren.

Ziel ist es, zu einem vertieften Verständnis der Definitionen von Open Access und Open Science im Kontext wissenschaftlicher Reputation zu gelangen, die aktuelle wissenschaftliche Debatte darzustellen, Treiber und Bremser für die Öffnung von Wissenschaft und Forschung zu identifizieren sowie die Erfahrungen aus dem Selbstversuch als Handlungsempfehlungen zu kommunizieren. --- TODO mit GA abklären --- 

Die vorläufige forschungsleitende Hypothese in dieser Arbeit ist, dass Open Access sich in der Übergangsphase zu Open Science befindet. Die sich daraus ableitende Fragestellung umfasst zum einen die theoretische Bedeutung von Offenheit im Rahmen der wissenschaftliche Kommunikation, aber auch die empirische Frage nach den Motiven und Beweggründen für Wissenschaftler, diese Offenheit in den unterschiedlichen Disziplinen zu ermöglichen oder zu verhindern . Abschließend soll erörtert werden, welche mögliche Auswirkungen auf das Selbstverständnis von Wissenschaft, auf das wissenschaftliche Kapital sowie auf die unterschiedlichen Disziplinen durch diesen Prozess der Öffnung zu erwarten sind. Dafür werden die für die Wissenschaftler relevanten Wege des Wissenstransfers ermittelt, abgefragt und abgeschätzt. Außerdem sollten Probleme und Hemmnisse bei der offenen Durchfürhung von wissenschaftlicher Arbeit herausgearbeitet und Handlungsmöglichkeiten am Beispiel der Erstellung von Doktorarbeiten erschlossen werden.

\section{Aufbau der Arbeit} 

Die Arbeit ist in acht Abschnitte unterteilt. In der Einleitung wird eine Einführung in die Thematik der Arbeit vorgenommen und die Relevanz des Themas herausgestellt. Im Teil "Grundlagen, Definitionen und Abgrenzung" werden intial Open Access, Open Science und wissenschaftliche Reputation beschrieben und abgegrenzt. Die zur Beantwortung der Forschungsfragen angewandten Methoden und das Vorgehen werden im dritten Kapitel beschrieben. Im daruffolgende Kapitel "Literaturanalyse" werden die aktuellen Debatten um die Begriffe, sowie um die Thematik der Arbeit auf Grundlage der Methode der Inhaltsanalyse erarbeitet, Defizite identifiziert und darauf aufbauend Vorraussetzungen für die empirische Befragung geschaffen. Im nächsten Abschnitt folgt die empirische Untersuchung zur Identifikation von Treibern und Bremsern für die Öffnung von Wissenschaft und Foschung, sowie die Einbettung eines Selbstexperiments im Rahmen der offenen Erstellung dieser Arbeit. Alle Teile folgen der forschungsleitende Hypothese, dass sich Open Access in einer Übergangsphasen von der reinen offenen Bereitstellung wissenschaftlicher Publikationen zum vollendeten Kommunikationsakt des "Wissenszuwachs" \cite{Luhmann1998} an die Gesamtgesellschaft befindet. Ausgehend von den Fragestellungen wird dazu ein transdisziplinärer Zugang zur wissenschaftlichen Bearbeitung gewählt, der tranlational von den Kulturwissenschaften über die Wirtschaftswissenschaften bis hin zu den Medienwissenschaften reicht und sich empirischen, analytischen aber auch experimenteller Methoden bedient. In den letzten beiden Kapiteln wird der Ansatz diskutiert und die gewonnen Ergebnisse vorgestellt. Auf Grundlage der Forschungsergebnisse und eignener Erfahrungen werden Empfehlungen zum Schreiben offener Promotionen, sowie ergänzend ein Ausblick auf die weitere Entwicklung offener Strukturen im Rahmen wissenschaftlicher Tätigkeit gewagt.
