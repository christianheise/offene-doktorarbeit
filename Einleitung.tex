\chapter{Einleitung}

Unbeschränkte und offene Kommunikation, sowie die Kenntnis des gegenwärtigen Wissensstandes, ist für wissenschaftliche Forschung, deren Aufgabe neues, überprüfbares Wissen zu produzieren, Verzerrungen und Fehler zu beseitigen, möglichst genau die Umwelt zu beschreiben, erklärende Theorien und Hypothesen zu entwicklen und damit dem gesellschaftlichen Auftrag des Wissenschaftssystems gerecht zu werden, unverzichtbar \cite{bbaw_publizieren_2015} \cite{Luescher_2014} \cite{Hanekop_2014} \cite{graefen2007_wissenschaftliche_artikel} \cite{glaeser2006} \cite{Luhmann1998} \cite{gibbons_1994}. Offenheit ist somit ein wesentlicher Bestandteil der Ethik der Wissenschaft \cite{Peters_2014} \cite{resnik_2005_ethics}. Die "gestörten Gleichgewichte im wissenschaftlichen Publikationssystem" \cite{cite:0}, das kaputte wissenschaftliche Anreizsystem \cite{Brembs_2013_a}, der steigender Publikationsdruck und die finanzielle Notlage von Bibliotheken, sowie der steigende Kostendruck auf Universitäten führen allerdings zu der Frage, ob der aktuelle Zustand des wissenschaftlichen Kommunikationssystems dieser Aufgabe uneingeschränkt gerecht werden kann \cite{Schekman_2013}.

Mit der zunehmenden Verbreitung und Etablierung des Internets als Kanal für die wissenschaftliche Kommunikation, die Forschungsaktivitäten und den Austausch von Informationen wurden neue Hoffnungen an die Öffnung des Wissenstransfers \cite{schulze_2013_open} \cite{albert_2006_open_implications} und des wissenschaftlichen Kommunikationsprozesses geweckt \cite{Hanekop_2014} \cite{EuropeanCommission_sciencepub_2006} \cite{Goodrum_2001} \cite{Lawrence_1999}. Diese Erwartungen umfassen unter anderem den Wunsch nach einem "unbeschränkten Zugang zur gesamten wissenschaftlichen Zeitschriftenliteratur", die Möglichkeiten der Steigerung von Effizienz und Effektivität von Wissenschaft und "dass die alten Zugangs- und Nutzungsbeschränkungen sukzessive ausgeräumt werden" \cite{boai_2002} können. Grundlage dafür ist die Annahme, dass die Folgen der technologischen Entwicklungen "zwangsläufig zu erheblichen Veränderungen im Wesen des wissenschaftlichen Publizierens führen und einen Wandel der bestehenden Systeme wissenschaftlicher Qualitätssicherung einleiten" \cite{berliner_erklaerung_2003} würden.

Im Zuge dieser technologischen Entwicklungen, politischen Forderungen und gesellschaftlichen Annahmen gab und gibt es auf der einen Seite ein großes Interesse an der offenen Kommunikation und Unterstützung für den Wunsch nach freiem Zugriff auf wissenschaftliche Informationen, auf der anderen Seite ist es in der Medien- und Technikgeschichte ein bekanntes Verhalten, dass es bei Einführung eines neuen Mediums mit größerer Reichweite zu Irritationen \cite{naeder_2010_open} und Irrelevanz- oder gar Verlustängsten bezüglich der etablierten Medien kommt \cite{hagner_2015_sache_buches}. So zeigten die ersten Erfahrungen des Experimentierens mit dem Internet als neuen Kommunikationskanal für den wissenschaftlichen Austausch schnell, dass es sehr viel schwieriger ist das wissenschaftliche "System" zu ändern und die Hürden für einen Wandel größer sind, als ursprünglich angenommen \cite{bjork_2004_open}.

Trotz der zunehmenden Digitalisierung wissenschaftlicher Kommunikationssysteme und -prozesse, bestehen weiterhin umfangreiche Zugangs- und Nutzungsbeschränkungen zu Wissen und nur langsam führen die ersten Modifikationen im System Wissenschaft zu Konsequenzen bei der Verfügbarkeit von Wissen für die Gesamtgesellschaft. Demnach muss rund 25 Jahre nach den ersten elektronischen Verfahren zum offenen Austausch wissenschaftlicher Publikationen und 350 Jahre nach dem Erscheinen der ersten wissenschaftlichen Fachzeitschrift, das "alte" System noch immer als weitstgehend stabil bezeichnet werden \cite{Hanekop_2014}. Die Gründe für diese Entwicklung in Wissenschaft und Forschung werden im folgenden dargestellt, empirisch geprüft und abschließend bewertet.

\section{Relevanz des Themas}

Im Rahmen des postulierten Wandels und der genannten Entwicklungen in Bezug auf die wissenschaftliche Kommunikation stehen nicht nur die Wissenschaftlerinnen und Wissenschaftler, sondern auch das Universitätsystem, sowie andere Bildungseinrichtungen und Bibliotheken vor bedeutenden Herausforderungen \cite{muller_2010_open} \cite{Harter2006} \cite{Gu_don_2004} \cite{osterloh2008anreize} \cite{Beverungen_2014}. Die Anspruchsvollsten sind in diesem Zusammenhang die Wahrung der Freiheit von Wissenschaft und Forschung bei möglichst uneingeschränkter Verbreitung wissenschaftlicher Erkenntnisse \cite{hagner_2015_sache_buches} auf der einen, sowie die Forderung nach besseren (Selbst-)Steuerungs- und Leistungsprozessen \cite{Adler_2009} \cite{gibbons_1994} auf der anderen Seite. In diesem Zusammenhang von besonderer Bedeutung sind die Veränderungen im Kommunikations- und Reputationssystem.

Die Institution Universität und wissenschafliche Einrichtungen laufen im Kontext dieser Entwicklungen Gefahr, ihre Bedeutung als exklusiver Ort der Wissensproduktion und -evaluation (weiter) zu verlieren. Spätestens mit der Privatisierung der Verarbeitung, Speicherung und Übertragung von Wissen hörten Universitäten damit auf, selber Bücher zu verlegen \cite{cite:0} und die Wirtschaft forderte (zunehmend) eine öffentliche Finanzierung der Wissensproduktion und erwartete gleichzeitig die privat(-wirtschaftliche) Aneignung und Nutzung des produzierten Wissens \cite{cite:2}. Das Prinzip der kollektiven Wissensproduktion, bei dem die Wirtschaft unentgeltlich an wissenschaftliche Informationen gelangt, wird auch von Verlagen für ihre Wertschöpfung genutzt. Neben dem entgeldlichen Vertrieb der wissenschaftlichen Informationen ermöglichen sie den Autoren, durch den "Rückgriff auf informal konstituierte Reputationen" \cite{luhmann_1970_selbststeuerung} und die vorangegangene Privatisierung des wissenschafltichen Reputationssystems \cite{suchen}, als Gegenleistung die Chance auf Anerkennung von der wissenschaftlichen Community und Reputation im wissenschaftlichen System \cite{cite:21a}.

Das Auseinanderdriften der Interessen zwischen der privatwirtschaftlichen Nutzung wissenschaftlicher Erkenntnisse und der genannten Aufgabe von Wissenschaft führten zu einer wissenschaftlichen Publikations- und Kommunikationskrise. Sie ist durch den wachsenden Kostendruck, Preissteigerungen, Publikations- \cite{Egger_1997} \cite{Fanelli_2012} \cite{Beverungen_2012} \cite{Brembs_20013} und Reportbias \cite{Chan_2008} \cite{Dickersin_2011}, Cargo Cult Science \cite{Feynman_1974} und die Einschränkung des Zugriffs auf wissenschaftliche Informationen \cite{Hess_2006} gekennzeichnet. Das aktuelle System steht dem Bestreben, dass es der Wissenschaft im Kern um Erkenntnisse und die uneingeschränkte Zurverfügungstellung der Erkenntnisse geht \cite{hanekop_2006}, entgegen \cite{offhaus_2012_institutionelle_repos}.

In diesem Zusammenhang entstand unter den Wissenschaftlern und Wissenschaftlerinnen die Besorgnis, dass es durch den enormen Publikationsdruck zunehmend wahrscheinlicher wurde, dass viele der veröffentlichten Forschungsergebnisse "eher falsch als richtig sind" \cite{Ioannidis_2005} und da jegliche Beschränkung des Zugangs zu Wissen auch die Erstellung neuen Wissens behindert \cite{cite:5} \cite{cite:8} \cite{Luhmann1998}, entwickelte sich ein zunehmend unhaltbarer Zustand \cite{Schekman_2013}.

Die Suche nach einem Ausweg aus der Publikations- und Kommunikationskrise führte zu anhaltenden Forderungen nach der Öffnung von Wissen und nach Alternativen für das geschlossene wissenschaftliche Publikations- und Kommunikationssystem. Die wissenschaftliche Kommunikation, die zwar "nach innen" (wissenschaftsintern) einen gewissen Grad an Offenheit bietet, aber "nach außen geschlossen ist" \cite{kelty_2004}, hat sich in den letzten 400 Jahren nur marginal verändert.

Seit der erstmals artikulierten Forderungen nach Öffnung dieser geschlossenen Form der Kommunikation befinden wir uns infolge der neuen Möglichkeiten durch die Digitalisierung und Globalisierung \cite{mcluhan_1962_gutenberg} inmitten eines "radikalen Wandels" \cite{poynder_2011_suber} tradierter Kommunikationssysteme. Dieser Wandel bietet die Chance für eine umfassende "Beschleunigung des Wissensumschlages" \cite{Wenzel_2003} und führt potenziell dazu, dass offene Innovation und offene wissenschaftliche Kommunikation den privaten und staatlichen Forschungsbereich effizienter machen \cite{chesbrough_2006_open}, sowie den gesamtgesellschaftlichen Fortschritt in bisher unbekannter Weise beschleunigen \cite{cite:7}.

Ungeachtet dessen ist unübersehbar, dass das System der wissenschaftlichen Kommunikation "weitgehend stabil" \cite{Hanekop_2014} geblieben ist und im aktuellen Steuerungssystem der Wissenschaft weiterhin anhand der tradierten wissenschaftlichen Bewertungssysteme Reputation, Mittel und Stellen verteilt werden \cite{cite:4}. Die analog gedruckten und bewährten Journale, sowie andere Publikationsformen der großen wissenschaftlichen Verlage werden mit nahezu unverändertem Geschäftsmodell nur zusätzlich digital verbreitet \cite{Hanekop_2014} \cite{boai_2012} und in der Wissenschaft rezipiert \cite{suchen}.

Trotz umfangreicher Literatur liegen bisher nur wenige Untersuchungen zur Öffnung wissenschaftlicher Kommunikation aus den Geisteswissenschaften vor \cite{naeder_2010_open}. Daraus ergibt sich die Relevanz und Notwendigkeit, die bisherigen Entwicklungen im Bereich der Öffnung wissenschaftlicher Kommunikation aus geistes- und kulturwissenschaftlicher Perspektive genauer zu untersuchen, den Erkenntnissen über die Öffnung wissenschaftlicher Kommunikation aus der Literatur empirisch erhobene Daten gegenüberzustellen und das Ergebnis zu diskutieren.

\section{Zielsetzung der Arbeit}

In dieser Arbeit wird untersucht, welche Auswirkungen der digitale Wandel und die Forderung nach Öffnung der Wissenschaft, beziehungsweise der wissenschaftlichen Kommunikation, auf Universitäten, wissenschaftliche Einrichtungen aber auch auf den einzelnen Wissenschaftler und die einzelne Wissenschaftlerin haben. Von besonderem Interesse sind in diesem Zusammenhang die Unterschiede von reinem Zugang zu publiziertem Wissen auf der einen und dem kompletten Zugriff auf den wissenschaftlichen Prozess auf der anderen Seite, sowie das Zusammenspiel unterschiedlicher Formen der Wissensverbreitung vor dem Hintergrund der geschichtlichen Entwicklung. Es wird betrachtet, ob es sich bei der Öffnung von Wissenschaft im Rahmen von Open Access und Open Science tatsächlich um einen grundlegenden Wandel in der wissenschaftlichen Kommunikation handelt. Im weiteren Fokus der Untersuchung stehen der Umstand, dass die meisten wissenschaftlichen Informationen der Allgemeinheit bisher nicht zugänglich sind und den daraus resultierenden Herausforderungen an die Netzkultur, das Wissen frei(er) zugänglich zu machen \cite{cite:6}.

Diese Thematik wird auch in Bezug zu der Massifizierung und Neoliberalisierung der Universität gesetzt und in einen historischen Kontext gestellt. Argumente für und gegen die Öffnung wissenschaftlicher Kommunikation aus Sicht der am wissenschaftlichen Kommunikationssystem beteiligten Akteure werden erhoben, um Gründe für die bisherigen Entwicklungen zu erarbeiten und zu einem vertieften Verständnis der Definitionen von Open Access und Open Science im Kontext wissenschaftlicher Kommunikation zu gelangen. Die aktuelle wissenschaftliche, disziplinübergreifende Debatte um die Öffnung von Wissenschaft und Forschung im deutschsprachigen Raum wird dabei dargestellt, Treiber und Bremser für diese Forderung nach Öffnung identifiziert sowie die Erfahrungen aus einem Selbstversuch als Handlungsempfehlungen für das offene Bearbeiten wissenschaftlicher Fragestellungen kommuniziert.

Ziel ist eine aktuelle Verhandlung der theoretischen Annahmen und Definitionen rund um die Etablierung und Praktizierung offener wissenschaftlicher Kommunikation mit den praktischen Gegebenheiten im wissenschaftlichen Alltag und den Erfahrungen und Meinungen der Wissenschaftler und Wissenschaftlerinnen.

Die forschungsleitende Hypothese dieser Arbeit ist, dass sich die Öffnung des Zugangs zu wissenschaftlichen Erkenntnissen (Open Access) in einer andauernden Übergangsphase zur Öffnung des Zugriffs auf den gesamten wissenschaftlichen Wertschöpfungsprozess (Open Science) befindet. Die sich daraus ableitenden Fragestellungen umfassen zum einen die theoretische Bedeutung und Historie von Offenheit im Rahmen der wissenschaftliche Kommunikation, zum anderen die empirische Frage nach den Motiven und Beweggründen für Wissenschaftler der unterschiedlichen Disziplinen, das aktuellen wissenschaftliche Kommunikationssystem bewahren zu wollen oder die Forderung hin zu Offenheit zu unterstützen. Weiterhin wird beispielhaft erarbeitet, welchen Hürden und welcher Aufwand durch die Öffnung der formellen Kommunikation für Wissenschaftler und Wissenschaftlerinnen entstehen.

Es wird erörtert, welche möglichen Auswirkungen durch diesen Prozess der Öffnung auf das Selbstverständnis der Wissenschaft an sich und auf die wissenschaftliche Reputation in den unterschiedlichen Disziplinen zu erwarten sind. Dafür werden relevante Wege des Wissenstransfers ermittelt, Probleme und Hemmnisse bei der offenen Durchführung von wissenschaftlicher Arbeit herausgearbeitet und Handlungsmöglichkeiten am Beispiel der Erstellung von Doktorarbeiten erschlossen und diskutiert.

\section{Aufbau der Arbeit}

Die Arbeit ist in acht Abschnitte unterteilt. In der Einleitung wird eine Einführung in die Thematik der Arbeit vorgenommen und die Relevanz des Themas sowie die Beweggründe und Positionen des Autors dargestellt.

Im Teil "Grundlagen, Definitionen und Abgrenzungen" werden initial die Termini Open Access, Open Science und wissenschaftliche Reputation beschrieben, abgegrenzt und historisch zugeordnet. Des weiteren wird auf die wissenschaftliche Kommunikation, ihre Rolle im wissenschaftlichen System sowie auf Veränderungen im Rahmen der Digitalisierung eingegangen.

Im dritten Kapitel werden Vorüberlegungen zur Methodenwahl dargestellt und die Forschungsfragen ausformuliert, sowie der zur Beantwortung der Forschungsfragen angewandte Methodenmix beschrieben, begründet und kritisch betrachtet. Die weitere Ausarbeitung der Treiber und Bremser für die Öffnung wissenschaftlicher Kommunikation erfolgt im vierten Kapitel anhand der Literaturrecherche und auf Grundlage der Erkenntnisse aus den einleitenden Kapiteln.

Im darauffolgenden Abschnitt wird eine empirische Untersuchung zur Prüfung der Treiber und Bremser für die Öffnung von Wissenschaft und Forschung, die mittels einer Online-Befragung im Rahmen dieser Arbeit durchgeführt wurde, dokumentiert und ausgewertet. Dabei werden die in den Grundlagen und in der Literaturrecherche identifizierten Defizite und die aktuellen Debatten um die Begriffe "Open Access" und "Open Science" bei der Befragung berücksichtigt.

Die offene Erstellung und Dokumentation der vorliegenden Arbeit im Sinne eines prospektiven Selbstexperimentes erweitern den empirischen Ansatz im darauffolgenden, sechsten Kapitel. Hier wird versucht einen primären verstehenden Zugang zu den Forschungsfragen und Zielen der Arbeit erhalten und diesen zu dokumentieren.

In den letzten beiden Kapiteln werden die gewonnen Ergebnisse und die Vorgehensweise in dieser Arbeit zusammengefasst und kritisch diskutiert. Auf Grundlage der Forschungsergebnisse und der eigenen Erfahrungen werden abschließend Empfehlungen zum Schreiben offener wissenschaftlicher Arbeiten, sowie ergänzend ein Ausblick auf die weitere Entwicklung offener Strukturen im Rahmen wissenschaftlicher Tätigkeit formuliert.

Alle Teile folgen der forschungsleitende Hypothese, dass sich Open Access in einer Übergangsphase von der reinen offenen Bereitstellung wissenschaftlicher Publikationen zum vollendeten Kommunikationsakt des "Wissenszuwachs" \cite{Luhmann1998} an die Gesamtgesellschaft befindet. Ausgehend von den Fragestellungen wird dazu ein transdisziplinärer Zugang zur wissenschaftlichen Bearbeitung gewählt, der translational von den Kulturwissenschaften über die Wirtschaftswissenschaften bis hin zu den Medienwissenschaften reicht und sich empirischer, analytischer, sowie auch experimenteller Methoden bedient.

\section{Beweggründe und eigene Position}

Die Beweggründe für die Erstellung der vorliegenden Arbeit ist die Folge der langjährigen Auseinandersetzung des Autors mit dem Konzept von Openness als wissenschaftlicher Mitarbeiter am Hybrid Publishing Lab der Leuphana Universität und als Vereinsvorstand der Open Knowledge Foundation Deutschland. Die Auseinandersetzung mit den Konzepten rund um Offenheit im wissenschaftlichen Kommunikationssystem zielt auch auf die kritische Auseinandersetzung und das Hinterfragen der eingenen Positionen.

Die eigene Position zum Beginn des Erstellungsprozess der Arbeit muss als klar befürwortend gegenüber den Forderungen nach Öffnung deklariert werden. Sie fußt auf den Erfahrungen des beruflichen und ehrenamtlichen Engagements in der Förderung, Forderung und Ausgestaltung offener und transparenter Kommunikation in den gesellschaftlichen Teilbereichen von Wissenschaft sowie in Politik und Verwaltung. Im Rahmen der Auslotung der Grenzen des Konzepts von Offenheit bei der wissenschaftlicher Kommunikation steht in dieser Arbeit auch eine differenzierte Betrachtung der eigenen Auffassung im Mittelpunkt der Auseinandersetzung und wird im Rahmen der jewiligen Betrachtungen immer wieder eine Rolle spielen.

Ziel dieser Auseinandersetzung ist es, die Forderung nach Öffnung wissenschaftliches Wissen und Kommunikation unter dem Einsatz von Technik als soziales und kulturelles Phänomen besser zu verstehen und diese Entwicklung einer differenzierten und kritischen Analyse zu unterziehen. Die Wandlungsprozesse des wissenschaftlichen Kommunikationssystems sollen auf soziotechnische Grundlagen hin untersucht und dabei eine kritische Distanz gewahrt werden. Dabei werden Methoden angewendet um die eigenen Position zu hinterfragen und diesen Prozess zu dokumentieren. In dieser Arbeit wird darüber hinaus auch die Praxistauglichkeit der Forderungen an das wissenschaftliche Kommunikationsystem geprüft und kritisch hinterfragt.

Die Auseinandersetzung mit den Fragestellungen folgt dem Ansatz der Science and Technology Studies (STS), beziehungsweise der Wissenschafts- und Technikforschung. Sie bezeichnet ein transdisziplinäres Forschungsfeld, das Ende der späten 1970er Jahre angetreten ist, um "Wissenschaft und Politik neu zu denken" \cite{Potthast_2010}. Die "empirische Untersuchung der vielfältigen Rollen von Wissen und Technologie in modernen Gesellschaften" \cite{beck_2014_science} ist dabei vorangiges Ziel der STS. Diese Herangehensweise beschäftigt sich mit den soziotechnischen Entwicklungen, den sozialen, kulturellen und politischen Dynamiken, die Wissenschaft und Technik formen, sowie der Frage, wie diese Dynamiken zukünftig Gesellschaft, Politk und Kultur beeinflussen \cite{Potthast_2010}.

Dieser Forschungszusammenhang oder dieses Forschungsfeld ermöglichen es, das Forschungsthema aus unterschiedlichen Blickwinkeln zu betrachten und die Betrachtungen unterschiedlichen Disziplinen zuzuordnen \cite{beck_2014_science}  \cite{Potthast_2010}. Wissenschaft wird demnach nicht mehr nur "als Ergebnis rein intellektueller kontemplativer Tätigkeit, sondern als Ergebnis praktischen Tuns und sozialen Handelns" \cite{beck_2014_science} verstanden. In den STS verstehen sich die Wissenschaftler und Wissenschaftlerinnen dabei nicht als Entdecker, sondern als aktive Teilnehmer des Dialogs um die Entwicklung \cite{MacKenzie_STS_1999} und streben danach die empirischen Realitäten von Technologie und Medien zu verstehen \cite{kelty_2014_freedom}.

Es ist unter anderem die Aufgabe von STS, die "Verschränkung von Wissenschaft, Technologie und Gesellschaft im Alltag zu untersuchen und damit auch die Rolle von Wissen und Technologie in gesellschaftlichen Ordnungsprozessen näher zu bestimmen" \cite{beck_2014_science}. STS hat sich bei der von Erforschung von Wissenschafts- und Technologiekultur zu einer etablierten Herangehensweise entwickelt um darzustellen, inwiefern Technologien politisch sind \cite{kelty_2014_freedom}.

Dieser Ansatz scheint demnach am Besten geeignet um die Beobachtungen im Rahmen dieser Forschungsarbeit wissenschaftlich zugänglich zu machen. Die Einordnung basiert unter anderem auf der Annahme, dass die Öffnung wissenschaftlicher Kommunikation im engen Zusammenhang mit der technologischen Entwicklung steht. Dieser Zusammenhang wird in vorliegenden Arbeit empirisch im Alltag durch eine Befragung und das eigene offene wissenschaftliche Kommunizieren bei der Erstellung beforscht. Die Herangehensweise ermöglicht eine als Laborstudie konzipierte etnographische Untersuchung des eigenen Forschungsalltags, bei der die teilnehmende Beobachtung sowie die Online-Befragung die Grundlage für den Methodenmix einer ethnograhischen Betrachtung darstellen \cite{bachmann_2011_ethnographie}, die eine Beschreibung der Auswirkungen auf die Kommunikation von Wissenschaftlern ermöglicht.

Durch die Betrachtung der Rolle von offenem Wissen und Technologie für wissenschaftliche Kommunikationsprozesse sowie durch die empirische Erforschung der Wissensproduktion und -verbreitung, seinen episemologischen Vorraussetzungen und den daraus resultierenden Konsequenzen \cite{beck_2014_science} soll mit dieser Arbeit ein Beitrag zum Fortschritt für die Wissenschafts- und Technikforschung geliefert werden.
