\chapter{Einleitung} 
\begin{quote}
scientia donum dei est unde vendi non potest
\end{quote}Unbeschränkte und offene Kommunikation sowie die Kenntnis des gegenwärtigen Wissensstandes ist für wissenschaftliche Forschung, deren Aufgabe neues Wissen zu produzieren und dem gesellschaftlichen Auftrag des Wissenschaftssystems gerecht zu werden unverzichtbar \cite{Hanekop_2014} \cite{glaeser2006} \cite{gibbons_1994} \cite{Luhmann1998}. Der aktuelle Zustand des wissenschaftlichen Kommunikationssystems wird diesem Auftrag nicht gerecht \cite{suchen}. Mit der Etablierung des Internets als neuen Kanal für die Kommunikation und den Austausch von Informationen wurden große Erwartungen für die völlig neuen Möglichkeiten des Wissenstransfers geweckt \cite{Hanekop_2014} \cite{schulze_2013_open} \cite{albert_2006_open_implications} \cite{Goodrum_2001} \cite{Lawrence_1999}. 

Im Rahmen des digitalen Wandels stehen das Universitätsystem, aber auch andere Bildungseinrichtungen und ihre Bibliotheken vor vielen großen Herausforderungen \cite{Harter2006} \cite{Gu_don_2004} \cite{osterloh2008anreize}. Eine Herausforderung ist die Wahrung der Freiheit von Wissenschaft und Forschung auf der einen,  sowie die Forderung nach besseren Steuerungs- und Leistungsprozessen \cite{Adler_2009} \cite{gibbons_1994} auf der anderen Seite. Von herausragender Bedeutung in diesem Zusammenhang stehen die “gestörten Gleichgewichte im wissenschaftlichen Publikationssystem” \cite{cite:0} und das "kaputten wissenschaftlichen Reputationssystem" \cite{suchen}. Friedrich Kittler hat diese Aspekte in einer Rede "Wissenschaft als Open-Source-Prozeß" im Jahr 1999 wie folgt zusammengefasst: "mit Freiheit von Quellcode steht und fällt auch die Freiheit der Wissenschaft". Mit anderen Worten, dass trotz Digitalisierung die "Verarbeitung des Wissens technisch reproduzierbar" \cite{cite:1} und transparent kontrollierbar \cite{suchen} bleiben muss. In Anbetracht dieser Entwicklung laufen insbesondere genuine Universitäten Gefahr, ihre Bedeutung als exklusiver Ort der Wissensproduktion \cite{suchen} und -evulation \cite{suchen} weiter zu verlieren.

Als Gründe für diese Entwicklung werden in der Literatur unter anderem folgende Aspekte genannt \cite{suchen}: Mit der Privatisierung der Verarbeitung, Speicherung und Übertragung von Wissen hörten Universitäten auf selber Bücher zu verlegen \cite{cite:0}. In einem weiteren Erklärungsversuch für diesen Trend stellt Peter Weingart fest, dass „die Wirtschaft (zunehmend) eine öffentliche Finanzierung der Wissenschaft und der Wissensproduktion". Gleichzeitig, so Weingard weiter, werden "im Endeffekt (...) die privat(-wirtschaftliche) Aneignung und Nutzung des produzierten Wissens erwartet“ \cite{cite:2}. Dieses Grundprinzip der uneigennützigen, kollektiven Wissensproduktion, um unentgeltlich an wissenschaftliche Informationen zu gelangen, wird von Verlagen genutzt. Neben dem entgeldlichen Vertrieb der wissenschaftlichen Informationen erbringen sie den Autoren als Gegenleistung (durch die Privatisierung des wissenschafltichen Reputationssystems \cite{suchen}) die Chance auf Anerkennung und Reputation \cite{cite:21a}. Das steht dem Grundsatz, dass es der Wissenschaft im Kern um Erkenntnisse und die uneingeschränkte Zurverfügungstellung der Erkenntnisse geht \cite{hanekop_2006}, diametral entgegen \cite{offhaus_2012_institutionelle_repos}. Dieser Widerspruch wird ebenfalls im aktuellen Steuerungssystem der Wissenschaft deutlich, in dem anhand der wissenschaftlichen Reputation Mittel und Stellen verteilt werden \cite{cite:4}.

Diese Entwicklungen mündeten letztendlich in eine wissenschaftlichen Publikations- und Kommunikationskrise \cite{suchen}, geprägt durch steigenden Kostendruck, Preissteigerungen und die Einschränkung des Zugriffs auf wissenschaftliche Informationen \cite{Hess_2006}. Dem gegenüber stehen die neuen Möglichkeiten der Digitalisierung und Globalisierung für die Wissensverbreitung. Da jede Beschränkung des Zugangs zu Wissen auch die Erstellung von neuem Wissen behindert \cite{cite:5} \cite{cite:8} ist dieser "Schwebezustand unhaltbar" \cite{suchen}. 

Die Suche nach einem Ausweg aus der Publikations- und Kommunikationskrise zeichnet sich in den anhaltenden Forderungen nach der Öffnung von Wissen und in der Suche nach Alternativen für das geschlossene wissenschaftliche Publikations- und Kommunikationssystem ab. Das folgt auch der Annahme, dass offene Innovation und offene wissenschaftliche Kommunikation den privaten und staatlichen Forschungsbereich effizienter machen sowie den industriellen Fortschritt beschleunigen \cite{cite:7}. 

Im Rahmen der erstmals vor über 20 Jahren artikulierten Forderungen nach Öffnung der in den letzten 350 Jahren nur marginal veränderten wissenschaftlichen Kommunikation \cite{poynder_2011_suber bzw suchen}, befinden wir uns derzeit inmitten eines "radikalen Wandels" \cite{poynder_2011_suber} dieser tradierten Kommunikationssysteme. Ungeachtet dessen muss festgehalten werden, dass "das etablierte Publikationssystem der Verlage auch nach zwei Jahrzehnten weitgehend stabil" \cite{Hanekop_2014} ist. Die analog gedruckten und bewährten Journale sowie andere Publikationsformen der großen wissenschaftlichen Verlage werden mit nahezu unverändertem Geschäftsmodell digital verbreitet \cite{Hanekop_2014} \cite{boai_2012}.

Diese Arbeit untersucht, welche Auswirkungen der digitale Wandel und die Forderung nach Öffnung der Wissenschaft beziehungsweise der wissenschafltichen Kommunikation auf Universitäten, wissenschaftliche Einrichtungen aber auch Wissenschaftler hat. Von besonderem Interesse in diesem Zusammenhang sind die Unterschiede von reinem Zugang zu Wissen auf der einen und dem kompletten Zugriff auf Wissenschaft auf der anderen Seite. Sowie die Frage, ob es sich dabei tatsächlich um einen Paradigmenwechsel handelt. Die Herausforderungen der Netzkultur, das Wissen frei(er) zugänglich zu machen und der Umstand, dass die meisten wissenschaftlichen Informationen der Allgemeinheit bisher nicht zugänglich sind \cite{cite:6} stehen ebenfalls im Fokus der Untersuchung. Diese Thematik wird in Relation zu der Massifizierung und Neoliberalisierung der Universität gesetzt und in einen historischen Kontext gestellt. Die Arbeit strebt eine ausgewogene Betrachtungsweise an, Argumente für und gegen die Öffnung wissenschaftlicher Kommunikation aus Sicht der am wissenschaftlichen Kommunikationssystem Beteiligten zu evaluieren.

\section{Aufbau der Arbeit} 

Die Arbeit ist in fünf Teile unterteilt. In der Einleitung wird eine Einführung in die Thematik der Arbeit vorgenommen, ein theoretischer Bezugsrahmen verdeutlicht und die Relevanz des Themas herausgestellt. Im Teil "Definition und Abgrenzung" sollen die Begrifflichkeiten Open Access, Open Science und wissenschaftliche Reputation anhand der vorliegenden Literatur erklärt, eingeordnet und zu anderen Begrifflichkeiten abgegrenzt werden. Das daruffolgende Kapitel "Forschungsstand" gibt die aktuellen Debatten um die Begrifflichkeiten sowie um die Thematik der Arbeit wieder. Dieser Teil folgt der forschungsleitende Hypothese, dass sich Open Access in einer Übergangsphasen von der reinen offenen Bereitstellung wissenschaftlicher Publikationen zur umfassenden und offenen Wissensverteilung an die Gesamtgesellschaft befindet. Die angewandte Methodik zur Beantwortung der Forschungsfragen wird im vierten Kapitel erläutert und angewandt. Ausgehend von den Fragestellungen wird dazu ein transdisziplinärer Zugang zur wissenschaftlichen Bearbeitung der Fragestellungen gewählt, der von den Kulturwissenschaften über die Wirtschaftswissenschaften bis hin zu den Medienwissenschaften reicht. In den letzten beiden Kapiteln werden die gewonnen Ergebnisse vorgestellt und diskutiert. Abschließend soll auf Grundlage der Forschungsergebnisse ein Ausblick auf die weitere Entwicklung von offenen Strukturen im Rahmen von wissenschaftlichen Publikationen aber auch darüber hinaus gewagt werden.

\section{Theoretischer Bezugsrahmen} 

Open Science, Open Access und wissenschaftliche Reputation werden in dieser Arbeit in technischen als auch in ihren gesellschaftlichen und politischen Aspekten sowie die kulturellen Auswirkungen der Medienbrüchen im Rahmen von wissenschaftlichen Publizieren auf theoretischem Niveau reflektiert. Die Analysen in dieser Arbeit werden aus den Perspektive des Produzenten (Wissenschaftler als Autoren) und auch aus der damit nicht immer harmonisierenden Perspektive des Rezipienten beziehungsweise Medienkonsumenten (Wissenschaftler als Leser) stattfinden. In diesem Zusammenhang wird adressiert inwiefern Macht, regulierende Prinzipien wie die Verknappung sowie die Ein- und Ausgrenzung im Rahmen von wissenschaftlichen Diskursen, nach dem Diskurs- und Machtbegriff von Michel Foucault, mit den Modellen der Open Access, Open Science und wissenschaftliche Reputation in der wissenschaftlichen Kommunikation vereinbar sind oder diesen diametral dem gegenüberstehen. Die theoretischen Vorannahmen stellen die Grundlage für die experimentellen und empirischen Betrachtungen dieser Arbeit dar und sollen an der wissenschaftlichen Praxis geprüft werden.

\section{Relevanz des Themas} 

