\chapter{Einleitung} 
\begin{quote}
scientia donum dei est unde vendi non potest
\end{quote}

\section{Relevanz des Themas} 

Unbeschränkte und offene Kommunikation sowie die Kenntnis des gegenwärtigen Wissensstandes ist für wissenschaftliche Forschung, deren Aufgabe neues Wissen zu produzieren und dem gesellschaftlichen Auftrag des Wissenschaftssystems gerecht zu werden unverzichtbar \cite{Hanekop_2014} \cite{glaeser2006} \cite{gibbons_1994} \cite{Luhmann1998}. Der aktuelle Zustand des wissenschaftlichen Kommunikationssystems wird diesem Auftrag nicht gerecht \cite{Schekman_2013}. Mit der Etablierung des Internets als neuen Kanal für die Kommunikation und den Austausch von Informationen wurden große Erwartungen für die völlig neuen Möglichkeiten des Wissenstransfers geweckt \cite{Hanekop_2014} \cite{schulze_2013_open} \cite{albert_2006_open_implications} \cite{Goodrum_2001} \cite{Lawrence_1999}, "die Grenzen der Wissenschaft verschwimmen" \cite{Scheliga_2014} und "plötzlich ist es nicht mehr einfach so möglich eine scharfe technische, institutionelle und juristische Grenze zwischen Wissenschaft und dem was Rest der Gesellschaft macht zu ziehen" \cite{kelty_2004}.

Im Rahmen des digitalen Wandels stehen das Universitätsystem, aber auch andere Bildungseinrichtungen und ihre Bibliotheken vor vielen großen Herausforderungen \cite{Harter2006} \cite{Gu_don_2004} \cite{osterloh2008anreize}. Eine Herausforderung ist die Wahrung der Freiheit von Wissenschaft und Forschung auf der einen,  sowie die Forderung nach besseren Steuerungs- und Leistungsprozessen \cite{Adler_2009} \cite{gibbons_1994} auf der anderen Seite. Von herausragender Bedeutung in diesem Zusammenhang stehen die “gestörten Gleichgewichte im wissenschaftlichen Publikationssystem” \cite{cite:0} und das "kaputten wissenschaftlichen Reputationssystem" \cite{suchen}. Friedrich Kittler hat diese Aspekte in einer Rede "Wissenschaft als Open-Source-Prozeß" im Jahr 1999 wie folgt zusammengefasst: "mit Freiheit von Quellcode steht und fällt auch die Freiheit der Wissenschaft". Mit anderen Worten, dass trotz Digitalisierung die "Verarbeitung des Wissens technisch reproduzierbar" \cite{cite:1} und transparent kontrollierbar \cite{suchen} bleiben muss. In Anbetracht dieser Entwicklung laufen insbesondere genuine Universitäten Gefahr, ihre Bedeutung als exklusiver Ort der Wissensproduktion \cite{suchen} und -evulation \cite{suchen} weiter zu verlieren.

Als Gründe für diese Entwicklung werden in der Literatur unter anderem folgende Aspekte genannt \cite{suchen}: Mit der Privatisierung der Verarbeitung, Speicherung und Übertragung von Wissen hörten Universitäten auf selber Bücher zu verlegen \cite{cite:0}. In einem weiteren Erklärungsversuch für diesen Trend stellt Peter Weingart fest, dass „die Wirtschaft (zunehmend) eine öffentliche Finanzierung der Wissenschaft und der Wissensproduktion". Gleichzeitig, so Weingard weiter, werden "im Endeffekt (...) die privat(-wirtschaftliche) Aneignung und Nutzung des produzierten Wissens erwartet“ \cite{cite:2}. Dieses Grundprinzip der uneigennützigen, kollektiven Wissensproduktion, um unentgeltlich an wissenschaftliche Informationen zu gelangen, wird von Verlagen genutzt. Neben dem entgeldlichen Vertrieb der wissenschaftlichen Informationen erbringen sie den Autoren als Gegenleistung (durch die Privatisierung des wissenschafltichen Reputationssystems \cite{suchen}) die Chance auf Anerkennung und Reputation \cite{cite:21a}. Das steht dem Grundsatz, dass es der Wissenschaft im Kern um Erkenntnisse und die uneingeschränkte Zurverfügungstellung der Erkenntnisse geht \cite{hanekop_2006}, diametral entgegen \cite{offhaus_2012_institutionelle_repos}. Dieser Widerspruch wird ebenfalls im aktuellen Steuerungssystem der Wissenschaft deutlich, in dem anhand der wissenschaftlichen Reputation Mittel und Stellen verteilt werden \cite{cite:4}.

Diese Entwicklungen mündeten letztendlich in eine wissenschaftlichen Publikations- und Kommunikationskrise \cite{suchen}, geprägt durch steigenden Kostendruck, Preissteigerungen, Publikations- \cite{Egger_1997} \cite{Fanelli_2012} und Reportbias \cite{Chan_2008} \cite{Dickersin_2011} , Cargo Cult Science \cite{Feynman_1974} und die Einschränkung des Zugriffs auf wissenschaftliche Informationen \cite{Hess_2006}. Darüber hinaus wächst die Besorgnis, dass es zunehmend wahrscheinlicher ist, dass die meisten veröffentlichten Forschungsergebnisse "eher falsch als richtig sind" \cite{Ioannidis_2005}. Dem gegenüber stehen die neuen Möglichkeiten der Digitalisierung und Globalisierung für die Wissensverbreitung. Da jede Beschränkung des Zugangs zu Wissen auch die Erstellung von neuem Wissen behindert \cite{cite:5} \cite{cite:8} ist dieser "Schwebezustand unhaltbar" \cite{suchen}. 

Die Suche nach einem Ausweg aus der Publikations- und Kommunikationskrise zeichnet sich in den anhaltenden Forderungen nach der Öffnung von Wissen und in der Suche nach Alternativen für das geschlossene wissenschaftliche Publikations- und Kommunikationssystem ab. Das folgt auch der Annahme, dass offene Innovation und offene wissenschaftliche Kommunikation den privaten und staatlichen Forschungsbereich effizienter machen sowie den industriellen Fortschritt beschleunigen \cite{cite:7}. 

Im Rahmen der erstmals vor über 20 Jahren artikulierten Forderungen nach Öffnung der in den letzten 400 Jahren nur marginal veränderten wissenschaftlichen Kommunikation, die zwar "nach innen" (wissenschaftsintern) einen gewissen Grad an Offenheit bot, aber "nach außen geschlossen ist" \cite{kelty_2004}, befinden wir uns derzeit inmitten eines "radikalen Wandels" \cite{poynder_2011_suber} dieser tradierten Kommunikationssysteme. Ungeachtet dessen muss festgehalten werden, dass "das etablierte Publikationssystem der Verlage auch nach zwei Jahrzehnten weitgehend stabil" \cite{Hanekop_2014} ist. Die analog gedruckten und bewährten Journale sowie andere Publikationsformen der großen wissenschaftlichen Verlage werden mit nahezu unverändertem Geschäftsmodell digital verbreitet \cite{Hanekop_2014} \cite{boai_2012}.

\section{Zielsetzung der Arbeit} 

Diese Arbeit untersucht, welche Auswirkungen der digitale Wandel und die Forderung nach Öffnung der Wissenschaft beziehungsweise der wissenschafltichen Kommunikation auf Universitäten, wissenschaftliche Einrichtungen aber auch Wissenschaftler hat. Von besonderem Interesse in diesem Zusammenhang sind die Unterschiede von reinem Zugang zu Wissen auf der einen und dem kompletten Zugriff auf Wissenschaft auf der anderen Seite. Sowie die Frage, ob es sich dabei tatsächlich um einen Paradigmenwechsel handelt. Die Herausforderungen der Netzkultur, das Wissen frei(er) zugänglich zu machen und der Umstand, dass die meisten wissenschaftlichen Informationen der Allgemeinheit bisher nicht zugänglich sind \cite{cite:6} stehen ebenfalls im Fokus der Untersuchung. Diese Thematik wird in Relation zu der Massifizierung und Neoliberalisierung der Universität gesetzt und in einen historischen Kontext gestellt. Die Arbeit strebt eine ausgewogene Betrachtungsweise an, Argumente für und gegen die Öffnung wissenschaftlicher Kommunikation aus Sicht der am wissenschaftlichen Kommunikationssystem Beteiligten zu evaluieren.

Ziel ist es, letztlich zu einem vertieften der Begrifflichkeiten Open Acces und Open Science im Kontext wissenschaftlicher Kommunikation, Reputation und Arbeit zu gelangen.

\section{Aufbau der Arbeit} 

Die Arbeit ist in sechs Teile unterteilt. In der Einleitung wird eine Einführung in die Thematik der Arbeit vorgenommen, ein theoretischer Bezugsrahmen verdeutlicht und die Relevanz des Themas herausgestellt. Im Teil "Theoretischer Rahmen, Definition und Abgrenzung" sollen die Begrifflichkeiten Open Access, Open Science und wissenschaftliche Reputation anhand der vorliegenden Literatur erklärt, operationalisiert und zu anderen Begrifflichkeiten abgegrenzt werden. Das daruffolgende Kapitel "Forschungsstand" gibt die aktuellen Debatten um die Begrifflichkeiten sowie um die Thematik der Arbeit wieder. Dieser Teil folgt der forschungsleitende Hypothese, dass sich Open Access in einer Übergangsphasen von der reinen offenen Bereitstellung wissenschaftlicher Publikationen zum vollendeten Kommunikationsakt des "Wissenszuwachs" \cite{Luhmann1998} an die Gesamtgesellschaft befindet. Die Methodik zur Beantwortung der Forschungsfragen wird im vierten Kapitel erläutert und angewandt. Ausgehend von den Fragestellungen wird dazu ein transdisziplinärer Zugang zur wissenschaftlichen Bearbeitung der Fragestellungen gewählt, der von den Kulturwissenschaften über die Wirtschaftswissenschaften bis hin zu den Medienwissenschaften reicht und sich empirischen, analytischen aber auch experimentellen Methoden bedient. In den letzten beiden Kapiteln werden die gewonnen Ergebnisse vorgestellt und diskutiert. Abschließend soll auf Grundlage der Forschungsergebnisse und eignene Erfahrungen Empfehlungen zum Schreiben von offenen Promotionen und ein Ausblick auf die weitere Entwicklung von offenen Strukturen im Rahmen von wissenschaftlichen Publikationen, aber auch darüber hinaus gewagt werden.

\section{Methodik} 
Die Verortung der Fragestellung dieser Arbeit, die von den Kulturwissenschaften über die Wirtschaftswissenschaften bis hin zu den Medienwissenschaften reicht, erfordert einen transdisziplinären Zugang zur wissenschaftlichen Bearbeitung. 
Drei wissenschaftliche Methodologien werden in dieser Arbeit angewandt: das Konzeptionelle/Theoretische im Rahmen der Literaturanalyse für die Begriffsbestimmung, das Ethnographische im Rahmen der Befragung zur Identifikation der Treiber und Bremser für die Öffnung von wissenschaftlicher Informationen und Prozesse sowie das Experimentelle. 
Die Herangehensweise folgt dabei der Auffassung des Medientheoretikers Geert Lovink, der diese dreifache Methodik für die Erforschung der digitalen Kultur für zwingend notwendig erachtet . 
Ziel ist es, letztlich zu einem vertieften theoretischen Verständnis der empirischen Ergebnisse zu gelangen. Im Rahmen der Arbeit am Inkubator bietet es sich hier an, weitere Hypothesen anhand von Experimenten zu gewinnen und diese mit neuen Geschäftsmodellen und politischen Prozessen forscherisch zu begleiten. So können mögliche Verallgemeinerungsmodelle im Rahmen der in Kapitel 3 definierten Fragestellungen theoretisch entwickelt und praktisch geprüft werden.
\subsection{Methode der Inhaltsanalyse}
Die unterschiedliche Verwendung der Begriffe Open Science und Open Access in der wissenschaftlichen Auseinandersetzung machen es notwendig die Begriffsbestimmungen für Open Science und Open Access im Rahmen dieser Arbeit vorzunehmen und zu konkretisieren. In Ergänzug zu der Literaturanalyse von Benedikt Fecher und Sascha Friesike für den Begriff "Open Science"\cite{cite:9} sowie der Litaraturanalyse von Giancarlo Frosio und Estelle Derclaye "Open Access Publishing" \cite{CREATe_2014} soll für diesen Zweck auch eine systematischen Literaturanalyse für die Begriffe "Open Access" und "Open Science" inklusive der Treiber und Bremser der Öffnung von Wissenschaft im Kontext des Begriffs "wissenschaftliche Reputation" durchgeführt werden. Neben der Berücksichtigung von Arbeiten aus den Medienwissenschaften im engeren Sinn sollen auch Arbeiten aus den Wirtschaftswissenschaften und den Kulturwissenschaften berücksichtigt werden.
\subsubsection{Forschungsfragen} 
Folgende Forschungsfragen sollen bei der Inhalsanalyse genauer analysiert werden:
\begin{itemize}
\item Warum kommt es zu der Bestrebungen hin zur Öffnung von Wissenschaft? 
\item Wie werden Open Science und Open Access definiert und voneinander abegrenzt? 
\item Welche Pro- und Contraargumente gibt es für die Öffnung von Wissenschaft - ist Offenheit in der Wissenschaft gut oder schlecht? 
\item Wo sind die Grenzen der Öffnung? 
\item Warum ist die Öffnung von Wissen in den verschiedenen wissenschaftlichen Disziplinen unterschiedlich weit verbreitet? 
\item Was bedeutet Offenheit und freier Zugang im Rahmen des wissenschaftlichen Diskurs-, Reputations- und Machtbegriffs?
\end{itemize}	

\subsubsection{Erhebungsmethode und Umfang} 
tbd

\subsubsection{Analyse der Definitionen von Open Access} 
tbd

\subsubsection{Analyse der Meinungen und Kritik an Open Access}

In diesem Teil der Arbeit soll im Rahmen der Literaturanalyse eine Auflistung der Kritikpunkte an der Open Access Bewegung in Wissenschaft und Forschung dokumentiert werden. Die Auswahl der berücksichtigten Werke bezieht sich auf dei genannten Fragestellungen und soll als verständlicher Überblick über den vorherrschenden Diskurs im Rahmen von Open Access und Open Science verstanden werden.

\subsubsection{Analyse der Definitionen von Open Access} 

Eine eindeutige Klassifizierung von Open Access gelingt derzeit nicht. Es "keine formelle Struktur, keine offizelle Organisation und kein ernannter Führer" gibt, der die Open Access Bewegung antreibt\cite{poynder_2011_suber}. Einzig und allein die Open Definition - open definition schreiben -

\subsubsection{Treiber und Bremser für Open Access} 

In den wissenschaftlichen Beiträgen zu Open Access werden viele positive Folgen aufgelistet. Folgende Treiber für eine Veränderung und Öffnung des wissenschaftlichen Kommunikationssystems werden dabei besonders häufig genannt:

\begin{itemize}
\item Verbreitung und Nutzungsmöglichkeiten der digitalen Infrastrukturen
\item Vorteile des grenzüberschreitenden Austauschs im Rahmen der Globalisierung von Wissenschaft und Forschung
\item ...
\end{itemize}

Neben den Aspekten die die Verbreitung von Open Access in den letzten Decaden unterstützt haben, gibt es aber auch einige Kriterien, die entweder zu einer Verlangsamung der Entwicklung geführt haben, oder sie in einigen Teilbereichen ganz zum erliegen gebracht haben. Dazu gehören:

\begin{itemize}
\item Fehlende Richtlinien auf regionaler, nationaler und internationaler Ebene
\item Führungslosigkeit der Open Access Bewegung
\item ...
\end{itemize}

\subsection{Analyse der Definitionen von Open Science} 

--- TODO ---- Michael Nielsen: “Open science is the idea that scientific knowledge of all kinds should be openly shared as early as is practical in the discovery process.”  https://lists.okfn.org/pipermail/open-science/2011-July/000907.html
http://www.openscience.org/blog/?p=454,

Research Information Network: “science carried out and communicated in a manner which allows others to contribute, collaborate and add to the research effort, with all kinds of data, results and protocols made freely available at different stages of the research process.” http://www.rin.ac.uk/our-work/data-management-and-curation/open-science-case-studies

Fecher/Friesike 5 Schulen von Open Science http://blogs.lse.ac.uk/impactofsocialsciences/2013/06/20/open-science-new-perspectives-for-scholarly-communication/ 
Siehe "Open Science"-Teil @ https://docs.google.com/document/d/1qDkQV-M_2VazjWwncRq_udo9tQqrjuZZkdLeKFc3cpI/edit#heading=h.1ahb76xafkbm

"Open science is the concept of making the whole research process as transparent and accessible as possible."\cite{Scheliga_2014}

Open science can be seen as a mechanism of cumulative knowledge production whereby scientists draw upon knowledge derived at by "prior researchers" and make their discoveries available to "future researchers". \cite{Scheliga_2014} auf Grundlage von \cite{Mukherjee_2009}
--- TODO ---

Es gibt zahlreiche Open Science Initiativen \cite{Scheliga_2014} viele von Ihnen erreichen aber keine kritische Masse \cite{wrap_2010} und enden eher als "virtuelle Geisterstädte" \cite{Nielsen_2011}.

\subsubsection{Analyse der Definitionen Treiber und Bremser für Open Science} 
tbd

\subsubsection{Analyse der Meinungen und Kritik an Open Science}

Während viele Wissenschaftler und Wissenschaftlerinnen Offenheit in der Forschung als wertvoll erachten, sind nur wenige sind wirklich bereit, die zusätzliche Zeit und Mühe zu investieren und potenziellen Risiken einzugehen, ihre Forschung offen und zugänglich zu machen \cite{Scheliga_2014} \cite{Procter_2010}. Forscherinnen und Forscher, die offene Wissenschaft pratizieren wollen, sind mit einer Reihe von Hindernissen konfrontiert \cite{Scheliga_2014}: 
\begin{enumerate}
\item individuelle Hindernisse: Angst vor Trittbrettfahren, Mehraufwand an Zeit und Mühe, Herausforderungen bei der Nutzung der digitalen Dienste, fehlender Anstoß negative Ergebnisse zu veröffentlichen, Herausforderung den Datenschutz sicherzustellen, Abneigung den Code zu teilen
\item systematische Hindernisse: Evaluationskriterien behindern Offenheit, kulturelle und institutionelle Einschränkungen, ineffektive (politische) Richtlinien, Mangel an Standards für das Teilen von Forschungsmaterialien, Mangel an rechtlicher Klarheit, finanzielle Aspekte der Offenheit
\end{enumerate}

Betrachtet wie Scheliga und Friesike das Phänomen Open Science an Hand des Konzepts des Soziale Dilemmatas, wird deutlich, dass was im kollektiven Interesse der wissenschaftlichen Gemeinschaft ist, nicht unbedingt im Interesse des einzelnen Wissenschaftlers ist und "wenn alle Wissenschaftler ihr Wissen nur in den Situationen teilen, in denen sie erwarten, dass sie selbst davon profitieren, ist die gemeinsame Wissenspool fragmentiert und alle Wissenschaftler stehen schlechter dar"\cite{Scheliga_2014}. 

Demgegenüber stehen dem gegenüber  --- TODO: ausarbeiten ---

\subsection{Methode der Onlinebefragung}
Um der Entwicklung der Öffnungvon Wissenschaft sowie deren Treiber und Bremser nachgehen zu können, soll eine Onlinebefragung unter den beteiligten Stakeholdern des akademischen Publizierens an wissenschaftlichen Institutionen explorativ durchgeführt werden. Dies ist nicht zuletzt deshalb für diese Arbeit relevant, weil theoretische Vorannahmen im Rahmen der Definition und Abgrenzung sowie der Literaturanalyse bestehen. Somit sollen die bestehenden Hypothesen getestet, beziehungsweise neue Hypothesen generiert werden. Durch einen Vergleich mit der Studie "Neue Formen des Wissenschaftlichen Publizierens" aus dem Jahr 2007 und 2008 vom Soziologisches Forschungsinstitut Göttingen (SOFI) soll darüber hinaus ein Einblick in die historische Entwicklung der Thematik im deutschsprachigen Raum ermöglicht werden. Die Befragung aus Göttingen bildet ausserdem die Grundlage für die Fragebogenkonstruktion dieser Erhebung. 

Die umfrangreiche Befragung aus den Jahren 2007 und 2008 entstand im Rahmen eines BMBF geförderten Verbundprojekts zwischen SOFI Göttingen und der Universitätsbibliothek Göttingen. Sie basierte auf einer "Vollerhebung der Wissenschaftler an den Instituten und Einrichtungen an fünf deutschen Standorten, die differenziert nach Fächern, Alters- und Statusgruppen (n=6500) erfasst wurden" \cite{Hanekop_2014}. Ziel der Befragung war es, die "Veränderungen beim Zugang zur Literatur wie auch bei den Veröffentlichungsstrategie" \cite{Hanekop_Wittke_2007_Fragebogen} zu untersuchen. In die Teilnehmer der Studie wurden anhand von Webseiten der Forschungseinrichtungen identifiziert und per Email zur Teilnahme aufgefordert. 

\subsubsection{Aufbau des Fragebogens}
Damls wurden 6500 Wissenschaftler und Wissenschaftlerinnen befragt, von denen 1803 geantwortet haben. Der 2007 verwendete Fragebogen bestand aus 51 Fragen. Im ersten Teil des Fragebogens wurden Fragen zu dem Fachgebiet und Tätigkeitsbereich zunächst als Leserin bzw. Leser wissenschaftlicher Publikationen erfasst. Im zweiten Teil wurden die Teilnehmer aus der Perspektive als Autorin beziehungsweise als Autor befragt. Abschließend wurden noch eineige personenbezogene Angaben abgefragt. \cite{Hanekop_Wittke_2007_Fragebogen} Die Skalen zur Beantwortung der Fragen waren unterschiedlich ausgewählt. 

Zu Beginn der Fragebogenkonstruktion wurde der Fragebogen und das Datenmaterial der Vorbefragung einer Itemanalyse zum Ausschluss unpassender Fragen (Items) unterzogen und Fragen bezglich der Fragestellung dieser Arbeit hinzugefügt. Dafür wurden die veröffentlichten Antworten analysiert. Fragen, die stark ungleich verteilt waren, wurden , wenn sie nicht inhaltlich interessant erschienen, ausgeschlossen.  Somit wurden auf der Basis der Analyse der Fragen der Fragepool auf 31 Fragen reduziert beziehungsweise verändert.

Die zentralen Forschungsfragen dieser Arbeit rückten in diesem Arbeitsschritt der Fragebogenerstellung in den Fokus und stellten die Grundlage für die Entwicklung des Fragepools dar. Die Formulierung der Fragen basierte, sofern nicht aus der Vorbefragung von SOFI unverändert übernommen, auf Handlungsmustern, Meinungen und Einstellungen zu folgenden Fragestellungen:
\begin{itemize}
\item Wie verändert die Digitalisierung, wie wir auf wissenschaftliche Daten und Informationen zugreifen?
\item In welchem Umfang herrscht unter den Wissenschaftlern und Wissenschaftlerinnen Wissen über die Öffnung von Wissenschaft vor? 
\item Welches Verständnis von Open Access besteht unter den Befragten? 
\item Wie stark ist das Interesse an Forschungsdaten? 
\item Welche Faktoren und Argumente begünstigen die Öffnung von Wissenschaft in einer wissenschaftlichen Disziplin? 
\item Welche Faktoren und Argumente sprechen gegen die Öffnung von Wissenschaft in einer wissenschaftlichen Disziplin? 
\item Wie wird der geschätzte Aufwand für die Öffnung von Wissenschaft in einer wissenschaftlichen Disziplin eingeschätzt?
\item Welche weiteren extrinsischen Faktoren unterstützen die Verbreitung von Offenheit in Wissenschaft und Forschung? 
\item Welche unterschiedlichen Auffassung bestehen zwischen den unterschiedlichen Fachdiziplinen, Alters- und Statusgruppen?
\item In welchem Umfang wird bereits heute im wissenschaftlichem Umfeld offen kommuniziert?
\item Welche Veränderungen beim Zugang zur Literatur wie auch bei den Veröffentlichungsstrategie sind im Vergleich zur der 2007 und 2008 durchgeführten Befragung des SOFI Göttingen zu erkennen?
\end{itemize}

Die Gliederung war ebenfalls an die Befragung aus den Jahren 2007 und 2008 angelehnt und beschränkte sich in der ersten Gruppe auf die Rahmenbedingungen der Teilnehmenden sowie deren wissenschaftlichen Tätigkeit. In der zweiten Fragegruppe wurden Aspekten aus der wissenschaftlichen Leserperspektive abgefragt. Die dritte Fragegruppe beschäftigte sich mit Fragen rund um den Zugang zu wissenschaftlichen Informationen, gefolgt von der vierten, die aus Fragen bezüglich des Zugangs zu wissenschaftlichen Informationen und des Zugriffs auf wissenschaftliche Infromationen bestand. In der fünften Fragegruppe wurden Fragen aus der Perspektive des Autors und der Autorin von wissenschaftlichen Inhalten gestellt. Abschließend wurden weitere personenbezogene Daten zur eindeutigen Segmentierung abgefragt. .

\subsubsection{Erhebungsmethode und Messinstrumente}

Auf Grund der zunehmenden Verbreitung und Nutzung des Internets, hat die Online-Befragung längst Eingang in die empirische Sozialforschung gefunden \cite{Pannewitz_2002}. Auschlaggebend für die Auswahl dieser Befragungsform ist vor allem die Ökonomie, "die es einfach macht, große Stichproben in kurzer Zeit zu erheben" \cite{eichhorn_2004_online}. Darüber hinaus wurde in der Vorbefragung durch das SOFI ebenfalls auf das Internet als primäre Quelle für die Identifikation von Teilnehmern und Teilnehmerinnen und E-Mail als Kontaktaufnahmekanal zurückgegriffen.

\subsubsection{Technische Realisation des Fragebogens}

Als System für die Online-Befragung kam die Open-Source-Lösung "Limesurvey" zum Einsatz. Die Fragen wurden in 6 Gruppen geteilt. Auf der Startseite wurde der Fragebogen und das Vorgehen erklärt. Am Ende der Befragung wurde der Befragte auf die Studie des SOFI Göttingen hingewiesen und über Möglichkeiten zur weiteren Verbreitung des Fragebogens informiert.

\subsubsection{Ablauf der Befragung}
Vor der eigentlichen Befragung fand im Juni 2014 ein Vorversuch statt, in dem der Fragebogen auf seine Verständlichkeit, Logik und Eignung überprüft wurde. Insgesamt nahmen zwölf Wissenschaftler und Wissenschaftlerinnen, darunter auch die Wissenschaftlerin der Vorbefragung Heidemarie Hanekop, an einem Test teil. Die größte Kritik betraf die Fragen der Gruppe x, die Tester kritisierten, dass Sie einige der Fragen nicht verstanden haben und sich teilweise zu einseitigen Antworten gezwungen fühlten. Darüber hinaus wurden einige der Matrixfragen vereinfacht und die Pflichtangaben reduziert. Die finale Befragung fand von August bis November 2014 statt.

\subsubsection{Rekrutierung der Teilnehmer}
Die Rekrutierung der Teilnehmer fand, wie bei der Befragung durch das SOFI, durch die Identifizierung der Teilnehmer auf den Webseiten der Forschungseinrichtungen für die unterschiedlichen Fachdiziplinen im deutschsprachigen Raum statt. Darüber hinaus wurde die Befragung in Sozialen Netzwerken (Twitter, Facebook, Google+ und Researchgate) und über Mailinglisten verbreitet. Trotz Auswahloption im Fragebogen wurde explizit keine Studenten addressiert, da sie noch über keinen ausreichenden Erfahrungsschatz in Bezug auf das wissenschaftliche Publikations- und Kommunikationssystem verfügen. In einer personalisierten E-Mail wurden die Angefragten gebeten an der Online-Befragung Teilzunehmen. Insgesammt wurde so xxxx Wissernschaftler und Wissenschaftlerinnen angefragt. Die Rekrutierung begann am 18.08.2014 und wurde am xx.xx.2014 abgeschlossen. Am Ende der Befragung hatten die Teilnehmer ausserdem die Möglichkeit die Umfrage in ihren Netzwerken zu verbreiten.

\subsubsection{Rücklaufquote}
Insgesamt wurden xxxx Wissenschaftlerinnen und Wissenschaftler per E-Mail angeschrieben. Davon haben xxxx an der Befragung teilgenommen, xxxx den Online-Fragebogen gestartet und xxxx den Online Fragebogen vollstädig beendet. xxx haben den Online-Fragebogen vor Beendigung abgebrochen. Die Rücklaufquote liegt somit bei xx,xx Prozent. 

\subsection{Das Experiment als wissenschaftliche Methode: Offenes Schreiben dieser Arbeit}
Zur weiteren Erkenntnisgewinnung und für das Ziel der Arbeit Handlungsempfehlungen für das offene Schreiben von Dissertationen erstellen zu können sowie die Kriterien und Argumente für oder gegen das offene Publizieren prüfen zu können, wurde für diese Arbeit selber eine offene Schreibweise gewählt. “Offen” bedeutet in diesem Fall, dass diese Arbeit direkt und unmittelbar bei der Erstellung für jeden, jederzeit frei zugänglich auf einer Webseite im Internet unter einer freien Lizenz (CC-BY-SA) veröffentlicht wurde. Der aktuelle Stand der Arbeit entsprach zu jedem Zeitpunkt dem Stand auf der Webseite. 

Um trotzdem den Anforderungen der Prüfungsordnung in allem Umfang gerecht zu werden, wurde in einem Schreiben an die Promotionskomission am 8. Januar 2013 alle betreffende Punkte in der Promotionsordung der Fakultät Kultutwissenschaften (Stand: 02.02.2011) hervorgehoben und versucht zu begründen, warum diese nicht im Widerspruch zur offenen Schreibweise meiner Arbeit stehen. Um die selbstständige, wissenschaftlicher Arbeit sicherzustellen, hatte kein anderer die Möglichkeit, den erstellten Inhalt zu editieren oder zu kommentieren. Die Transparenz während der Erstellung stellt in diesem Fall kein Widerspruch zu der Selbständigkeit bei der Ausarbeitung dar. Im Gegenteil, sie ermöglichte eine neue Form, die Eigenständigkeit direkt während der wissenschaftlichen Arbeit und Erstellung des Inhalts sicherzustellen. Dem Gesuch die Arbeit "offen" verfassen zu dürfen, wurde seitens der Promotionskommission am 12. Dezember 2013 mehrheitlich entsprochen.

\subsubsection{konzeptionelle und technische Rahmenbedingungen}

Konzeptionell war das Projekt so angelegt, dass die Arbeit unter allen Umständen jederzeit frei und offen Verfügbar einsehbar sein sollte. Inital sollte ein Blog auf der Grundlage der verbreiteten Open Source-Lösung Wordpress zum Einsatz kommen und nicht nur die Dokumentation rund um die Arbeit, sondern auch als technische Plattform für die gesamte Arbeit selbst zur Verfügung stehen.

\subsubsection{Herausforderungen während der offenen Anfertigung der Dissertation}

Die Arbeit wurde, bis zur Klärung der Erlaubnis durch die Promotionskommission im Dezember 2013, in einem Google Dokument ohne Freigaben verfasst. Google Docs, ist ein kostenloser, webbasiertes Textverarbeitungssystem der Firma Google. Es ist angelehnt an die gängigen Programme von Microsoft Office oder Open Office, bietet aber einige Einschränkungen besonders für das wissenschaftliche Publizieren.

Nach der schriftlichen Erlaubnis durch die Promotionskommission vom 11. Dezember 2013, wurde in einem ersten Schritt das Google Dokument offen zur Verfügung gestellt beziehungsweise für jeden lesbar freigegeben und aus dem Blog verlinkt. Bei der Übertragung der bereits geschriebenen Inhalte in das Blogsystem, stellte sich schnell heraus, dass die Blog-Software für die Veröffentlichung in einzelnen Blogposts nicht geeignet war. Zwar ermöglichten zusätzliche Plugins die Veröffentlichung von Inhalten in wissenschaftlichen Formen und Formaten, aber an folgenden Aspekten scheiterte der Einsatz der Blogsoftware als Publikationplattform:
\begin{itemize}
\item Die Darstellung der einzelenen Kapitel als Blogposts hat sich als sehr aufwendig und im Schreibptozess als unpraktikabel herausgestellt
\item Eine einfache aber standardisierte Referenzierung von von Literaturverweisen ist nicht möglich
\item Fußnoten konnten nicht über mehrere Einträge hinweg zusammenhängend dargestellt werden
\item Der Export in ein lesbares Dokument ist immer mit Aufwand verbunden
\item Die strukturierte Eingabe der Inhalte ist ebenfalls nur unter Aufwand möglich
\item Bei der Anzahl an unterschiedlichen Revisionen wäre das Revisionssystem sicher an seine Grenzen gestoßen
\item ...
\end{itemize}