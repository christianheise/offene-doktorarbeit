\chapter{Einleitung}

Unbeschränkte und offene Kommunikation, sowie die Kenntnis des gegenwärtigen Wissensstandes, ist für wissenschaftliche Forschung, deren Aufgabe neues, überprüfbares Wissen zu produzieren \cite[:551]{Luescher_2014} \cite[:298]{Luhmann1998} und zu verbreiten \cite[:100]{graefen2007_wissenschaftliche_artikel}, Verzerrungen und Fehler zu vermeiden, unter den Regeln der Logik und Methodik wissenschaftlich arbeitet \cite[:494]{Weber_1992}, zu beschreiben und zu bestimmen \cite[:124]{Luhmann1998} und damit dem gesellschaftlichen Auftrag des Wissenschaftssystems gerecht zu werden \cite[:3]{Hanekop_2014}, unverzichtbar \cite{glaeser2006} \cite{gibbons_1994}. Offenheit und Transparenz muss demnach als ein wesentlicher Bestandteil der Ethik der Wissenschaft bezeichnet werden \cite{Peters_2014} \cite{resnik_2005_ethics}.

Die "gestörten Gleichgewichte im wissenschaftlichen Publikationssystem" \cite{cite:0}, das kaputte wissenschaftliche Anreizsystem \cite{osterloh2008anreize}, der steigende Publikationsdruck, die finanzielle und ideelle Notlage von Bibliotheken \cite{russell_2008_business} \cite{Sietmann_oa_2007}, Herausforderungen bei der Wahrung der Freiheit und Unabhängigkeit von Wissenschaft und Forschung \cite{Goetting_2015}, fehlende Transparenz, der Anstieg an Wissenschaftsskandalen \cite{brembs2015open} sowie die Ökonomisierung der Universitäten \cite{bauer2006} führen allerdings zu der Frage, ob der aktuelle Zustand des wissenschaftlichen Kommunikationssystems der Aufgabe von Wissenschaft uneingeschränkt gerecht werden kann \cite{Schekman_2013}.

Mit der zunehmenden Verbreitung und Etablierung des Internets als Kanal für die wissenschaftliche Kommunikation, Forschungsaktivitäten und den Austausch von Informationen wurden neue Hoffnungen für die Verbesserung dieser "fatalen und unhaltbaren Situation" \cite{brembs2015open}, sowie für die Öffnung des Wissenstransfers \cite{schulze_2013_open} \cite{albert_2006_open_implications} und des wissenschaftlichen Kommunikationsprozesses geweckt \cite{Hanekop_2014} \cite{EuropeanCommission_sciencepub_2006} \cite{Goodrum_2001} \cite{Lawrence_1999}. Diese Erwartungen umfassen unter anderem den Wunsch nach "unbeschränkten Zugang zur gesamten wissenschaftlichen Zeitschriftenliteratur" \cite{boai_2002}, nach mehr Transparenz im wissenschaftlichen Erkenntnisprozess \cite{eu_agenda_open_science_2015}, die Möglichkeiten der Steigerung von Effizienz und Effektivität von Wissenschaft \cite{Partha_1994_economics_science} und "dass die alten Zugangs- und Nutzungsbeschränkungen sukzessive ausgeräumt werden" \cite{boai_2002} können. Grundlage dafür ist die Annahme, dass die Folgen der technologischen Entwicklungen "zwangsläufig zu erheblichen Veränderungen im Wesen des wissenschaftlichen Publizierens führen und einen Wandel der bestehenden Systeme wissenschaftlicher Qualitätssicherung einleiten" \cite{berliner_erklaerung_2003} würden.

Im Zuge dieser technologischen Entwicklungen, politischen Forderungen und gesellschaftlichen Annahmen gab und gibt es auf der einen Seite ein großes Interesse an der offenen Kommunikation und Unterstützung für den Wunsch nach freiem Zugriff auf wissenschaftliche Informationen, auf der anderen Seite ist es in der Medien- und Technikgeschichte ein bekanntes Verhalten, dass es bei Einführung eines neuen Mediums mit größerer Reichweite zu Irritationen \cite{naeder_2010_open} und Irrelevanz- oder gar Verlustängsten kommt \cite{hagner_2015_sache_buches}. So zeigten die ersten Erfahrungen des Experimentierens mit dem Internet als neuen Kommunikationskanal für den wissenschaftlichen Austausch schnell, dass es sehr viel schwieriger sein würde, das wissenschaftliche Kommunikationssystem zu öffnen und die Hürden für einen Wandel des Systems größer sind, als ursprünglich angenommen \cite{bjork_2004_open}.

Somit bestehen trotz der zunehmenden Digitalisierung wissenschaftlicher Kommunikationssysteme und -prozesse, weiterhin umfangreiche Zugangs- und Nutzungsbeschränkungen zu Wissen und nur langsam führen die ersten Modifikationen im System Wissenschaft zu Konsequenzen bei der Verfügbarkeit von Wissen für die Gesamtgesellschaft. Auch rund 25 Jahre nach den ersten elektronischen Verfahren zum offenen Austausch wissenschaftlicher Publikationen und 350 Jahre nach dem Erscheinen der ersten wissenschaftlichen Fachzeitschrift, muss das "alte" System demnach noch immer als weitestgehend stabil bezeichnet werden \cite{brembs2015open} \cite{Hanekop_2014}. Die Gründe und Einflussfaktoren für diese Entwicklung in Wissenschaft und Forschung werden im folgenden dargestellt, empirisch und experimentell überprüft sowie abschließend diskutiert und zusammengefasst.

\section{Relevanz des Themas}

Im Rahmen des postulierten Wandels stehen nicht nur die Wissenschaftlerinnen und Wissenschaftler, sondern auch das ganze Universitätssystem, sowie andere Bildungseinrichtungen und wissenschaftliche Bibliotheken vor bedeutenden Herausforderungen \cite{muller_2010_open} \cite{Harter2006} \cite{Gu_don_2004} \cite{osterloh2008anreize} \cite{Beverungen_2014}. Die wissenschaftliche Kommunikation hat sich dabei in den letzten Jahrhunderten nur marginal verändert. "Nach innen" (wissenschaftsintern) bietet sie zwar einen gewissen Grad an Offenheit, aber nach außen ist sie geschlossen \cite{kelty_2004}.

Die Institution Universität und wissenschaftliche Einrichtungen laufen im Kontext dieser Entwicklungen Gefahr, ihre Bedeutung als Ort der Wissensproduktion und -evaluation (weiter) zu verlieren \cite[:343]{Kruecken_2001}. Denn spätestens mit der Privatisierung der Verarbeitung, Speicherung und Übertragung von Wissen hörten Universitäten damit auf, selber Bücher zu verlegen \cite{cite:0}. Darüber hinaus fordert die Wirtschaft (zunehmend) eine öffentliche Finanzierung der Wissensproduktion und erwartet gleichzeitig die privat(-wirtschaftliche) Aneignung und Nutzung des produzierten Wissens \cite{cite:2}. Dieses Prinzip der kollektiven Wissensproduktion, bei dem die Wirtschaft unentgeltlich an wissenschaftliche Informationen gelangt, wird auch von Verlagen für ihre Wertschöpfung genutzt. Neben dem entgeltlichen Vertrieb der wissenschaftlichen Informationen ermöglichen sie den Autoren, durch den "Rückgriff auf informal konstituierte Reputationen" \cite{luhmann_1970_selbststeuerung} und die vorangegangene Privatisierung des wissenschaftlichen Reputationssystems \cite{suchen}, als Gegenleistung die Chance auf Anerkennung von der wissenschaftlichen Community und Reputation im wissenschaftlichen System \cite{cite:21a}.

Im Zusammenhang mit dem Wandel sind für die Wissenschaftlerinnen und Wissenschaftler und damit abgeleitet auch für die Institutionen, die Wahrung der Freiheit von Wissenschaft und Forschung bei möglichst uneingeschränkter Verbreitung wissenschaftlicher Erkenntnisse \cite{hagner_2015_sache_buches} \cite{bbaw_publizieren_2015} auf der einen, sowie die Forderung nach besseren (Selbst-)Steuerungs- und Leistungsprozessen \cite{Adler_2009} \cite{gibbons_1994} auf der anderen Seite als besondere Herausforderungen zu nennen. Ebenfalls von besonderer Bedeutung sind auch die Auswirkungen des Wandels auf das Kommunikations- und Reputationssystem der Wissenschaft.

Das Auseinanderdriften der Interessen zwischen der privatwirtschaftlichen Nutzung wissenschaftlicher Erkenntnisse und der ursprünglichen Aufgabe von Wissenschaft in den letzten Jahrzehnten führten zu einer wissenschaftlichen Publikations- und Kommunikationskrise. Sie ist durch den wachsenden Kostendruck, Preissteigerungen \cite{lewis_2015_future}, Publikations- \cite{Egger_1997} \cite{Fanelli_2012} \cite{Beverungen_2012} \cite{Brembs_20013} und Reportbias \cite{Chan_2008} \cite{Dickersin_2011}, Cargo Cult Science \cite{Feynman_1974} und die Einschränkung des Zugriffs auf wissenschaftliche Informationen \cite{Hess_2006} gekennzeichnet. Das aktuelle System mit den genannten Problemen steht dem Bestreben, dass es der Wissenschaft im Kern um Erkenntnisse und die uneingeschränkte Zurverfügungstellung dieser geht \cite{hanekop_2006}, entgegen \cite{offhaus_2012_institutionelle_repos}.

Infolgedessen entstand unter den Wissenschaftlern und Wissenschaftlerinnen auch die Besorgnis, dass es durch Publikationsdruck und den Druck anwendungsorientierter zu Forschen wahrscheinlicher wird, dass viele der veröffentlichten Forschungsergebnisse "eher falsch als richtig sind" \cite{Ioannidis_2005}. Die genannten Entwicklungen befördern die Geschlossenheit des wissenschaftlichen Kommunikationssystems, erschweren nachhaltig den Zugang zu Wissen, beeinträchtigen die Erstellung von neuem Wissen \cite{cite:5} \cite{cite:8} \cite{Luhmann1998} und führen zu einem zunehmend unhaltbaren Zustand bei der wissenschaftlichen Kommunikation \cite{Schekman_2013}.

Sucht man nach Gründen für die Beibehaltung des bisherigen Modells durch die Wissenschaftsgemeinschaft, wird deutlich, dass vor allem Unwissen über die wirtschaftlichen Entwicklungen, rechtliche Bedenken und das etablierte wissenschaftliche Reputationssystem zentrale extrinsische Motivationsfaktoren für die Unterstützung des bisherigen Systems durch die wissenschaftliche Gemeinschaft darstellen \cite{minssen_2012_arbeit}. Als weiteren Grund wird die komfortable Situation der Wissenschaftler in dem System genannt, bei dem Wissenschaftler und Wissenschaftlerinnen selten auf Zugang zu wissenschaftlichen Publikationen verzichten müssen und von der Auseinandersetzung mit den finanziellen Aspekten weitestgehend befreit sind \cite{herb_2010} \cite{Sietmann_oa_2007} \cite{hanekop_2006}. Dennoch trägt die Verschärfung der Krise und die langsam spürbaren Auswirkungen auf die wissenschaftliche Gemeinschaft dazu bei, dass die Forderung nach Veränderung des Systems zunehmende Unterstützung erfährt.

Die Suche nach einem Ausweg aus dieser Kommunikations- und Publikationskrise führte zu der anhaltenden Forderungen nach der Öffnung von Wissen und nach Alternativen für das geschlossene wissenschaftliche Publikations- und Kommunikationssystem. Ergänzend zu den erstmals artikulierten Forderungen nach der Öffnung dieser geschlossenen Form der Kommunikation in Wissenschaft und Forschung befinden wir uns infolge der neuen Möglichkeiten durch die Digitalisierung und Globalisierung \cite{mcluhan_1962_gutenberg} inmitten eines "radikalen Wandels" \cite{poynder_2011_suber} tradierter wissenschaftlicher Kommunikationssysteme. Dieser Wandel bietet nicht nur die Chance für die Lösung der Herausforderungen im aktuellen wissenschaftlichen Kommunikationssystem, sondern ermöglicht auch eine umfassende "Beschleunigung des Wissensumschlages" \cite{Wenzel_2003} und führt potenziell dazu, dass offene Innovation und offene wissenschaftliche Kommunikation den privaten und staatlichen Forschungsbereich effizienter machen \cite{chesbrough_2006_open}, sowie den gesamtgesellschaftlichen Fortschritt in bisher unbekannter Weise beschleunigen \cite{cite:7}.

Ungeachtet dieser Entwicklungen ist unübersehbar, dass das System der wissenschaftlichen Kommunikation noch immer "weitgehend stabil" \cite{Hanekop_2014} geblieben ist und im aktuellen Steuerungssystem der Wissenschaft weiterhin anhand der tradierten wissenschaftlichen Bewertungssysteme Reputation, Mittel und Stellen verteilt werden \cite{hollricher_wandel_2009} \cite{cite:4}. Die analog gedruckten und bewährten Journale, sowie andere Publikationsformen der großen wissenschaftlichen Verlage werden bisher einfach nur mit nahezu unverändertem Geschäftsmodell zusätzlich digital verbreitet \cite{Hanekop_2014} \cite{boai_2012}.

Trotz umfangreicher Literatur liegen bisher nur wenige Untersuchungen und Experimente zur Öffnung wissenschaftlicher Kommunikation, vor allem aus den Geisteswissenschaften \cite{naeder_2010_open}, vor. Daraus ergibt sich die Relevanz und Notwendigkeit, die bisherigen Entwicklungen im Bereich der Öffnung wissenschaftlicher Kommunikation aus geistes- und kulturwissenschaftlicher Perspektive genauer zu untersuchen, den Erkenntnissen über die Öffnung wissenschaftlicher Kommunikation aus der Literatur empirisch erhobene Daten gegenüberzustellen, die Erkenntnisse praktisch-experimentell zu überprüfen und das Ergebnis zu diskutieren und einen Ausblick für die weitere Entwicklung zu wagen.

\section{Zielsetzung der Arbeit}

In dieser Arbeit wird untersucht, welche Auswirkungen die Digitalisierung und die Forderung nach Öffnung der Wissenschaft, beziehungsweise der wissenschaftlichen Kommunikation, auf Universitäten, wissenschaftliche Einrichtungen aber auch auf den einzelnen Wissenschaftler und die einzelne Wissenschaftlerin haben. Von besonderem Interesse sind in diesem Zusammenhang die Unterschiede von reinem Zugang zu publiziertem Wissen auf der einen und dem kompletten Zugriff auf den gesamten wissenschaftlichen Erkenntnisprozess auf der anderen Seite, sowie das Zusammenspiel unterschiedlicher Formen der Wissensverbreitung vor dem Hintergrund der geschichtlichen Entwicklung.

Es wird dabei betrachtet, inwieweit es sich bei der Öffnung von Wissenschaft im Rahmen von Open Access und Open Science tatsächlich um einen grundlegenden Wandel in der wissenschaftlichen Kommunikation handelt. Im weiteren Fokus der Untersuchung stehen der Umstand, dass die meisten wissenschaftlichen Informationen der Allgemeinheit bisher nicht zugänglich sind und welche Herausforderungen, das Wissen frei(er) zugänglich zu machen, daraus resultieren sowie welche Konsequenzen für das Wissenschaftssystem daraus zu erwarten sind.

Diese Thematik wird auch in Bezug zu der Massifizierung und Neoliberalisierung der Universität gesetzt und in einen historischen Kontext gestellt. Argumente für und gegen die Öffnung wissenschaftlicher Kommunikation aus Sicht der am wissenschaftlichen Kommunikationssystem beteiligten Akteure werden erhoben, um Gründe für die bisherigen Entwicklungen zu erarbeiten und zu einem vertieften Verständnis der Definitionen von Open Access und Open Science im Kontext wissenschaftlicher Kommunikation zu gelangen. Die wissenschaftliche, disziplinübergreifende Debatte um die Öffnung von Wissenschaft und Forschung wird dabei dargestellt, auf den deutschsprachigen Raum begrenzt sowie Katalysatoren und Hindernisse für die Etablierung der Forderung nach Öffnung bei den wissenschaftlichen Akteuren identifiziert und abgefragt.

Ziel ist eine aktuelle Verhandlung der theoretischen Annahmen und Definitionen rund um die Etablierung und Praktizierung offener wissenschaftlicher Kommunikation mit den praktischen Gegebenheiten im wissenschaftlichen Alltag. In diesem Zusammenhang wird insbesondere die Diskrepanz zwischen der Idee der Öffnung von wissenschaftlicher Kommunikation und der wissenschaftliche Realität \cite{Scheliga_2014} adressiert sowie die Gründe für die schleppende Umsetzung der Konzepte rund um die Öffnung von Wissenschaft erarbeitet. Die Erfahrungen und Meinungen der Wissenschaftler und Wissenschaftlerinnen werden dabei den Erfahrungen aus einem Selbstversuch gegenübergestellt und abschließend Handlungsempfehlungen für das offene Bearbeiten wissenschaftlicher Fragestellungen gegeben.

Die forschungsleitende Hypothese dieser Arbeit ist, dass sich die Öffnung des Zugangs zu wissenschaftlichen Erkenntnissen (Open Access) in einer andauernden Übergangsphase zur Öffnung des Zugriffs auf den gesamten wissenschaftlichen Erkenntnisprozess (Open Science) befindet. Die sich daraus ableitenden Fragestellungen umfassen zum einen die theoretische Bedeutung und Historie von Offenheit im Rahmen der wissenschaftliche Kommunikation, zum anderen die empirische Frage nach den Motiven und Beweggründen für Wissenschaftler der unterschiedlichen Disziplinen, das aktuellen wissenschaftliche Kommunikationssystem oder die Forderung hin zu Offenheit zu unterstützen. Anschließend wird experimentell erarbeitet, welchen Hürden und welcher Aufwand durch die Öffnung der formellen Kommunikation für Wissenschaftler und Wissenschaftlerinnen entstehen.

Es wird erörtert, welche möglichen Auswirkungen durch diesen Prozess der Öffnung auf das Selbstverständnis der Wissenschaft an sich und auf die wissenschaftliche Reputation in den unterschiedlichen Disziplinen zu erwarten sind. Dafür werden relevante Wege des Wissenstransfers ermittelt, Probleme und Hemmnisse bei der offenen Durchführung von wissenschaftlicher Arbeit herausgearbeitet und Handlungsmöglichkeiten am Beispiel der Erstellung von einer Doktorarbeit erschlossen und diskutiert.

\section{Aufbau der Arbeit}

Die Arbeit ist in sieben Abschnitte unterteilt. In der Einleitung werden eine Einführung in die Thematik der Arbeit vorgenommen und die Relevanz des Themas sowie die Beweggründe und Positionen des Autors dargestellt.

Im zweiten Kapitel werden Vorüberlegungen zur Methodenwahl angestellt und die Forschungsfragen ausformuliert, sowie der zur Beantwortung der Forschungsfragen angewandte Methodenmix beschrieben, begründet und kritisch betrachtet.

Im Teil "Grundlagen: Chronologie, Begriffsbestimmungen und Debatten" wird initial die Entwicklung wissenschaftlicher Kommunikation chronologisch dargestellt, auf die Forderung der Öffnung wissenschaftlichen Kommunikation und auf Veränderungen im Rahmen der Digitalisierung eingegangen und in den Kontext der wissenschaftlicher Reputation, des wissenschaftlichen Ethos und Diskurs gestellt. Des weiteren werden die Herausforderungen im bestehenden wissenschaftlichen Kommunikationssystem anhand der Debatten in der Literatur ausgearbeitet Anknüpfungspunkte für die empirische Untersuchung abgeleitet.

Im darauffolgenden Abschnitt wird eine empirische Untersuchung zur Prüfung der identifizierten Katalysatoren und Hindernisse für die Öffnung von Wissenschaft und Forschung, die mittels einer Online-Befragung im Rahmen dieser Arbeit durchgeführt wurde, dokumentiert und ausgewertet. Dabei wird auf die Defizite und die aktuellen Debatten um die Begriffe "Open Access" und "Open Science" auf Grundlage der Erkenntnisse aus den vorhergehenden Kapiteln zurückgegriffen.

Die offene Erstellung und Dokumentation der vorliegenden Arbeit im Sinne eines prospektiven Realexperimentes erweitern den empirischen Ansatz im darauffolgenden, fünften Kapitel. Hier wird versucht einen primären verstehenden Zugang zu den Forschungsfragen und Zielen der Arbeit zu erhalten und diesen zu dokumentieren. Im Ergebnis werden Vorteile und Nachteile der offenen Anfertigung der Arbeit dargestellt, Praxistauglichkeit und Aufwand dokumentiert und Handlungsempfehlungen für das offene Verfassen wissenschaftlicher (Qualifikations-)Arbeiten abgeleitet.

In den letzten beiden Kapiteln werden die gewonnen Ergebnisse und die Vorgehensweise und die Fragestellungen dieser Arbeit kritisch diskutiert sowie abschließend zusammengefasst. Auf Grundlage der Forschungsergebnisse und der eigenen Erfahrungen werden Empfehlungen zum Schreiben offener wissenschaftlicher Arbeiten, sowie ergänzend ein Ausblick auf die weitere Entwicklung offener Strukturen im Rahmen wissenschaftlicher Tätigkeit formuliert.

Alle Teile folgen der forschungsleitende Hypothese, dass sich Open Access in einer Übergangsphase von der reinen offenen Bereitstellung wissenschaftlicher Publikationen zur Öffnung des vollendeten Kommunikationsakt des "Wissenszuwachs" \cite{Luhmann1998} an die Gesamtgesellschaft (Open Science) befindet. Ausgehend von den Fragestellungen wird dazu transdisziplinärer Zugang zur wissenschaftlichen Bearbeitung gewählt, der translational von den Kulturwissenschaften über die Politikwissenschaften und die Wirtschaftswissenschaften bis hin zu den Medienwissenschaften reicht, an die Wissenschafts- und Technikforschung angelehnt ist und sich empirischer, analytischer, sowie auch experimenteller Methoden bedient.

\section{Beweggründe und eigene Position}

Die Beweggründe für die Erstellung der vorliegenden Arbeit ist die Folge der langjährigen Auseinandersetzung des Autors mit dem Konzept von "Offenheit" als wissenschaftlicher Mitarbeiter am Hybrid Publishing Lab der Leuphana Universität und als Vereinsvorstand der Open Knowledge Foundation Deutschland. Die hier angestrebte Auseinandersetzung mit den Konzepten rund um Offenheit im wissenschaftlichen Kommunikationssystem zielt somit auch auf die kritische Auseinandersetzung und das Hinterfragen der eigenen Positionen.

Die eigene Position zum Beginn des Erstellungsprozess der Arbeit muss dabei als klar befürwortend gegenüber den Forderungen nach Öffnung des wissenschaftlichen Kommunikations - und Erkenntnisprozesses deklariert werden. Sie fußt auf den Erfahrungen des beruflichen und ehrenamtlichen Engagements in der Förderung, Forderung und Ausgestaltung offener und transparenter Kommunikation in den gesellschaftlichen Teilbereichen von Wissenschaft sowie in Politik und Verwaltung. Im Rahmen der Auslotung der Grenzen des Konzepts von Offenheit steht in dieser Arbeit auch eine differenzierte Betrachtung der eigenen Auffassung im Mittelpunkt der Auseinandersetzung und wird im Rahmen der jewiligen Betrachtungen immer wieder eine Rolle spielen.

Ziel dieser Auseinandersetzung ist es, die Forderung nach Öffnung wissenschaftliches Wissen und Kommunikation unter dem Einsatz von Technik als soziales und kulturelles Phänomen besser zu verstehen und diese Entwicklung einer differenzierten und kritischen Analyse zu unterziehen. Die Wandlungsprozesse des wissenschaftlichen Kommunikationssystems sollen auf soziotechnische Grundlagen hin untersucht und dabei trotzdem eine kritische Distanz gewahrt werden. Es werden verschiedene Methoden angewendet um die eigenen Position zu hinterfragen und diesen Prozess zu dokumentieren. In dieser Arbeit wird abschließend auch die Praxistauglichkeit der Forderungen an das wissenschaftliche Kommunikationssystem im Rahmen der offenen Anfertigung dieser Arbeit geprüft und kritisch hinterfragt.

Die Auseinandersetzung mit den Fragestellungen folgt dem Ansatz der Science and Technology Studies (STS), beziehungsweise der Wissenschafts- und Technikforschung. Sie bezeichnet ein transdisziplinäres Forschungsfeld, das Ende der späten 1970er Jahre angetreten ist, um "Wissenschaft und Politik neu zu denken" \cite{Potthast_2010}. Die "empirische Untersuchung der vielfältigen Rollen von Wissen und Technologie in modernen Gesellschaften" \cite{beck_2014_science} ist dabei vorrangiges Ziel der STS. Diese Herangehensweise beschäftigt sich mit den soziotechnischen Entwicklungen, den sozialen, kulturellen und politischen Dynamiken, die Wissenschaft und Technik formen, sowie der Frage, wie diese Dynamiken zukünftig Gesellschaft, Politik und Kultur beeinflussen \cite{Potthast_2010}.

Dieser Forschungszusammenhang ermöglicht es, das Forschungsthema aus unterschiedlichen Blickwinkeln zu betrachten und die Betrachtungen unterschiedlichen Disziplinen zuzuordnen \cite{beck_2014_science}  \cite{Potthast_2010}. Wissenschaft wird demnach nicht mehr nur "als Ergebnis rein intellektueller kontemplativer Tätigkeit, sondern als Ergebnis praktischen Tuns und sozialen Handelns" \cite{beck_2014_science} verstanden. In den STS sehen sich die Wissenschaftler und Wissenschaftlerinnen nicht als Entdecker, sondern als aktive Teilnehmer des Dialogs um die Entwicklung \cite{MacKenzie_STS_1999}. Sie streben danach die empirischen Realitäten von Technologie und Medien zu verstehen \cite{kelty_2014_freedom}.

Es ist unter anderem die Aufgabe von STS, die "Verschränkung von Wissenschaft, Technologie und Gesellschaft im Alltag zu untersuchen und damit auch die Rolle von Wissen und Technologie in gesellschaftlichen Ordnungsprozessen näher zu bestimmen" \cite{beck_2014_science}. Die STS haben sich bei der von Erforschung von Wissenschafts- und Technologiekultur zu einer etablierten Herangehensweise entwickelt um darzustellen, inwiefern Technologien politisch sind \cite{kelty_2014_freedom}.

Dieser Ansatz scheint am Besten geeignet um die Beobachtungen im Rahmen dieser Forschungsarbeit wissenschaftlich zugänglich zu machen. Die Einordnung basiert unter anderem auf der Annahme, dass die Öffnung wissenschaftlicher Kommunikation im engen Zusammenhang mit der technologischen und politischen Entwicklung steht \cite{weingart_2005_wissenschaft}. Dieser Zusammenhang wird in vorliegenden Arbeit empirisch im Alltag durch eine Befragung der wissenschaftlichen Akteure und das eigene offene wissenschaftliche Kommunizieren bei der Erstellung beforscht. Die Herangehensweise ermöglicht eine als Realexperiment konzipierte ethnographische Untersuchung des eigenen Forschungsalltags, bei der die teilnehmende Beobachtung sowie die Online-Befragung die Grundlage für den Methodenmix einer ethnographischen Betrachtung darstellen \cite{bachmann_2011_ethnographie}. Dieser Mix ermöglicht es die Auswirkungen auf die Kommunikation von Wissenschaftlern und Wissenschaftlerinnen möglichst vollumfänglich zu beschreiben.

Durch die Betrachtung der Rolle von offenem Wissen und Technologie für wissenschaftliche Kommunikationsprozesse sowie durch die empirische Erforschung der Wissensproduktion und -verbreitung, seinen epistemologischen Voraussetzungen und den daraus resultierenden Konsequenzen \cite{beck_2014_science} soll mit dieser Arbeit ein Beitrag zum Fortschritt für die Wissenschafts- und Technikforschung geliefert werden.
