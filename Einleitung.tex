\chapter{Einleitung} 
Das Universitätsystem, aber auch andere Bildungseinrichtungen und ihre Bibliotheken, stehen vor vielen großen Herausforderungen. Eine der wichtigsten ist die der Wahrung der Freiheit der Wissenschaft in Zeiten der digitalen Revolution und der Forderung nach besseren Steuerungs- und Leistungsprozessen in Forschung und Lehre aber auch der Umgang mit den “gestörten Gleichgewichten im wissenschaftlichen Publikationssystem”\cite{cite:0}. Friedrich Kittler hat das in einer Rede "Wissenschaft als Open-Source-Prozeß" im Jahr 1999 wie folgt zusammengefasst: "mit Freiheit von Quellcode steht und fällt auch die Freiheit der Wissenschaft". Damit ist gemeint, dass die "Verarbeitung des Wissens (im Rahmen der digitalen Revolution) technisch reproduzierbar"\cite{cite:1} und kontrollierbarer wird. Vorallem die europäische Universität verliert so spätestens seit den 90er Jahren immer weiter ihre Bedeutung als exklusiver und freier Ort der Wissensproduktion.

Als Treiber für diese Entwicklung sind können unter anderem follgende Faktoren genannt werden: 
\begin{enumerate}
\item Auf der einen Seite, der Umgang mit dem System der Bücher und der Verlage - bis zur Erfindung des Buchdrucks ebenfalls eine der exklusiven Aufgaben der Universität. Mit der Privatisierung der Verarbeitung, Speicherung und Übertragung von Wissen hörten Universitäten auf selber Bücher zu verlegen. In einem Erklärungsversuch für das Verständnis seitens der Verlage stellt Peter Weingart diesbezüglich fest, dass „die Wirtschaft (zunehmend) eine öffentliche Finanzierung der Wissenschaft und der Wissensproduktion, im Endeffekt aber gleichzeitig die privat(-wirtschaftliche) Aneignung und Nutzung des produzierten Wissens erwartet“\cite{cite:2}. Verlage nutzen das Grundprinzip der uneigennützigen, kollektiven Wissensproduktion, um unentgeltlich an wissenschaftliche Informationen von den wissenschaftlichen Autoren zu gelangen. Neben dem entgeldlichen Vertrieb der wissenschaftlichen Informationen erbringen sie den Autoren als Gegenleistung die Chance auf Anerkennung und Reputation. Das steht der Annahme, dass es der Wissenschaft im Kern aber um Erkenntnisse und diese der Gesellschaft, insbesondere aber den Wissenschaftlern als öffentliches Gut uneingeschränkt zugänglich sein sollten\cite{cite:3}, diametral entgegen. Das manifestiert sich spätestens dadurch, dass anhand der wissenschaftlicher Reputation Mittel und Stellen verteilt werden\cite{cite:4}.
\item Die Überzeugung, dass offene Innovation und offene wissenschaftliche Kommunikation den privaten und stattlichen Forschungsbereich effizienter macht sowie den industriellen Fortschritt beschleunigt \cite{cite:7}.
\item Die Feststellung, dass jede Beschränkung im Zugang zu Wissen auch die Erstellung von neuem Wissen verhindert\cite{cite:5}\cite{cite:8}.  Die meisten wissenschaftlichen Informationen sind der Allgemeinheit nicht zugänglich und kann nur in Universitäten und Forschungseinrichtungen durch wissenschaftliche Mitarbeiter, Studenten und Professoren abgerufen werden\cite{cite:6}. 
\end{enumerate}	

Im Rahmen der Arbeit soll untersucht werden, wie Universitäten und wissenschaftliche Einrichtungen nach der digitalen Revolution mit der relativ neuen Herausforderung aus der Netzkultur heraus Wissen frei(er) zugänglich machen können oder müssen, eine Herausforderung die in Relation zu der Massifizierung und Neoliberalisierung der Universität zu setzen ist.

In der Einleitung wird der Aufbau der Arbeit und der theoretische Bezugsrahmen genauer erläutert. Abschließend wird in dem Kapitel die Relevanz des Themas dargestellt.
