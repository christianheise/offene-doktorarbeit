\chapter{Einleitung} 

\section{Relevanz des Themas} 

Unbeschränkte und offene Kommunikation, sowie die Kenntnis des gegenwärtigen Wissensstandes, ist für wissenschaftliche Forschung, deren Aufgabe neues Wissen zu produzieren und dem gesellschaftlichen Auftrag des Wissenschaftssystems gerecht zu werden, unverzichtbar \cite{Hanekop_2014} \cite{glaeser2006} \cite{gibbons_1994} \cite{Luhmann1998}. Der aktuelle Zustand des wissenschaftlichen Kommunikationssystems wird diesem Auftrag nicht gerecht \cite{Schekman_2013}. Mit der Etablierung des Internets als neuen Kanal für die Kommunikation und den Austausch von Informationen wurden große Erwartungen für völlig neue Möglichkeiten des Wissenstransfers und der wissenschaftlichen Kommunikation geweckt \cite{Hanekop_2014} \cite{schulze_2013_open} \cite{albert_2006_open_implications} \cite{Goodrum_2001} \cite{Lawrence_1999}.

Im Rahmen dieses Wandels stehen aber auch das Universitätsystem, sowie andere Bildungseinrichtungen und Bibliotheken vor bedeutenden Herausforderungen \cite{Harter2006} \cite{Gu_don_2004} \cite{osterloh2008anreize}. Eine der anspruchvollsten ist die Wahrung der Freiheit von Wissenschaft und Forschung auf der einen, sowie die Forderung nach besseren Steuerungs- und Leistungsprozessen \cite{Adler_2009} \cite{gibbons_1994} auf der anderen Seite. Die “gestörten Gleichgewichte im wissenschaftlichen Publikationssystem” \cite{cite:0} und das kaputte wissenschaftliche Reputationssystem \cite{suchen} sind in diesem Zusammenhang von besonderer Bedeutung. 

Die Institution Universtät läuft im Kontext dieser Entwicklung gefahr, ihre Bedeutung als exklusiver Ort der Wissensproduktion \cite{suchen} und -evulation \cite{suchen} weiter zu verlieren. Mit der Privatisierung der Verarbeitung, Speicherung und Übertragung von Wissen hörten Universitäten auf, selber Bücher zu verlegen \cite{cite:0}. Die Wirtschaft forderte (zunehmend) eine öffentliche Finanzierung der Wissensproduktion. Die privat(-wirtschaftliche) Aneignung und Nutzung des produzierten Wissens wird erwartet \cite{cite:2}. Das Prinzip der kollektiven Wissensproduktion, bei dem die Wirtschaft unentgeltlich an wissenschaftliche Informationen gelangt, wird auch von Verlagen für ihre Wertschöpfung genutzt. Neben dem entgeldlichen Vertrieb der wissenschaftlichen Informationen erbringen sie den Autoren als Gegenleistung (durch den "Rückgriff auf informal konstituierte Reputationen" \cite{luhmann_1970_selbststeuerung} und die vorangegangene Privatisierung des wissenschafltichen Reputationssystems \cite{suchen}) die Chance auf Anerkennung von der wissenschaftlichen Community und Reputation im wissenschaftlichen System \cite{cite:21a}. 

Diese Entwicklungen führten zu einer wissenschaftlichen Publikations- und Kommunikationskrise, welche durch steigenden Kostendruck, Preissteigerungen, Publikations- \cite{Egger_1997} \cite{Fanelli_2012} und Reportbias \cite{Chan_2008} \cite{Dickersin_2011}, Cargo Cult Science \cite{Feynman_1974} und die Einschränkung des Zugriffs auf wissenschaftliche Informationen \cite{Hess_2006} geprägt ist. In diesem Zusammenhang etablierte sich die Besorgnis, dass es durch den enormen Publikationsdruck zunehmend wahrscheinlicher wurde, dass viele der veröffentlichten Forschungsergebnisse "eher falsch als richtig sind" \cite{Ioannidis_2005}. Das aktuelle System steht dem Bestreben, dass es der Wissenschaft im Kern um Erkenntnisse und die uneingeschränkte Zurverfügungstellung der Erkenntnisse geht \cite{hanekop_2006}, entgegen \cite{offhaus_2012_institutionelle_repos}. Weil jegliche Beschränkung des Zugangs zu Wissen auch die Erstellung neuen Wissens behindert \cite{cite:5} \cite{cite:8} \cite{Luhmann1998}, entwickelte sich ein zunehmend unhaltbarer "Schwebezustand" \cite{suchen}. 

Die Suche nach einem Ausweg aus der Publikations- und Kommunikationskrise führte zu den anhaltenden Forderungen nach der Öffnung von Wissen und nach Alternativen für das geschlossene wissenschaftliche Publikations- und Kommunikationssystem. Die wissenschaftliche Kommunikation, die zwar "nach innen" (wissenschaftsintern) einen gewissen Grad an Offenheit bietet, aber "nach außen geschlossen ist" \cite{kelty_2004}, hat sich in den letzten 400 Jahren nur marginal verändert. Nach der erstmals vor über 20 Jahren artikulierten Forderungen nach Öffnung dieser geschlossenen Form der Kommunikation befinden wir uns derzeit inmitten eines "radikalen Wandels" \cite{poynder_2011_suber} tradierter Kommunikationssysteme. Infolge der neuen Möglichkeiten im Rahmen der Digitalisierung und Globalisierung \cite{mcluhan_1963_gutenberg} eröffnet sich erstmal die Chance für eine umfassende "Beschleunigung des Wissensumschlages" \cite{Wenzel_2003} und die Hoffung, dass offene Innovation und offene wissenschaftliche Kommunikation den privaten und staatlichen Forschungsbereich effizienter machen, sowie den industriellen Fortschritt beschleunigen \cite{cite:7}. 

Ungeachtet dessen ist allerdings unübersehbar, dass derzeit trotz des Wandels "das etablierte Publikationssystem der Verlage auch nach zwei Jahrzehnten weitgehend stabil" \cite{Hanekop_2014} geblieben ist und im aktuellen Steuerungssystem der Wissenschaft weiterhin anhand der tradierten wissenschaftlichen Bewertungssysteme Reputation, Mittel und Stellen verteilt werden \cite{cite:4}. Die analog gedruckten und bewährten Journale, sowie andere Publikationsformen der großen wissenschaftlichen Verlage werden mit nahezu unverändertem Geschäftsmodell digital verbreitet \cite{Hanekop_2014} \cite{boai_2012} und in der Wissenschaft rezipiert \cite{suchen}. 

Es ergibt sich die Relevanz und Notwendigkeit, diese Entwicklung genauer zu untersuchen, sowie den Erkentnisse über die Öffnung wissenschaftlicher Kommunikation aus der Literatur empirisch erhobene Daten gegenüberzustellen und das Ergebnis zu diskutieren.

\section{Zielsetzung der Arbeit} 

In dieser Arbeit wird untersucht, welche Auswirkungen der digitale Wandel und die Forderung nach Öffnung der Wissenschaft, beziehungsweise der formellen wissenschafltichen Kommunikation, auf Universitäten, wissenschaftliche Einrichtungen aber auch auf den einzelnen Wissenschaftler hat. Von besonderem Interesse in diesem Zusammenhang sind die Unterschiede von reinem Zugang zu Wissen auf der einen und dem kompletten Zugriff auf den wissenschaftlichen Prozess auf der anderen Seite, sowie das Zusammenspiel unterschiedlicher Prozesse der Wissensverbreitung vor dem Hintergrund der geschichtlichen Entwicklung. Es wird betrachtet, ob es sich bei der Öffnung von Wissenschaft im Rahmen von Open Access und Open Science tatsächlich um einen Wandel in der wissenschaftlichen Kommunikation handelt. Im weiteren Fokus der Untersuchung stehen die Herausforderungen der Netzkultur, das Wissen frei(er) zugänglich zu machen und der Umstand, dass die meisten wissenschaftlichen Informationen der Allgemeinheit bisher nicht zugänglich sind \cite{cite:6}. Diese Thematik wird in Bezug zu der Massifizierung und Neoliberalisierung der Universität gesetzt und in einen historischen Kontext gestellt. Argumente für und gegen die Öffnung wissenschaftlicher Kommunikation aus Sicht der am wissenschaftlichen Kommunikationssystem Beteiligten werden evaluiert, um zu einem vertieften Verständnis der Definitionen von Open Access und Open Science im Kontext wissenschaftlicher Reputation zu gelangen, die aktuelle wissenschaftliche Debatte darzustellen, Treiber und Bremser für die Öffnung von Wissenschaft und Forschung zu identifizieren sowie die Erfahrungen aus dem Selbstversuch als Handlungsempfehlungen zu kommunizieren. --- TODO mit GA abklären --- 

Die vorläufige forschungsleitende Hypothese in dieser Arbeit ist, dass die Öffnung des Zugangs zu wissenschaftlichen Erkenntnissen (Open Access) sich in der Übergangsphase zum Öffnung des Zugriffs auf wissenschaftliche Informationen (Open Science) befindet. Die sich daraus ableitende Fragestellung umfasst zum einen die theoretische Bedeutung von Offenheit im Rahmen der wissenschaftliche Kommunikation, aber auch die empirische Frage nach den Motiven und Beweggründen für Wissenschaftler, diese Offenheit in den unterschiedlichen Disziplinen zu ermöglichen oder zu verhindern. Abschließend wird erörtert, welche mögliche Auswirkungen durch diesen Prozess der Öffnung auf das Selbstverständnis der Wissenschaft, auf die wissenschaftliche Reputation sowie auf die unterschiedlichen Disziplinen zu erwarten sind. Dafür werden relevante Wege des Wissenstransfers ermittelt, Probleme und Hemmnisse bei der offenen Durchführung von wissenschaftlicher Arbeit herausgearbeitet und Handlungsmöglichkeiten am Beispiel der Erstellung von Doktorarbeiten erschlossen und diskutiert.

\section{Aufbau der Arbeit} 

Die Arbeit ist in acht Abschnitte unterteilt. In der Einleitung wird eine Einführung in die Thematik der Arbeit vorgenommen und die Relevanz des Themas herausgestellt. Im Teil "Grundlagen, Definitionen und Abgrenzung" werden intial Open Access, Open Science und wissenschaftliche Reputation beschrieben und abgegrenzt. Die zur Beantwortung der Forschungsfragen angewandten Methoden und das Vorgehen werden im dritten Kapitel beschrieben. Im daruffolgende Kapitel "Literaturanalyse" werden die aktuellen Debatten um die Begriffe, sowie um die Thematik der Arbeit mittels der Methode der Inhaltsanalyse erarbeitet, Defizite identifiziert und darauf aufbauend Vorraussetzungen für die empirische Befragung geschaffen. Im nächsten Abschnitt folgt die empirische Untersuchung zur Identifikation von Treibern und Bremsern für die Öffnung von Wissenschaft und Foschung, sowie die Einbettung eines Selbstexperiments im Rahmen der offenen Erstellung der vorliegenden Arbeit. Alle Teile folgen der forschungsleitende Hypothese, dass sich Open Access in einer Übergangsphasen von der reinen offenen Bereitstellung wissenschaftlicher Publikationen zum vollendeten Kommunikationsakt des "Wissenszuwachs" \cite{Luhmann1998} an die Gesamtgesellschaft befindet. Ausgehend von den Fragestellungen wird dazu ein transdisziplinärer Zugang zur wissenschaftlichen Bearbeitung gewählt, der translational von den Kulturwissenschaften über die Wirtschaftswissenschaften bis hin zu den Medienwissenschaften reicht und sich empirischer, analytischer, sowie auch experimenteller Methoden bedient. In den letzten beiden Kapiteln werden die gewonnen Ergebnisse vorgestellt und die Arbeit kritisch diskutiert. Auf Grundlage der Forschungsergebnisse und eignener Erfahrungen werden Empfehlungen zum Schreiben offener Promotionen, sowie ergänzend ein Ausblick auf die weitere Entwicklung offener Strukturen im Rahmen wissenschaftlicher Tätigkeit gewagt.
