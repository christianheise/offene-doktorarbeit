\chapter{Einleitung}
Unbeschränkte und offene Kommunikation, sowie die Kenntnis des gegenwärtigen Wissensstandes, ist für wissenschaftliche Forschung, deren Aufgabe neues Wissen zu produzieren und damit dem gesellschaftlichen Auftrag des Wissenschaftssystems gerecht zu werden, unverzichtbar \cite{Hanekop_2014} \cite{glaeser2006} \cite{gibbons_1994} \cite{Luhmann1998}. Ob der aktuelle Zustand des wissenschaftlichen Kommunikationssystems diesem Auftrag noch gerecht wird ist fraglich \cite{Schekman_2013}. Mit der Etablierung des Internets als neuem Kanal für die Kommunikation und den Austausch von Informationen wurden große Erwartungen für völlig neue Möglichkeiten des Wissenstransfers und der wissenschaftlichen Kommunikation geweckt \cite{Hanekop_2014} \cite{schulze_2013_open} \cite{albert_2006_open_implications} \cite{Goodrum_2001} \cite{Lawrence_1999}.

\section{Relevanz des Themas}

Im Rahmen dieses postulierten Wandels stehen das Universitätsystem, sowie andere Bildungseinrichtungen und Bibliotheken vor bedeutenden Herausforderungen \cite{Harter2006} \cite{Gu_don_2004} \cite{osterloh2008anreize} \cite{Beverungen_2014}. Die Anspruchsvollsten sind in diesem Zusammenhang die Wahrung der Freiheit von Wissenschaft und Forschung bei möglichst uneingeschränkter Verbreitung wissenschaftlicher Erkenntnisse auf der einen, sowie die Forderung nach besseren Steuerungs- und Leistungsprozessen \cite{Adler_2009} \cite{gibbons_1994} auf der anderen Seite.  Die “gestörten Gleichgewichte im wissenschaftlichen Publikationssystem” \cite{cite:0} und das kaputte wissenschaftliche Reputationssystem \cite{suchen} sind in diesem Zusammenhang von besonderer Bedeutung.

Die Institution Universität läuft im Kontext dieser Entwicklung Gefahr, ihre Bedeutung als exklusiver Ort der Wissensproduktion und -evaluation weiter zu verlieren \cite{suchen}. Mit der Privatisierung der Verarbeitung, Speicherung und Übertragung von Wissen hörten Universitäten auf, selber Bücher zu verlegen \cite{cite:0}. Die Wirtschaft fordert (zunehmend) eine öffentliche Finanzierung der Wissensproduktion und erwartet gleichzeitig die privat(-wirtschaftliche) Aneignung und Nutzung des produzierten Wissens \cite{cite:2}. Das Prinzip der kollektiven Wissensproduktion, bei dem die Wirtschaft unentgeltlich an wissenschaftliche Informationen gelangt, wird auch von Verlagen für ihre Wertschöpfung genutzt. Neben dem entgeldlichen Vertrieb der wissenschaftlichen Informationen ermöglichen sie den Autoren, durch den "Rückgriff auf informal konstituierte Reputationen" \cite{luhmann_1970_selbststeuerung} und die vorangegangene Privatisierung des wissenschafltichen Reputationssystems \cite{suchen}, als Gegenleistung die Chance auf Anerkennung von der wissenschaftlichen Community und Reputation im wissenschaftlichen System \cite{cite:21a}.

Diese Entwicklungen führten zu einer wissenschaftlichen Publikations- und Kommunikationskrise. Sie ist durch den wachsenden Kostendruck, Preissteigerungen, Publikations- \cite{Egger_1997} \cite{Fanelli_2012} \cite{Beverungen_2012} und Reportbias \cite{Chan_2008} \cite{Dickersin_2011}, Cargo Cult Science \cite{Feynman_1974} und die Einschränkung des Zugriffs auf wissenschaftliche Informationen \cite{Hess_2006} gekennzeichnet. In diesem Zusammenhang entstand unter den Wissenschaftlern und Wissenschaftlerinnen die Besorgnis, dass es durch den enormen Publikationsdruck zunehmend wahrscheinlicher wurde, dass viele der veröffentlichten Forschungsergebnisse "eher falsch als richtig sind" \cite{Ioannidis_2005}. Das aktuelle System steht dem Bestreben, dass es der Wissenschaft im Kern um Erkenntnisse und die uneingeschränkte Zurverfügungstellung der Erkenntnisse geht \cite{hanekop_2006}, entgegen \cite{offhaus_2012_institutionelle_repos}. Weil jegliche Beschränkung des Zugangs zu Wissen auch die Erstellung neuen Wissens behindert \cite{cite:5} \cite{cite:8} \cite{Luhmann1998}, entwickelte sich ein zunehmend unhaltbarer "Schwebezustand" --- Todo: was für ein Schwebezustand --- \cite{suchen}.

Die Suche nach einem Ausweg aus der Publikations- und Kommunikationskrise führte zu anhaltenden Forderungen nach der Öffnung von Wissen und nach Alternativen für das geschlossene wissenschaftliche Publikations- und Kommunikationssystem. Die wissenschaftliche Kommunikation, die zwar "nach innen" (wissenschaftsintern) einen gewissen Grad an Offenheit bietet, aber "nach außen geschlossen ist" \cite{kelty_2004}, hat sich in den letzten 400 Jahren nur marginal verändert.

Seit der erstmals vor über 20 Jahren artikulierten Forderungen nach Öffnung dieser geschlossenen Form der Kommunikation befinden wir uns infolge der neuen Möglichkeiten durch die Digitalisierung und Globalisierung \cite{mcluhan_1963_gutenberg} inmitten eines "radikalen Wandels" \cite{poynder_2011_suber} tradierter Kommunikationssysteme. Dieser Wandel bietet die Chance für eine umfassende "Beschleunigung des Wissensumschlages" \cite{Wenzel_2003} und führt potenziell dazu, dass offene Innovation und offene wissenschaftliche Kommunikation den privaten und staatlichen Forschungsbereich effizienter machen, sowie den gesamtgesellschaftlichen Fortschritt in bisher unbekannter Weise beschleunigen \cite{cite:7}.

Ungeachtet dessen ist unübersehbar, dass derzeit "das etablierte Publikationssystem der Verlage auch nach zwei Jahrzehnten weitgehend stabil" \cite{Hanekop_2014} geblieben ist und im aktuellen Steuerungssystem der Wissenschaft weiterhin anhand der tradierten wissenschaftlichen Bewertungssysteme Reputation, Mittel und Stellen verteilt werden \cite{cite:4}. Die analog gedruckten und bewährten Journale, sowie andere Publikationsformen der großen wissenschaftlichen Verlage werden mit nahezu unverändertem Geschäftsmodell nur zusätzlich digital verbreitet \cite{Hanekop_2014} \cite{boai_2012} und in der Wissenschaft rezipiert \cite{suchen}.

Daraus ergibt sich die Relevanz und Notwendigkeit, diese Entwicklungen genauer zu untersuchen, den Erkenntnissen über die Öffnung wissenschaftlicher Kommunikation aus der Literatur empirisch erhobene Daten gegenüberzustellen und das Ergebnis zu diskutieren.

\section{Zielsetzung der Arbeit}

In dieser Arbeit wird untersucht, welche Auswirkungen der digitale Wandel und die Forderung nach Öffnung der Wissenschaft, beziehungsweise der wissenschaftlichen Kommunikation, auf Universitäten, wissenschaftliche Einrichtungen aber auch auf den einzelnen Wissenschaftler haben. Von besonderem Interesse sind in diesem Zusammenhang die Unterschiede von reinem Zugang zu Wissen auf der einen und dem kompletten Zugriff auf den wissenschaftlichen Prozess auf der anderen Seite, sowie das Zusammenspiel unterschiedlicher Formen der Wissensverbreitung vor dem Hintergrund der geschichtlichen Entwicklung. Es wird betrachtet, ob es sich bei der Öffnung von Wissenschaft im Rahmen von Open Access und Open Science tatsächlich um einen Wandel in der wissenschaftlichen Kommunikation handelt. Im weiteren Fokus der Untersuchung stehen die Herausforderungen der Netzkultur, das Wissen frei(er) zugänglich zu machen und der Umstand, dass die meisten wissenschaftlichen Informationen der Allgemeinheit bisher nicht zugänglich sind \cite{cite:6}.

Diese Thematik wird in Bezug zu der Massifizierung und Neoliberalisierung der Universität gesetzt und in einen historischen Kontext gestellt. Argumente für und gegen die Öffnung wissenschaftlicher Kommunikation aus Sicht der am wissenschaftlichen Kommunikationssystem Beteiligten werden evaluiert, um zu einem vertieften Verständnis der Definitionen von Open Access und Open Science im Kontext wissenschaftlicher Reputation zu gelangen. Die aktuelle wissenschaftliche Debatte dargestellt, Treiber und Bremser für die Öffnung von Wissenschaft und Forschung identifiziert sowie die Erfahrungen aus einem Selbstversuch als Handlungsempfehlungen für das offene Bearbeiten wissenschaftlicher Fragestellungen kommuniziert.

Die forschungsleitende Hypothese dieser Arbeit ist, dass sich die Öffnung des Zugangs zu wissenschaftlichen Erkenntnissen (Open Access) in der Übergangsphase zur Öffnung des Zugriffs auf den gesamten wissenschaftlichen Wertschöpfungsprozess (Open Science) befindet. Die sich daraus ableitenden Fragestellungen umfassen zum einen die theoretische Bedeutung von Offenheit im Rahmen der wissenschaftliche Kommunikation, zum anderen die empirische Frage nach den Motiven und Beweggründen für Wissenschaftler der unterschiedlichen Disziplinen, das aktuellen wissenschaftliche Kommunikationssystem bewahren zu wollen oder die Forderung hin zu Offenheit zu unterstützen. Weiterhin wird beispielhaft erarbeitet, welchen Hürden und welcher Aufwand durch die Öffnung der formellen Kommunikation für Wissenschaftler und Wissenschaftlerinnen entsteht. Abschließend wird erörtert, welche möglichen Auswirkungen durch diesen Prozess der Öffnung auf das Selbstverständnis der Wissenschaft an sich und auf die wissenschaftliche Reputation in den unterschiedlichen Disziplinen zu erwarten sind. Dafür werden relevante Wege des Wissenstransfers ermittelt, Probleme und Hemmnisse bei der offenen Durchführung von wissenschaftlicher Arbeit herausgearbeitet und Handlungsmöglichkeiten am Beispiel der Erstellung von Doktorarbeiten erschlossen und diskutiert.

\section{Aufbau der Arbeit}

Die Arbeit ist in acht Abschnitte unterteilt. In der Einleitung wird eine Einführung in die Thematik der Arbeit vorgenommen und die Relevanz des Themas dargestellt. Im Teil "Grundlagen, Definitionen und Abgrenzungen" werden initial die Termini Open Access, Open Science und wissenschaftliche Reputation beschrieben, abgegrenzt und historisch zugeordnet.

Die zur Beantwortung der Forschungsfragen angewandten Methoden werden im dritten Kapitel beschrieben. Im darauffolgenden Kapitel "Inhaltsanalyse" werden die aktuellen Debatten um die Begriffe "Open Access" und "Open Science", sowie die Thematik der Arbeit dargestellt, Defizite identifiziert und darauf aufbauend Voraussetzungen für die empirische Befragung geschaffen.

Im darauffolgenden Abschnitt wird eine empirische Untersuchung zur Identifikation von Treibern und Bremsern für die Öffnung von Wissenschaft und Forschung mittels einer Online-Befragung durchgeführt und ausgewertet.

Die konstant offene Erstellung und Dokumentation der vorliegenden Arbeit im Sinne eines prospektiven Selbstexperimentes erweitert den empirischen Ansatz.

Alle Teile folgen der forschungsleitende Hypothese, dass sich Open Access in einer Übergangsphase von der reinen offenen Bereitstellung wissenschaftlicher Publikationen zum vollendeten Kommunikationsakt des "Wissenszuwachs" \cite{Luhmann1998} an die Gesamtgesellschaft befindet. Ausgehend von den Fragestellungen wird dazu ein transdisziplinärer Zugang zur wissenschaftlichen Bearbeitung gewählt, der translational von den Kulturwissenschaften über die Wirtschaftswissenschaften bis hin zu den Medienwissenschaften reicht und sich empirischer, analytischer, sowie auch experimenteller Methoden bedient.

In den letzten beiden Kapiteln werden die gewonnen Ergebnisse und die Vorgehensweise in dieser Arbeit zusammengefasst und kritisch diskutiert. Auf Grundlage der Forschungsergebnisse und der eigenen Erfahrungen werden Empfehlungen zum Schreiben offener wissenschaftlicher Arbeiten, sowie ergänzend ein Ausblick auf die weitere Entwicklung offener Strukturen im Rahmen wissenschaftlicher Tätigkeit formuliert.

\section{Beweggründe und eigene Position}

--- weiter ausarbeiten: Debatte konkretisieren, selbst positionieren und  ----
