\chapter{Empirische Untersuchung: Die Öffnung von Wissenschaft und Forschung aus der Perspektive von Wissenschaftlern}

Ziel der Arbeit ist es, die erarbeiteten theoretischen Grundlagen im Rahmen einer Umfrage, sowie die Ausprägungen von Open Access und Open Science vor dem Hintergrund von wissenschaftlicher Reputation und über die Grenzen einzelner Fachdisziplinen hinaus zu püfen. Besondere Berücksichtigung findet die Identifikation der Treiber und Bremser für die Öffnung von wissenschaftlicher Informationen und Prozesse. Dafür werden die aus der theoretischen Betrachtung analysierten Konzepte Open Access und Open Science einer Untersuchung zugeordnet. Die Befragung basiert aus Gründen der Vergleichbarkeit und Übrerprüfung teilweise auf Fragen der Studie des Soziologischen Forschungsinstituts Göttingen (SOFI) "Wissenschaftliche Publikationen im Internet: Wissenschaftler als Leser und Autoren" aus dem Jahr 2007.

Die zentralen Forschungsfragen dieser Arbeit rückten in diesem Arbeitsschritt der Fragebogenerstellung in den Fokus und stellten die Grundlage für die Entwicklung des Fragepools dar. Die Formulierung der Fragen basierte, sofern nicht aus der Vorbefragung von SOFI unverändert übernommen, auf Handlungsmustern, Meinungen und Einstellungen zu folgenden Fragestellungen:
\begin{itemize}
\item Wie verändert die Digitalisierung, wie wir auf wissenschaftliche Daten und Informationen zugreifen?
\item In welchem Umfang besteht Wissen über die Öffnung von Wissenschaft unter den Wissenschaftlern und Wissenschaftlerinnen? 
\item Welches Verständnis von Open Access besteht unter den Befragten? 
\item Wie stark ist das Interesse an Forschungsdaten? 
\item Welche Faktoren und Argumente begünstigen die Öffnung von Wissenschaft in der jeweiligen wissenschaftlichen Disziplin? 
\item Welche Faktoren und Argumente sprechen gegen die Öffnung von Wissenschaft in einer wissenschaftlichen Disziplin? 
\item Wie wird der geschätzte Aufwand für die Öffnung von Wissenschaft in einer wissenschaftlichen Disziplin eingeschätzt?
\item Welche weiteren extrinsischen Faktoren unterstützen die Verbreitung von Offenheit in Wissenschaft und Forschung? 
\item Welche unterschiedlichen Auffassungen bestehen zwischen den unterschiedlichen Fachdiziplinen, Alters- und Statusgruppen?
\item In welchem Umfang wird bereits heute im wissenschaftlichem Umfeld offen kommuniziert?
\item Welche Veränderungen beim Zugang zur Literatur wie auch bei den Veröffentlichungsstrategie sind im Vergleich zur der 2007 und 2008 durchgeführten Befragung des SOFI Göttingen zu erkennen?
\end{itemize}

\section{Erhebungsmethode und Messinstrumente}

Auf Grund der zunehmenden Verbreitung und Nutzung des Internets, hat die Online-Befragung längst Eingang in die empirische Sozialforschung gefunden \cite{Pannewitz_2002}. Auschlaggebend für die Auswahl dieser Befragungsform war vor allem der ökonomische Aspekt, "die es einfach macht, große Stichproben in kurzer Zeit zu erheben" \cite{eichhorn_2004_online}. Darüber hinaus wurde in der Vorbefragung durch das SOFI ebenfalls auf das Internet als primäre Quelle für die Identifikation von Teilnehmern und Teilnehmerinnen und E-Mail als Kontaktaufnahmekanal zurückgegriffen.

\subsection{Untersuchungsobjekte}

Die Teilnehmer des Fragebogens waren primär Wissenschaftler und Wissenschaftlerinnen aus sämtlichen Fachdiziplinen oder Mitarbeiter des wissenschaftlichen Betriebs aus dem deutschsprachigen Raum die im Zeitraum vom 18.8.2014 bis 18.01.2015 online befragt wurden. Bibliothekare und Bibliotherinnen (0,95 Prozent der Befragten), sowie Studierende (3,68 Prozent Befragten) wurden zwar nicht direkt angesprochen, konnten aber dennoch an der Umfrage teilnehmen. Während der Befragung wurden dazu 4002 Wissenschaftlerinnen und Wissenschaftler per E-Mail im Zeitraum vom 18.08.2014 bis 18.01.2015 angeschrieben. Sie wurden über die Webseiten der Forschungseinrichtungen identifiziert und per Email um Teilnahme gebeten. Es wurden hauptsächlich Personen angeschrieben, deren direkte E-Mail Adresse auf der Webseite der wissenschaftlichen Einrichtungen angegeben war. Vereinzelt wurden auch Sekretariatsadressen verwendet und um Weiterleitung gebeten. Alle identifizierten Kontakte wurden nur einmal angeschrieben.  Des Weiteren wurde der Umfragelink mit einer kurzen Information zur Umfrage auf offene-doktorarbeit.de veröffentlicht sowie über die privaten Social-Media Kanäle und an persönliche Kontakte des Autors versendet.

Die Auswahl der jeweiligen Fächer beruht auf der aktuellen Auflistung der Fachsystematik der Deutschen Forschungsgemeinschaft (DFG). Da die Erhebung fächerübergreifend angelegt war, um die Unterschiede zwischen den Disziplinen zu evaluieren, wurden Wissenschaftler aus alle dort gelisteten Fachdiziplinen angefragt. Per Zufall wurdnen dazu von den Institutswebseiten im deutschsprachigen Raum pro Fach 150 Wissenschaftler und Wissenschaftlerinnen per E-Mail angeschrieben und um Teilnahme an der Befragung gebeten. 1.768 der Angefragten haben an der Umfrage teilgenommen und den Fragebogen gestartet, 1.467 haben mindestens eine Frage beantwortet und somit zumindest teilweise an der Befragung teilgenommen. 301 Personen haben vor Beantwortung der ersten Fragegruppe abgebrochen. Die Rücklaufquote liegt somit bei 44,18 Prozent brutto beziehungsweise bei 36,67 Prozent netto. 1.112 der 1.768 Teilnehmer und Teilnehmerinnen (62,89 Prozent) haben den Online Fragebogen vollständig beendet. Nach Beendigung des Umfragezeitraums haben somit 656 (37,10 Prozent) den Online-Fragebogen vor Beendigung abgebrochen. 

Verschiedenen Verzerrungen sind dahergehend zu vermute, da die kontaktierten Menschen ausschließlich online kontaktiert wurden. Da die Umfrage jedoch öffentlich online stattfand, konnte jeder teilnehmen.

\subsection{Untersuchungsmaterial}

Für die Online-Befragung wurde die Open Source Software LimeSurvey Version 2.05+ Build 140811 verwendet, die auf dem Webserver des Centre for Digital Cultures installiert wurde. Die Darstellung der Befragung wurde so angepasst, dass die Darstellung und die Bewantwortung des Fragebogens auch auf Mobiltelefonen möglich war. Des Weiteren wurde bei dem Design des Fragebogens und der Anpassung der Dartstellung der Software explizit darauf geachtet, dass alle Texte einfach und angenehm lesbar waren und die Beantwortung der Fragen einfach und strukturiert ablaufen konnte.

Die Ergebnisse wurden in der Datenbank des Servers des Centres for Digital Cultures zwischengespeichert und am xx.xx.2015 gelöscht. Nach Abschluss der Befragung wurden die Datensätze anonymisiert. Dazu wurden sämtliche persönliche Daten, wie zum Beispiel E-Mailadressen entfernt und die freiwillige personenbezogene Angabe von dem Rest der Daten getrennt. Folgende Felder wurden entfernt beziehungsweise getrennt, neu angeordnet und aggregiert veröffentlicht: Geschlecht, Alter, weitere Aspekte zum Thema, Anmerkungen und Kritik, Funktion im Rahmen eines Open Access Engagements (optional), Antwort ID und Zeitpunkt der Beantwortung. Die anonymisierte Datensätze wurden nach Abschluss der Befragung im Januar 2015 auf dem Datenrepositorium zenodo.org, auf dem Forscher Daten und Publikationen einstellen können, veröffentlicht.

\subsection{Aufbau des Fragebogens und Untersuchungsdurchführung}

In der Befragung durch das SOFI wurden 6500 Wissenschaftler und Wissenschaftlerinnen befragt, von denen 1803 geantwortet haben. Der 2007 verwendete Fragebogen bestand aus 51 Fragen. Im ersten Teil des Fragebogens wurden Fragen zu dem Fachgebiet und Tätigkeitsbereich zunächst als Leserin bzw. Leser wissenschaftlicher Publikationen erfasst. Im zweiten Teil wurden die Teilnehmer aus der Perspektive als Autorin beziehungsweise als Autor befragt. Abschließend wurden noch eineige personenbezogene Angaben abgefragt. \cite{Hanekop_Wittke_2007_Fragebogen} Die Skalen zur Beantwortung der Fragen waren unterschiedlich ausgewählt. 

Zu Beginn der Fragebogenkonstruktion wurde der Fragebogen und das Datenmaterial der Vorbefragung einer Itemanalyse zum Ausschluss unpassender Fragen (Items) unterzogen und Fragen bezglich der Fragestellung dieser Arbeit hinzugefügt. Dafür wurden die veröffentlichten Antworten analysiert. Fragen, die stark ungleich verteilt waren, wurden , wenn sie nicht inhaltlich interessant erschienen, ausgeschlossen. Somit wurden auf der Basis der Analyse der Fragen der Fragepool auf 31 Fragen reduziert beziehungsweise verändert. Acht der insgesamt 40 Fragen bedingen Antworten aus vorherigen Fragen und wurden deshalb nicht allen Teilnehmern gestellt. Die Reihenfolge der Fragen und der Fragengruppen wurden so ausgewählt, dass sie zueinander passen, der Reihenfolge-Effekt minimiert wird und die Beantwortung bis zum Ende interessant bleibt. Da 75 Prozent der Befragten, die mindestens eine Frage beantwortet haben, auch den Fragebogen vollständig beantwortet haben, verdeutlicht den Erfolg der Vorbereitung und Anpassung der Daten. 

Die Gliederung war ebenfalls an die Befragung aus den Jahren 2007 angelehnt und beschränkte sich in der ersten Gruppe auf die Rahmenbedingungen der Teilnehmenden sowie deren wissenschaftlichen Tätigkeit. In der zweiten Fragegruppe wurden Aspekten aus der wissenschaftlichen Leserperspektive evaluiert. Die dritte Fragegruppe beschäftigte sich mit dem Zugang zu wissenschaftliche Informationen, gefolgt von der vierten, die aus Fragen bezüglich des Zugangs zu wissenschaftlichen Informationen und des Zugriffs auf wissenschaftliche Infromationen bestand. In der fünften Fragegruppe wurden Fragen aus der Perspektive des Autors und der Autorin von wissenschaftlichen Inhalten gestellt. Abschließend wurden weitere personenbezogene Daten zur eindeutigen Segmentierung erhoben. Die Befragten wurden vor Start der Befragung auf die Gleiderung des Fragebogens und die Reihenfolge der Fragegruppen hingewiesen.

Der Fragebogen "Wissenschaftliche Kommunikation im Rahmen der Digitalisierung" wurde nach Vorbereitung am 18.08.2014 unter http://umfrage.offene-doktorarbeit.de veröffentlicht.

\section{Auswertung der Umfrage}

In die Auswertung der Befragung gingen die Angaben der 1.467 Teilnehmer des Online-Fragebogens ein, die mindestens eine Frage beantwortet haben.


\section{statistische Ergebnisse der Umfrage}



\subsection{Weitere Ergebnisse der Umfrage}

Wie in der Befragung vom SOFI 2008 wurden die Teilnehmer und Teilnehmerinnen gefragt, wie sie sich in ihrem Fachgebiet auf dem Laufendenhalten und welchen Zugang sie zuden (Voll-) Texten haben \cite{hanekop_2008}. Auch die Antworten der 1.446 der Befragten zeigen "wie weitreichend sich bei der gezielten Suche nach Literatur digitale Suchmöglichkeiten durchgesetzt haben" \cite{hanekop_2008}. Fast 50 Prozent der Befragten in 2014 gaben an, die Google Suche häufig als Suchmöglichkeiten zu nutzen, um gezielt nach Literatur zu suchen. Bei der Befragung 2007 gaben 46 Prozent der Befragten an sehr häufig die Google Suche zu verwednen. Diese Entwicklung liegt im Trend, denn die IT-gestütze Suche lag in den 1980er Jahren bei einem Prozent, stieg bis 1993 auf neun Prozent an und betrug im Jahr 2003 bereits 24 Prozent \cite{hanekop_2008}.


Ein ähnliches Bild zeigt sich bei dem Vergleich der Umfrageergebnisse bei der Frage wie sich die Teilnehmer in Ihrem Fachgebiet auf dem Laufenden halten. 2007 gaben 57 Prozent an, sich sehr häufig über Online-Zeitschriften auf dem aktuellen Stand der wissenschaftlichen Debatte zu halten. In der Befragung im Rahmen dieser Arbeit gaben 67,71 Prozent der 1.446 Befragten an sich in Online-Zeitrschriften zu informieren. Gefolgt wird diese Option durch die Teilnahme an Tagungen oder Kongressen (56,05 Prozent) und Gespräche mit Fachkollegen (55.09 Prozent). Social Media Platformen spielen mit bisher knapp 6 Prozent eher eine kleinere Rolle. Online-Datenbanken, Online-Archive, die 2007 noch zweithäufigste Option sich auf dem Laufenden zu halten, bleibt annähernd auf dem gleichen Niveau.

Im Jahr 2007 fanden rund 81 Prozent der Befragten die Forderung nach kostenfreiem Zugang zu allen wissenschaftlichen Publikationen für Leser gut bis sehr gut. In der Befragung 2014 fiel das Ergebnis mit einer Befragtenzahl von 1.154 mit 75,95 Prozent zwar niedriger aber dennoch weiterhin überwältigend positiv aus. Knapp 20 Prozent waren sich bei der Frage unsicher und 38 der Befragen lehnten die Forderung nach Open Access ab, 9 davon sogar "entschieden". Unter den Befragten gaben 14,75 Prozent an in der Open Access Bewegung engagiert zu sein. 72,30 Prozent verneinten die Aussage und 12,96 Prozent machten keine Angabe.

\section{Kritische Betrachtung und Beurteilungsfehler}

Immer wieder kommt es bei dem Prozeß der Erstellung von Fragebögen oder bei der Beurteilung der Daten zu Störungen, zu sogenannten Beurteilungsfehlern. Deshalb soll die Güte der Befragung durch
die Gütekriterien Objektivität, Reliabilität und Validität beurteilt werden.
