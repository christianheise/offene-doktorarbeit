\chapter{Empirische Untersuchung}

\section{Aufbau des Fragebogens}
Damls wurden 6500 Wissenschaftler und Wissenschaftlerinnen befragt, von denen 1803 geantwortet haben. Der 2007 verwendete Fragebogen bestand aus 51 Fragen. Im ersten Teil des Fragebogens wurden Fragen zu dem Fachgebiet und Tätigkeitsbereich zunächst als Leserin bzw. Leser wissenschaftlicher Publikationen erfasst. Im zweiten Teil wurden die Teilnehmer aus der Perspektive als Autorin beziehungsweise als Autor befragt. Abschließend wurden noch eineige personenbezogene Angaben abgefragt. \cite{Hanekop_Wittke_2007_Fragebogen} Die Skalen zur Beantwortung der Fragen waren unterschiedlich ausgewählt. 

Zu Beginn der Fragebogenkonstruktion wurde der Fragebogen und das Datenmaterial der Vorbefragung einer Itemanalyse zum Ausschluss unpassender Fragen (Items) unterzogen und Fragen bezglich der Fragestellung dieser Arbeit hinzugefügt. Dafür wurden die veröffentlichten Antworten analysiert. Fragen, die stark ungleich verteilt waren, wurden , wenn sie nicht inhaltlich interessant erschienen, ausgeschlossen.  Somit wurden auf der Basis der Analyse der Fragen der Fragepool auf 31 Fragen reduziert beziehungsweise verändert.

Die zentralen Forschungsfragen dieser Arbeit rückten in diesem Arbeitsschritt der Fragebogenerstellung in den Fokus und stellten die Grundlage für die Entwicklung des Fragepools dar. Die Formulierung der Fragen basierte, sofern nicht aus der Vorbefragung von SOFI unverändert übernommen, auf Handlungsmustern, Meinungen und Einstellungen zu folgenden Fragestellungen:
\begin{itemize}
\item Wie verändert die Digitalisierung, wie wir auf wissenschaftliche Daten und Informationen zugreifen?
\item In welchem Umfang herrscht unter den Wissenschaftlern und Wissenschaftlerinnen Wissen über die Öffnung von Wissenschaft vor? 
\item Welches Verständnis von Open Access besteht unter den Befragten? 
\item Wie stark ist das Interesse an Forschungsdaten? 
\item Welche Faktoren und Argumente begünstigen die Öffnung von Wissenschaft in einer wissenschaftlichen Disziplin? 
\item Welche Faktoren und Argumente sprechen gegen die Öffnung von Wissenschaft in einer wissenschaftlichen Disziplin? 
\item Wie wird der geschätzte Aufwand für die Öffnung von Wissenschaft in einer wissenschaftlichen Disziplin eingeschätzt?
\item Welche weiteren extrinsischen Faktoren unterstützen die Verbreitung von Offenheit in Wissenschaft und Forschung? 
\item Welche unterschiedlichen Auffassung bestehen zwischen den unterschiedlichen Fachdiziplinen, Alters- und Statusgruppen?
\item In welchem Umfang wird bereits heute im wissenschaftlichem Umfeld offen kommuniziert?
\item Welche Veränderungen beim Zugang zur Literatur wie auch bei den Veröffentlichungsstrategie sind im Vergleich zur der 2007 und 2008 durchgeführten Befragung des SOFI Göttingen zu erkennen?
\end{itemize}

Die Gliederung war ebenfalls an die Befragung aus den Jahren 2007 und 2008 angelehnt und beschränkte sich in der ersten Gruppe auf die Rahmenbedingungen der Teilnehmenden sowie deren wissenschaftlichen Tätigkeit. In der zweiten Fragegruppe wurden Aspekten aus der wissenschaftlichen Leserperspektive abgefragt. Die dritte Fragegruppe beschäftigte sich mit Fragen rund um den Zugang zu wissenschaftlichen Informationen, gefolgt von der vierten, die aus Fragen bezüglich des Zugangs zu wissenschaftlichen Informationen und des Zugriffs auf wissenschaftliche Infromationen bestand. In der fünften Fragegruppe wurden Fragen aus der Perspektive des Autors und der Autorin von wissenschaftlichen Inhalten gestellt. Abschließend wurden weitere personenbezogene Daten zur eindeutigen Segmentierung abgefragt. .

\section{Erhebungsmethode und Messinstrumente}

Auf Grund der zunehmenden Verbreitung und Nutzung des Internets, hat die Online-Befragung längst Eingang in die empirische Sozialforschung gefunden \cite{Pannewitz_2002}. Auschlaggebend für die Auswahl dieser Befragungsform ist vor allem die Ökonomie, "die es einfach macht, große Stichproben in kurzer Zeit zu erheben" \cite{eichhorn_2004_online}. Darüber hinaus wurde in der Vorbefragung durch das SOFI ebenfalls auf das Internet als primäre Quelle für die Identifikation von Teilnehmern und Teilnehmerinnen und E-Mail als Kontaktaufnahmekanal zurückgegriffen.