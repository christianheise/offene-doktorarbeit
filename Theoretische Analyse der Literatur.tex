\chapter{Literaturanylse zu Open Access und Open Science als Grundlage für die empirische Forschung}
Das Geschäftsmodell hinter der wissenschaftlichen Kommunikation ermöglicht den Verlegern Betriebsgewinnmargen von über 35 Prozent \cite{russell_2008_business} und hohe jährliche Wachstumsraten \cite{Wellcome_Trust_2003}. Sucht man nach Gründen für die Unterstützung des bisherigen Modells durch die Wissenschaftsgemeinschaft, wird deutlich, dass vor allem die in --- TODO in welchen Kapitel? --- beschriebene wissenschaftliche Reputation einen zentralen, extrinsischen Motivationsfaktor für Wissenschaftler darstellt \cite{minssen_2012_arbeit}. Die akademische Reputation „ist [dabei] die zentrale Kommunikationsform, die das Wissenschaftssystem charakterisiert“ \cite{Rutenfranz_1997}. Die Ergebnisse aus wissenschaftlicher Forschung werden dabei als Publikationen vor allen Mitgliedern der Wissenschaft präsentiert, „um diese intern von der Wissenschaftsgemeinde als wissenschaftlich beziehungsweise unwissenschaftlich zertifizieren zu lassen" \cite{Rutenfranz_1997}. 

Der aktuelle Forschungsstand zur Öffnung von Wissenschaft, zu den Treibern und Bremsern dieser Entwicklung und dem damiteinhergehenden Paradigmenwechsel mit Fokus auf den Themenbereich der wissenschaftlichen Reputation ist von besonderem Interesse. Grundlage dessen sind ausgewählte relevante und aktuelle Werke der Fachdiskussion die sich mit dem Phänomen Öffnung von Wissenschaft beschäftigen. Ziel des Kapitels ist die Entwicklung einer geeigneten wissenschaftlichen Fragestellung.

\section{Beschreibung des Forschungsstands}
Anhand einer Literaturanalyse wird dargestellt, welche Argumentationen es für und gegen, sowie welche Möglichkeiten und Grenzen für die Öffnung der Wissenschaft angeführt werden. Eine kritische Analyse soll dabei Pro- und Kontraargumente zusammenfassen und einen Überblick über die aktuelle Debatte um Open Science und Open Access ermöglichen. Diese Analyse basiert auf der Annahme, dass sich Open Access in einer Übergangsphase von der reinen offenen Bereitstellung wissenschaftlicher Publikationen und dem damit verbundenen offenen Zugang zur Wissenschaft zur umfassenden und offenen Wissensverteilung und dem damit einherdehnenden Zugriff auf Wissenschaft für die Gesamtgesellschaft (Open Science) befindet. Darüber hinaus sollen medienkulturwissenschaftlich Open Science und Open Access in ihren technischen als auch in ihren gesellschaftlichen und politischen Aspekten sowie die kulturellen Auswirkungen der Medienbrüchen im Rahmen von hybridem Publizieren reflektiert werden. Abschließend werden Treiber und Bremser für die Öffnung von Wissenschaft erhoben und in der Gesamtbetrachtung der Arbeit zusammengeführt.

\section{Forschungsfragen} 
Folgende Forschungsfragen sollen bei der Inhalsanalyse genauer analysiert werden:
\begin{itemize}
\item Warum kommt es zu der Bestrebungen hin zur Öffnung von Wissenschaft? 
\item Wie können Open Science und Open Access definiert und voneinander abegrenzt werden? 
\item Welche Pro- und Contraargumente gibt es für die Öffnung von Wissenschaft - ist Offenheit in der Wissenschaft gut oder schlecht? 
\item Wo sind die Grenzen der Öffnung? 
\item Warum ist die Öffnung von Wissen in den verschiedenen wissenschaftlichen Disziplinen unterschiedlich weit verbreitet? 
\item Was bedeutet Offenheit und freier Zugang im Rahmen des wissenschaftlichen Diskurs-, Reputations- und Machtbegriffs?
\end{itemize}	

\section{Erhebungsmethode und Umfang} 

Für die Analyse wurden xxx Quellen ausgewählt und analysiert.

\section{Analyse der Definitionen von Open Access} 
an
\subsection{Offener Zugang zu wissenschaftlicher Kommunikation}

Der bisherige Prozess wissenschaftlicher Kommunikation steht vor großen Herausforderungen. Die Zeitschriften- und Monographienkrise, der zunehmende finanzielle Druck sowie die Veränderungen im wissenschaftlichen Kommunikationsprozess durch neue Arten und Möglichkeiten der Distribution, die steigenden Beschaffungskosten für wissenschaftliche Literatur \cite{cite:17}, sowie die Veränderungen in der Rezeption von Inhalten \cite{holub_2013_reception}, zwingen zum Umdenken in der wissenschaftlichen Kommuinkationspraxis \cite{suchen}. Nachfolgend wird das Modell des Offenen Zugangs zu Wissenschaft erläutert, analysiert und abgerenzt.

Der Schwerpunkt beruht dabei auf den Themenbereichen wissenschaftliche Reputation und (Effizienz der) Kommunikation. Dieser Zugang beruht auf der Annahme, dass Offenheit eine große Chance für dringend notwendige Veränderungen im wissenschaftlichen Qualitäts- und Reputationssystem (siehe Kapitel 2.3) darstellt. Das betrifft vor allem die Aktivität der Wissenschaftler und die Qualtiät der Forschungergebnisse, deren Erkenntnisse bisher häufig erst nach langen intransparenten Verfahren bewertet und publiziert, sowie nur an einen beschränkten Kreis von Rezipienten vermittelt werden. Das hat auch einen signifikanten Einfluss auf Kosten die im Rahmen öffentlich-finanzierter Forschung entstehen \cite{suchen}. Ein besonderer Fokus liegt auf dem generellen Zugang zu wissenschaftlichen Informationen im Rahmen des "klassischen" Kommuniktations- und Publikationsprozess. Im Rahmen von Open Access ist dabei nicht zwingend der Zugriff auf Informationen oder Daten, die bei Erstellung der Publikation entstehen, eingeschlossen. 

Als Grundlage für diese Entwicklung werden vor allem die infrastrukturellen Veränderungen angeführt, die "seit spätestens Mitte der 1990er-Jahre entscheidend Einfluss auch auf die Wissenschaftskommunikation und das wissenschaftliche Arbeiten genommen haben" \cite{schulze_2013_open}. Wissenschaftliche Informationen werden seither nicht nur in "digitaler Form konsumiert, sondern auch kollaborativ und kooperativ, zeitlich versetzt, durch teilweise räumlich weit verstreute Arbeitsgruppen und Forschungsverbünde, genutzt und weiterverarbeitet" \cite{schulze_2013_open}. Die Verbreitung und Akzeptanz von Open Access variiert zwischen den einzelnen wissenschaftlichen Disziplinen erheblich \cite{cite:21a} .


\subsection{Open Access Modelle}

In der Literatur wird Open Access in unterschiedliche Formen unterteilt und es existieren mehrere Definitionen \cite{CREATe_2014} \cite{albert_2006_open_implications}, sowie unterschiedliche Auffassungen über die verschiedenen Modelle von Open Access \cite{CREATe_2014} \cite{cite:22b} \cite{lewis_2012_inevitability}. Trotz der Unterteilung orientieren sich alle Formen an den "three Bs" (siehe Kapitel XXXX), den derzeit geltenden Definitionen von Open Access. Am Beispiel der Budapest Open Access Initiative werden zwei Wege für Open Access dargestellt \cite{albert_2006_open_implications}: 
\begin{enumerate}
\item Einrichtung "einer neuen Generation von Fachzeitschriften," die einen kostenfreien und unmittelbaren Zugang zu den Beiträgen ermöglichen ("goldener" Weg)
\item öffentlich zugängliche (Selbst-)Archivierung durch den Urheber ("grüner" Weg)
\end{enumerate}

Eine zweite Ebene der Einteilung in hybride, radikale und sonstige Formen von Open Access soll allen in der Literatur aufkommenden Formen gerecht werden. Es werden auch die Publikationsformen genannt, die zwar häufig als Open Access bezeichnet werden, die aber nicht den gängigen Deklarationen \cite{boai_2012} und Definitionen gerecht werden. In Bezug auf das gängigste Finanzierungsmodellen von OA, werden bei Open Access-Publikationen in der Regel vom Autor Veröffentlichungsgebühr erhoben und nicht auf Peer-Review verzichtet, um die akademische Reputation zu bewahren \cite{albert_2006_open_implications} \cite{Open_Access_net_2009}.

Der "grüne Weg" beschreibt das Modell, in dem der Autor im Rahmen einer (Selbst-)Archivierung von Beiträgen in Repositorien (öffentlichen Dokumentenservern) anstrebt \cite{suchen}. Das vom Autor inital eingereichte Dokument (Manuskriptfassung) steht dabei als Pre-Print oder Post-Print-Version auf meist institutionellen oder disziplinären Dokumentenservern \cite{suchen} oder privaten Homepages \cite{suchen} jedem zur Verfügung. Im Unterschied zu Post-Prints, hat bei Pre-Print keine Peer Review stattgefunden \cite{suchen} und der Beitrag hat somit keine externe wissenschaftliche Qualitätssicherungsmaßnahme durchlaufen. Beim "grünen Weg" hat der publizierende Verlag darüber hinaus die Möglichkeit innerhalb einer Speerfrist von überlicherweise 6-12 Monaten \cite{suchen} oder länger den lektorierten und fertig-publizierten Beitrag unter einer eigenen Lizenz zu verkaufen \cite{suchen}. Erst nach Ablauf der Frist, wird auch die finale Fassung des Beitrags frei und offen zur Verfügung gestellt. Hier gibt es je nach Verlag und Publikationsform verschiedenen Möglichkeiten der Ausgestaltung des Publikationswegs \cite{suchen}.

Beim "goldene Weg" stellt der Autor die Publikation über einen Verlag unmittelbar nach Fertigstellung frei und offen zur Verfügung. Auch die Verlagsversion muss ohne Sperrfrist in einem Repositorium zur Verfügung gestellt werden. Der Verlag hat allerdings zusätzlich die Möglichkeit den Beitrag kommerziell zu vertreiben und zu verkaufen, muss aber ebenfalls eine freie und offene Version zur Verfügung stellen. Alternativ ermöglicht es der "verzögerte goldenen Open Access" Weg dem Verlag zeitverzögert für die Öffentlichkeit die finale Version unter einer offenen Lizenz zur Verfügung zu stellen \cite{lewis_2012_inevitability}. 

Im Rahmen anderer Modelle, vornehmlich bei der Publikation in Zeitschriften und Monographien, wird den Autoren auch zunehmend die Möglichkeit eingeräumt auch im Nachhinein durch zusätzliche Zahlung die Publikation offen und frei zur Verfügung zu stellen\cite{lewis_2012_inevitability}.

Bei beiden Wegen kann parallel zu der elektronischen Veröffentlichung von Beiträgen und Büchern eine meist kostenpflichtige Publikation in gedruckter Form erfolgen \cite{suchen}.

Der Kernunterschied zwischen diesen beiden Modellen, besteht darin, dass die grüne und das verzögerte goldene Open Access, das klassische Geschäftsmodell der Verlage erstmal nicht direkt beeinträchtigt. Der goldene Weg auf Grundlage unmittelbarer, freier und offener Veröffentlichungspflicht dagegen kommt ohne das tradierte Geschäftsmodell der Verlage aus \cite{lewis_2012_inevitability}.

Weitere, aber im Vergleich wenig genutzte Modelle sind:
Hybride Modelle
Open Choice \cite{Hess_2006} 
...

\subsection{Open Access Kanäle und Formate}
In diesem Abschnitt wird auf unterschiedliche Modelle der Veröffentlichung wissenschaftlicher Inhalte in Form von Open Access Publikationen sowie auf verschiedene Open Access Kanäle und Publikationsformaten eingegangen.

Dabei wird unterschieden: Open Access Aggregatoren, Open Access Repositorien, Open Access Jounrals, Open Access Bücher, Open Access Journals. Sie beziehen sich entweder auf bestimmten Publikationsformen der wissenschaftlichen Kommunikation oder auf Herausforderungen, die im Rahmen der Distribution und Archivierung im Umfeld der neuen Möglichkeiten von offenem und freien Publizieren entstanden sind. 

Da es eine enge Verknüpfung zwischen von Repositorien und der Entwicklung der Open-Access-Bewegung gibt \cite{offhaus_2012_institutionelle_repos}, soll hier auf die Rolle der Repositorien als Kanal für die Verbreitung von Publiaktionen eingegangen werden. 
Repositorien sind verwaltete Orte zur Aufbewahrung geordneter Dokumente, die öffentlich zugänglich sind, im Unterschied zu Archiven die ausschließlich historische Dokumente verwalten \cite{suchen}. Institutionelle Repositorien sind ein Instrument für wissenschaftliche Einrichtungen wie etwa Universitäten, um ihre Publikationen frei zugänglich zu machen \cite{dobratz_2007_open}.

Institutionelle Repositorien haben potenziell erhebliche Vorteile für die Institutionen, wenn sie in die ganzheitlichen Rahmenbedingungen der Universität integriert sind \cite{steele_2006}. Repositorien können neben der klassischen Aufgabe für die Lernumgebungen und die Marketingaktivitäten einer Universität eine wichtige Rolle spielen: Sie ermöglichen die Dokumentation des universitären Outputs und verbessern den Zugang zu institutionellem Austausch \cite{steele_2006}. Ökonomisch rentieren sie sich vor allem dann, wenn skaleneffekte eintreten und in Verbünden agiert wird \cite{blythe_2005value}. Neben den institutionellen sind auch fachliche oder andere Arten von Repositorien eng mit der Open Access Bewegung verknüpft. 

Sie stehen umgangssprachlich für die digitale Speicherung von Dokumenten und mittlerweile auch Daten. Über sie wird der Zugang zu Open Access Publikationen ermöglicht.


\subsection{Kritik an Open Access}

In diesem Teil der Arbeit soll im Rahmen der Literaturanalyse eine Auflistung der Kritikpunkte an der Open Access Bewegung in Wissenschaft und Forschung dokumentiert werden. Die Auswahl der berücksichtigten Werke bezieht sich auf dei genannten Fragestellungen und soll als verständlicher Überblick über den vorherrschenden Diskurs im Rahmen von Open Access und Open Science verstanden werden.

\subsubsection{Kritik am ökonomischen Modell}

Ein Kritikpunkt an dem Open Access Modell bezieht sich auf das Kostenargument und die frühe Hoffnung, dass die technologischen Treiber gesteuert und organisiert von der Forschungs Community selbst, anstatt durch Fachverlage, die durchschnittlichen Kosten für einen publizierten Artikel signifikant senke könnten. In einigen Beiträgen wurdne schon früh Kostensenkungen von bis zu 90 Prozent\cite{hilf_2004} prognostiziert. Grundlage dafür war die Fragestellung, dass "aus der Sicht des individuellen Nutzenkalküls von Wissenschaftlern, Verlagen und weiteren Einrichtungen wie Bibliotheken als auch aus Sicht gesamtwirtschaftlicher Wohlfahrtsüberlegunge (...) ob der Markt der Wissenschaftskommunikation nicht effizienter organisiert werden könnte."\cite{Hess_2006} Folgende Punkte schürten darüber hinaus die Hoffung, dass System leistungsfähiger zu machen und "von seinen durch den Papierdruck auferlegten Fesseln" zu befreien \cite{hilf_2004}:
\begin{itemize}
\item langer Zeitverzug vom Einreichen eines Manuskriptes bis zum Gelesen werden,
\item komplizierter Vertriebsweg vom Verlag über Grossisten zu Bibliotheken,
\item horrende Kosten (ca. 3.000,- Euro für die gesamte Verlagsarbeit je Artikel) mit den daraus folgenden horrenden Zeitschriftenpreisen,
\item und daraus folgend wenige Leser, auch noch ungleich in der Welt verteilt (digital divide),
\item unvollständige Information (aus Platzmangel), was Nachnutzungen und das Nachprüfen erschwert und somit auch Fälschungen erleichtert,
\item nur anonymes Referieren vor der Veröffentlichung, was den Missbrauch erleichtert. 
\end{itemize}

\subsubsection{Sicherung von Freiheit von Forschung und Lehre sowie Forschungsdiversität}

Eine Öffnung der wissenschaftlichen Kommunikation hat weitreichende Implikationen, nicht nur auf die Frage wie geforscht wird, sondern auch was geforscht wird \cite{suchen}. Da ein Großteil der Wissenschaft durch die öffentliche Hand finanziert wird, stecken hinter den Steuerungsmechanismen von Wissenschaft und Wissenschaftsförderung immer auch politische Interessen. Zwar soll die Vermischung dieser Interessen in Deutschland durch die Unabhängigkeit der Deutsche Froschungsgemeinschaft verhindert werden und Mittel völlig frei von politischer Couleur verteilt werden \cite{suchen}, dennoch kann, so die Befürchtung einiger Autoren \cite{suchen}, nicht sichergestellt werden, dass eine Einbeziehung der Öffentlichkeit nicht doch einen Einfluss auf die Mittelvergabe hätte. Drastischer ausgedrückt sieht Hagner in dem Beitrag "Open access als Traum der Verwaltungen" dass es im Rahmen des Öffnungsprozesses auch auf eine vollends verwaltete Forschung hinausläuft \cite{suchen}. Grundlagenforschung sowie andere komplexe oder explorative Forschungsbereiche würden in Zukunft weniger Berücksichtigung finden und die Freiheit von Wissenschaft und Forschung endgültig gefährdet \cite{suchen}, so der düstere Ausblick einiger Wissenschaftler \cite{suchen} \cite{suchen}. 

Um diese Aspekte beziehungsweise Prognosen über die Implikationen von Open Access zu evaluieren werden in diesem Teil der Arbeit auf Grundlage von Textbeispielen die Kritik an der Öffnung von Wissenschaft und der (forschungs-)politischen, rechtlichen und freiheitlichen Entwicklungen beleuchtet.

\subsubsection{Beispiel: Der "Heidelberger Apell" für Publikationsfreiheit und die Wahrung der Urheberrechte }

Am 22. März 2009 wurde auf der Webseite der „Frankfurter Allgemeinen Zeitung“ der Artikel "Geistiges Eigentum: Autor darf Freiheit über sein Werk nicht verlieren" \cite{faz_heidelberger_apell_2009} veröffentlicht. Im Anhang zu dem Artikel fand sich ein Aufruf, auch der "Heidelberger Appell" genannt. Vorangegangen war eine öffentlich ausgetragene Diskussion zwischen dem Literaturwissenschaftler Prof. Dr. Roland Reuß sowie weiteren Wissenschaftlern in einem Spezial der Onlineausgabe der Frankfurter Allgemeinen Zeitung: "Die Debatte über Open Access".

Der Appell richtete sich vor allem an "die Bundesregierung und die Regierungen der Länder, das bestehende Urheberrecht, die Publikationsfreiheit und die Freiheit von Forschung und Lehre entschlossen und mit allen zu Gebote stehenden Mitteln zu verteidigen" \cite{ITK_2009}. Die Autoren forderten Politik, Öffentlichkeit und weitere Kreative auf, sich für die "Wahrung der Urheberrechte", unter anderem in Bezug auf die Google Buchsuche "gegen eine angebliche „Enteignung“ der Autoren durch das Vorgehen von Google einerseits und durch das Publikationsmodell Open Access andererseits" \cite{WD_bundestag_2009} zu engagieren. 

Die Kritik am urheberrechtlichem Aspekt der Google Buchsuche soll in dieser Arbeit nicht berücksichtigt werden. Hier soll nur untersucht werden, inwiefern die Kritik am Publikationsmodell Open Access berechtigt ist. Der Apell unterscheidet dabei in zwei Ebenen: \textit{International} kritisieren die Autoren "die nach deutschem Recht illegale Veröffentlichung urheberrechtlich geschützter Werke geistiges Eigentum auf Plattformen wie GoogleBooks und YouTube" und die Entwendung dieser "ohne strafrechtliche Konsequenzen". \textit{National} werden die "Eingriffe in die Presse- und Publikationsfreiheit, deren Folgen grundgesetzwidrig wären" durch die "»Allianz der deutschen Wissenschaftsorganisationen« (Mitglieder: Wissenschaftsrat, Deutsche Forschungsgemeinschaft, Leibniz-Gesellschaft, Max Planck-Institute u. a.)" angeprangert.\cite{ITK_2009}

Die Kritik der Autoren des Heidelberger Apells bezieht laut einer Untersuchung des Wissenschafltichen Diensts des Bundestags insbesondere auf drei Aspekte \cite{WD_bundestag_2009}:
\begin{enumerate}
\item Erzwungene Vertriebswege
"Eine Forschung, der man diktieren könnte, wo ihre Ergebnisse publiziert werden sollen, sei nicht mehr frei." Die Verpflichtung auf "bestimmte Publikationsform (...) dient nicht der Verbesserung der wissenschaftlichen Information" \cite{ITK_2009}.
\item Abhängigkeitsverhältnis
\item Subventionierung von Vertriebswegen
\end{enumerate}

Der Appell "hat eine außergewöhnlich heftige Diskussion über die urheberrechtliche Problematik im Hinblick auf die aktuellen Entwicklungen im Internet ausgelöst. Er hat auch viele Parlamentarier und Politiker für das Thema sensibilisiert"\cite{WD_bundestag_2009}. An vielen Stellen widerlegt der Wissenschaftliche Dienst die Befürchtungen der Autoren des Heidelberger Apells. Beim Kritikpunkt der "Erzwungene Vertriebswege" widerspricht der Wissenschaftliche Dienst mit dem Verweis auf Gudrun Gersmann, weil "auch (Anmerkung: unter Open Access) eine Veröffentlichung bei einem Verlag mit einfachem Nutzungsrecht weiterhin möglich sei". In Bezug auf die im Apell erwähnte Kritik am neuen Abhängigkeitsverhältnis halten die wissenschaftlichen Autoren des Bundestags Reuß entgegen, dass es im bisherigen System "zwischen Autor und Fachzeitschriftverlag oft ein einseitiges Abhängigkeitsverhältnis zu Lasten des Autors gibt" und Wissenschaftler "oftmals alle Rechte an ihren Beiträgen abtreten" \cite{WD_bundestag_2009} müssen. "Der Befürchtung im Heidelberger Appell, das Publikationsmodell Open Access gefährde Fachzeitschriftenverlage", laut Autoren dritter Aspekt der Kritik an Open Access im Apell, "wird entgegengehalten, dass die digitale Plattform auf lange Sicht auch ein Ausweg aus der Zeitschriftenkrise sein könnte" \cite{WD_bundestag_2009}.

Dabei ist die Kritik im Rahmen des Apells mindestens an zwei Punkten berechtigt, so ist es erstens wahr, dass man seitens der Forschungsförderer nicht besonders bemüht war und ist \cite{suchen}, sich "ein genaues Bild von den Nebenwirkungen (Anmerkung: von Open Access)" \cite{Reuss_2009} zu verschaffen und zweitens stellt die Sicherung von Freiheit von Forschung und Lehre sowie die Anpassung der Steuerungsmechanismen eine Herausforderung an die Bestrebungen zur Öffnung von Wissenschaft und Forschung dar \cite{suchen}.

\subsection{Analyse der Definitionen von Open Access} 

Eine eindeutige Klassifizierung von Open Access gelingt derzeit nicht. Es "keine formelle Struktur, keine offizelle Organisation und kein ernannter Führer" gibt, der die Open Access Bewegung antreibt\cite{poynder_2011_suber}. Einzig und allein die Open Definition - open definition schreiben -

\subsection{Treiber und Bremser für Open Access} 

In den wissenschaftlichen Beiträgen zu Open Access werden viele positive Folgen aufgelistet. Folgende Treiber für eine Veränderung und Öffnung des wissenschaftlichen Kommunikationssystems werden dabei besonders häufig genannt:

\begin{itemize}
\item Verbreitung und Nutzungsmöglichkeiten der digitalen Infrastrukturen
\item Vorteile des grenzüberschreitenden Austauschs im Rahmen der Globalisierung von Wissenschaft und Forschung
\item ...
\end{itemize}

Neben den Aspekten die die Verbreitung von Open Access in den letzten Decaden unterstützt haben, gibt es aber auch einige Kriterien, die entweder zu einer Verlangsamung der Entwicklung geführt haben, oder sie in einigen Teilbereichen ganz zum erliegen gebracht haben. Dazu gehören:

\begin{itemize}
\item Fehlende Richtlinien auf regionaler, nationaler und internationaler Ebene
\item Führungslosigkeit der Open Access Bewegung
\item ...
\end{itemize}

\section{Analyse der Definitionen von Open Science} 

--- TODO ---- Michael Nielsen: “Open science is the idea that scientific knowledge of all kinds should be openly shared as early as is practical in the discovery process.”  https://lists.okfn.org/pipermail/open-science/2011-July/000907.html
http://www.openscience.org/blog/?p=454,

Research Information Network: “science carried out and communicated in a manner which allows others to contribute, collaborate and add to the research effort, with all kinds of data, results and protocols made freely available at different stages of the research process.” http://www.rin.ac.uk/our-work/data-management-and-curation/open-science-case-studies

Fecher/Friesike 5 Schulen von Open Science http://blogs.lse.ac.uk/impactofsocialsciences/2013/06/20/open-science-new-perspectives-for-scholarly-communication/ 
Siehe "Open Science"-Teil @ https://docs.google.com/document/d/1qDkQV-M_2VazjWwncRq_udo9tQqrjuZZkdLeKFc3cpI/edit#heading=h.1ahb76xafkbm

"Open science is the concept of making the whole research process as transparent and accessible as possible."\cite{Scheliga_2014}

Open science can be seen as a mechanism of cumulative knowledge production whereby scientists draw upon knowledge derived at by "prior researchers" and make their discoveries available to "future researchers". \cite{Scheliga_2014} auf Grundlage von \cite{Mukherjee_2009}
--- TODO ---

Es gibt zahlreiche Open Science Initiativen \cite{Scheliga_2014} viele von Ihnen erreichen aber keine kritische Masse \cite{wrap_2010} und enden eher als "virtuelle Geisterstädte" \cite{Nielsen_2011}.

\subsection{Analyse der Definitionen}  
tbd

\subsection{Treiber und Bremser für Open Science} 
Bei der Verbreitung von Open Science, werden grundsätzlich zwei Strategien für die Etablierung von Offenheit in Wissenschaft und Forschung abgegrenzt \cite{schulze_2013_open}: 
\begin{enumerate}
\item "Top-down durch Förderstrategien, Vorgaben und Empfehlungen"
Hiermit sind Prozesse beziehungsweise gemeint, bei denen durch die direkte Incentivierung im Rahmen von Forschungsförderung Anreize für die Berücksichtigung von Offenheit in den geförderten Projekten geschaffen werden. Beispielsweise kann durch die Bereitstellung zusätzlicher Mittel für die offene Bereitstellung und Publikation von Forschungsergebnissen ein Anreiz geschaffen werden  \cite{suchen}. Neben der Incentivierung bietet die bindende Vorgabe eine weitere Möglichkeit zur Etablierung von Verhaltensänderungen \cite{suchen}. So kann durch Änderung der politischen und rechtlichen Vorgaben eine Öffnung von Wissenschaft und Forschung erzwungen werden \cite{suchen}. Eine weitere Möglichkeit der "Top-Down"-Etablierung von Offenheit und Forschung stellen Empfehlungen dar, bei denen Insitutionen, Organisationen oder Gruppen Empfehlungen aussprechen, anhand derer WissenschaftlerInnen über nicht bindende Hinweise überzeugt werden sollen, die Öffnung von Wissenschaft und Forschung zu etablieren. Alle diese Strategien haben einen formellen Charakter \cite{suchen}.
\item "Bottom-up durch Graswurzelprojekte und den Einsatz von Evangelists"
Im Gegenzug zur Strategie von "oben" gibt es auch Bestrebungen, die von einzelnen WissenschaftlerInnen oder Gruppen initiiert sind. Sie sind zumeist informell und zielen auf eine beispielhafte Herangehensweise für die Verbreitung von Verhaltensänderungen ab \cite{suchen}. Bottum-up-Projekte kommen aus dem wissenschaftlichen Alltag und erfahren keine politische, rechtliche oder monitäre Incentivierung zur Umsetzung der Tätigkeiten für die Öffnung von Wissenschaft und Forschung. Der Einsatz von Evangelisten baisert auf der Idee einer konkreten Stelle oder Position um eine Änderung zu Begleiten \cite{suchen} oder einen Mulitplikator innerhalb und außerhalb von Insitutionen oder Organisationen zu etablieren, der das gewünschte Ziel proaktiv kommuniziert und verbreitet \cite{suchen}. Evangelisten stellen einen wesentliche Maßnahme dar, um "die Befindlichkeiten" "auszutarieren" und um "teils diffuse, teils reale Ängste" bei "Offenheit und Transparenz der Wissenschaft "\cite{schulze_2013_open} zu beseitigen.
\end{enumerate} 

In beiden Fällen steht und fällt der Erfolg damit, ob sich der jeweiligen Zielgruppe ein unmittelbarer Mehrwert und Nutzen erschließen wird \cite{schulze_2013_open}.

\subsection{Kritik an Open Science}

Während viele Wissenschaftler und Wissenschaftlerinnen Offenheit in der Forschung als wertvoll erachten, sind nur wenige sind wirklich bereit, die zusätzliche Zeit und Mühe zu investieren und potenziellen Risiken einzugehen, ihre Forschung offen und zugänglich zu machen \cite{Scheliga_2014} \cite{Procter_2010}. Forscherinnen und Forscher, die offene Wissenschaft pratizieren wollen, sind mit einer Reihe von Hindernissen konfrontiert \cite{Scheliga_2014}: 
\begin{enumerate}
\item individuelle Hindernisse: Angst vor Trittbrettfahren, Mehraufwand an Zeit und Mühe, Herausforderungen bei der Nutzung der digitalen Dienste, fehlender Anstoß negative Ergebnisse zu veröffentlichen, Herausforderung den Datenschutz sicherzustellen, Abneigung den Code zu teilen
\item systematische Hindernisse: Evaluationskriterien behindern Offenheit, kulturelle und institutionelle Einschränkungen, ineffektive (politische) Richtlinien, Mangel an Standards für das Teilen von Forschungsmaterialien, Mangel an rechtlicher Klarheit, finanzielle Aspekte der Offenheit
\end{enumerate}

Betrachtet wie Scheliga und Friesike das Phänomen Open Science an Hand des Konzepts des Soziale Dilemmatas, wird deutlich, dass was im kollektiven Interesse der wissenschaftlichen Gemeinschaft ist, nicht unbedingt im Interesse des einzelnen Wissenschaftlers ist und "wenn alle Wissenschaftler ihr Wissen nur in den Situationen teilen, in denen sie erwarten, dass sie selbst davon profitieren, ist die gemeinsame Wissenspool fragmentiert und alle Wissenschaftler stehen schlechter dar"\cite{Scheliga_2014}. 

Demgegenüber stehen dem gegenüber  --- TODO: ausarbeiten ---



Die \textbf{Forderung nach Öffnung} adressiert mehrere Unzulänglichkeiten am bestehenden wissenschaftlichen Kommunikationssystem:
\begin{enumerate}
\item Transition-Argument
Die Nutzung der neuen Möglichkeiten für eine offene Wissensverbreitung neben den konventionellen Wegen der nicht-elektronischen Publikationen . Dabei gilt die Grundvoraussetzung der Aufbereitung des Wissens als strukturierte Daten zur Wissensweiterverwendung und -verarbeitung über alle Kanäle.
\item Speed & Circulation-Argument
Wissensverbreitung wird künstlich durch Embargos und ineffiziente Validationssysteme zurückgehalten. Die Digitalsierung und Verbreitung über elektronische Kanäle stellt einen Vorteil für Wissensverbreitung und -verwertung dar. Wenn das Wissen schneller zur Verfügungsteht wird es schneller zirkulieren und effizienter genutzt werden können \cite{Woelfle_2011}.  
\item Higher Impact & Citation-Argument
Ein Hauptargument der Open Acces-Befürworter ist die höhere Zitationsrate von wissenschaftlichen Publikationen, die unter den Kriterien von Open Access veröffentlicht wurden\cite{cite:21a}. In der einschlägigen Literatur findet man viele Untersuchugnen, die das zum Untersuchungsgegenstand gemacht haben und zu einem positiven Ergbniss kommen \cite{Lawrence_2001}\cite{Jeffrey_2008}\cite{Eysenbach_2006}\cite{Antelman_2004}
\item Tax-Payer-Agrument
Durch Steuergelder finanzierte Forschung ist dem Steuerzahler im Rahmen konventioneller wissenschaftlicher Kommunikation nicht immer unentgeldlich zugänglich, obwohl er im Rahmen öffentlich-geförderter Forschungsprogramme die Forschung bereits finanziert hat. Darüber hinaus stellt sich die Frage nach dem bestmöglichen Einsatz der monetären Ressourcen \cite{Glasziou_2014} \cite{altman_1994_scandal}.
\item Economic Promotion Argument
Bisher profitieren wirtschaftliche Unternehmungen nur unzureichend von staatlich-finanzierter wissenschaftlicher Kommunikation, dabei könnte eine schnellere, kommerziell verwertbare und umfassendere Bereitstellung der wissenschaftlichen Inhalte einen eklatanten Beitrag zur non-monetären Wirtschaftsförderung darstellen. Im Rahmen der offenen und schnelleren Verbreitung von wissenschaftlichen Informationen können neue Geschäftmodelle entstehen.
\item Digital Divide Argument
Der offene Zugang zu Publikationen ermöglicht neue Möglichkeiten für die Überwindung der sozialen, nationalen und globalen Wissenskluften  zwischen bildungsfernereren und -affineren Bevölkerungsteilen und -schichten der Welt . Der Mehrwert und die Chance von wissenschaftlichen Informationen für die Bewegung der offenen Bildungsmaterialien ist bisher auch noch nicht ausgeschöpft\cite{heise_lernen_2013}.
\item Validation & Reputation-Argument
Die Entwicklung neuer Verfahren, die die Aktivität und Qualität eines Forschers umfassender, transparenter und demokratischer messbar und kommunizierber machen, als im bestehenden Reputations- und Förderungssystem \cite{chalmers_2009_avoidable_waste}. Wissenschaftsevaluation wird durch Offenheit effizienter.
\item Paradoxon of Information Argument
Überwindung des bestehenden Informationsparadoxons bei der Verbreitung und Vermarktung von wissenschaftlichen Inhalten. Hierbei handelt es sich um das Problem, dass es schwer ist eine Information kommerziell zu verwerten ohne zu viel über Inhalt und Qualität auszusagen. Eine Entkommerzialisierung des Vertriebs von Wissen  würde das Informationsparadoxon aufheben.
\item Science communication Crisis-Argument
Durch die Öffnung von der wissenschaftlichen Kommunikations- und Reputationsprozesse besteht die Möglichkeit der vorherrschende Zeitschriften- und Monographienkrise durch neue Geschäftsmodelle zu begegnen.
\item Interdicipline & International Exchange/Collaboration Argument
Die Globalisierung in der Wissenschaft führt immerstärker zu internationalem Austausch und zur internationalen Zusammenarbeit von Wissenschaftlern . Doch das gilt nicht nur für die grenzenüberschreitende Zusammenarbeit in Bezug auf die lokale Verortung sondern auch für die Interdisziplinarität der Forschungsvorhaben. Die Öffnung von Wissenschaft ermöglicht also auch Fächerfremden Wissenschaftlern Zugruff auf Publikationen und damit auf Wissensressourcen für die eigene Arbeit .
\item Sustainable Access & Archiving Argument
Nur Offenheit im Sinne von Verwertbarkeit ermöglicht es in dezentralen Strukturen wie der des Internets alle Informationen nachhaltig und unabhängig voneinander zu speichern. Im Falle von Natur- oder anderen Katastrophen ermöglicht die digitale Ablage auf mehreren Kontinenten eine präservierung von Wissen undabhängig von lokalen Gegebenheiten oder Bedingungen.
\end{enumerate}

\textbf{Demgegenüber} stehen aber auch Argumente gegen die Öffnung der wissenschaftlichen Prozesse und Publikationen:
\begin{enumerate}
\item 	Quality-Argument
Die Befürchutung, das die Qualität auf Grund von schlechten oder nicht vorhandenen wissenschaftlichen Überprüfungsmechanismen leidet. Hauptargument ist das durch ein Autorengebühren finanziertes Publikationsmodell keinen klaren Anreiz für Ablehnung bietet.
\item Archiving-Argument
Die Sicherstellung der Langzeitarchivierung und die Garantierung der langfristigen Auffindbarkeit sowie Bereitstellung der Dokumente kann im Auge der Kritiker von Offenheit in Wissenschaft und Forschung nicht durch alternative digitale Strukturen gewährleistet werden. 
\item Authenticity-Argument
Forscherinnen und Forscher befürchten durch die dezentrale und offene Handhabung ihrer Texte und Arbeiten, dass diese im Zeitablauf inhaltlich nicht mehr unverändert zuordnenbar ihrem Autor sind.
\item Rightsmanagement-Argument
Hierbei handelt es sich um die Verpflichtung für Mitarbeiter staatlich finanzierter Forschungsinsitutionen alle Texte elektronisch frei und offen zu publizieren. In dem 2009 veröffentlichten "Heidelberger Appell" \cite{faz_heidelberger_apell_2009} kritisieren zahlreiche Autoren, Wissenschaftler, Verleger und Publizisten, dass das “verfassungsmäßig verbürgte Grundrecht von Urhebern auf freie und selbstbestimmte Publikation” … “derzeit massiven Angriffen ausgesetzt und nachhaltig bedroht” ist. Weiter sehen die Unterzeichner „weitreichende Eingriffe in die Presse- und Publikationsfreiheit, deren Folgen grundgesetzwidrig wären“ \cite{ITK_2009} und die Befürchtung, dass die Freiheit von Forschung und Lehre gefährdet ist \cite{Jochum_2009}. 
\item (Re-)Financing-Argument
Die unklare Refinanzierung der Öffnung von Wissenschaft ist eines der Kernargumente gegen das offene Publizieren von Arbeiten und Daten. Die Befürchtung ist, das ein solches System überhaupt nicht finanziert werden kann, konnte bisher nicht ausgeräumt werden.
\item Sustainability-Argument
\item Ressource-Allocation-Argument
Die Befürchtung, dass die Vergabe von Fördermittel und für die Karriere wichtige Aspekte der Reputationsbildung durch offenen System nicht Rechnung getragen wern kann ist eine weiteres Argument der Kritiker der Öffnung von Wissenschaft und Forschung. Eine Mittelvergabe zu gunsten populärer Forschung und damit eine Aushöhlung des wissenschaftlichen Systems in Ihrer Fächer und Facettenvielfalt wäre eine unmittelbare Folge dessen.
\item Open-Caring-Argument
Wissenschaftlerinnen und Wissenschaftler fürchten durch den Zwang zu umfassenderen Bereitstellung von Publikationen und gegebenenfalls soagar Daten einen nicht unwesentlichen zeitlichen Mehraufwand für die Öffnung ihrer Arbeiten. Sie möchten aber möglichst wenig Zeit für die Veröffentlichung, Bereithaltung und Verbreitungung ihrer wissenschaftlichen Arbeiten aufbringen.
	Aufwand für Offenheit im Alltag des Wissenschaftlers
\item Scientific-Freedom/Loss of Idea-Diversity-Argument
Angst dass durch Offenheit und Transparenz Forschungsförderung und Öffentlichkeit nur die wissenschaftlichen Projekte fördern und unterstützen, die von der Öffentlichkeit verstanden werden. Dabei stellt Wissen, vorallem aber Grundlagenwissen ein "öffentliches Gut" dar, "dessen Wert von der Öffentlichkeit nur schwer beurteilt werden kann"\cite{osterloh2008anreize}. Darüber hinaus herrscht die Annahme, dass im Rahmen von zunehmender Kollaboration und der Effizienz der elektronischen Suche die Diversität von wissenschaftlichen Meinungen und Projekten zu einem gleichen oder ähnlichem Thema eingeschränkt wird\cite{Evans_2008}.
\item Interpretations-Argument
Eine der weiteren Ängste der wissenschaftlichen Community ist die Angst vor der Fehlinterpretation ihres kommunizierten Wissens sowie der Verlust der Kontrolle über die Informationen\cite{gibbons_1994}. Dabei steht vor allem die Befürchtung im Vordergrund, dass die offen veröffentlichten Arbeiten genutzt werden um die Arbeit zu miskreditieren oder gezielt zur Falschinfromation der Öffentlichkeit zu nutzen.
\item Transparent-Research-Intentions-Argument
Mit den Forderung nach Offenlegung des gesamten Forschungsprozess erfolgt auch die Forderung nach "Transparenz der Interaktion zwischen Sponsoren (insbesondere kommerzielle Förderer wie die Pharma- und Medizinprodukteindustrie) und Auftragnehmern" \cite{Stengel_2013} 
\end{enumerate}

\subsection{Offener Zugriff auf wissenschaftliche Kommunikation}
Open Science beinhaltet nicht nur um den offenen Zugang zu Wissenschaft und den daraus resultierenden Veränderungen wissenschaftlicher Kommunikationsprozessen im Rahmen von Publikationen, sondern auch den unmittelbaren und offenen Zugriff auf den gesamten Prozess der Wissensschaffung. Aus technischer Sicht ist jeder Aspekt der Wissenschaftskommunikation, der digital auf einem Desktop-Computer stattfindet, auch öffentlich über das Web potenziell verfügbar \cite{mietchen2012wissenschaft}. 

Zur Verdeutlichung des Prozess der Wissensschaffung wird in der vorliegenden Arbeit eine Einteilung extrapoliert in vier Phasen vorgenommen:
\begin{enumerate}
\item Fragestellung & Planung
Basis für den Prozess der Wissenschaffung ist eine Frage zur Erklärug einer speziefischen Beobachtung oder eine offene Frage\cite{suchen}. Für die wissenschaftliche Bearbeitung eines Themas ist es entscheidend, dass eine präzise Fragestellung im Zentrum steht \cite{suchen}. --- TODO: weiter beschreiben ---
\item Ausführung
Testen der Hypothese durch den Einsatz von geeigneten wissenschaftlichen Kontrollen und unter Minimierung der möglichen Fehler.
\item Verarbeitung und Analyse --- TODO: weiter beschreiben ---
Analyse der gewonnen Daten und Informationen im Hinblick auf die Verifikation und Falsifikation der Hypothese. --- TODO: weiter beschreiben ---
\item Auswertung
----TODO: Beschreiben-----
\end{enumerate}

Anhand der hier vorgenommenen Einteilung werden die Charakteristika des Wissenschafts-Prozesss erläutert und dargestellt, um zu verdeutlichen, was die Öffnung von Wissenschaft im Sinne von Open Science bedeutet. Die Forderung nach Öffnung des Prozess der Wissensschaffung begründet sich dabei nicht  durch Unzulänglichkeiten am bestehenden wissenschaftlichen Kommunikationssystem, sondern basiert auch auf weiteren Annahmen:

\begin{enumerate}
\item Der offene Zugang zum gesamten Wissenschaftsprozess erhöht die Möglichkeiten der Validierung und Reproduzierbarkeit der gesamten Forschung(skette) und die Entwicklung neuer Qualitätskriterien. (enhanced Validation/Reputation-Argument)
\item Im Rahmen des Teilens (z.B. von Rohdaten) erhöht sich die Effizienz und Verwendbarkeit von Forschung und im Rahmen von Wissenschaft entstandenen Informationen (Shared-Science-Argument)
\item im klassischen wissenschaftlichen Kommunikationssystem gibt es kaum Anreize negative, widerlegende oder unerfolgreiche wissenschaftliche Ergebnisse zu veröffentlichen, eine grundsätzliche Öffnung könnte dazu beitragen, dass Wissenschaft ihrem Anspruch an Falsifizierbarkeit gerecht wird z.B. in Pharmalogie (negative-science/falsifiability-argument)
\end{enumerate}

\subsection{Wissenschaft als Open-Source-Prozeß}

Open Source ist ein Begriff aus der Softwareentwicklung der als Gegensatz zum “Verfahren der Wissenssicherung” \cite{stallman2002} eine quelloffenen Handhabe von Softwarecode beschreibt. Der Ende der 90iger Jahre des letzten Jahrhunderts eingeführte Begriff wird, auch wenn es im Detail Unterschiede im Konzept gibt \cite{suchen}, mit “freier Software“ (nicht Freeware) gleichgesetzt \cite{suchen}. Dabei folgt die Open Source-Entwicklung der Maxime, dass die Kernsteuerungsinformationen und -befehle (Quelltext) von Software öffentlich einsehbar und zugänglich und je nach gewähltem Lizenzmodell modifiziert, kopiert oder weitergegeben werden müssen\cite{suchen}. 

Die Entwicklungsmethode unterscheidet, so Steven Weber, zwischen Open-Source-Software und dem traditionellen Modell des geistigen Eigentums mit der Feststellung, dass Open-Source-Software das Prinzip der Exklusivität des geistigen Eigentums auf den Kopf stellt, weil diese Software 'um das Recht auf Vertrieb konfiguriert, nicht auszuschließen ist" \cite{suchen}. 

--- Prüfen ---
Auch Maurer und Scotchmer merken an, dass Open-Source-Software-Entwicklung Rechtsmittel ein Defekt der Schutz des geistigen Eigentums, die nicht allgemein zu fördern hat die Offenlegung des Quellcodes. 
--- Prüfen ---

Die Open Source Definition beinhaltet festgelegte Kriterien für die Klassifizierung von Open Source Produkten \cite{suchen}:
\begin{enumerate}
\item Freie Weitergabe
----TODO: Beschreiben-----
\item Quellcode, das Programm muss den Quellcode beinhalten, bzw. muss den Code offen zur Verfügung stellen
----TODO: Beschreiben-----
\item Verwendete Lizenz muss Derivate zulassen
----TODO: Beschreiben-----
\item Unversehrtheit des Quellcodes des Autors muss garantiert werden
----TODO: Beschreiben-----
\item Auschluss von Diskriminierung von Personen oder Gruppen
----TODO: Beschreiben-----
\item Keine Enschränkung des Einsatzfeldes
----TODO: Beschreiben-----
\item Lizenz muss weitergegeben werden könnne
----TODO: Beschreiben-----
\item Lizenz muss auf das Produktpaket angewandt werden
----TODO: Beschreiben-----
\item Lizenz darf die Weitergabe zusammen mit anderer Software nicht einschränken
----TODO: Beschreiben-----
\end{enumerate}

Im Vergleich zum klassischen Softwareentwicklungsprozess gelten folgende charakteristische Merkmale \cite{suchen}:
\begin{enumerate}
\item “Anzahl der beteiligten Entwickler: Im Vergleich zu traditioneller Softwareentwicklung ist eine weitaus größere Anzahl von Entwicklern beteiligt. Zudem gibt es keine klare Grenze zwischen Entwicklern und Anwendern, da die Hürden für eine Partizipation im Entwicklungsprozess sehr gering sind. Auch wenn ein großer Teil der Entwicklungsarbeit von Freiwilligen übernommen wird, gibt es dennoch den Trend zum Einsatz bezahlter Entwickler.
\item Zuteilung der Arbeit: Im OSP wird die Entwicklungsarbeit nicht länger von einer definierten Instanz zugeteilt, sondern die Teilnehmer wählen ihre Arbeitspakete selbst aus.
\item Architektur: In der Regel orientierten sich die Teilnehmer eines OSP nicht an einer vorgegebenen System-Architektur. Die Gestaltung der Architektur geschieht dezentral und ist oftmals häufigen Richtungswechseln unterworfen.
\item Koordination: Es gibt wenig oder keine institutionalisierten traditionellen Koordinationsmechanismen, wie z.B. Projekt- und Zeitpläne, Lasten- und Pflichtenhefte u.ä.” \cite{suchen}
\end{enumerate}

Der Literaturwissenschaftler und Medientheoretiker Friedrich Kittler beschreibt die Entwicklungsmethode Open-Source als fest mit dem Wissenschaftsprozess verankert \cite{suchen}. Open Source Entwicklungsprozesse unterscheiden sich von den klassisch-traditionellen (closed-source) Softwareentwicklungsprozessen insbesondere dadurch, dass sie jederzeit öffentlich einsehbar und transparent nachvollziehbar sind. Open Source zeigt diesbezüglich mit Open Science konvergenzen, als dass es nicht nur den freien und offenen Zugang zu wissenschaftlichen Informationen betrifft, sondern auch den Zugriff auf den gesammten Prozess zur Erlangung der wissenschaftlichen Informationen sowie die Daten offenlegt und transparent nachvollziehbar macht \cite{kelty_2004}. Adaptiert man den Open-Source-Prozess auf wissenschaftliche Wertschöpfungsprozesse und definiert in diesem Zusammenhang wissenschaftliche Publikationen als Quellcode, ist das Konzept übertragbar \cite{Singh_2008} \cite{Bradley_2008} \cite{Bradley_2007}. Daraus folgt, dass Open Access aus technologisch-entwicklungsmethodischer Sicht mit kostenloser Software (Freeware) \cite{suchen} verglichen werden kann. Freeware und Open Access Publikationen sind zwar kostenlos verfügbar, ihr Erstellungprozess wird jedoch nicht offen und transparent kommuniziert \cite{suchen}. Dieser Exkurs in die Softwareentwicklung versicht die Abgrenzung von Open Access zu Open Science zu verdeutlichen und stellt Parallelen zu Open Source versus kostenloser Software (Freeware) her. Es gibt aber noch eine  weitere Gemeinsamkeit: "Free Software (im Sinne von Open Source), Open Access und Creative Commons sind alles Rechts- und Infrastrukturexperimente"\cite{kelty_2004}.


\subsection{Open Science Modelle}
--- TODO: definieren ----
\subsection{Open Science Formate}
Data Repositorien, (offne) Forschungsanträge, offenes Publizieren (siehe OA), Laborbücher



\subsection{Kritik}
Die Verlage haben mit Hilfe von wissenschaftlicher Journale ein zentrales Steuerungs- und Bewertungssystem in der Wissenschaft etablieren können. Dabei werden die Grundprinzipien der Wissenschaft für die verlegerischen Verwertungsinteressen genutzt und das, obwohl diese “wissenschaftlichen Grundprinzipien und Normen eigentlich ökonomischen Verwertungsinteressen zu widersprechen scheinen” \cite{hanekop_2006}. Darüber hinaus haben die Forscher in vielen Fällen wenig oder keine Verantwortung für den Einkauf der wissenschaftlichen Informationen, die er oder sie "verschenkt" \cite{steele_2006}. Die Einführung der Zitationsregister und Impact Faktoren, sowie die Definition der Kernzeitschriften, hat zur weitgehenden Erstarrung des wissenschaftliche Zeitschriftenmarktes geführt und gleichzeitig die Kapazität der kommerziellen Verlagen, sowie deren Gewinnmargen ansteigen lassen \cite{CREATe_2014}. Die Steuerungsmechanismen werden über die Messbarkeit mittels --- TODO siehe alt 2.3.4 ---- beschriebenen Methoden direkt oder indirekt ausgeübt. Dabei stehen insbesondere die Methoden, die auf der quantitativen Grundlage der Zitationsraten wissenschaftlicher Publikationen gemessen werden in der Kritik \cite{Dong_2005} und auch andere Indikatoren für die Messung von Forschungsleistungen sind hoch umstritten \cite{Hornbostel_1997} \cite{Hicks_1996} \cite{Havemann_2002}. Der Hauptkritikpunkt: Die Verfahren, um die Wirkung von Wissenschaft und damit auch die Reputation von Wissenschaftlern zu messen, sind kein eigentliches wissenschaftliches Produkt\cite{suchen} und erfassen zum Beispiel die Tätigkeit einzelner Forschergruppen zu stark \cite{schmoch_2009}. Das führt unter anderem dazu, dass der Impact Factor “kein perfektes Werkzeug (ist) um die Qualität der Artikel zu messen” und trotzdem wird er zur Bewertung von Wissenschaft genutzt, denn “(...) es gibt nichts Besseres, und er hat den Vorteil, dass er bereits lange existiert und ist daher eine gute Technik für die wissenschaftliche Bewertung”\cite{garfield_1999}. Wie “gering der Wirkungsgrad” und die Methoden zur Messung “zur Reproduktion des traditionellen wissenschaftlichen Diskurses ausfall(en), wird von dem Moment an klar, an dem ein neues und offenes Kommunikationsmedium wie das Internet als alternativer Publikations- und Verbreitungskanal für Wissenschaft zur Verfügung steht \cite{Rost_1998}. 


\section{Defizite}
Viele der Erklärungsansätze für den Paradigmenwechsel hin zur Öffnung der Wissenschaft basieren auf Annahmen, in denen ein direkter Zusammenhang von technischen Entwicklungen unmittelbar auf (wissenschafts-)politische und kulturelle Bewegungen geschlossen werden. Darüber hinaus beschränkt sich die Perspektive primär auf den Zugang zum Ergebnis von Wissenschaft und weniger auf die Öffnung des gesamten Prozesses. Die theoretische Auseinandersetzung mit der Geschlossenheit des wissenschaftlichen Diskurses  auf der Einen und mit den Treibern und Bremsern im realen wissenschaftlichen Prozess werden in der gängigen Literatur auf der anderen Seite, wird nicht genügend berücksichtigt. Hier wird vor allem die Verbindung zwischen wissenschaftlicher Reputation und Geschlossenheit des Wissensproduktionsprozesses nur selten erörtert. Als weiteres Manko kann angeführt werden, "dass die Deliberation bezüglich und die Verbreitung von Wissen ein stabiles Set von Infrastrukturen braucht"\cite{kelty_2004}, nach denen man heute noch vergeblich sucht.
