\subsubsection{Offener Zugriff auf Wissenschaft}
In Ergänzung zu 2.1.1. geht es bei Open Science dabei eben nicht nur um den offenen Zugang zu Wissenschaft und den daraus resultierenden Veränderungen von Kommunikationsprozessen im Rahmen von Publikationen, sondern auch um den generellen und offenen Zugriff auf den gesamten Prozess der Wissensschaffung. Dieser Ansatz folgt dabei der Annahme, dass aus technischer Sicht praktisch jeder Aspekt der Wissenschaftskommunikation, der digital auf einem Desktop-Computer stattfindet, auch öffentlich über das Web erfolgen kann . Dieser Wissenschafts-Prozess wird in dieser Arbeit grob in vier Phasen gegliedert:
\begin{enumerate}
\item Planung
\item Ausführung
\item Verarbeitung
\item Auswertung
\end{enumerate}
In diesem Kapitel sollen die Charakteristika des Wissenschafts-Prozesss erläutert werden und dargestellt werden, was eine Öffnung im Sinne des Zugriffs bedeutet. In diesem Zusammenhang soll Open Science nicht nur als Sammelbegriff, sondern auch als  weiteren Evolutionsschritt nach Open Access verstanden werden. Die Forderung nach Open Science begründet sich dabei nicht nur durch die in 2.1.1 genannten Unzulänglichkeiten am wissenschaftlichen Kommunikationssystem sondern basiert auf folgenden weiteren Annahmen:
\begin{enumerate}
\item Der offene Zugang zum gesamten Wissenschaftsprozess erhöht die Möglichkeiten der Validierung und Reproduzierbarkeit der gesamten Forschung(skette) und die Entwicklung neuer Qualitätskriterien. (enhanced Validation/Reputation-Argument)
\item Im Rahmen des Teilens (z.B. von Rohdaten) erhöht sich die Effizienz und Verwendbarkeit von Forschung und im Rahmen von Wissenschaft entstandenen Informationen (Shared-Science-Argument)
\item im klassischen wissenschaftlichen Kommunikationssystem gibt es kaum Anreize negative, widerlegende oder unerfolgreiche wissenschaftliche Ergebnisse zu veröffentlichen, eine grundsätzliche Öffnung könnte dazu beitragen, dass Wissenschaft ihrem Anspruch an Falsifizierbarkeit gerecht wird  z.B. in Pharmalogie (negative-science/falsifiability-argument)
\end{enumerate}
klassisches Geschäftsmodell vs. open Geschäftsmodell