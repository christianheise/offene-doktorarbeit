\subsubsection{Messbarkeit wissenschaftlicher Qualität vs. Publikationsquantität}
Wissenschaft ist ein Prozess, bei dem aus “unterschiedlichen Inputfaktoren, mittels verschiedener Transformationen Beiträge zur Schaffung neuer wissenschaftlicher Erkenntnisse als Output entstehen” . Die Bewertungen des jeweiligen Outputs führt “zur Ausage über die Forschungsperformanz”. Neben den Indikatoren für den Output wissenschaftlicher Perfomanz, müssen aber auch intermediäre Aspekte berücksichtigt werden . Nach diesem Ansatz etablierten sich nach dem zweiten Weltkrieg die ersten Indikatoren für die Effizienzmessung wissenschaftlicher Wissensproduktion. Seit den 1960er Jahren wird diese Messung in der Gestalt von Indikatoren, die die Forschungsleistung quantifizieren sollen, durchgeführt. Seit den 1990er Jahren fällt diese Bewertung in Gestalt von Zahlen als unkontrollierte Nebenprodukte digitaler Wissenskommunikation an . Heute zählen in der Wissenschaft vor allem die wissenschaftliche Reputation und die als Impact bezeichnete Wirkung wissenschaftlicher Publikationen.  Die Wirkung wird dabei anhand der Zitationen der jeweiligen Publikation gemessen. Eine häufige Zitation stellt dabei einen Indikator für einen große Wirkung der wissenschaftlichen Arbeit dar. In diesem Kapitel werden die gängigen Methoden und Möglichkeiten der Messbarkeit von wissenschaftlicher Reputation und Wirkung im Kontext der Arbeit dargestellt.