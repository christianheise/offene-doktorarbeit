\subsection{Open Access} 
\begin{quote}
„Open Access“ meint, dass [= Peer-Review-Fachliteratur] kostenfrei und öffentlich im Internet zugänglich sein sollte, sodass Interessenten die Volltexte lesen, herunterladen, kopieren, verteilen, drucken, in ihnen suchen, auf sie verweisen und sie auch sonst auf jede denkbare legale Weise benutzen können, ohne finanzielle, gesetzliche oder technische Barrieren jenseits von denen, die mit dem Internet-Zugang selbst verbunden sind. In allen Fragen des Wiederabdrucks und der Verteilung und in allen Fragen des Copyrights überhaupt sollte die einzige Einschränkung darin bestehen, den Autoren Kontrolle über ihre Arbeit zu belassen und deren Recht zu sichern, dass ihre Arbeit angemessen anerkannt und zitiert wird.
\cite{boai_2012}
\end{quote}
Der Prozess der Wissenschaft hängt von dem freien Austausch und der Verbreitung von Informationen ab\cite{cite:11}. Das System der wissenschaftlichen Kommunikation, das so seit mehreren hundert Jahren besteht, basierte auf Forschung, Kommunikation der Egebnisse in Büchern und anderen wissenschaftlichen Publikationen, der Begutachtung, dem Druck der Publikation und der Verbreitung sowie dem Verkauf an Bibliotheken und andere wissenschaftliche Institutionen gegen Kosten \cite{cite:11a}. Der offene Zugang zu wissenschaftlicher Kommunikation ist seit der Entwicklung des gedruckten Wortes aber auch eng verknüpft mit der Frage nach Urheberrechten für wissenschaftliche Informationen\cite{Case_2000}. Open Access beschreibt in diesem Zusammenhang eine Situation in der der Zugriff auf die unterschiedlichsten Formen wissenschaftlicher Publikationen unter freien, kostenlosen und ohne finanzielle, gesetzliche oder technische Barrieren (Online) möglich ist \cite{WD_bundestag_2009}, aber auch ein "alternatives Geschäftsmodell"\cite{lewis_2012_inevitability} für wissenschaftliche Publikationen. Das beruht allerdings auch auf der Maßgabe, dass der Autor die "Eigentumsrechte an den Artikeln, die bisher für die Publikation in wissenschaftlichen Journals an die jeweiligen Fachverlage abgetreten wurden, (...) nun bei den Autoren der Artikel selbst verbleiben \cite{Hess_2006}. 

Durch den weltweit steigenden Haushaltsdruck an Bibliotheken und wissenschaftlichen Insitutionen sowie dem “ungewöhnlichen Geschäftsmodell”\cite{cite:12} mit immer höheren Margen der Wissenschaftsverlage\cite{albert_2006_open_implications} und dem Umstand, dass private Wissenschaftsverlage über öffentlich finanzierte Wissenschaftlerkarrieren entscheiden\cite{heise_2012}, befindet sich das System in einer Krise\cite{cite:14}. Open Access beschäftigt sich in diesem Rahmen mit der Öffnung (Open) und dem freien Zugang (Access) zu den wissenschaftlichen Publikationen. Die größtmögliche Verbreitung wissenschaftlicher Informationen stellt dabei eine der grundlegenden Forderungen von Open Access dar\cite{cite:15} und der Einsatz von (offenen) Lizenzen ist dafür einer der Haupteinflussfaktoren\cite{cite:16}. Das Modell kann dabei in drei Modelle eingeteilt werden: Green Open Access, Golden Open Access, Gray Open Access und andere Mischformen.
