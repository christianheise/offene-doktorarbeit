\subsection{Open Access} 
\begin{quote}
„Open Access“ meint, dass [= Peer-Review-Fachliteratur] kostenfrei und öffentlich im Internet zugänglich sein sollte, sodass Interessenten die Volltexte lesen, herunterladen, kopieren, verteilen, drucken, in ihnen suchen, auf sie verweisen und sie auch sonst auf jede denkbare legale Weise benutzen können, ohne finanzielle, gesetzliche oder technische Barrieren jenseits von denen, die mit dem Internet-Zugang selbst verbunden sind. In allen Fragen des Wiederabdrucks und der Verteilung und in allen Fragen des Copyrights überhaupt sollte die einzige Einschränkung darin bestehen, den Autoren Kontrolle über ihre Arbeit zu belassen und deren Recht zu sichern, dass ihre Arbeit angemessen anerkannt und zitiert wird.
\cite{boai_2012}
\end{quote}
Der Prozess der Wissenschaft hängt von dem freien Austausch und der Verbreitung von Informationen ab\cite{cite:11}. Das System der wissenschaftlichen Kommunikation, das so seit mehreren hundert Jahren besteht, basierte auf Forschung, Kommunikation der Egebnisse in Büchern und anderen wissenschaftlichen Publikationen, der Begutachtung, dem Druck der Publikation und der Verbreitung sowie dem Verkauf an Bibliotheken und andere wissenschaftliche Institutionen gegen Kosten\cite{cite:11a}. Der offene Zugang zu wissenschaftlicher Kommunikation ist seit der Entwicklung des gedruckten Wortes aber auch eng verknüpft mit der Frage nach Urheberrechten für wissenschaftliche Informationen\cite{Case_2000}. Open Access beschreibt in diesem Zusammenhang eine Situation in der der Zugriff auf die unterschiedlichsten Formen wissenschaftlicher Publikationen unter freien, kostenlosen und unmittelbaren Bedingungen (Online) möglich ist{cite:11a2} aber auch ein "alternatives Geschäftsmodell"\cite{lewis_2012_inevitability} für wissenschaftliche Publikationen. Das beruht allerdings auch auf der Maßgabe, dass der Autor das letztendliche Urheberrecht behält{cite:11ab}.

Als weitere wesentliche Besonderheit der Wissenschaftskommunikation ist die Organisation des Marktes, die von spezifischen Akteuren und Prozessen geprägt ist \cite{Hess_2006}, zu nennen. Vereinfacht kann der klassische wissenschaftliche Kommunikationsprozess im Rahmen von Publikationen wie folgt unterteilt werden\cite{cite:11b} \cite{Hess_2006}:
\begin{enumerate}
\item Erstellung durch Wissenschaftler - Inhalte erzeugen: 
Nach der Entwicklung eines konkreten Forschungsvorhabens sowie einer wissenschaftlichen Fragestellung enstehen im Rahmen der wissenschaftlichen Forschung oder der jewiligen Untersuchung Informationen\cite{cite:11c}, die im Forschungsprozess gesammelt, analysiert, ausgewertet, aufbereitet und verschriftlicht wurden\cite{cite:11d}. Diese Infromationen werden strukturiert zusammengefasst und niedergeschrieben \cite{Hess_2006}.
\item Qualitätskontrolle durch Wissenschaftler - Inhalte bewerten: 
Die Qualitätskontrolle ist einer der wesentlichen Bestandteile der wissenschaftlichen Kommunikation. Sie sichert die gewonnen Erkenntnisse\cite{cite:11e} und stellt einen klaren Abrenzungsaspekt zu nicht-wissenschaftlichen Informationen und Erkenntnissen dar\cite{cite:11f}. Sie findet im Kommunikationsprozess an zwei Stellen statt. Hier ist die erste Stelle gemeint, in der vor der Produktion der Informationen in Form der Publikation, die Erkenntnisse von anderen Wissenschaftlern überprüft und gesichert werden (Peer-Review) \cite{Hess_2006}.
\item Bündlung durch Verlage - Inhalte auswählen:
\item Publikation durch Verlage - Inhalte distribuieren: 
Nach Erstellung und Erkenntnissicherung findet die für die Distribution notwendige Publikation der Informationen statt. Vor der digitalen Revolution bestand dieser Schritt ausschließlich in dem Druck auf Papier.\cite{cite:11h}
\item Distribution durch Verlage: 
Der Vertrieb und die Verbreitung der Inhalte ermöglicht den Zugriff auf die Information der Forschung durch andere Wissenschaftler. Der Schritt stellt einen wichtigen Teil zur Zirkulation des neu gewonnen Wissens dar\cite{cite:11i}. Er sichert die Verfügbarkeit und den Zugriff auf die Informationen und ist essentieller Teil des Selektionsprozesses für die Erschaffung neuen Wissens.\cite{cite:11l}
\item Suppoert und Archivierung durch Bibliotheken
\item Konsum beziehungsweise Rezeption durch Wissenschaftler: 
Der nächste Schritt im wissenschaftlichen Kommunikationsprozess, der wiederum den gesamten Prozess von vore beginnen lässt ist die Rezeption der veröffentlichten Inhalte. Hier geht es zum einen um die Rezeption der wissenschaftlichen Vorschung aus der "Erstellung", zum anderen kommt hier auch die zweite Stufe der Qualitätsicherung zum Tragen.\cite{cite:11j} Der Konsum von wissenschaftlicher Informationen ist dabei auch als Grundlage für die "Erstellung" neuen Wissens zu betrachten. Somit ist der Endpunkt des wissenschaftlichen Kommunikationsprozess auch gleichzeitig Ausgangspunkt für einen neuen Prozess\cite{cite:11k}.
\end{enumerate}

An diesem Prozess sind vor allem drei Gruppen beteiligt: erstens die Wissenschaftler, als Produzenten und Konsumenten der Informationen, zweitens die Verleger, die als Intermediäre wissenschaftliche Informationen sammeln, bündeln und verkaufen, sowie drittens die Bibliotheken, die die Informationen wieder den Wissenschaftlern zur Verfügung stellen \cite{Odlyzko_1997}. Aus diesem Prozess und den beteiligten Gruppen, werden folgende Problemfelder ersichtlich:


klassisches Geschäftsmodell/Wertschöpfungskette vs. Open Access Geschäftsmodell/Wertschöpfungskette

Durch den weltweit steigenden Haushaltsdruck an Bibliotheken und wissenschaftlichen Insitutionen sowie dem “ungewöhnlichen Geschäftsmodell”\cite{cite:12} mit immer höheren Margen der Wissenschaftsverlage\cite{albert_2006_open_implications} und dem Umstand, dass private Wissenschaftsverlage über öffentlich finanzierte Wissenschaftlerkarrieren entscheiden\cite{heise_2012}, befindet sich das System in einer Krise\cite{cite:14}. Open Access beschäftigt sich in diesem Rahmen mit der Öffnung (Open) und dem freien Zugang (Access) zu den wissenschaftlichen Publikationen. Die größtmögliche Verbreitung wissenschaftlicher Informationen stellt dabei eine der grundlegenden Forderungen von Open Access dar\cite{cite:15} und der Einsatz von (offenen) Lizenzen ist dafür einer der Haupteinflussfaktoren\cite{cite:16}. Das Modell kann dabei in drei Modelle eingeteilt werden: Green Open Access, Golden Open Access, Gray Open Access und andere Mischformen.
