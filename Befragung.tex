\chapter{Befragung: Öffnung von Wissenschaft aus der Perspektive von Wissenschaftlern}

Ziel der Arbeit ist es, die erarbeiteten theoretischen Grundlagen, sowie die Ausprägungen von Open Access und Open Science vor dem Hintergrund von wissenschaftlicher Reputation und über die Grenzen einzelner Fachdisziplinen hinaus im Rahmen einer Umfrage zu püfen. Besondere Berücksichtigung findet dabei auch die Identifikation weiterer Treiber und Bremser für die Öffnung von wissenschaftlicher Informationen und Prozesse. Dafür werden die aus der theoretischen Betrachtung analysierten Konzepte Open Access und Open Science der Untersuchung zugeordnet. Der Einsatz des "Forschungsinstruments Fragebogen" gehörtz zu den am häufigsten eingesetzten Methoden in der Sozialforschung \cite{raab_2012_fragebogen}.

Die Befragung ist methodisch von der Ausrichtung mit einer Studie des Soziologischen Forschungsinstituts Göttingen (SOFI) "Wissenschaftliche Publikationen im Internet: Wissenschaftler als Leser und Autoren" aus dem Jahr 2007/2008 vergleichbar. Die Studie untersuchte "die neue Möglichkeiten wissenschaftlichen Publizierens, die zunehmend als Alternative zu Fachzeitschriften und -verlagen diskutiert werden" und "zielte darauf ab, die Veränderungen aus der Perspektive von Wissenschaftlern als Autoren und Leser zu untersuchen"\cite{SOFI_Webseite}.

Die zentralen Forschungsfragen dieser Arbeit stellten die Grundlage für die Entwicklung des Fragepools dar. Der Fragebogen war für die Erfassung konkreter Verhaltensweisen und allgemeine Zustände und Sachverhalte \cite{raab_2012_fragebogen} konstruiert. Die Formulierung der Fragen basierte, sofern nicht aus der Studie des SOFI unverändert übernommen, auf den in den vorhergehenden Kapieteln theoretisch erarbeiteten Handlungsmustern, Definitionen, Intentionen, Meinungen und Einstellungen zu folgenden Fragestellungen:
\begin{itemize}
\item Wie verändert die Digitalisierung, wie wir auf wissenschaftliche Daten und Informationen zugreifen?
\item In welchem Umfang besteht Wissen über die Öffnung von Wissenschaft unter den Wissenschaftlern und Wissenschaftlerinnen?
\item Welches Verständnis von Open Access besteht unter den Befragten?
\item Wie stark ist das Interesse an Forschungsdaten?
\item Welche Faktoren und Argumente begünstigen die Öffnung von Wissenschaft in der jeweiligen wissenschaftlichen Disziplin, welche Argumente sprechen dagegen?
\item Wie wird der geschätzte Aufwand für die Öffnung von Wissenschaft in einer wissenschaftlichen Disziplin eingeschätzt?
\item Welche unterschiedlichen Auffassungen bestehen zwischen den unterschiedlichen Fachdiziplinen, Alters- und Statusgruppen?
\item In welchem Umfang wird bereits heute im wissenschaftlichem Umfeld offen kommuniziert?
\item Welche Veränderungen beim Zugang zur Literatur wie auch bei den Veröffentlichungsstrategie sind im Vergleich zur der 2007 und 2008 durchgeführten Befragung des SOFI Göttingen zu erkennen?
\end{itemize}

\section{Erhebungsmethode und Messinstrumente}

Folgende Überlegungen wurden bei der Auswahl der Erhebungsmethode angestellt: Persönliche Interviews erschienen wenig geeignet, da der damit verbundene personelle, zeitliche und finanzielle Aufwand als zu hoch eingestuft wurde. Gegen eine postalische Befragung sprachen die hohen Kosten (unter anderem Porto) sowie die häufig geringen Rücklaufquoten. Darüber hinaus wurde in der vorangegangenen Studie durch das SOFI Göttingen ebenfalls auf das Internet als primäre Quelle für die Identifikation von Teilnehmern und Teilnehmerinnen und E-Mail als Kontaktaufnahmekanal zurückgegriffen. Darüber hinaus hat die zunehmenden Verbreitung und Nutzung des Internets, hat die elektronische Online-Befragung längst Eingang in die empirische Sozialforschung gefunden \cite{Pannewitz_2002}.

Auschlaggebend für die Auswahl dieser Befragungsform war der ökonomische Aspekt, da es die Online-Befragung "einfach macht, große Stichproben in kurzer Zeit zu erheben" \cite{eichhorn_2004_online} und diese Methode eine Beantwortung der Fragen jederzeit im Internet möglich macht. Gleichzeitig ermöglichte diese Form der Befragung, die einfache Verbreitung am Institut und unter Kolleginnen und Kollegen. In den Insturktionen des Fragebogens wurde aber auch die Option angeboten, den Fragebogen analog zu erhalten. Der direkte Download des Fragebogens wurde nicht angeboten um die Umfrageergebnisse nicht im Vorhinein zu beeinflussen.

Weitere Vorteile bei der Methode der Online-Datenerhebung ist die unabhängige und einfache Teilnahme der Befragten, wobei davon ausgegangen wurde, dass die notwendigen technischen Voraussetzungen zur Teilnahme an einer Internetbefragung bei allen Wissenschaftlern an deutschsprachigen Wissenschaftseinrichtungen gegeben ist. Es kam nur zu drei expliziten Verweigerungen der Teilnahme an der Befragung: In einem Fall gab es Institutsbeschluss nicht mehr an Befragungen teilzunehmen. In einem weiteren Fall wurde die Methode der sozialwissenschaftlichen Befragung grundsätzlich abgelehnt und in dem dritten Fall mit dem Verweis auf zu hohen Aufwand für das Ausfüllen von Fragebogen beantwortet.

Die Anonymität der Befragen wurde jederzeit gewahrt und keine eindeutigen persönlichen Daten erhoben, die einen Nutzer direkt identifizierbar machen. Obwohl die Ergebnisse nach Abschluss der Befragung anonymisiert veröffentlicht wurden keine Informationen und Ergebnisse veröffentlicht, die Rückschlüsse auf individuelle Teilnehmer an der Befragung zulassen. Darauf wurde auch im Pretest explizit eingegangen.

\subsection{Untersuchungsobjekte}

Die Teilnehmer des Fragebogens waren primär deutschsprachige Wissenschaftler und Wissenschaftlerinnen aus sämtlichen Fachdisziplinen oder Mitarbeiter des wissenschaftlichen Betriebs aus dem deutschsprachigen Raum die im Zeitraum vom 18.8.2014 bis 18.01.2015 online befragt wurden. Bibliothekare und Bibliothekarinnen (0,95 Prozent der Befragten), sowie Studierende (3,68 Prozent Befragten) wurden zwar nicht direkt angesprochen, konnten aber dennoch an der Umfrage teilnehmen. Mit dem Start der Befragung wurden dazu 4002 Wissenschaftlerinnen und Wissenschaftler per E-Mail im Zeitraum vom 18.08.2014 bis 18.01.2015 angeschrieben.

Die Auswahl der jeweiligen Fächer beruht auf der aktuellen Auflistung der Fachsystematik der Deutschen Forschungsgemeinschaft (DFG). Da die Erhebung fächerübergreifend angelegt war, um die Unterschiede zwischen den Disziplinen zu evaluieren, wurden Wissenschaftler aus alle dort gelisteten Fachdisziplinen angefragt. Per Zufall wurdnen dazu von den Institutswebseiten im deutschsprachigen Raum pro Fach 150 Wissenschaftler und Wissenschaftlerinnen per E-Mail angeschrieben und um Teilnahme an der Befragung gebeten. 1.768 der Angefragten haben an der Umfrage teilgenommen und den Fragebogen gestartet, 1.467 haben mindestens eine Frage beantwortet und teilweise an der Befragung teilgenommen. 301 Personen haben vor Beantwortung der ersten Fragegruppe abgebrochen. Die Rücklaufquote liegt somit bei 44,18 Prozent brutto beziehungsweise bei 36,67 Prozent netto. 1.112 der 1.768 Teilnehmer und Teilnehmerinnen (62,89 Prozent), die die Befragung gestartet haben, haben den Online Fragebogen vollständig beendet. Nach Beendigung des Umfragezeitraums haben demnach 656 Personen (37,10 Prozent) den Online-Fragebogen vor Beendigung abgebrochen.

Die hohe Resonanz ist vermutlich auf die persönliche Ansprache sowie die konkrete Zuordnung zur Fachdisziplin im Anschreiben zurückzuführen. Dabei handelt es zwar um ein aufwendiges Vorgehen, hat aber sicherlich zu dieser guten Quote beigetragen. Die Fragebögen, deren Beantwortung vor Beendigung aller Fragen abgebrochen wurde, bleiben in der weiteren Betrachtung unberücksichtigt.

Um die Repräsentativität der Studie sicherzustellen wurden die Rückläufer der Befragung auf die vorhandenen Informationen wie die fachliche Zuordnung, Beruflicher Status und Alter ausgewertet. Verschiedenen Verzerrungen sind nur zu vermuten, da die kontaktierten Menschen ausschließlich online angeschrieben wurden. Da die Umfrage jedoch ohne Zugangsbeschränkung öffentlich online ausgefüllt werden konnte, war es jedem Interessenten möglich teilzunehmen.

\subsection{Untersuchungsmaterial}

Für die Durchführung der Online-Befragung wurde die Open Source Software LimeSurvey Version 2.05+ verwendet, die auf einem Webserver des Centre for Digital Cultures installiert wurde. Die Software ist weit verbreitet und ermöglicht umfassende Einstellungs- und Anpassungsmöglichkeiten. So konnten zum Beispiel ein Teil der Fragen in Abhängigkeit von Antworten auf vorherige Fragen kontextsensitiv definiert werden. Die Software ermöglichte es die beantworteten Fragebögen aus der Verwaltungsoberfläche einzeln oder zusammengefasst einzusehen und für die Auswertung zu exportieren. Neben den üblichen Möglichkeiten der Durchführung der Befragung an internetfähigen Endgeräten wurde die Darstellung der Befragung darüber hinaus so angepasst, dass die Darstellung und die Beantwortung des Fragebogens auch auf internetfähigen Mobiltelefonen möglich war. Des Weiteren wurde bei dem Design des Fragebogens und der Anpassung der Dartstellung der Software explizit darauf geachtet, dass alle Texte einfach und angenehm lesbar waren, damit die Beantwortung der Fragen einfach und strukturiert ablaufen konnte.

Die Ergebnisse wurden in der Datenbank des Servers des Centres for Digital Cultures zwischengespeichert und am xx.xx.2015 gelöscht. Nach Abschluss der Befragung wurden die Datensätze anonymisiert. Dazu wurden sämtliche persönliche Daten, wie zum Beispiel E-Mailadressen entfernt und die freiwillige personenbezogene Angabe von dem Rest der Daten getrennt. Folgende Felder wurden entfernt beziehungsweise getrennt, neu angeordnet und aggregiert veröffentlicht: Geschlecht, Alter, weitere Aspekte zum Thema, Anmerkungen und Kritik, Funktion im Rahmen eines Open Access Engagements (optional), Antwort ID und Zeitpunkt der Beantwortung. Die anonymisierte Datensätze wurden nach Abschluss der Befragung im Januar 2015 auf dem datorium-Datenrepositorium des GESIS - Leibniz-Institut für Sozialwissenschaften, auf dem Forscher Daten und Publikationen einstellen können, überprüft und veröffentlicht. Eine weitere Veröffentlichung der Daten erfolgte auf dem Datenrepositorium Zenodo.

\subsection{Aufbau des Fragebogens und Untersuchungsdurchführung}

Für die Befragung durch das SOFI wurden 6500 Wissenschaftler und Wissenschaftlerinnen angefragt, von denen 1803 geantwortet haben. Der 2007 verwendete Fragebogen bestand aus 51 Fragen. Im ersten Teil des Fragebogens wurden Fragen zu dem Fachgebiet und Tätigkeitsbereich zunächst als Leserin beziehungsweise Leser wissenschaftlicher Publikationen erfasst. Im zweiten Teil wurden die Teilnehmer aus der Perspektive als Autorin beziehungsweise als Autor befragt. Abschließend wurden noch eineige personenbezogene Angaben abgefragt. \cite{Hanekop_Wittke_2007_Fragebogen} Die Skalen und Antwortformate zur Beantwortung der Fragen waren unterschiedlich ausgewählt.

Zu Beginn der Fragebogenkonstruktion wurde der Fragebogen und das Datenmaterial der Vorbefragung einer Itemanalyse zum Ausschluss unpassender Fragen (Items) unterzogen und Fragen bezüglich der Fragestellung dieser Arbeit hinzugefügt. Dafür wurden die veröffentlichten Antworten analysiert. Fragen, die stark ungleich verteilt waren, wurden, wenn sie nicht inhaltlich interessant erschienen, ausgeschlossen oder mit anderen Fragen zusammengelegt. Somit wurden auf der Basis der Analyse der Fragen der Fragepool auf 32 Fragen reduziert beziehungsweise verändert. Acht der insgesamt 40 Fragen bedingen Antworten aus vorherigen Fragen und wurden deshalb nicht allen Teilnehmern gestellt. Die Reihenfolge der Fragen und der Fragengruppen wurden so ausgewählt, dass sie strukturiert abgebildet werden, der Reihenfolge-Effekt minimiert wird und die Beantwortung bis zum Ende interessant bleibt. Insgesamt wurden bei dem Aufbau des Fragebogens die Aufzählung der Richtlinien zur Formulierung der Items nach Bortz und Döring \cite{raab_2012_fragebogen} berücksichtigt. Das 75 Prozent der Befragten, die mindestens eine Frage beantwortet haben, auch den gesamten Fragebogen vollständig beantwortet haben, verdeutlicht den Erfolg der Vorbereitung und Anpassung der Daten.

Die Qualität und Brauchbarkeit des Fragebogens wurde in einem Pretest (Probedurchlauf) mit wissenschafltichen Mitarbeitern und Mitarbeiterinnen aus dem Arbeitsumfeld des Autors überprüft. Die Einleitung für den Fragebogen, die Instruktionen und die Anrede wurden ebenfalls im Pretest evaluiert und optimiert, da sie sehr viel "zur Motivation der Bearbeitung beteitragen kann" \cite{raab_2012_fragebogen}. Dazu wurde der Fragebogen an 15 Wissenschaftler im Testmodus übermittelt und unter der Instruktion des "lauten Denkens" um Bearbeitung des Fragebogens gebeten \cite{raab_2012_fragebogen}. Nach dem Pretest wurden die Befragung um weitere Fragen angepasst, die sich auch auf die Veröffentlichung von wissenschaftlichen Informationen und Daten beziehen.

Im finalen Fragebogen kamen die Antwortformate: Offene Fragen, geschlossene Fragen und Mischformen mit offenen und vorgegebenene Kategorien, sowie freie (offenen) Antwortformate zum Einsatz. Es wurde versucht weitesgehend auf Ratingskalen zu verzichten. Insgesamt wurden in dem Fragebogen drei fünfstufige Ratingskalen mit verbaler Skalenbezeichnung eingesetzt. Die Charakterisierungen der Abstufungen wurde zur Vergleichbarkeit aus der SOFI-Befragung von 2007 übernommen.

Die Gliederung war ebenfalls an die Befragung aus den Jahren 2007 angelehnt und nur leicht ergänzt. Insgesamt wurden die 40 Fragen in 5 Fragegruppen plus eine abschließende Fragegruppe für persönliche Angaben und Anmerkungen und Kritik unterteilt. In der ersten Fragegruppe wurde auf die Rahmenbedingungen der Teilnehmenden sowie deren wissenschaftlichen Tätigkeit eingegangen. In der zweiten Fragegruppe wurden Aspekten aus der wissenschaftlichen Leserperspektive evaluiert. Die dritte Fragegruppe beschäftigte sich mit dem Zugang zu wissenschaftliche Informationen, gefolgt von der vierten, die aus Fragen bezüglich des Zugangs zu wissenschaftlichen Informationen und des Zugriffs auf wissenschaftliche Daten bestand. In der fünften Fragegruppe wurden Fragen aus der Perspektive des Autors oder der Autorin von wissenschaftlichen Inhalten gestellt. Abschließend folgte die Erhebung weiterer Daten zur eindeutigen Segmentierung der Teilnehmer und Teilnehmerinnen. Die Befragten wurden vor Start der Befragung auf die Gliederung des Fragebogens und die Reihenfolge der Fragegruppen, sowie die Rahmenbedingungen des Fragebogens, wie die anonyme Behandlung der Daten, hingewiesen.

Nach der Auswertung und Einarbeitung der Anmerkungen der Pretester wurde der Fragebogen "Wissenschaftliche Kommunikation im Rahmen der Digitalisierung" wurde nach Vorbereitung am 18.08.2014 unter http://umfrage.offene-doktorarbeit.de veröffentlicht. Nach der Veröffentlochung wurden nach Zufallsprinzip jeweils ca. 150 Wissenschaftler und Wissenschaftlerinnen je Fachdisziplinen aller DFG-Fachkollegien identifiziert. Die Namen und E-Mail-Adressen zu den Personen waren über die Internetseite der Hochschulen und wissenschaftlichen Organisationen öffentlich zugänglich. Die Kontaktaufnahme zu den ausgewählten Personen erfolgte per personalisierter E-Mail mit einem Hinweistext, Instruktionen und einem direkten Link auf die Webadresse des Fragebogens. Vereinzelt wurden auch Sekretariatsadressen verwendet und um Weiterleitung gebeten. Alle identifizierten Kontakte wurden nur einmal angeschrieben. Zusätzlich wurde der Umfrage-Link mit einer kurzen Information zur Umfrage auf offene-doktorarbeit.de veröffentlicht, sowie über die privaten Social-Media Kanäle des Autors und an persönliche Kontakte des Autors versendet. Des weiteren wurden eine generalisierte Einladung über wissenschaftliche Mailinglisten, sowie den Newsletter des Centre for Digital Cultures verbreitet. Um eine Möglichst große Verbreitung zu erreichen, hatten die Teilnehmer und Teilnehmerinnen nach Abschluss des Fragebogens die Möglichkeit einen Link auf die Befragung  über die sozialen Kanäle und per E-Mail weiterzuverbreiten. Ausserdem befand sich dort ein Hinweis auf die Webseite des Promotionsvorhabens.

\section{Kritische Betrachtung und Beurteilungsfehler}

Immer wieder kommt es bei dem Prozeß der Erstellung von Fragebögen oder bei der Beurteilung der Daten zu Störungen, zu sogenannten Beurteilungsfehlern. Deshalb soll die Güte der Befragung durch die Gütekriterien Objektivität, Reliabilität und Validität beurteilt werden.

\subsection{Objektivität}

Die Unabhängigkeit beschreibt das Ausmaß, in dem das Ergebnis der Untersuchung frei und unabhängig von Einflüßen ausserhalb der befragten Person ist \cite{rost_2004_lehrbuch}. Die Objektivität der durchgeführten Befragung ist dadurch gegeben, als dass durch die elektronische Onlinebefragung eine zeitliche und räumliche Unabhängigkeit gewährleistet wurde. Die Befragung wurde für alle Teilnehmer nach identischer Anrede, Einladung und Instruktion und ohne Untersuchungsleiter durchgeführt und war nicht von besonderen Situationsvariablen abhängig.

\subsection{Reliabilität}

"Ein Test als Messinstrument ist reliabel, wenn er das, was er mißt, genau misst." \cite{schelten_1997_testbeurteilung} Die Reliabilität gibt den Grad der Genauigkeit an, mit der durch die empirische Datenerhebung ein Merkmal erfasst wird \cite{rost_2004_lehrbuch}, unabhängig davon was er erfasst. Sie spiegelt die Replizierbarkeit von Messergebnissen und Zuverlässigkeit einer Datenerhebung wieder.

--- TODO weiter ausarbeiten----

\subsection{Validität}
In der Literatur wird in zwei Typen der Validität unterschieden \cite{rost_2004_lehrbuch}. Die interne und die externe Validität. Von einer hohen internen Validität wird ausgegangen, wenn die erzielten Ergebnisse klar und eindeutig interpretierbar sind \cite{raab_2012_fragebogen}. Im Rahmen der durchgeführten Befragung zeigt die Validität, ob das Messinstrument Fragebogen wirklich das misst was dazu beiträgt, die Fragestellungen der Arbeit zu beantworten. Die Validität wurde durch die Übernahme der Grundstruktur und von Items der Studie "Wissenschaftliche Publikationen im Internet: Wissenschaftler als Leser und Autoren" des SOFI in Göttigen gewährleistet. Die Validität der neu erstellten, angepassten  und zusammengelegten Items wurde durch Auswertung des Pretests an der auch eine Wissenschaftlerin der Studie des SOFIs teilgenommen hat sichergestellt.

\section{Ergebnisse der Befragung}

Im Zeitraum vom 18. August 2014 bis 18. Januar 2015 haben 1.768 Personen die Befragung zur wissenschaftlichen Kommunikation im Rahmen meines Promotionsvorhabens gestartet. 1.446 Teilnehmer haben die Umfrage teilweise und 1.112 komplett abgeschlossen.

Die erhobenen Daten der 1.112 Teilnehmer des Online-Fragebogens werden mit Hilfe der computerunterstützten Datenaufbereitung statistisch ausgewertet.

\subsection{Demographische und persönliche Daten}

In die Auswertung der Befragung gingen die Angaben der 1.112 Teilnehmer des Online-Fragebogens ein, die alle Fragen beantwortet haben. Zu den erhobenen persöhnlichen Daten:

\begin{itemize}
\item \textbf{Geschlecht:} 444 der Befragten waren weiblich (39,9 Prozent), 606 oder 54,5 Prozent männlich und 62 Personen oder 5,6 Prozent machten keine Angabe zu ihrem Geschlecht gemacht.
\item \textbf{Alter:} Die prozentuale Verteilung des Alters gestaltete sich wie folgt: 4,4 Prozent (46) waren zum Zeitpunkt der Befragung jünger als 31 Jahre, die größte Altersgruppe mit 31,2 Prozent stellten die 31 bis 40 Jährigen, 17,4 Prozent der Befragten waren zwischen 41 und 50 Jahren alt, 14,6 waren älter als 50 Jahre. 0,7 Prozent machten bei der Frage nach ihrem Alter keine Angaben.
\item \textbf{Berufstatus:} Unter den Befragten gaben 24,8 Prozent an, Privatdozenten, Juniorprofessoren oder Professoren zu sein. 55,9 Prozent der Teilnehmer waren wissenschaftliche Mitarbeiter, 20,0 Prozent wissenschaftliche Mitarbeiter und Doktoranden, 22,8 Prozent promovierte wissenschaftliche Mitarbeiter und 13,1 Prozent Mitarbeiter ohne Promotionsvorhaben oder abgeschlossener Promotion. 10 Teilnehmer (0,9 Prozent) gaben an Wissenschaftler in der Privatwirtschaft zu sein. 35 Befragte (3,2 Prozent) wurden unter Sonstiges subsummiert.
\item \textbf{Tätigkeitsdauer in der Wissenschaft:} Nur 6,2 Prozent der Befragten gaben an "weniger als 1 Jahr" in der Wissenschaft tätig zu sein, 20,4 Prozent seit mehr als einem aber weniger als drei Jahre. Drei bis sechs Jahre waren laut der Auswertung der Befragung 24,0 Prozent. 15,4 Prozent der Teilnehmer und Teilnehmerinnen war mehr als sechs aber weniger als zehn Jahre wissenschaftlich tätig. Die größte Gruppe gab an, "mehr als 10 Jahre" wissenschaftlich tätig zu sein (31.8 Prozent). 1,5 Prozent waren "nicht in der Wissenschaft tätig" und 0,7 Prozent machten keine Angaben.
\item \textbf{Forschungseinrichtung:} Die große Mehrzahl der Teilnehmer (78,06 Prozent) gaben an einer deutsche Universität/Hochschule tätig zu sein. Die zweitgrößte Gruppe stellten mit 5,31 Prozent die 59 Befragten, die an einem Institut der Leibniz-Gemeinschaft tätig sind. 4,77 Prozent gaben an an einer "Sonstigen" Einrichtung tätig zu sein. Nur 1,4 Prozent der Befragten waren an einem Max-Planck-Institut und 0.4 Prozent an einem Institut der Fraunhofer Gesellschaft tätig. An einer Universität/Hochschule im deutschsprachigen Ausland waren 4,1 Prozent und im nicht nicht-deutschsprachigen Ausland 1,4 Prozent tätig. 1,3 Prozent arbeiteten an einer deutschen Fachhochschule. 11 Befragte (1 Prozent) gaben an einem „An“-Institut (eigenständige Forschungseinrichtung, angegliedert an einer deutschen Hochschule) zu arbeiten.
\end{itemize}

Die größte Gruppe der Teilnehmerinnen und Teilnehmer (37,59 Prozent) sind in der Fachgruppe der Geistes- und Sozialwissenschaften verortet. 28,96 Prozent gaben an, in den Naturwissenschaften tätig zu sein. Aus den Lebenswissenschaften kamen 17,72 Prozent der Befragten. Die kleinste Gruppe unter den Teilnehmern stellten mit 12,68 Prozent Wissenschaftler aus der Fachgruppe der Ingenieurwissenschaften dar. 34 der Befragten (3,1 Prozent) konnten nicht eindeutig einer der vier Fachgruppen zugeordnet werden.

--- Todo: Verteilung auf Fächer und Grafik bauen ----

Die überwiegende Mehrheit der Befragten (97,4 Prozent) gab an in der Forschung tätig zu sein. Mehr als die Hälfte aller gab an "überwiegend" (53.4 Prozent) in der Forschung zu arbeiten. Demgegenüber gaben 1,7 Prozent oder 19 Personen an "gar nicht" in der Forschung tätig zu sein und 2,6% machten keine Angabe. 9,6 Prozent gab an überwiegend in der Lehre tätig zu sein. Insgesamt ware 83,6 Prozent lehrend tätig. 51 der 1.112 Teilnehmer und Teilnehmerinnen der Umfrage (4,5 Prozent) gab an in der klinischen Versorgung tätig zu sein, 2,2 Prozent überwiegend. "Überwiegend" Administrativ waren nur 4,5 Prozent, 13,0 Prozent "gleichgwichtig" und 31,3 Prozent "weniger". 4,1 Prozent machten gaben an im Sonstigen-Bereichen tätig zu sein, 93,3 Prozent machten dazu im Freitextfeld genauere Angaben.

--- Todo: Arbeitsbereich Grafik bauen ----

\subsection{Veränderungen wissenschaftlicher Kommunikation durch die Digitalisierung}

Diese Arbeit folgt der These von Heidemarie Hanekop, dass Steuerung der Aufmerksamkeit der Wissenschaftler, die durch die neuen Such- und Rezeptionsmöglichkeiten beeinflusst wird, eine wichtige Schnittstelle zwischen informeller und formeller Kommunikation spielt\cite{Hanekop_2014}. Der Einfluss der Digitalisierung auf das wissenschaftlichen Kommunikationssystem lässt sich demnach auf der Grundlage der Entwicklung der Nutzung von Such- und Rezeptionsmöglichkeiten nach Literatur untersuchen.

"Um sich im eigenen Fach auf dem Laufenden zu halten", nutzte laut der Befragung im Jahr 2007/2008 79 Prozent häufig oder sehr häufig Online-Journale. 2014/2015 stieg die relative Anzahl auf 88,4 Prozent. Im Gegenzug nutzen nur noch 32,2 Prozent die Printausgaben von Zeitschriften, 2007/2008 waren es noch 50 Prozent der Befragten die auf die analoge Version der Journale zurückgriff.

Auch bei der Frage welche der Suchmöglichkeiten häufig genutzt werden, um gezielt nach Literatur zu suchen, lässt sich ein klare Veränderung hin zu den digitalen Möglichkeiten feststellen. Ähnlich wie in der Befragung 2007/2008 durch das SOFI gab auch 2014/2015 weniger als ein Viertel der Befragten an über die "Konventionelle Suche" (in Bibliotheksregalen, Archiven etc.) nach Literatur zu suchen. Demgegenüber stieg der Anteil von Wissenschaftlerinnen und Wissenschaftlern, die angaben den Dienst Google Scholar für die Literatursuche zu verwenden, von 31 Prozent in 2007/2008 auf 51,9 Prozent in 2014/2015.

Diese Ergebnisse belegen erneut \cite{Hanekop_2014}, dass sich die webbasierten Such- und Rezeptionsmöglichkeiten in der wissenschaftlichen Kommunikation durchgesetzt haben.

\subsection{Verständnis von Offenheit bei der wissenschaftlichen Kommunikation}

16,9 Prozent der Befragten gaben an, sich häufig über Open-Access Repositorien (z.B. arxiv.org) auf dem Laufenden zu halten, 17,1 Prozent oder 190 der 1.112 Befragten nutzen Open-Access Portale (z.B. Directory of Open Access Journals) um sich über den aktuellen Stand der Forschung zu informieren. Bei der Suche nach Literatur nutzten nur 4,3 Pozent häufig über Suchmaschinen für Open Access, aber mehr als 50 Prozent in fachspezifische Suchportalen, die ebenfalls Open-Access Publikationen enthalten.

54,5 Prozent gaben an, sie finden "die Forderung nach kostenfreiem Zugang zu allen wissenschaftlichen Publikationen für Leser (Open Access)" sehr gut. Knapp unter einem Viertel (22,3 Prozent) finden die Forderung "gut". 19,2 Prozent waren sich bei der Frage unsicher und antworteten mit "teils/teils"und 38 der Befragen lehnten die Forderung nach Open Access ab, 9 davon sogar "entschieden". Nur 7 Teilnehmer und Teilnehmerinnen oder 0,6 Prozent gaben an Open Access nicht zu kennen.

Bei der geneaueren Betrachtung der Meinung zu Open Access nach beruflichen Status, stellen die Doktoranden die größte Gruppe (89,3 Prozent) der Befürworter dar. Mit 80,32 Prozent finden auch die promovierten wissenschaftlichen Mitarbeiter und Mitarbeiterinnen Open Access "gut" oder "sehr gut". Bei den Wissenschaftlichen Mitarbeitern ohne Promotion (75,3 Prozent), bei Doktoranden mit einer wissenschaftlichen Mitarbeiterstelle (73,0 Prozent) unterstützen ist und bei Privatdozenten (73,0 Prozent) sowie bei Professoren (71,7 Prozent) ist die Forderung nach freien und offenen Zugang zu wissenschaftlichen Publikationen ebenfalls mehrheitlich stark ausgeprägt.  Am geringsten ist die Befürwortung unter Juniorprofessoren (65,4 PRozent) ausgeprägt.

14,8 Prozent sind in der Open Access Bewegung engagiert, wobei 72,3 Prozent die Aussage verneinten ein Engagement in der OA-Bewergung und 13,0 Prozent enthielten sich der Angabe. 43,4 Prozent oder 483 Personen kommentierten Ihre Meinung zu Open Access, wobei 72,9 Prozent der abgegebenen Kommentare von Befragten stammten, die Open Access gut oder sehr gut finden. Unter den Befragten, die Open Access ablehnen, kommentierten 75,7 Prozent ihre Haltung zur Forderung nach kostenfreiem Zugang zu allen wissenschaftlichen Publikationen. Von den der Personen, die "teils/teils" angaben, kommentierten fast die Hälfte (48,1 Prozent) ihre unsichere Haltung. 159 der 854 Befragten (18,6 Prozent), die Open Access gut oder sehr gut finden, gaben an, selbst aktiv in der Open Access Bewegung zu sein. Überraschend sind an dieser Stelle 12 Personen (5,6 Prozent) die "teils/teils" bezüglich ihrer Meinung zu der Forderung nach Open Access angegeben haben, aber sich dennoch zum Teil der Bewegung zählen.

In der folgenden Frage wurde das Verständnis von Open Access nach der Definition der Budapest Open Access Initative \cite{boai_2012} abgefragt. Knapp drei Viertel der Teilnehmer (74,9 Prozent) stimmten dieser Definition uneingeschränkt zu, 19,0 Prozent waren sich unsicher, 2,4 Prozent lehnten die Definition ab, 3,2 Prozent beantworteten die Frage mit "weiß nicht" und fünf Teilnehmer enthielten (0,5) sich der Aussage. Wurde "teils/teils" oder "weiß nicht" als Antwort ausgewählt konnte in einer optionalen Freitext-Frage beantwortet werden, welche Aspekten der Definition genau keine Zustimmung und welche Zustimmung fanden. Davon machten 37,7 Prozent der möglichen Befragten gebrauch.

\subsection{Interesse an Offenheit bei der wissenschaftlichen Kommunikation}

Die Mehrheit der Befragten gab an Interesse am Zugang zu Forschungdaten anderer Wissenschaftler und Wissenschaftlerinnen zu haben (71,3 Prozent). Über ein Drittel dieser Teilnehmer und Teilnehmerinnen erklärten ihr genaues Interesse in einer optionalen Frage. 28.7 Prozent oder 319 der Befragten hatte kein Interesse am Zugang zu Forschungsdaten anderer Wissenschaftler_innen.

Die Frage, ob sie die befragten Wissenschaftler vorstellen können ihre "Forschungsdaten und alle weiteren Informationen, die während der wissenschaftlichen Arbeit anfallen (z.B. Laborbücher, Entwürfe oder andere Dokumente und Daten) unter Berücksichtigung von Datenschutz öffentlich zur Verfügung zu stellen", beantworteten 28,0 Prozent uneingeschränkt mit "ja" und 36.3 Prozent schränkten ein, dass sie das "nur unter bestimmten Bedinungen" tun würden. 29,0 Prozent lehnten diese Frage ab und 6,7 Prozent wußten darauf keine Antwort. Die bedingte und freie Frage nach der Erläuterung "bestimmten Bedingungen" beantworteten 214 Teilnehmer und Teilnehmerinnen.

\subsection{Offenheit im wissenschaftlichen Alltag}

Dem großen Verständnis von Open Access, der mehrheitlichen Unterstützung der Forderung nach Öffnung von Wissenschaft dem Interesse an Forschungsdaten anderer steht die Frage gegen über, wie wichtig den Wissenschaftlern das Kriterium freier Zugang zum Volltext bei den eigenen Veröffentlichungen ist. Die größte Gruppe der Befragten (49,8 Prozent) erachten dies als "weniger wichtig" oder "nicht wichtig". Demgegenüber erachteten nur 44,7 Prozent das Kriterium "freier Zugang zum Volltext im Internet" als wichtig oder sehr wichtig bei der eigenen Veröffentlichung. 5,5 Prozent der 1.112 Befragten haben die Frage nicht beantwortet.

Diese Zahlen werden bei der weiteren Betrachtung der Kriterien, die Wissenschaftler bei der Veröffentlichung eines Beitrags als wichtig oder sehr wichtig erachten bestätigt. Während der fachlich einschlägige Schwerpunkt (91,2 Prozent), das Renommee der Zeitschrift/des Verlags (81,7 Prozent) und akzeptable oder keine Veröffentlichungskosten für Autoren (79,2) relativ am häufigsten als wichtig oder sehr wichtig erachtet wird, stellt die Veröffentlichung unter einer Open-Access Lizenz für nur 33,1 Prozent ein wichtiges oder sehr wichtiges Kriterien bei der Veröffentlichung eigener Inhalte dar. 56,4 Prozent erachten dieses Kriterium als weniger wichtig oder nicht wichtig. Der akzeptable Preis der Publikation spielt für 43,4 Prozent eine Rolle, für 50,0 Prozent der Befragen ist er weniger wichtig oder nicht wichtig.

Weitere Kriterien nach der Wichtigkeit aus Sicht der Befragten sortiert:
\begin{itemize}
\item 77,8 Prozent der Teilnehmer der Studie sehen die internationale Verbreitung als mindestens wichtige, wenn nicht sogar sehr wichtiges Kriterium im Rahmen der eigenen Veröffentlichungen an. 19,4 finden dieses weniger wichtig oder unwichtig.
\item Das Peer-Review Verfahren wird von 75,4 Prozent als wichtiges Kriterium erachtet. Nur 18,6 Prozent der Befragten sind gegensätzlicher Meinung.
\item 75,4 Prozent der befragten Personen sehen die Transparenz des Review-Prozesses als wichtig an, 17,9 Prozent nicht.
\item Eine leichte Auffindbarkeit der Publikation im Internet ist 71,23 Prozent wichtig. 25,0 Prozent ist das weniger bis nicht wichtig.
\item Eine rasche Publikation der eigenen Publikation wollen 68,0 Prozent. 28,7 Prozent sehen das anders.
\item Rankings, wie der Impact-Faktor der Zeitschrift, erachten 58,4 Prozent als wichtig und 35,1 Prozent als weniger wichtig oder unwichtig.
\item Die Reputation der Herausgeber ist 48,3 Prozent ein wichtiges Kriterium, für 46,5 Prozent eher unwichtig bis nicht wichtig.
\end{itemize}

--- Todo: Grafik bauen ----

Eine weitere Frage im Fragebogen betraf die Auffasssung der Befragten, ob ihre Veröffentlichungen in Zeitschriften oder Büchern für potentielle Leser gut zugänglich sind. 32,0 Prozent waren der Meinung, dass ihre Publikationen gut zugänglich sind, mit "teils/teils antworteten 46,9 Prozent und 11,5 Prozent mit "nein, nicht so gut zugänglich" (9,2 Prozent) oder "nein, sehr schlecht"(2,3 Prozent). 107 oder 9,6 Prozent wussten die Frage nicht zu beantworten.

Bei der Frage, ob Aufsätze, Texte oder Bücher publiziert wurden, die vom Verlag selbst frei zugänglich gemacht wurden, antworteten 140 Teilnehmer und Teilnehmerinnen (12,6 Prozent) mit "ja, einen Beitrag" und 23,1 Prozent mit "ja, mehrere Beiträge". 54,4 Prozent oder 605 der Befragten, hatte zum Zeitpunkt der Befragung noch keine Aufsätze, Texte oder Bücher publiziert, die vom Verlag selbst frei zugänglich gemacht wurden.  18,5 Prozent derer, die bisher noch nicht bei einem Verlag Open Access veröffentlicht hatten, gaben an dies zu planen. 9,9 Prozent der Befragten beantworteten die Frage nicht.

Die 397 Befragten, die angegeben haben Inhalte frei publiziert zu haben, hatten die Möglichkeit anzugeben, wieviele Aufsätze, Texte oder Bücher die befragten Wissenschaftler das waren:
\begin{itemize}
\item Bücher - 63 Befragte (15,9 Prozent) beantworteten die optionale Frage, 26 gaben an, bisher kein Buch veröffentlicht zu haben, das frei zugänglich gemacht wurde. Bezieht man die Antworten nicht mit ein, die 0 Bücher angaben, hatten die 37 Befragten ca. 2 Bücher veröffentlicht, die vom Verlag selbst frei zugänglich gemacht wurden. 334 der möglichen Befragten machten keine Angaben.
\item Texte - 192 der 397 Befragten (48,4 Prozent) gab an mindestens 1 Text frei veröffentlicht zu haben. Im Durchschnitt veröffentlicheten die Befragten rund 3 Texte "frei zugänglich".
\item Daten - 2,5 Prozent (10) der möglichen Befragen gab an dazu, mindestens einen Datensatz frei veröffentlicht zu haben, als 2,4 Datensätze pro Person.
\item Sonstiges - Keiner der Teilnehmerinnen und Teilnehmer gab an sonstige Beiträge frei veröffentlicht zu haben.
\end{itemize}

Den Aufwand, die eigene Publikationen im Internet frei zur Verfügung zu stellen, schätzte der größte Teil der Befragten (30,8 Prozent) als gering ein. 255 der Befragten Wissenschaftler und Wissenschaftlerinnen (22.9 Prozent) schätzen den Aufwand ihre Publikationen im Internet frei zur Verfügung zu stellen als mittelgroß ein.  22,6 Prozent waren sich unsicher und wählten "teils/teils" und 18.6 Prozent wusste nicht den Aufwand einzuschätzen. 5.1 Prozent schätzten den Aufwand als "groß" ein.

Während bei der Veröffentlichung dr Publikationen die Mehrheit der Befragten die Veröffentlichung als nicht groß einschätzte, zeichnete sich bei der Auswertung der Frage nach dem geschätzten Aufwand für die Veröffentlichgung von Forschungsdaten im Internet ein anderes Bild. 55,0 Prozent der Befragten schätzte den Aufwand die Forschungdaten zu veröffentlichen als "groß" ein. Die kleinste Gruppe der Befragten (9,6 Prozent) vermutete dabei einen geringen Aufwand, 15,0 Prozent schätzten den Aufwand "teils/teils" ein und ein Fünftel (20,3 Prozent) wusste die Frage nicht zu beantworten.

--- Todo: Grafik aus beiden Items bauen ---

\subsection{Treiber und Bremser für die Öffnung von Wissenschaft und Forschung}

In Kapitel --- TODO: definieren --- wurden Treiber und Bremser für die ÖFfnung von Wissenschaft und Forschung in der Literatur identifiziert und herausgearbeitet. Diese wurden im Rahmen der empirischen Erhebung abgefragt und geklärt werden, welche Faktoren und Argumente aus Sicht von Wissenschaftlern die Öffnung von Wissenschaft in der jeweiligen wissenschaftlichen Disziplin begünstigen und welche sie behindern sprechen.

Die Verteilung der Antwortmöglichkeiten auf die Argumente für die Öffnung der Wissenschaft und Forschung seitens der Befragten waren:
\begin{enumerate}
\item Die Beschleunigung der Wissensverbreitung und -verwertung, 721 mal wurde diese Antwortmöglichkeit ausgewählt.
\item Gefolgt von dem Argument der "Eröffnung neuer Möglichkeiten", das 63,8 Prozent der 1.112 Befragten unter den Antwortmöglichkeiten auswählten.
\item Dem Argument einer besseren Verfügbarkeit für jeden, folgte 55,2 Prozent.
\item Eine Erleichterung der wissenschaftliche Kommunikation, sahen 49,2 Prozent als Argument für die Öffnung der wissenschaftlichen Kommunikation an.
\item Die Förderung des interdisziplinären Austausch von Wissenschaftlern und Wissenschaftlerinnen erachteten 45,1 Prozent als Argument an.
\item 44.0 Prozent oder 489 der Befragten sehen in der Überwindung sozialer, nationaler und globaler Wissenskluften ein Argument.
\item Die Chance einer umfassendere und transparentere Qualitätsmessung von Wissenschaft sahen 33,7 Prozent der befragten Wissenschaftler.
\item 250 oder 22.5 Prozent der Befragten sahen in der nachhaltigen und unabhängigen Archivierung der Informationen ein Argument für die Öffnung von Wissenschaft und Forschung.
\item 20,0 Prozent Ermöglicht indirekte Wirtschaftsförderung durch freien und offenen Wissenstransfer
\item Die Möglichkeit der Beilegung der vorherrschenden Zeitschriften- und Monographienkrise erachteten 16,3 Prozent.
\end{enumerate}

Bei dieser Frage gaben 8,3 Prozent der 1.112 Befragten gaben an, dass ihrer Meinung nach kein Argument für die Öffnung der wissenschaftlichen Kommunikation und aller Informationen aus dem Forschungs-/Arbeitsprozess spricht. Weitere 47 oder 4,2 Prozent machen weitere Angaben unter "Sonstiges".

---- Todo: weiter ausarbeiten v.a. Sonstige aufteilen und Grafik bauen ----

Bei den Argumenten gegen die Öffnung der wissenschaftlichen Kommunikation zeichnete sich ein uneindeutigeres Bild:
\begin{enumerate}
\item Als stärkstes Argument (43,4 Prozent) wurde von den Befragten die fehlenden Reputationskriterien für die Bewertung von offener Wissenschaft genannt.
\item Knapp dahinter (40,2 Prozent) wurde die "Gefahr der Fehlinterpretation und Falschinformation durch Wissenschaft" ausgewählt.
\item Einen erhöhter zeitlicher Mehraufwand für die Bereitstellung der wissenschaftlichen Publikationen und/oder Forschungsdaten sahen 379 oder 34.08 Prozent als Hürde an.
\item 30,0 Prozent sehen eine Erschwerung der eindeutigen Zuordnung von Texten, Arbeiten und Daten zu den jeweilgen als Argument gegen die Öffnung wissenschaftlicher Kommunikation.
\item  An fünfthäufigster Stelle wurde dem Argument zugestimmt, dass Qualität der wissenschaftlichen Arbeit leidet (27,3 Prozent).
\item Mit dem Argument, dass die Langzeitarchivierung und langfristige Auffindbarkeit nicht (dezentral) gewährleistet werden kann, stimmten 26,4 Prozent zu.
\item Das hohe Kosten und keine Refinanzierung ein Argument gegen Öffnung im wissenschaftlichen Kommunikationsprozess, unterstützten 274 der 1.112 Befragten (24.6 Prozent).
\item 9,0 Prozent der Befragten sah in der Öffnung der Kommunikation eine Bedrohung der Publikations- und Pressefreiheit.
\item Dass die offene Bereitstellung von Daten keinen nachhaltigen Mehrwert bietet, dem stimmen 8,7 Prozent zu und sehen darin ein Argument gegen eine Öffnung des Systems.
\item Dass Offenheit und Transparenz bei Forschungsförderung Freiheit von Wissenschaft und Forschung gefährden sehen 5,4 Prozent als Argument.
\end{enumerate}

154 Teilnehmer und Teilnehmerinnen nannten sonstige Argumente (13.9 Prozent). 7.7 Prozent sind der Meinung dass kein Argument gegen die Öffnung der wissenschaftlichen Kommunikation und aller Informationen aus dem Forschungs-/Arbeitsprozess spricht.

---- Todo: weiter ausarbeiten v.a. Sonstige aufteilen und Grafik bauen ----

\subsection{Wissenschaftliche Reputation und die Öffnung von wissenschaftlicher Kommunikation}

Die 1.112 Teilnehmer und Teilnehmerinnen der Befragen hatten die Möglichkeit unter mehreren Antwortoptionen auszuwählen, welche Faktoren für wissenschaftlichen Reputation in Ihrer Disziplin wichtig sind. Am häufigsten wurde dabei die "Anzahl der Beiträge" ausgewählt, 79,8 Prozent aller Befragten wählten diese Option. Die "Relevanz der publizierten Ergebnisse" wurde von 73.7 Prozent und "Vorträge auf wichtigen Konferenzen" von 68.2 Prozent der Wissenschaftler als wichtiger Faktor für wissenschaftlichen Reputation in der jeweiligen Disziplin angegeben.

Folgende weitere Kriterien wurden von den Befragten der Häufigkeit nach unter den Auswahlmöglichkeiten ausgewählt:
\begin{itemize}
\item die Bezugnahme und Zitation durch Kollegen wurde von 65,5 Prozent am vierthäufigsten genannt
\item 64,7 Prozent nannten den Indikator Drittmittelprojekte wichtig für die Reputation
\item 675 oder 60,7 Prozent der befragten Wissenschaftler und Wissenschaftlerinnen nannten Ranking- oder Impactfaktoren von Journalen als wichtigen Faktor für Reputation in ihrer Disziplin
\item das Renommee der Forschungseinrichtung war für weniger als die Hälfte der Befragten (48,1 Prozent) relevant
\item auch das Renommee von Herausgebern oder Mitautoren spielte nur für 36,06 Prozent einen wichtigen Faktor für die Reputation in der eigenen Disziplin
\item Netzwerke, Kontakte und ob man "dazu gehört" erachteten 35,16 Prozent als Reputationsfaktor
\item die Gutachtertätigkeit, Herausgeberschaft oder andere Funktionen sehen 27,7 Prozent als wichtig für die Reputation an
\item Ein Viertel (23,7 Prozent) der Befragen gaben an dass die Anzahl Monografien wichtig ist
\item Anwendungsrelevanz und Verwertbarkeit 13,3 Prozent
\item "materielle Ausstattung, Großgeräte etc." 12,9 Prozent
\item 143 der Befragten (12,7 Prozent) gaben an, dass öffentliche Aufmerksamkeit wichtig für Reputation ist
\item personelle Ausstattung zählte mit 12,5 Prozent eher selten zu den wichtigen Faktoren für Reputation in einer Disziplin
\item für 7,6 Prozent unter den Fragebogenteilnehmern stellen Patente einen Faktor dar
\item unter den vorgegebenen Antwortmöglichkeiten stellte die "politische Relevanz" mit 2,8 Prozent der Befragten den unwichtigsten Faktor für Reputation dar
\item 12 Teilnehmer oder Teilnehmerinnen (1,1 Prozent ) gaben in einem Freitextfeld sonstige Faktoren an
\end{itemize}

---- Todo: weiter ausarbeiten v.a. Grafik bauen ----

\subsection{Auffassungen zwischen den unterschiedlichen Fachdiziplinen}

Die Frage ob die Befragten ihre Veröffentlichungen in Zeitschriften oder Büchern für potentielle Leser gut zugänglich befinden, bejahten in den Naturwissenschaften 38,8 Prozent, in den Lebenswissenschaften 36,0 Prozent, in den Ingenieurwissenschaften 31,9 Prozent und in den Geistes- und Sozialwissenschaften 25,1 Prozent.

Laut der Studie besteht in der Fachgruppe der Ingenieurwissenschaften das größte Interesse an den Forschungsdaten anderer (75,2 Prozent), gefolgt von den Lebenswissenschaften mit 74,6 Prozent. Unter den 418 Sozial- und Geisteswissenschaftlern bejahten 69,9 Prozent und unter den Naturwissenschaftlern 69,0 Prozent ein Interesse an den Daten anderer.

Die Untersützung für die Forderung nach Open Access ist über die befragten Fachgruppen unterschiedlich stark verbreitet. In den Lebenswissenschaften gaben 88,3 Prozent an die Forderung nach kostenfreiem Zugang zu allen wissenschaftlichen Publikationen für Leser (Open Access) sehr gut oder gut zu finden. In den Naturwissenschaften beführworteten 82,0 Prozent den kostenfreiem Zugang zu allen wissenschaftlichen Publikationen für Leser, den Ingenieurwissenschaften 71,6 Prozent und in Geisteswissenschaften 68,2 Prozent.

Signifikante Unterschiede zwischen den verschiedenen Fachgruppen bestand auch bei der Frage nach der Wichtigkeit von Offenheit bei eigenen wissenschaftlichen Publikationsvorhaben. In den Geistes- und Sozialwissenschaften erachten nur 37,7 Prozent die "freie Verfügbarkeit des Volltexts im Internet" und 29,7 Prozent die "Veröffentlichung unter einer Open-Access Lizenz" als "sehr wichtig" oder "wichtig" an. Auch in den Ingenieurwissenschaften antworteten die Befragten auf die Frage nach der Wichtigkeit der freie und offene Verfügbarkeit ihrer Veröffentlichungen (57,8 Prozent) und die "Veröffentlichung unter einer Open-Access Lizenz" (47,6 Prozent) mehrheitlich mit "weniger wichtig" oder "nicht wichtig". In der Fachgruppe der Lebenswissenschaften gab die Mehrheit an, dass ihnen freie Verfügbarkeit der Texte im Internet wichtig, oder sehr wichtig ist (61,1 Prozent). Auch die Veröffentlichung unter einer Open-Access Lizenz war 53,6 Prozent mindestens wichtig. Die Mehrheit der Befragten, die angaben in den Naturwissenschaften tätig zu sein, war ebefalls, wenn auch knapp, mehrheitlich wichtig oder sehr wichtig (51,7 Prozent), dass ihr Beitrag als Volltext frei im Internet verfügbar ist. Die Veröffentlichung unter einer Open Access Lizenz war den Wissenschaftlern allerdings weniger oder nicht wichtig (63,9 Prozent).

Die Anzahl der Publikationen wird von den Befragten aller Fachgruppen als wichtiger Faktor für wissenschaftliche Reputation gesehen. Die Unterschiede bei den Antworten der Befragten bezüglich der Frage nach der Wichtigkeit dieses Faktors lagen zwischen den Fachgruppen liegen zwischen 82,8 Prozent in den Geistes- und Sozialwissenschaften, 80,8 Prozent in den Naturwissenschaften, 77,7 Prozent in den Lebenswissenschaften und 73,1 Prozent in den Ingenieurwissenschaften. Die Relevanz der publizierten Ergebnisse wird in den verschiedenen Fachgruppen als unterschiedlich wichtig betrachtet. In den Naturwissenschaften gaben 80,7 Prozent der Befragten an, dass die Relevanz der publizierten Ergebnisse ein wichtiger Faktor für Reputation in ihrer Disziplin ist. In den Lebenswissenschaften 77,7 Prozent, in den Geistes- und Sozialwissenschaften 69,6 Prozent und in den Ingenieurwissenschaften 68,1 Prozent. Die Relevanz von Rankings und des Impact-Factors von Journalen wird in den Fachgruppen unterschiedlich stark bewertet. In den Lebenswissenschaften stellt dieser Faktor für 84,8 Prozent der Befragten einen wichtigen Reputationsfaktor dar. 72,1 Prozent der teilnehmenden Wissenschaftler und Wissenschaftlerinnen der Naturwissenschaften, 56,0 Prozent der Ingenieurwissenschaftler und 43,8 Prozent der Sozial- und Geisteswissenschaftler erachten den Faktor Ranking beziehungsweise den gemessenen Impact der Journale als wichtig für die Reputation. Diese Zahlen decken sich auch mit dem Faktor "Anzahl der Monografien" und Wichtigkeit der der jeweiligen Publikationsform in den unterschiedlichen Fachdisziplin. Da die Publikationsform der Monografien nur in den Geistes- und Sozialwissenschaften eine signifikante Rolle spielt (76,6 Prozent) gaben auch nur in dieser Fachgruppe mehr als die Hälfte der Befragten an (50,2 Prozent), dass es sich bei der Anzahl der Monografien um einen wichtigen Faktor für die wissenschaftlichen Reputation sind in Ihrer Disziplin handelt. In den Ingenieurwissenschaften gaben nur 12,8 Prozent der Befragten an, dass es sich dabei um ein wichtigen Faktor handelt, in den Lebenswissenschaften 5,58 Prozent und in den Naturwissenschaften nur 4,4 Prozent. In allen Fachgruppen spielte der Faktor "Drittmittelprojekte" mehrheitlich eine wichtige Rolle. Am stärksten war der Faktor in den Lebenswissenschaften (74,1 Prozent), gefolgt von den Naturwissenschaften (67,1 Prozent) und den Ingenieurwissenschaften (61,7 Prozen). 58,9 Prozent der Geistes- und Sozialwissenschaftler gab an, dass der Faktor "Drittmittelprojekte" eine wichtige Rolle für Reputation in der jeweiligen Fachdisziplin spielt.

Nur die Befragten in den Lebenswissenschaften (52,5 Prozent) gaben mehrheitlich an Aufsätze, Texte oder Bücher publiziert zu haben, die vom Verlag selbst frei zugänglich gemacht wurden. Weitere 11,3 Prozent haben bisher keine Beiträge frei zugänglich veröffentlicht, planen das aber. In den Geistes- und Sozialwissenschaften gab 37,1 Prozent an Beiträge als Open Access veröffentlicht zu haben und weitere 12,1 Prozent eine Veröffentlichung als frei zugänglich zu planen. 36,0 Prozent der Naturwissenschaftler gaben an, mindestens einmal schon über einen Verlag frei verfügbar veröffentlicht zu haben, weitere 9,4 Prozent planen das. Knapp ein Drittel (34,5 Prozent) der befragten Ingenieurwissenschaftler gaben an frei zugänglich veröffentlicht zu haben und 10,1 Prozent planen es.

--- Todo Grafik bauen ----

Bis auf die Ingenieurwissenschaften (48,9 Prozent) hat laut den Befragten in allen Fachgruppen der Publikationsdruck in den vergangenen fünf Jahren zugenommen. In den Lebenswissenschaften 71,1 Prozent, 62,4 Prozent Naturwissenschaften und 61,5 Prozent in den Naturwissenschaften. Unter den Befragten der Lebenswissenschaften gab mehr ein Viertel (26,4 Prozent) an, das der Druck zu veröffentlichen "sehr stark" angestiegen ist.

\subsection{Auffassungen in den unterschiedlichen Altersgruppen der Befragten}

Anteilig an den Altersgruppen stellten die 56-60 Jährigen die größte Gruppe der Befürworter (83,3 Prozent) der Forderung nach kostenfreiem Zugang zu allen wissenschaftlichen Publikationen für Leser. Unter den 113 Befragten 36-40 Jährigen fanden 81,4 Prozent die Forderung "sehr gut" oder "gut". 81,1 Prozent der 26-30 Jährigen fanden die Forderung erwartungsgemäß mindestens gut.

--- Grafik bauen ---

\subsection{Auffassungen zwischen den unterschiedlichen Statusgruppen}

Eine genauere Betrachtung des Interesses an Forschungsdaten anderer in Kombination mit dem Item "beruflicher Status" zeigte, dass vor allem unter Doktoranden ein solches Interesse besteht. 79,2 Prozent der 118 befragten Doktoranden und 78,8 Prozent der 175 Doktorand_in mit einer Stelle als Wissenschaftlicher Mitarbeiter bejaten das Interesse. Drei viertel der wissenschaftlichen Mitarbeiter ohne Promotion (75,4 Prozent) und 72,4 Prozent der promoviertern wissenschaftlichen Mitarbeiter waren ebenfalls überwiegend an den Daten anderer interessiert. Unter den Juniorprofessoren gaben noch 61,5 Prozent der Befragten ein Interesse an den Daten an. Die etablierten Wissenschaftlern, wie Professoren (58,1 Prozent) und Privatdozenten (59,3 Prozent) waren weniger aber ebenfalls überwiegend an den Daten anderer interessiert.  Studenten, wissenschaftler aus der Privatwirtschaft, "Sonstige" und Bibliothekare waren mit 74,1 Prozent ebenfalls stark an Forschungsdaten interessiert.


\subsection{Veränderungen im Vergleich zur SOFI Studie}

Die Vergleich bezüglich der Samples in Bezug auf beruflicher Status, Alter und Geschlecht der befragten WissenschaftlerInnen ist ein Indikator für die Vergleichbarkeit der Samples. Bei beiden Erhebungen ist die Verteilung bezüglich des beruflichen Status im wissenschaftlichen System vergleichbar.
--- Todo: Grafik bauen ---
Auch der Vergleich der Altersgruppen beider Studien zeigt klare Überschneidungen bei den Befragten.  --- Todo: Grafik bauen ---

Wie in der Befragung vom SOFI 2008 wurden die Teilnehmer und Teilnehmerinnen gefragt, wie sie sich in ihrem Fachgebiet auf dem Laufenden halten und welchen Zugang sie zu den (Voll-) Texten haben \cite{hanekop_2008}. Auch die Antworten der 1.446 der Befragten zeigen "wie weitreichend sich bei der gezielten Suche nach Literatur digitale Suchmöglichkeiten durchgesetzt haben" \cite{hanekop_2008}. Fast 50 Prozent der Befragten in 2014 gaben an, die Google Suche häufig als Suchmöglichkeiten zu nutzen, um gezielt nach Literatur zu suchen. Bei der Befragung 2007 gaben 46 Prozent der Befragten an sehr häufig die Google Suche zu verwednen. Diese Entwicklung liegt im Trend, denn die IT-gestütze Suche lag in den 1980er Jahren bei einem Prozent, stieg bis 1993 auf neun Prozent an und betrug im Jahr 2003 bereits 24 Prozent \cite{hanekop_2008}.

--- Todo: Zahlen prüfen wegen Grundgesamtheit Grafik bauen ---

Ein ähnliches Bild zeigt sich bei dem Vergleich der Umfrageergebnisse bei der Frage wie sich die Teilnehmer in Ihrem Fachgebiet auf dem Laufenden halten. 2007 gaben 57 Prozent an, sich sehr häufig über Online-Zeitschriften auf dem aktuellen Stand der wissenschaftlichen Debatte zu halten. In der Befragung im Rahmen dieser Arbeit gaben 67,71 Prozent der 1.446 Befragten an sich in Online-Zeitrschriften zu informieren. Gefolgt wird diese Option durch die Teilnahme an Tagungen oder Kongressen (56,1 Prozent) und Gespräche mit Fachkollegen (55.1 Prozent). Social Media Platformen spielen mit bisher knapp 6 Prozent eher eine kleinere Rolle. Online-Datenbanken, Online-Archive, die 2007 noch zweithäufigste Option sich auf dem Laufenden zu halten, bleibt annähernd auf dem gleichen Niveau.

--- Todo: Zahlen prüfen wegen Grundgesamtheit 1.446 Grafik bauen ---

Im Jahr 2007 fanden rund 81 Prozent der Befragten die Forderung nach kostenfreiem Zugang zu allen wissenschaftlichen Publikationen für Leser gut bis sehr gut. In der Befragung 2014 fiel das Ergebnis mit einer Befragtenzahl von 1.112 mit 76,8 Prozent zwar niedriger aber dennoch weiterhin mehrheitlich positiv aus.

Ein weiteres Ergebnis der Studie des Soziologische Forschungsinstituts Göttingen im Jahr 2007, dass "gerade auch die etablierten und damit etwas älteren Wissenschaftler nutzen internetbasierte Plattformen intensiv". In der aktuellen Befragung gaben 86,4 Prozent der über 50-jährigen Befragten an, sich mit "Online-Ausgaben von Zeitschriften" "häufig auf dem Laufenden zu halten". In dieser Altersgruppe greifen jedoch auch noch 50 Prozent zu "Print-Ausgaben von Zeitschriften". In der Altersgruppe unter 50 Jahren nutzen es gerade mal 29,0 Prozent die Print-Ausgaben von Zeitschriften. Print-Bücher hingegen finden altersgruppenunabhängig bei rund 52,8 bis 53.7 Prozent Verwendung bei den Befragten. Bei digitalen Büchern sind es bei den über 50 Jährigen 19,8 Prozent und bei den unter 50 Jährigen mit 38,8 Prozent fast doppelt so häufig das Mittel der Wahl um sich in dem jeweiligen Fachgebiet auf dem Laufenden zu halten.

--- Todo: Grafik bauen ---

In der Studie 2007 gaben insgesamt 80 Prozent an sich mit Onlineausgaben auf dem Laufenden zu halten. Sieben Jahre später stieg die Nutzung nochmals um über 8 Prozent auf 88,4 Prozent an. Die Situationen in denen die Befragten nicht Online-Version eines Aufsatzes zugreifen können, weil es keine Lizenz gibt wurde ebenfalls seltener. Gaben 2007 noch 45 Prozent an, haufig bis sehr häufig nicht auf Aufsätze und Texte online zugreifen zu können, waren es in der aktuellen Befragung nur noch 32,4 Prozent. 66,8 Prozent der teilnehmenden Wissenschaftler gaben an nur gelegentlich bis nie Probleme mit dem Zugang zu Onlinetexten zu haben. In der Befragung 2007 waren es nur 52 Prozent.

--- Todo: Grafik bauen ---

Bei der Fragen, wie häufig es kommt für die Befragten vorkommt, dass Sie auf die Online-Version eines Aufsatzes nicht zugreifen können, ist eine leichte Verschiebung zu gunsten der Verfügbarkeit für die Wissenschaftler und Wissenschaftlerinnen festzustellen. Während 2007/2008 9 Prozent der Teilnehmer "sehr häufig" oder 36 Prozent "häufig" auf die Online-Version eines Aufsatzes nicht zugreifen konnten, gaben 2014/2015 nur 6,3 Prozent an "sehr häufig" beziehungsweise 26,1 Prozent "häufig" auf Onlineinhalte nicht zugreifen konnten. "Gelegentlich" konnten 2007/2008 38 Prozent und 2014/2015 fast die Hälfte (49.5 Prozent) nicht auf die Webversion von Inhalten zugreifen, weil es keine Lizenz dafür gab. Selten oder nie Probleme mit dem Zugriff hatten 2014/2015 17,4 Prozent. In 2007/2008 waren es 14 Prozent.

Ob Aufsätze oder Bücher publiziert wurden, die vom Verlag selbst frei zugänglich gemacht wurden antworteten 2007/2008 34 Prozent mehr als einen Beitrag frei zugänglich verfügbar gemacht zu haben. Im Vergleich dazu gaben 2014/2015 25,7 Prozent der antwortenden Personen an mehrere Beiträge frei zugänglich veröffentlicht zu haben. 14,0 Prozent hatten laut der Befragung 2014/2015 einen Beitrag veröffentlicht (23,1 Prozent 2007/2008). Wie in 2007/2008 (11 Prozent) gaben auch 2014/2015 mit 11,2 Prozent fast ähnlich viele Personen an eine frei zugängliche Publikation zu planen. In der aktuellen Erhebung gab fast die Hälfte (49,2 Prozent) an bisher keine offenen Publikationen veröffentlicht haben und das auch nicht zu planen. 2007/2008 waren es nur 32 Prozent, die keine frei zugängliche Publikation veröffentlicht oder geplant haben.

--- Todo: Grafik mit Vergleich bauen ---

In Bezug auf die Wichtigkeit der Faktoren für wissenschaftlichen Reputation in den verschiedenen Disziplinen war in der Befragung 2007/2008 für 92 Prozent der Befragten die "Relevanz der Ergebnisse" wichtig oder sehr wichtig. 2014/2015 gaben 73,7 Prozent der Befragten an, dass "Relevanz der Ergebnisse" ein wichtiger Faktor für Reputation ist. Der am häufigsten augewählte Faktor für Reputation war in der aktuellen Erhebung "die Anzahl der Aufsätze / Beiträge" (79,8 Prozent). In 2007/2008 waren für 82 Prozent der Befragen die Bezugnahme beziehungsweise die Zitation durch Kollegen wichtig oder sehr wichtig, in der aktuellen Befragung identifizierten diesen Faktor 65,5 Prozent.

Die Frage, ob der Publikationsdruck in Ihrem Fachgebiet in den vergangenen fünf Jahren zugenommen hat, beantworteten in der SOFI-Studie ähnlich wie in der aktuellen Erhebung. 18,4 Prozent antworteten aktuell und 21,9 Prozent in 2007 mit ja, sehr stark. Unverändert ist der Publikationsdruck aktuell bei 21,9 Prozent versus 19,8 Prozent 2007. Die größte Gruppe antwortete mit "ja"; 2007 43,6 Prozent und 2014/2015 42,9 Prozent. In der ektuellen Befragung waren sich 16.5 Prozent unsicher, in der Befragung vor 7 Jahren waren es 14,1 Prozent.

Finden Sie, dass Ihre Veröffentlichungen in Zeitschriften oder Büchern für potentielle Leser gut zugänglich sind, beantworteten 7,9 Prozent weniger als 2007, dass ihre Veröffentlichungen gut zugänglich sind. "Teils/teils" antworteten in beiden Erhebungen ähnlich viele: 46,5 Prozent in 2007 versus 46,9 Prozent in 2014. Nicht gut zugänglich befanden 2014 9,2 Prozent, 2007 waren noch 6,0 Prozent.

--- Todo: Grafik mit Vergleich bauen ---

Die Frage nach dem Aufwand, für die freie Veröffentlichung von Publikationen im Internet beantworteten in der aktuellen Befragung 30,8 Prozent und 2007 32,5 Prozent. Mittelgroßen Aufwand vermuteten 2007 22,9 Prozent und 2014 24,7 Prozent. "Teils/teils" gaben aktuell 22,6 Prozent und 2007 19,6 Prozent an. Großer Aufwand für die freie Veröffentlichung schätzen 2014 5,1 Prozent und in der SOFI-Studie 4,4 Prozent. Keine Antwort wussten in beiden Erhebungen rund 18,5 Prozent.

--- Todo: Grafik mit Vergleich bauen ---
