\chapter{Befragung: Öffnung von Wissenschaft aus der Perspektive von Wissenschaftlern}

Ein weiteres Ziel der Arbeit ist es, die erarbeiteten theoretischen Grundlagen, sowie die Ausprägungen von Open Access und Open Science vor dem Hintergrund von wissenschaftlicher Reputation und über die Grenzen einzelner Fachdisziplinen hinaus im Rahmen einer Umfrage zu überprüfen. Besondere Berücksichtigung findet dabei auch die Identifikation weiterer Treiber und Bremser für die Öffnung von wissenschaftlichen Informationen und Prozessen. Dafür werden die aus der theoretischen Betrachtung analysierten Konzepte Open Access und Open Science einer Befragung von Wissenschaftlern und Wissenschaftlerinnen zugeordnet. Abschließend werden die Ergebnisse der Befragung mit der im Jahr 2007 durchgeführten Erhebung "Wissenschaftliche Publikationen im Internet: Wissenschaftler als Leser und Autoren" durch das Soziologische Forschungsinstituts Göttingen (SOFI) verglichen.

Um die genannten Aspekte mit einer möglichst großen Stichprobe zu erforschen, wurde die Online-Befragung als Methode gewählt. Die Befragung richtete sich dabei ausschließlich an deutschsprachige Wissenschaftler und Wissenschaftlerinnen in  unterschiedlichen Karrierestufen und Fachdisziplinen sowie an Personen im wissenschaftlichen Umfeld, die mit den Eigenheiten des wissenschaftlichen Kommunikationssystem vertraut sind.

Der Fragebogen wird für die Erfassung konkreter Verhaltensweisen und allgemeine Zustände und Sachverhalte \cite{raab_2012_fragebogen} konstruiert und die zentralen Forschungsfragen dieser Arbeit stellen auch die Grundlage für die Entwicklung des Fragepools dar. Die Formulierung der Fragen basierten, sofern nicht aus der Studie des SOFI unverändert übernommen, auf den in den vorhergehenden Kapiteln erarbeiteten Handlungsmustern, Definitionen, Intentionen, Meinungen und Einstellungen zu folgenden Fragestellungen:
\begin{itemize}
\item Wie verändert die Digitalisierung, wie wir auf wissenschaftliche Daten und Informationen zugreifen?
\item In welchem Umfang besteht Wissen über die Öffnung von Wissenschaft unter den Wissenschaftlern und Wissenschaftlerinnen?
\item Welches Verständnis von Open Access besteht unter den Befragten?
\item Wie stark ist das Interesse an Forschungsdaten?
\item Welche Faktoren und Argumente begünstigen die Öffnung von Wissenschaft in der jeweiligen wissenschaftlichen Disziplin, welche Argumente sprechen dagegen?
\item Wie wird der geschätzte Aufwand für die Öffnung von Wissenschaft in einer wissenschaftlichen Disziplin eingeschätzt?
\item Welche unterschiedlichen Auffassungen bestehen zwischen den unterschiedlichen Fachdiziplinen, Alters- und Statusgruppen?
\item In welchem Umfang wird bereits heute im wissenschaftlichem Umfeld offen kommuniziert?
\item Welche Veränderungen beim Zugang zur Literatur wie auch bei den Veröffentlichungsstrategie sind im Vergleich zur der 2007 und 2008 durchgeführten Befragung des SOFI Göttingen zu erkennen?
\end{itemize}

\section{Erhebungsmethode und Messinstrumente}

Die Auswahl der Erhebungsmethode basierte auf folgenden Überlegungen: Persönliche Interviews erschienen wenig geeignet, da der damit verbundene personelle, zeitliche und finanzielle Aufwand als zu hoch eingestuft wurde. Gegen eine postalische Befragung sprachen die hohen Kosten (unter anderem Porto) sowie die häufig geringen Rücklaufquoten \cite{suchen}.

Auschlaggebend für die Auswahl dieser Befragungsform war auch, dass das Forschungsinstruments "Fragebogen" zu den am häufigsten eingesetzten Methoden in der Sozialforschung gehört \cite{raab_2012_fragebogen} und durch die zunehmenden Verbreitung und Nutzung des Internets, die elektronische Online-Befragung längst Eingang in die empirische Sozialforschung gefunden hat \cite{Pannewitz_2002}. Darüber hinaus begründete sich die Auswahl in dem ökonomischen Aspekt, das die Online-Befragung besonders geeignet ist, um "große Stichproben in kurzer Zeit zu erheben" \cite{eichhorn_2004_online} und diese Erhebungsmethode eine Beantwortung der Fragen durch die Teilnehmer und Teilnehmerinnen zu jeder Zeit ermöglicht. Auch dass die Vergleichsstudie durch das SOFI Göttingen ebenfalls auf das Internet als primäre Quelle für die Identifikation von Teilnehmern und Teilnehmerinnen und E-Mail als Kontaktaufnahmekanal zurückgegriffen hat, spielte eine Rolle bei der Wahl der Erhebungmethode. Wessentlich war auch, dass diese Form der Befragung, die leichte Verbreitung am Institut und unter Kolleginnen und Kollegen ermöglichte.

Weitere Vorteile bei der Methode der Online-Datenerhebung ist die unabhängige und einfache Teilnahme der Befragten. Die Unabhängigkeit wurde vor allem dadurch gewährleistet, das die Befragungsituation für alle Teilnehmer und Teilnehmerinnen gleich war. Es wurde auch davon ausgegangen, dass die notwendigen technischen Voraussetzungen zur Teilnahme an einer Internetbefragung (Internetzugang und internetfähiges Endgerät) bei allen Wissenschaftlern an deutschsprachigen Wissenschaftseinrichtungen gegeben sind. Es kam nur zu drei expliziten Verweigerungen der Teilnahme an der Befragung: In einem Fall gab es Institutsbeschluss nicht mehr an Befragungen teilzunehmen. In einem weiteren Fall wurde die Methode der sozialwissenschaftlichen Befragung grundsätzlich abgelehnt und in dem dritten Fall mit dem Verweis auf zu hohen Aufwand für das Ausfüllen von Fragebogen beantwortet.

Die Anonymität der Befragten wurde jederzeit gewahrt und keine eindeutigen persönlichen Daten erhoben, die einen Nutzer oder eine Nutzerin direkt identifizierbar gemacht hätten. Auf Grund der geplanten Veröffentlichung der Ergebnisse unmittelbar nach Abschluss der Befragung, wurde von Beginn an darauf geachtet, dass zu keinem Zeitpunkt Rückschlüsse auf individuelle Teilnehmer oder Teilnehmreinnen an der Befragung möglich sind.

\subsection{Untersuchungsobjekte}

Die Teilnehmer des Fragebogens waren primär deutschsprachige Wissenschaftler und Wissenschaftlerinnen aus sämtlichen Fachdisziplinen oder Mitarbeiter des wissenschaftlichen Betriebs aus dem deutschsprachigen Raum. Sie wurden im Zeitraum vom 18.8.2014 bis 18.01.2015 online befragt. Bibliothekare und Bibliothekarinnen (0,95 Prozent der Befragten) und Studierende (3,68 Prozent Befragten) wurden zwar nicht direkt angesprochen, konnten aber dennoch an der Umfrage teilnehmen. Mit dem Start der Befragung wurden insgesamt 4.002 Wissenschaftlerinnen und Wissenschaftler per E-Mail im Zeitraum vom 18.08.2014 bis 18.01.2015 angeschrieben.

Die Auswahl der jeweiligen Fachdisziplinen beruht auf der aktuellen Auflistung der Fachsystematik der Deutschen Forschungsgemeinschaft (DFG) \cite{suchen_Webseite_DFG}. Da die Erhebung fächerübergreifend angelegt war, um die Unterschiede zwischen den Disziplinen zu evaluieren, wurden Vertreter und Vertreterinnen aus allen dort gelisteten Fachdisziplinen für die Teilnahme angefragt. Per Zufall wurden dazu von den Institutswebseiten im deutschsprachigen Raum pro Fach 150 Wissenschaftler und Wissenschaftlerinnen per E-Mail angeschrieben und um Teilnahme an der Befragung gebeten. 1.768 der Angefragten haben an der Umfrage teilgenommen und den Fragebogen gestartet, 1.467 Teilnehmer und Teilnehmerinnen haben mindestens eine Frage beantwortet und somit teilweise an der Befragung teilgenommen. 301 Personen haben vor Beantwortung der ersten Fragegruppe abgebrochen. Die Rücklaufquote liegt somit bei 44,2 Prozent brutto beziehungsweise bei 36,7 Prozent netto. 1.112 von den 1.768 Teilnehmer und Teilnehmerinnen (62,9 Prozent), die die Befragung gestartet haben, haben den Online Fragebogen vollständig beendet. Die übrigen 656 Personen (37,1 Prozent) haben den Online-Fragebogen vor der Beantwortung aller Fragen abgebrochen.

Die hohe Resonanz ist vermutlich auf die persönliche Ansprache sowie die konkrete Zuordnung zur Fachdisziplin im Anschreiben zurückzuführen. Dabei handelt es sich zwar um ein aufwendiges Vorgehen, das aber sicherlich zu dieser guten Quote beigetragen hat. Die Fragebögen, deren Beantwortung vor Beendigung aller Fragen abgebrochen wurden, bleiben in der weiteren Betrachtung unberücksichtigt.

Um die Repräsentativität der Studie sicherzustellen wurden die Rückläufer der Befragung auf vorhandene Informationen zur fachliche Zuordnung, den beruflichen Status und das Alter ausgewertet. Verschiedenen Verzerrungen sind nur zu vermuten, da die kontaktierten Menschen ausschließlich online angeschrieben wurden. Da die Umfrage jedoch ohne Zugangsbeschränkung öffentlich online ausgefüllt werden konnte, war es jedem Interessenten möglich teilzunehmen.

\subsection{Untersuchungsmaterial} --- TODO: Titel überarbeiten ----

Für die Durchführung der Online-Befragung wurde die Open Source Software LimeSurvey Version 2.05+ verwendet, die auf einem Webserver des Centre for Digital Cultures installiert wurde. Die Software ist weit verbreitet und ermöglicht umfassende Einstellungs- und Anpassungsmöglichkeiten. So konnte zum Beispiel ein Teil der Fragen in Abhängigkeit von den Antworten auf vorherige Fragen kontextsensitiv definiert werden. Die Software ermöglichte es, die beantworteten Fragebögen aus der Verwaltungsoberfläche einzeln oder zusammengefasst einzusehen und für die Auswertung zu exportieren. Neben den üblichen Möglichkeiten zur Durchführung von Befragungen an internetfähigen Endgeräten wurde die Darstellung der Befragung darüber hinaus so angepasst, dass die Darstellung und die Beantwortung des Fragebogens auch auf internetfähigen Mobiltelefonen möglich war. Bei dem Design des Fragebogens und der Anpassung der Darstellung der Software wurde darüber hinaus explizit darauf geachtet, dass alle Texte einfach und angenehm lesbar waren, damit die Beantwortung der Fragen einfach und strukturiert ablaufen konnte.

Die Ergebnisse wurden in der Datenbank des Servers des Centres for Digital Cultures zwischengespeichert und am xx.xx.2015 gelöscht. Nach Abschluss der Befragung wurden die Datensätze anonymisiert. Dazu wurden sämtliche persönliche Daten, wie zum Beispiel in Freitextfeldern genannte E-Mailadressen entfernt und die freiwilligen personenbezogenen Angaben von dem Rest der Daten getrennt und neu sortiert. Folgende Felder wurden getrennt, neu angeordnet und unabhängig von den anderen Erhebungen veröffentlicht: Geschlecht, Alter, weitere Aspekte zum Thema, Anmerkungen und Kritik, Funktion im Rahmen eines Open Access Engagements, Antwort ID und Zeitpunkt der Beantwortung. Die anonymisierte Datensätze wurden nach Abschluss der Befragung im Januar 2015 auf dem datorium-Datenrepositorium des GESIS - Leibniz-Institut für Sozialwissenschaften veröffentlicht. Die Forschungsdaten durchliefen vor der Veröffentlichung ein durch GESIS durchgeführtes Review. Eine weitere Veröffentlichung der Daten erfolgte auf dem Datenrepositorium Zenodo.

\subsection{Aufbau des Fragebogens}

Für die Befragung durch das SOFI im Jahr 2007 wurden 6.500 Wissenschaftler und Wissenschaftlerinnen angefragt, von denen 1.803 geantwortet haben. Der 2007 verwendete Fragebogen bestand aus 51 Fragen. Im ersten Teil des Fragebogens wurden den Befragten Fragen zu Fachgebiet und Tätigkeitsbereich aus der Pespektive des Leserin und Leser wissenschaftlicher Publikationen gestellt. Im zweiten Teil wurden die Teilnehmer und Teilnehmerinnen aus der Perspektive der Autorin beziehungsweise des Autors wissenschaftlicher Beiträge befragt. Abschließend wurden noch eineige personenbezogene Angaben erhoben \cite{Hanekop_Wittke_2007_Fragebogen}.

Zu Beginn der Fragebogenkonstruktion wurden der Fragebogen und das Datenmaterial der Vorbefragung einer Itemanalyse zum Ausschluss unpassender Fragen (Items) unterzogen und Fragen in Zusammenhang mit den Fragestellungen dieser Arbeit hinzugefügt. Dafür wurden die veröffentlichten Antworten der Befragung durch das SOFI analysiert. Fragen, die stark ungleich verteilt waren, wurden, wenn sie nicht inhaltlich interessant erschienen, ausgeschlossen oder mit anderen Fragen zusammengelegt. Dadurch wurden auf der Basis der Analyse der Fragen der Fragepool auf 32 Fragen reduziert beziehungsweise modifiziert. Acht der insgesamt 40 Fragen standen im Zusammenhang mit vorhergehenden Fragen und wurden deshalb nicht allen Teilnehmern und Teilnehmerinnen gestellt. Die Reihenfolge der Fragen und der Fragengruppen wurde so gewählt, dass sie strukturiert abgebildet werden konnten, der Reihenfolge-Effekt minimiert wurde und die Beantwortung bis zum Ende interessant bleibt. Beim Aufbau des Fragebogens wurden die Aufzählung der Richtlinien zur Formulierung der Items nach Bortz und Döring \cite{raab_2012_fragebogen} berücksichtigt.

Die Qualität und Brauchbarkeit des Fragebogens wurde in einem Pretest (Probedurchlauf) mit wissenschafltichen Mitarbeitern und Mitarbeiterinnen aus dem Arbeitsumfeld des Autors überprüft. Die Einleitung für den Fragebogen, die Instruktionen und die Anrede wurden ebenfalls im Pretest evaluiert und optimiert, da sie sehr viel "zur Motivation der Bearbeitung beteitragen kann" \cite{raab_2012_fragebogen}. Dazu wurde der Fragebogen an 15 Wissenschaftler im Testmodus übermittelt und unter der Instruktion des "lauten Denkens" um Bearbeitung des Fragebogens gebeten \cite{raab_2012_fragebogen}. Nach dem Pretest wurden die Befragung um weitere Fragen angepasst, die sich auch auf die Veröffentlichung von wissenschaftlichen Informationen und Daten beziehen.

Im finalen Fragebogen kommen die Antwortformate offene Fragen, geschlossene Fragen und Mischformen mit offenen und vorgegebenene Kategorien, sowie freie (offene) Antwortformate zum Einsatz. Es wurde versucht weitestgehend auf Ratingskalen zu verzichten. Insgesamt wurden in dem Fragebogen drei fünfstufige Ratingskalen mit verbaler Skalenbezeichnung eingesetzt. Die Charakterisierungen der Abstufungen wurde zur Vergleichbarkeit aus der SOFI-Befragung von 2007 übernommen.

Die Gliederung des Fragebogens war ebenfalls an die Befragung aus den Jahren 2007 angelehnt und lediglich durch die Besonderheiten in Bezug auf die Veröffentlichung und Nutzung von Forschungsdaten ergänzt. Insgesamt wurden die 40 Fragen in 5 Fragegruppen und eine abschließende Fragegruppe für persönliche Angaben sowie Anmerkungen und Kritik unterteilt. In der ersten Fragegruppe wurde auf die Rahmenbedingungen der Teilnehmenden sowie deren wissenschaftliche Tätigkeit eingegangen. In der zweiten Fragegruppe wurden Aspekte aus der wissenschaftlichen Leserperspektive evaluiert. Die dritte Fragegruppe beschäftigte sich mit dem Zugang zu wissenschaftliche Informationen, gefolgt von der vierten, die aus Fragen zum Zugang zu wissenschaftlichen Informationen und zum Zugriff auf wissenschaftliche Daten bestand. In der fünften Fragegruppe wurden Fragen aus der Perspektive des Autors oder der Autorin von wissenschaftlichen Inhalten gestellt. Abschließend folgte die Erhebung weiterer freiwilliger personenbezogener Daten als Grundlage für die Möglichkeit der späteren Segmentierung der Teilnehmer und Teilnehmerinnen. Die Befragten wurden vor Beginn der Befragung auf die Gliederung des Fragebogens und die Reihenfolge der Fragegruppen, sowie die Bedingungen des Fragebogens, wie die anonyme Behandlung der Daten, hingewiesen.

Den Erfolg der Erfolg dieser umfassenden Vorbereitung und der Anpassung des Fragen verdeutlicht, dass 75 Prozent der Befragten, die mindestens eine Frage beantwortet haben, auch den gesamten Fragebogen vollständig beantwortet haben.

\subsection{Untersuchungsdurchführung}

Nach der Auswertung und Einarbeitung der Anmerkungen der Pretester wurde der Fragebogen "Wissenschaftliche Kommunikation im Rahmen der Digitalisierung" wurde am 18. August 2014 unter der Internetadresse http://umfrage.offene-doktorarbeit.de veröffentlicht. Nach der Veröffentlichung wurden nach Zufallsprinzip jeweils circa 150 Wissenschaftler und Wissenschaftlerinnen jeder Fachdisziplin der DFG-Fachkollegien identifiziert. Die Namen und E-Mail-Adressen zu den Personen waren über die Internetseiten der Hochschulen und wissenschaftlichen Organisationen öffentlich zugänglich. Die Kontaktaufnahme zu den ausgewählten Personen erfolgte über eine personalisierte E-Mail mit einem Hinweistext, Instruktionen und einem direkten Link auf die Internetadresse des Fragebogens als klickbarer Link in der E-Mail. Vereinzelt wurden auch Sekretariatsadressen von Forschungsinstitutionen verwendet und um Weiterleitung der Einladung zur Befragung an die Wissenschaftler und Wissenschaftlerinnen innerhalb der Organisation gebeten. Alle identifizierten Kontakte wurden lediglich einmal kontaktiert. Zusätzlich wurde der Umfrage-Link mit einer kurzen Information zur Umfrage auf http://offene-doktorarbeit.de veröffentlicht, sowie über die privaten Social-Media Kanäle des Autors und an persönliche Kontakte des Autors verschickt. Des Weiteren wurde eine generalisierte Version der Einladung zur Umfrage über wissenschaftliche Mailinglisten, sowie den Newsletter des Centre for Digital Cultures verbreitet. Um eine Möglichst große Streuung der Umfrage zu erzielen, hatten die Teilnehmer und Teilnehmerinnen nach Abschluss des Fragebogens zusätzlich die Möglichkeit den Link zu der Befragung über die sozialen Kanäle und per E-Mail selbst weiterzuverbreiten.

\section{Kritische Betrachtung und Beurteilungsfehler}

Immer wieder kommt es bei dem Prozeß der Erstellung von Fragebögen oder bei der Beurteilung der erhobenen Daten zu Störungen, der sogenannten Beurteilungsfehlern. Deshalb soll die Güte der Befragung durch die Gütekriterien Objektivität, Reliabilität und Validität beurteilt werden.

\subsection{Objektivität}

Die Unabhängigkeit beschreibt das Ausmaß, in dem das Ergebnis der Untersuchung frei und unabhängig von Einflüßen ausserhalb der befragten Person ist \cite{rost_2004_lehrbuch}. Die Objektivität der durchgeführten Befragung ist gegeben, da durch die elektronische Onlinebefragung eine zeitliche und räumliche Unabhängigkeit gewährleistet wurde. Die Befragung wurde für alle Teilnehmer und Teilnehmerinnen nach identischer Anrede, Einladung und Instruktion und ohne Untersuchungsleiter durchgeführt und war somit nicht von besonderen Situationsvariablen abhängig.

\subsection{Reliabilität}

Die Reliabilität gibt den Grad der Genauigkeit an, mit der durch die empirische Datenerhebung ein Merkmal erfasst wird \cite{rost_2004_lehrbuch}, unabhängig davon was er erfasst. Schelten definiert einen Tests als reliabel, "wenn er das, was er mißt, genau misst" \cite{schelten_1997_testbeurteilung}.  Sie spiegelt die Replizierbarkeit von Messergebnissen und Zuverlässigkeit einer Datenerhebung wieder. Von einer hohen Reliabilität der durchgeführten Befragung kann ausgegangen werden, da bei den übernommenen Fragen aus der Messung des SOFI im Jahr 2007 zu selben oder ähnliche Ergebnissen erzielt werden konnten und die Reliablitität der Online-Befragung mit der schriftlichen Befragung als vergleichbar eingestuft werden kann \cite{suchen_DOI_10.1007/978-3-663-10948-8_10}. Weitere Reliabilitätstests konnten vernachlässigt werden, weil die Befragung größtenteils aus statistischen Abfragen und Bewertungsfragen bestand.

\subsection{Validität}
In der Literatur werden zwei Typen von Validität unterschieden \cite{rost_2004_lehrbuch}. Diese sind die interne und die externe Validität. Von einer hohen internen Validität wird ausgegangen, wenn die erzielten Ergebnisse klar und eindeutig interpretierbar sind \cite{raab_2012_fragebogen}. Von einer hohen externe Validität wird ausgegangen, wenn die Ergebnisse des Experiments auf die Realität übertragbar sind \cite{bortz1995forschungsmethoden}.

Im Rahmen der durchgeführten Befragung zeigt die Validität, ob das Messinstrument Fragebogen wirklich das misst was dazu beiträgt, die Fragestellungen der Arbeit zu beantworten. Die Validität wurde durch die Übernahme der Grundstruktur und von Items der Studie "Wissenschaftliche Publikationen im Internet: Wissenschaftler als Leser und Autoren" des SOFI in Göttigen gewährleistet. Wie bei der Reliablitität wird auch die Validität einer schriftlichen Befragung mit der einer Online-Befragung als vergleichbar eingestuft. Die Validität der neu erstellten, angepassten  und zusammengelegten Items wurde durch die Auswertung des Pretests sowie durch die Einbeziehung der Inhaltsanalyse in die Erstellung der Fragen sichergestellt.

\section{Ergebnisse der Befragung}

Im Zeitraum vom 18. August 2014 bis zum 18. Januar 2015 haben 1.768 Personen die Befragung zur wissenschaftlichen Kommunikation im Rahmen des Promotionsvorhabens gestartet. 1.467 Teilnehmer haben die Umfrage teilweise und 1.112 komplett abgeschlossen. Die erhobenen Daten der 1.112 Teilnehmer des Online-Fragebogens werden mit Hilfe der computerunterstützten Datenaufbereitung statistisch ausgewertet und hier dargestellt. Die Darstellung der Ergebnisse orientiert sich dabei an den definierten Fragestellungen dieser Arbeit und werden wie folgt verteilt.

Erst wird die Einordnung der Befragten und eine Darstellung der soziodemographischen Daten vorgenommen. Im darauffolgenden Abschnitt werden die Erhebungsergebnisse zu den Veränderungen in der wissenschaftlicher Kommunikation durch die Digitalisierung geschildert, gefolgt von der Darstellung der Ergebnissen über das Verständnis von Offenheit und Interesse an Offenheit bei der wissenschaftlichen Kommunikation unter den Befragten, sowie der Erläuterung der tatsächlich praktizierten Offenheit im wissenschaftlichen Alltag auf Grundlage der Ergebnisse. Eine weiteres Forschungsziel war die Herausarbeitung und Erforschung der Treiber und Bremser für die Öffnung von Wissenschaft und Forschung, die im nächsten Abschnitt anhand der Umfrageergebnisse dargestellt werden. Im folgenden Abschnitt werden die Fragen ausgewertet, bei denen die Öffnung von wissenschaftlicher Kommunikation in den Kontext von wissenschaftlicher Reputation und der jeweiligen Fachdiziplin gestellt wurden. Im Weiteren werden die Ergebnisse der Befragung in Bezug auf die Auffassungen zu verschiedenen Fragen in den unterschiedlichen Alters- und Statusgruppen dargestellt. Abschließend werden die Veränderungen zur SOFI-Studie elaboriert.

\subsection{Soziodemographischen Daten}

In die Auswertung der Befragung gingen die Angaben der 1.112 Teilnehmer des Online-Fragebogens ein, die die Umfrage komplett abgeschlossen haben. Die Auswertung ergab folgende soziodemographische Daten der Befragten:

\begin{itemize}
\item \textbf{Geschlecht:} 444 der Befragten waren weiblich (39,9 Prozent), 606 und 54,5 Prozent männlich. 62 Personen oder 5,6 Prozent machten keine Angabe zu ihrem Geschlecht.
\item \textbf{Alter:} Die prozentuale Verteilung des Alters gestaltete sich wie folgt: 4,4 Prozent (46) waren zum Zeitpunkt der Befragung jünger als 31 Jahre, die größte Altersgruppe mit 31,2 Prozent stellten die 31 bis 40 Jährigen dar. 17,4 Prozent der Befragten waren zwischen 41 und 50 Jahre alt, während 14,6 Prozent angab älter als 50 Jahre zu sein. 0,7 Prozent machten bei der Frage nach ihrem Alter keine Angaben.
\item \textbf{Berufsstatus:} Unter den Befragten gaben 24,8 Prozent an, Privatdozenten, Juniorprofessoren oder Professoren zu sein. 55,9 Prozent der Teilnehmer waren wissenschaftliche Mitarbeiter, 20,0 Prozent wissenschaftliche Mitarbeiter mit Pomotionsvorhaben, 22,8 Prozent bereits fertig promovierte wissenschaftliche Mitarbeiter und 13,1 Prozent Mitarbeiter ohne Promotionsvorhaben oder abgeschlossener Promotion. 10 Teilnehmer (0,9 Prozent) gaben an Wissenschaftler in der Privatwirtschaft zu sein. 35 Befragte (3,2 Prozent) wurden unter "Sonstiges" subsummiert.
\item \textbf{Tätigkeitsdauer in der Wissenschaft:} Nur 6,2 Prozent der Befragten gaben an "weniger als 1 Jahr" in der Wissenschaft tätig zu sein. 20,4 Prozent war seit mehr als einem aber weniger als drei Jahre in der Wissenschaft tätig. 24,0 Prozent gaben an zwischen drei und sechs Jahren wissenschaftlich tätig zu sein. 15,4 Prozent der Teilnehmer und Teilnehmerinnen war mehr als sechs aber weniger als zehn Jahre in der Wissenschaft. Die größte Gruppe gab an, "mehr als 10 Jahre" wissenschaftlich tätig zu sein (31.8 Prozent). 1,5 Prozent gaben an, "nicht in der Wissenschaft tätig" zu und 0,7 Prozent enthielten sich einer Angabe.
\item \textbf{Forschungseinrichtung:} Die große Mehrzahl der Teilnehmer (78,06 Prozent) gaben an, an einer deutsche Universität/Hochschule beschäftigt zu sein. Mit 5,31 Prozent waren 59 Befragten an einem Institut der Leibniz-Gemeinschaft und 4,77 Prozent an an einer "Sonstigen" Einrichtung tätig zu sein. Nur 1,4 Prozent der Befragten wirkten an einem Max-Planck-Institut und 0,4 Prozent an einem Institut der Fraunhofer Gesellschaft. An einer Universität/Hochschule im deutschsprachigen Ausland waren 4,1 Prozent und im nicht-deutschsprachigen Ausland 1,4 Prozent tätig. 1,3 Prozent arbeiteten an einer deutschen Fachhochschule. 11 Befragte (1 Prozent) gaben an einem „An“-Institut (eigenständige Forschungseinrichtung, angegliedert an einer deutschen Hochschule) beschäftigt zu sein.
\end{itemize}

Die größte Gruppe (37,6 Prozent) bildeten die Befragten der Fachgruppe der Geistes- und Sozialwissenschaften. 29,0 Prozent gaben an, in den Naturwissenschaften verortet zu sein. Aus den Lebenswissenschaften kamen 17,7 Prozent der Befragten. Die kleinste Gruppe unter den Teilnehmern stellten mit 12,7 Prozent Wissenschaftler aus der Fachgruppe der Ingenieurwissenschaften dar. 34 der Befragten (3,1 Prozent) konnten nicht eindeutig einer der vier Fachgruppen zugeordnet werden.

--- Todo: Verteilung auf Fächer und Grafik bauen ----

Die überwiegende Mehrheit der Befragten (97,4 Prozent) gab an, in der Forschung tätig zu sein. Mehr als die Hälfte aller gab an "überwiegend" (53,4 Prozent) forschend zu arbeiten. Demgegenüber gaben 1,7 Prozent beziehungsweise 19 Personen an "gar nicht" zu forschen und 2,6 Prozent machten keine Angabe. Insgesamt gaben 83,6 Prozent der Teilnehmer und Teilnehmerinnen in der Umfrage an, lehrend tätig zu sein, davon 9,6 Prozent sogar "überwiegend". Demgegenüber waren laut der Auswertung nur 51 Personen (4,5 Prozent) in der klinischen Versorgung tätig. Administrativen tätigkeiten gingen 4,5 Prozent "überwiegend", 13,0 Prozent "gleichgwichtig" und 31,3 Prozent "weniger" nach. 4,1 Prozent gaben an, im "Sonstigen" Bereichen tätig zu sein, 93,3 Prozent machten dazu im Freitextfeld genauere Angaben.

--- Todo: Arbeitsbereich Grafik bauen ----

\subsection{Veränderungen wissenschaftlicher Kommunikation durch die Digitalisierung}

Diese Arbeit folgt der These der Leiterin der Studie durch das SOFI, dass sich der Einfluss der Digitalisierung auf das wissenschaftlichen Kommunikationssystem  durch die beobachtbare Entwicklung der Nutzung von Such- und Rezeptionsmöglichkeiten nach Literatur untersuchen lässt und diese Schnittstelle zwischen informeller und formeller Kommunikation eine Zentrale spielt. \cite{Hanekop_2014}. Der

Diese These konnte auch im Rahem der aktuellen Befragung gestützt werden. "Um sich im eigenen Fach auf dem Laufenden zu halten", nutzte laut der Befragung im Jahr 2007/2008 79 Prozent häufig oder sehr häufig Online-Journale. 2014/2015 stieg dieser Wert auf 88,4 Prozent. Im Gegenzug nutzen nur noch 32,2 Prozent die analogen Printausgaben von Zeitschriften, 2007/2008 waren es noch 50 Prozent der Befragten die auch auf die analoge Version der Journale zurückgriffen.

Auch bei der Beantwortung der Frage, welche der Suchmöglichkeiten häufig genutzt werden, um gezielt nach Literatur zu suchen, lässt sich ein klare Veränderung hin zu den digitalen Medien feststellen. Ähnlich wie in der Befragung 2007/2008 durch das SOFI gab auch 2014/2015 weniger als ein Viertel der Befragten an, über die "Konventionelle Suche" (in Bibliotheksregalen, Archiven etc.) nach Literatur zu suchen. Demgegenüber stieg der Anteil von Wissenschaftlerinnen und Wissenschaftlern, die angaben den Dienst Google Scholar für die Literatursuche zu verwenden, von 31 Prozent in 2007/2008 auf 51,9 Prozent in 2014/2015.

Wie die Ergebnisse der SOFI-Studie \cite{Hanekop_2014}, belegen diese aktuellen Ergebnisse erneut, dass sich die webbasierten Such- und Rezeptionsmöglichkeiten in der wissenschaftlichen Kommunikation durchgesetzt haben. Sie können darüber hinaus auch als Argument für die These nach der generell fortgeschrittene Nutzung digitaler Medien bei der wissenschaftlichen Kommunikation gewertet werden.

\subsection{Verständnis von Offenheit bei der wissenschaftlichen Kommunikation}

16,9 Prozent der Befragten gaben an, sich häufig über Open-Access Repositorien (z.B. arxiv.org) auf dem Laufenden zu halten, 17,1 Prozent beziehungsweise 190 der 1.112 Befragten nutzen Open-Access Portale (z.B. Directory of Open Access Journals) um sich über den aktuellen Stand der Forschung zu informieren. Für die Suche nach Literatur nutzten nur 4,3 Pozent der Teilnehmerinnen und Teilnehmer häufig "Suchmaschinen für Open Access", aber mehr als 50 Prozent die jeweils "fachspezifische Suchportale", die ebenfalls Open-Access Publikationen enthalten.

54,5 Prozent bewerteten "die Forderung nach kostenfreiem Zugang zu allen wissenschaftlichen Publikationen für Leser (Open Access)" als "sehr gut". Knapp unter einem Viertel (22,3 Prozent) erachteten die Forderung als "gut". 19,2 Prozent waren sich bei der Frage unsicher und antworteten mit "teils/teils" und 38 der Befragen lehnten die Forderung nach Open Access ab, 9 davon sogar "entschieden". Nur 7 Teilnehmer und Teilnehmerinnen oder 0,6 Prozent gaben an Open Access nicht zu kennen.

Bei der Betrachtung der Meinung zu Open Access differenziert nach beruflichen Status, herschte in der Gruppe der Doktoranden die größte Einigkeit (89,3 Prozent). Unter den promovierten wissenschaftlichen Mitarbeiter und Mitarbeiterinnen bewerteten 80,3 Prozent die Forderung nach Open Access als "gut" oder "sehr gut". Bei den Wissenschaftlichen Mitarbeitern ohne Promotion (75,3 Prozent), bei Doktoranden mit einer wissenschaftlichen Mitarbeiterstelle (73,0 Prozent) unterstützen und bei Privatdozenten (73,0 Prozent) sowie bei Professoren (71,7 Prozent) ist die Forderung nach freien und offenen Zugang zu wissenschaftlichen Publikationen ebenfalls mehrheitlich stark ausgeprägt. Mit 65,4 Prozent war die Befürwortung unter Juniorprofessoren am geringsten ausgeprägt.

Während 14,8 Prozent angaben in der Open Access-Bewegung engagiert zu sein, verneinten 72,3 Prozent ein Engagement in der Open Access-Bewegung. 13,0 Prozent enthielten sich der Angabe zu ihrem Engagement.

43,4 Prozent oder 483 Personen kommentierten ihre Meinung zu Open Access im optionalen Freitextfeld, wobei 72,9 Prozent der abgegebenen Kommentare von Befragten stammten, die Open Access gut oder sehr gut finden. Unter den Befragten, die Open Access ablehnen, kommentierten 75,7 Prozent ihre Haltung zur Forderung nach kostenfreiem Zugang zu allen wissenschaftlichen Publikationen. Von den Personen, die "teils/teils" angaben, machten fast die Hälfte (48,1 Prozent) ihre unsichere Haltung in den Kommentaren deutlich. 159 der 854 Befragten (18,6 Prozent), die Open Access gut oder sehr gut finden, gaben an, selbst aktiv in der Open Access Bewegung zu sein. Überraschend sind an dieser Stelle 12 Personen (5,6 Prozent) die "teils/teils" bezüglich ihrer Meinung zu der Forderung nach Open Access angegeben haben, aber sich dennoch zum Teil der Bewegung zählen.

In einer weiteren Frage wurde das Verständnis von Open Access am Beispiel der Definition der Budapest Open Access Initative \cite{boai_2012} abgefragt. Knapp drei Viertel der Teilnehmer und Teilnehmerinnen (74,9 Prozent) stimmten dieser Definition uneingeschränkt zu, 19,0 Prozent stimmten dieser Definition nur "teils/teils" zu und 2,4 Prozent lehnten die Definition ab. 3,2 Prozent der Befragten beantworteten die Frage mit "weiß nicht" und fünf Teilnehmer enthielten (0,5) sich der Beantwortung der Frage. Wurde "teils/teils" oder "weiß nicht" als Antwort ausgewählt, konnte in einer optionalen Freitext-Frage beantwortet werden, welche Aspekten der Definition genau keine Zustimmung und welche Zustimmung fanden. Davon machten 37,7 Prozent der möglichen Befragten gebrauch.

--- TODO: freitextfeld auswerten ----

\subsection{Interesse an Offenheit bei der wissenschaftlichen Kommunikation}

Die Mehrheit der Befragten gab an, Interesse am Zugang zu Forschungdaten anderer Wissenschaftler und Wissenschaftlerinnen zu haben (71,3 Prozent). Über ein Drittel dieser Teilnehmer und Teilnehmerinnen erklärte ihr genaues Interesse die Beantwortung der optionalen Frage. 28,7 Prozent (319 der Befragten) hatte kein Interesse am Zugang zu Forschungsdaten anderer Wissenschaftler und Wissenschaftlerinnen.

Die Frage, ob sich die befragten Wissenschaftler vorstellen können ihre "Forschungsdaten und alle weiteren Informationen, die während der wissenschaftlichen Arbeit anfallen (z.B. Laborbücher, Entwürfe oder andere Dokumente und Daten) unter Berücksichtigung von Datenschutz öffentlich zur Verfügung zu stellen", beantworteten 28,0 Prozent uneingeschränkt mit "ja". 36,3 Prozent der Befragten schränkten ein, dass sie das "nur unter bestimmten Bedinungen" tun würden und 29,0 Prozent lehnten die Veröffentlichung von Forschungsdaten und weiteren Informationen komplett ab. 6,7 Prozent wußten darauf keine Antwort. Die bedingte, Freitextfrage nach der Erläuterung der "bestimmten Bedingungen" unter denen die Befragten Forschungsdaten und weiteren Informationen veröffentlichen würden beantworteten 214 Teilnehmer und Teilnehmerinnen.

--- TODO: freitextfeld auswerten ----

\subsection{Offenheit im wissenschaftlichen Alltag}

Bei der Auswertung der Erhebung konnte ein mehrheitlich stark verbreitetes Verständnis von Open Access und die mehrheitliche Unterstützung der Forderung nach Öffnung von Wissenschaft sowie Interesse an Forschungsdaten anderer dargestellt werden. Demgegenüber steht die Frage, wie wichtig den befragten Wissenschaftlern das Kriterium "freier Zugang zum Volltext" bei den eigenen Veröffentlichungen ist. Der größere Teil der Befragten (49,8 Prozent) erachten dies als "weniger wichtig" oder "nicht wichtig". Demgegenüber erachteten nur 44,7 Prozent das Kriterium "freier Zugang zum Volltext im Internet" als wichtig oder sehr wichtig bei der eigenen Veröffentlichung. 5,5 Prozent der 1.112 Befragten enthielten sich der Beantwortung der Frage.

Diese negative Bewertung des Kriteriums "freier Zugang" wird durch die Betrachtung weiterer Merkmale, die Wissenschaftler bei den eigenen Veröffentlichungen als "wichtig" oder "sehr wichtig" erachten, bestätigt. Während der fachlich einschlägige Schwerpunkt (91,2 Prozent), das Renommee der Zeitschrift/des Verlags (81,7 Prozent) und akzeptable oder keine Veröffentlichungskosten für Autoren (79,2) relativ häufig als "wichtig" oder "sehr wichtig" erachtet wird, stellt die Veröffentlichung unter einer Open-Access Lizenz nur für 33,1 Prozent ein wichtiges Kriterium bei der Publikation eigener Inhalte dar. Die Mehrheit der Befragten (56,4 Prozent) erachten dieses Kriterium sogar als "weniger wichtig" oder "nicht wichtig". Der akzeptable Preis für den Erwerb der Publikation spielt ebenfalls nur für 43,4 Prozent eine wichtige Rolle, für 50,0 Prozent der Befragen ist er "weniger wichtig" oder "nicht wichtig".

Weitere Kriterien bei der Frage, der Wichtigkeit von Kriterien bei der wissenschaftlichen Veröffentlichung von Beiträgen oder Büchern aus Sicht der Befragten sortiert:
\begin{itemize}
\item 77,8 Prozent der Teilnehmer und Teilnehmerinnen der Erhebung sehen die internationale Verbreitung als mindestens wichtige, wenn nicht sogar sehr wichtiges Kriterium im Rahmen der eigenen Veröffentlichungen an, die übrigen 19,4 Prozent finden dieses "weniger wichtig" oder "unwichtig".
\item Das Peer-Review Verfahren wird von 75,4 Prozent der Befragten als wichtiges Kriterium erachtet nur 18,6 Prozent sind diezbezüglich gegensätzlicher Meinung.
\item 75,4 Prozent der Befragten Personen sehen die Transparenz des Review-Prozesses als wichtig an während 17,9 Prozent dieses Kriterium als "weniger wichtig" oder "unwichtig" erachten.
\item Eine leichte Auffindbarkeit der eigenen Publikation im Internet ist 71,2 Prozent der befragten Wissenschaftler und Wissenschaftlerinnen mindestens wichtig. Für 25,0 Prozent der Befragten ist das "weniger" bis "nicht wichtig".
\item Die rasche Veröffentlichung der eigenen Publikation ist für 68,0 Prozent von Bedeutung. Für 28,7 Prozent hat dieses Kriterium keine besondere Bedeutung.
\item Das Ranking, wie der Impact-Faktor einer wissenschaftlichen Zeitschrift, wurden 58,4 Prozent der befragten Personen als "wichtig" und von 35,1 Prozent als "weniger wichtig" oder "unwichtig" bewertet.
\item Die Reputation der Herausgeber war für die Mehrheit der Befragten eher "unwichtig" bis "nicht wichtig" (48,3 Prozent). Demgegenüber erachteten 46,5 Prozent dieses Kriterium als sehr wichtig bis wichtig.
\end{itemize}

--- Todo: Grafik bauen ----

Eine weitere Frage im Fragebogen betraf die Einschätzung der Befragten, ob ihre eigenen Veröffentlichungen in Zeitschriften oder Büchern für Leser potentiell gut zugänglich sind. 32,0 Prozent der Befragten beantworteten die Frage mit der Option "ja, gut zugänglich". Mit "teils/teils" antworteten 46,9 Prozent und 11,5 Prozent der Teilnehmer und Teilnehmerinnen wählte die Option "nein, nicht so gut zugänglich" (9,2 Prozent) oder "nein, sehr schlecht"(2,3 Prozent). 107 oder 9,6 Prozent wussten die Frage nicht zu beantworten.

Bei der Frage, ob Aufsätze, Texte oder Bücher publiziert wurden, die vom Verlag selbst frei zugänglich gemacht wurden, antworteten 140 Teilnehmer und Teilnehmerinnen (12,6 Prozent) mit "ja, einen Beitrag" und 23,1 Prozent mit "ja, mehrere Beiträge". 54,4 Prozent oder 605 der Befragten, hatte zum Zeitpunkt der Befragung noch keine Aufsätze, Texte oder Bücher publiziert, die vom Verlag selbst frei zugänglich gemacht wurden.  18,5 Prozent derer, die bisher noch nicht bei einem Verlag Open Access veröffentlicht hatten, gaben an dies zu planen. 9,9 Prozent der Befragten beantworteten die Frage nicht.

Die 397 (35,70 Prozent) der 1.112 Befragten die angaben, bereits Inhalte frei publiziert zu haben, wurden gebeten auszusagen, wieviele Aufsätze, Texte oder Bücher sie bisher frei veröffentlicht haben:
\begin{itemize}
\item Bücher - 63 Befragte (15,9 Prozent) beantworteten die optionale Frage. Davon gaben 26 an, bisher kein Buch veröffentlicht zu haben, das frei zugänglich gemacht wurde. Bezieht man die Antworten nicht mit ein, die 0 Bücher angaben, hatten die 37 Befragten jeweils circa 2 Bücher veröffentlicht, die vom Verlag selbst frei zugänglich gemacht wurden.
\item Texte - 192 der 397 Befragten (48,4 Prozent) gaben an, mindestens 1 Text frei veröffentlicht zu haben. Im Durchschnitt hatten die Befragten jeweils rund 3 Texte "frei zugänglich" veröffentlicht.
\item Daten - 2,5 Prozent (10 Personen) gab an, mindestens einen Datensatz frei veröffentlicht zu haben.
\item Sonstiges - Keiner der Teilnehmerinnen und Teilnehmer gab an, "sonstige Beiträge" frei veröffentlicht zu haben.
\end{itemize}

Den Aufwand, die eigene Publikationen im Internet frei zur Verfügung zu stellen, schätzte der größte Teil der Befragten (30,8 Prozent) als gering ein. Demgegenüber schätzten 255 der Befragten Wissenschaftler und Wissenschaftlerinnen (28 Prozent) den Aufwand ihre Publikationen im Internet frei zur Verfügung zu stellen als "mittelgroß" oder "groß" ein. 22,6 Prozent waren sich unsicher und wählten "teils/teils" und 18,6 Prozent wusste nicht den Aufwand einzuschätzen.

Während die Mehrheit der Befragten den Aufwand für die freie Veröffentlichung als nicht groß einschätzte, zeichnete sich bei der Auswertung der Frage nach dem geschätzten Aufwand für die Veröffentlichung von Forschungsdaten im Internet ein anderes Bild. 55,0 Prozent der Befragten schätzte den Aufwand die Forschungdaten zu veröffentlichen als "groß" ein. Die kleinste Gruppe der Befragten (9,6 Prozent) vermutete dabei einen geringen Aufwand, 15,0 Prozent schätzten den Aufwand "teils/teils" ein und ein Fünftel (20,3 Prozent) wußte die Frage nicht zu beantworten.

--- Todo: Grafik aus beiden Items bauen ---

\subsection{Treiber und Bremser für die Öffnung von Wissenschaft und Forschung}

In Kapitel --- TODO: definieren --- wurden Treiber und Bremser für die Öffnung der wissenschaftlicher Kommunikation in der Literatur identifiziert und herausgearbeitet. Diese wurden im Rahmen der empirischen Erhebung abgefragt. Im Rahmen der Befragung sollten darüber hinaus auch evaluiert werden, welche Faktoren und Argumente aus der Sicht von Wissenschaftlern die Öffnung von Wissenschaft und Forschung in der jeweiligen wissenschaftlichen Disziplin begünstigen und welche sie behindern.

Die Auswertung nach Häufigkeit der ausgewählten Antwortoptionen für die Öffnung der wissenschaftlichen Kommunikation aus Sicht der befragten Wissenschaftler und Wissenschaftlerinnen ergab folgende Reihenfolge:
\begin{enumerate}
\item 721 (64,8 Prozent) mal wurde das Argument "Beschleunigung der Wissensverbreitung und -verwertung" von den Befragten ausgewählt.
\item Das Argument der "Eröffnung neuer Möglichkeiten für die Wissensverbreitung" wurde von 63,8 Prozent der 1.112 Befragten unter den Antwortmöglichkeiten am zweithäufigsten ausgewählt.
\item Die umfassenderen Verfügbarkeit von bereits finanzierter Forschung stellte für 55,2 Prozent ein Argument für die offene und freie Veröffentlichung der eigenen wissenschaftliche Kommunikation dar.
\item Eine generelle Erleichterung der wissenschaftliche Kommunikation, erachteten 49,2 Prozent als Argument für die Öffnung der wissenschaftlichen Kommunikation.
\item Die Förderung des interdisziplinären Austausch von Wissenschaftlern und Wissenschaftlerinnen erachteten 45,1 Prozent als Argument an.
\item 44.0 Prozent oder 489 der Befragten sahen in der Überwindung sozialer, nationaler und globaler Wissenskluften ein Argument.
\item Die Chance einer umfassendere und transparentere Qualitätsmessung von Wissenschaft sahen 33,7 Prozent der befragten Wissenschaftler als Argument für die Öffnung an.
\item 250 oder 22,5 Prozent der Befragten sahen in der nachhaltigen und unabhängigen Archivierung der Informationen ein Argument für die Öffnung von Wissenschaft und Forschung.
\item 20,0 Prozent betrachteten die Ermöglichung von indirekter Wirtschaftsförderung durch freien und offenen Wissenstransfer als Argument.
\item Die Möglichkeit der Beilegung der vorherrschenden Zeitschriften- und Monographienkrise akzeptierten 16,3 Prozent der Befragten als Argument für die Öffnung der eigenen wissenschaftlicher Kommunikation.
\end{enumerate}

8,3 Prozent der 1.112 Befragten gaben an, dass ihrer Meinung nach keins der genannten Argumente für die eigene Öffnung der wissenschaftlichen Kommunikation und aller Informationen aus dem Forschungs-/Arbeitsprozess sprechen. Weitere 47 oder 4,2 Prozent machen weitere Angaben unter "Sonstiges".

---- Todo: weiter ausarbeiten v.a. Sonstige aufteilen und Grafik bauen ----

Bei der Frage nach den Argumenten gegen die Öffnung der wissenschaftlichen Kommunikation zeichnete sich ein uneindeutigeres Bild:
\begin{enumerate}
\item Am häufigsten (43,4 Prozent) wurden von den Befragten die fehlenden Reputationskriterien für die Bewertung von offener Wissenschaft gewählt.
\item Knapp dahinter (40,2 Prozent) wurde die "Gefahr der Fehlinterpretation und Falschinformation durch Wissenschaft" ausgewählt.
\item Der erhöhte zeitliche Mehraufwand für die Bereitstellung der wissenschaftlichen Publikationen und/oder Forschungsdaten führten 379 beziehungsweise 34,1 Prozent als Argument an.
\item 30,0 Prozent erachteten das Argument einer Erschwerung der eindeutigen Zuordnung von Texten, Arbeiten und Daten zu den Urhebern als Argument gegen die eigene Öffnung wissenschaftlicher Kommunikation an.
\item  An fünfthäufigster Stelle wurde dem Argument zugestimmt, dass die Qualität der wissenschaftlichen Arbeit leidet (27,3 Prozent).
\item Dem Argument, dass die Langzeitarchivierung und langfristige Auffindbarkeit nicht (dezentral) gewährleistet werden kann, stimmten 26,4 Prozent der Befragten zu.
\item Das Argument, dass die freie und offene Verfügbarkeit wissenschaftlicher Informationen zu hohe Kosten führt und keine Refinanzierung absehbar ist, stimmten 274 der 1.112 Befragten (24,6 Prozent) zu.
\item 9,0 Prozent der Befragten akzeptierten das Argument, dass die Öffnung der Kommunikation eine Bedrohung für Publikations- und Pressefreiheit in der Wissenschaft darstellt.
\item Dass die offene Bereitstellung von Daten keinen nachhaltigen Mehrwert bietet, dem stimmen 8,7 Prozent zu und sehen darin ein Argument gegen eine Öffnung des Systems.
\item Dass "Offenheit und Transparenz bei Forschungsförderung Freiheit von Wissenschaft und Forschung gefährden" sahen 5,4 Prozent als Argument gegen die eigene Öffnung der wissenschaftlichen Kommunikation an.
\end{enumerate}

154 Teilnehmer und Teilnehmerinnen nannten sonstige Argumente (13,9 Prozent), die ihrer Meinung nach gegen Open Access und Open Science anzuführen sind. 7,7 Prozent aller Befragten gab an, dass kein Argument gegen die Öffnung der wissenschaftlichen Kommunikation und aller Informationen aus dem Forschungs-/Arbeitsprozess spricht.

---- Todo: weiter ausarbeiten v.a. Sonstige aufteilen und Grafik bauen ----

\subsection{Wissenschaftliche Reputation und die Öffnung von wissenschaftlicher Kommunikation}

Die 1.112 Teilnehmer und Teilnehmerinnen hatten die Möglichkeit unter mehreren Antwortoptionen auszuwählen, welche Faktoren für wissenschaftliche Reputation in ihrer Disziplin wichtig sind. Am häufigsten wurde dabei die "Anzahl der Beiträge" ausgewählt, 79,8 Prozent aller Befragten wählten diese Option. Die "Relevanz der publizierten Ergebnisse" wurde von 73,7 Prozent und "Vorträge auf wichtigen Konferenzen" von 68,2 Prozent der Wissenschaftler als wichtiger Faktor für wissenschaftlichen Reputation in der jeweiligen Disziplin ausgewählt.

Folgende weitere Auswahlmöglichkeiten wurden von den Befragten der Häufigkeit nach ausgewählt:
\begin{itemize}
\item die "Bezugnahme und Zitation" durch Kollegen wurde von 65,5 Prozent am vierthäufigsten genannt
\item 64,7 Prozent nannten den Indikator "Drittmittelprojekte" wichtig für die Reputation
\item 675 oder 60,7 Prozent der Befragten Wissenschaftler und Wissenschaftlerinnen nannten "Ranking- oder Impactfaktoren" als wichtigen Faktor für Reputation in ihrer Disziplin
\item das Renommee der Forschungseinrichtung war für weniger als die Hälfte der Befragten (48,1 Prozent) relevant
\item auch das Renommee von Herausgebern oder Mitautoren spielte nur für 36,06 Prozent einen wichtigen Faktor für die Reputation in der eigenen Disziplin
\item "Netzwerke, Kontakte und ob man dazu gehört" erachteten 35,16 Prozent als wesentlichen Reputationsfaktor
\item die Gutachtertätigkeit, Herausgeberschaft oder andere Funktionen sehen 27,7 Prozent als wichtig für die Reputation an
\item knapp ein Viertel der Befragen (23,7 Prozent) gab an, dass die "Anzahl Monografien" wichtig für die Reputation in der jeweiligen ist
\item "Anwendungsrelevanz und Verwertbarkeit" war für 13,3 Prozent der Teilnehmerinnen und Teilnehmer ein wichtiges Kriterium für wissenschaftliche Reputation
\item "materielle Ausstattung, Großgeräte etc." wählten 12,9 Prozent der Befragten
\item 143 der Befragten (12,7 Prozent) gaben an, dass öffentliche Aufmerksamkeit wichtig für Reputation in ihrer Fachdisziplin ist
\item die "personelle Ausstattung" zählte mit 12,5 Prozent eher selten zu den wichtigen Faktoren für wissenschaftliche Reputation in allen Disziplin
\item für 7,6 Prozent der Befragten stellten "Patente" ein Kriterium für Anerkennung dar
\item unter den vorgegebenen Antwortmöglichkeiten stellte die "politische Relevanz" mit 2,8 Prozent der Befragten den unwichtigsten Faktor für Reputation dar
\item 12 Teilnehmer und Teilnehmerinnen (1,1 Prozent) gaben in einem Freitextfeld "Sonstige" Faktoren an
\end{itemize}

---- Todo: sostigeweiter ausarbeiten v.a. Grafik bauen ----

\subsection{Auffassungen zwischen den unterschiedlichen Fachdiziplinen}

Die Frage, ob die Befragten ihre Veröffentlichungen in Zeitschriften oder Büchern für potentielle Leser gut zugänglich befinden, bejahten in den Naturwissenschaften 38,8 Prozent, in den Lebenswissenschaften 36,0 Prozent, in den Ingenieurwissenschaften 31,9 Prozent und in den Geistes- und Sozialwissenschaften 25,1 Prozent.

Laut der Erhebung gaben in der Fachgruppe der Ingenieurwissenschaften mehr als 2/3 der Befragten an (75,2 Prozent), Interesse an den Forschungsdaten anderer  zu haben. In den Lebenswissenschaften bestand mit 74,6 Prozent der Befragten, das zwitgrößte Interesse an bereits erhobenen wissenschaftlichen Daten. Unter den 418 teilnehmenden Sozial- und Geisteswissenschaftlern bekundeten 69,9 Prozent und unter den Naturwissenschaftlern 69,0 Prozent ein Interesse an den Forschungsdaten ihrer wissenschaftlichen Kollegen und Kolleginnen.

Die Untersützung für die Forderung nach Open Access ist in den befragten Fachgruppen unterschiedlich stark ausgeprägt. 88,3 Prozent der teilnehmenden Wissenschaftler und Wissenschaftlerinnen aus den Lebenswissenschaften bewerteten die Forderung nach kostenfreiem Zugang zu allen wissenschaftlichen Publikationen für Leser (Open Access) mit "sehr gut" oder "gut". In den Naturwissenschaften beführworteten 82,0 Prozent den kostenfreiem Zugang zu allen wissenschaftlichen Publikationen für Leserinnen und Leser, in den Ingenieurwissenschaften 71,6 Prozent und in Geisteswissenschaften hatten 68,2 Prozent der Befragten eine gute oder sehr gute Meinung zu Open Access.

Signifikante Unterschiede zwischen den verschiedenen Fachgruppen ergab die Auswertung der Erhebung bei der Frage nach der Wichtigkeit von Offenheit bei den eigenen wissenschaftlichen Publikationsvorhaben. In den Geistes- und Sozialwissenschaften erachten 37,7 Prozent die "freie Verfügbarkeit des Volltexts im Internet" und 29,7 Prozent die "Veröffentlichung unter einer Open-Access Lizenz" als "sehr wichtig" oder "wichtig" an. Für die überwiegende Anzahl der befragten Geistes- und Sozialwissenschaftler sind beide Faktoren eher "wichtig" oder "weniger wichtig". Auch in den Ingenieurwissenschaften antworteten die Befragten auf die Frage nach der Wichtigkeit der freie und offene Verfügbarkeit ihrer Beiträge im Internet (57,8 Prozent) und die "Veröffentlichung unter einer Open-Access Lizenz" (47,6 Prozent) mehrheitlich mit "weniger wichtig" oder "nicht wichtig". Demgegenüber gaben die Befragten der Fachgruppe der Lebenswissenschaften mehrheitlich an, dass ihnen die freie Verfügbarkeit der Texte im Internet "wichtig", oder "sehr wichtig" ist (61,1 Prozent). Auch die Veröffentlichung unter einer Open-Access Lizenz war für 53,6 Prozent der befragten Lebenswissenschaftler und -wissenschaftlerinnen mindestens wichtig. Für die Mehrheit der Befragten, die sich den Naturwissenschaften zugeordnet hatte, war ebefalls, wenn auch knapp, mehrheitlich wichtig oder sehr wichtig (51,7 Prozent), dass ihr Beitrag als Volltext frei im Internet verfügbar ist. Die Veröffentlichung unter einer Open Access Lizenz war den Wissenschaftlern allerdings weniger oder nicht wichtig (63,9 Prozent).

Die Anzahl der Publikationen wird von den Befragten aller Fachgruppen als wichtiger Faktor für wissenschaftliche Reputation gesehen. Die Unterschiede bei den Antworten der Befragten bezüglich der Frage nach der Wichtigkeit dieses Faktors lagen zwischen den Fachgruppen liegen zwischen 82,8 Prozent in den Geistes- und Sozialwissenschaften, 80,8 Prozent in den Naturwissenschaften, 77,7 Prozent in den Lebenswissenschaften und 73,1 Prozent in den Ingenieurwissenschaften. Die Relevanz der publizierten Ergebnisse wird in den verschiedenen Fachgruppen als unterschiedlich wichtig betrachtet. In den Naturwissenschaften gaben 80,7 Prozent der Befragten an, dass die Relevanz der publizierten Ergebnisse ein wichtiger Faktor für Reputation in ihrer Disziplin ist. In den Lebenswissenschaften 77,7 Prozent, in den Geistes- und Sozialwissenschaften 69,6 Prozent und in den Ingenieurwissenschaften 68,1 Prozent. Die Relevanz von Rankings und des Impact-Factors von Journalen wird in den Fachgruppen unterschiedlich stark bewertet. In den Lebenswissenschaften stellt dieser Faktor für 84,8 Prozent der Befragten einen wichtigen Reputationsfaktor dar. 72,1 Prozent der teilnehmenden Wissenschaftler und Wissenschaftlerinnen der Naturwissenschaften, 56,0 Prozent der Ingenieurwissenschaftler und 43,8 Prozent der Sozial- und Geisteswissenschaftler erachten den Faktor Ranking beziehungsweise den gemessenen Impact der Journale als wichtig für die Reputation. Diese Zahlen decken sich auch mit dem Faktor "Anzahl der Monografien" und Wichtigkeit der der jeweiligen Publikationsform in den unterschiedlichen Fachdisziplin. Da die Publikationsform der Monografien nur in den Geistes- und Sozialwissenschaften eine signifikante Rolle spielt (76,6 Prozent) gaben auch nur in dieser Fachgruppe mehr als die Hälfte der Befragten an (50,2 Prozent), dass es sich bei der Anzahl der Monografien um einen wichtigen Faktor für die wissenschaftlichen Reputation sind in Ihrer Disziplin handelt. In den Ingenieurwissenschaften gaben nur 12,8 Prozent der Befragten an, dass es sich dabei um ein wichtigen Faktor handelt, in den Lebenswissenschaften 5,58 Prozent und in den Naturwissenschaften nur 4,4 Prozent. In allen Fachgruppen spielte der Faktor "Drittmittelprojekte" mehrheitlich eine wichtige Rolle. Am stärksten war der Faktor in den Lebenswissenschaften (74,1 Prozent), gefolgt von den Naturwissenschaften (67,1 Prozent) und den Ingenieurwissenschaften (61,7 Prozen). 58,9 Prozent der Geistes- und Sozialwissenschaftler gab an, dass der Faktor "Drittmittelprojekte" eine wichtige Rolle für Reputation in der jeweiligen Fachdisziplin spielt.

Nur die Befragten in den Lebenswissenschaften (52,5 Prozent) gaben mehrheitlich an Aufsätze, Texte oder Bücher publiziert zu haben, die vom Verlag selbst frei zugänglich gemacht wurden. Weitere 11,3 Prozent haben bisher keine Beiträge frei zugänglich veröffentlicht, planen das aber. In den Geistes- und Sozialwissenschaften gab 37,1 Prozent an Beiträge als Open Access veröffentlicht zu haben und weitere 12,1 Prozent eine Veröffentlichung als frei zugänglich zu planen. 36,0 Prozent der Naturwissenschaftler gaben an, mindestens einmal schon über einen Verlag frei verfügbar veröffentlicht zu haben, weitere 9,4 Prozent planen das. Knapp ein Drittel (34,5 Prozent) der befragten Ingenieurwissenschaftler gaben an frei zugänglich veröffentlicht zu haben und 10,1 Prozent planen es.

--- Todo Grafik bauen ----

Bis auf die Ingenieurwissenschaften (48,9 Prozent) hat laut den Befragten in allen Fachgruppen der Publikationsdruck in den vergangenen fünf Jahren zugenommen. In den Lebenswissenschaften 71,1 Prozent, 62,4 Prozent Naturwissenschaften und 61,5 Prozent in den Naturwissenschaften. Unter den Befragten der Lebenswissenschaften gab mehr ein Viertel (26,4 Prozent) an, das der Druck zu veröffentlichen "sehr stark" angestiegen ist.

\subsection{Auffassungen in den unterschiedlichen Altersgruppen der Befragten}

Anteilig an den Altersgruppen stellten die 56-60 Jährigen die größte Gruppe der Befürworter (83,3 Prozent) der Forderung nach kostenfreiem Zugang zu allen wissenschaftlichen Publikationen für Leser. Unter den 113 Befragten 36-40 Jährigen fanden 81,4 Prozent die Forderung "sehr gut" oder "gut". 81,1 Prozent der 26-30 Jährigen fanden die Forderung erwartungsgemäß mindestens gut.

--- Grafik bauen ---

\subsection{Auffassungen zwischen den unterschiedlichen Statusgruppen}

Eine genauere Betrachtung des Interesses an Forschungsdaten anderer in Kombination mit dem Item "beruflicher Status" zeigte, dass vor allem unter Doktoranden ein solches Interesse besteht. 79,2 Prozent der 118 befragten Doktoranden und 78,8 Prozent der 175 Doktorand_in mit einer Stelle als Wissenschaftlicher Mitarbeiter bejaten das Interesse. Drei viertel der wissenschaftlichen Mitarbeiter ohne Promotion (75,4 Prozent) und 72,4 Prozent der promoviertern wissenschaftlichen Mitarbeiter waren ebenfalls überwiegend an den Daten anderer interessiert. Unter den Juniorprofessoren gaben noch 61,5 Prozent der Befragten ein Interesse an den Daten an. Die etablierten Wissenschaftlern, wie Professoren (58,1 Prozent) und Privatdozenten (59,3 Prozent) waren weniger aber ebenfalls überwiegend an den Daten anderer interessiert.  Studenten, wissenschaftler aus der Privatwirtschaft, "Sonstige" und Bibliothekare waren mit 74,1 Prozent ebenfalls stark an Forschungsdaten interessiert.


\subsection{Veränderungen im Vergleich zur SOFI Studie}

Die Vergleich bezüglich der Samples in Bezug auf beruflicher Status, Alter und Geschlecht der befragten WissenschaftlerInnen ist ein Indikator für die Vergleichbarkeit der Samples. Bei beiden Erhebungen ist die Verteilung bezüglich des beruflichen Status im wissenschaftlichen System vergleichbar.
--- Todo: Grafik bauen ---
Auch der Vergleich der Altersgruppen beider Studien zeigt klare Überschneidungen bei den Befragten.  --- Todo: Grafik bauen ---

Wie in der Befragung vom SOFI 2008 wurden die Teilnehmer und Teilnehmerinnen gefragt, wie sie sich in ihrem Fachgebiet auf dem Laufenden halten und welchen Zugang sie zu den (Voll-) Texten haben \cite{hanekop_2008}. Auch die Antworten der 1.446 der Befragten zeigen "wie weitreichend sich bei der gezielten Suche nach Literatur digitale Suchmöglichkeiten durchgesetzt haben" \cite{hanekop_2008}. Fast 50 Prozent der Befragten in 2014 gaben an, die Google Suche häufig als Suchmöglichkeiten zu nutzen, um gezielt nach Literatur zu suchen. Bei der Befragung 2007 gaben 46 Prozent der Befragten an sehr häufig die Google Suche zu verwednen. Diese Entwicklung liegt im Trend, denn die IT-gestütze Suche lag in den 1980er Jahren bei einem Prozent, stieg bis 1993 auf neun Prozent an und betrug im Jahr 2003 bereits 24 Prozent \cite{hanekop_2008}.

--- Todo: Zahlen prüfen wegen Grundgesamtheit Grafik bauen ---

Ein ähnliches Bild zeigt sich bei dem Vergleich der Umfrageergebnisse bei der Frage wie sich die Teilnehmer in Ihrem Fachgebiet auf dem Laufenden halten. 2007 gaben 57 Prozent an, sich sehr häufig über Online-Zeitschriften auf dem aktuellen Stand der wissenschaftlichen Debatte zu halten. In der Befragung im Rahmen dieser Arbeit gaben 67,71 Prozent der 1.446 Befragten an sich in Online-Zeitrschriften zu informieren. Gefolgt wird diese Option durch die Teilnahme an Tagungen oder Kongressen (56,1 Prozent) und Gespräche mit Fachkollegen (55.1 Prozent). Social Media Platformen spielen mit bisher knapp 6 Prozent eher eine kleinere Rolle. Online-Datenbanken, Online-Archive, die 2007 noch zweithäufigste Option sich auf dem Laufenden zu halten, bleibt annähernd auf dem gleichen Niveau.

--- Todo: Zahlen prüfen wegen Grundgesamtheit 1.446 Grafik bauen ---

Im Jahr 2007 fanden rund 81 Prozent der Befragten die Forderung nach kostenfreiem Zugang zu allen wissenschaftlichen Publikationen für Leser gut bis sehr gut. In der Befragung 2014 fiel das Ergebnis mit einer Befragtenzahl von 1.112 mit 76,8 Prozent zwar niedriger aber dennoch weiterhin mehrheitlich positiv aus.

Ein weiteres Ergebnis der Studie des Soziologische Forschungsinstituts Göttingen im Jahr 2007, dass "gerade auch die etablierten und damit etwas älteren Wissenschaftler nutzen internetbasierte Plattformen intensiv". In der aktuellen Befragung gaben 86,4 Prozent der über 50-jährigen Befragten an, sich mit "Online-Ausgaben von Zeitschriften" "häufig auf dem Laufenden zu halten". In dieser Altersgruppe greifen jedoch auch noch 50 Prozent zu "Print-Ausgaben von Zeitschriften". In der Altersgruppe unter 50 Jahren nutzen es gerade mal 29,0 Prozent die Print-Ausgaben von Zeitschriften. Print-Bücher hingegen finden altersgruppenunabhängig bei rund 52,8 bis 53.7 Prozent Verwendung bei den Befragten. Bei digitalen Büchern sind es bei den über 50 Jährigen 19,8 Prozent und bei den unter 50 Jährigen mit 38,8 Prozent fast doppelt so häufig das Mittel der Wahl um sich in dem jeweiligen Fachgebiet auf dem Laufenden zu halten.

--- Todo: Grafik bauen ---

In der Studie 2007 gaben insgesamt 80 Prozent an sich mit Onlineausgaben auf dem Laufenden zu halten. Sieben Jahre später stieg die Nutzung nochmals um über 8 Prozent auf 88,4 Prozent an. Die Situationen in denen die Befragten nicht Online-Version eines Aufsatzes zugreifen können, weil es keine Lizenz gibt wurde ebenfalls seltener. Gaben 2007 noch 45 Prozent an, haufig bis sehr häufig nicht auf Aufsätze und Texte online zugreifen zu können, waren es in der aktuellen Befragung nur noch 32,4 Prozent. 66,8 Prozent der teilnehmenden Wissenschaftler gaben an nur gelegentlich bis nie Probleme mit dem Zugang zu Onlinetexten zu haben. In der Befragung 2007 waren es nur 52 Prozent.

--- Todo: Grafik bauen ---

Bei der Fragen, wie häufig es kommt für die Befragten vorkommt, dass Sie auf die Online-Version eines Aufsatzes nicht zugreifen können, ist eine leichte Verschiebung zu gunsten der Verfügbarkeit für die Wissenschaftler und Wissenschaftlerinnen festzustellen. Während 2007/2008 9 Prozent der Teilnehmer "sehr häufig" oder 36 Prozent "häufig" auf die Online-Version eines Aufsatzes nicht zugreifen konnten, gaben 2014/2015 nur 6,3 Prozent an "sehr häufig" beziehungsweise 26,1 Prozent "häufig" auf Onlineinhalte nicht zugreifen konnten. "Gelegentlich" konnten 2007/2008 38 Prozent und 2014/2015 fast die Hälfte (49.5 Prozent) nicht auf die Webversion von Inhalten zugreifen, weil es keine Lizenz dafür gab. Selten oder nie Probleme mit dem Zugriff hatten 2014/2015 17,4 Prozent. In 2007/2008 waren es 14 Prozent.

Ob Aufsätze oder Bücher publiziert wurden, die vom Verlag selbst frei zugänglich gemacht wurden antworteten 2007/2008 34 Prozent mehr als einen Beitrag frei zugänglich verfügbar gemacht zu haben. Im Vergleich dazu gaben 2014/2015 25,7 Prozent der antwortenden Personen an mehrere Beiträge frei zugänglich veröffentlicht zu haben. 14,0 Prozent hatten laut der Befragung 2014/2015 einen Beitrag veröffentlicht (23,1 Prozent 2007/2008). Wie in 2007/2008 (11 Prozent) gaben auch 2014/2015 mit 11,2 Prozent fast ähnlich viele Personen an eine frei zugängliche Publikation zu planen. In der aktuellen Erhebung gab fast die Hälfte (49,2 Prozent) an bisher keine offenen Publikationen veröffentlicht haben und das auch nicht zu planen. 2007/2008 waren es nur 32 Prozent, die keine frei zugängliche Publikation veröffentlicht oder geplant haben.

--- Todo: Grafik mit Vergleich bauen ---

In Bezug auf die Wichtigkeit der Faktoren für wissenschaftlichen Reputation in den verschiedenen Disziplinen war in der Befragung 2007/2008 für 92 Prozent der Befragten die "Relevanz der Ergebnisse" wichtig oder sehr wichtig. 2014/2015 gaben 73,7 Prozent der Befragten an, dass "Relevanz der Ergebnisse" ein wichtiger Faktor für Reputation ist. Der am häufigsten augewählte Faktor für Reputation war in der aktuellen Erhebung "die Anzahl der Aufsätze / Beiträge" (79,8 Prozent). In 2007/2008 waren für 82 Prozent der Befragen die Bezugnahme beziehungsweise die Zitation durch Kollegen wichtig oder sehr wichtig, in der aktuellen Befragung identifizierten diesen Faktor 65,5 Prozent.

Die Frage, ob der Publikationsdruck in Ihrem Fachgebiet in den vergangenen fünf Jahren zugenommen hat, beantworteten in der SOFI-Studie ähnlich wie in der aktuellen Erhebung. 18,4 Prozent antworteten aktuell und 21,9 Prozent in 2007 mit ja, sehr stark. Unverändert ist der Publikationsdruck aktuell bei 21,9 Prozent versus 19,8 Prozent 2007. Die größte Gruppe antwortete mit "ja"; 2007 43,6 Prozent und 2014/2015 42,9 Prozent. In der ektuellen Befragung waren sich 16.5 Prozent unsicher, in der Befragung vor 7 Jahren waren es 14,1 Prozent.

Finden Sie, dass Ihre Veröffentlichungen in Zeitschriften oder Büchern für potentielle Leser gut zugänglich sind, beantworteten 7,9 Prozent weniger als 2007, dass ihre Veröffentlichungen gut zugänglich sind. "Teils/teils" antworteten in beiden Erhebungen ähnlich viele: 46,5 Prozent in 2007 versus 46,9 Prozent in 2014. Nicht gut zugänglich befanden 2014 9,2 Prozent, 2007 waren noch 6,0 Prozent.

--- Todo: Grafik mit Vergleich bauen ---

Die Frage nach dem Aufwand, für die freie Veröffentlichung von Publikationen im Internet beantworteten in der aktuellen Befragung 30,8 Prozent und 2007 32,5 Prozent. Mittelgroßen Aufwand vermuteten 2007 22,9 Prozent und 2014 24,7 Prozent. "Teils/teils" gaben aktuell 22,6 Prozent und 2007 19,6 Prozent an. Großer Aufwand für die freie Veröffentlichung schätzen 2014 5,1 Prozent und in der SOFI-Studie 4,4 Prozent. Keine Antwort wussten in beiden Erhebungen rund 18,5 Prozent.

--- Todo: Grafik mit Vergleich bauen ---
