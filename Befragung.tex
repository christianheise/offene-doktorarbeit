\chapter{Quantitative Befragung: Öffnung von Wissenschaft und Forschung aus der Perspektive von Wissenschaftlern}

Ziel der Arbeit ist es, die erarbeiteten theoretischen Grundlagen, sowie die Ausprägungen von Open Access und Open Science vor dem Hintergrund von wissenschaftlicher Reputation und über die Grenzen einzelner Fachdisziplinen hinaus im Rahmen einer Umfrage zu püfen. Besondere Berücksichtigung findet dabei auch die Identifikation weiterer Treiber und Bremser für die Öffnung von wissenschaftlicher Informationen und Prozesse. Dafür werden die aus der theoretischen Betrachtung analysierten Konzepte Open Access und Open Science der Untersuchung zugeordnet. Der Einsatz des "Forschungsinstruments Fragebogen" gehörtz zu den am häufigsten eingesetzten Methoden in der Sozialforschung \cite{raab_2012_fragebogen}.

Die Befragung ist methodisch von der Ausrichtung mit einer Studie des Soziologischen Forschungsinstituts Göttingen (SOFI) "Wissenschaftliche Publikationen im Internet: Wissenschaftler als Leser und Autoren" aus dem Jahr 2007 vergleichbar. Die Studie untersuchte "die neue Möglichkeiten wissenschaftlichen Publizierens, die zunehmend als Alternative zu Fachzeitschriften und -verlagen diskutiert werden" und "zielte darauf ab, die Veränderungen aus der Perspektive von Wissenschaftlern als Autoren und Leser zu untersuchen"\cite{SOFI_Webseite}. 

Die zentralen Forschungsfragen dieser Arbeit stellten die Grundlage für die Entwicklung des Fragepools dar. Der Fragebogen war für die Erfassung konkreter Verhaltensweisen und allgemeine Zustände und Sachverhalte \cite{raab_2012_fragebogen} konstruiert. Die Formulierung der Fragen basierte, sofern nicht aus der Studie des SOFI unverändert übernommen, auf den in den vorhergehenden Kapieteln theoretisch erarbeiteten Handlungsmustern, Definitionen, Intentionen, Meinungen und Einstellungen zu folgenden Fragestellungen:
\begin{itemize}
\item Wie verändert die Digitalisierung, wie wir auf wissenschaftliche Daten und Informationen zugreifen?
\item In welchem Umfang besteht Wissen über die Öffnung von Wissenschaft unter den Wissenschaftlern und Wissenschaftlerinnen? 
\item Welches Verständnis von Open Access besteht unter den Befragten? 
\item Wie stark ist das Interesse an Forschungsdaten? 
\item Welche Faktoren und Argumente begünstigen die Öffnung von Wissenschaft in der jeweiligen wissenschaftlichen Disziplin, welche Argumente sprechen dagegen? 
\item Wie wird der geschätzte Aufwand für die Öffnung von Wissenschaft in einer wissenschaftlichen Disziplin eingeschätzt?
\item Welche unterschiedlichen Auffassungen bestehen zwischen den unterschiedlichen Fachdiziplinen, Alters- und Statusgruppen?
\item In welchem Umfang wird bereits heute im wissenschaftlichem Umfeld offen kommuniziert?
\item Welche Veränderungen beim Zugang zur Literatur wie auch bei den Veröffentlichungsstrategie sind im Vergleich zur der 2007 und 2008 durchgeführten Befragung des SOFI Göttingen zu erkennen?
\end{itemize}

\section{Erhebungsmethode und Messinstrumente}

Folgendee Überlegungen wurden bei der Auswahl der Erhebungsmethode angestellt: Persönliche Interviews erschienen wenig geeignet, da der damit verbundene personelle, zeitliche und finanzielle Aufwand als zu hoch eingestuft wurde. Gegen eine postalische Befragung sprachen die hohen Kosten (unter anderem Porto) sowie die häufig geringen Rücklaufquoten. Darüber hinaus wurde in der vorangegangenen Studie durch das SOFI Göttingen ebenfalls auf das Internet als primäre Quelle für die Identifikation von Teilnehmern und Teilnehmerinnen und E-Mail als Kontaktaufnahmekanal zurückgegriffen. Darüber hinaus hat die zunehmenden Verbreitung und Nutzung des Internets, hat die elektronische Online-Befragung längst Eingang in die empirische Sozialforschung gefunden \cite{Pannewitz_2002}. 

Auschlaggebend für die Auswahl dieser Befragungsform war der ökonomische Aspekt, da es die Online-Befragung "einfach macht, große Stichproben in kurzer Zeit zu erheben" \cite{eichhorn_2004_online} und diese Methode eine Beantwortung der Fragen jederzeit im Internet möglich macht. Gleichzeitig ermöglichte diese Form der Befragung, die einfache Verbreitung am Institut und unter Kolleginnen und Kollegen. In den Insturktionen des Fragebogens wurde aber auch die Option angeboten, den Fragebogen analog zu erhalten. Der direkte Download des Fragebogens wurde nicht angeboten um die Umfrageergebnisse nicht im Vorhinein zu beeinflussen. 

Weitere Vorteile bei der Methode der Online-Datenerhebung ist die unabhängige und einfache Teilnahme der Befragten, wobei davon ausgegangen wurde, dass die notwendigen technischen Voraussetzungen zur Teilnahme an einer Internetbefragung bei allen Wissenschaftlern an deutschsprachigen Wissenschaftseinrichtungen gegeben ist. Es kam nur zu drei expliziten Verweigerungen der Teilnahme an der Befragung: In einem Fall gab es Institutsbeschluss nicht mehr an Befragungen teilzunehmen. In einem weiteren Fall wurde die Methode der sozialwissenschaftlichen Befragung grundsätzlich abgelehnt und in dem dritten Fall mit dem Verweis auf zu hohen Aufwand für das Ausfüllen von Fragebogen beantwortet.

Die Anonymität der Befragen wurde jederzeit gewahrt und keine eideutigen persönlichen Daten erhoben, die einen Nutzer direkt identifizierbar machen. Obwohl die Ergebnisse nach Abschluss der Befragung anonymisiert veröffentlicht wurden keine Informationen und Ergebnisse veröffentlicht, die Rückschlüsse auf individuelle Teilnehmer an der Befragung zulassen. Darauf wurde auch im Pretest explizit eingegangen.

\subsection{Untersuchungsobjekte}

Die Teilnehmer des Fragebogens waren primär deutschsprachige Wissenschaftler und Wissenschaftlerinnen aus sämtlichen Fachdiziplinen oder Mitarbeiter des wissenschaftlichen Betriebs aus dem deutschsprachigen Raum die im Zeitraum vom 18.8.2014 bis 18.01.2015 online befragt wurden. Bibliothekare und Bibliotherinnen (0,95 Prozent der Befragten), sowie Studierende (3,68 Prozent Befragten) wurden zwar nicht direkt angesprochen, konnten aber dennoch an der Umfrage teilnehmen. Während der Befragung wurden dazu 4002 Wissenschaftlerinnen und Wissenschaftler per E-Mail im Zeitraum vom 18.08.2014 bis 18.01.2015 angeschrieben.

Die Auswahl der jeweiligen Fächer beruht auf der aktuellen Auflistung der Fachsystematik der Deutschen Forschungsgemeinschaft (DFG). Da die Erhebung fächerübergreifend angelegt war, um die Unterschiede zwischen den Disziplinen zu evaluieren, wurden Wissenschaftler aus alle dort gelisteten Fachdiziplinen angefragt. Per Zufall wurdnen dazu von den Institutswebseiten im deutschsprachigen Raum pro Fach 150 Wissenschaftler und Wissenschaftlerinnen per E-Mail angeschrieben und um Teilnahme an der Befragung gebeten. 1.768 der Angefragten haben an der Umfrage teilgenommen und den Fragebogen gestartet, 1.467 haben mindestens eine Frage beantwortet und somit zumindest teilweise an der Befragung teilgenommen. 301 Personen haben vor Beantwortung der ersten Fragegruppe abgebrochen. Die Rücklaufquote liegt somit bei 44,18 Prozent brutto beziehungsweise bei 36,67 Prozent netto. 1.112 der 1.768 Teilnehmer und Teilnehmerinnen (62,89 Prozent), die die Befragung gestartet haben, haben den Online Fragebogen vollständig beendet. Nach Beendigung des Umfragezeitraums haben somit 656 (37,10 Prozent) den Online-Fragebogen vor Beendigung abgebrochen. Die hohe Resonanz ist vermutlich auf die persönliche Ansprache sowie die konkrete Zurordnung zur Fachdisziplin im Anschreiben zurückzuführen. Dabei handelt es zwar um ein aufwendiges Vorgehen, hat aber sicherlich zu dieser guten Quote geführt. Die Fragebögen, deren Beantwortung vor Beendigung aller Fragen abgebrochen wurde, blieben unberücksichtigt. 

Um die Repräsentativität der Studie sicherzustellen wurden die Rückläufer der Befragung auf die vorhandenen Informationen wie die fachliche Zuordnung, Beruflicher Status und Alter ausgewertet. Verschiedenen Verzerrungen sind dahergehend zu vermute, da die kontaktierten Menschen ausschließlich online kontaktiert wurden. Da die Umfrage jedoch öffentlich online stattfand, konnte jeder teilnehmen.

\subsection{Untersuchungsmaterial}

Für die Online-Befragung wurde die Open Source Software LimeSurvey Version 2.05+ verwendet, die auf dem Webserver des Centre for Digital Cultures installiert wurde. Die Software ist weit verbreitet und ermöglicht eine umfassende Einstellungs und Anpassungsmöglichkeiten. So konnten Fragen in Abhängigkeit von Antworten auf vorherige Fragen kontextsensitiv definiert werden. Die beantworteten Fragebögen können aus der Verwaltungsoberfläche einzeln oder zusammengefasst eingesehen und in für die Auswertung exportiert werden. Die Darstellung der Befragung wurde so angepasst, dass die Darstellung und die Bewantwortung des Fragebogens auch auf Mobiltelefonen möglich war. Des Weiteren wurde bei dem Design des Fragebogens und der Anpassung der Dartstellung der Software explizit darauf geachtet, dass alle Texte einfach und angenehm lesbar waren und die Beantwortung der Fragen einfach und strukturiert ablaufen konnte.

Die Ergebnisse wurden in der Datenbank des Servers des Centres for Digital Cultures zwischengespeichert und am xx.xx.2015 gelöscht. Nach Abschluss der Befragung wurden die Datensätze anonymisiert. Dazu wurden sämtliche persönliche Daten, wie zum Beispiel E-Mailadressen entfernt und die freiwillige personenbezogene Angabe von dem Rest der Daten getrennt. Folgende Felder wurden entfernt beziehungsweise getrennt, neu angeordnet und aggregiert veröffentlicht: Geschlecht, Alter, weitere Aspekte zum Thema, Anmerkungen und Kritik, Funktion im Rahmen eines Open Access Engagements (optional), Antwort ID und Zeitpunkt der Beantwortung. Die anonymisierte Datensätze wurden nach Abschluss der Befragung im Januar 2015 auf dem Datenrepositorium zenodo.org, auf dem Forscher Daten und Publikationen einstellen können, veröffentlicht.

\subsection{Aufbau des Fragebogens und Untersuchungsdurchführung}

In der Befragung durch das SOFI wurden 6500 Wissenschaftler und Wissenschaftlerinnen befragt, von denen 1803 geantwortet haben. Der 2007 verwendete Fragebogen bestand aus 51 Fragen. Im ersten Teil des Fragebogens wurden Fragen zu dem Fachgebiet und Tätigkeitsbereich zunächst als Leserin bzw. Leser wissenschaftlicher Publikationen erfasst. Im zweiten Teil wurden die Teilnehmer aus der Perspektive als Autorin beziehungsweise als Autor befragt. Abschließend wurden noch eineige personenbezogene Angaben abgefragt. \cite{Hanekop_Wittke_2007_Fragebogen} Die Skalen und Antwortformate zur Beantwortung der Fragen waren unterschiedlich ausgewählt. 

Zu Beginn der Fragebogenkonstruktion wurde der Fragebogen und das Datenmaterial der Vorbefragung einer Itemanalyse zum Ausschluss unpassender Fragen (Items) unterzogen und Fragen bezglich der Fragestellung dieser Arbeit hinzugefügt. Dafür wurden die veröffentlichten Antworten analysiert. Fragen, die stark ungleich verteilt waren, wurden, wenn sie nicht inhaltlich interessant erschienen, ausgeschlossen oder mit anderen Fragen zusammengelegt. Somit wurden auf der Basis der Analyse der Fragen der Fragepool auf 32 Fragen reduziert beziehungsweise verändert. Acht der insgesamt 40 Fragen bedingen Antworten aus vorherigen Fragen und wurden deshalb nicht allen Teilnehmern gestellt. Die Reihenfolge der Fragen und der Fragengruppen wurden so ausgewählt, dass sie zueinander passen, der Reihenfolge-Effekt minimiert wird und die Beantwortung bis zum Ende interessant bleibt. Insgesamt wurden bei dem Aufbau des Fragebogens die Aufzählung der Richtlinien zur Formulierung der Items nach Bortz und Döring \cite{raab_2012_fragebogen} berücksichtigt. Da 75 Prozent der Befragten, die mindestens eine Frage beantwortet haben, auch den Fragebogen vollständig beantwortet haben, verdeutlicht den Erfolg der Vorbereitung und Anpassung der Daten. 

Die Qualität und Brauchbarkeit des Fragebogens wurde in einem Pretest (Probedurchlauf) mit wissenschafltichen Mitarbeitern und Mitarbeiterinnen überprüft. Dazu wurde der Fragebogen an 15 Wissenschaftler im Testmodus übermittelt und unter der Instruktion des "lauten Denkens" um Bearbeitung des Fragebogens gebeten \cite{raab_2012_fragebogen}. Nach dem Pretest wurden nach der initialen Konstruktion der Befragung weitere Fragen angepasst, die sich auf die Veröffentlichung von wissenschaftlichen Informationen und Daten beziehen.

Im finalen Fragebogen kamen die Antwortformate: Offene Fragen, geschlossene Fragen und Mischformen mit offenen und vorgegebenene Kategorien, sowie freie (offenen) Antwortformate zum Einsatz. Bezüglich der eingesetzten Ratingskalen wurden versucht größtenteils darauf zu verzichten. Insgesamt wurden in dem Fragebogen drei fünfstufige Ratingskalen mit verbaler Skalenbezeichnung eingesetzt. Die Charakterisierungen der Abstufungen wurde zur Vergleichbarkeit aus der Befragung 2007 übernommen.

Die Gliederung war ebenfalls an die Befragung aus den Jahren 2007 angelehnt und nur leicht ergänzt. Insgesamt wurden die 40 Fragen in 5 Fragegruppen plus eine abschließende Fragegruppe für persönliche Angaben und Anmerkungen und Kritik unterteilt. In der ersten Fragegruppe auf die Rahmenbedingungen der Teilnehmenden sowie deren wissenschaftlichen Tätigkeit. In der zweiten Fragegruppe wurden Aspekten aus der wissenschaftlichen Leserperspektive evaluiert. Die dritte Fragegruppe beschäftigte sich mit dem Zugang zu wissenschaftliche Informationen, gefolgt von der vierten, die aus Fragen bezüglich des Zugangs zu wissenschaftlichen Informationen und des Zugriffs auf wissenschaftliche Infromationen bestand. In der fünften Fragegruppe wurden Fragen aus der Perspektive des Autors und der Autorin von wissenschaftlichen Inhalten gestellt. Abschließend wurden weitere personenbezogene Daten zur eindeutigen Segmentierung erhoben. Die Befragten wurden vor Start der Befragung auf die Gleiderung des Fragebogens und die Reihenfolge der Fragegruppen, sowie die Rahmenbedingungen des Fragebogens, wie anonyme Behandlung und Veröffentlichung der Daten, hingewiesen.

Bevor mit der Online-Befragung begonnen werden konnte, wurde ein Pretest mit einigen Wissenschaftlern und Wissenschaftlerinnen durchgeführt. Die Einleitung für den Fragebogen, die Instruktionen und die Anrede wurden ebenfalls im Pretest evaluiert und optimiert, da sie sehr viel "zur Motivation der Bearbeitung beteitragen kann" \cite{raab_2012_fragebogen}. Nach der Auswertung und Einarbeitung der Anmerkungen der Pretester wurde nach Zufallsprinzip jeweils ca. 150 Wissenschaftler und Wissenschaftlerinnen je Fachdiziplinen aller DFG-Fachkollegien identifiziert. Die Namen und E-Mail-Adressen zu den Personen waren über die Internetseite der Hochschulen und wissenschaftlichen Organisationen öffentlich zugänglich. 

Der Fragebogen "Wissenschaftliche Kommunikation im Rahmen der Digitalisierung" wurde nach Vorbereitung am 18.08.2014 unter http://umfrage.offene-doktorarbeit.de veröffentlicht. Die Kontaktaufnahme zu den ausgewählten Personen erfolgte per personalisierter E-Mail mit einem Hinweistext, Instruktionen und einem direkten Link auf die Webadresse des Fragebogens. Vereinzelt wurden auch Sekretariatsadressen verwendet und um Weiterleitung gebeten. Alle identifizierten Kontakte wurden nur einmal angeschrieben. Zusätzlich wurde der Umfragelink mit einer kurzen Information zur Umfrage auf offene-doktorarbeit.de veröffentlicht, sowie über die privaten Social-Media Kanäle, an persönliche Kontakte des Autors versendet und über wissenschaftliche Mailinglisten, sowie den Newsletter des Centre for digital Cultures verbreitet. Nach Abschluss des Fragebogens hatten die Teilnehmer und Teilnehmerinnen die einfache Möglichkeit einen Link auf die Befragung  über die sozialen Kanäle und per E-Mail weiterzuverbreiten. Ausserdem befand sich dort ein Hinweis auf die Webseite meines Promotionsvorhabens.

\section{Kritische Betrachtung und Beurteilungsfehler}

Immer wieder kommt es bei dem Prozeß der Erstellung von Fragebögen oder bei der Beurteilung der Daten zu Störungen, zu sogenannten Beurteilungsfehlern. Deshalb soll die Güte der Befragung durch
die Gütekriterien Objektivität, Reliabilität und Validität beurteilt werden.

\subsection{Objektivität}

Die Unabhängigkeit beschreibt das Ausmaß, in dem das Ergebnis der Untersuchung frei und unabhängig von Einflüßen ausserhalb der befragten Person ist \cite{rost_2004_lehrbuch}. Die Objektivität der durchgeführten Befragung ist gegeben, da durch die elektronische Onlinebefragung eine zeitliche und räumliche Unabhängigkeit gewährleistet wurde. Die Befragung wurde für alle Teilnehmer nach identischer Anrede, Einladung und Instruktion und ohne Untersuchungsleiter durchgeführt und hing nicht von Situationsvariablen ab.

\subsection{Reliabilität}

"Ein Test als Messinstrument ist rliabel, wenn er das, was er mißt, genau misst." \cite{schelten_1997_testbeurteilung} Die Reliabilität gibt den Grad der Genauigkeit an, mit der durch die empirische Datenerhebung ein Merkmal erfasst wird \cite{rost_2004_lehrbuch}, unabhängig davon was er erfasst. Sie spiegelt die Replizierbarkeit von Messergebnissen und Zuverlässigkeit einer Datenerhebung wieder.

\subsection{Validität}
In der Literatur wird in zwei Typen der Validität unterschieden \cite{rost_2004_lehrbuch}. Die interne und die externe Validität. Von einer hohen internen Validität wird ausgegangen, wenn die erzielten Ergebnisse klar und eindeutig interpretierbar sind \cite{raab_2012_fragebogen}. Im Rahmen der durchgeführten Befragung zeigt die Validität, ob das Messinstrument Fragebogen wirklich das misst was dazu beiträgt, die Fragestellungen der Arbeit zu beantworten. Die Validität wurde durch die Übernahme der Grundstruktur und von Items der Studie "Wissenschaftliche Publikationen im Internet: Wissenschaftler als Leser und Autoren" des SOFI in Göttigen gewährleistet. Die Validität der neu erstellten, angepassten  und zusammengelegten Items wurde durch Auswertung des Pretests an der auch eine Wissenschaftlerin der Studie des SOFIs teilgenommen hat sichergestellt.

\section{Ergebnisse der Befragung}

Im Zeitraum vom 18. August 2014 bis 18. Januar 2015 haben 1.768 Personen die Befragung zur wissenschaftlichen Kommunikation im Rahmen meines Promotionsvorhabens gestartet. 1.446 Teilnehmer haben die Umfrage teilweise und 1.112 komplett abgeschlossen.

Die erhobenen Daten der 1.112 Teilnehmer des Online-Fragebogens werden mit Hilfe der computerunterstützten Datenaufbereitung statistisch ausgewertet. 

\subsection{Demographische Daten}

In die Auswertung der Befragung gingen die Angaben der 1.112 Teilnehmer des Online-Fragebogens ein, die alle Fragen beantwortet haben. Zu den erhobenen persöhnlichen Daten:

\begin{itemize}
\item \textbf{Geschlecht:} 39,93 Prozent der Befragten waren weiblich, 54,50 Prozent männlich und 62 Personen oder 5,58 Prozent haben keine Angabe zu Ihrem Geschlecht gemacht. 
\item \textbf{Alter:} Die prozentuale Verteilung des Alters: 4,4 Prozent (46) waren zum Zeitpunkt der Befragung jünger als 31 Jahre alt, die größte Altersgruppe mit 31,21 Prozent stellten die 31 bis 40 Jährigen, 17,36 Prozent der Befragten waren zwischen 41 und 50 Jahren alt, 14,57 waren älter als 50 Jahre. 0,72 Prozent machten bei der Frage nach ihrem Alter keine Angaben. 
\item \textbf{Berufstatus:} Unter den Befragten faben 24,82 Prozent an, Privatdozenten, Juniorprofessoren oder Professoren zu sein. 55,94 Prozent der Teilnehmer gaben an wissenschaftliche Mitarbeiter zu sein, 19,96 Prozent wissenschaftliche Mitarbeiter und Doktoranden, 22,84 Prozent promovierte wissenschaftliche Mitarbeiter und 13,13 Prozent Mitarbeiter ohne Promotionsvorhaben oder abgeschlossener Promotion. 10 Teilnehmer (0,90 Prozent) gaben an Wissenschaftler in der Privatwirtschaft zu sein. 35 Befragte (3,15 Prozent) wurden unter Sonstiges subsummiert.
\item \textbf{Tätigkeitsdauer in der Wissenschaft:} Nur 6,21 Prozent der Befragten gaben an "weniger als 1 Jahr" in der Wissenschaft tätig zu sein, 20,41 Prozent seit mehr als einem aber weniger als drei Jahre. Drei bis 6 Jahre gaben 24,01 Prozent an. 15,38 Prozent der Teilnehmer sind mehr als sechs aber weniger als 10 Jahre wissenschaftlich tätig. Die größte Gruppe gab an, "mehr als 10 Jahre" wissenschaftlich tätig zu sein. 1,53 Prozent waren "nicht in der Wissenschaft tätig" und 0,72 Prozent machten keine Angaben.
\item \textbf{Forschungseinrichtung:} Die große Mehrzahl der Teilnehmer (78,06 Prozent) gaben an einer deutsche Universität/Hochschule. Die zweitgrößte Gruppe stellten mit 5,31 Prozent die 59 Befragten, die an einem Institut der Leibniz-Gemeinschaft tätig sind. 4,77 Prozent gaben an an einer "Sonstigen" Einrichtung tätig zu sein. Nur 1,71 Prozent der Befragten waren an einem Max-Planck-Institut oder einem Institut der Fraunhofer Gesellschaft tätig. An einer Universität/Hochschule im deutschsprachigen Ausland waren 4,05 Prozent und im nicht nicht-deutschsprachigen Ausland 1,44 Prozent tätig. 1,26 Prozent arbeiteten an einer deutschen Fachhochschule tätig. 11 Befragte (0,99 Prozent) gaben an einem „An“-Institut (eigenständige Forschungseinrichtung, angegliedert an einer deutschen Hochschule) zu arbeiten.
\end{itemize}

\subsection{Veränderungen wissenschaftlicher Kommunikation durch die Digitalisierung}

\subsection{Verständnis von Offenheit}

16,91 Prozent der Befragten gaben an, sich häufig über Open-Access Repositorien (z.B. arxiv.org) auf dem Laufenden zu halten, 17,09 Prozent oder 190 der 1.112 Befragten nutzen Open-Access Portale (z.B. Directory of Open Access Journals) um sich über den aktuellen Stand der Forschung zu informieren. Bei der Suche nach Literatur nutzten nur 4,32 Pozent häufig über Suchmaschinen für Open Access, aber mehr als 50 Prozent in fachspezifische Suchportalen, die ebenfalls Open-Access Publikationen enthalten. Die Mehrheit der Befragten gab an Interesse am Zugang zu Forschungdaten anderer Wissenschaftler und Wissenschaftlerinnen zu haben (71,31 Prozent). Über ein Drittel dieser Teilnehmer und Teilnehmerinnen erklärten ihr genaues Interesse in einer optionalen Frage.

54,50 Prozent gaben an, sie finden "die Forderung nach kostenfreiem Zugang zu allen wissenschaftlichen Publikationen für Leser (Open Access)" sehr gut. Knapp unter einem Viertel (22,30 Prozent) finden die Forderung "gut". 19,24 Prozent waren sich bei der Frage unsicher und antworteten mit "teils/teils"und 38 der Befragen lehnten die Forderung nach Open Access ab, 9 davon sogar "entschieden". Nur 7 Teilnehmer und Teilnehmerinnen oder 0,63 Prozent gaben an Open Access nicht zu kennen. 14,75 Prozent sind in der Open Access Bewegung engagiert, wobei 72,30 Prozent die Aussage verneinten ein Engagement in der OA-Bewergung und 12,96 Prozent enthielten sich der Angabe. 43,44 Prozent oder 483 Personen kommentierten Ihre Meinung zu Open Access, wobei 72,88 Prozent der abgegebenen Kommentare von Befragten stammten, die Open Access gut oder sehr gut finden. Unter den Befragten, die Open Access ablehnen, kommentierten 75,68 Prozent ihre Haltung zur Forderung nach kostenfreiem Zugang zu allen wissenschaftlichen Publikationen. Von den der Personen, die "teils/teils" angaben, kommentierten fast die Hälfte (48,13 Prozent) ihre unsichere Haltung. 159 der 854 Befragten (18,62 Prozent), die Open Access gut oder sehr gut finden, gaben an, selbst aktiv in der Open Access Bewegung zu sein. Überraschend sind an dieser Stelle 12 Personen (5,61 Prozent) die "teils/teils" bezüglich ihrer Meinung zu der Forderung nach Open Access angegeben haben, aber sich dennoch zum Teil der Bewegung zählen.

In der folgenden Frage wurde das Verständnis von Open Access nach der Definition der Budapest Open Access Initative \cite{boai_2012} abgefragt. Knapp drei Viertel der Teilnehmer (74,91 Prozent) stimmten dieser Definition uneingeschränkt zu, 18,97 Prozent waren sich unsicher, 2,43 Prozent lehnten die Definition ab, 3,24 Prozent beantworteten die Frage mit "weiß nicht" und fünf Teilnehmer enthielten (0,45) sich der Aussage. Wurde "teils/teils" oder "weiß nicht" als Antwort ausgewählt konnte in einer optionalen Freitext-Frage beantwortet werden, welche Aspekten der Definition genau keine Zustimmung und welche Zustimmung fanden. Davon machten 37,65 Prozent der möglichen Befragten gebrauch.

Die Frage, ob sie die befragten Wissenschaftler vorstellen können ihre "Forschungsdaten und alle weiteren Informationen, die während der wissenschaftlichen Arbeit anfallen (z.B. Laborbücher, Entwürfe oder andere Dokumente und Daten) unter Berücksichtigung von Datenschutz öffentlich zur Verfügung zu stellen", beantworteten 27,97 Prozent uneingeschränkt mit "ja" und 36.33 Prozent schränkten ein, dass sie das "nur unter bestimmten Bedinungen" tun würden.  28,96 Prozent lehnten diese Frage ab und 6,74 Prozent wußten darauf keine Antwort. Die bedingte und freie Frage nach der Erläuterung "bestimmten Bedingungen" beantworteten 214 Teilnehmer und Teilnehmerinnen.

\subsection{Interesse daran Offenheit zu praktizieren}

Dem großen Verständnis von Open Access, der mehrheitlichen Unterstützung der Forderung nach Öffnung von Wissenschaft dem Interesse an Forschungsdaten anderer steht die Frage gegen über, wie wichtig den Wissenschaftlern das Kriterium freier Zugang zum Volltext bei der eigenen Veröffentlichung eines Beitrags ist. Die Mehreit der Befragten (49,82 Prozent) erachten dies als weniger wichtig oder nicht wichtig. Demgegenüber erachteten 44,70 Prozent das Kriterium "freier Zugang im Internet" als wichtig oder sehr wichtig. Keine Antwort auf diese Frage gaben 5,49 Prozent.

Diese Zahlen werden bei der weiteren Betrachtung der Kriterien, die Wissenschaftler bei der Veröffentlichung eines Beitrags als wichtig oder sehr wichtig erachten bestätigt. Während der fachlich einschlägige Schwerpunkt (91,19 Prozent), das Renommee der Zeitschrift/des Verlags (81,74 Prozent) und akzeptable oder keine Veröffentlichungskosten für Autoren (79,23) relativ am häufigsten als wichtig oder sehr wichtig erachtet wird, stellt die Veröffentlichung unter einer Open-Access Lizenz für nur 33,09 Prozent ein wichtiges oder sehr wichtiges Kriterien bei der Veröffentlichung eigener Inhalte dar. 56,39 Prozent erachten dieses Kriterium als weniger wichtig oder nicht wichtig. Der akzeptable Preis der Publikation spielt für 43,43 Prozent eine Rolle, für 50,00 Prozent der Befragen ist er weniger wichtig oder nicht wichtig.

Weitere Kriterien nach ihrer Wichtigkeit sortiert:
\begin{itemize}
\item 77,78 Prozent der Teilnehmer der Studie sehen die internationale Verbreitung als mindestens wichtige, wenn nicht sogar sehr wichtiges Kriterium im Rahmen der eigenen Veröffentlichungen an. 19,43 finden dieses weniger wichtig oder unwichtig.
\item Das Peer-Review Verfahren wird von 75,36 Prozent als wichtiges Kriterium erachtet. Nur 18,62 Prozent der Befragten sind gegensätzlicher Meinung.
\item 75,35 Prozent der befragten Personen sehen die Transparenz des Review-Prozesses als wichtig an, 17,90 Prozent nicht.
\item Eine leichte Auffindbarkeit der Publikation im Internet ist 71,23 Prozent wichtig. 25,00 Prozent ist das weniger bis nicht wichtig.
\item Eine rasche Publikation der eigenen Publikation wollen 67,99 Prozent. 28,69 Prozent sehen das anders.
\item Rankings, wie der Impact-Faktor der Zeitschrift, erachten 58,36 Prozent als wichtig und 35,07 Prozent als weniger wichtig oder unwichtig.
\item Die Reputation der Herausgeber ist 48,29 Prozent ein wichtiges Kriterium, für 46,49 Prozent eher unwichtig bis nicht wichtig.
\end{itemize}

--- Todo: Grafik bauen ----

\subsection{Treiber und Bremser für die Öffnung von Wissenschaft und Forschung}

In Kapitel --- TODO: definieren --- wurden Treiber und Bremser für die ÖFfnung von Wissenschaft und Forschung in der Literatur identifiziert und herausgearbeitet. Diese wurden im Rahmen der empririschen Erhebung abgefragt und geklärt werden, welche Faktoren und Argumente aus Sicht von Wissenschaftlern die Öffnung von Wissenschaft in der jeweiligen wissenschaftlichen Disziplin begünstigen und welche sie behindern sprechen. 

Die Verteilung der Antwortmöglichkeiten auf die Argumente für die Öffnung der Wissenschaft und Forschung seitens der Befragten waren:
\begin{enumerate}
\item Die Beschleunigung der Wissensverbreitung und -verwertung, 721 mal wurde diese Antwortmöglichkeit ausgewählt.
\item Gefolgt von dem Argument der "Eröffnung neuer Möglichkeiten", das 63,76 Prozent der 1.112 Befragten unter den Antwortmöglichkeiten auswählten.
\item Dem Argument einer besseren Verfügbarkeit für jeden, folgte 55,22 Prozent.
\item Eine Erleichterung der wissenschaftliche Kommunikation, sahen 49,19 Prozent als Argument für die Öffnung der wissenschaftlichen Kommunikation an.
\item Die Förderung des interdisziplinären Austausch von Wissenschaftlern und Wissenschaftlerinnen erachteten 45,05 Prozent als Argument an. 
\item 43.97 Prozent oder 489 der Befragten sehen in der Überwindung sozialer, nationaler und globaler Wissenskluften ein Argument.
\item Die Chance einer umfassendere und transparentere Qualitätsmessung von Wissenschaft sahen 33,72 Prozent der befragten Wissenschaftler.
\item 250 oder 22.48 Prozent der Befragten sahen in der nachhaltigen und unabhängigen Archivierung der Informationen ein Argument für die Öffnung von Wissenschaft und Forschung.
\item 19,96 Prozent Ermöglicht indirekte Wirtschaftsförderung durch freien und offenen Wissenstransfer
\item Die Möglichkeit der Beilegung der vorherrschenden Zeitschriften- und Monographienkrise erachteten 16.28 Prozent.
\end{enumerate}

Bei dieser Frage gaben 8,27 Prozent der 1.112 Befragten gaben an, dass ihrer Meinung nach kein Argument für die Öffnung der wissenschaftlichen Kommunikation und aller Informationen aus dem Forschungs-/Arbeitsprozess spricht. Weitere 47 oder 4,23 Prozent machen weitere Angaben unter "Sonstiges". Nach Diziplin sortiert zeigte sich ein differenzierteres Bild  ---- Todo: weiter ausarbeiten v.a. Sonstige aufteilen und Grafik bauen ----

Bei den Argumenten gegen die Öffnung der wissenschaftlichen Kommunikation zeichnete sich ein uneindeutigeres Bild:
\begin{enumerate}
\item Als stärkstes Argument (43.35 Prozent) wurde von den Befragten die fehlenden Reputationskriterien für die Bewertung von offener Wissenschaft genannt.   
\item Knapp dahinter (40.20 Prozent) wurde die "Gefahr der Fehlinterpretation und Falschinformation durch Wissenschaft" ausgewählt.
\item Einen erhöhter zeitlicher Mehraufwand für die Bereitstellung der wissenschaftlichen Publikationen und/oder Forschungsdaten sahen 379 oder 34.08 Prozent als Hürde an.
\item 29,95 Prozent sehen eine Erschwerung der eindeutigen Zuordnung von Texten, Arbeiten und Daten zu den jeweilgen als Argument gegen die Öffnung wissenschaftlicher Kommunikation.
\item  An fünfthäufigster Stelle wurde dem Argument zugestimmt, dass Qualität der wissenschaftlichen Arbeit leidet (27,25 Prozent).
\item Mit dem Argument, dass die Langzeitarchivierung und langfristige Auffindbarkeit nicht (dezentral) gewährleistet werden kann, stimmten 26.35 Prozent zu.
\item Das hohe Kosten und keine Refinanzierung ein Argument gegen Offnung im wissenschaftlichen Kommunikationsprozess, unterstützten 274 der 1112 Befragten (24.64 Prozent).
\item 8,99 Prozent der Befragten sah in der Öffnung der Kommunikation eine Bedrohung der Publikations- und Pressefreiheit.
\item Dass die offene Bereitstellung von Daten keinen nachhaltigen Mehrwert bietet, dem stimmen 8,72 Prozent zu und sehen darin ein Argument gegen eine Öffnung des Systems.
\item Dass Offenheit und Transparenz bei Forschungsförderung Freiheit von Wissenschaft und Forschung gefährden sehen 5,40 Prozent als Argument.  
\end{enumerate}

154 Teilnehmer und Teilnehmerinnen nannten sonstige Argumente (13.85 Prozent). 7.73 Prozent sind der Meinung dass kein Argument gegen die Öffnung der wissenschaftlichen Kommunikation und aller Informationen aus dem Forschungs-/Arbeitsprozess spricht.

---- Todo: weiter ausarbeiten v.a. Sonstige aufteilen und Grafik bauen ----

\subsection{Aufwand für die Öffnung von Wissenschaft}

\subsection{Auffassungen zwischen den unterschiedlichen Fachdiziplinen, Alters- und Statusgruppen?}

\subsection{Veränderungen im Vergleich zur SOFI Studie}

Die Vergleich bezüglich der Samples in Bezug auf Beruflicher Status, Alter und Geschlecht der befragten WissenschaftlerInnen ist ein Indikator für die Vergleichbarkeit der Samples. Bei beiden Erhebungen ist die Verteilung bezüglich des beruflichen Status im wissenschaftlichen System vergleichbar.
--- Todo: Grafik bauen --- 
Auch der Vergleich der Altersgruppen beider Studien zeigt klare Überschneidungen bei den Befragten.  --- Todo: Grafik bauen --- 

Wie in der Befragung vom SOFI 2008 wurden die Teilnehmer und Teilnehmerinnen gefragt, wie sie sich in ihrem Fachgebiet auf dem Laufendenhalten und welchen Zugang sie zuden (Voll-) Texten haben \cite{hanekop_2008}. Auch die Antworten der 1.446 der Befragten zeigen "wie weitreichend sich bei der gezielten Suche nach Literatur digitale Suchmöglichkeiten durchgesetzt haben" \cite{hanekop_2008}. Fast 50 Prozent der Befragten in 2014 gaben an, die Google Suche häufig als Suchmöglichkeiten zu nutzen, um gezielt nach Literatur zu suchen. Bei der Befragung 2007 gaben 46 Prozent der Befragten an sehr häufig die Google Suche zu verwednen. Diese Entwicklung liegt im Trend, denn die IT-gestütze Suche lag in den 1980er Jahren bei einem Prozent, stieg bis 1993 auf neun Prozent an und betrug im Jahr 2003 bereits 24 Prozent \cite{hanekop_2008}.

Ein ähnliches Bild zeigt sich bei dem Vergleich der Umfrageergebnisse bei der Frage wie sich die Teilnehmer in Ihrem Fachgebiet auf dem Laufenden halten. 2007 gaben 57 Prozent an, sich sehr häufig über Online-Zeitschriften auf dem aktuellen Stand der wissenschaftlichen Debatte zu halten. In der Befragung im Rahmen dieser Arbeit gaben 67,71 Prozent der 1.446 Befragten an sich in Online-Zeitrschriften zu informieren. Gefolgt wird diese Option durch die Teilnahme an Tagungen oder Kongressen (56,05 Prozent) und Gespräche mit Fachkollegen (55.09 Prozent). Social Media Platformen spielen mit bisher knapp 6 Prozent eher eine kleinere Rolle. Online-Datenbanken, Online-Archive, die 2007 noch zweithäufigste Option sich auf dem Laufenden zu halten, bleibt annähernd auf dem gleichen Niveau.

Im Jahr 2007 fanden rund 81 Prozent der Befragten die Forderung nach kostenfreiem Zugang zu allen wissenschaftlichen Publikationen für Leser gut bis sehr gut. In der Befragung 2014 fiel das Ergebnis mit einer Befragtenzahl von 1.112 mit 76,80 Prozent zwar niedriger aber dennoch weiterhin überwältigend positiv aus.

Ein weiteres Ergbnis der Studie des Soziologische Forschungsinstituts Göttingen im Jahr 2007, dass "gerade auch die etablierten und damit etwas älteren Wissenschaftler nutzen internetbasierte Plattformen intensiv". In der aktuellen Befragung gaben 86,42 Prozent der über 50-jährigen Befragten an, sich mit "Online-Ausgaben von Zeitschriften" "häufig auf dem Laufenden zu halten". In dieser Altersgruppe greifen jedoch auch noch 50 Prozent zu "Print-Ausgaben von Zeitschriften". In der Altersgruppe unter 50 Jahren nutzen es gerade mal 28.98 Prozent die Print-Ausgaben von Zeitschriften. Print-Bücher hingegen finden altersgruppenunabhängig bei rund 52,76 bis 53.70 Prozent Verwendung bei den Befragten. Bei digitalen Büchern sind es bei den über 50 Jährigen 19,75 Prozent und bei den unter 50 Jährigen mit 38.75 Prozent fast doppelt so häufig das Mittel der Wahl um sich in dem jeweilgien Fachgebiet auf dem Laufenden zu halten.

In der Studie 2007 gaben insgesamt 80 Prozent an sich mit Onlineausgaben auf dem Laufenden zu halten. Sieben Jahre später stieg die Nutzung nochmals um über 8 Prozent auf 88.40 Prozent an. Die Situationen in denen die Befragten nicht Online-Version eines Aufsatzes zugreifen können, weil es keine Lizenz gibt wurde ebenfalls seltener. Gaben 2007 noch 45 Prozent an, haufig bis sehr häufig nicht auf Aufsätze und Texte online zugreifen zu können, waren es in der aktuellen Befragung nur noch 32,37 Prozent. 66,82 Prozent der teilnehmenden Wissenschaftler gaben an nur gelegentlich bis nie Probleme mit dem Zugang zu Onlinetexten zu haben. In der Befragung 2007 waren es nur 52 Prozent.
