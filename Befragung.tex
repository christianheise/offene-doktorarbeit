\chapter{Befragung: Öffnung von Wissenschaft aus der Perspektive von Wissenschaftlern}

Ein weiteres Bestreben der Arbeit ist es, die herausgearbeiteten theoretischen Grundlagen, die Ausprägungen von Open Access und Open Science sowie die im vorherigen Kapitel erabeiteten Herausforderungen im Aktuellen System wissenschaftlicher Kommunikation sowie Katalysatoren und Hindernisse für die Verbreitung der Öffnung wissenschaftlicher Kommunikation vor dem Hintergrund wissenschaftlicher Reputation im Rahmen einer Umfrage zu überprüfen.

Besondere Berücksichtigung findet dabei das die Identifikation weiterer Treibern, Bremser und Anreizen für die Öffnung von wissenschaftlicher Informationen und Prozesse. Dafür werden die aus der theoretischen Betrachtung analysierten Konzepte Open Access und Open Science einer Befragung von Wissenschaftlern und Wissenschaftlerinnen zugeordnet. Abschließend werden die Ergebnisse der Befragung mit der im Jahr 2007 zwischen Juli und November durchgeführten Erhebung "Wissenschaftliche Publikationen im Internet: Wissenschaftler als Leser und Autoren" durch das Soziologische Forschungsinstituts Göttingen (SOFI) \cite{hanekop_2008} verglichen.

Um die genannten Aspekte mit einer möglichst großen Stichprobe zu konsolidieren, wurde die Online-Befragung als Methode gewählt. Die Befragung richtete sich dabei ausschließlich an deutschsprachige Wissenschaftler und Wissenschaftlerinnen in unterschiedlichen Karrierestufen und Fachdisziplinen sowie an Personen im wissenschaftlichen Umfeld, die mit den Eigenheiten des wissenschaftlichen Kommunikationssystem vertraut sind.

Der Fragebogen wurde für die Erfassung konkreter Verhaltensweisen und allgemeine Zustände und Sachverhalte \cite{raab_2012_fragebogen} konstruiert und die zentralen Forschungsfragen dieser Arbeit stellten die Grundlage für die Entwicklung des Fragepools dar. Die Formulierung der Fragen basierten, sofern nicht aus der Studie des SOFI unverändert übernommen, auf den in den vorhergehenden Kapiteln erarbeiteten Handlungsmustern, Definitionsversuchen, Intentionen, Meinungen und Einstellungen zu folgenden detaillierte Fragestellungen:
\begin{itemize}
\item Wie verändert die Einfluss der Digitalisierung das wissenschaftliche Kommunikationssystem?
\item In welchem Umfang besteht Interesse an der Öffnung von Wissenschaft und Wissen über Open Access unter den Wissenschaftlern und Wissenschaftlerinnen?
\item Welches Verständnis von Open Access besteht unter den Befragten?
\item Wie stark ist das Interesse an Forschungsdaten (anderer) ausgeprägt?
\item Wie hoch ist die Diskrepanz zwischen der Idee der Öffnung von wissenschaftlicher Kommunikation und der wissenschaftliche Realität?
\item Welche Faktoren und Argumente begünstigen die Öffnung von Wissenschaft in der jeweiligen wissenschaftlichen Disziplin, welche Argumente sprechen dagegen?
\item Wie wird der Aufwand für die Öffnung von Wissenschaft in einer wissenschaftlichen Disziplin eingeschätzt?
\item Welche unterschiedlichen Auffassungen bezüglich wissenschaftlicher Kommunikation bestehen zwischen den unterschiedlichen Fachdisziplinen, Alters- und Statusgruppen?
\item In welchem Umfang wird bereits heute im wissenschaftlichem Umfeld offen kommuniziert?
\item Welche Veränderungen beim Zugang zur Literatur wie auch bei den Veröffentlichungsstrategie sind im Vergleich zur der 2007 und 2008 durchgeführten Befragung des SOFI Göttingen zu erkennen?
\end{itemize}

\section{Erhebungsmethode und Messinstrumente}

Die Auswahl der Erhebungsmethode basierte auf folgenden Überlegungen: Persönliche Interviews und ein rein qualitatives Vorgehen erschienen wenig geeignet, da der damit verbundene personelle, zeitliche und finanzielle Aufwand als zu hoch eingestuft wurde. Gegen eine postalische Befragung sprachen die hohen Kosten (unter anderem Porto), der hohe zeitliche Aufwand sowie die häufig geringen Rücklaufquoten \cite{suchen}. Darüber hinaus haben "digitale Aufzeichnungen eine deutlich höhere Qualität", "digitale Daten lassen sich komfortabler und effizienter bearbeiten" und die "Darstellungsmöglichkeiten ermöglichen eine vertiefte Wahrnehmung sozialer Interaktionen" \cite{Hartung_2011_digitalisierung}.

Auschlaggebend für die Auswahl der Onlinebefragung als Befragungsform war auch, dass das Forschungsinstrument "Fragebogen" zu den am häufigsten eingesetzten Methoden in der Sozialforschung gehört \cite{raab_2012_fragebogen} und durch die zunehmenden Verbreitung und Nutzung des Internets, die elektronische Online-Befragung längst Eingang in die empirische Sozialforschung gefunden hat \cite{Pannewitz_2002}. Darüber hinaus begründete sich die Auswahl in dem ökonomischen Aspekt, das die Online-Befragung besonders geeignet ist, um "große Stichproben in kurzer Zeit zu erheben" \cite{eichhorn_2004_online} und diese Erhebungsmethode eine Beantwortung der Fragen durch die Teilnehmer und Teilnehmerinnen zu jeder Zeit ermöglicht. Auch dass die Vergleichsstudie durch das SOFI Göttingen ebenfalls auf das Internet als primäre Quelle für die Identifikation von Teilnehmern und Teilnehmerinnen und E-Mail als Kontaktaufnahmekanal zurückgegriffen hat, spielte eine Rolle bei der Wahl der Erhebungsmethode. Hilfreich war weiterhin, dass diese Form der Befragung, die einfache Verbreitung am Zentrum für digitale Kulturen (Centre for Digital Cultures) der Leupahana Universität und unter Kolleginnen und Kollegen ermöglichte.

Ein weiterer Vorteil bei der Methode der Online-Datenerhebung ist die unabhängige und einfache Teilnahme der Befragten. Die Unabhängigkeit wurde vor allem dadurch gewährleistet, das die Befragungssituation für alle Teilnehmer und Teilnehmerinnen gleich war. Es wurde auch davon ausgegangen, dass die notwendigen technischen Voraussetzungen zur Teilnahme an einer Internetbefragung (Internetzugang und internetfähiges Endgerät) bei allen Wissenschaftlern an deutschsprachigen Wissenschaftseinrichtungen gegeben sind. Es kam nur zu drei expliziten Verweigerungen der Teilnahme an der Befragung: In einem Fall gab es einen Institutsbeschluss nicht mehr an Befragungen teilzunehmen. In einem weiteren Fall wurde die Methode der sozialwissenschaftlichen Befragung grundsätzlich abgelehnt und in dem dritten Fall mit dem Verweis auf zu hohen Aufwand für das Ausfüllen des Fragebogens beantwortet.

Die Anonymität der Befragten wurde jederzeit gewahrt und keine eindeutigen persönlichen Daten erhoben, die einen Nutzer oder eine Nutzerin direkt identifizierbar gemacht hätten. Auf Grund der geplanten Veröffentlichung der Rohdaten und Ergebnisse unmittelbar nach Abschluss der Befragung, wurde von Beginn an darauf geachtet, dass zu keinem Zeitpunkt Rückschlüsse auf individuelle Teilnehmer oder Teilnehmerinnen an der Befragung möglich sind. So sollte der Aufwand für der Anonymisierung so gering wie möglich gehalten werden.

\subsection{Untersuchungsobjekte}

Die Teilnehmer des Fragebogens waren primär deutschsprachige Wissenschaftler und Wissenschaftlerinnen aus verschiedenen Fachdisziplinen oder Mitarbeiter des wissenschaftlichen Betriebs aus dem deutschsprachigen Raum. Sie wurden im Zeitraum vom 18.8.2014 bis 18.01.2015 online befragt. Bibliothekare und Bibliothekarinnen (1 Prozent der Befragten) und Studierende (4 Prozent Befragten) wurden zwar nicht direkt angesprochen, waren aber dennoch Willkommen an der Umfrage teilzunehmen. Im Rahmen der Befragung sind insgesamt 4.002 Wissenschaftlerinnen und Wissenschaftler per E-Mail im Zeitraum vom 18.08.2014 bis 18.01.2015 angeschrieben worden.

Die Auswahl der jeweiligen Fachdisziplinen beruht auf der aktuellen Auflistung der Fachsystematik der Deutschen Forschungsgemeinschaft (DFG) \cite{suchen_Webseite_DFG}. Da die Erhebung fächerübergreifend angelegt war, um die Unterschiede zwischen den Disziplinen zu evaluieren, wurden Vertreter und Vertreterinnen aus allen gelisteten Fachdisziplinen für die Teilnahme angefragt. Im Zufallsprinzip wurden dazu von den Institutswebseiten im deutschsprachigen Raum pro Fach 150 Wissenschaftler und Wissenschaftlerinnen per E-Mail angeschrieben und um Teilnahme an der Befragung gebeten. 1.768 der Angefragten haben an der Umfrage teilgenommen und den Fragebogen gestartet, 1.467 Teilnehmer und Teilnehmerinnen haben mindestens eine Frage beantwortet und somit teilweise an der Befragung teilgenommen. 301 Personen haben vor Beantwortung der ersten Fragegruppe abgebrochen. Die Rücklaufquote liegt somit bei 44 Prozent brutto beziehungsweise bei 37 Prozent netto. 1.112 von den 1.768 Teilnehmer und Teilnehmerinnen (63 Prozent), die die Befragung gestartet haben, haben den Online Fragebogen vollständig beendet. Die übrigen 656 Personen (37 Prozent) haben den Online-Fragebogen vor der Beantwortung aller Fragen abgebrochen.

Die hohe Resonanz ist vermutlich auf die persönliche Ansprache sowie die konkrete Zuordnung zur Fachdisziplin im Anschreiben zurückzuführen. Dabei handelt es sich zwar um ein aufwendiges, aber effizientes Vorgehen. Die angefangenen Fragebögen, die vor Beantwortung aller Fragen abgebrochen worden sind, bleiben in der weiteren Betrachtung unberücksichtigt.

\subsection{Untersuchungsmaterial}

Für die Durchführung der Online-Befragung wurde die Open Source Software LimeSurvey Version 2.05+ verwendet, die auf einem Webserver (Apache 2.2, PHP 5.5, MySQL 5.5) des Centre for Digital Cultures durch den Autor installiert worden war. Diese Software ist weit verbreitet und ermöglicht umfassende Einstellungs- und Anpassungsmöglichkeiten. So konnte zum Beispiel ein Teil der Fragen in Abhängigkeit von den Antworten auf vorherige Fragen kontextsensitiv definiert werden. Die Software ermöglichte es, die beantworteten Fragebögen aus der Verwaltungsoberfläche einzeln oder zusammengefasst einzusehen und für die Auswertung zu exportieren. Neben den üblichen Möglichkeiten zur Durchführung von Befragungen an internetfähigen Endgeräten wurde die Darstellung der Befragung darüber hinaus so angepasst, dass die Darstellung und die Beantwortung des Fragebogens auch auf internetfähigen Mobiltelefonen möglich war. Bei dem Design des Fragebogens und der Anpassung der Darstellung der Software wurde darüber hinaus explizit darauf geachtet, dass alle Texte einfach und angenehm lesbar waren, damit die Beantwortung der Fragen einfach und strukturiert ablaufen konnte.

Die Ergebnisse wurden in der Datenbank des Servers des Centres for Digital Cultures zwischengespeichert und am 10.08.2015 gelöscht. Nach Abschluss der Befragung sind die Datensätze anonymisiert worden. Dazu wurden sämtliche persönliche Daten, wie zum Beispiel in Freitextfeldern genannte E-Mailadressen entfernt und die freiwilligen personenbezogenen Angaben von dem Rest der Daten getrennt und neu sortiert. Folgende Felder wurden getrennt, neu angeordnet und unabhängig von den anderen Erhebungen veröffentlicht: Geschlecht, Alter, weitere Aspekte zum Thema, Anmerkungen und Kritik, Funktion im Rahmen eines Open Access Engagements, Antwort ID und Zeitpunkt der Beantwortung. Die anonymisierte Datensätze wurden nach Abschluss der Befragung im Januar 2015 auf dem datorium-Datenrepositorium des GESIS - Leibniz-Institut für Sozialwissenschaften veröffentlicht \cite{heise_2015_os_data_gesis}. Die Forschungsdaten durchliefen vor der Veröffentlichung ein durch GESIS durchgeführtes Review. Eine weitere Veröffentlichung der Daten erfolgte auf dem Datenrepositorium Zenodo \cite{heise_2015_os_data_zenodo}.

\subsection{Aufbau des Fragebogens}

Für die Befragung durch das SOFI im Jahr 2007 sind 6.500 Wissenschaftler und Wissenschaftlerinnen angefragt worden, von denen 1.803 mindestens teilweise geantwortet haben. Der 2007 verwendete Fragebogen bestand aus 51 Fragen  \cite{Hanekop_Wittke_2007_Fragebogen}. Im ersten Teil des Fragebogens wurden den Befragten Fragen zu Fachgebiet und Tätigkeitsbereich aus der Perspektive des Leserin und Leser wissenschaftlicher Publikationen gestellt. Im zweiten Teil wurden die Teilnehmer und Teilnehmerinnen aus der Perspektive der Autorin beziehungsweise des Autors wissenschaftlicher Beiträge befragt. Abschließend wurden noch einige personenbezogene Angaben erhoben \cite{Hanekop_Wittke_2007_Fragebogen}.

Zu Beginn der Fragebogenkonstruktion für die Befragung im Rahmen der vorliegenden Arbeit wurden der Fragebogen und das Datenmaterial der Befragung durch das SOFI in 2007 einer Itemanalyse zum Ausschluss unpassender Fragen (Items) unterzogen und Fragen in Zusammenhang mit den Fragestellungen dieser Arbeit hinzugefügt. Dafür wurden die veröffentlichten Antworten der Befragung durch das SOFI analysiert \cite{Hanekop_Wittke_2007_Fragebogen} und Fragen, die stark ungleich verteilt waren, wurden, wenn sie nicht inhaltlich interessant erschienen, ausgeschlossen oder mit anderen Fragen zusammengelegt. Dadurch wurden auf der Basis der Analyse der Fragen der Fragepool auf 40 Fragen reduziert beziehungsweise modifiziert. Acht der insgesamt 40 Fragen standen in Abhängigkeit von der Beantwortung vorhergehender Fragen und wurden deshalb nicht allen Teilnehmern und Teilnehmerinnen gestellt. Die Reihenfolge der Fragen und der Fragengruppen wurde so gewählt, dass sie strukturiert abgebildet werden konnten, der Reihenfolge-Effekt minimiert wurde und die Beantwortung bis zum Ende interessant blieb. Beim Aufbau des Fragebogens wurden die Aufzählung der Richtlinien zur Formulierung der Items nach Bortz und Döring \cite{raab_2012_fragebogen} berücksichtigt.

Die Qualität und Brauchbarkeit des Fragebogens wurde in einem Pretest (Probedurchlauf) mit wissenschaftlichen Mitarbeitern und Mitarbeiterinnen aus dem Arbeitsumfeld des Autors überprüft. Die Einleitung für den Fragebogen, die Instruktionen und die Anrede wurden ebenfalls im Pretest evaluiert und optimiert, da sie sehr viel "zur Motivation der Bearbeitung beitragen kann" \cite{raab_2012_fragebogen}. Dazu wurde der Fragebogen an 15 Wissenschaftler im Testmodus übermittelt und unter der Instruktion des "lauten Denkens" um Bearbeitung des Fragebogens gebeten \cite{raab_2012_fragebogen}. Nach dem Pretest ist der Fragenpool um weitere Fragen, die sich auch auf die Veröffentlichung von wissenschaftlichen Informationen und Daten beziehen, ergänzt worden.

Im finalen Fragebogen kommen die Antwortformate offene Fragen, geschlossene Fragen und Mischformen mit offenen und vorgegebenen Kategorien, sowie freie (offene) Antwortformate zum Einsatz. Es wurde versucht weitestgehend auf Ratingskalen zu verzichten. Insgesamt wurden in dem Fragebogen drei fünfstufige Ratingskalen mit verbaler Skalenbezeichnung eingesetzt. Die Charakterisierungen der Abstufungen wurde aus Gründen der Vergleichbarkeit aus der SOFI-Befragung von 2007 übernommen.

Die Gliederung des Fragebogens war ebenfalls an die Befragung aus den Jahren 2007 angelehnt und lediglich durch die Besonderheiten in Bezug auf die Veröffentlichung und Nutzung von Forschungsdaten ergänzt. Insgesamt wurden die 40 Fragen in 6 Fragegruppen und eine abschließende Fragegruppe für persönliche Angaben sowie Anmerkungen und Kritik unterteilt.

\begin{enumerate}
\item In der ersten Fragegruppe wurde auf die Rahmenbedingungen der Teilnehmenden sowie auf deren wissenschaftliche Tätigkeit eingegangen - es wurden Fachdisziplinen, Position und Arbeitsbereiche, sowie Forschungsrichtung abgefragt.
\item In der zweiten Fragegruppe wurden Aspekte aus der wissenschaftlichen Leserperspektive evaluiert - wie etwa Publikationsformen in der jeweiligen Fachdisziplin, Informationsverhalten, Suchmöglichkeiten und Zugriffmöglichkeiten auf wissenschaftliche Publikationen.
\item Die dritte Fragegruppe beschäftigte sich mit dem Zugang zu wissenschaftliche Informationen. In dieser Fragegruppe konnten die Befragten ihre Zugangsmöglichkeiten zur wissenschaftlichen Informationen beurteilen, ihr Interesse an Zugang zu Forschungsdaten angeben und die Barrieren beim Zugriff nennen. Darüber hinaus wurde die Nutzung der Möglichkeiten zur Auflistung der eigenen Publikation und Hürden beim Veröffentlichen der eigenen Volltexte gefragt.
\item Die vierte bestand aus Fragen zum Zugang zu wissenschaftlichen Informationen und zum Zugriff auf wissenschaftliche Kommunikation - insbesondere wurde die Einstellung zu Open Access, eine Beispieldefinition, das Interesse an der Veröffentlichung der eigenen Forschungsdaten, sowie die aus der Literatur erarbeitete Liste von Argumenten für und gegen die Öffnung der eigenen wissenschaftlichen Kommunikation abgefragt.
\item In der fünften Fragegruppe wurden Fragen aus der Perspektive des Autors oder der Autorin von wissenschaftlichen Inhalten gestellt - wie die Kriterien für die Auswahl des Veröffentlichungsortes, der wissenschaftlichen Reputation, zum Publikationsdruck, zur Publikationsaktivität zum Aufwand der freien Veröffentlichung von Texten und Daten.
\item Abschließend folgte die Erhebung weiterer freiwilliger personenbezogener Daten - wie Alter, Zeitraum der Forschungstätigkeit und Anmerkungen zum Fragebogen sowie zum Thema der Befragung - als Grundlage für die Möglichkeit der späteren Segmentierung der Teilnehmer und Teilnehmerinnen.
\end{enumerate}

Die Befragten wurden vor Beginn der Befragung auf die Gliederung des Fragebogens und die Reihenfolge der Fragegruppen, sowie die Bedingungen des Fragebogens, wie die anonyme Behandlung der Daten, hingewiesen. Dass 75 Prozent der Befragten, die mindestens eine Frage beantwortet haben, auch den gesamten Fragebogen vollständig beantwortet haben, verdeutlicht den Erfolg der Vorbereitung.

\section{Untersuchungsdurchführung}

Nach der Auswertung und Einarbeitung der Anmerkungen der Pretester wurde der Fragebogen "Wissenschaftliche Kommunikation im Rahmen der Digitalisierung" am 18. August 2014 unter der Internetadresse http://umfrage.offene-doktorarbeit.de veröffentlicht. Nach der Veröffentlichung wurden nach Zufallsprinzip jeweils im Durchschnitt 150 Wissenschaftler und Wissenschaftlerinnen jeder Fachdisziplin der DFG-Fachkollegien \cite{DFG_2014} identifiziert. Die Namen und E-Mail-Adressen zu den Personen waren über die Internetseiten der Hochschulen und wissenschaftlichen Organisationen öffentlich zugänglich.

Die Kontaktaufnahme zu den ausgewählten Personen erfolgte über eine personalisierte E-Mail mit einem Hinweistext, Instruktionen und einem direkten Link auf die Internetadresse des Fragebogens als klickbarer Link in der E-Mail. Vereinzelt wurden auch Sekretariatsadressen von Forschungsinstitutionen verwendet mit Bitte um die Weiterleitung der Einladung zur Befragung an die Wissenschaftler und Wissenschaftlerinnen innerhalb der jeweiligen Organisation. Alle identifizierten Kontakte wurden ausschließlich einmal kontaktiert. Eine Liste der kontaktierten Adressen wurde aus datenschutzrechtlichen Gründen nicht veröffentlicht.

Zusätzlich wurde der Umfrage-Link mit einer kurzen Information zur Umfrage auf http://offene-doktorarbeit.de veröffentlicht, über die privaten Social-Media Kanäle des Autors verbreitet und an persönliche Kontakte des Autors verschickt. Des Weiteren wurde eine generalisierte Version der Einladung zur Umfrage über wissenschaftliche Mailinglisten, sowie den Newsletter des Centre for Digital Cultures verbreitet. Um eine Möglichst große Streuung der Umfrage zu erzielen, hatten die Teilnehmer und Teilnehmerinnen nach Abschluss des Fragebogens zusätzlich die Möglichkeit den Link zu der Befragung über soziale Kanäle und per E-Mail selbst weiterzuverbreiten.

\section{Kritische Betrachtung der Vorgehensweise}

Da es bei dem Prozess der Erstellung von Fragebögen oder bei der Beurteilung der erhobenen Daten immer wieder zu Störungen, den sogenannten Beurteilungsfehlern komme kann, wurde die Güte der Befragung durch die Gütekriterien Objektivität, Reliabilität, Validität und Repräsentativität geprüft.

\subsection{Objektivitaet}

Die Unabhängigkeit beschreibt das Ausmaß, in dem das Ergebnis der Untersuchung frei und unabhängig von Einflüssen außerhalb der befragten Person ist \cite{rost_2004_lehrbuch}. Die Interpretationen und Schlüsse müssen auf "Fakten und Daten beruhen, sowie einer Prüfung standhalten" und "die Sammlung, Analyse und Interpretation der Daten ist transparent und nachvollziehbar hinsichtlich der wissenschaftlichen Argumentation zu gestalten" \cite{Bargheer_2015}.

Die Objektivität der durchgeführten Befragung ist gegeben, da durch die elektronische Onlinebefragung eine zeitliche und räumliche Unabhängigkeit bei der Beantwortung gewährleistet ist. Die Befragung wurde für alle Teilnehmer und Teilnehmerinnen nach identischer Anrede, Einladung und Instruktion und ohne Untersuchungsleiter durchgeführt und war somit nicht von besonderen Situationsvariablen abhängig. Die Sammlung und Analyse wurde transparent und offen gestaltet, da die Auswertung sowie die Daten unmittelbar nach Abschluss der Erhebung veröffentlicht wurden.

\subsection{Reliabilitaet}

Die Reliabilität gibt den Grad der Genauigkeit an, mit der durch die empirische Datenerhebung ein Merkmal erfasst wird \cite{rost_2004_lehrbuch}, unabhängig davon was er erfasst. Schelten definiert einen Test als reliabel, "wenn er das, was er misst, genau misst" \cite{schelten_1997_testbeurteilung}.  Sie spiegelt die Replizierbarkeit von Messergebnissen und Zuverlässigkeit einer Datenerhebung wieder. Von einer hohen Reliabilität der durchgeführten Befragung kann ausgegangen werden, da bei den übernommenen Fragen aus der Messung des SOFI im Jahr 2007 zu selben oder ähnliche Ergebnisse erzielt werden konnten und die Reliabilität der Online-Befragung mit der schriftlichen Befragung als vergleichbar eingestuft werden kann \cite{Batinic_2003}. Weitere Reliabilitätstests konnten vernachlässigt werden, weil die Befragung größtenteils aus statistischen Abfragen und Bewertungsfragen bestand.

\subsection{Validitaet}
In der Literatur werden zwei Typen von Validität unterschieden \cite{rost_2004_lehrbuch}: Die interne und die externe Validität. Von einer hohen internen Validität wird ausgegangen, wenn die erzielten Ergebnisse klar und eindeutig interpretierbar sind \cite{raab_2012_fragebogen}. Von einer hohen externe Validität wird ausgegangen, wenn die Ergebnisse des Experiments auf die Realität übertragbar sind \cite{bortz1995forschungsmethoden}.

Im Rahmen der durchgeführten Befragung zeigt die Validität, ob das Messinstrument Fragebogen wirklich das misst was dazu beiträgt, die Fragestellungen der Arbeit zu beantworten. Die Validität wurde durch die Übernahme der Grundstruktur und von Items der Studie "Wissenschaftliche Publikationen im Internet: Wissenschaftler als Leser und Autoren" des SOFI in Göttingen gewährleistet. Wie bei der Reliabilität wird auch die Validität einer schriftlichen Befragung mit der einer Online-Befragung als vergleichbar eingestuft. Die Validität der neu erstellten, angepassten  und zusammengelegten Items wurde durch die Auswertung des Pretests sowie durch die Einbeziehung der Inhaltsanalyse in die Erstellung der Fragen sichergestellt.

\subsection{Repräsentativität}

Um die Repräsentativität der Studie sicherzustellen wurden die Rückläufer der Befragung auf vorhandene Informationen zur fachliche Zuordnung, den beruflichen Status und das Alter ausgewertet und mit vergleichbaren Studien verglichen. Verschiedene Verzerrungen sind nur zu vermuten, da die kontaktierten Menschen ausschließlich online angeschrieben wurden. Da die Umfrage jedoch ohne Zugangsbeschränkung öffentlich online ausgefüllt werden konnte, war es jedem Interessenten möglich teilzunehmen. Darüber hinaus können die Ergebnisse der Erhebung insofern als repräsentativ gelten, als dass sie auf sehr großen Stichprobe (n=1.112) beruht.

---- TODO: Weiter ausarbeiten + Grafik aus Vergleich Kriterien fachliche Zuordnung, den beruflichen Status und das Alter SOFI, Science 2.0 Studie bauen ----

\section{Auswertung der Befragung}

Im Zeitraum vom 18. August 2014 bis zum 18. Januar 2015 haben 1.768 Personen die Befragung zur wissenschaftlichen Kommunikation im Rahmen des Promotionsvorhabens gestartet. 1.467 Teilnehmer haben die Umfrage teilweise und 1.112 komplett abgeschlossen. Die erhobenen Daten der 1.112 Teilnehmer des Online-Fragebogens werden mit Hilfe der computerunterstützten Datenaufbereitung statistisch ausgewertet und hier dargestellt. Die Darstellung der Ergebnisse orientiert sich dabei an den definierten Fragestellungen dieser Arbeit und werden wie folgt verteilt.

Zu Beginn wird die Einordnung der Befragten und eine Darstellung der soziodemographischen Daten vorgenommen. Im darauffolgenden Abschnitt werden die Erhebungsergebnisse zu den Veränderungen in der wissenschaftlicher Kommunikation durch die Digitalisierung geschildert, gefolgt von der Darstellung der Ergebnissen über das Verständnis von Offenheit und Interesse an Offenheit bei der wissenschaftlichen Kommunikation unter den Befragten, sowie die Überprüfung der Idee der Öffnung von wissenschaftlicher Kommunikation und der wissenschaftliche Realität auf Grundlage der Ergebnisse. Eine weiteres Forschungsziel war die Herausarbeitung und Erforschung der Katalysatoren und Hindernisse bei der Etablierung der Öffnung von Wissenschaft und Forschung, die im nächsten Abschnitt anhand der Umfrageergebnisse dargestellt werden. Im folgenden Abschnitt werden die Fragen ausgewertet, bei denen die Öffnung von wissenschaftlicher Kommunikation in den Kontext von wissenschaftlicher Reputation und der jeweiligen Fachdisziplin gestellt wurden. Nachfolgend werden die Ergebnisse der Befragung in Bezug auf die Auffassungen zu verschiedenen Fragen in den unterschiedlichen Alters- und Statusgruppen dargestellt. Abschließend erfolgte die Elaborierung der Veränderungen im Vergleich zur SOFI-Studie.

---- TODO: Reihenfolge mit Fragestellungen überprüfen ----

\subsection{Soziodemographischen Daten}

Alle Angaben der 1.112 Teilnehmer des Online-Fragebogens, die die Umfrage komplett abgeschlossen haben sind in die folgende Auswertung der Befragung eingelossen. Die Auswertung ergab dabei folgende soziodemographische Daten der Befragten:

\begin{itemize}
\item \textbf{Geschlecht:} 444 der Befragten waren weiblich (40 Prozent), 606 und 55 Prozent männlich. 62 Personen oder 6 Prozent machten keine Angabe zu ihrem Geschlecht.
\end{itemize}

\begin{figure}[h!]
\includegraphics{graphid:8f2pe}
\caption{Geschlecht der befragten Wissenschaftler und Wissenschaftlerinnen}
\end{figure}

\begin{itemize}
\item \textbf{Alter:} Die prozentuale Verteilung des Alters gestaltete sich wie folgt: 4,4 Prozent (46) waren zum Zeitpunkt der Befragung jünger als 31 Jahre, die größte Altersgruppe mit 31,2 Prozent stellten die 31 bis 40 Jährigen dar. 17 Prozent der Befragten waren zwischen 41 und 50 Jahre alt, während 15 Prozent angab älter als 50 Jahre zu sein. 1 Prozent machten bei der Frage nach ihrem Alter keine Angaben.
\end{itemize}

\begin{figure}[h!]
\includegraphics{graphid:bJE49}
\caption{Alter der befragten Wissenschaftler und Wissenschaftlerinnen}
\end{figure}

\begin{itemize}
\item \textbf{Berufsstatus:} Unter den Befragten gaben 25 Prozent an, Privatdozenten, Juniorprofessoren oder Professoren zu sein. 56 Prozent der Teilnehmer waren wissenschaftliche Mitarbeiter, 20 Prozent wissenschaftliche Mitarbeiter mit Promotionsvorhaben, 23 Prozent bereits fertig promovierte wissenschaftliche Mitarbeiter und 13 Prozent Mitarbeiter ohne Promotionsvorhaben oder abgeschlossener Promotion. 10 Teilnehmer (1 Prozent) gaben an Wissenschaftler in der Privatwirtschaft zu sein. 35 Befragte (3 Prozent) wurden unter "Sonstiges" subsummiert.
\end{itemize}

\begin{figure}[h!]
\includegraphics{smallgraphid:aSGJT}
\caption{Position der Teilnehmer/innen}
\end{figure}

\begin{itemize}
\item \textbf{Tätigkeitsdauer in der Wissenschaft:} Nur 6 Prozent der Befragten gaben an "weniger als 1 Jahr" in der Wissenschaft tätig zu sein. 20 Prozent war seit mehr als einem aber weniger als drei Jahre in der Wissenschaft tätig. 24 Prozent gaben an zwischen drei und sechs Jahren wissenschaftlich tätig zu sein. 15 Prozent der Teilnehmer und Teilnehmerinnen war mehr als sechs aber weniger als zehn Jahre in der Wissenschaft. Die größte Gruppe gab an, "mehr als 10 Jahre" wissenschaftlich tätig zu sein (32 Prozent). 2 Prozent gaben an, "nicht in der Wissenschaft tätig" zu und 1 Prozent enthielten sich einer Angabe.
\end{itemize}

\begin{figure}[h!]
\includegraphics{smallgraphid:UQmsE}
\caption{Tätigkeitsdauer in der Wissenschaft nach Jahren}
\end{figure}

\begin{itemize}
\item \textbf{Forschungseinrichtung:} Die große Mehrzahl der Teilnehmer (78 Prozent) gaben an, an einer deutsche Universität/Hochschule beschäftigt zu sein. Mit 5 Prozent waren 59 Befragten an einem Institut der Leibniz-Gemeinschaft und 5 Prozent an einer "Sonstigen" Einrichtung tätig zu sein. Nur 1 Prozent der Befragten wirkten an einem Max-Planck-Institut und 0,4 Prozent an einem Institut der Fraunhofer Gesellschaft. An einer Universität/Hochschule im deutschsprachigen Ausland waren 4 Prozent und im nicht-deutschsprachigen Ausland 1 Prozent tätig. 1 Prozent arbeitete an einer deutschen Fachhochschule. 11 Befragte (1 Prozent) gaben an einem „An“-Institut (eigenständige Forschungseinrichtung, angegliedert an einer deutschen Hochschule) beschäftigt zu sein.
\end{itemize}

\begin{figure}[h!]
\includegraphics{graphid:3FUsi}
\caption{Fachkolleg der befragten Wissenschaftler}
\end{figure}

Die größte Gruppe (38 Prozent) bildeten die Befragten der Fachgruppe der Geistes- und Sozialwissenschaften. 29 Prozent gaben an, in den Naturwissenschaften verortet zu sein. Aus den Lebenswissenschaften kamen 18 Prozent der Befragten. Die kleinste Gruppe unter den Teilnehmern stellten mit 13 Prozent Wissenschaftler aus der Fachgruppe der Ingenieurwissenschaften dar. 34 der Befragten (3 Prozent) konnten nicht eindeutig einer der vier Fachgruppen zugeordnet werden oder machten keine konkrete Angabe.

---- TODO: Verteilung und Titel ueberarbeiten ----

Die überwiegende Mehrheit der Befragten (97 Prozent) gab an, in der Forschung tätig zu sein. Mehr als die Hälfte aller gab an "überwiegend" (53 Prozent) forschend zu arbeiten. Demgegenüber gaben 2 Prozent beziehungsweise 19 Personen an "gar nicht" zu forschen und 3 Prozent machten keine Angabe. Insgesamt gaben 84 Prozent der Teilnehmer und Teilnehmerinnen in der Umfrage an, lehrend tätig zu sein, davon 10 Prozent sogar "überwiegend". Demgegenüber waren nur 51 Personen (5 Prozent) in der klinischen Versorgung tätig. Administrativen Tätigkeiten gingen 5 Prozent "überwiegend", 13 Prozent "gleichgewichtig" und 31 Prozent "weniger" nach. 4 Prozent gaben an, im "Sonstigen" Bereichen tätig zu sein, 93 Prozent ergänzten im Freitextfeld genauere Angaben.

---- TODO: Arbeitsbereich Grafik bauen ----

\subsection{Veränderungen wissenschaftlicher Kommunikation durch die Digitalisierung}

Der Einfluss der Digitalisierung auf das wissenschaftlichen Kommunikationssystem lässt sich durch die beobachtbare Entwicklung der Nutzung von Such- und Rezeptionsmöglichkeiten nach Literatur untersuchen \cite{Hanekop_2014}. Dabei spielt die Schnittstelle zwischen informeller und formeller Kommunikation eine zentrale Rolle \cite{Hanekop_2014}.

Diese Annahme konnte auch im Rahmen der aktuellen Befragung gestützt werden. "Um sich im eigenen Fach auf dem Laufenden zu halten", nutzten laut der Befragung im Jahr 2007 79 Prozent häufig oder sehr häufig Online-Journale. 2014/2015 stieg dieser Wert auf 88 Prozent. Im Gegenzug nutzen nur noch 32 Prozent die analogen Printausgaben von Zeitschriften, 2007 waren es noch 50 Prozent der Befragten die auch auf die analoge Version der Journale zurückgriffen.

Auch bei der Beantwortung der Frage, welche der Suchmöglichkeiten häufig genutzt werden, um gezielt nach Literatur zu suchen, lässt sich ein klare Veränderung hin zu den digitalen Medien feststellen. Ähnlich wie in der Befragung 2007 durch das SOFI gab auch 2014/2015 weniger als ein Viertel der Befragten an, über die "Konventionelle Suche" (in Bibliotheksregalen, Archiven etc.) nach Literatur zu suchen. Demgegenüber stieg der Anteil von Wissenschaftlerinnen und Wissenschaftlern, die angaben den Dienst Google Scholar für die Literatursuche zu verwenden, von 31 Prozent in 2007 auf 52 Prozent in 2014/2015.

Wie schon in der SOFI-Studie \cite{Hanekop_2014}, belegen diese aktuellen Ergebnisse erneut, dass sich die webbasierten Such- und Rezeptionsmöglichkeiten in der wissenschaftlichen Kommunikation durchgesetzt haben. Sie können darüber hinaus auch als Argument für die These nach der generell fortgeschrittene Nutzung digitaler Medien bei der wissenschaftlichen Kommunikation gewertet werden.

\subsection{Verständnis von Offenheit im Rahmen wissenschaftlicher Kommunikation}

16 Prozent der Befragten gaben an, sich häufig über Open-Access Repositorien (z.B. arxiv.org) auf dem Laufenden zu halten, 17 Prozent beziehungsweise 190 der 1.112 Befragten nutzen Open-Access Portale (z.B. Directory of Open Access Journals) um sich über den aktuellen Stand der Forschung zu informieren. Für die Suche nach Literatur nutzten 4 Prozent der Teilnehmerinnen und Teilnehmer häufig "Suchmaschinen für Open Access", aber mehr als 50 Prozent die jeweils "fachspezifische Suchportale", die ebenfalls Open-Access Publikationen enthalten.

55 Prozent bewerteten "die Forderung nach kostenfreiem Zugang zu allen wissenschaftlichen Publikationen für Leser (Open Access)" als "sehr gut". Knapp unter einem Viertel (22 Prozent) erachteten die Forderung als "gut". 19 Prozent waren sich bei der Frage unsicher und antworteten mit "teils/teils" und 38 der Befragen lehnten die Forderung nach Open Access ab, 9 davon sogar "entschieden". Nur 7 Teilnehmer und Teilnehmerinnen oder 1 Prozent gaben an Open Access nicht zu kennen.

Bei der Betrachtung der Meinung zu Open Access differenziert nach beruflichen Status, war in die Gruppe der Doktoranden am meisten homogen (89 Prozent). Unter den promovierten wissenschaftlichen Mitarbeiter und Mitarbeiterinnen bewerteten 80 Prozent die Forderung nach Open Access als "gut" oder "sehr gut". Bei den Wissenschaftlichen Mitarbeitern ohne Promotion (75 Prozent), bei Doktoranden mit einer wissenschaftlichen Mitarbeiterstelle (73 Prozent) unterstützen und bei Privatdozenten (73 Prozent) sowie bei Professoren (72 Prozent) ist die Forderung nach freien und offenen Zugang zu wissenschaftlichen Publikationen ebenfalls mehrheitlich stark ausgeprägt. Mit 65 Prozent war die Befürwortung unter Juniorprofessoren am geringsten ausgeprägt.

Während 15 Prozent angaben in der Open Access-Bewegung engagiert zu sein, verneinten 72 Prozent ein Engagement in der Open Access-Bewegung. 13 Prozent enthielten sich der Angabe zu ihrem Engagement. Die relativ größte Gruppe der Engagierten ist nach Altersgruppe zwischen 51-55 Jahren (33 Prozent der Befragten in der Altersgruppe). Sie kommt aus dem Fachbereich der Lebenswissenschaften (21 Prozent der befragten Lebenswissenschaftler) und gehört zu Gruppe der Professoren (25 Prozent aller befragten Professoren).

43 Prozent oder 483 Personen kommentierten ihre Meinung zu Open Access im optionalen Freitextfeld, wobei 73 Prozent der abgegebenen Kommentare von Befragten stammten, die Open Access gut oder sehr gut finden. Unter den Befragten, die Open Access ablehnen, kommentierten 76 Prozent ihre Haltung zur Forderung nach kostenfreiem Zugang zu allen wissenschaftlichen Publikationen. Von den Personen, die "teils/teils" angaben, machten fast die Hälfte (48 Prozent) ihre unsichere Haltung in den Kommentaren deutlich. 159 der 854 Befragten (19 Prozent), die Open Access gut oder sehr gut finden, gaben an, selbst aktiv in der Open Access Bewegung zu sein. Überraschend sind an dieser Stelle 12 Personen (6 Prozent) die "teils/teils" bezüglich ihrer Meinung zu der Forderung nach Open Access angegeben haben, aber sich dennoch zum Teil der Bewegung zählen. Die 483 abgegebenen Kommentare zu der Meinung der Befragten bezüglich der Forderung nach kostenfreiem Zugang zu allen wissenschaftlichen Publikationen in dem Freitextfeld können in folgende Kategorien eingeordnet werden: ideelle , praktische, ökonomische, rechtliche, qualitative und sonstige Meinungen.

---- TODO: Grafik aus Freitextkategorien bauen bzw. Tabelle anlegen  ----

In einer weiteren Frage wurde das Verständnis von Open Access am Beispiel der Definition der Budapest Open Access Initiative \cite{boai_2012} abgefragt. Knapp drei Viertel der Teilnehmer und Teilnehmerinnen (75 Prozent) stimmten dieser Definition uneingeschränkt zu, 19 Prozent stimmten dieser Definition nur "teils/teils" zu und 2 Prozent lehnten die Definition ab. 3 Prozent der Befragten beantworteten die Frage mit "weiß nicht" und fünf Teilnehmer (1 Prozent) enthielten sich der Beantwortung der Frage.

Wurde "teils/teils" oder "weiß nicht" als Antwort ausgewählt, konnte in einer optionalen Freitext-Frage beantwortet werden, welche Aspekten der Definition genau keine Zustimmung und welche Zustimmung fanden. Davon machten 38 Prozent der möglichen 247 Befragten gebrauch. Die 93 Antworten in den Freitextfeldern konnten in fünf Kategorien eingeordnet werden: definitorische Aspekte, rechtliche Aspekte, technische Aspekte, ökonomische Aspekte und sonstige Aspekte.

---- TODO: Grafik von Freitextfeld nach Aspekten ----

\subsection{Interesse an Offenheit bei der wissenschaftlichen Kommunikation}

Die Mehrheit der Befragten gab an, Interesse am Zugang zu Forschungsdaten anderer Wissenschaftler und Wissenschaftlerinnen zu haben (71 Prozent). Über ein Drittel dieser Teilnehmer und Teilnehmerinnen erklärte ihr genaues Interesse durch die Beantwortung der optionalen Frage. 29 Prozent (319 der Befragten) hatte kein Interesse am Zugang zu Forschungsdaten anderer Wissenschaftler und Wissenschaftlerinnen.

---- TODO: genaues Interesse darstellen ----

Die Frage, ob sich die befragten Wissenschaftler vorstellen können ihre "Forschungsdaten und alle weiteren Informationen, die während der wissenschaftlichen Arbeit anfallen (z.B. Laborbücher, Entwürfe oder andere Dokumente und Daten) unter Berücksichtigung von Datenschutz öffentlich zur Verfügung zu stellen", beantworteten 28 Prozent uneingeschränkt mit "ja". 36 Prozent der Befragten schränkten ein, dass sie das "nur unter bestimmten Bedingungen" tun würden und 29 Prozent lehnten die Veröffentlichung von Forschungsdaten und weiteren Informationen komplett ab. 7 Prozent wussten darauf keine Antwort. Die bedingte, Freitextfrage nach der Erläuterung der "bestimmten Bedingungen" unter denen die Befragten Forschungsdaten und weiteren Informationen veröffentlichen würden beantworteten 214 Teilnehmer und Teilnehmerinnen. Die Antworten in den Freitextfeldern konnten in folgende Kategorien des Interesses an Forschungsdaten eingeordnet werden: Reproduktion von Ergebnissen, nicht publizierte Daten, Darstellung der Methodik, Vergleiche, wissenschaftlicher Austausch und Synergie, Überprüfung der Validität, Konsistenz und Qualität der Daten, Weiterverwendung der Daten, Analyse des aktuellen Forschungsstands und sonstige Interessen.

---- TODO: Grafik nach Kategorien des Forschungsinteresses bauen + weiter ausarbeiten ----

\subsection{Diskrepanz zwischen der Idee der Öffnung von wissenschaftlicher Kommunikation und der wissenschaftliche Realität}

 Bei der Auswertung der Erhebung konnte ein mehrheitlich stark verbreitetes Verständnis von Open Access und die mehrheitliche Unterstützung der Forderung nach Öffnung von Wissenschaft sowie Interesse an Forschungsdaten anderer dargestellt werden. Demgegenüber steht die Frage, wie wichtig den befragten Wissenschaftlern das Kriterium "freier Zugang zum Volltext" bei den eigenen Veröffentlichungen ist. Der größere Teil der Befragten (50 Prozent) erachten dies als "weniger wichtig" oder "nicht wichtig". Demgegenüber erachteten 45 Prozent das Kriterium "freier Zugang zum Volltext im Internet" als wichtig oder sehr wichtig bei der eigenen Veröffentlichung. 6 Prozent der 1.112 Befragten enthielten sich der Beantwortung der Frage.

Diese negative Bewertung des Kriteriums "freier Zugang" wird durch die Betrachtung weiterer Merkmale, die Wissenschaftler bei den eigenen Veröffentlichungen als "wichtig" oder "sehr wichtig" erachten, bestätigt. Während der fachlich einschlägige Schwerpunkt (91 Prozent), das Renommee der Zeitschrift/des Verlags (82 Prozent) und akzeptable oder keine Veröffentlichungskosten für Autoren (79) relativ häufig als "wichtig" oder "sehr wichtig" erachtet wird, stellt die Veröffentlichung unter einer Open-Access Lizenz nur für 33 Prozent ein wichtiges Kriterium bei der Publikation eigener Inhalte dar. Die Mehrheit der Befragten (56 Prozent) erachten dieses Kriterium sogar als "weniger wichtig" oder "nicht wichtig". Der akzeptable Preis für den Erwerb der Publikation spielt ebenfalls nur für 43 Prozent eine wichtige Rolle, für 50 Prozent der Befragen ist er "weniger wichtig" oder "nicht wichtig".

Weitere Kriterien für wissenschaftlichen Veröffentlichung von Beiträgen oder Büchern aus Sicht der Befragten sortiert:
\begin{itemize}
\item 78 Prozent der Teilnehmer und Teilnehmerinnen der Erhebung sehen die internationale Verbreitung als mindestens wichtige, wenn nicht sogar sehr wichtiges Kriterium im Rahmen der eigenen Veröffentlichungen an, die übrigen 19 Prozent finden dieses "weniger wichtig" oder "unwichtig".
\item Das Peer-Review Verfahren wird von 75 Prozent der Befragten als wichtiges Kriterium erachtet nur 19 Prozent sind diesbezüglich gegensätzlicher Meinung.
\item 75 Prozent der Befragten Personen sehen die Transparenz des Review-Prozesses als wichtig an während 18 Prozent dieses Kriterium als "weniger wichtig" oder "unwichtig" erachten.
\item Eine leichte Auffindbarkeit der eigenen Publikation im Internet ist 71 Prozent der befragten Wissenschaftler und Wissenschaftlerinnen mindestens wichtig. Für 25 Prozent der Befragten ist das "weniger" bis "nicht wichtig".
\item Die rasche Veröffentlichung der eigenen Publikation ist für 68 Prozent von Bedeutung. Für 29 Prozent hat dieses Kriterium keine besondere Bedeutung.
\item Das Ranking, wie der Impact-Faktor einer wissenschaftlichen Zeitschrift, wurden 58 Prozent der befragten Personen als "wichtig" und von 35 Prozent als "weniger wichtig" oder "unwichtig" bewertet.
\item Die Reputation der Herausgeber war für die Mehrheit der Befragten eher "unwichtig" bis "nicht wichtig" (48 Prozent). Demgegenüber erachteten 47 Prozent dieses Kriterium als sehr wichtig bis wichtig.
\end{itemize}

---- TODO: Grafik bauen ----

Eine weitere Frage im Fragebogen betraf die Einschätzung der Befragten, ob ihre eigenen Veröffentlichungen in Zeitschriften oder Büchern für Leser potentiell gut zugänglich sind. 32,0 Prozent der Befragten beantworteten die Frage mit der Option "ja, gut zugänglich". Mit "teils/teils" antworteten 47 Prozent und 12 Prozent der Teilnehmer und Teilnehmerinnen wählte die Option "nein, nicht so gut zugänglich" (9 Prozent) oder "nein, sehr schlecht"(2 Prozent). 107 oder 10 Prozent wussten die Frage nicht zu beantworten.

Bei der Frage, ob Aufsätze, Texte oder Bücher publiziert wurden, die vom Verlag selbst frei zugänglich gemacht wurden, antworteten 140 Teilnehmer und Teilnehmerinnen (13 Prozent) mit "ja, einen Beitrag" und 23 Prozent mit "ja, mehrere Beiträge". 54 Prozent oder 605 der Befragten, hatte zum Zeitpunkt der Befragung noch keine Aufsätze, Texte oder Bücher publiziert, die vom Verlag selbst frei zugänglich gemacht wurden.  19 Prozent derer, die bisher noch nicht bei einem Verlag Open Access veröffentlicht hatten, gaben an dies zu planen. 10 Prozent der Befragten beantworteten die Frage nicht.

Die 397 (36 Prozent) der 1.112 Befragten die angaben, bereits Inhalte frei publiziert zu haben, wurden gebeten auszusagen, wie viele Aufsätze, Texte oder Bücher sie bisher frei veröffentlicht haben:
\begin{itemize}
\item Bücher - 63 Befragte (16 Prozent) beantworteten die optionale Frage. Davon gaben 26 an, bisher kein Buch veröffentlicht zu haben, das frei zugänglich gemacht wurde. Bezieht man die Antworten nicht mit ein, die 0 Bücher angaben, hatten die 37 Befragten jeweils circa 2 Bücher veröffentlicht, die vom Verlag selbst frei zugänglich gemacht wurden.
\item Texte - 192 der 397 Befragten (48 Prozent) gaben an, mindestens 1 Text frei veröffentlicht zu haben. Im Durchschnitt hatten die Befragten jeweils rund 3 Texte "frei zugänglich" veröffentlicht.
\item Daten - 3 Prozent (10 Personen) gab an, mindestens einen Datensatz frei veröffentlicht zu haben.
\item Sonstiges - Keiner der Teilnehmerinnen und Teilnehmer gab an, "sonstige Beiträge" frei veröffentlicht zu haben.
\end{itemize}

Den Aufwand, die eigene Publikationen im Internet frei zur Verfügung zu stellen, schätzte der größte Teil der Befragten (31 Prozent) als gering ein. 255 der Befragten Wissenschaftler und Wissenschaftlerinnen (28 Prozent) schätzten den Aufwand ihre Publikationen im Internet frei zur Verfügung zu stellen als "mittelgroß" oder "groß" ein. 23 Prozent waren sich unsicher und wählten "teils/teils" und 19 Prozent wusste nicht den Aufwand einzuschätzen.

Während die Mehrheit der Befragten den Aufwand für die freie Veröffentlichung von Publikationen als nicht groß einschätzte, zeichnete sich bei der Auswertung der Frage nach dem geschätzten Aufwand für die Veröffentlichung von Forschungsdaten im Internet ein anderes Bild. 55 Prozent der Befragten schätzte den Aufwand die Forschungsdaten zu veröffentlichen als "groß" ein. Die kleinste Gruppe der Befragten (10 Prozent) vermutete dabei einen geringen Aufwand, 15 Prozent schätzten den Aufwand "teils/teils" ein und ein Fünftel (20 Prozent) wusste die Frage nicht zu beantworten.

---- TODO: Grafik aus beiden Items bauen ----

\subsection{Treiber und Bremser für die Etablierung der Öffnung von Wissenschaft und Forschung}

Im vorherigen Kapitel wurden Katalysatoren und Hindernisse für die Öffnung der wissenschaftlichen Kommunikation in der Literatur identifiziert und herausgearbeitet. Diese wurden anhand der empirischen Ergebnisse überprüft. Im Rahmen der Befragung sollten darüber hinaus auch evaluiert werden, welche Faktoren und Argumente aus der Sicht von Wissenschaftlern die Öffnung von Wissenschaft und Forschung in der jeweiligen wissenschaftlichen Disziplin begünstigen und welche sie behindern.

Die Auswertung nach Häufigkeit der ausgewählten Antwortoptionen für die Öffnung der wissenschaftlichen Kommunikation aus Sicht der befragten Wissenschaftler und Wissenschaftlerinnen ergab folgende Reihenfolge:
\begin{enumerate}
\item 721 (65 Prozent) mal wurde das Argument "Beschleunigung der Wissensverbreitung und -verwertung" von den Befragten ausgewählt.
\item Das Argument der "Eröffnung neuer Möglichkeiten für die Wissensverbreitung" wurde von 64 Prozent der 1.112 Befragten unter den Antwortmöglichkeiten am zweithäufigsten ausgewählt.
\item Die umfassendere Verfügbarkeit von bereits finanzierter Forschung stellte für 55 Prozent ein Argument für die offene und freie Veröffentlichung der eigenen wissenschaftlichen Kommunikation dar.
\item Eine generelle Erleichterung der wissenschaftlichen Kommunikation, sahen 49 Prozent der Befragen als Argument für die Öffnung der wissenschaftlichen Kommunikation an.
\item Die Förderung des interdisziplinären Austausch von Wissenschaftlern und Wissenschaftlerinnen erachteten 45 Prozent als Argument an.
\item 44 Prozent oder 489 der Befragten sahen in der Überwindung sozialer, nationaler und globaler Wissenskluften ein Argument.
\item Dem Argument, dass die Öffnung von Wissenschaft eine Chance für eine umfassendere und transparentere Qualitätsmessung ist, stimmten 34 Prozent der befragten Wissenschaftler zu.
\item 250 oder 23 Prozent der Befragten sahen in der nachhaltigen und unabhängigen Archivierung der Informationen ein Argument für die Öffnung von Wissenschaft und Forschung.
\item 20 Prozent betrachteten die Möglichkeit indirekter Wirtschaftsförderung durch freien und offenen Wissenstransfer als Argument.
\item Die Möglichkeit der Beilegung der vorherrschenden Zeitschriften- und Monographienkrise akzeptierten 16 Prozent der Befragten als Argument für die Öffnung der eigenen wissenschaftlicher Kommunikation.
\end{enumerate}

8 Prozent der 1.112 Befragten gaben an, dass ihrer Meinung nach keines der genannten Argumente für die eigene Öffnung der wissenschaftlichen Kommunikation und aller Informationen aus dem Forschungs-/Arbeitsprozess sprechen. Weitere 47 oder 4 Prozent machen weitere Angaben unter "Sonstiges".

---- TODO: Sonstige Kategorien ausarbeiten und aufteilen und Grafik bauen ----

Bei dem Item "Argumente gegen die Öffnung der wissenschaftlichen Kommunikation" zeichnete sich ein kein eindeutiges Bild:
\begin{enumerate}
\item Überwiegend (43 Prozent) wurden von den Befragten die fehlenden Reputationskriterien für die Bewertung von offener Wissenschaft gewählt.
\item Bei 40 Prozent wurde die "Gefahr der Fehlinterpretation und Falschinformation durch Wissenschaft" ausgewählt.
\item Der erhöhte zeitliche Mehraufwand für die Bereitstellung der wissenschaftlichen Publikationen und/oder Forschungsdaten führten 379 beziehungsweise 34 Prozent als Argument an.
\item 30 Prozent erachteten das Argument einer Erschwerung der eindeutigen Zuordnung von Texten, Arbeiten und Daten zu den Urhebern als Argument gegen die eigene Öffnung wissenschaftlicher Kommunikation an.
\item An fünfter Stelle wurde dem Argument zugestimmt, dass die Qualität der wissenschaftlichen Arbeit leidet (27 Prozent).
\item Dem Argument, dass die Langzeitarchivierung und langfristige Auffindbarkeit nicht (dezentral) gewährleistet werden kann, stimmten 26 Prozent der Befragten zu.
\item Das Argument, dass die freie und offene Verfügbarkeit wissenschaftlicher Informationen zu hohen Kosten führt und keine Refinanzierung absehbar ist, stimmten 274 der 1.112 Befragten (25 Prozent) zu.
\item 9 Prozent der Befragten akzeptierten das Argument, dass die Öffnung der Kommunikation eine Bedrohung für Publikations- und Pressefreiheit in der Wissenschaft darstellt.
\item Dass die offene Bereitstellung von Daten keinen nachhaltigen Mehrwert bietet, dem stimmen 8 Prozent zu und sehen darin ein Argument gegen eine Öffnung des Systems.
\item Dass "Offenheit und Transparenz bei Forschungsförderung Freiheit von Wissenschaft und Forschung gefährden" sahen 5 Prozent als Argument gegen die eigene Öffnung der wissenschaftlichen Kommunikation an.
\end{enumerate}

154 Teilnehmer und Teilnehmerinnen nannten sonstige Argumente (14 Prozent), gegen Open Access und Open Science anzuführen sind. 8 Prozent aller Befragten gab an, dass kein Argument gegen die Öffnung der wissenschaftlichen Kommunikation und aller Informationen aus dem Forschungs-/Arbeitsprozess spricht.

---- TODO: Sonstige Kategorien ausarbeiten und aufteilen und Grafik bauen ----

\subsection{Wissenschaftliche Reputation und die Öffnung von wissenschaftlicher Kommunikation}

Die Teilnehmer und Teilnehmerinnen hatten die Möglichkeit unter mehreren Antwortoptionen auszuwählen, welche Faktoren für wissenschaftliche Reputation in ihrer Disziplin wichtig sind. Am häufigsten wurde dabei die "Anzahl der Beiträge" ausgewählt, 80 Prozent aller Befragten wählten diese Option. Die "Relevanz der publizierten Ergebnisse" wurde von 74 Prozent und "Vorträge auf wichtigen Konferenzen" von 68 Prozent der Wissenschaftler als wichtiger Faktor für wissenschaftliche Reputation in der jeweiligen Disziplin ausgewählt.

Folgende weitere Auswahlmöglichkeiten wurden von den Befragten der Häufigkeit nach ausgewählt:
\begin{itemize}
\item die Bezugnahme und Zitation durch Kollegen wurde von 66 Prozent am vierthäufigsten genannt
\item 65 Prozent nannten den Indikator "Drittmittelprojekte" wichtig für die Reputation
\item 675 oder 61 Prozent der Befragten Wissenschaftler und Wissenschaftlerinnen nannten "Ranking- oder Impactfaktoren" als wichtigen Faktor für Reputation in ihrer Disziplin
\item das Renommee der Forschungseinrichtung war für weniger als die Hälfte der Befragten (48 Prozent) relevant
\item auch das Renommee von Herausgebern oder Mitautoren spielte nur für 36 Prozent einen wichtigen Faktor für die Reputation in der eigenen Disziplin
\item "Netzwerke, Kontakte und ob man dazu gehört" erachteten 35 Prozent als wesentlichen Reputationsfaktor
\item die Gutachtertätigkeit, Herausgeberschaft oder andere Funktionen sehen 28 Prozent als wichtig für die Reputation an
\item knapp ein Viertel der Befragen (24 Prozent) gab an, dass die "Anzahl Monografien" wichtig für die Reputation in der jeweiligen Disziplin ist
\item "Anwendungsrelevanz und Verwertbarkeit" waren für 13 Prozent der Teilnehmerinnen und Teilnehmer ein wichtiges Kriterium für wissenschaftliche Reputation
\item "materielle Ausstattung, Großgeräte etc." wählten 13 Prozent der Befragten
\item 143 der Befragten (13 Prozent) gaben an, dass öffentliche Aufmerksamkeit wichtig für Reputation in ihrer Fachdisziplin ist
\item die "personelle Ausstattung" zählte mit 13 Prozent eher selten zu den wichtigen Faktoren für wissenschaftliche Reputation in allen Disziplin
\item für 8 Prozent der Befragten stellten "Patente" ein Kriterium für Anerkennung dar
\item unter den vorgegebenen Antwortmöglichkeiten stellte die "politische Relevanz" mit 3 Prozent der Befragten den unwichtigsten Faktor für Reputation dar
\item 12 Teilnehmer und Teilnehmerinnen (1 Prozent) gaben in einem Freitextfeld "Sonstige" Faktoren an
\end{itemize}

---- TODO: Grafik bauen und weiter ausarbeiten ----

\subsection{Auffassungen zwischen den unterschiedlichen Fachgruppen}

Die weitläufige Annahme \cite{naeder_2010_open}, dass in unterschiedlichen Fachgruppen unterschiedliche Publikationsformen wichtig sind, konnte auch bei dieser Befragung bestätigt werden. Abgefragt wurden Monografien, Sammelbände, Tagungsbände und Proceedings, Handbücher und Lehrbücher, internationale Zeitschriften, deutschsprachige Zeitschriften, Arbeitsberichte etc. und Sonstige Veröffentlichungsformen.

\begin{itemize}
\item In den Geistes- und Sozialwissenschaften war die Verbreitung der Monografie mit 77 Prozent am stärksten ausgeprägt, gefolgt von den Ingenieurwissenschaften mit 35 Prozent, den Naturwissenschaften mit 29 Prozent und den Lebenswissenschaften mit 17 Prozent.
\item Sammelbände waren ebenfalls am Stärksten bei den Geistes- und Sozialwissenschaften verbreitet (65 Prozent). Wieder gefolgt von den Ingenieurwissenschaften mit 23 Prozent, den Naturwissenschaften mit 14 Prozent und in den Lebenswissenschaften mit 8 Prozent.
\item Tagungsbände und Proceedings stellten bei 73 Prozent der Ingenieurwissenschaftler und -wissenschaftlerinnen eine wichtige Publikationsform dar. In den Geistes- und Sozialwissenschaften  wurde diese Publikationsform bei 47 Prozent, in Naturwissenschaften bei 40 Prozent der Befragen als wichtig erachtet, in den Lebenswissenschaften bei nur 22 Prozent.
\item Handbücher und Lehrbücher wurden von den Befragten der Geistes- und Sozialwissenschaften mit 44 Prozent als wichtig bewertet, bei den Ingenieurwissenschaften mit 42 Prozent, in den Lebenswissenschaften mit 29 Prozent und in den Naturwissenschaften mit 27 Prozent als wichtig angegeben.
\item Mit jeweils 98 Prozent gaben die Teilnehmer und Teilnehmerinnen aus den Lebenswissenschaftenund Naturwissenschaften in der Befragung an, dass internationale Zeitschriften in ihrer Fachdisziplin wichtig sind. In den Ingenieurwissenschaften waren es 90 Prozent und in den Geistes- und Sozialwissenschaften immerhin 77 Prozent.
\item Deutschsprachige Zeitschriften spielten dabei nur bei den Geistes- und Sozialwissenschaften eine signifikante Rolle (65 Prozent). Nur knapp über ein drittel der befragten Ingenieurwissenschaften und -wissenschaftlerinnen erachteten deutschsprachige Zeitschriften als wichtig an. In den Lebenswissenschaften waren es nur noch 24 Prozent und in den Naturwissenschaften ein Fünftel der Befragten.
 \item  Arbeitsberichte spielten weder in den Naturwissenschaften (12 Prozent), den Ingenieurwissenschaften (11 Prozent), in den Geistes- und Sozialwissenschaften (6 Prozent) noch in den Lebenswissenschaften (5 Prozent) eine signifikante Rolle.
 \item Alle übrigen Formen (Sonstige) spielten von 4 Prozent aller Befragten keine Rolle.
 \end{itemize}

---- TODO: Grafik bauen unterschiedliche Publikationsformen pro Fachgruppe und Sonstige darstellen ----

Die Frage, ob die Befragten ihre Veröffentlichungen in Zeitschriften oder Büchern für potentielle Leser gut zugänglich befinden, wurden in den Naturwissenschaften 39 Prozent, in den Lebenswissenschaften 36 Prozent, in den Ingenieurwissenschaften 32 Prozent und in den Geistes- und Sozialwissenschaften 25 Prozent bejaht.

Laut der Erhebung gaben in der Fachgruppe der Ingenieurwissenschaften mehr als 2/3 der Befragten an (75 Prozent), Interesse an den Forschungsdaten anderer zu haben. In den Lebenswissenschaften bestand mit 75 Prozent der Befragten, das zweitgrößte Interesse an bereits erhobenen wissenschaftlichen Daten. Unter den 418 teilnehmenden Sozial- und Geisteswissenschaftlern bekundeten 70 Prozent und unter den Naturwissenschaftlern 69 Prozent ein Interesse an den Forschungsdaten ihrer wissenschaftlichen Kollegen und Kolleginnen.

Die Unterstützung für die Forderung nach Open Access ist in den befragten Fachgruppen unterschiedlich stark ausgeprägt. 88 Prozent der teilnehmenden Wissenschaftler und Wissenschaftlerinnen aus den Lebenswissenschaften bewerteten die Forderung nach kostenfreiem Zugang zu allen wissenschaftlichen Publikationen für Leser (Open Access) mit "sehr gut" oder "gut". In den Naturwissenschaften befürworteten 82 Prozent den kostenfreien Zugang zu allen wissenschaftlichen Publikationen für Leserinnen und Leser, in den Ingenieurwissenschaften 71,6 Prozent und in Geisteswissenschaften hatten 68 Prozent der Befragten eine befürwortende Meinung zu Open Access.

Während 29 Prozent der befragten Naturwissenschaftler angab Open-Access Repositorien zu benutzen um auf dem Laufenden zu bleiben, gaben in den Ingenieurwissenschaften 20 Prozent an, diese Verzeichnisse zu verwenden, 10 Prozent der 418 befragten Geistes- und Sozialwissenschaftler und 9 Prozent der Lebenswissenschaftler. Institutionelle Repositorien werden von den Geisteswissenschaftlern am häufigsten genutzt (24 Prozent), gefolgt von den Lebenswissenschaftlern mit 17 Prozent, den Ingenieurwissenschaften  mit 13 Prozent und den Naturwissenschaften mit 12 Prozent.

Open-Access-Portale werden von den Befragten ebenfalls nur verhalten genutzt. 21 Prozent der Lebenswissenschaftler, 17 Prozent der Geistes- und Sozialwissenschaftler, 16 Prozent der Ingenieurwissenschaften und 14 Prozent der Naturwissenschaftler gaben an Open-Access-Portale wie das Directory of Open Access Journals zu nutzen um sich in Ihrem Fachgebiet auf dem Laufenden zu halten. Diese Zahlen decken sich auch mit der geringen Verbreitung von Open Access Suchmaschinen bei der Literaturrecherche, durchschnittlich 4 Prozent der befragten Wissenschaftler und Wissenschaftlerinnen nutzen solche Angebote.

22 Prozent der befragten Lebenswissenschaftler, 18 Prozent der Naturwissenschaftler, 15 Prozent der Geistes- und Sozialwissenschaftler und 9 Prozent der Ingenieurwissenschaftler gaben an, sich über akademische Social-Media-Plattformen über aktuelle Themen aus dem eigenen Fachgebiet auf dem Laufenden zu halten. Von 14 Prozent der Geistes- und Sozialwissenschaftler und jeweils 5 Prozent der befragten Ingenieur- und Naturwissenschaftler sowie 4 Prozent der Lebenswissenschaftler wurden für diese Aufgabe auch generelle Social Media-Plattformen wie Facebook oder Twitter genutzt. Eine ähnliche Verteilung ergab sich bei der Nutzung von Blogs (Geisteswissenschaften 14 Prozent; Naturwissenschaften 4 Prozent; Lebenswissenschaften und Ingenieurwissenschaften jeweils 3 Prozent).

Bei der Frage nach der Berurteilung der Zugangsmöglichkeiten zu den einzelnen Publikationsformen in den unterschiedlichen Fachgruppen ergab sich folgendes Bild:
\begin{itemize}
\item 6 Prozent der Befragten aus der Geistes- und Sozialwissenschaft bewerteten die Zugangsmöglichkeiten zu Print-Büchern als "schlecht" oder "sehr schlecht", bei elektronischen Büchern (eBooks/Online-Bücher) waren es 12 Prozent. Print-Zeitschriften sind laut 12 Prozent Befragten der Zugang schlecht zugänglich. Laut 4 Prozent der Befragten mangelt es an der Zugänglichkeit zu Online-Zeitschriften. Insgesamt bewerten die Befragten dieser Fachgruppe die Zugänglichkeit mit 67 Prozent "gut" oder "sehr gut".
\item 9 Prozent der Ingenieurwissenschaftler bewerteten die Zugänglichkeit zu analogen Büchern als "schlecht". Für 11 Prozent der Befragten war die Zugänglichkeit zu eBooks ebenfalls "schlecht" oder "sehr schlecht". Den Zugang zu Printzeitschriften bemängelten 19 Prozent, zu den Online-Zeitschriften 6 Prozent der befragten Ingenieurwissenschaftler. Auch in dieser Fachgruppe bewerten die Befragten die Zugänglichkeit mit 67 Prozent als "gut" oder "sehr gut".
\item In den Lebenswissenschaften gaben 22 Prozent der Befragten an, nur über schlechte Zugangsmöglichkeiten zu gedruckten Büchern zu verfügen, bei Online-Büchern laut der Befragung 21 Prozent der Teilnehmerinnen und Teilnehmer der Befragung. Bei Printzeitschriften waren es 24 Prozent, Online 3 Prozent. 46 Prozent bewerteten die Zugänglichkeit mit teils/teils, 31 Prozent mit gut oder sehr gut.
\item Wie in den Ingenieurwissenschaften gaben auch 9 Prozent der Befragten aus den Naturwissenschaften an, "schlechten" oder "sehr schlechten" Zugang zu gedruckten Büchern zu haben. 15 Prozent bewerteten die Zugänglichkeit zu Online-Büchern als "schlecht". 17 Prozent hatte zudem nur schlechten Zugang zu Print-Zeitschriften, 4 Prozent zu Online-Zeitschriften. 70 Prozent als dieser Fachgruppe bewerteten die allgemeine Zugänglichkeit zu Publikationen als "gut" oder "sehr gut".
\end{itemize}

---- TODO: Grafik bauen ----

Signifikante Unterschiede zwischen den verschiedenen Fachgruppen ergab die Auswertung der Erhebung bei der Frage nach der Wichtigkeit von Offenheit bei den eigenen wissenschaftlichen Publikationsvorhaben. In den Geistes- und Sozialwissenschaften erachten 38 Prozent die "freie Verfügbarkeit des Volltexts im Internet" und 30 Prozent die "Veröffentlichung unter einer Open-Access Lizenz" als "sehr wichtig" oder "wichtig" an. Für die überwiegende Anzahl der befragten Geistes- und Sozialwissenschaftler sind beide Faktoren eher "unwichtig" oder "weniger wichtig". Auch in den Ingenieurwissenschaften antworteten die Befragten auf die Frage nach der Wichtigkeit der freie und offene Verfügbarkeit ihrer Beiträge im Internet (58 Prozent) und die "Veröffentlichung unter einer Open-Access Lizenz" (48 Prozent) mehrheitlich mit "weniger wichtig" oder "nicht wichtig". Demgegenüber gaben die Befragten der Fachgruppe der Lebenswissenschaften mehrheitlich an, dass ihnen die freie Verfügbarkeit der Texte im Internet "wichtig", oder "sehr wichtig" ist (61 Prozent). Auch die Veröffentlichung unter einer Open-Access Lizenz war für 54 Prozent der befragten Lebenswissenschaftler und -wissenschaftlerinnen mindestens wichtig. Für die Mehrheit der Befragten, die sich den Naturwissenschaften zugeordnet hatten, war die Verfügbarkeit des eigenen Volltexts im Internet mehrheitlich "wichtig" oder "sehr wichtig" (52 Prozent). Die Veröffentlichung unter einer Open Access Lizenz war den Wissenschaftlern und Wissenschaftlerinnen der Naturwissenschaften allerdings mehrheitlich "weniger" oder "nicht wichtig" (64 Prozent).

Relativ gesehen gaben 22 Prozent der Befragten aus den Lebenswissenschaften an, auf den eigenen Webseiten auch Volltexte zur Verfügung zu stellen. Gefolgt von den Geistes- und Sozialwissenschaftlern, von denen 15 Prozent Volltexte auf den eigenen Internetauftritten oder denen der Institution veröffentlichen, den Naturwissenschaften 11 Prozent und in den Ingenieurwissenschaften 6 Prozent.

---- TODO: Grafik bauen ----

Auf die Frage, ob sie sich vorstellen können, die "Forschungsdaten und alle weiteren Informationen, die während Ihrer wissenschaftliche Arbeit anfallen (z.B. Laborbücher, Entwürfe oder andere Dokumente und Daten) unter Berücksichtigung von Datenschutz öffentlich zur Verfügung zu stellen" antworteten 67,2 Prozent der Befragten Geisteswissenschaftler "ja" oder "ja, aber nur unter bestimmten Bedingungen". Unter den Ingenieurwissenschaftlern bejahten 65 Prozent die Frage grundsätzlich und bei den Befragten Lebenswissenschaften stimmten 63 Prozent der Veröffentlichung von Forschungsdaten und alle weiteren Informationen unter bestimmten Bedingungen zu. Immer noch mehrheitlich aber am wenigsten Vorstellen konnten sich das die Naturwissenschaftler (59 Prozent). Unvorstellbar war die Veröffentlichung der Forschungsdaten für 35 Prozent der befragten Naturwissenschaftler, 31 Prozent der Lebenswissenschaftler, 28 Prozent der Ingenieurwissenschafter und 26 Prozent der Geisteswissenschaftler. Die Option "weiss nicht" klickten durchschnittlich 7 Prozent der Befragten an.

Die "Anzahl der Publikationen" wurde im Rahmen der durchgeführten Umfrage von den Befragten aller Fachgruppen als überdurchschnittlich wichtiger Faktor für wissenschaftliche Reputation angesehen. So gaben 83 Prozent der Befragten aus den Geistes- und Sozialwissenschaften, 81 Prozent der Befragten aus den Naturwissenschaften, 78 Prozent der Teilnehmer und Teilnehmerinnen aus den Lebenswissenschaften und 73 Prozent der Ingenieurwissenschaftler und -wissenschaftlerinnen an, dass dieses Kriterium mindestens "wichtig" wenn nicht sogar "sehr wichtig" für die Reputation im jeweiligen Fach ist.

Das Kriterium "Relevanz der publizierten Ergebnisse" wurde in den verschiedenen Fachgruppen als überwiegend wichtig bewertet. In den Naturwissenschaften gaben 81 Prozent der Befragten an, dass die Relevanz ein wichtiger Faktor für Reputation in ihrer Disziplin ist. In den Lebenswissenschaften 78 Prozent, in den Geistes- und Sozialwissenschaften 70 Prozent und in den Ingenieurwissenschaften 68 Prozent. Die durchschnittliche Häufigkeit über alle Fachgruppen, bei denen der Faktor als wichtig für wissenschaftlichen Reputation in der jeweiligen Disziplin ausgewählt wurde, lag 74 Prozent.

Die Relevanz von Rankings und des Impact-Factors auf die Reputation variierte stark in Abhängigkeit von der jeweiligen wissenschaftlichen Fachgruppe. In den Lebenswissenschaften stellte dieser Faktor für 85 Prozent der Befragten ein wichtiges Kriterium für die Erlangung von wissenschaftlicher Reputation in der Disziplin dar. Auch in den Naturwissenschaften gaben 72 Prozent der Befragten an, das diese Rankings ein wichtiger Faktor sind. Demgegenüber gaben 56 Prozent der Ingenieurwissenschaftler und -wissenschaftlerinnen und nur 44 Prozent der Sozial- und Geisteswissenschaftler und -wissenschaftlerinnen an, dass Rankings wichtig für die Erlangung von wissenschaftlicher Reputation in ihrer Disziplin sei.

Diese Zahlen decken sich auch mit dem Faktor "Anzahl der Monografien" und spiegeln die unterschiedlich starke Verbreitung von Publikationsformen in den unterschiedlichen Fachdisziplinen wider. Da die Publikationsform der Monografie nur in den Geistes- und Sozialwissenschaften eine signifikante Rolle spielt (77 Prozent) gaben auch nur in dieser Fachgruppe mehr als die Hälfte der Befragten (50 Prozent) an, dass es sich bei der Anzahl der Monografien um einen wichtigen Faktor für die wissenschaftlichen Reputation sind in Ihrer Disziplin handelt. In den Ingenieurwissenschaften gaben 13 Prozent, in den Lebenswissenschaften nur 6 Prozent und in den Naturwissenschaften 4 Prozent der Befragten an, dass es sich bei der "Anzahl der Monografien" um ein wichtigen Faktor für wissenschaftliche Anerkennung in ihrer Disziplin handelt. Unter allen Befragten nannten durchschnittlich 24 Prozent diesen Faktor als wichtig für Reputation.

Einheitlicher war unter den Befragten aller Fachgruppen die Einschätzung der Rolle des Faktors "Drittmittelprojekte" für die wissenschaftliche Reputation. Dieser spielte in allen Fachgruppen mehrheitlich eine wichtige Rolle. Am häufigsten wurden "Drittmittelprojekte" in den Lebenswissenschaften (74 Prozent) genannt, gefolgt von 67 Prozent der Befragten in den Naturwissenschaften und 62 Prozent bei den Teilnehmern aus den Ingenieurwissenschaften. Mit 59 Prozent unter den Befragten Geistes- und Sozialwissenschaftler stellte die "Drittmittelprojekte" in den Geistes- und Sozialwissenschaften nur eine unterdurchschnittlich wichtige Rolle für die Reputation dar. Die durchschnittliche Auswahl dieses Faktors in allen Fachgruppen lag bei 65 Prozent.

Nur die Befragten in den Lebenswissenschaften (53 Prozent) gaben mehrheitlich an, Aufsätze, Texte oder Bücher publiziert zu haben, die vom Verlag selbst frei zugänglich gemacht wurden (z.B. in einem Open Access Journal, bei speziellen Open Access Verlagen). Weitere 11 Prozent der befragten Lebenswissenschaftlerinnen und -wissenschaftler haben bisher keine Veröffentlichungen frei zugänglich publiziert, planen das aber. In den Geistes- und Sozialwissenschaften gaben 37 Prozent der Teilnehmerinnen und Teilnehmer an, Beiträge als Open Access veröffentlicht zu haben und weitere 12 Prozent eine Veröffentlichung als frei zugänglich zu planen. 36 Prozent der Naturwissenschaftler und -wissenschaftlerinnen erklärten im Rahmen der Umfrage, dass sie mindestens einmal über einen Verlag frei verfügbar veröffentlicht haben, weitere 9 Prozent planen das in Zukunft. Unter den teilnehmenden Ingenieurwissenschaftler und -wissenschaftlerinnen haben laut der Auswertung der Befragung über ein Drittel (35 Prozent) der Befragten schon frei zugänglich veröffentlicht und weitere 10 Prozent planen diese Art der Veröffentlichung.

---- TODO: Grafik bauen ----

Der gefühlte Publikationsdruck hat laut der Auswertung der Erhebung in fast allen Fachgruppen mehrheitlich zugenommen. Nur in den Ingenieurwissenschaften gaben weniger als die Hälfte der Befragten (49 Prozent) an, das der Druck zu publizieren gestiegen ist. Demgegenüber zeigte die Auswertung der Angaben in den weiteren Fachgruppen eine klar-mehrheitlichen Anstieg des Publikationsdrucks in den vergangenen fünf Jahren. In den Lebenswissenschaften wird dieser mit 71 Prozent am stärksten wahrgenommen. In den Naturwissenschaften gaben 62 Prozent und in den Geistes- und Sozialwissenschaften 61 Prozent der Befragten an, dass der Druck für die Veröffentlichung von wissenschaftlichen Erkenntnissen gestiegen sei. Unter den Befragten der Lebenswissenschaften gab mehr als ein Viertel (26 Prozent) an, das in den letzten fünf Jahren der Druck zu veröffentlichen "sehr stark" angestiegen sei.

\subsection{Auffassungen in den unterschiedlichen Altersgruppen der Befragten}

In der Altersgruppe der 56-60 Jährigen waren prozentual die meisten Befürworter (83 Prozent) der Forderung nach kostenfreiem Zugang zu allen wissenschaftlichen Publikationen für Leser und Leserinnen. Unter den 113 Befragten 36-40 Jährigen bewerteten immerhin noch 81 Prozent diese Forderung als "sehr gut" oder "gut" und unter den Befragten in der Altersgruppe von 26-30 Jährigen unterstützten 81 Prozent die Forderung. Der Durchschnitt der Befürworter über alle Altersgruppen lag bei 77 Prozent.

---- TODO: weiter ausarbeiten & Grafik bauen ----

\subsection{Auffassungen zwischen den unterschiedlichen Statusgruppen}

Eine genauere Betrachtung des Interesses an Forschungsdaten anderer in Kombination mit der Frage nach dem "beruflichen Status" zeigte, dass vor allem bei den Doktoranden ein Interesse an den Daten anderer besteht. 79 Prozent der 118 befragten Doktoranden und 79 Prozent der 175 Doktoranden mit einer Stelle als Wissenschaftlicher Mitarbeiter gaben an, "Interesse am Zugang zu Forschungsdaten anderer Wissenschaftler_innen" zu haben. Drei Viertel der wissenschaftlichen Mitarbeiter ohne Promotion (75 Prozent) und 72 Prozent der promovierten wissenschaftlichen Mitarbeiter waren mehrheitlich an den Forschungsdaten anderer interessiert. Unter den Juniorprofessoren zeigten 62 Prozent der Befragten ein Interesse am Zugang zu Forschungsdaten anderer Wissenschaftler und Wissenschaftlerinnen. 58 Prozent der befragten Professoren und 59 Prozent der befragten Privatdozenten waren etwas weniger aber ebenfalls mehrheitlich an den Daten anderer interessiert.

---- TODO: weiter ausarbeiten &  Grafik bauen ----

\subsection{Veränderungen im Vergleich zur SOFI Studie}

Ein Vergleich der Ergebnisse der aktuellen Erhebung und der im Jahr 2007 durchgeführten SOFI-Studie in Bezug auf die sozio-demographischen Angaben zum beruflichen Status, Alter und Geschlecht der befragten Wissenschaftler und Wissenschaftlerinnen kann als Indikator für die Vergleichbarkeit der beiden Studien gewertet werden. Die Verteilung des beruflichen Status der Befragten und der Altersgruppen zeigt klare Überschneidungen. Es kann insofern von einer Vergleichbarkeit der beiden Samples ausgegangen werden.

---- TODO: Grafik bauen ----

Wie in der Befragung vom SOFI 2007 wurden die Teilnehmer und Teilnehmerinnen gefragt, wie sie sich in ihrem Fachgebiet auf dem Laufenden halten und welchen Zugang sie zu den (Voll-) Texten haben. In der aktuellen Erhebung gaben 50 Prozent der Befragten an, die Google Suche häufig als Suchmöglichkeiten zu nutzen, um gezielt nach Literatur zu suchen. Im Vergleich dazu gaben bei der Befragung im Jahr 2007 46 Prozent der Befragten an, häufig die Google Suche zu verwenden. Diese Entwicklung bestätigt den Trend nachdem die IT-gestützte Suche in der Wissenschaft seit den 1980er Jahren von einem Prozent bis 1993 auf neun Prozent anstieg und im Jahr 2003 von fast einem Viertel (24 Prozent) der Befragten Wissenschaftler für die Literaturrecherche genutzt wurde \cite{hanekop_2008}.

---- TODO: Zahlen wegen anderer Grundgesamtheit pruefen und Grafik bauen ----

Bei dem Vergleich der Umfrageergebnisse von 2007 und 2014/2015 in Bezug auf die Frage, wie sich die Teilnehmer in Ihrem Fachgebiet auf dem Laufenden halten, zeigte sich ebenfalls ein klarer Trend zur stärkeren Nutzung digitaler Anwendungen. In der Göttinger Befragung im Jahr 2007 gaben noch 57 Prozent an, sich "sehr häufig" oder "häufig" über Online-Zeitschriften auf dem aktuellen Stand zu halten. In der Befragung im Rahmen dieser Arbeit waren es bereits 68 Prozent der 1.446 Befragten, die sich über Online-Zeitschriften über den aktuellen Stand der Forschung informierten.

Neben den Veränderungen durch die Digitalisierung stellen aber weiterhin die Teilnahme an Tagungen oder Kongressen (56 Prozent) und Gespräche mit Fachkollegen (55 Prozent) für die befragten Wissenschaftler und Wissenschaftlerinnen wichtige Möglichkeiten dar, "sich im Fachgebiet auf dem Laufenden zu halten". Social Media-Plattformen sind mit knapp 6 Prozent von geringer Relevanz. Die Information über Online-Datenbanken, Online-Archive, die 2007 noch zweithäufigste Option sich auf dem Laufenden zu halten, bleibt auf gleichem Niveau.

---- TODO: Zahlen pruefen wegen Grundgesamtheit 1.446 und Grafik bauen ----

Im Jahr 2007 bewerteten rund 81 Prozent der Befragten die Forderung nach kostenfreiem Zugang zu allen wissenschaftlichen Publikationen für Leser als "gut" oder "sehr gut". In der Befragung 2014 fiel das Ergebnis mit 77 Prozent und einer Teilnehmerzahl von 1.112 zwar niedriger aber dennoch weiterhin mehrheitlich positiv aus.

Ein weiteres Ergebnis der Studie des Soziologischen Forschungsinstituts in Göttingen aus dem Jahr 2007 war, dass "gerade auch die etablierten und damit etwas älteren Wissenschaftler internetbasierte Plattformen intensiv nutzen". Nach den Ergebnissen der aktuellen Befragung kann diese Entwicklung bestätigt werden, so gaben 86 Prozent der über 50-jährigen Befragten an, sich mit "Online-Ausgaben von Zeitschriften" "häufig auf dem Laufenden zu halten". In dieser Altersgruppe griffen jedoch auch noch immer 50 Prozent zu "Print-Ausgaben von Zeitschriften". In der Altersgruppe unter 50 Jahren nutzen nur noch 29 Prozent die Print-Ausgaben von Zeitschriften. Print-Bücher hingegen wurden altersgruppenunabhängig von 53 bis 54 Prozent der Befragten verwendet. Um auf dem Laufenden zu bleiben, griffen 20 Prozent der über 50 Jährigen auf digitale Bücher zurück. Mit 39 Prozent verwendeten fast doppelt so viele der unter 50 jährigen Befragten digitale Bücher um sich in dem jeweiligen Fachgebiet auf dem Laufenden zu halten, wie über 50 Jährige.

---- TODO: Grafik bauen ----

In der Studie 2007 gaben insgesamt 80 Prozent an sich mit Onlineausgaben von wissenschaftlichen Beiträgen auf dem Laufenden zu halten. Sieben Jahre später stieg die Nutzung nochmals um über 8 Prozent auf 88 Prozent an. Die Situationen in denen die Befragten nicht auf die Online-Version eines Aufsatzes zugreifen können, weil es keine Lizenz vorliegt wurde ebenfalls seltener. Gaben 2007 noch 45 Prozent an, häufig bis sehr häufig nicht auf Aufsätze und Texte online zugreifen zu können, waren es in der aktuellen Befragung nur noch 32 Prozent. 67 Prozent der teilnehmenden Wissenschaftler gaben an nur gelegentlich bis nie Probleme mit dem Zugang zu Onlinetexten zu haben. In der Befragung 2007 waren es nur 52 Prozent.

---- TODO: Grafik bauen ----

Seit der Umfrage aus dem Jahr 2007 ist eine sehr leichte Verschiebung zu­guns­ten der besseren Verfügbarkeit von digitalen Texten für die Wissenschaftler und Wissenschaftlerinnen festzustellen. So gaben im Jahr 2007 45 Prozent der Befragten an, "sehr häufig" oder "häufig" nicht auf die Online-Version eines Textes zugreifen zu können. Laut der Erhebung von 2014/2015 waren es 42 Prozent der befragten Wissenschaftlerinnen und Wissenschaftler, die angaben "sehr häufig" oder "häufig" nicht auf Onlineinhalte zugreifen zu können. "Gelegentlich" konnten 2007 38 Prozent und 2014/2015 die Hälfte (50 Prozent) nicht auf die Webversion von Inhalten zugreifen, weil es zum Beispiel keine Lizenz dafür gab. "Selten" oder "nie" Probleme mit dem Online-Zugriff auf Texte hatten im Jahr 2014/2015 17 Prozent der Befragten, im Jahr 2007/2008 waren es noch 14 Prozent.

In der Göttinger Erhebung von 2007 gaben 34 Prozent der Befragten an, mehr als einen Beitrag veröffentlicht zu haben, der vom Verlag selbst frei zugänglich gemacht wurden, im Jahr 2014/2015 waren es nur 26 Prozent. 14 Prozent hatten laut der Befragung 2014/2015 einen Beitrag veröffentlicht, in der Befragung 2007/2008 waren es noch 23 Prozent. Eine frei zugängliche Publikation zu planen, gaben in 2007/2008 (11 Prozent) ähnlich viele Befragte an, wie in der aktuellen Erhebung (11 Prozent). Markante Unterschiede gab es in diesem Zusammenhang auch bei der Anzahl der Personen die angaben, bisher keine offenen Publikationen veröffentlicht haben und das auch nicht planen. Gaben im Jahr 2007/2008 knapp ein Drittel (32 Prozent) an, keine frei zugängliche Publikation veröffentlicht zu haben oder zu planen, war es in der aktuellen Befragung fast die Hälfte der Befragten (49 Prozent).

---- TODO: Grafik mit Vergleich bauen ----

Bei der Frage, "welche Faktoren für wissenschaftlichen Reputation sind in Ihrer Disziplin wichtig" sind, wurde im Jahr 2007 von 92 Prozent und im Jahr 2014/2015 von 74 Prozent der Befragten die "Relevanz der Ergebnisse" als "wichtig" oder "sehr wichtig" bewertet. Demgegenüber erachteten die Teilnehmer und Teilnehmerinnen in der aktuellen Erhebung die "Anzahl der Aufsätze / Beiträge" (80 Prozent) am häufigsten als wichtigen Faktor für die Reputation in der jeweiligen Disziplin. In 2007 stellten für 82 Prozent der Befragen die "Bezugnahme beziehungsweise die Zitation" durch Kollegen und Kolleginnen einen wichtigen oder sehr wichtigen Faktor für wissenschaftliche Reputation dar, in der aktuellen Befragung bewerteten diesen Faktor 66 Prozent der Befragten als "wichtig".

Die Frage, ob der Publikationsdruck in dem Fachgebiet der jeweiligen Befragten in den vergangenen fünf Jahren zugenommen hat, wurde von den Teilnehmer und Teilnehmerinnen in der SOFI-Studie von 2007 ähnlich wie in der aktuellen Erhebung bewertet. In der aktuellen Umfrage gaben 18,4 Prozent und in 2007 21,9 Prozent der Befragten an, einen sehr starke Zunahme des Publikationsdrucks zu registrieren. Für 22 Prozent der Befragten in 2014/2015 und 20 Prozent der Befragten in 2007 ist die Zunahme unverändert geblieben und der Publikationsdruck hat nicht zugenommen. Die in beiden Erhebungen größte Gruppe antwortete mit "ja" (2007: 44 Prozent und 2014/2015: 43 Prozent). Unsicher waren bezüglich einer Aussage zur Zunahme des Publikationsdrucks waren sich in 2014/2015 17 Prozent und in der Befragung vor 7 Jahren 14 Prozent der Befragten.

Ob die befragten Wissenschaftler und Wissenschaftlerinnen ihre Veröffentlichungen in Zeitschriften oder Büchern für potentielle Leser als gut zugänglich empfinden, beantworteten in der aktuellen Erhebung 32 Prozent mit "ja", im Jahr 2007 waren es noch xx Prozent. Mit "teils/teils" antworteten in beiden Erhebungen jeweils mit 47 Prozent in 2007 und 47 Prozent in 2014. Die Anzahl der Befragten, die ihre Veröffentlichungen in Zeitschriften oder Bücher für potentielle Leser für nicht gut zugänglich erachteten, stieg leicht von 6 Prozent in 2007 auf 9 Prozent in 2014 an.

---- TODO: Grafik mit Vergleich bauen ----

Die Frage nach dem Aufwand, für die freie Veröffentlichung von Publikationen im Internet bewerteten in der aktuellen Befragung 31 Prozent und 2007 33 Prozent der Befragten als "gering". Mittelgroßen Aufwand vermuteten 2007 23 Prozent und 2014 25 Prozent. Unsicher waren laut der aktuellen Erhebung 23 Prozent und 2007 20 Prozent der befragten Wissenschaftler und Wissenschaftlerinnen. Einen großen Aufwand schätzen 2014 5 Prozent und in der SOFI-Studie 2007 4 Prozent. Keine Schätzung wussten in beiden Befragungen rund 19 Prozent der Befragten abzugeben.

---- TODO: Vergleich "ob die Befragten ihre Veröffentlichungen in Zeitschriften oder Büchern für potentielle Leser gut zugänglich befinden" ----

\section{Zwischenergebnis: Interesse in der wissenschaftlichen Gemeinschaft an dem Zugang zu, Zugriff auf wissenschaftliche Kommunikation und Verbreitung der Öffnung der wissenschaftlicher Kommunikation}

Bei der Betrachtung des Interesses an der Öffnung wissenschaftlicher Kommunikation bestanden Vornanahmen, die von einer geringen Motivation der wissenschaftlichen Gemeinschaft für Veränderungen am System der wissenschaftlichen Kommunikation \cite{hagner_2015_sache_buches} und von einer geringen praktischen Verbreitung der Konzepte um die Öffnung wissenschaftlicher Kommunikation ausgingen \cite{Scheliga_2014}. In der Auswertung der Befragung von 1.112 Wissenschaftlern im Rahmen dieser Arbeit konnte allerdings ein mehrheitlich stark verbreitetes Verständnis von Open Access und die mehrheitliche Unterstützung der Forderung nach Öffnung von Wissenschaft sowie ein Interesse an Forschungsdaten anderer nachgewiesen werden. Diese Merkmale sind jedoch vor allem nach Alter, wissenschaftlichem Status und Fachzugehörigkeit unterschiedlich stark ausgeprägt.

\subsection{Verbreitung von und das Interesse an Offenheit nach Alter}

Das Interesse an Forschungsdaten in der Befragung war grundsätzlich hoch und 71 Prozent der Befragten bekundete Interesse an den Forschungsdaten anderer. 64 Prozent aller Befragten konnte sich grundsätzlich vorstellen unter bestimmten Bedingungen ihre Forschungsdaten und alle weiteren Informationen, die während Ihrer wissenschaftliche Arbeit anfallen unter Berücksichtigung von Datenschutz öffentlich zur Verfügung zu stellen, 28 Prozent sogar nur unter den Einschränkungen des Datenschutz. Dabei variierten die Unterschiede zwischen den Altersgruppen nur leicht.

\begin{figure}[h!]
\includegraphics{graphid:WYo8m}
\caption{Bereitschaft zur Öffnung von Forschungsdaten nach Alter}
\end{figure}

In anderen Befragungen wurde vor allem bei jüngeren Altersgruppen befragter Forscherinnen und Forscher eine spezielles Interesse identifiziert, die Daten nicht ohne Einschränkungen zu veröffentlichen, während die über 50-Jährigen weniger bedenken äußerten \cite{Tenopir_2011}. Es wurde angenommen, dass das mit Bedenken hinsichtlich der Besitzverhältnisse und der beruflichen Entwicklung zusammenhängt \cite{Tenopir_2011}.

Auf die Frage ob schonmal Texte oder Bücher als Open Access veröffentlicht wurden, antworteten 78 Prozent der unter 30 Jährigen mit nein. Bei den 31-40 Jährigen verneiten noch 57 Prozent, bei den 41-50 Jährigen mit 49 Prozent nur noch knapp weniger als die Hälfte und bei den über 50 Jährigen 46 Prozent jemals eine Publikation unter den Bedinungen von Open Access veröffentlicht zu haben.

\begin{figure}[h!]
\includegraphics{graphid:k8Kht}
\caption{Haben Sie Aufsätze, Texte oder Bücher als Open Access publiziert?}
\end{figure}

Insgesamt gab die Mehrheit aller Befragten (60 Prozent) gab an noch nicht unter den Bedingungen von Open Access publiziert zu haben. Je nach Alterstgruppe kündigten jedoch 9-12 Prozent der Befragten an, es bisher noch nicht getan zu haben, es aber zu planen.

\begin{figure}[h!]
\includegraphics{graphid:pUUEG}
\caption{Prozent der Befragten die geplant haben Aufsätze, Texte oder Bücher als Open Access Publikation zu veröffentlichen}
\end{figure}

---- TODO: Grafik mit Altersvergleich Vorstellung Daten zu veröffentlichen bauen  ----

\subsection{Verbreitung von und das Interesse an Offenheit nach wissenschaftlichem Status }

---- TODO: ausarbeiten  ----

\subsection{Verbreitung von und das Interesse an Offenheit in den verschiedenen Disziplinen}

Das Interesse an Forschungsdaten anderer Wissenschaftler war in allen Disziplinen ähnlich stark ausgeprägt. Die Vision von Open Access hingegen fand auf hohem Niveau unterschiedlich viel Unterstützung. Im Detail gibt es aber gravierende Unterschiede zwischen den Fachgruppen bei der Bewertung und der praktischen Umsetzung von offener wissenschaftlicher Kommunikation. Diese Entwicklung und eventuelle Gründe werden im Folgenden dargestellt und diskutiert.

---- TODO: Grafik mit Fachgruppenvergleich der Unterstützung von Forderung OA und Interesse and Forschungsdaten bauen  ----

Die durchgeführte Befragung zeigt, dass bei der überwiegenden Mehrzahl der Befragten ein grundsätzliches Verständnis für die Forderung nach Offenheit in der wissenschaftlichen Kommunikation vorherrscht (96 Prozent) und 75 Prozent eine gängige Definitionen von Open Access befürworten. 71 Prozent der Befragten zeigte zudem Interesse am Zugang zu Forschungsdaten anderer Wissenschaftler und Wissenschaftlerinnen zeigt. 29 Prozent der Befragten gaben an, kein Interesse an den Daten anderer zu haben.

49 Prozent gaben an, gelegentlich und 32 Prozent häufig nicht auf die digitale/Online-Version eines Textes zugreifen zu können. Diese Zahlen überraschen, da bisher angenommen wurde, dass die meisten Wissenschaftler und Wissenschaftlerinnen an einer komfortablen Stelle des wissenschaftlichen Produktions- und Distributionssystems \cite{herb_2010} stehen, in der sie durch Lizenzen der Forschungsinstitutionen und Universitäten, die Konsequenzen der Zeitschriften- und Publikationskrise beim Zugriff auf wissenschafltiche Beiträge selbst nicht zu spüren bekommen.

---- TODO: Grafik Vergleich zu SOFI Daten bauen ----

Demgegenüber haben nur 36 Prozent der Befragten angegeben, bisher Aufsätze, Texte oder Bücher publiziert zu haben, die frei zugänglich waren und 38 Prozent stellen laut eigenen Angaben Volltexte auf den eigenen oder Institutswebseiten zur Verfügung. 32 Prozent bewerteten die Zugänglichkeit zu ihren Veröffentlichungen für potentielle Leser als gut. Diese Zahlen stützen die Annahmen in der Literatur, nach denen Wissenschaftler und Wissenschaftler  Open Access als Rezipienten mehrheitlich bejahen, als Autoren jedoch wenig oder nur partiell genuines Interesse an Open Access haben \cite{wein_2010_erwerbung}.

Im Folgenden werden die unterschiedlichen Disziplinen genauer betrachtet. Dabei wird vermutet, das eklatante unterschiede beim Interesse am und der Verbreitung von Offenheit beim Publikationsverhalten unter den Geisteswissenschaften, Lebenswissenschaften, Ingenieurwissenschaften und den Naturwissenschaften vorherrschen.

\subsubsection{Geisteswissenschaften}

Dass die Geisteswissenschaften am geringsten unter allen befragen Fachgruppen, aber dennoch mehrheitlich, die Forderung nach kostenfreiem Zugang zu allen wissenschaftlichen Publikationen für Leser (Open Access) zustimmen, deckt sich mit dem Stand der Verbreitung von Open Access in der Fachrichtung. Dabei spielt die Publikationsform der Monografien nur in den Geistes- und Sozialwissenschaften eine wichtige Rolle, ebenfalls wurden nur in dieser Fachgruppe deutsche Zeitschriften als wichtig erachtet  (65 Prozent). Das deckt sich mit den Aussagen in der Literatur \cite{hagner_2015_sache_buches} \cite{naeder_2010_open} \cite{hollricher_wandel_2009} \cite{Lossau_oa_2007}. Die Auswertung der Ergebnisse der Befragung zeigt aber auch bei den Geistes- und Sozialwissenschaftler und -wissenschaftlerinnen ein mehrheitliches Interesse der an den Forschungsdaten anderer (70 Prozent). Betrachtet man den Austausch von Daten als eine erweiterte Möglichkeit, Wissen zu überprüfen und Verzerrungen und Fehler zu beseitigen, erscheint es dennoch verwunderlich, dass 30 Prozent der befragten Wissenschaftler und Wissenschaftlerinnen der Fachgruppe kein Interesse daran haben.

---- TODO: Grafik bauen ----

Auch insgesamt zeichnet die Auswertung der Antworten von 418 Wissenschaftlern und Wissenschaftlerinnen aus der geistes- und sozialwissenschaftliche Fachrichtung ein eher ambivalentes Bild bezüglich dem Wunsch nach Öffnung wissenschaftlicher Kommunikation und der tatsächlich praktizierten Offenheit. So erachten nur 25 Prozent der Befragten Geisteswissenschaftler ihre eigenen Beiträge als gut zugänglich. Dass die überwiegende Mehrheit der Befragten die freie Verfügbarkeit des eigenen Volltexts im Internet (62 Prozent) und die Veröffentlichung unter einer Open Access Lizenz (70 Prozent) als eher weniger wichtig oder unwichtig erachtet, lässt dennoch auf ein gewisses Desinteresse schließen. Für alle anderen Fachgruppen hat die freie Verfügbarkeit der eigenen Texte eine höheren Stellenwert.

---- TODO: Grafik bauen ----

Demgegenüber gaben 43 Prozent der Befragten an, Volltexte selber auf Webseiten zur Verfügung zu stellen oder stellen zu lassen und 37 Prozent der Teilnehmer und Teilnehmerinnen haben schon mindestens einmal ihre Inhalte frei zugänglich publiziert. Erstaunlich ist, dass Sie auch die Gruppe derer stellen, die sich am Besten vorstellen können, Forschungsdaten und alle weiteren Informationen, die während Ihrer wissenschaftliche Arbeit anfallen unter bestimmten Bedingungen, öffentlich zur Verfügung zu stellen (67 Prozent).

---- TODO: Grafik bauen  ----

Daraus lässt sich schließen, dass unter den Wissenschaftlern und Wissenschaftlerinnen dieser Fachrichtung zwar ein grundsätzliches Interesse am Zugang zu Forschungsdaten anderer Forscher und Forscherinnen besteht (70 Prozent) und auch mehrheitliche, wenn auch unter allen Fachgruppen am geringsten Ausgeprägte Zustimmung zu der Forderung nach kostenfreiem Zugang zu allen wissenschaftlichen Publikationen für Leser (Open Access) vorherrscht (68 Prozent), diese aber in der praktischen Arbeit keine große Rolle spielt (37 Prozent). Das mag dadurch begründet sein, dass die eigene Zugänglichkeit zu Publikationen durch die Wissenschaftler in der Fachgruppe überwiegend als gut oder sehr gut bewertet wird und dass die Entwicklungen rund um die Öffnung von Wissenschaft und Forschung, wie im Grundlagen Kapitel dargestellt, eher aus den STM-Fächern kommen.

Dennoch verwundert diese ambivalente Haltung, denn gerade für die Geistes- und Sozialwissenschaften lässt sich ein besonderes Interesse an die Verbreitung von Wissen innerhalb der wissenschaftlichen Community und auch an die Gesamtgesellschaft vermuten \cite{suchen}. Diese Vermutung wird unterstrichen, dass in den Geistes- und Sozialwissenschaften 83 Prozent der Befragten die Anzahl der Aufsätze und Beiträge, am stärksten unter allen befragten Fachgruppen, als wichtigen Faktor für Reputation in ihrer Disziplin erachteten. Wobei sich es sich bei den Zahlen statt um einen Indikator für das Interesse an Verbreitung wissenschaftlicher Informationen, auch um die ungewollte Konsequenz der Leistungsbewertung nach rein quantitativen Kriterien handeln kann \cite{wr_2015_wissenschaft_integritaet}.

\subsubsection{Lebenswissenschaften}

In der Gruppe der Lebenswissenschaften findet die Forderung nach kostenfreiem Zugang zu allen wissenschaftlichen Publikationen für Leser (Open Access), die stärkste Zustimmung unter allen vier Fachgruppen (88 Prozent). Mit 98 Prozent sind in den Lebenswissenschaften internationale Zeitschriften wichtigste Publikationsform. Die Befragten bewerteten die Zugänglichkeit zu ihren eigenen Beiträge mehrheitlich als nicht so gut, schlecht oder teil/teils (68 Prozent). Zwei Drittel der Befragten Lebenswissenschaftler und -wissenschaftlerinnen gab an, Interesse am Zugang zu Forschungsdaten anderer zu haben. Auch hier ist es verwunderlich, dass ein Drittel kein Interesse an dem Zugang zu den Daten anderer Forscher hat.

---- TODO: Grafik bauen  ----

Die 197 Befragten aus den Lebenswissenschaften bekennen sich eindeutiger zur Öffnung der wissenschaftlichen Kommunikation als die Geistes- und Sozialwissenschaftler. Der freie Zugang zu den eigenen wissenschaftlichen Beitragen wird in der Fachgruppe von 61 Prozent als wichtig oder sehr wichtig betrachtet und 54 Prozent der Lebenswissenschaftler gaben an, dass es ihnen mindestens wichtig ist, unter einer Open-Access-Lizenz zu veröffentlichen. Demnach ist in dieser Fachgruppe nicht nur die stärkste Zustimmung zur Öffnung wissenschaftlicher Kommunikation zu verzeichnen, sondern auch der stärkste praktische Verbreitungsgrad. Dass 53 Prozent der Befragten bereits unter frei zugänglich, zum Beispiel unter einer Open-Access-Lizenz, publiziert haben, bestätigt diese Einschätzung.

---- TODO: Grafik bauen  ----

Der Zustand kann darauf zurückgeführt werden, dass die Zugangsmöglichkeiten insgesamt zu wissenschaftlichen Publikationen durch die Wissenschaftler in der Fachgruppe überwiegend als schlecht beurteilt werden. Das stützt ebenfalls die These aus der Literatur \cite{suchen}, dass die Öffnung am stärksten in den Fachgruppen vorangetrieben wird, bei denen sich die Krisen am stärksten für die Wissenschaftler und Wissenschaftlerinnen bemerkbar machen. Dennoch nur 36 Prozent erachten die eigenen Veröffentlichung als gut zugänglich.

---- TODO: Vergleich Entwicklung mit SOFI-Studie einbauen als Beleg ----

\subsubsection{Naturwissenschaften}

In den Naturwissenschaften unterstützten 82 Prozent der 322 Befragten die Forderung nach Open Access. Wie in den Lebenswissenschaften wurde auch in den Naturwissenschaften die internationale Zeitschrift als wichtigste Publikationsform genannt (98 Prozent). 39 Prozent der Befragten gaben an, dass sie ihre Veröffentlichungen für potentielle Leser als gut zugänglich bewerten. Mit nur 69 Prozent gab im Fachgruppenvergleich die kleinste Gruppe der Befragten an, Interesse am Zugang zu Forschungsdaten anderer Wissenschaftler und Wissenschaftlerinnen zu haben.

---- TODO: Grafik bauen  ----

36 Prozent erachten es als wichtig oder sehr wichtig unter einer Open-Lizenz zu veröffentlichen und für 52 Prozent spielt der freie Zugang und die Veröffentlichung im Internet zu den eigenen wissenschaftlichen Beitragen eine wichtige oder sehr wichtige Rolle. 36 Prozent gaben an bereits frei zugänglich publiziert zu haben.

Insgesamt befinden sich die Gruppe der Naturwissenschaftler und -wissenschaftlerinnen damit im Fachgruppenvergleich auf niedrigem Niveau im Mittelfeld, wenn es um die Verbreitung offener Kommunikation und die tatsächlichen Umsetzung geht.

Das ist auf der einen Seite verwunderlich, denn die Entwicklung der Forderung von Offenheit in Wissenschaft und Forschung wird neben den Fächern der Lebenswissenschaften auch den Naturwissenschaften und auf der Publikationsform Zeitschrift zugeschrieben wird \cite{suchen}, auf der anderen gaben 78 Prozent der Befragten an, über gute oder sehr gute Zugangsmöglichkeiten zu wissenschaftlichen Online-Zeitschriften über eine Lizenz ihrer Forschungseinrichtung zu erhalten (93 Prozent) und scheinen somit keinen direkten Veränderungsdruck zu erleben.

---- TODO: Vergleich Entwicklung mit SOFI-Studie einbauen als Beleg ----

\subsubsection{Ingenieurwissenschaften}

Die befragten Wissenschaftler und Wissenschaftlerinnen unterstützen ebenfalls mehrheitlich die Forderung nach Open Access (72 Prozent). Knapp einem Drittel (32 Prozent) ist es wichtig oder sehr wichtig unter einer Open-Lizenz zu veröffentlichen, während 42 Prozent der freie Zugang zum Volltext im Internet zu den eigenen wissenschaftlichen Beitragen wichtig oder sehr wichtig.

34 Prozent haben bereits frei zugänglich publiziert und 32 Prozent finden, dass Ihre Veröffentlichungen in Zeitschriften oder Büchern für potentielle Leser gut zugänglich. Wie bei den Lebenswissenschaften gaben zwei Drittel der Befragten an, Interesse am Zugang zu Forschungsdaten anderer Wissenschaftler und Wissenschaftlerinnen zu haben.

Auch in den Ingenieurwissenschaften ist mit 90 Prozent die internationale Zeitschrift die wichtigste Publikationsform. Nur 37 Prozent erachten deutschsprachige Zeitschriften als wichtig. Monografien spielen nur für ein Drittel (35 Prozent) der Befragten eine wichtige Rolle.

---- TODO: weiter ausarbeiten ----
