\chapter{Literaturanalyse zu Open Access und Open Science als Grundlage für die empirische Forschung}
Von hervorzuhebenen Interesse ist der aktuelle Forschungsstand zur Öffnung von Wissenschaft, zu den Treibern und Bremsern dieser Entwicklung und dem damit einhergehenden Paradigmenwechsel mit Fokus auf wissenschaftliche Reputation. Anhand ausgewählter relevanter und aktueller Werke der Fachdiskussion, die sich mit dem Phänomen Öffnung von Wissenschaft auseinandersetzen. Ziel des Kapitels ist die Entwicklung geeigneter wissenschaftlicher Fragestellungen für die Befragung von publizierender Wissenschaftler verschiedener Fachbereiche.

Welche Argumente werden für und welche gegen eine Öffnung der Wissenschaft angeführt. Welche Möglichkeiten fördern und welche Grenzen limitieren diesen Prozess. Pro- und Kontraargumente werden zusammenfasst und ein Überblick über die aktuelle Debatte um Open Science und Open Access gegeben. Diese Analyse basiert auf der Annahme, dass sich Open Access in einer Übergangsphase von der reinen offenen Bereitstellung wissenschaftlicher Publikationen und dem damit verbundenen offenen Zugang zur Wissenschaft zur umfassenden und offenen Wissensverteilung und dem damit einhergehenden Zugriff auf Wissenschaft für die Gesamtgesellschaft (Open Science) befindet. Medienkulturwissenschaftlich werden Open Science und Open Access in ihren technischen, gesellschaftlichen und politischen Aspekten, sowie den kulturellen Auswirkungen der Medienbrüche im Rahmen digitalen Publizierens reflektiert. Am Ende dieses Kaptitels werden Treiber und Bremser für die Öffnung von Wissenschaft benannt, für die Befragung extrapoliert und in der Gesamtbetrachtung der Arbeit zusammengeführt und strukturiert ausgewertet.

\section{Beschreibung des Forschungsstandes}
Das Geschäftsmodell hinter der derzeitigen wissenschaftlichen Kommunikation ermöglicht den Verlegern Betriebsgewinnmargen von über 35 Prozent \cite{russell_2008_business} und hohe jährliche Wachstumsraten \cite{Wellcome_Trust_2003}. Sucht man nach Gründen für die Unterstützung des bisherigen Modells durch die Wissenschaftsgemeinschaft, wird deutlich, dass vor allem die in --- TODO in welchen Kapitel? --- beschriebene wissenschaftliche Reputation einen zentral extrinsischen Motivationsfaktor für Wissenschaftler darstellt \cite{minssen_2012_arbeit}. Die akademische Reputation „ist [dabei] die zentrale Kommunikationsform, die das Wissenschaftssystem charakterisiert“ \cite{Rutenfranz_1997}. Die Ergebnisse aus wissenschaftlicher Forschung werden dabei als Publikationen vor allen Mitgliedern der Wissenschaft präsentiert, „um diese intern von der Wissenschaftsgemeinde als wissenschaftlich beziehungsweise unwissenschaftlich zertifizieren zu lassen" \cite{Rutenfranz_1997}.

--- TODO Forschungstand zur Debatte Open Access / Open Science ---

\section{Forschungsfragen} 

---- TODO: Fragestellungen nicht eher für die ganze Arbeit. In der Literaturanalyse eher die Debatte evaluieren? -----
Folgende Forschungsfragen sollen bei der Literaturanalyse genauer betrachtet werden:
\begin{itemize}
\item Wie werden Open Science und Open Access definiert und voneinander abegrenzt? 
\item Was sind die Gründe für die Forderungen nach Öffnung in Wissenschaft und Forschung? 
\item Welche Pro- und Kontraargumente gibt es für die Öffnung von Wissenschaft? 
\item Warum ist die Öffnung von Wissen in den verschiedenen wissenschaftlichen Disziplinen unterschiedlich stark etabliert? 
\item Was bedeutet Offenheit und freier Zugang im Rahmen des wissenschaftlichen Diskurs-, Reputations- und Machtbegriffs?
\end{itemize}	

\section{Erhebungsmethode und Umfang} 

Die Literaturanalyse ermöglicht es existierenden Erkenntnisse darzulegen und im aufzuzeigen in welchen Bereichen weitere Forschung angestrebt werden sollte"\cite{webster2002analyzing}. Für die Analyse wurden xxx Quellen ausgewählt und analysiert. --- Todo ausführen ----

\section{Analyse von Open Access: Zugang zu wissenschaftlicher Kommunikation} 

Der bisherige Prozess wissenschaftlicher Kommunikation steht vor großen Herausforderungen. Die Zeitschriften- und Monographienkrise, der zunehmende finanzielle Druck, sowie die Veränderungen im wissenschaftlichen Kommunikationsprozess durch neue Arten und Möglichkeiten der Distribution, die steigenden Beschaffungskosten für wissenschaftliche Literatur \cite{cite:17}, sowie die Veränderungen in der Rezeption von Inhalten \cite{holub_2013_reception}, zwingen zum Umdenken in der wissenschaftlichen Kommuinkationspraxis \cite{suchen}. Die steigende Forderung nach mehr Offenheit im wissenschaftlichen Kommunikationsprozess, entwickelte sich zu einem konkreten Ansatz für dieses Umdenken. Nachfolgend wird die Debatte um das Modell des Offenen Zugangs zu Wissenschaft erläutert, analysiert und abgerenzt.

Der Schwerpunkt beruht dabei auf den Themenbereichen wissenschaftliche Reputation und (Effizienz der) Kommunikation. Dieser Zugang beruht auf der Annahme, dass Offenheit eine große Chance für Veränderungen im wissenschaftlichen Qualitäts- und Reputationssystem darstellt. Das bezieht sich insbesondere auf die Aktivität der Wissenschaftler, sowie die Qualtiät der Forschungsergebnisse. Die Erkenntnisse wurden bisher häufig erst nach langen intransparenten Verfahren bewertet und publiziert, sowie an einen beschränkten Kreis von Rezipienten vermittelt. Das hat einen signifikanten Einfluss auf Kosten, die im Rahmen öffentlich-finanzierter Forschung entstehen \cite{suchen}. Der Fokus liegt insbesondere auf dem generellen Zugang zu wissenschaftlichen Informationen im Rahmen des "klassischen" Kommuniktations- und Publikationsprozesses. 

Dabei muss zwisschen Open Access und Open Science unterschieden werden. Im Rahmen von Open Access ist dabei nicht zwingend der Zugriff auf Informationen oder Daten, die bei Erstellung der Publikation entstehen, eingeschlossen. 

Als Grundlage für die Entwicklung von Open Access werden vor allem die infrastrukturellen Veränderungen angeführt, die "seit spätestens Mitte der 1990er-Jahre entscheidend Einfluss auch auf die Wissenschaftskommunikation und das wissenschaftliche Arbeiten genommen haben" \cite{schulze_2013_open}. Wissenschaftliche Informationen werden seither nicht nur in "digitaler Form konsumiert, sondern auch kollaborativ und kooperativ, zeitlich versetzt, durch teilweise räumlich weit verstreute Arbeitsgruppen und Forschungsverbünde, genutzt und weiterverarbeitet" \cite{schulze_2013_open}. Die Verbreitung und Akzeptanz von Open Access variiert zwischen den einzelnen wissenschaftlichen Disziplinen erheblich \cite{cite:21a}.

\subsection{Definition von Open Access}
--- TODO: Prüfen, ob das hier reinpasst ----
Open Access wurde von Peter Suber als "digital, online, kostenlos, und frei von den meisten Urheber- und Lizenzbeschränkungen" \cite{suber_2012_open} definiert \cite{Adema_2014_open_access}. Open Access bedeutet den "Verzicht auf die finanzielle, technische und rechtliche Hindernisse, die dazu bestimmt sind, den Zugang zu wissenschaftlichen Forschungsartikel für zahlende Kunden zu begrenzen" und dass, "im Interesse der Beschleunigung der Forschung und den Austausch von Wissen, Verlage ihre Kosten aus anderen Quellen schöpfen" \cite{Suber_2002}. Abgesehen von dieser Meta-Definition gibt es unterschiedliche Auffassungen über Open Access, wie es erreicht werden kann und der Definition des Attributs "Open" \cite{Adema_2014_open_access}. Das ist darauf zurückzuführen, dass es "keine formelle Struktur, keine offizelle Organisation und kein ernannter Führer" gibt, der die Open Access Bewegung antreibt \cite{poynder_2011_suber}. Was dennoch alle diese Auffassungen vereint, ist "die Mission, die Bedingungen zu verbessern, unter denen wissenschaftliche Arbeiten zirkulieren können"\cite{Adema_2014_open_access}. Eine der verbreiteten Definitionen von Open ist die "Open Definition".

\subsubsection{Open Definition}

Die Open Definition hat den Anspruch die Prinzipien, die "Open" definieren für Daten und Inhalt zu definieren. Nach der "Open Definition" gilt der Inhalt als als "Open", der "für jeden Zweck von jedem kostenlos genutzt, modifiziert und geteilt werden" \cite{open_definition} kann. Ziel der Definition ist es, "die Bedeutung von offen in Bezug auf Wissen" zu präzisieren. Der Begriff "Wissen" beinhaltet Inhalte wie Musik, Filme, Bücher, jegliche Art von Daten, ob wissenschaftlicher, historischer, geographischer oder anderer Art und Regierungs- und andere Verwaltungsinformationen \cite{open_definition}.

Die Definition wurde von der Open Scource Definition abgeleitet und ist als synonym für "frei" oder "libre" im Rahmen der Definition für "Freie kulturelle Werke" zu verstehen.

Ein Werk oder Inhalt gilt nach dieser Definition als "offen", wenn es bei der Verbreitung folgenden Kriterien erfüllt:
--- Todo: Grafik von http://de.slideshare.net/christianheise/gfm-open-whaaaaaaat102013bmch-27451852 ---

\subsection{Open Access Modelle und Formen}

In der Literatur wird Open Access in unterschiedliche Formen unterteilt und es existieren mehrere Definitionen \cite{CREATe_2014} \cite{albert_2006_open_implications}. Es bestehen unterschiedliche Auffassungen über die verschiedenen Modelle von Open Access \cite{CREATe_2014} \cite{cite:22b} \cite{lewis_2012_inevitability}. Trotz dieser orientieren sich alle Modelle und Definitionen an den "three Bs" --- TODO siehe Kapitel XXXX --- , den derzeit geltenden Definitionen von Open Access \cite{Adema_2014_open_access}. Am Beispiel der Budapest Open Access Initiative werden zwei Wege für Open Access dargestellt \cite{albert_2006_open_implications}: 
\begin{enumerate}
\item Einrichtung "einer neuen Generation von Fachzeitschriften," die einen kostenfreien und unmittelbaren Zugang zu den Beiträgen ermöglichen ("goldener" Weg)
\item öffentlich zugängliche (Selbst-)Archivierung durch den Urheber ("grüner" Weg)
\end{enumerate}

Der "grüne Weg" beschreibt das Modell, in dem der Autor im Rahmen einer (Selbst-)Archivierung von Beiträgen in Repositorien (öffentlichen Dokumentenservern) anstrebt \cite{suchen}. Das vom Autor inital eingereichte Dokument (Manuskriptfassung) steht dabei als Pre-Print oder Post-Print-Version auf meist institutionellen oder disziplinären Dokumentenservern \cite{suchen} oder privaten Homepages \cite{suchen} jedem zur Verfügung. Im Unterschied zu Post-Prints, hat bei Pre-Print keine Peer Review stattgefunden \cite{suchen} und der Beitrag hat somit keine externe wissenschaftliche Qualitätssicherungsmaßnahme durchlaufen. Beim "grünen Weg" hat der publizierende Verlag darüber hinaus die Möglichkeit innerhalb einer Speerfrist von überlicherweise 6-12 Monaten \cite{suchen} oder länger den lektorierten und fertig-publizierten Beitrag unter einer eigenen Lizenz zu verkaufen \cite{suchen}. Erst nach Ablauf der Frist, wird auch die finale Fassung des Beitrags frei und offen zur Verfügung gestellt. Hier gibt es je nach Verlag und Publikationsform verschiedenen Möglichkeiten der Ausgestaltung des Publikationsweges \cite{suchen}.

Beim "goldene Weg" stellt der Autor die Publikation über einen Verlag unmittelbar nach der Fertigstellung frei und offen zur Verfügung. Auch die Verlagsversion muss ohne Sperrfrist in einem Repositorium unmittelbar zur Verfügung gestellt werden. Der Verlag hat allerdings zusätzlich die Möglichkeit, den Beitrag kommerziell zu vertreiben und zu verkaufen, muss parallel eine freie und offene Version der Publikation zur Verfügung stellen. Alternativ ermöglicht es der "verzögerte goldenen Open Access" Weg dem Verlag, zeitverzögert für die Öffentlichkeit die finale Version der Publikation unter einer offenen Lizenz zur Verfügung zu stellen \cite{lewis_2012_inevitability}. Der Verlag hat bei diesem verzögerten Modell den Vorteil, einen bestimmten Zeitraum die Publikation vertreiben zu können, ohne zeitgleich eine offene und freie Version anbieten zu müssen.

Im Rahmen anderer Modelle, vornehmlich bei der Publikation in Zeitschriften und Monographien, wird den Autoren auch zunehmend die Möglichkeit eingeräumt auch im Nachhinein, durch zusätzliche Zahlung, die Publikation offen und frei zur Verfügung zu stellen\cite{lewis_2012_inevitability}. Das hat für den Autor den Nutzen, dass er von den Vorteilen bei der Verbreitung von Publikationen unter den Bedingungne von Open Access profitiert. Der Verlag widerum generiert über diese Weg zusätzliche Einnahmen nehmen den intialen Verkaufserlösen.

Bei den genannten Wegen des Open Access Publizierens kann parallel zu der kostenlosen und offenen elektronischen Veröffentlichung von Beiträgen und Büchern eine weitere kostenpflichtige Publikation in gedruckter oder digitaler Form erfolgen \cite{suchen}. Vorraussetzung ist, dass neben den kostenpflichtigen Versionen, eine kostenfreie Version der Publikation unter den in --- Kapitel xxxx-- definierten Bedingungen exisitiert.

Der verzögerte goldenen Weg und der grünen Weg beeinträchtigen das klassische Geschäftsmodell der Verlage vorerst nicht direkt. Der goldene Weg auf Grundlage unmittelbarer, freier und offener Veröffentlichungspflicht dagegen kommt ohne das tradierte Geschäftsmodell der Verlage aus \cite{lewis_2012_inevitability}.

Neben den dargestellen unterschiedlichen Modellen zu Open Access gibt es weitere Formen: Die Einteilung in hybride, gratis und libre, radikale und sonstige Formen von Open Access stellt eine weitere Ebene der Unterteilung dar und soll allen in der Literatur aufkommenden Formen gerecht werden. 

Seit kurzem findet in der Literatur die Unterteilung in gratis und libre Open Access statt. Mit Gratis Open Access seit kurzem die Möglichkeit bezeichnet, den Zugang zu Publikationen und Forschungsergebnisse zu erleichtern und die Kostenpflichtigkeit zu entfernen. Libre Open Access bedeutet, dass weitere Barrieren, wie Urheber- und Lizenzbeschränkungen aufgehoben werden. \cite{Adema_2014_open_access} Diese Unterteilung wird von einigen Autoren nachhaltig kritisiert, da durch das Hinzufügen eines weiteren Attributs die eigentliche scharfe Abrenzung von "Close" und "Open" geschwächt wird, was sich auch auf andere Bereiche der Open-Bewegung (Ope Data, Open Government, Open Spending uvm.) auswirken kann \cite{suchen}. 

In der Literatur \cite{suchen} finden sich auch weitere Open Access Publikationsformen, die zwar als Open Access bezeichnet werden, aber nicht den gängigen Deklarationen \cite{boai_2012} und Definitionen gerecht werden. Dieser Prozess wird Open Washing genannt \cite{suchen}. 

Für Publikationen unter den Bedingungen von Open Access, werden derzeit durch die Verlage Veröffentlichungsgebühren von den Autoren erhoben \cite{suchen}. Dabei wird weder auf den Peer-Review-Prozess, noch auf die Möglichkeit Umsatz zu generieren, Urheber zu schützen oder andere Stärken der traditionellen Publikationsformen verzichtet \cite{albert_2006_open_implications} \cite{Open_Access_net_2009}. Aus dieser theoretischen Perspktive ändert Open Access die Erlöstruktur der Verlage von nachgelagertnen, verkaufsorientierten Einnahmen hin zu vorab Einnahmen für die Erstellung und den Vertrieb der Publikation durch den Autor. Strukturell steht Open Access für Verlage vorerst in keinem Widerspruch zur Bewarhung der wissenschaftlichen Qualität oder den Vorteile des klassischen Publikationssystems \cite{Suber_2002}. Viele Open Access Verlage nutzen zwar Open-Access-Option, wollen damit aber die etablierten Verhältnisse möglichst fortschreiben und halten am Subskrikptionsmodell weiter fest \cite{schmidt_2007_goldenen}.

--- TODO: Weitere, aber im Vergleich wenig genutzte Modelle sind:
Hybride Modelle
Open Choice \cite{Hess_2006} 
----

\subsection{Open Access Kanäle und Formate}
In diesem Abschnitt wird auf unterschiedliche Modelle der Veröffentlichung wissenschaftlicher Inhalte in Form von Open Access Publikationen, sowie auf verschiedene Open Access Kanäle und Publikationsformate eingegangen. Es wird unterschieden in: Open Access Aggregatoren, Open Access Repositorien, Open Access Jounrals, Open Access Bücher, Open Access Monografien. Sie beziehen sich entweder auf bestimmte Publikationsformen der wissenschaftlichen Kommunikation oder auf konkrete Herausforderungen, die im Rahmen der Distribution und Archivierung im Umfeld der neuen Möglichkeiten von offenem und freien Publizieren entstanden sind. 

Da es eine enge Verknüpfung zwischen Repositorien und der Entwicklung der Open-Access-Bewegung gibt \cite{offhaus_2012_institutionelle_repos}, soll hier auf die Rolle der Repositorien als spezifischer Kanal für die Verbreitung von Publikationen eingegangen werden. Repositorien sind verwaltete Orte zur Aufbewahrung geordneter Dokumente, die im Unterschied zu Archiven die ausschließlich historische Dokumente verwalten und öffentlich zugänglich sind \cite{suchen}. Institutionelle Repositorien sind ein Instrument für wissenschaftliche Einrichtungen, wie Universitäten, um ihre Publikationen für einen institutionell abgegrenzten Bereich frei zugänglich zu machen \cite{dobratz_2007_open}.

Institutionelle Repositorien haben potenziell erhebliche Vorteile für die Institutionen, wenn sie in die ganzheitlichen Rahmenbedingungen der Universität integriert sind \cite{steele_2006}. Repositorien können neben dieser klassischen Aufgabe für die Lernumgebungen und die Marketingaktivitäten einer Universität eine wichtige Rolle spielen: Sie ermöglichen zum Beispiel die Dokumentation des universitären Outputs und verbessern den Zugang zu institutionellem Austausch \cite{steele_2006}. Ökonomisch rentieren sie sich vor allem dann, wenn Skaleneffekte eintreten und in Verbünden agiert wird \cite{blythe_2005value}. Neben den institutionellen sind auch fachliche oder andere Arten von Repositorien eng mit der Open Access Bewegung verknüpft. Repositiorien stehen  für die digitale Speicherung von Dokumenten und zunehmend auch Daten. Über die Repositorien wird der Zugang zu den unterschiedlichen Modellen von Open Access Publikationen ermöglicht.

\subsection{Kritik an Open Access}

Im Rahmen der Literaturanalyse wird eine Auflistung der Kritikpunkte an der Open Access Bewegung in Wissenschaft und Forschung dokumentiert. Die Auswahl der berücksichtigten Werke bezieht sich auf die --- TODO in Kapitel --- genannten Fragestellungen und wird in den Überblick der Debatte von Open Access und Open Science einfließen.

Aus der Perspektive der Leser von Open Access Publikationen gibt es kaum Kritik am Konzept \cite{weishaupt_2009_goldenOA}. Aus der Autoreperspektive hingegen herschen viele Vorbehalten und Missverstädnisse und die Akzeptanz von Open Access Publikationen stellt eine der größten Herausforderungen dar \cite{weishaupt_2009_goldenOA} und es  \cite{Suber_2002}. Das betrifft vor allem das Author-Paymodell zur Refinanzierung der Publikation, bei Autoren für die Veröffentlichung der Texte selbst zahlen müssen, damit die Texte frei zugägnlich sind \cite{suchen} und das "obwohl auch bei konventionellen (nicht Open Access) Veröffentlichungen oft genug die Druckkosten selbst aufgebracht werden mussten" \cite{weishaupt_2009_goldenOA}. Eine weitere Hürde bei der Akzeptanz stellen Probleme bei der Sicherung der "Authentizität und Integrität der Texte" dar \cite{weishaupt_2009_goldenOA}. Darüber hinaus gibt es Herausfoderungen bei der Lanzeitarchivierung und der Einbettung offener Kommunikatione in das wissenschaftliche Reputtionssystem \cite{weishaupt_2009_goldenOA} \cite{Suber_2002} \cite{Adema_2014_open_access}.

\subsubsection{Kritik am ökonomischen Modell}

Ein Kritikpunkt an dem Open Access Modell bezieht sich auf das Kostenargument und die frühe Hoffnung, dass die technologischen Treiber gesteuert und organisiert von der Forschungs Community selbst, anstatt durch Fachverlage, die durchschnittlichen Kosten für einen publizierten Artikel signifikant senken könnten. So stellte sich die Frage, ob
"aus der Sicht des individuellen Nutzenkalküls von Wissenschaftlern, Verlagen und weiteren Einrichtungen wie Bibliotheken als auch aus Sicht gesamtwirtschaftlicher Wohlfahrtsüberlegungen (...) der Markt der Wissenschaftskommunikation nicht effizienter organisiert werden könnte."\cite{Hess_2006} In einigen Beiträgen wurden schon sehr früh Kostensenkungen von bis zu 90 Prozent \cite{hilf_2004} \cite{suchen} prognostiziert. Folgende Punkte schürten darüber hinaus die Hoffung, das System leistungsfähiger zu machen und "von seinen durch den Papierdruck auferlegten Fesseln" zu befreien \cite{hilf_2004}:
\begin{itemize}
\item langer Zeitverzug vom Einreichen eines Manuskriptes bis zum Gelesen werden,
\item komplizierter Vertriebsweg vom Verlag über Grossisten zu Bibliotheken,
\item horrende Kosten (ca. 3.000,- Euro für die gesamte Verlagsarbeit je Artikel) mit den daraus folgenden horrenden Zeitschriftenpreisen,
\item und daraus folgend wenige Leser, auch noch ungleich in der Welt verteilt (digital divide),
\item unvollständige Information (aus Platzmangel), was Nachnutzungen und das Nachprüfen erschwert und somit auch Fälschungen erleichtert,
\item nur anonymes Referieren vor der Veröffentlichung, was den Missbrauch erleichtert. 
\end{itemize}

Verlage die Open Access publizieren versuchen nachhaltig zu operieren und passen dazu ihre Preise an. "Aufällig ist jedoch, dass gerade die großen erfolgreichen Projekte wie BioMed Central und Public Library of Science nach ihrer Einführung am Markt deutlichen Gebrauch von Preissteigerungen gemacht haben"\cite{schmidt_2007_goldenen}. Diese Entwicklung hat sich verlangsamt hält aber weiter an\cite{suchen}. In diesem Fall unterscheiden sich subskripitonsbasierte und Open Access-Verlage nicht fundamental \cite{schmidt_2007_goldenen}.

\subsubsection{Gefahr der Einschränkung von Freiheit in Forschung, Lehre und Forschungsdiversität}

Eine Öffnung der wissenschaftlichen Kommunikation hat weitreichende Implikationen, nicht nur auf die Frage wie geforscht wird, sondern auch was geforscht wird \cite{suchen}. Die Vermischung der Interessen an Forschung in Deutschland soll durch die Unabhängigkeit der Deutsche Froschungsgemeinschaft verhindert werden und Mittel völlig frei von politischer Couleur verteilt werden \cite{suchen}, dennoch kann, so die Befürchtung einiger Autoren \cite{suchen}, nicht sichergestellt werden, dass eine zu große Einbeziehung der Öffentlichkeit nicht doch einen Einfluss auf die Mittelvergabe haben könnte. Ein Großteil der Wissenschaft wird durch Steuergelder finanziert, was dazu führt, dass politische Interessen, die Steuerungsmechanismen von Wissenschaft und Forschungsförderung beeinflussen können. Der Wissenschaftstheoretiker Michael Hagner formuliert in einem Beitrag für die F.A.Z. "Open Access als Traum der Verwaltungen" dass es, bei der Verpflichtung zur elektronischen Veröffentlichung von Forschungsergebnissen für Wissenschaftler durch Universitäten "auf eine vollends verwaltete Forschung hinaus" laufen würde \cite{suchen}. Andere antizipieren eine weitere Gefährdung von Wissenschaft und Forschung, weil Grundlagenforschung, andere komplexe oder explorative Forschungsbereiche in Zukunft weniger Berücksichtigung finden werden, wenn die Öffnung der wissenschaftlichen Forschungsprozesse weiter vorangetrieben wird \cite{suchen} \cite{suchen} \cite{suchen}. 

Um diese Aspekte, beziehungsweise Prognosen über die Implikationen von Open Access zu evaluieren, werden in diesem Teil der Arbeit auf Grundlage von Textbeispielen die Kritik an der Öffnung von Wissenschaft und der (forschungs-)politischen, rechtlichen und freiheitlichen Entwicklungen beleuchtet.

\subsubsubsection{Beispiel: Der "Heidelberger Appell" für Publikationsfreiheit und die Wahrung von Urheberrechten}

Am 22. März 2009 wurde auf der Webseite der „Frankfurter Allgemeinen Zeitung“ der Artikel "Geistiges Eigentum: Autor darf Freiheit über sein Werk nicht verlieren" \cite{faz_heidelberger_apell_2009} veröffentlicht. Vorangegangen war eine öffentlich ausgetragene Diskussion zwischen dem Literaturwissenschaftler Prof. Dr. Roland Reuß sowie weiteren Wissenschaftlern in einem Spezial der Onlineausgabe der Frankfurter Allgemeinen Zeitung: "Die Debatte über Open Access". Im Anhang fand sich ein öffentlicher Aufruf, auch der "Heidelberger Appell" genannt. 

Der Appell richtete sich vor allem an "die Bundesregierung und die Regierungen der Länder, das bestehende Urheberrecht, die Publikationsfreiheit und die Freiheit von Forschung und Lehre entschlossen und mit allen zu Gebote stehenden Mitteln zu verteidigen" \cite{ITK_2009}. Die Autoren forderten, unter anderem in Bezug auf die Google Buchsuche, die Politik, Öffentlichkeit und weitere Kreative auf sich für die "Wahrung der Urheberrechte", "gegen eine angebliche „Enteignung“ der Autoren durch das Vorgehen von Google einerseits und durch das Publikationsmodell Open Access andererseits" \cite{WD_bundestag_2009} zu engagieren. 

Im Rahmen dieser Arbeit ist von besonderem Interesse, inwiefern die Kritik des "Heideberger Appells" am Publikationsmodell Open Access berechtigt ist. 
Die Autoren des Appells unterscheiden zwei Ebenen: \textit{International} kritisieren sie "die nach deutschem Recht illegale Veröffentlichung urheberrechtlich geschützter Werke geistigen Eigentums auf Plattformen wie GoogleBooks und YouTube", sowie die Entwendung dieser "ohne strafrechtliche Konsequenzen". Im \textit{nationalen Rahmen}, so prangern die Autoren weiter an, werden diese "Eingriffe in die Presse- und Publikationsfreiheit, deren Folgen grundgesetzwidrig wären" durch die "»Allianz der deutschen Wissenschaftsorganisationen« (Mitglieder: Wissenschaftsrat, Deutsche Forschungsgemeinschaft, Leibniz-Gesellschaft, Max Planck-Institute u.a.)" sogar unterstützt.\cite{ITK_2009}

Die Kritik der Autoren des Heidelberger Apells bezieht laut einer Untersuchung des Wissenschafltichen Diensts des Bundestags insbesondere auf drei Aspekte \cite{WD_bundestag_2009}:
\begin{enumerate}
\item Erzwungene Vertriebswege
"Eine Forschung, der man diktieren könnte, wo ihre Ergebnisse publiziert werden sollen, sei nicht mehr frei." Die Verpflichtung auf "bestimmte Publikationsform (...) dient nicht der Verbesserung der wissenschaftlichen Information" \cite{ITK_2009}.
\item Abhängigkeitsverhältnis
\item Subventionierung von Vertriebswegen
\end{enumerate}

Der Appell "hat eine außergewöhnlich heftige Diskussion über die urheberrechtliche Problematik im Hinblick auf die aktuellen Entwicklungen im Internet ausgelöst. Er hat auch viele Parlamentarier und Politiker für das Thema sensibilisiert"\cite{WD_bundestag_2009}. An vielen Stellen widerlegt der Wissenschaftliche Dienst die Befürchtungen der Autoren des Heidelberger Apells. Beim Kritikpunkt der "Erzwungene Vertriebswege" widerspricht der Wissenschaftliche Dienst mit dem Verweis auf Gudrun Gersmann, weil "auch (Anmerkung: unter Open Access) eine Veröffentlichung bei einem Verlag mit einfachem Nutzungsrecht weiterhin möglich sei". In Bezug auf die im Apell erwähnte Kritik am neuen Abhängigkeitsverhältnis halten die wissenschaftlichen Autoren des Bundestags Reuß entgegen, dass es im bisherigen System "zwischen Autor und Fachzeitschriftverlag oft ein einseitiges Abhängigkeitsverhältnis zu Lasten des Autors gibt" und Wissenschaftler "oftmals alle Rechte an ihren Beiträgen abtreten" \cite{WD_bundestag_2009} müssen. "Der Befürchtung im Heidelberger Appell, das Publikationsmodell Open Access gefährde Fachzeitschriftenverlage", laut Autoren dritter Aspekt der Kritik an Open Access im Apell, "wird entgegengehalten, dass die digitale Plattform auf lange Sicht auch ein Ausweg aus der Zeitschriftenkrise sein könnte" \cite{WD_bundestag_2009}.

Dabei ist die Kritik im Rahmen des Apells mindestens an zwei Punkten berechtigt, so ist es erstens wahr, dass man seitens der Forschungsförderer nicht besonders bemüht war und ist \cite{suchen}, sich "ein genaues Bild von den Nebenwirkungen (Anmerkung: von Open Access)" \cite{Reuss_2009} zu verschaffen und zweitens stellt die Sicherung von Freiheit von Forschung und Lehre sowie die Anpassung der Steuerungsmechanismen eine Herausforderung an die Bestrebungen zur Öffnung von Wissenschaft und Forschung dar \cite{suchen}.

Die Kritik am urheberrechtlichem Aspekt der Google Buchsuche soll in dieser Arbeit nicht berücksichtigt werden. 

\subsubsection{Kritik der neoliberalen Rethorik}

Openess kann als "schwimmender Signifikant (...) ohne eindeutige Definition, adaptierbar von unterschiedlichen politischen Ideologien" verstanden werden \cite{Adema_2014_open_access}. Als neoliberale Rethorik wird der Begriff Open Access als effizientes Wettberwerbsmodell verbunden mit den Ideen von Transparenz und Effizienz von Unternehmen und Regierung eingesetzt \cite{tkacz_2012_open}. Über diesen Ansatz kann Openness den wissenschaftlichen Prozess outputorientierter und seine Ergebnisse effektiver gestaltet und überwacht werden \cite{adema_2010_oaoverview} . 

\section{Analyse von Open Science} 


\section{Definition von Open Science} 
--- TODO ---- Michael Nielsen: “Open science is the idea that scientific knowledge of all kinds should be openly shared as early as is practical in the discovery process.”  https://lists.okfn.org/pipermail/open-science/2011-July/000907.html
http://www.openscience.org/blog/?p=454,

Research Information Network: “science carried out and communicated in a manner which allows others to contribute, collaborate and add to the research effort, with all kinds of data, results and protocols made freely available at different stages of the research process.” http://www.rin.ac.uk/our-work/data-management-and-curation/open-science-case-studies

Fecher/Friesike 5 Schulen von Open Science http://blogs.lse.ac.uk/impactofsocialsciences/2013/06/20/open-science-new-perspectives-for-scholarly-communication/ 
Siehe "Open Science"-Teil @ https://docs.google.com/document/d/1qDkQV-M_2VazjWwncRq_udo9tQqrjuZZkdLeKFc3cpI/edit#heading=h.1ahb76xafkbm

"Open science is the concept of making the whole research process as transparent and accessible as possible."\cite{Scheliga_2014}

Open science can be seen as a mechanism of cumulative knowledge production whereby scientists draw upon knowledge derived at by "prior researchers" and make their discoveries available to "future researchers". \cite{Scheliga_2014} auf Grundlage von \cite{Mukherjee_2009}
--- TODO ---

Es gibt zahlreiche Open Science Initiativen \cite{Scheliga_2014} viele von Ihnen erreichen aber keine kritische Masse \cite{wrap_2010} und enden eher als "virtuelle Geisterstädte" \cite{Nielsen_2011}.

Bei der Verbreitung von Open Science, werden grundsätzlich zwei Strategien für die Etablierung von Offenheit in Wissenschaft und Forschung abgegrenzt \cite{schulze_2013_open}: 
\begin{enumerate}
\item "Top-down durch Förderstrategien, Vorgaben und Empfehlungen"
Hiermit sind Prozesse beziehungsweise gemeint, bei denen durch die direkte Incentivierung im Rahmen von Forschungsförderung Anreize für die Berücksichtigung von Offenheit in den geförderten Projekten geschaffen werden. Beispielsweise kann durch die Bereitstellung zusätzlicher Mittel für die offene Bereitstellung und Publikation von Forschungsergebnissen ein Anreiz geschaffen werden  \cite{suchen}. Neben der Incentivierung bietet die bindende Vorgabe eine weitere Möglichkeit zur Etablierung von Verhaltensänderungen \cite{suchen}. So kann durch Änderung der politischen und rechtlichen Vorgaben eine Öffnung von Wissenschaft und Forschung erzwungen werden \cite{suchen}. Eine weitere Möglichkeit der "Top-Down"-Etablierung von Offenheit und Forschung stellen Empfehlungen dar, bei denen Insitutionen, Organisationen oder Gruppen Empfehlungen aussprechen, anhand derer WissenschaftlerInnen über nicht bindende Hinweise überzeugt werden sollen, die Öffnung von Wissenschaft und Forschung zu etablieren. Alle diese Strategien haben einen formellen Charakter \cite{suchen}.
\item "Bottom-up durch Graswurzelprojekte und den Einsatz von Evangelists"
Im Gegenzug zur Strategie von "oben" gibt es auch Bestrebungen, die von einzelnen WissenschaftlerInnen oder Gruppen initiiert sind. Sie sind zumeist informell und zielen auf eine beispielhafte Herangehensweise für die Verbreitung von Verhaltensänderungen ab \cite{suchen}. Bottum-up-Projekte kommen aus dem wissenschaftlichen Alltag und erfahren keine politische, rechtliche oder monitäre Incentivierung zur Umsetzung der Tätigkeiten für die Öffnung von Wissenschaft und Forschung. Der Einsatz von Evangelisten baisert auf der Idee einer konkreten Stelle oder Position um eine Änderung zu Begleiten \cite{suchen} oder einen Mulitplikator innerhalb und außerhalb von Insitutionen oder Organisationen zu etablieren, der das gewünschte Ziel proaktiv kommuniziert und verbreitet \cite{suchen}. Evangelisten stellen einen wesentliche Maßnahme dar, um "die Befindlichkeiten" "auszutarieren" und um "teils diffuse, teils reale Ängste" bei "Offenheit und Transparenz der Wissenschaft "\cite{schulze_2013_open} zu beseitigen.
\end{enumerate} 

In beiden Fällen steht und fällt der Erfolg damit, ob sich der jeweiligen Zielgruppe ein unmittelbarer Mehrwert und Nutzen erschließen wird \cite{schulze_2013_open}.

\subsection{Kritik an Open Science}

Während viele Wissenschaftler und Wissenschaftlerinnen Offenheit in der Forschung als wertvoll erachten \cite{suchen}, sind nur wenige tatsächlich bereit, die zusätzliche Zeit und Mühe dafür zu investieren und potenziellen Risiken einzugehen, ihre Forschung offen und zugänglich zu machen \cite{Scheliga_2014} \cite{Procter_2010}. Forscherinnen und Forscher, die offene Wissenschaft praktizieren wollen, werden mit einer Reihe von Hindernissen konfrontiert \cite{Scheliga_2014}: 
\begin{enumerate}
\item individuelle Hindernisse: Angst vor Trittbrettfahren, Mehraufwand an Zeit und Mühe, Herausforderungen bei der Nutzung der digitalen Dienste, fehlender Anstoß, negative Ergebnisse zu veröffentlichen, Herausforderung den Datenschutz sicherzustellen, Abneigung den Code zu teilen
\item systematische Hindernisse: Evaluationskriterien behindern Offenheit, kulturelle und institutionelle Einschränkungen, ineffektive (politische) Richtlinien, Mangel an Standards für das Teilen von Forschungsmaterialien, Mangel an rechtlicher Klarheit, finanzieller Aspekte der Offenheit
\end{enumerate}

Betrachtet wie Scheliga und Friesike das Phänomen Open Science an Hand des Konzepts der Soziale Dilemmata, wird deutlich, dass das was im kollektives Interesse der wissenschaftlichen Gemeinschaft ist, nicht unbedingt im Interesse des einzelnen Wissenschaftlers steht. "Wenn alle Wissenschaftler ihr Wissen nur in den Situationen teilen, in denen sie erwarten, dass sie selbst davon profitieren, ist die gemeinsame Wissenspool fragmentiert und alle Wissenschaftler stehen schlechter dar"\cite{Scheliga_2014}. 

Demgegenüber stehen dem gegenüber  --- TODO: ausarbeiten ---

\subsection{Treiber und Bremser für Open Access} 

In den wissenschaftlichen Beiträgen zu Open Access werden viele positive Folgen aufgelistet. Folgende Treiber für eine Veränderung und Öffnung des wissenschaftlichen Kommunikationssystems werden dabei besonders häufig genannt:

\begin{itemize}
\item Verbreitung und Nutzungsmöglichkeiten der digitalen Infrastrukturen
\item Vorteile des grenzüberschreitenden Austauschs im Rahmen der Globalisierung von Wissenschaft und Forschung
\item ...
\end{itemize}

Neben den Aspekten die die Verbreitung von Open Access in den letzten Decaden unterstützt haben, gibt es aber auch einige Kriterien, die entweder zu einer Verlangsamung der Entwicklung geführt haben, oder sie in einigen Teilbereichen ganz zum erliegen gebracht haben. Dazu gehören:

\begin{itemize}
\item Fehlende Richtlinien auf regionaler, nationaler und internationaler Ebene
\item Führungslosigkeit der Open Access Bewegung
\item ...
\end{itemize}


\subsection{Treiber und Bremser für die Öffnung der wissenschaftlichen Kommunikation} 
Für die Öffnung von Wissenschaft und Forschung ist es wichtig die Treiber und Bremser für die Entwicklung zu identifizieren. Dafür ist eine Erarbeitung der Unzulänglichkeiten am bestehenden wissenschaftlichen Kommunikationssystem wichtig\cite{cite:17}. Aus den folgenden Aspekten hat sich diese Forderung nach Öffnung entwickelt:
\begin{enumerate}
\item \textbf{Transition-Argument} - Die Nutzung der neuen Möglichkeiten für eine offene Wissensverbreitung neben den konventionellen Wegen der nicht-elektronischen Publikationen \cite{berliner_erklaerung_2003}. Dabei gilt die Grundvoraussetzung der Aufbereitung des Wissens als strukturierte Daten zur Wissensweiterverwendung und -verarbeitung über alle Kanäle.
\item \textbf{Speed & Circulation-Argument} - Wissensverbreitung wird künstlich durch Embargos und ineffiziente Validationssysteme zurückgehalten. Die Digitalsierung und Verbreitung über elektronische Kanäle stellt einen Vorteil für Wissensverbreitung und -verwertung dar. Wenn das Wissen schneller zur Verfügungsteht wird es schneller zirkulieren und effizienter genutzt werden können \cite{Woelfle_2011}.  
\item \textbf{Higher Impact & Citation-Argument} - Ein Hauptargument ist die höhere Zitationsrate wissenschaftlicher Publikationen, die nach den Kriterien von Open Access veröffentlicht wurden\cite{cite:21a}. In der Literatur findet man viele Untersuchungen, die diesbezüglich zu einem positiven Ergebnis kommen \cite{Lawrence_2001} \cite{Jeffrey_2008} \cite{Eysenbach_2006} \cite{Antelman_2004}.
\item \textbf{Tax-Payer-Agrument} - Durch Steuergelder finanzierte Forschung ist dem Steuerzahler konventionelle wissenschaftliche Kommunikation selten unentgeldlich zugänglich, obwohl er defacto im Rahmen öffentlich-geförderter Forschungsprogramme die Forschung bereits finanziert hat \cite{suber_2003_taxpayer} \cite{Adema_2014_open_access}. Darüber hinaus stellt sich die weiterführende Frage nach dem bestmöglichen Einsatz der monetären Ressourcen \cite{Glasziou_2014} \cite{altman_1994_scandal}.
\item \textbf{Economic Promotion Argument} - Bisher profitieren wirtschaftliche Unternehmungen nur unzureichend von staatlich-finanzierter wissenschaftlicher Kommunikation, dabei könnte eine schnellere, kommerziell verwertbare und umfassendere Bereitstellung der wissenschaftlichen Inhalte einen Beitrag zur non-monetären Wirtschaftsförderung und Innovation darstellen \cite{heise_2012}. Im Rahmen der offenen und schnelleren Verbreitung wissenschaftlicher Informationen sind neue Geschäftmodelle denkbar \cite{suchen}.
\item \textbf{Digital Divide Argument} - Der offene Zugang zu Publikationen ermöglicht neue Möglichkeiten für die Überwindung der sozialen, nationalen und globalen Wissenskluften  zwischen bildungsfernereren und -affineren Bevölkerungsteilen und -schichten der Welt \cite{boai_2012}. Der Mehrwert und die Chance von wissenschaftlichen Informationen für die Bewegung der offenen Bildungsmaterialien ist bisher auch noch nicht ausgeschöpft\cite{heise_lernen_2013}.
\item \textbf{Validation & Reputation-Argument} - Die Entwicklung neuer Verfahren, die die Aktivität und Qualität eines Forschers umfassender, transparenter und demokratischer messbar und kommunizierber machen, als im bestehenden Reputations- und Förderungssystem \cite{chalmers_2009_avoidable_waste}. Wissenschaftsevaluation wird durch Offenheit effizienter, da Wissenschaft einerseits "per definitionen die Bemühung um integere Information ist" \cite{umstatter_2007_qualitatssicherung} und eine Falsifikation nur dann umfassen und einfach möglich ist, wenn der Aufwand für die Falsifikation bzw. der Zugriff auf die wissenschaftliche Informationen möglich ist \cite{umstatter_2007_qualitatssicherung}.
\item \textbf{Paradoxon of Information Argument} - Überwindung des bestehenden Informationsparadoxons bei der Verbreitung und Vermarktung von wissenschaftlicher Inhalte. Hierbei handelt es sich um das Problem, dass es schwer ist eine Information kommerziell zu verwerten ohne zu viel über Inhalt und Qualität auszusagen. Eine Entkommerzialisierung des Vertriebs von Wissen  würde das Informationsparadoxon aufheben.
\item \textbf{Science communication Crisis-Argument} - Durch die Öffnung wissenschaftlicher Kommunikations- und Reputationsprozesse entsteht die Möglichkeit, der vorherrschende Zeitschriften- und Monographienkrise durch neue Geschäftsmodelle zu begegnen \cite{suchen}.
\item \textbf{Interdicipline & International Exchange/Collaboration Argument} - Die Globalisierung in der Wissenschaft führt immerstärker zu internationalem Austausch und zur internationalen Zusammenarbeit von Wissenschaftlern \cite{Waltman_2011}. Doch das gilt nicht nur für die grenzenüberschreitende Zusammenarbeit in Bezug auf die lokale Verortung sondern auch für die Interdisziplinarität der Forschungsvorhaben. Die Öffnung von Wissenschaft ermöglicht also auch Fächerfremden Wissenschaftlern Zugruff auf Publikationen und damit auf Wissensressourcen für die eigene Arbeit .
\item \textbf{Sustainable Access & Archiving Argument} - Nur Offenheit im Sinne von Verwertbarkeit ermöglicht es in dezentralen Strukturen wie der des Internets alle Informationen nachhaltig und unabhängig voneinander zu speichern. Im Falle von Natur- oder anderen Katastrophen ermöglicht die digitale Ablage auf mehreren Kontinenten eine präservierung von Wissen undabhängig von lokalen Gegebenheiten oder Bedingungen.

\item \textbf{Dataquality-Argument} - Die Veröffentlichung der Daten hinter den wissenschaftlichen Publikationen kann zu einer insgesamten Erhöhung der Datenqualität im Rahmen wissenschaftlicher Arbeit führen. Ähnliche Erfahrungen wurde bereits im Bereich der Veröffentlichung von Daten bei der Entwicklungszusammenarbeit gemacht\cite{heise_2014_bundestag}.
\end{enumerate}

\textbf{Demgegenüber} stehen aber auch Argumente gegen die Öffnung der wissenschaftlichen Prozesse und Publikationen:
\begin{enumerate}
\item \textbf{Quality-Argument} - Die Befürchutung, das die Qualität auf Grund von schlechten oder nicht vorhandenen wissenschaftlichen Überprüfungsmechanismen leidet. Hauptargument ist das durch ein Autorengebühren finanziertes Publikationsmodell keinen klaren Anreiz für Ablehnung bietet.
\item \textbf{Renomee-Argument} - Wissenschaftliche Reputation ist der größten Motivationsfaktoren bei der Veröffentlichung von Inhalten. Offene Publikationsplatformen und Journale können auf Grund der Kürze des bestehens und auf Grund von Vorbehalten dieses Renomee nicht vorweisen. Die Renomeefrage stellt eine der größten Hürden für die offene wissenschaftliche Kommunikation dar \cite{weishaupt_2009_goldenOA}.
\item \textbf{Archiving- & Sustainability-Argument} - Den Vorteilen des elektronischen Publizierens steht Probleme und Zwefel an der Langfristverfügbarkeit und Langzeitarchivierung \cite{weishaupt_2009_goldenOA} gegenüber. Die Sicherstellung der Langzeitarchivierung und die Garantierung der langfristigen Auffindbarkeit sowie Bereitstellung der Dokumente kann im Auge der Kritiker von Offenheit in Wissenschaft und Forschung kann bisher nicht durch alternative digitale Strukturen gewährleistet werden \cite{umstatter_2007_qualitatssicherung}. 
\item \textbf{Authenticity- or Integrity-Argument} - Ein weiteres Problem stellt die Sicherung der Authentizität dar \cite{umstatter_2007_qualitatssicherung} \cite{weishaupt_2009_goldenOA}. Weil Elektronische Dokumente oft innerhalb weniger Tage oder Wochenin mehreren Versionen zugänglich sind, befürchten Forscherinnen und Forscher durch die dezentrale und offene Handhabung ihrer Texte und Arbeiten, dass diese im Zeitablauf inhaltlich nicht mehr unverändert ihrem Autor zuordnenbar sind, "solange sie nicht in Digitalen Bibliothekenmit gesicherter Authentizität abgeliefert" wurden \cite{umstatter_2007_qualitatssicherung}.
\item \textbf{Rightsmanagement-Argument} - Hierbei handelt es sich um die Verpflichtung für Mitarbeiter staatlich finanzierter Forschungsinsitutionen alle Texte elektronisch frei und offen zu publizieren. In dem 2009 veröffentlichten "Heidelberger Appell" \cite{faz_heidelberger_apell_2009} kritisieren zahlreiche Autoren, Wissenschaftler, Verleger und Publizisten, dass das “verfassungsmäßig verbürgte Grundrecht von Urhebern auf freie und selbstbestimmte Publikation” … “derzeit massiven Angriffen ausgesetzt und nachhaltig bedroht” ist. Weiter sehen die Unterzeichner „weitreichende Eingriffe in die Presse- und Publikationsfreiheit, deren Folgen grundgesetzwidrig wären“ \cite{ITK_2009}. Die rechtlichen Bedenken und die Unwissenheit stellen einen weiteren Vorbehalt gegen die offene Veröffentlichung von Forschung- und Forschungsergebnissen dar \cite{weishaupt_2009_goldenOA}.
\item \textbf{(Re-)Financing-Argument} - Die unklare Refinanzierung der Öffnung von Wissenschaft ist eines der Kernargumente gegen das offene Publizieren von Arbeiten und Daten. Die Befürchtung ist, das ein solches System überhaupt nicht finanziert werden kann, konnte bisher nicht ausgeräumt werden \cite{weishaupt_2009_goldenOA}.
\item \textbf{Ressource-Allocation-Argument} - Die Befürchtung, dass die Vergabe von Fördermittel und für die Karriere wichtige Aspekte der Reputationsbildung durch offenen System nicht Rechnung getragen wern kann ist eine weiteres Argument der Kritiker der Öffnung von Wissenschaft und Forschung. Eine Mittelvergabe zu gunsten populärer Forschung und damit eine Aushöhlung des wissenschaftlichen Systems in Ihrer Fächer und Facettenvielfalt wäre eine unmittelbare Folge dessen.
\item \textbf{Open-Caring-Argument} - Wissenschaftlerinnen und Wissenschaftler fürchten durch den Zwang zu umfassenderen Bereitstellung von Publikationen und gegebenenfalls soagar Daten einen nicht unwesentlichen zeitlichen Mehraufwand für die Öffnung ihrer Arbeiten. Sie möchten aber möglichst wenig Zeit für die Veröffentlichung, Bereithaltung und Verbreitungung ihrer wissenschaftlichen Arbeiten aufbringen. Der Aufwand für Offenheit im Alltag des Wissenschaftlers ist bisher nur marginal evaluiert \cite{osterloh2008anreize}.
\item \textbf{Scientific-Freedom/Loss of Idea-Diversity-Argument}
Dieses Argument betrifft zwei Ebenen: 1. Die Angst dass durch Offenheit und Transparenz Forschungsförderung und Öffentlichkeit die Steuerungsmechanismen der Wissenschaft ausgehebelt werden und nur die wissenschaftlichen Projekte gefördern und unterstützt werden, die vom Wähler und Steuerzahler verstanden werden und 2. die Befürchtung, dass die Freiheit von Forschung und Lehre im Sinne der Publikations- und Veröffentlichungsfreiheit gefährdet ist\cite{Jochum_2009}. Dabei stellt Wissen, vorallem im Rahmen der Grundlagenforschung ein "öffentliches Gut" dar, "dessen Wert von der Öffentlichkeit nur schwer beurteilt werden kann"\cite{osterloh2008anreize}. Als folgedessen besteht auch die Befürchtung, dass im Rahmen von zunehmender Kollaboration und der Effizienz der elektronischen Suche die Diversität von wissenschaftlichen Meinungen und Projekten zu einem gleichen oder ähnlichem Thema eingeschränkt wird \cite{Evans_2008}.
\item \textbf{Interpretations-Argument} - Eine der weiteren Ängste der wissenschaftlichen Community ist die Angst vor der Fehlinterpretation ihres kommunizierten Wissens sowie der Verlust der Kontrolle über die Informationen\cite{gibbons_1994}. Dabei steht vor allem die Befürchtung im Vordergrund, dass die offen veröffentlichten Arbeiten genutzt werden um die Arbeit zu miskreditieren oder gezielt zur Falschinfromation der Öffentlichkeit zu nutzen.
\item \textbf{Transparent-Research-Intentions-Argument} - Mit den Forderung nach Offenlegung des gesamten Forschungsprozess erfolgt auch die Forderung nach "Transparenz der Interaktion zwischen Sponsoren (insbesondere kommerzielle Förderer wie die Pharma- und Medizinprodukteindustrie) und Auftragnehmern" \cite{Stengel_2013} 
\end{enumerate}

\subsection{Offener Zugriff auf wissenschaftliche Kommunikation}
Open Science beinhaltet nicht nur den offenen Zugang zu Wissenschaft (Open Access) und den daraus resultierenden Veränderungen wissenschaftlicher Kommunikationsprozessen im Rahmen von Publikationen, sondern auch den unmittelbaren und offenen Zugriff auf den gesamten Prozess der Wissensschaffung. Aus technischer Sicht ist jeder Aspekt der Wissenschaftskommunikation, der digital auf einem Desktop-Computer stattfindet, auch öffentlich über das Web potenziell verfügbar \cite{mietchen2012wissenschaft}. 

Zur Verdeutlichung des Prozesses der Wissensschaffung wird in der vorliegenden Arbeit eine Einteilung extrapoliert in vier Phasen vorgenommen:
\begin{enumerate}
\item Fragestellung & Planung
Basis für den Prozess der Wissenschaffung ist eine Frage zur Erklärug einer speziefischen Beobachtung oder eine offene Frage\cite{suchen}. Für die wissenschaftliche Bearbeitung eines Themas ist es entscheidend, dass eine präzise Fragestellung im Zentrum steht \cite{suchen}. --- TODO: weiter beschreiben ---
\item Ausführung
Testen der Hypothese durch den Einsatz von geeigneten wissenschaftlichen Kontrollen und unter Minimierung der möglichen Fehler.
\item Verarbeitung und Analyse --- TODO: weiter beschreiben ---
Analyse der gewonnen Daten und Informationen im Hinblick auf die Verifikation und Falsifikation der Hypothese. --- TODO: weiter beschreiben ---
\item Auswertung
----TODO: Beschreiben-----
\end{enumerate}

Anhand der hier vorgenommenen Einteilung werden die Charakteristika des Wissenschafts-Prozesss erläutert und dargestellt, um zu verdeutlichen, was die Öffnung von Wissenschaft im Sinne von Open Science beinhaltet. Die Forderung nach Öffnung des gesammten Prozesses der Wissensschaffung begründet sich dabei nicht durch Unzulänglichkeiten am bestehenden wissenschaftlichen Kommunikationssystem, sondern basiert auf weiterführenden Annahmen:

\begin{enumerate}
\item Der offene Zugang zum gesamten Wissenschaftsprozess erhöht die Möglichkeiten der Validierung und Reproduzierbarkeit der gesamten Forschung(skette) und die Entwicklung neuer Qualitätskriterien. (enhanced Validation/Reputation-Argument)
\item Im Rahmen des Teilens (z.B. von Rohdaten) erhöht sich die Effizienz und Verwendbarkeit von der in Forschung und Wissenschaft entstandenen Informationen. (Shared-Science-Argument)
\item Im klassischen wissenschaftlichen Kommunikationssystem gibt es kaum Anreize negative, widerlegende oder unerfolgreiche wissenschaftliche Ergebnisse zu veröffentlichen. Eine vollumfängliche Öffnung des wissenschaftlichen Erkenntnisprozessess könnte dazu beitragen, dass Wissenschaft ihrem Anspruch an Falsifizierbarkeit gerecht wird. (negative-science/falsifiability-argument)
\end{enumerate}

\subsection{Wissenschaft als Open-Source-Prozeß}

Open Source ist ein Begriff aus der Softwareentwicklung der als Gegensatz zum “Verfahren der Wissenssicherung” \cite{stallman2002} eine quelloffenen Handhabe von Softwarecode beschreibt. Der Ende der 90iger Jahre des letzten Jahrhunderts eingeführte Begriff wird, auch wenn es im Detail Unterschiede im Konzept gibt \cite{suchen}, mit “freier Software“ (nicht Freeware) gleichgesetzt \cite{suchen}. Dabei folgt die Open Source-Entwicklung der Maxime, dass die Kernsteuerungsinformationen und -befehle (Quelltext) von Software öffentlich einsehbar und zugänglich und je nach gewähltem Lizenzmodell modifiziert, kopiert oder weitergegeben werden müssen\cite{suchen}. 

Die Entwicklungsmethode unterscheidet, so der Berkeley-Professor --- TODO Quelle prüfen ---- Steven Weber, zwischen Open-Source-Software und dem traditionellen Modell des geistigen Eigentums mit der Feststellung, dass Open-Source-Software das Prinzip der Exklusivität des geistigen Eigentums auf den Kopf stellt, weil diese Software 'um das Recht auf Vertrieb konfiguriert, nicht auszuschließen ist" \cite{suchen}. 

--- Prüfen ---
Auch Maurer und Scotchmer merken an, dass Open-Source-Software-Entwicklung Rechtsmittel ein Defekt der Schutz des geistigen Eigentums, die nicht allgemein zu fördern hat die Offenlegung des Quellcodes. 
--- Prüfen ---

Die Open Source Definition beinhaltet festgelegte Kriterien für die Klassifizierung von Open Source Produkten \cite{suchen}:
\begin{enumerate}
\item Freie Weitergabe
----TODO: Beschreiben-----
\item Quellcode, das Programm muss den Quellcode beinhalten, bzw. muss den Code offen zur Verfügung stellen
----TODO: Beschreiben-----
\item Verwendete Lizenz muss Derivate zulassen
----TODO: Beschreiben-----
\item Unversehrtheit des Quellcodes des Autors muss garantiert werden
----TODO: Beschreiben-----
\item Auschluss von Diskriminierung von Personen oder Gruppen
----TODO: Beschreiben-----
\item Keine Enschränkung des Einsatzfeldes
----TODO: Beschreiben-----
\item Lizenz muss weitergegeben werden könnne
----TODO: Beschreiben-----
\item Lizenz muss auf das Produktpaket angewandt werden
----TODO: Beschreiben-----
\item Lizenz darf die Weitergabe zusammen mit anderer Software nicht einschränken
----TODO: Beschreiben-----
\end{enumerate}

Im Vergleich zum klassischen Softwareentwicklungsprozess gelten folgende charakteristische Merkmale \cite{suchen}:
\begin{enumerate}
\item “Anzahl der beteiligten Entwickler: Im Vergleich zu traditioneller Softwareentwicklung ist eine weitaus größere Anzahl von Entwicklern beteiligt. Zudem gibt es keine klare Grenze zwischen Entwicklern und Anwendern, da die Hürden für eine Partizipation im Entwicklungsprozess sehr gering sind. Auch wenn ein großer Teil der Entwicklungsarbeit von Freiwilligen übernommen wird, gibt es dennoch den Trend zum Einsatz bezahlter Entwickler.
\item Zuteilung der Arbeit: Im OSP wird die Entwicklungsarbeit nicht länger von einer definierten Instanz zugeteilt, sondern die Teilnehmer wählen ihre Arbeitspakete selbst aus.
\item Architektur: In der Regel orientierten sich die Teilnehmer eines OSP nicht an einer vorgegebenen System-Architektur. Die Gestaltung der Architektur geschieht dezentral und ist oftmals häufigen Richtungswechseln unterworfen.
\item Koordination: Es gibt wenig oder keine institutionalisierten traditionellen Koordinationsmechanismen, wie z.B. Projekt- und Zeitpläne, Lasten- und Pflichtenhefte u.ä.” \cite{suchen}
\end{enumerate}

Der Literaturwissenschaftler und Medientheoretiker Friedrich Kittler beschreibt die Entwicklungsmethode Open-Source als fest mit dem Wissenschaftsprozess verankert \cite{suchen}. Open Source Entwicklungsprozesse unterscheiden sich von den klassisch-traditionellen (closed-source) Softwareentwicklungsprozessen insbesondere dadurch, dass sie jederzeit öffentlich einsehbar und transparent nachvollziehbar sind. Open Source zeigt diesbezüglich mit Open Science Konvergenzen, als dass es nicht nur den freien und offenen Zugang zu wissenschaftlichen Informationen betrifft, sondern auch den Zugriff auf den gesammten Prozess zur Erlangung der wissenschaftlichen Informationen sowie die Daten offenlegt und transparent nachvollziehbar macht \cite{kelty_2004}. Adaptiert man den Open-Source-Prozess auf wissenschaftliche Wertschöpfungsprozesse und definiert in diesem Zusammenhang wissenschaftliche Publikationen als Quellcode, ist das Konzept übertragbar \cite{Singh_2008} \cite{Bradley_2008} \cite{Bradley_2007}. 

Daraus folgt, dass Open Access aus technologisch-entwicklungsmethodischer Sicht mit kostenloser Software (Freeware) \cite{suchen} verglichen werden kann. Freeware und Open Access Publikationen sind zwar kostenlos verfügbar, ihr Erstellungprozess wird jedoch nicht offen und transparent kommuniziert \cite{suchen}. Dieser Exkurs in die Softwareentwicklung versucht die Abgrenzung von Open Access zu Open Science zu verdeutlichen und stellt Parallelen zu Open Source versus kostenloser Software (Freeware) her. Es gibt aber noch eine  weitere Gemeinsamkeit: "Free Software (im Sinne von Open Source), Open Access und Creative Commons sind alles Rechts- und Infrastrukturexperimente"\cite{kelty_2004}.

\subsection{Open Science Modelle}
--- TODO: definieren ----
\subsection{Open Science Formate}
Data Repositorien, (offne) Forschungsanträge, offenes Publizieren (siehe OA), Laborbücher



\subsection{Kritik}
---- TODO: Neu einordnen -----
Die Verlage haben mit Hilfe wissenschaftlicher Journale ein zentrales Steuerungs- und Bewertungssystem in der Wissenschaft etablieren können. Dabei werden die Grundprinzipien der Wissenschaft für die verlegerischen Verwertungsinteressen genutzt und das, obwohl diese “wissenschaftlichen Grundprinzipien und Normen eigentlich ökonomischen Verwertungsinteressen zu widersprechen scheinen” \cite{hanekop_2006}. Darüber hinaus haben die Forscher in vielen Fällen wenig oder keine Verantwortung für den Einkauf der wissenschaftlichen Informationen, die er oder sie "verschenkt" \cite{steele_2006}. Die Einführung der Zitationsregister und Impact Faktoren, sowie die Definition der Kernzeitschriften, hat zur weitgehenden Erstarrung des wissenschaftliche Zeitschriftenmarktes geführt und gleichzeitig die Kapazität der kommerziellen Verlagen, sowie deren Gewinnmargen ansteigen lassen \cite{CREATe_2014}. Die Steuerungsmechanismen werden über die Messbarkeit mittels Methoden direkt oder indirekt ausgeübt. Dabei stehen insbesondere die Methoden, die auf der quantitativen Grundlage der Zitationsraten wissenschaftlicher Publikationen gemessen werden in der Kritik \cite{Dong_2005} und auch andere Indikatoren für die Messung von Forschungsleistungen sind hoch umstritten \cite{Hornbostel_1997} \cite{Hicks_1996} \cite{Havemann_2002}. Der Hauptkritikpunkt: Die Verfahren, um die Wirkung von Wissenschaft und damit auch die Reputation von Wissenschaftlern zu messen, sind kein eigentliches wissenschaftliches Produkt\cite{suchen} und erfassen zum Beispiel die Tätigkeit einzelner Forschergruppen zu stark \cite{schmoch_2009}. Darüber hinaus sind "weder importance noch impact noch quality sind direkt meßbar" und man kann sich ihnen nur "nähern" \cite{Hornbostel_1997}. Das führt unter anderem dazu, dass der aus der "Zahl der Zitationen aud die Beiträge einer Zeitschrift ermittelte" \cite{weishaupt_2009_goldenOA} Impact Factor “kein perfektes Werkzeug (ist) um die Qualität der Artikel zu messen” und trotzdem wird er zur Bewertung von Wissenschaft genutzt, denn “(...) es gibt nichts Besseres, und er hat den Vorteil, dass er bereits lange existiert und ist daher eine gute Technik für die wissenschaftliche Bewertung”\cite{garfield_1999} \cite{weishaupt_2009_goldenOA}. 

Die Kritik am Impact Faktor lässt sich laut der Bibliotheks- und Informationswissenschaftlerin Dr. Karin Weishaupt, am Beispiel des "Thomson Reuters Journal Citation Factors" in sechs Punkten zusammenfassen \cite{weishaupt_2009_goldenOA}:
\begin{enumerate}
\item Der Impact Factor bezieht sich immer auf die gesamte Zeitschrift und hat somit keine Aussagekraft über die "Rezeption oder Qualität des einzelnen Artikels" .
\item Der Impact Factor lässt berücksichtigt nur die Zeitrschriften, die im eigenen Index gelistet sind und enthält weder Monographien, Tagungsbeiträge, sonsitge Beiträge oder Internetquellen.
\item Durch Selbstzitierungen sind Manipulationen möglich.
\item Es werden nur Zitate aus den letzten beiden Jahren berücksichtigt und je nach Fachgebiet ist es von Vorteil wenn im eignene Gebiet die Verwertungszyklen kürzer sind.
\item Publikationen die nicht in englischer Sprache verfasst sind haben es schwieriger, da englische Journale überproportional vertreten sind
\item Spezialisierte Zeitschriften sind ebenfalls systematisch benachteiligt gegenüber Journalen von großen Fach-Communities oder Journalen mit Überblicksartikeln.
\end{enumerate}

Es bleibt festzuhalten dass die im wissenschaftlichen System genutzten Indikatoren die "komplexe Realität" der Leistungsbewertung in der Wissenschaft nicht abbilden können und sie demgegenüber eine "eine eigene Realität" konstruieren \cite{Hornbostel_1997}. Versteht man Wissenschaft als soziales System, so stellen "Reputation" und nicht die Wahrheit der Beobachtungen und Erklärungen "nicht selten auch eingestandenes vorrangies Ziel wissenschaftlicher Tätigkeit" dar \cite{luhmann_1970_selbststeuerung}. Wie “gering der Wirkungsgrad” und die Methoden zur Messung “zur Reproduktion des traditionellen wissenschaftlichen Diskurses ausfall(en), wird von dem Moment an klar, an dem ein neues und offenes Kommunikationsmedium wie das Internet als alternativer Publikations- und Verbreitungskanal für Wissenschaft zur Verfügung steht" \cite{Rost_1998}. 

\section{Zusammenfassung und Ableitung von Anknüpfungspunkten für die empirische Untersuchung}
Viele der Erklärungsansätze für den Paradigmenwechsel hin zur Öffnung der Wissenschaft basieren auf Annahmen, in denen ein direkter Zusammenhang von technischen Entwicklungen unmittelbar auf (wissenschafts-)politische und kulturelle Bewegungen geschlossen werden. Darüber hinaus beschränkt sich die Perspektive primär auf den Zugang zum Ergebnis von Wissenschaft und weniger auf die Öffnung des gesamten Prozesses. Die theoretische Auseinandersetzung mit der Geschlossenheit des wissenschaftlichen Diskurses  auf der Einen und mit den Treibern und Bremsern im realen wissenschaftlichen Prozess werden in der gängigen Literatur auf der anderen Seite, wird nicht genügend berücksichtigt. Hier wird vor allem die Verbindung zwischen wissenschaftlicher Reputation und Geschlossenheit des Wissensproduktionsprozesses nur selten erörtert. Als weiteres Manko kann angeführt werden, "dass die Deliberation bezüglich und die Verbreitung von Wissen ein stabiles Set von Infrastrukturen braucht"\cite{kelty_2004}, nach denen man heute noch vergeblich sucht. Das Potenzial bei der Verwendung von digitalen Technologien um Wissenschaft offen zu teilen, ist nicht annährend ausgeschöpft und es "besteht eine erhebliche Diskrepanz zwischen der Idee der offenen Wissenschaft und wissenschaftliche Realität" \cite{Scheliga_2014}.
