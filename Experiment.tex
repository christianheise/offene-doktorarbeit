\chapter{Experimentelle Untersuchung: Offenes Verfassen einer Dissertation}

Um vor allem die Kriterien und Argumente (zum Beispiel Aufwand) für oder gegen das offene Publizieren prüfen zu können und Handlungsempfehlungen für das offene Schreiben von wissenschaftlichen Arbeiten am Beispiel von Dissertationen erstellen zu können, wurde für diese Arbeit als Selbstexperiment eine offene Schreibweise gewählt. Dabei wurde die Arbeit direkt und unmittelbar während der Erstellung für jeden, jederzeit frei zugänglich auf einer Webseite (http://offene-doktorarbeit.de) im Internet veröffentlicht.

Ziel des Experiments ist auch die Analyse ob und wie weit die offene Erstellung der Doktorarbeit möglich ist. Dieses Vorgehen soll helfen weitere Treiber und Bremser, sowie exemplarisch den Aufwand für offene Verfassen einer wissenschaftlichen Arbeit zu identifizieren. Darüber hinaus soll die Praxistauglichkeit der Forderung nach Öffnung an das wissenschaftliche Kommunikationsystem analysiert werden.

Zur Erreichung des Ziels wurde ein (auto-)ethnographischer Ansatz gewählt. Dieser ist durch drei Merkmale gekenntzeichnet: Erstens sucht sie  "einen primär verstehenden Zugang", zweitens geht mit ihr (inzwischen) ein stark gebrochener Holismus einher und drittens sind "Ethnographien durch einen Methodenmix gekennzeichnet" \cite{bachmann_2011_ethnographie}. Dieser Methodenmix eignet sich für die Herangehensweise des Felds der Science and Technology Studies.

\section{konzeptionelle und technische Rahmenbedingungen}

Konzeptionell war das Projekt so angelegt, dass die Arbeit und alle Daten unter allen Umständen jederzeit frei und offen im Internet einsehbar sein sollte. Die Bedingungen untern denen diese Arbeit erstellt wird, sollen sich so nahe wie möglich an den Forderungen von Open Science und den genannten Erklärungen zur Öffnung wissenschaftlicher Kommunikation orientieren. Nach der rechtlichen Klärung der offenen Schreibweise, sollte die Arbeit und alle damit verbundenen Tätigkeiten so schnell und umfassend wie möglich jederzeit im Internet abrufbar sein.

Um die Eigenleistung und die Selbstständigkeit bei der Erstellung der wissenschaftlichen Qualifikationsarbeit zu gewährleisten, wurde technisch sichergestellt, dass es für andere als den Autoren keine Möglichkeit gab, den erstellten Inhalt zu editieren oder zu kommentieren. Im Gegenteil, sie ermöglicht eine neue Form, die Eigenständigkeit direkt während der wissenschaftlichen Arbeit und Erstellung des Inhalts darzustellen.

Initial kam dafür ein Blog auf der Grundlage der Open Source-Lösung Wordpress zum Einsatz. Blogs, auch Weblogs genannt, beschreiben eine Reihe von Softwarelösungen, die nach der Installation Internetnutzern einfach ermöglichen Einträge im Internet zu veröffentlichen. Wordpress wurde zunächst als eine solche Blogging-Plattform entwickelt, in den letzten Jahren hat sich das System aber zu einem umfangreichen Content Management System weiterentwickelt \cite{Patel_2011_cms}. Content Management Systeme ermöglichen nicht nur die Darstellung von Texten in chronologischer Reihenfolge, sondern auch die Ablage und Organisation von Daten oder anderen Medien. Einer der wesentlichen Vorteile von Wordpress ist die große Anzahl von Plug-Ins \cite{Patel_2011_cms}. Über die Plug-Ins kann jeder Aspekt einer Wordpress Webseite in Bezug auf die Erstellung, Organisation und Optimierung von Inhalten mit dem Einsatz von Plug-Ins erweitert werden. Sie werden von unabhängigen Programmierern entwickelt und meist unter einer open source Lizenz freigegeben. Da die verwendete Software nicht nur für die Dokumentation rund um die Arbeit dienen, sondern auch als technische Plattform für die Veröffentlichung der gesamten wissenschaftlichen Arbeit selbst zur Verfügung stehen sollte, erschien Wordpress als die beste Lösung \cite{Jones_2013_CMS}.

In einem weiteren Schritt sollten die Inhalte in einem Dokument auf Google Docs im Blog eingebunden und offen zur Verfügung gestellt. Google Docs, ist ein kostenloses, webbasiertes Textverarbeitungssystem der Firma Google. Es ist angelehnt an die gängigen Programme von Microsoft Office oder Open Office, bietet aber einige Einschränkungen besonders für das wissenschaftliche Publizieren. So fehlen bei Google Docs die Möglichkeiten der strukturierten Ablage von Daten und die einfache Verwaltung von Quellen und referenzen. Vor der Erstellung der Arbeit wurden unterschiedliche technische Möglichkeiten für die einfache Darstellung getestet. In der ursprünglichen Analyse wurde wie oben dargestellt die Veröffentlichung der Arbeit in einem Blogsystem präferiert.

Bei der Übertragung der bereits geschriebenen Inhalte in das Blogsystem, stellte sich schnell heraus, dass die Blog-Software für die Veröffentlichung der gesamten Arbeit in einzelnen Blogposts unhandlich und unzureichend war. Zwar ermöglichten zusätzliche Anpassungen an dem System (Plug-Ins) die Veröffentlichung von Inhalten in wissenschaftlichen Formen und Formaten, dennoch stieß das eingesetzte System schnell an seine Grenzen. Bei der Internetrecherche nach Alternativen Lösungswegen wurde das noch junge System Authorea ausgewählt und die Inhalte aus dem bisherigen System dahin konvertiert und übertragen.

\section{Herausforderungen der offenen Anfertigung der Dissertation}

Bei der Erstellung der Arbeit unter der Voraussetzung, dass diese jederzeit über das Internet abrufbar ist, stellten sich unterschiedliche Herausforderungen. Da es sich hierbei um den ersten Versuch eines offenen Promotionsverfahren handelte, konnte nicht auf Erfahrungswerte zurückgegriffen werden.

Damit die Kriterien und Argumente (zum Beispiel Aufwand) für oder gegen das offene Publizieren geprüft werden können und gegebenenfalls Handlungsempfehlungen für das offene Schreiben von wissenschaftlichen Arbeiten am Beispiel von Dissertationen erstellt werden können, werden im Folgenden die rechtlichen und technischen Herausforderungen dokumentiert und strukturiert zusammengefasst.

\subsection{Rechtliche Herausforderungen}

Die Promotionsordnung der Fakultät Kulturwissenschaften der Leuphana Universität mit dem Stand vom 02.02.2011 untersagte nicht ausdrücklich das offene Verfassen einer Dissertationsarbeit, erlaubte das aber auch nicht explizit. Um sicherzustellen, dass die offene Schreibweise nicht gegen die Regeln der Promotionsordnung der Fakultät verstößt, wurde Anfang des Jahres 2013 ein Schreiben an Promotionskommission geschickt und um eine Erlaubnis der zeitgleichen Einsehbarkeit des aktuellen Stands meiner Arbeit gebeten.

Um den Anforderungen der aktuell geltenden Prüfungsordnung der Leuphana Universität zu entsprechen, wurden vorab in einem Schreiben an die Promotionskommission die Bedingungen für die offene Erstellung der abgestimmt und die Vereinbarkeit mit der Promotionsordnung geprüft. Die Promotionskommission hat am 12. Dezember 2013 dem Gesuch die Arbeit "offen" verfassen zu dürfen, nach der rechtlichen Prüfung durch das Justitiariat der Universität, mehrheitlich für eine direkte und unmittelbare Veröffentlichung des Schreibprozesses entsprochen. Die gewonnene Transparenz während des Erstellungsprozesses stellt in diesem Fall keinen Widerspruch zu der Selbständigkeit bei der Ausarbeitung dar. Die Kommission, empfahl darüber hinaus der nachfolgenden Promotionskommission, Entstehungsform der Dissertation anzunehmen. Dennoch machte der Vorsitzenden eine Mitteilung zur Unsicherheit dieser Art der Veröffentlichung, da abschließend "voraussichtlich eine Promotionskommission unter anderer Zusammensetzung die Annahme der Dissertation zu prüfen und beschließen hat".

Um den Vorraussetzungen der Open Definition und den Forderungen in den Erklärungen für die Offenheit im wissenschaftlichen Kommunikationsprozess (Three BBB's) gerecht zu werden, wurden die Inhalte der Arbeit unter einer Creative Commons Lizenz lizenziert. Die Creative Commons (CC) Lizenzen wurden von dem Stanford University Jura-Professor Lawrence Lessig und andere hatten im Jahr 2002 entwickelt und als eine Reihe von Urheberrechtslizenzen mit unterschiedlichen Eigenschaften kostenlos für die Öffentlichkeit veröffentlicht \cite{Minjeong_2007}. Die CC-Lizenzen erfreuen sich seit dem steigender Beliebtheit. Im Oktober 2004 waren 5 Millionen Werke unter einer CC-Lizenz verfügbar /cite{Suchen Forbes: Movement Seeks Copyright Alternatives}. Nach eigenen Angaben von Creative Commons (https://stateof.creativecommons.org/) stieg die Anzahl der unter CC-lizensierten Werke auf 50 Millionen im Jahr 2006, 400 Millionen in 2010 und 882 Millionen in 2014. Seit 2010 ist auch ein Shift hin zu offenen Lizenzmodellen innerhalb der CC-Lizenzen ersichtlich. Waren 2010 noch 60 Prozent der 400 Millionen Werke unter den restriktiven CC-Lizenzen veröffentlicht, sang der Anteil in 2014 auf 44 Prozent. Im Rahmen dieser Arbeit kommt eine Creative Commons Lizenz unter den Bedingungen "Weitergabe unter gleichen Bedingungen 3.0 Unported" zum Einsatz, mit knapp einem Drittel Anteil die am häufigsten verwendete CC-Lizenz.

---- TODO: Vorveröffentlichung (Inhalte, Daten) durch Dritte ----

\subsection{Technische Umsetzung und Herausforderungen}

Die Arbeit wurde, bis zur Klärung der Erlaubnis durch die Promotionskommission im Dezember 2013, in einem geschlossenen Google Dokument ohne Freigaben verfasst.

An folgenden Aspekten scheiterte der Einsatz der Blogsoftware als Publikationsplattform:
\begin{itemize}
\item Die Darstellung der einzelnen Kapitel als Blogposts hat sich als sehr aufwendig und insofern Schreibprozess als unpraktikabel herausgestellt
\item Eine einfache standardisierte Referenzierung von Literaturverweisen ist nicht möglich
\item Fußnoten können nicht über mehrere Einträge hinweg zusammenhängend dargestellt werden, eine Zuordnung von Seitenzahlen stellte ebenfalls eine Herausforderung dar
\item Der Export in ein lesbares Dokument ist immer mit Aufwand verbunden
\item Die strukturierte Eingabe der Inhalte ist ebenfalls nur unter zusätzlichem Aufwand möglich
\item Bei der Anzahl an unterschiedlichen Revisionen wäre das Revisionssystem sicher an seine Grenzen gestoßen
\item ...
\end{itemize}

Da keine standardisierte Lösung für das offene Verfassen wissenschaftlicher Arbeiten verfügbar ist und die gängigen web-basierten Softwarelösungen zur Textverarbeitung nicht den Ansprüchen für wissenschaftliche Arbeit genügen, wurde mehrmals die Plattform gewechselt. Nachdem sich der Einsatz von Worpdress und GoogleDocs als unzureichend herausgestellt hatte, wurde die Arbeit im August 2014 auf die Platform Authorea übertragen.

Authorea ist eine Webanwendung die im Browser die kollaborative Textverarbeitung ermöglicht und die Ablage der Dokumente in einem Repositorium ermöglicht. So können wissenschaftliche Texte webbasiert verfasst und öffentlich dargestellt werden. Die technische Infrastruktur hinter der Plattform ermöglicht es, bei der Erstellung der Arbeit alle Zwischenstände und Veränderungen an dem Text abzulegen, zu dokumentieren und darzustellen. Dieses Revisionssystem kann optional auch an die Softwareentwicklungplattform GitHub angebunden werden, denn auch hier wird bei jedem Speichern eine Version der Arbeit als eigene Revisionen abgespeichert.

Da in dieser Arbeit Parallelen zwischen der Öffnung der Softwareentwicklung im Rahmen der Open-Source-Bewegung und der Öffnung von wissenschaftlicher Kommunikation gezogen werden, war es naheliegend gängige Tools und Umgebungen für die Entwicklung von Software auch für die eigene wissenschaftliche Arbeit zu einzusetzen. Die Arbeit wurde im Erstellungsprozess nach dem Umzug auf Authorea auch zusätzlich mit einem GitHub Repositorium verknüpft.

Diese Verknüpfung der beiden Anwendungen ermöglichte:
\begin{itemize}
\item die automatische, dezentrale Sicherung und Archivierung der Arbeit auch außerhalb von Authorea.
\item die Erstellung und Synchronisierung eines lokalen Abbilds der gesamten Arbeit und der Daten auf dem Rechner über den GitHub Desktop-Client
\item die Bearbeitung und Synchronisation der lokalen Inhalte mit denen auf GitHub und Authorea, ohne auf die Vorteile des Revisionssystems (Darstellung der einzelnen Schritte der Erstellung der Arbeit) verzichten zu müssen
\item die Verbindung mit GitHub ermöglicht über mobile Applikationen die Bearbeitung der Arbeit von unterwegs
\item die Arbeit auch ohne Internetzugang zu bearbeiten und die Veränderungen an der Arbeit trotzdem detailliert und transparent darzustellen.
\item ein Monitoring des Fortschritts durch die statistische Aufbereitung und Darstellung der Entwicklung und Veränderung im Erstellungsprozess der Promotionsarbeit
\end{itemize}

Für die Erstellung wurde das Textsatzsystem TeX mit der Makrosprache Lamport Tex (LaTeX) verwendet. LaTeX ist ein Layoutsystem, das besonders für wissenschaftliches Veröffentlichen geeignet ist. Im Gegensatz zu gängigen Textverarbeitungsprogrammen ermöglicht dieses System die Arbeit an Textdateien, die an bestimmten stellen so ausgezeichnet werden, dass sie später als strukturierter Datensatz in jede Mögliche Form übertragen werde kann. Während Textverarbeitungsprogrammen (wie zum Beispiel Microsoft Word) auf dem "What you see is what you get" (WYSIWYG), zählt man LaTeX zu den sogenannten Markup-­Sprachen beziehungsweise Auszeichnungssprachen, die nicht innerhalb einer bestimmten Umgebung verwendet werden müssen \cite{Sievers_2012}. Diese Art des Textsatzes ist vor allem dann sinnvoll, wenn die finale Verwertung oder Ausgabe des Inhalts unbekannt oder variabel ist  \cite{braune_2007_latex}.

Nachdem die Arbeit einen gewissen Umfang erreicht hat, kam es zu Problemen bei der Darstellung der Webseite über Authorea. Nach Rücksprache mit den Entwicklern stellte sich heraus, dass die Arbeit auf Grund ihres umfangs nicht mehr beziehungsweise nur noch unregelmäßig geladen und dargestellt werden konnte. Grund dafür waren unzureichende Ressourcen für die Umwandlung von LaTeX in eine browserkompatible HTML-Darstellung. Demnach musste eine alternative Darstellung gewählt werden, die den Text aus dem GitHub-Repositorium zusammenfassend darstellt.

Zur Lösung wurde für die alternative Live-Darstellung der Arbeit ein eigener Konverter (Reader) programmiert, der den LaTeX-Datensatz aus dem Repositorium live importiert und in einer einfachen HTML-Ansicht unter http://live.offene-doktorarbeit zur Verfügung stellt. Mit einer eigenen PHP Programmierung wurden dafür die einzelnen .tex-Dokumente über die Schnittstelle von GitHub aus dem Repositorium ausgelesen und konvertiert. Über diesen Konverter konnten die LaTeX-Auszeichnungen in HTML interpretiert und im Browser dargestelltwerden. Die Bibliographie wurde über die Roh-Daten-Schnittstelle von GitHub an eine bestehende Open Source Softwarelösung (Bibtexbrowser), die die Literaturangaben strukturiert darstellbar macht. Diese Lösung ermöglicht aus den Texten auch die Literaturquellen direkt angesteurt und mit Metainformationen ausgegeben werden. Die Forschungsdaten wurden zusätzlich im GitHub Repositorium abgelegt und wurden aus der eigenen Readerlösung verlinkt. Die individuelle Programmierung ermöglichte das Einsehen der Arbeit, der Bibliographie und der Daten, auch ohne Authorea. Die Inhalte wurden zwischen GitHub und Authorea automatisch syncronisiert und waren stehts aktuell. Der Reader ermöglichte es jederzeit auf den aktuellen Stand der Texte und Daten zuzugreifen. Der Quellcode für die Reader-Applikation wurde ebenfalls im Repositorium zu dieser Arbeit veröffentlicht.

Obwohl nach Evaluation aller webbasierten Schreibplattformen zunächst eine geeignete Lösung gefunden wurde, war es ohne eigene Programmierarbeit nicht möglich den gesamten Inhalt der Arbeit dauerhaft öffentlich verfügbar zu machen. Demnach wird davon ausgegangen, dass der Erstellungsprozess einer Promotion schon an den aktuellen technischen Möglichkeiten von Standardlösungen scheitern kann, wenn der Wissenschaftler nicht mit genügend Kenntnissen zur Bewältigung der Herausforderungen ausgestattet ist.

\subsection{Vor- und Nachteile}

Vorteilhaft kann sich die offenen Schreibweise in den Fällen erweisen, in denen eine öffentliche Diskussion die Arbeit des Forschers oder der Forscherin positiv beeinflusst. Im Rahmen der Erstellung einer Promotion ist das nur begrenzt von Vorteil, da die Arbeit "selbstständig" erstellt werden muss. Es muss sich herausstellen, in welchem Umfang Kommentare oder kollaborative Schreibweisen diese Selbstsändigkeit gefährden oder ein "unerlaubtes Hilfsmittel" darstellen. Bei der Erstellung dieser Qualifikationsarbeit wurde die Kommentar- und kollaborative Schreibfunktionen, die bei Authorea zur Verfügung standen, proaktiv deaktiviert um die Vorraussetzungen der Promotionsordnung vollumfänglich zu erfüllen. Einzige Möglichkeit den Autor zu kontaktieren war per E-Mail und die Möglichkeit auf der Webseite http://offene-doktorarbeit.de Literaturempfehlungen per Mail einzusenden. Davon wurde zwei Mal gebrauch gemacht.

Die Aufmerksamkeit für die offene Schreibweise der Doktorarbeit im direkten Umfeld des Autors war groß. Inhaltlich hatte das zwar kaum Effekte, dennoch stieg die Anzahl der regelmäßigen Nachfragen bezüglich des Bearbeitungstandes der Arbeit, nach dem der Text einsehbar war. Der soziale Druck die Arbeit voranzubringen stiegt ebenfalls. Hinweise auf Fehler inhaltlicher oder rechtschreibtechnischer Natur blieben weitestgehend aus. Im Schnitt wurde die Webseite http://offene-doktorarbeit.de von 300 Besuchern pro Monat aufgerufen. Die Anzahl der Nutzer stieg fast linear mit der Anzahl der Blogbeiträge und dem Fortschritt beim Schreibprozess der Arbeit.

---- TODO: Grafik bauen ----

Nachteile ergaben sich aus der fehlenden Verfügbarkeit von einfachen technischen, rechtlichen und konzeptionellen Standards bei der offenen Erstellung wissenschaftlicher Texte und bei der Veröffentlichung von Forschungsdaten. Mit LaTeX ist zwar ein Textsatz für Experten vorhanden, allerdings fehlt es an einem einfachen Frontend mit dem Funktionsumfang eines gängigen Texrverarbeitungsprogramms um dieses System offen und zu jedem Zeitpunkt verfügbar zu nutzen. Ergänzend war mit BibTeX ein Literaturdatenbanksystem für Literaturangaben für LaTeX-Dokumente verfügbar. Auch für BibTex gibt es kein einfaches Frontend und die Bearbeitung ist eher aufwändig.

---- TODO: gewohnte Arbeitsweise gegenüberstellen. Z.B: Wer gewohnt ist mit Literaturverwaltungsprogramme wie Citavi oder Zotero zu arbeiten, wird auf Grund die fehlenden Implementierung dieser Programme  durch mehr manuelle Arbeit kompensieren müssen ----

Die Arbeit jederzeit in jedem Zustand für jeden im Internet einsehbar zu halten scheint im am Anfang befremdlich, ist aber eine Gewöhungsfrage. Die Befürchtung im Bearbeitungszustand falsche oder fehlerhafte Inhalte zu verbreiten, stellt ebenfalls eine überwindbare Hürde bei der offenen Schreibweise dar. Als Nachteil gegenüber der analogen und geschlossenen Arbeit, ist der Aufwand für die Anonymisierung und Veröffentlichung der Umfrage-Daten kurz nach Abschluss der Erhebung zu deklarieren. Auch die Abklärung der rechtlichen Rahmenbedingungen nimmt bisher viel Zeit in Anspruch, ist aber ebenfalls überwindbar.

Die Gefahr, dass Inhalte fehlinterpretiert werden oder vorab wissenschaftlich anerkannt "veröffentlicht" werden und die Arbeit somit nicht mehr als unveröffentlichte Leistung anerkannt wird, konnte während des Verfassens dieser Arbeit nicht bestätigt werden. Dennoch sollte dieser Umstand bei zukünftigen Vorhaben beachtet werden.

\section{Kritische Betrachtung und Alternative Vorgehensweisen}

---- TODO: Alternativen technisch wie konzeptionell aufzeigen ----

\section{Ergebnis}

Bisher muss der Aufwand für die Erstellung einer wissenschaftlichen Arbeit in einer geschlossenen Umgebung auf dem eigenen Rechner als geringer eingeschätzt werden, als die Texterstellung im Internet unter einer offenen Lizenz für jeden jederzeit einsehbar. Trotz mehrfachen Wechsels der Software, konnte keine einfache Lösung gefunden werden, die der Bedienbarkeit und Flexibilität der geschlossenen wissenschaftlichen Textbearbeitung auf dem Desktop entspricht.

Es ist davon auszugehen, dass der Forderung ---- Todo: Budapest usw. ---- heute ohne erheblichen Mehraufwand noch nicht nachgekommen werden kann. Nur mit genügend Kenntnissen kann der offene Schreibprozess und die zeitnahe und umfassende Veröffentlichung von Daten nach wissenschaftlichen Maßstäben erfolgen. Demnach müssen Wissenschaftler und Wissenschaftlerinnen in der Lage sein, die technischen Begrenzungen zu überwinden und die Rahmenbedingungen für den offenen Schreibprozess selbst zu schaffen. Weder die Forschungsinstitutionen noch private Anbieter sind bisher in der Lage Platformen für die Öffnung des gesamten wissenschaftlichen Prozesses anzubieten. Zu unterschiedlich sind die Anforderungen in den verschiedenen Disziplinen, zu mannigfaltig die Funktionen in den vorhanden Applikationen, zu uneinheitlich die Standards für das digitale Publizieren und zu verschieden der Kenntnisstand bei der Verwendung digitaler Methoden und Tools.

Insgesamt müssen die Vorteile und Nachteile der offenen Schreibweise ausgewogen betrachtet werden. Die offene Erstellung dieser Arbeit hat gezeigt, dass der Forderung nach Öffnung der wissenschaftlichen Kommunikation im Rahmen einer Promotionsarbeit grundsätzlich entsprochen werden kann. Letztendlich, so das Ergebnis des eigenen Experiments, sind durch die offene Schreibweise bisher weder fundamentale Vorteile, noch unlösbare Hürden für den publizierenden Wissenschaftler absehbar. Weitere Experimente mit der offenen Forschungsarbeit sind notwendig um abschließend zu evaluieren, ob eine solche Art des Verfassens von Forschungs- und Qualifizierungsarbeiten einen fundamentalen Vorteil für die Wissenschaft und die Öffentlichkeit bringen.

---- TODO: ausarbeiten ----
