\chapter{Experimentelle Untersuchung: Offenes Verfassen einer Dissertation}

Um vor allem die Kriterien und Argumente (zum Beispiel Auwand) für oder gegen das offene Publizieren prüfen zu können und Handlungsempfehlungen für das offene Schreiben von wissenschaftlichen Arbeiten am Beispiel von Dissertationen erstellen zu können, wurde für diese Arbeit als Selbstexperiment eine offene Schreibweise gewählt. Dabei wurde die Arbeit direkt und unmittelbar während der der Erstellung für jeden, jederzeit frei zugänglich auf einer Webseite (http://offene-doktorarbeit.de) im Internet veröffentlicht.

\section{konzeptionelle und technische Rahmenbedingungen}

Konzeptionell war das Projekt so angelegt, dass die Arbeit unter allen Umständen jederzeit frei und offen Verfügbar einsehbar sein sollte. Inital sollte ein Blog auf der Grundlage der verbreiteten Open Source-Lösung Wordpress zum Einsatz kommen und nicht nur die Dokumentation rund um die Arbeit, sondern auch als technische Plattform für die gesamte Arbeit selbst zur Verfügung stehen.

Ziel des Experiment ist die Analyse ob und wie weit die offene Erstellung der Doktorarbeit möglich ist. Dieses Vorgehen soll helfen weitere Treiber und Bremser, sowie exemplarisch den Aufwand für offene Verfassen einer wissenschaftlichen Arbeit zu identifizieren.

\section{Herausforderungen der offenen Anfertigung der Dissertation}

Die Arbeit wurde, bis zur Klärung der Erlaubnis durch die Promotionskommission im Dezember 2013, in einem geschlossenen Google Dokument ohne Freigaben verfasst. Google Docs, ist ein kostenloser, webbasiertes Textverarbeitungssystem der Firma Google. Es ist angelehnt an die gängigen Programme von Microsoft Office oder Open Office, bietet aber einige Einschränkungen besonders für das wissenschaftliche Publizieren.

Nach der schriftlichen Erlaubnis durch die Promotionskommission vom 11. Dezember 2013, wurde in einem ersten Schritt das Google Dokument offen zur Verfügung gestellt beziehungsweise für jeden lesbar freigegeben und aus dem Blog verlinkt. Bei der Übertragung der bereits geschriebenen Inhalte in das Blogsystem, stellte sich schnell heraus, dass die Blog-Software für die Veröffentlichung in einzelnen Blogposts nicht geeignet war. Zwar ermöglichten zusätzliche Plugins die Veröffentlichung von Inhalten in wissenschaftlichen Formen und Formaten, aber an folgenden Aspekten scheiterte der Einsatz der Blogsoftware als Publikationplattform:
\begin{itemize}
\item Die Darstellung der einzelenen Kapitel als Blogposts hat sich als sehr aufwendig und insofern Schreibprozess als unpraktikabel herausgestellt
\item Eine einfache standardisierte Referenzierung von von Literaturverweisen ist nicht möglich
\item Fußnoten können nicht über mehrere Einträge hinweg zusammenhängend dargestellt werden, eine Zuordnung von Seitenzahlen stellte ebenfalls eine Herausforderung dar
\item Der Export in ein lesbares Dokument ist immer mit Aufwand verbunden
\item Die strukturierte Eingabe der Inhalte ist ebenfalls nur unter zusätzlichem Aufwand möglich
\item Bei der Anzahl an unterschiedlichen Revisionen wäre das Revisionssystem sicher an seine Grenzen gestoßen
\item ...
\end{itemize}

Da keine standardisierte Lösung für Verfassen offener Arbeiten vorherrscht und die gängigen System nicht den Ansprüchen für wissenschaftliche Arbeit genügen, wurde mehrmals die Plattform gewechselt. Nachdem sich der Einsatz von GoogleDocs und Worpdress als unzureichend herausgestellt hatte, wurde die Arbeit im August 2014 auf Authorea übertragen. Authorea ist eine Webanwendung die im Browser die kollaborative Textverarbeitung ermöglicht und die Ablage der Dokumente in einem Repositorium ermöglicht. So können wissenschaftliche Texte webbasiert verfasst und öffentlich dargestellt werden. Die technische Infrastruktur hinter der Plattform ermöglicht es, bei der Erstellung der Arbeit wurden alle Zwischenstände und Veränderungen an dem Text abzulegen, zu dokumentieren und darzustellen. Dieses Revisionssystem kann optional auch an die Softwareentwicklungplattform GitHub angebunden werden, denn auch hier wird bei jedem Speichern eine Version der Arbeit als eigene Revisionen abgespeichert. Da in dieser Arbeit Parallelen zwischen der Öffnung der Softwareentwicklung im Rahmen der Open-Source-Bewegung und der Öffnung von wissenschaftlicher Kommunikation gezogen werden, war es naheliegend gängige Tools und Umgebungen für die Entwicklung von Software auch für die eigene wissenschaftliche Arbeit zu einzusetzen. Die Arbeit wurde also im Erstellungsprozess nach dem Umzug auf Authorea auch zusätzlich mit einem GitHub Repositorium verknüpft.

Diese Verknüpfung der beiden Anwedungen ermöglichte:
\begin{itemize}
\item die automatische, dezentrale Sicherung und Archivierung der Arbeit auch ausserhalb von Authorea.
\item die Erstellung und Synchronisierung eines lokalen Abbilds der gesamten Arbeit und der Daten auf dem Rechner über den GitHub Desktop-Client.
\item die Bearbeitung und Synchronisation der lokalen Inhalte mit denen auf GitHub und Authorea, ohne auf die Vorteile des Revisionssystems (Darstellung der einzelnen Schritte der Erstellung der Arbeit) verzichten zu müssen.
\item die Arbeit auch ohne ohne Internetzugang zu bearbeiten und die Veränderungen an der Arbeit trotzdem detailiert und transparent darzustellen.
\item ein Monitoring des Fortschritts durch die statistische Aufbereitung und Dartstellung der Entwicklung und Veränderung im Erstellungsprozess der Promotionsarbeit
\end{itemize}


Bisher kann der Aufwand für die Erstellung einer wissenschaftlichen Arbeit in einer geschlossenen Umgebung auf dem eigenen Rechner als geringer eingeschätzt werden, als sie im Internet unter einene offenen Lizenz für jeden jederzeit einsehbar zu verfassen. Trotz mehrfachen Wechsels der Software, konnte keine einfache Lösung gefunden werden, die der Bedienbarkeit und Flexibilität der geschlossenen wissenschaftlichen Textbearbeitung auf dem Desktop entspricht. Die Anonymisierung und Veröffentlichung der Umfrage-Daten kurz nach Abschluss der Erhebung stellte ebenfalls einen Mehraufwand dar.
