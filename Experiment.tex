\chapter{Experimentelle Untersuchung: Offenes Verfassen einer Dissertation}

Um vor allem die in der Literaturrecherche und der Befragung extrapolierten Katalysatoren und Hindernisse, den vermuteten Aufwand sowie damit verbundenen Bedingungen für das offene Publizieren prüfen zu können und Handlungsempfehlungen für das offene Schreiben von wissenschaftlichen Arbeiten am Beispiel von Dissertationen erstellen zu können, wurde für diese Arbeit eine offene Schreibweise als Selbstexperiment gewählt. Dabei wurde die Arbeit direkt und unmittelbar während der Erstellung für jeden, jederzeit frei zugänglich auf einer Webseite (http://offene-doktorarbeit.de) im Internet veröffentlicht.

Diese experimentelle Untersuchung hat das Ziel darzustellen ob und wie weit die offene Erstellung einer Doktorarbeit unter den Kriterien und den Forderungen von Open Science möglich ist. Sie ermöglicht die Ergänzung der bisher dargestellten Entwicklung und Erkenntnisse um die Praxis und die Auseinandersetzung mit der "Wünschbarkeit von Zuständen" \cite{cite:10}. Das Vorgehen soll helfen, weitere Treiber und Bremser für Veränderungen im wissenschaftlichen Kommunikationssystem zu identifizieren, sowie exemplarisch den Aufwand für offene Verfassen einer wissenschaftlichen Arbeit darzustellen und als Beispiel für ein konkretes Vorhaben der Öffnung von Wissenschaft und Forschung dienen. Darüber hinaus soll die Praxistauglichkeit der Forderung nach Öffnung des wissenschaftlichen Kommunikationssystems im Erstellungsprozess dieser Arbeit dargestellt und analysiert und die Ergebnisse der Befragung um die praktische Herangehensweise erweitern werden.

Folgende detaillierte Fragestellungen sollen im Rahmen des Experiments bearbeitet werden:
\begin{itemize}
\item Wie kann eine offene wissenschaftliche (Qualifikations-)Arbeit angefertigt werden?
\item Welche konkreten Herausforderungen bestehen bei der offenen Anfertigung einer solchen Arbeit?
\item Welche Vorteile und welche Nachteile ergeben sich beim offenen Verfassen wissenschaftlicher Arbeiten?
\item Welche Handlungsempfehlungen für das offenen Verfassen einer wissenschaftlichen (Qualifikations-)Arbeit können abgeleitet werden?
\end{itemize}

Für diese Herangehensweise wurde ein (auto-)ethnographischer Ansatz gewählt. Dieser ist primär durch drei Merkmale gekennzeichnet, die bei der Darstellung im Folgenden berücksichtigt werden:  Erstens sucht er  "einen primär verstehenden Zugang", zweitens geht mit ihm (inzwischen) ein stark gebrochener Holismus einher und drittens sind "Ethnographien durch einen Methodenmix gekennzeichnet" \cite{bachmann_2011_ethnographie}. Dieser Methodenmix eignet sich für die Herangehensweise des Felds der Science and Technology Studies und ermöglicht die umfassende Darstellung und die erhöht die Reproduzierbarkeit des Vorhabens.

\section{Methode und konzeptionelle sowie technische Rahmenbedingungen}

Konzeptionell war das Projekt von Anfang an so angelegt, dass die Arbeit und alle Daten unter allen Umständen jederzeit frei und offen im Internet einsehbar sein sollten. Die Bedingungen, untern denen diese Arbeit erstellt wird, sollten sich dabei so nahe wie möglich an den Forderungen von Open Science und den genannten Erklärungen zur Öffnung wissenschaftlicher Kommunikation orientieren. Nach der rechtlichen Klärung der offenen Schreibweise sollten die Arbeit und alle damit verbundenen Tätigkeiten so schnell und umfassend wie möglich jederzeit frei und offen im Internet abrufbar sein.

Initial kam dafür ein Blog auf der Grundlage der Open-Source-Lösung Wordpress zum Einsatz. Blogs, auch Weblogs genannt, beschreiben eine Reihe von Softwarelösungen, die nach der Installation Internetnutzern einfach ermöglichen, Einträge im Internet zu veröffentlichen. Wordpress wurde zunächst als eine solche Blogging-Plattform entwickelt. In den letzten Jahren hat sich das System jedoch zu einem umfangreichen Content-Management-System weiterentwickelt \cite{Patel_2011_cms}. Content-Management-Systeme ermöglichen nicht nur die Darstellung von Texten in chronologischer Reihenfolge, sondern auch die Ablage und Organisation von Daten oder anderen Medien. Einer der wesentlichen Vorteile von Wordpress ist die große Anzahl von Plug-Ins \cite{Patel_2011_cms}. Über die Plug-Ins kann jeder Aspekt einer Wordpress Webseite in Bezug auf die Erstellung, Organisation und Optimierung von Inhalten mit dem Einsatz von Plug-Ins erweitert werden. Sie werden von unabhängigen Programmierern entwickelt und überwiegend unter einer Open-Source-Lizenz freigegeben. Da die verwendete Software nicht nur für die Dokumentation im Rahmen der Erstellung der vorliegenden Arbeit dienen, sondern auch als technische Plattform für die Veröffentlichung der gesamten wissenschaftlichen Arbeit selbst zur Verfügung stehen sollte, erschien Wordpress in der theoretischen Betrachtung als beste Lösung für das Vorhaben.

Zu einem späteren Zeitpunkt sollten die ersten Inhalte in einem Dokument auf Google Docs im Blog eingebunden und offen zur Verfügung gestellt werden. Google Docs, ist ein kostenloses, webbasiertes Textverarbeitungssystem der Firma Google. Es ist angelehnt an die gängigen Programme von Microsoft Office oder Open Office, bietet aber einige Einschränkungen besonders für das wissenschaftliche Publizieren. So fehlen bei Google Docs die Möglichkeiten der strukturierten Ablage von Daten und die einfache Verwaltung von Quellen und Referenzen. Vor der Erstellung der Arbeit wurden unterschiedliche technische Möglichkeiten für die einfache Darstellung getestet. In der ursprünglichen Analyse wurde wie oben dargestellt die Veröffentlichung der Arbeit in einem Blogsystem präferiert und später aus pragmatischen Gründen auf die Schreibplattform Authorea übertragen.

Insgesamt wurde der Prozesse der Wissensschaffung öffentlich dokumentiert und jederzeit einsehbar veröffentlicht. Exemplarisch fand das in den folgenden fünf Phasen des wissenschaftlichen Erkenntnisprozesses statt:
\begin{enumerate}
\item Die \textit{Fragestellung und Planung} der Arbeit wurde im Blog seit August 2012 veröffentlicht. Auf den generellen Übersichtsseiten wurde dort das Vorhaben vorgestellt und regelmäßig zum Stand der Arbeit Beiträge veröffentlicht. Das Exposé für die Doktorarbeit wurde in einem Google Dokument verfasst und ebenfalls in dem Blog eingebunden.
\item Die \textit{Ausführung} der Befragung, des Schreibprozesses (seit Bestätigung durch die Promotionskommission Ende 2013) und des Experiments war zu jeder Zeit offen einsehbar. Auf der jeweiligen Schreibplattform wurde der aktuelle Stand der Arbeit öffentlich festgehalten und Blog wurden die Entwicklungen der Arbeit regelmäßig dokumentiert. Die Befragung wurde ebenfalls auf dem Blog vorgestellt und dokumentiert sowie die Umfragedaten nach Abschluss der Befragung als anonymisierte Rohdaten veröffentlicht. Noch nicht bearbeitete Stellen und unausformulierte Gedanken wurden ebenfalls im Text gesondert gekennzeichnet und jederzeit dokumentiert.
\item Die \textit{Analyse} der Daten wurde auf Grundlage der Rohdaten durchgeführt und direkt in dem Text der Arbeit verarbeitet. Zwischenergebnisse wurden aufbereitet und vorab inklusive der jeweiligen Daten kommuniziert. Die Analyse des Experimets der offenen Anfertigung der Arbeit erfolgte ebenfalls direkt im Text und in Beiträgen auf dem Blog.
\item Das \textit{Auswertungsverfahren} wurde ebenfalls im Blog dokumentiert und direkt in der Arbeit inklusive der Daten veröffentlicht. Die Grafiken und Statistiken wurden ebenfalls gesammelt auf der Webplattform dargestellt und direkt in den Text eingebunden.
\item Die \textit{Verwendung und Kommunikation der Ergebnisse} fand ebenfalls unmittelbar und jederzeit auf dem Blog sowie direkt im Text der Arbeit statt. Das finale Dokument wird abschließend auch auf der Webseite veröffentlicht.
\end{enumerate}

Die Arbeit orientierte sich dabei an der Forderung von Open Science dass der umfassenden Zugriff auf den gesamten wissenschaftlichen Erkenntnisprozess inklusive aller Daten und Informationen, die bereits bei der Erstellung, Bewertung und Kommunikation der wissenschaftlichen Erkenntnisse entstanden sind, jederzeit gegeben ist. Auf der Webseite http://offene-doktorarbeit.de wurde zu jeder Zeit der gesammte Text, sowie die verwendete Literatur aber auch die Ergebnisse der empirischen Arbeit zeitnah veröffentlicht. Somit war es anderen mit einem Internetanschluss möglich im gesammten Verlauf auf den aktuellen Stand und mit dem Wechsel auf GitHub sogar die einzelnen Entwicklungsschritte der Arbeit und Daten zuzugreifen.

Der Forderung, die technischen Entwicklungen zu nutzen, um wissenschaftliche Erkenntnisse aller Art im Rahmen des wissenschaftlichen Erkenntnisprozesses schnellstmöglich offen zu verbreiten, wurde somit entsprochen. Die Informationen wurden nicht nur technisch über die Webseite und die Ablage der Informationen in Repositorien (Zenodo und dem GESIS-Data-Sharing-Repositorium datorium), sondern auch rechtlich über die Wahl einer Open-Definition-kompatiblen Creative-Commons-Lizenz (CC-BY-SA) für andere les- und (weiter-)nutzbar gemacht. Eine direkte Weiternutzung des Textes und der Daten konnte jedoch im Erstellungszeitraum nicht festgestellt werden. Im Rahmen der Evaluation der Plattformen und der eigenen Arbeit bei der Programmierung eines Readers wurde die technische Entwicklung genutzt, um den Zugang zu dem Text sowie zu den Daten und den Zugriff auf die gesamte wissenschaftliche Kommunikation sicherzustellen.

Die Verwendung einer system-, geräte- und lokationsunabhängigen Webseite stellte sicher, dass sämtliche Inhalte der Kommunikation während und nach der Wissensproduktion durch andere innerhalb und außerhalb der wissenschaftlichen Gemeinschaft konsumiert und weiterverwendet werden konnten. Im Blog wurden darüber hinaus Informationen im Zusammenhang mit der Erstellung der Arbeit und aus den mit der Arbeit verknüpften Vorträgen und Kolloquiumsbesuchen veröffentlicht und dokumentiert. Einzig die Möglichkeit, zu den Forschungsbemühungen beizutragen, war aufgrund der notwendigen Selbstständigkeit bei der Erstellung der wissenschaftlichen Qualifikationsarbeit eingeschränkt. So konnten Nutzer und Nutzerinnen die Blogbeiträge rund um die Dokumentation der offenen Schreibweise kommentieren und Literaturvorschläge einreichen sowie über ein Kontaktformular Nachrichten übermitteln, direkt in und an der Arbeit gab es jedoch keine Möglichkeit, zu intervenieren oder zu kommentieren.

Die erweiterten Möglichkeiten der Anpassung und Kommentierung sowie die Möglichkeit der Veränderung der wissenschaftlichen Daten und des strukturierten Textes direkt auf GitHub wurden ebenfalls eingeschränkt. Die Möglichkeit die Arbeit unabhängig von dem bisherigen Autor unter anderem Namen weiterzuentwickeln (forken) konnte durch die Ablage des Textes als Code auf GitHub zwar nicht eingeschränkt werden, wurde jedoch nicht aktiv verwendet. Zwar machte ein Nutzer am 14. Mai 2015 von der Funktion Gebrauch, veränderte aber weder den Inhalt noch die Daten. Der Nutzer ist dem Autor zudem persönlich bekannt.

Der gesamte Forschungsprozess wurde so transparent und so zugänglich wie möglich gestaltet \cite{Scheliga_2014}. Es konnte zu jeder Zeit auf alle Daten, Ergebnisse und Protokolle in allen Phasen des Forschungsprozesses frei zugegriffen werden \cite{RIN_2010_open_research}. Die Eigenleistung und die in der Promotionsordnung geforderte \cite{promotionsordnung_leuphana_kuwi_2011} Selbstständigkeit bei der Erstellung der wissenschaftlichen Qualifikationsarbeit zu gewährleisten, wurde technisch so sichergestellt, dass es für keinen anderen, als den Autor, die Möglichkeit gab, den erstellten Inhalt zu editieren oder Stellen direkt zu kommentieren. Die offene Darstellung ermöglichte sogar eine neue Form der Sicherung der Eigenständigkeit, da alle Veränderungen am Text und Einflüsse auf den Inhalt direkt und während der Erstellung des Inhalts jederzeit offen und transparent dargestellt und dokumentiert wurden.

\section{Durchführung der offenen Anfertigung der Dissertation}

Die Erstellung dieser Arbeit sowie der damit verbundenen Daten und Informationen stand unter der Bedingung, dass diese jederzeit und so umfänglich wie möglich über das Internet abrufbar sind. Hierbei kam es zu unterschiedlichen Herausforderungen. Von den im vorherigen Kapitel evaluierten Anreizen beziehungsweise Möglichkeiten und Vorteilen wurde trotz respektabler Besucherzahlen nur wenig Gebrauch gemacht. Da es sich hierbei um den ersten Versuch eines komplett offenen Promotionsverfahrens handelt, konnte diesbezüglich auch nicht auf Erfahrungswerte zurückgegriffen werden.

Damit die Kriterien und Argumente für oder gegen das offene Publizieren (zum Beispiel Verbreitung, Beschleunigung, Aufwand) geprüft und gegebenenfalls Handlungsempfehlungen für das offene Schreiben von wissenschaftlichen Arbeiten am Beispiel von Dissertationen erstellt werden können, werden im Folgenden die rechtlichen und technischen Herausforderungen dokumentiert und zusammengefasst.

\subsection{Rechtliche Herausforderungen}

Die Promotionsordnung der Fakultät Kulturwissenschaften der Leuphana Universität mit dem Stand vom 02.02.2011 \cite{promotionsordnung_leuphana_kuwi_2011} untersagte nicht ausdrücklich das offene Verfassen einer Dissertationsarbeit wie es hier geplant war, erlaubte dies aber auch nicht explizit. Für die Klärung, dass die offene Schreibweise nicht gegen die Regeln der Promotionsordnung der Fakultät verstößt und möglicherweise zu einem Ausschluss der Arbeit aus dem Promotionsverfahren führt, wurde nach Fertigstellung des Exposés im Januar 2013 ein offizielles Schreiben an Promotionskommission übermittelt \cite{heise_2013_schreiben_kommission}, in dem um eine Erlaubnis der zeitgleichen Veröffentlichung des aktuellen Stands der Arbeit im Internet gebeten wurde.

Um den Anforderungen der aktuell geltenden Prüfungsordnung der Leuphana Universität vollends zu entsprechen, wurden in dem Schreiben an die Promotionskommission die Bedingungen für die offene Erstellung der Arbeit angeboten und die vermutete Vereinbarkeit mit der Promotionsordnung dargestellt. Nach einer rechtlichen Prüfung durch das Justitiariat der Universität entsprach die Promotionskommission am 12. Dezember 2013 mehrheitlich dem Gesuch die Arbeit "offen" verfassen zu dürfen. Sie stützte damit auch die Vermutung, dass die gewonnene Transparenz während des Erstellungsprozesses in diesem Fall keinen Widerspruch zu der Selbständigkeit bei der Ausarbeitung der Dissertation darstellt. Die Kommission empfahl darüber hinaus der nachfolgenden Promotionskommission, die Entstehungsform der Dissertation anzunehmen. Dennoch machte der Vorsitzenden eine Mitteilung zur Unsicherheit dieser Art der Veröffentlichung, da zum Zeitpunkt der Fertigstellung "voraussichtlich eine Promotionskommission unter anderer Zusammensetzung die Annahme der Dissertation zu prüfen und zu beschließen" \cite{heise_2013_erlaubnis_kommission} haben wird.

Um den Voraussetzungen der Open Definition und den Forderungen in den Erklärungen für die Offenheit im wissenschaftlichen Kommunikationsprozess von Budapest, Berlin und Bethesda (siehe Kapitel Grundlagen) vollumfänglich gerecht zu werden und eine möglichst umfassende Öffnung des wissenschaftlichen Erkenntnisprozzesses im Rahmen dieser Arbeit zu erreichen, wurden die Inhalte und Daten der Arbeit unmittelbar, für alle frei und kostenlos unter einer Creative Commons Lizenz veröffentlicht. Im Rahmen dieser Arbeit kommt dabei eine Creative Commons Lizenz unter den Bedingungen "Weitergabe unter gleichen Bedingungen 3.0 Unported" zum Einsatz.

An dieser Stelle sei auch auf die rechtliche Möglichkeit hingewiesen, dass die Arbeit oder Derivate des Textes von Dritten vor der eigentlichen Abgabe in veröffentlicht werden könnten. Damit würde zwar nicht direkt gegen die Auflagen der Promotionsordnung verstoßen werden, dennoch bedürfte das einer erneuten Prüfung und damit voraussichtlich einer Verzögerung im Promotionsprozess. Die Ordnung sieht zwar vor, dass die Dissertation "in begründeten Fällen teilweise vorher veröffentlicht" \cite{promotionsordnung_leuphana_kuwi_2011} sein kann, auf die gesamten Veröffentlichung der Arbeit vorab wird aber nicht Bezug genommen. Ähnliches gilt für die erhobenen Daten. Die rechtlichen Herausforderungen um die Promotionsordnung können somit als ein praktisches Beispiel für die Fokussierung des wissenschaftlichen Publikations- und Kommunikationssystemsystems auf das gedruckte Wort interpretiert werden.

\subsection{Technische Herausforderungen und Umsetzung der offenen Anfertigung der Arbeit}

Die Arbeit wurde, bis zur Klärung der Erlaubnis durch die Promotionskommission im Dezember 2013, zur Vorbereitung der öffentlichen Publikation in einem Google Dokument ohne Freigaben verfasst. Diese Veröffentlichungsform hatte sich bereits bei der Erstellung und Veröffentlichung des Exposés für das Promotionsvorhaben \cite{heise_2012_expose} als praktische Lösung für die Erarbeitung eines ersten Entwurfs herausgestellt und würde bei Freigabe durch die Promotionskommission, so die Annahme, eine unmittelbare Veröffentlichung in dem Blog ermöglichte.

Die Blogsoftware Wordpress (Version 3.8 bis 4.3) wurde im Vorfeld des Erstellungsprozesses der Arbeit auf dem Webserver (Ubuntu Linux 14.04, Apache 2.4, PHP 5.5, MySQL 5.5) des Autoren installiert und über die Domain http://offene-doktorarbeit.de im Internet für alle Internetnutzer verfügbar gemacht. Da weder beim Autor noch in der Literatur Vorerfahrungen mit dem offenen Verfassen von wissenschaftlichen Qualifikationsarbeiten vorherrschten, war geplant, unmittelbar nach Erlaubnis der offenen Anfertigung durch die Promotionskommission, die Arbeit aus dem Google Dokument in das bereits genutzte Blogsystem übertragen und der Schreibprozess dort bis zum Abschluss weiterzuführen.

Bei der Übertragung der bereits geschriebenen Inhalte in das Blogsystem Anfang 2014, stellte sich schnell heraus, dass die Blog-Software für die Veröffentlichung der gesamten Arbeit in einzelnen Blogposts sehr unpraktisch und unzureichend war. Zwar ermöglichten zusätzliche Anpassungen an dem System (Plug-Ins) die Veröffentlichung von Inhalten in wissenschaftlichen Formen und Formaten, dennoch stieß das eingesetzte System bei der Länge des Inhalts schnell an seine Grenzen. Folgenden weitere Gründe verhinderten letztendlich die geplante Überführung der geschriebenen Inhalte auf die Blogsoftware als primäre Veröffentlichungsplattform:
\begin{itemize}
\item Die Blogsoftware war für ein so umfassendes wissenschaftliches Vorhaben nicht konzipiert. Die nötigen Anpassungen (Literatur- sowie Zitatverwaltung, Darstellung von Grafiken und Statistiken, öffentlich einsehbares Revisionssystem, Exportfunktion) stellten sich während der Übertragung der bestehenden Inhalte schnell als unverhältnismäßig aufwendige Aufgabe dar.
\item Die Abbildung der Struktur der bis Ende 2013 verfassten, einzelnen Kapitel aus dem unstrukturierten Google Dokument als Blogposts stellte sich ebenfalls als sehr unpraktikabel dar und erforderte zahlreiche aufwendige Anpassungen.
\item Einzelne Inhalte und Formatierungen waren grundsätzlich nicht ohne Anpassungen übertragbar. Das betraf vor allem die Literaturangaben sowie die Formatierungen im Text.
\item Der Schreib- und Editierprozess stellte sich ebenfalls als ungemein unpraktikabel dar. Die Blog Software war zwar für eine schnelle Veröffentlichung von kurzen Beiträgen optimiert, bei längeren Texten erhöhten lange Ladezeiten und unflexibles scrollen den Zeitaufwand für die Arbeit am Text.
\item Trotz Anpassungen und geeigneter Plung-Ins ist die einfache standardisierte Referenzierung von Literaturverweisen, Fußnoten und die Sammlung von Quellen ist mit Wordpress über mehrere Blogeinträge bisher ohne umfassende Modifikationen des Systemkerns oder der Verwendung von externen Softwarelösungen nicht möglich.
\item Der Export der übergreifenden Inhalte in ein lesbares Dokument, dass online einsehbar ist, war nur mit hohem zusätzlichem Aufwand möglich.
\item Die strukturierte Eingabe der Inhalte und Ablage der Daten war ebenfalls nur unter erhöhtem Aufwand möglich und verringerte die Produktivität beim Schreiben.
\item Erfahrungen aus anderen Wordpressprojekten legten die Vermutung nahe, dass die Anzahl an unterschiedlichen Revisionen der einzelnen Teile der Arbeit, die Darstellung und Funktionsweise des Revisionssystems überfordert hätten.
\item Auf Grund der Länge des Inhalts, der hohen Anzahl an Revisionen und der Größe der Daten verhinderten Performanceeinbußen, die ohne eine leistungsfähigere IT-Infrastruktur nicht zu beheben waren, die einfache Arbeit am Text.
\end{itemize}

Nachdem der Einsatz von Wordpress und GoogleDocs zwar grundlegend der Anforderung des ständigen und offenen Zugriffs auf die Arbeit entsprach, sich aber letztendlich als unzureichend und unpraktikabel für die Erstellung wissenschaftlicher Inhalte herausgestellt hatte, wurden weitere Plattformen evaluiert. Zu diesem Zeitpunkt, Anfang 2014, standen jedoch keine standardisierte Lösung für das offene Verfassen wissenschaftlicher Arbeiten zur Verfügung und die gängigen webbasierten Softwarelösungen zur Online-Textverarbeitung und -darstellung genügten meist nicht den Ansprüchen für wissenschaftliche Arbeiten. Dennoch wurden neun Plattformen und Systeme auf ihre Praxistauglichkeit (siehe Praxistauglich) für das Vorhaben evaluiert.

Nach den praktischen Tests der unterschiedlichen Plattformen kam ab Ende Juli 2014 die kollaborative, wissenschaftliche Schreibplattform Authorea zum Einsatz. Authorea hatte gegenüber allen anderen Plattformen den grundlegenden Vorteil, dass die Inhalte einfach offen für jederman einsehbar dargestellt und in die bestehende Blogplattform eingebunden werden konnten. Ein umfassendes Revisionssystem ermöglichte zudem die einfache Darstellung der Veränderungen am Text und der Versionsgeschichte aller mit dem Text verbundenen Informationen. Diese Funktion war für die offene Schreibweise als ein grundlegender Vorteil erachtet worden. Auf Grundlage der positiven Evaluation wurden die Inhalte aus dem bisherigen System (GoogleDoc) auf die Plattform übertragen.

Der Text in Authorea wurde mit dem Textsatzsystem TeX und der Makrosprache Lamport Tex (LaTeX) verfasst. LaTeX ist ein Layoutsystem, das besonders für wissenschaftliches Veröffentlichen geeignet ist. Im Gegensatz zu gängigen Textverarbeitungsprogrammen ermöglicht dieses System die Arbeit an strukturellen Textdateien, die an bestimmten Stellen so ausgezeichnet werden, dass sie später als strukturierter Datensatz in jede Mögliche Form und Format übertragen und exportiert werden können. Während die üblichen Textverarbeitungsprogramme (wie zum Beispiel Microsoft Word) auf dem "What you see is what you get" (WYSIWYG) basieren, zählt man LaTeX zu den sogenannten Markup-­Sprachen beziehungsweise Auszeichnungssprachen, die nicht innerhalb einer bestimmten Umgebung verwendet werden muss \cite{Sievers_2012}. Diese Art des Textsatzes ist vor allem dann sinnvoll, wenn die finale Verwertung oder Ausgabe des Inhalts unbekannt oder variabel sein soll \cite{braune_2007_latex}. Darüber hinaus ermöglicht sie eine plattformübergreifende Erstellung und Ablage des Inhalts. Das Editieren des Inhalts blieb dabei vergleichbar komfortabel, da auf der Plattform gängige Texformatierungen über einen einfachen Editor ermöglicht wurden.

Im Grundlagen-Kapitel "Wissenschaftliche Kommunikation als Open-Source-Prozeß" wurden bereits Parallelen zwischen der Öffnung der Softwareentwicklung im Rahmen der Open-Source-Bewegung und der Öffnung von wissenschaftlicher Kommunikation gezogen und so war es naheliegend gängige Tools und Umgebungen für die Entwicklung von Software auch für die Erstellung der eigenen wissenschaftlichen Arbeit einzusetzen. Die strukturierte Ablage des Textes anhand von LaTeX stellte die Grundlage dafür dar. Die Arbeit wurde deshalb im weiteren Erstellungsprozess und nach der Migration des Textes aus dem Google Dokument auf Authorea zusätzlich mit einem GitHub Repositorium als Ablage für den Code und die Daten hinter dem öffentlich einsehbaren Text verknüpft.

Die Verknüpfung der Inhalte von der Schreibplattform mit dem Software-Repositorium GitHub hatte folgende weitere Vorteile:
\begin{itemize}
\item die automatische, dezentrale Sicherung und Archivierung der Arbeit auch außerhalb von Authorea.
\item die Erstellung, Bearbeitung und Synchronisierung eines lokalen Abbilds der gesamten Arbeit und der dahinterliegenden Daten auf dem Rechner über den GitHub Desktop-Client.
\item die Bearbeitung und Synchronisation der lokalen Inhalte mit denen auf GitHub und Authorea, ohne auf die Vorteile des Revisionssystems (Darstellung der einzelnen Schritte der Erstellung der Arbeit) verzichten zu müssen.
\item sie ermöglichte die Bearbeitung der Arbeit über mobile Applikationen von unterwegs.
\item die Arbeit auch ohne Internetzugang zu bearbeiten und die Veränderungen an der Arbeit trotzdem detailliert und transparent darzustellen.
\item ein Monitoring des Fortschritts der Arbeit durch die statistische Aufbereitung und Darstellung der Entwicklung auf GitHub.
\item für den Leser und die Leserin war es jederzeit die Veränderungen am Text und an den Daten nachzuvollziehen, somit konnten eventuelle Manipulationen oder wissenschaftliches Fehlverhalten transparent dargestellt und auch im Nachhinein verfügbar gehalten werden.
\item eine transparente Kontrolle der Verwendung des Textes und der Daten durch Dritte innerhalb von GitHub
\item die Möglichkeit über eine Schnittstelle (API) auf die Inhalte von anderen Diensten und durch Applikationen zuzugreifen und die Inhalte anderen Applikationen zur Verfügung zu stellen
\end{itemize}

Nachdem die Arbeit einen gewissen Umfang erreicht hatte, kam es zu Problemen bei der Darstellung der Inhalte über Authorea und auf der Webseite offene-doktorarbeit.de. Nach Rücksprache mit den Entwicklern stellte sich heraus, dass die Arbeit auf Grund ihrer Komplexität nicht mehr beziehungsweise nur noch eingeschränkt auf der Plattform geladen und dargestellt werden konnte. Dieser Umstand war selbst für die Entwickler unvorhersehbar. Als Grund dafür wurden unzureichende Ressourcen für die Umwandlung der LaTeX-Texte und -Daten in eine browserkompatible HTML-Darstellung vermutet \cite{authorea_502_2014}. Für die Erstellung dieser Arbeit und den Anspruch den Text jederzeit für jeden online einsehbar zu halten, hatte das zur Folge, dass für die Darstellung des Textes auf offene-doktorbeit.de eine Alternative gefunden werden musste.

Trotz der umfangreichen Evaluation der gängigen Softwarelösungen konnte keine geeignete Plattform gefunden werden, die die umfassende Darstellung des Textes übernehmen konnte. Als einzige Lösung blieb die Programmierung eines eigener Konverters (Readers), der den LaTeX-Datensatz aus der Datenablage (Repositorium) live importiert und in einer einfachen HTML-Ansicht unter http://live.offene-doktorarbeit zur Verfügung stellt. Für diesen Reader wurden mit einer eigenen Programmierung auf Grundlage der Skriptsprache PHP die einzelnen .tex-Dokumente über die Schnittstelle von GitHub aus dem Repositorium ausgelesen und in Hypertext Markup Language (HTML) konvertiert. Über diesen Konverter konnten die LaTeX-Auszeichnungen in HTML interpretiert und in jederzeit in gängigen Browsern dargestellt werden.

Die erstellte Software war in der Lage den jeweils aktuellen Text so automatisch zu importieren und zu interpretieren, dass die Inhalte und alle damit verknüpften Daten geräte- und plattformübergreifend über gängige Webbrowser für jeden jederzeit lesbar dargestellt werden konnten. Im weiteren ermöglichte die Software die Darstellung eines Inhaltsverzeichnis, errechnete die Wort- und Seitenzahl, bot eine Fortschrittsanzeige des Gesamtvorhabens, zeigte detailierte Informationen über die letzten Änderungen an der Arbeit, ermöglichte die Ansicht auf mobilen Endgeräten und ermöglichte eine umfassende Verknüpfung mit dem Dokumentationsblog. Im Verlauf der Arbeit wurden weitere Funktionen hinzugefügt. Zum Beispiel ermöglichte eine bestimmte Auszeichung die Markierung von Stellen innerhalb des Textes an denen noch Nachbesserungs oder Veränderungsbedarf bestand.

Für die grafische Darstellung der Ergebnisse aus der Befragung innerhalb des Textes und von Tabellen, wurde die Open-Source-Software Datawrapper \cite{datawrapper_2015} in der Version 1.9.6 eingesetzt, angepasst und in den Reader eingebunden. Diese offene Softwarelösung ermöglicht die einfache und dynamische Erstellung von Graphen und Diagrammen, die Darstellung dieser und die Ausgabe der dazugehörigen Daten in einem maschinenlesbaren Format. Datawrapper wurde ebenfalls auf dem Webserver unter http://graphs.offene-doktorarbeit.de installiert, eine Anbindung an den Reader programmiert und die Darstellung für LaTeX angepasst. Über einen LaTeX-kompatiblen Befehl konnten daraufhin Diagramme und Tabellen in die jeweils aktuelle Version der Arbeit integriert werden. Die Diagramme und Tabellen konnten trotzdem jederzeit modifiziert und angepasst werden. Für die finale Publikation wurde stand Bildexport zur Vefügung, der automatisch die Einbettung einer statischen Kopie der Grafiken in das finale LaTeX-Template ermögliche. Für die Mobilversion wurde eine angepasste Darstellung der Grafiken entwickelt und in den Reader integriert.

Die Bibliographie und Literaturangaben wurden über die Roh-Daten-Schnittstelle von GitHub und eine bestehende Open Source Softwarelösung (Bibtexbrowser) dargestellt. Diese Lösung ermöglicht es, die Literaturquellen direkt aus den Texten anzusteuern und die Metainformationen im Frontend auszugeben. Die Forschungsdaten wurden ebenfalls im Repositorium abgelegt und direkt aus der eigenen Readerlösung verlinkt. Die individuelle Programmierung ermöglichte die Darstellung der gesamten Arbeit, der Bibliographie und der Daten, auch ohne die direkten Verwendung von Authorea. Dennoch stellte diese Lösung sicher, dass die Inhalte zwischen GitHub und Authorea weiterhin automatisch synchronisiert wurden und stehts aktuell waren. Der Reader ermöglichte es jederzeit auf den aktuellen Stand der Texte und Daten zuzugreifen. Der Quellcode für die Reader-Applikation wurde wie die Daten und die gesamte Arbeit auf GitHub und dem Forschungsrepositorium zu dieser Arbeit als Open-Source-Software veröffentlicht \cite{heise_2015_reader}.

Obwohl, nach Evaluation aller webbasierten Schreibplattformen, zunächst eine geeignete Lösung gefunden wurde, war es ohne eigene Programmierarbeit nicht möglich, den gesamten Inhalt der Arbeit jederzeit offen einsehbar und verfügbar zu halten. Demnach wird davon ausgegangen, dass der Erstellungsprozess einer Promotion schon an den aktuellen technischen Möglichkeiten von Standardlösungen scheitern kann, wenn der Wissenschaftler und die Wissenschaftlerin nicht über ausreichende (Programmier-)Kenntnissen zur Bewältigung technischer Herausforderungen verfügt.

\subsection{Praxistauglichkeit}

Die folgende Darstellung der evaluierten Applikationen für die offene Erstellung einer wissenschaftlichen Arbeit wurde als wichtig erachtet, da die Öffnung wissenschaftlicher Kommunikation eng mit den technologischen Gegebenheiten \cite{naeder_2010_open}, mit dem vorherrschendem technischem Wissen und der Praktikabilität der Erstellung des Inhalts verknüpft ist. Die Auflistung der technologischen Gegebenheiten erscheint als sinnvoll um die Praxistauglichkeit an den Bedürfnissen für die offene Erstellung und Darstellung zu überprüfen.

In dem Experiment sollte die Arbeit als strukturierter Text mit einer strukturierte Zitationsverwaltung und mit einer öffentlichenVersionskontrolle, sowie die im Zusammenhang mit der Arbeit erhobenen Daten jederzeit öffentlich einsehbar sein. Ein maschinenlesbarer Export sollte ausserdem die Darstellung der Inhalte in andere Systeme ermöglichen.

Die folgende Auswahl ist exemplarisch und die Evaluation der verfügbaren Plattformen erhebt dabei weder den Anspruch auf Vollständigkeit, noch auf eine generelle Anwendbarkeit für die Erstellung wissenschaftliche Publikationen:
\begin{itemize}
\item Google Docs hat sich bereits im Rahmen der Erstellung des Promotionsexposés als einfache und praktikable Lösung für die Erstellung und Darstellung kurzer Texte herausgestellt. Google Docs ermöglicht die simultane Arbeit an einem Dokument durch mehrere Autoren und ist dadurch für kollaborative Arbeiten geeignet. Veränderungen am Text konnten allerdings nur unzureichend dargestellt werden und mit zunehmender Länge des Textes verringerte sich auch die Performanz der Web-Plattform. Außerdem waren keine oder nur begrenzte Lösungen für die Darstellung umfangreicher Literaturangaben und die Verwaltung von Zitationen gegeben. Zum Zeitpunkt der Evaluation beschränkten sich die direkten Darstellungsmöglichkeiten primär auf die Google Plattform und eine Integration auf andere Seiten war nur unzureichend möglich.
\item Wikis sind Onlineplattformen auf Grundlage einer Open-Source-Software mit der Inhalte jeglicher Art, zumeist jedoch Text, von Nutzerinnen und Nutzern nicht nur über einen Webbrowser gelesen, sondern auch direkt bearbeitet werden können. Das prominentestes Beispiel ist die freie Enzyklopädie Wikipedia. Für das akademische Schreiben sind sie nur begrenzt geeignet, da Wikis vornehmlich für nichtwissenschaftlichen und verhältnismäßig kurze Texte konzipiert sind. Grundsätzlich sind über Extensions Funktionserweiterungen der jeweils eingesetzten Wikisoftware möglich. Diese Erweiterungen gibt es auch für viele der notwendigen Funktionen für das wissenschaftliche Publizieren, allerdings werden diese nur unregelmäßig gewartet und müssen trotzdem umfangreich angepasst werden. Wikis sind darüber hinaus auch strukturell nicht für die Erstellung und Darstellung von langen Texten geeignet und weisen Defizite bei der Handhabung auf.
\item FidusWriter ist eine Online-Anwendung bestimmt für das akademische Schreiben und Publizieren. Ähnlich zu GoogleDocs ermöglicht die Applikation simultanes Arbeiten an einem Dokument in Echtdarstellung durch mehrere Autoren, ist aber im Gegensatz zu dem Google Produkt mit grundlegenden Funktionen für das akademische Publizieren angereichert. Dazu gehören unter anderem eine Zitationsverwaltung und ein Formeleditor. LaTeX-Kenntnisse sind für die Erstellung nicht nötig. Der Export der erstellten Dokumente ist auf vorgegebene Ausgabeformate beschränkt und nur in HTML, LaTeX und ePub möglich. Eine Importfunktion sowie die Anbindung an GitHub oder andere Speicherdienste ist bisher nicht vorgesehen. Die offene Darstellung oder Einbettung der erstellten Inhalte auf anderen Plattformen ist bei Fidus Writer nicht möglich. Die Entwicklung der Plattform befindet sich noch im Teststadium, ist nur mit aktuellen Versionen der Browser Google Chrome und Safari kompatibel und hat sich in letzter Zeit stark verlangsamt \cite{fidus_2015}.
\item Booktype ist eine Open-Source-Software die es mehrerne Autoren ermöglicht an einem Buch zu schreiben. Die Plattform ist ebenfalls auf nicht-wissenschaftliche Texte ausgelegt und somit fehlen grundlegende Funktionen wie eine wissenschaftliche Literaturverwaltung und umfangreiche Zitierfunktionen. Die Software bietet zwar diverse Exportformate, ist allerdings primär für die nicht-wissenschaftliche Textarbeit konzipiert. Der Bedienungskomfort und Versionskontrolle sind eingeschränkt und die Ablage von Daten und weiteren Informationen ist nur begrenzt bis überhaupt nicht möglich. Import und Exportfunktionen beschränken sich auf die fertige Publikation.
\item Authorea ist ebenfalls eine Webanwendung die im Browser die kollaborative Textverarbeitung  und die Ablage wissenschaftlicher Dokumente in einem Repositorium ermöglicht. Autoren können über die Webseite wissenschaftliche Texte verfassen, editieren und darstellen sowie Bibliographien zu erstellen, verwalten und durchsuchen. Dazu wird der Text in einzelne Stücke zerlegt, die jeweils von einem Autor bearbeitet werden können. Die Bearbeitung kann in LaTeX oder Markdown erfolgen und ermöglicht während der Erstellung grundlegende Funktionen der Echtbilddarstellung (WYSIWYG). Darüber hinaus können Quellen und Referenzen von externen Quellen zum Beispiel über einen Digital Object Identifier (DOI) importiert werden. Die technische Infrastruktur hinter der Plattform erlaubt es, bei der Erstellung der Arbeit Zwischenstände und Veränderungen abzulegen, sie zu dokumentieren und öffentlich darzustellen. Dieses Revisionssystem kann optional auch an die Softwareentwicklungplattform GitHub angebunden werden. Die Texte können zu jeder Zeit über bestimmte Templates in verschiedene Formate exportiert werden und öffentlich oder privat verfasst werden. Die Funktion nicht-öffentliche Texte zu verfassen, steht allerdings nur zahlenden Nutzern zur Verfügung. Der Darstellung der Texte ist responsiv und ermöglicht das einfache, geräteübergreifende Lesen der Inhalte.
\item Overleaf (ehemals writeLaTeX) vereint verschiedene Funktionen der bisher genannten Lösungen. So beherrscht das System eine einfache Änderungsverfolgung und eine Versionshistorie. In seiner aktuellen Version beinhaltet die Plattform alle notwendigen Funktionen einen wissenschaftlichen Text in LaTeX zu verfassen. Die Ausgabeformate und -möglichkeiten ermöglichen den Export nach GitHub, als gepacktes Dokument (ZIP) und als PDF. Einige der Funktionen stehen dabei ebenfalls nur zahlenden Nutzerinnen und Nutzern zur Verfügung, so zum Beispiel die Integration von DropBox, mehr Speichervermögen und das Verfassen von nicht-öffentlichen Dokumenten. Dennoch adressiert die Plattform eher technisch versierte Nutzer mit LaTeX-KnowHow. Wie bei Authorea hat man die Möglichkeit seine Inhalte mit Hilfe von LaTex zu erstellen, in Ergänzung dazu aber auch Fehler im LaTeX-Code zu identifizieren und zu beheben. Im Gegensatz zu Authorea ist der öffentliche Lesemodus aber weder responsiv, noch für die einfache Lesedarstellung oder Integration auf anderen Webseiten geeignet. Geplant ist außerdem die Möglichkeit die gesamte Historie eines Projekts darzustellen. Die Änderung des Namens der Plattform Ende 2014 ist mit der Ausweitung der Funktionsvielfalt auf generelle Publikationsprozesse und eine auch auf die nicht-wissenschaftlichen Bereich verbunden \cite{overleaf_2014}.
\item shareLaTeX ist vom Funktionsumfang ähnlich zu Overleaf, beschränkt sich aber auf die Aufgabe eines webbasierten LaTeX-Editors, ohne die Möglichkeiten vorformatierten Text (Rich-Text) zu importieren oder zu editieren. Die Applikation sprich eine eher technisch versierte und wissenschaftliche Zielgruppe an, die in der Lage ist LaTeX zu editieren. Neben Google Docs ist shareLaTeX die einzige Plattform die bisher in deutscher Sprache verfügbar ist. Auch hier sind unterschiedliche Ausgabeformate und -templates vorhanden und die Nutzer können ihre erstellten Templates anderen Nutzern zur Verfügung stellen. Darüber hinaus erlaubt die Webapplikation seit Anfang 2015 den Import von GitHub. Funktionen wie die Synchronisierung der abgelegten Daten mit GitHub, Dropbox, ein vollständiger Versionsverlauf und eine unbegrenzte Anzahl von Dokumenten stehen aber nur zahlenden Nutzern zur Verfügung. Wie bei Overleaf eignet sich die öffentliche Darstellung von shareLaTeX-Dokumenten allerdings nicht zur einfachen Darstellung des Inhalts für Dritte oder zum Einbetten in bestehende Umgebungen.
\item Etherpads sind eine Art öffentlicher Notizblock mit Versionskontrolle. Es kann am ehesten mit Google Docs verglichen werden, hat aber einen viel geringeren Funktionsumfang. Wie bei GoogleDocs fehlen hier die Funktionen, die für das wissenschaftliche Publizieren notwendig sind. Die Inhalte werden in einem markdown-ähnlichem Format verfasst aber die Strukturierung der Dokumente ist nur im eingeschränkten Maße möglich. Die Exportmöglichkeiten sind ebenfalls begrenzt und eine Darstellung im Lesemodus bei Beschränkung der Editierfunktion ist nur über technische Umwege Möglich. Etherpads eignen sich vom Funktionsumfang und von den Darstellungsmöglichkeiten im wissenschaftlichen Prozess nur für die schnelle kollaborative Erstellung und Bearbeitung von Notizen.
\end{itemize}

\begin{figure}[h!]
\includegraphics{smalltableid:au9xQ}
\caption{Praxistauglichkeit der evaluierten Systeme}
\end{figure}

Die Plattform Authorea hat sich demnach bei der Evaluation als praxistauglichestes System herausgestellt. Dennoch zeigt diese Auflistung, dass alle betrachteten Tools durch ihren begrenzten Funktionsumfang nicht ausnahmslos für die offene Erstellung einer wissenschaftlichen Arbeit am Beispiel dieser Arbeit geeignet sind. Bei allen Lösungen bedarf es der Anpassung, Zusammenfassung und Ergänzung von Funktionen, damit die Arbeit in vollem Umfang jederzeit öffentlich einsehbar und dokumentierbar ist.

\subsection{Vor- und Nachteile}

Vorteilhaft kann sich die offenen Schreibweise in den Fällen erweisen, in denen eine (fach-)öffentliche Diskussion die Arbeit des Forschers oder der Forscherin positiv beeinflusst. Im Rahmen der Erstellung einer Promotion ist das nur begrenzt von Vorteil, da die Arbeit "selbstständig" \cite{promotionsordnung_leuphana_kuwi_2011} erstellt werden muss. Es wird sich zukünftig herausstellen, in welchem Umfang Kommentare oder kollaborative Schreibweisen diese Selbstständigkeit gefährden oder ein "unerlaubtes Hilfsmittel" \cite{promotionsordnung_leuphana_kuwi_2011} darstellen. Bei der Erstellung dieser Qualifikationsarbeit wurden die Kommentar- und kollaborative Schreibfunktionen, die bei Authorea zur Verfügung standen, proaktiv deaktiviert um die Voraussetzungen der Promotionsordnung vollumfänglich zu erfüllen \cite{heise_2013_schreiben_kommission}. Einzige Möglichkeit, den Autor über die Plattform zu kontaktieren, war per E-Mail und die Möglichkeit auf der Webseite http://offene-doktorarbeit.de Literaturempfehlungen per Mail einzureichen. Davon wurde allerdings im gesamten Verlauf nur zwei Mal Gebrauch gemacht.

Ein weiterer Vorteil der offenen Schreibweise in den genutzten Systemen bezog sich auf die Möglichkeit, die unterschiedlichen Versionen und Revisionen der Arbeit einfach und transparent zu durchsuchen. Somit war es möglich den Erstellungsprozess einzelner Sätze oder Absätze auch im Nachhinein transparent nachvollziehbar und überprüfbar zu machen. Damit steht die Arbeit und die damit verbundende wissenschaftliche Erkenntnis der (Fach-)Öffentlichkeit möglichst umfassend offen für Kritik. Für den Autoren hatte dieses Vorgehen den Vorteil, dass in den 3 Jahren Schreibphase jederzeit die Möglichkeit bestand, Satzkonstruktionen und Gedankengänge rückwirkend zu durchsuchen, nachvollziehbar und darstellbar zu machen.

Die Aufmerksamkeit für die offene Schreibweise der Doktorarbeit im direkten Umfeld des Autors war verhältnismäßig groß. Inhaltlich hatte das zwar kaum Effekte, dennoch stieg die Anzahl der regelmäßigen Nachfragen bezüglich des Bearbeitungstandes der Arbeit, nachdem der Text einsehbar war. Der soziale Druck die Arbeit voranzubringen stieg ebenfalls. Hinweise auf Fehler inhaltlicher oder rechtschreibtechnischer Natur blieben weitestgehend aus. Die Webseite http://offene-doktorarbeit.de wurde von bis zu 362 Besuchern im Monat aufgerufen. Die Anzahl der Nutzer stieg fast linear mit der Anzahl der Blogbeiträge und dem fortschreitenden Schreibprozess der Arbeit.

\begin{figure}[h!]
\includegraphics{graphid:Qaj1E}
\caption{Besucherzahlen auf offene-doktorarbeit.de}
\end{figure}

Einen weiteren Vorteil stellte die Quantifizierung der Arbeitsverhaltens dar. Durch die Ablage auf Github und durch das Speichern der einzelenen Arbeitsfortschritte in einem Revisionssystem war es möglich verschiedene rudimentäre Statistiken rund um das Arbeits- und Beitragsverhalten als Autor zu erheben und darzustellen. So ermöglichte zum Beispiel die Darstellung einer sogenannten Punchcard an welchen Wochentag und zu welcher Uhrzeit die meisten Bearbeitungen und Veränderungen eingereicht wurden . Die Information kann dabei zum Beispiel helfen, die Zeit zu identifizieren, an denen der Autor oder die Autorin am besten arbeiten kann. Somti können diese Erhebungen zum einen informative und motivierende Effekte haben, anderseits aber auch negative Konsequenzen nach sich ziehen und zum Beospiel grundsätzlich die Überwachung von Arbeits- und Publikationsverhalten von Wissenschaftlern ermöglichen. Die Folgen einer flächendeckenden Überwachung des Verhaltens werden in dieser Arbeit nicht betrachtet, bieten aber dennoch einen Ansatzupunkt für weitere Forschung.

---- TODO: Grafiken einbauen ----

Nachteile ergaben sich aus der fehlenden Verfügbarkeit von einfachen technischen, rechtlichen und konzeptionellen Standards bei der offenen Erstellung wissenschaftlicher Qualifizierungstexte und bei der Veröffentlichung von Forschungsdaten. Es fehlt an einfachen und zugänglichen Diensten und Applikationen, die es dem Autor oder der Autoren einfach macht den Text zu erstellen und Daten zu verwalten. Auch die Darstellung des gesamten Prozesses der Erstellung wissenschaftlicher Publikationen ist bisher wenig verbreitet und nicht standardisiert. Der Rückgriff auf die Entwicklerplattform GitHub, bei der ein Revisionsverhalten tief im System verankert ist, stellte dabei einen pragmatischen Ausweg dar.

Mit LaTeX stand zwar ein wissenschaftlicher Textsatz zur Verfügung, der auch dem Anspruch einer strukturierten Ablage von Text gerecht wurde und damit der Annahme entsprach Text als Code zu behandeln, allerdings ist dieser kompliziert zu handhaben und fehlt an vielen Stellen an einem einfachen Frontend mit dem Funktionsumfang eines gängigen Textverarbeitungsprogramms um dieses System offen, einfach und zu jedem Zeitpunkt verfügbar zu nutzen. Ergänzend war mit BibTeX ein Literaturdatenbanksystem für Literaturangaben für LaTeX-Dokumente verfügbar, das ebenfalls sehr komplex, aber umfassend unterstützt wird. Auch für BibTex gibt es mit wenigen Ausnahmen kein einfaches Frontend und die Bearbeitung, Validierung und Darstellung ist eher aufwändig.

Die Arbeit jederzeit in jedem Zustand für jeden im Internet einsehbar zu halten, erschien am Anfang befremdlich, wurde aber nach einiger Zeit normal. Darüber hinaus stellt die ständige Befürchtung im Bearbeitungszustand falsche oder fehlerhafte Inhalte zu verbreiten, eine weitere Hürde bei der offenen Schreibweise dar, die sich aber ebenfalls im Schreibprozess reduzierte und durch eine geeignete Kennzeichnung über den aktuellen Stand der Arbeit bei der Darstellung verringert werden konnte. Bei der Erstellung dieser Arbeit wurde deshalb möglichst auffällig immer wieder auf den Zustand der Arbeit hingewiesen.

Als weiterer Nachteil gegenüber der analogen und geschlossenen Arbeit ist der Aufwand für die Anonymisierung und Veröffentlichung der Umfragedaten kurz nach Abschluss der Erhebung zu nennen. Auch die Abklärung der rechtlichen Rahmenbedingungen nimmt bisher viel Zeit in Anspruch, ist aber ebenfalls überwindbar.

Die Gefahr, dass Inhalte fehlinterpretiert werden oder vorab wissenschaftlich anerkannt "veröffentlicht" werden und die Arbeit somit gegebenenfalls nicht mehr als unveröffentlichte, originäre Leistung anerkannt wird, konnte während des Verfassens dieser Arbeit nicht bestätigt werden. Dennoch empfiehlt es sich, diesen Umstand bei zukünftigen Vorhaben zu beachten.

\section{Kritische Betrachtung der Vorgehensweise}

Seit Beginn der Erstellung der Arbeit haben die technische Entwicklung und die Möglichkeiten für das digitale wissenschaftliche Publizieren einige Fortschritte gemacht. In den letzten zwei Jahren ist eine Vielzahl an Tools und Applikationen entwickelt und veröffentlicht worden, die die digitale Veröffentlichung von wissenschaftlichen Inhalten im Internet fokussieren. Diese bringen viele Vorteile für die offene Bearbeitung von wissenschaftlichen Fragestellungen. Vor allem in Bezug auf die Darstellung wissenschaftlicher Inhalte in dem Blogsystem Wordpress hat es einige interessante Entwicklungen und Modifikationen gegeben, die hier nicht in vollem Umfang berücksichtigt werden konnten.

Die Begrenzung in Bezug auf die Verwendung und Evaluation der im Rahmen dieser Arbeit berücksichtigten Plattformen rührt daher, dass bei fortlaufender Erarbeitung und beim Anstieg des Textvolumens ein Wechsel der Plattform trotz system- und plattformübergreifender Textformatierung immer aufwendiger und schwieriger wurde. Die im Erstellungszeitraum dieser Arbeit aus dem Umfeld der Öffnung und Digitalisierung wissenschaftlicher Forschung neu entwickelten Dienste und Tools vereinfachen dennoch zunehmend die Öffnung der wissenschaftlichen Kommunikation und Information und erleichtern die Verwaltung von Forschungsdaten und des gesamten Forschungsprozesses. Bis zum Abschluss dieser Arbeit lag jedoch keine Lösung vor, mit der diese Arbeit ganz in dem praktizierten Umfang von Offenheit hätte erstellt werden können.

Kritisch muss dabei angemerkt werden, dass Offenheit im Rahmen jedweder Kommunikation im Rahmen dieser Arbeit auch nur in begrenztem Maße möglich war. Zwar wurden die Zwischenstände, Präsentationen und Entwicklungen beim Erstellungsprozess der Arbeit im Blog unter http://offene-doktorarbeit.de dokumentiert und der Text sowie Abbildungen und Daten waren jederzeit über http://live.offene-doktorarbeit.de einsehbar, doch hätte man noch weitere Möglichkeiten der Öffnung realisieren können. Die Einbeziehung externer Personen in die Erstellung des Textes, die Öffnung der Mailkommunikation im Zusammenhang mit der Arbeit, die audiovisuelle Dokumentation von Vorträgen und Diskussionen und vieles mehr wäre zwar denkbar, jedoch nicht ohne Mehraufwand möglich und unter Umständen auch nicht mit der Promotionsordnung vereinbar gewesen.

Die Übertragbarkeit aller Erfahrungen aus dem Experiment der Erstellung einer offenen geisteswissenschaftlichen Arbeit auf andere Fächer ist allerdings nur begrenzt möglich. Der Aufwand bei der Datenbereitstellung, die Vorgehensweise, die Darstellungsformen und der wissenschaftliche Erkenntnisprozess können sich stark von dem Aufwand in anderen Disziplinen unterscheiden. Dennoch gibt es Überschneidungen bei der grundlegenden Herangehensweise für die offene Erstellung einer Arbeit.

Von diesem Einzelexperiment und der Verortung in den Geisteswissenschaften, der begrenzten Möglichkeiten der Nutzung von kollaborativen Schreibprozessen im Rahmen einer Qualifikationsarbeit sowie der begrenzten Reichweite des Vorhabens sind die ganzen Konsequenzen einer Öffnung für die wissenschaftliche Gemeinschaft und Gesamtgesellschaft nur eingeschränkt ableitbar. Vor allem in Bezug auf die in den Grundlagen genannten Hindernisse und Befürchtungen wie Themen-, Ideen- und Datendiebstahl und Fehlinterpretation können hier nur eingeschränkt ausgeräumt werden. Dennoch wurde der Beweis angetreten, dass es grundsätzlich möglich ist.

\section{Zwischenergebnis: Handlungsempfehlungen für das Verfassen einer offenen wissenschaftlichen Arbeit}

Wie erwartet muss der Aufwand für die Erstellung einer wissenschaftlichen Arbeit in einer geschlossenen Umgebung auf dem eigenen Rechner als geringer eingeschätzt werden als die offene Texterstellung im Internet unter einer offenen Lizenz für jeden jederzeit einsehbar. Trotz mehrfachen Wechsels der Softwareumgebung, konnte letztendlich keine einfache Lösung gefunden werden, die der Bedienbarkeit und Flexibilität der geschlossenen wissenschaftlichen Textbearbeitung auf dem Desktop entspricht.

Das Experiment hat verdeutlicht, dass der Forderung nach Öffnung des gesamten wissenschaftlichen Erkenntnisprozesses und der damit verbundenen Öffnung der wissenschaftlichen Kommunikation zum Erstellungszeitpunkt dieser Arbeit noch nicht ohne erheblichen Mehraufwand nachgekommen werden kann. Nur mit ausreichend programmiertechnischen Kenntnissen kann der Anspruch an die offene Schreibweise, der zeitnahen und umfassenden Veröffentlichung von Kommunikation und Daten nach wissenschaftlichen Maßstäben erfüllt werden. Demnach müssen Wissenschaftler und Wissenschaftlerinnen in der Lage sein, die technischen Herausforderungen zu überwinden und die Rahmenbedingungen für den offenen Schreibprozess selbst zu schaffen. Weder die Forschungsinstitutionen noch private Anbieter sind bisher in der Lage, Plattformen für die Öffnung des gesamten wissenschaftlichen Prozesses anzubieten. Zu unterschiedlich sind die Anforderungen in den verschiedenen Disziplinen, zu mannigfaltig die Funktionen in den vorhandenen Applikationen, zu uneinheitlich die Standards für das digitale Publizieren und zu verschieden der Kenntnisstand bei der Verwendung digitaler Methoden und Tools.

Die skizzierten technischen Herausforderungen legen die Vermutung nahe, dass für die Öffnung des wissenschaftlichen Kommunikationsprozesses, zumindest bisher, mindestens rudimentäre Programmierkenntnisse bei den Wissenschaftlern und Wissenschaftlerinnen erforderlich sind. Ohne Kenntnisse über Webtechnologien, Quellcodes und Datenbanken wäre die offene Darstellung dieser Arbeit nicht möglich gewesen. Trotz der Entwicklung neuer Plattformen ist bisher von einer offenen Schreibweise ohne programmiertechnische Grundkenntnisse abzuraten. Liegen diese Kenntnisse vor, kann eine solche Art der Anfertigung einer wissenschaftlichen (Qualifikations-)Arbeit allerdings als sehr bereichernd und motivierend wirken.

Den rechtlichen Herausforderungen im Rahmen der Vereinbarkeit mit dem auf den Druck der finalen Publikation ausgelegten Prozess konnte im Rahmen dieser Arbeit nur nach einiger Zeit der Prüfung mit einer schriftlichen Ausnahmeregelung der Promotionskommission begegnet werden. Diese wurde auf Anfrage durch den Autor und nach umfänglicher Prüfung erteilt. Bei der Regelung bleibt allerdings ein Rest Unsicherheit, da die bei der Abgabe verantwortliche Promotionskommission gegebenenfalls unter anderer Zusammensetzung als bei der Zustimmung zu dem offenen Schreibprozess die Annahme der Dissertation erneut zu prüfen und zu beschließen hat.

Insgesamt müssen die Vor- und Nachteile der offenen Schreibweise ausgewogen betrachtet werden. Die offene Erstellung dieser Arbeit hat gezeigt, dass der Forderung nach Öffnung der wissenschaftlichen Kommunikation im Rahmen einer Promotionsarbeit grundsätzlich entsprochen werden kann.

Letztendlich, so das Ergebnis des eigenen Experiments, sind unter mit dieser Arbeit vergleichbaren Bedingungen durch die offene Schreibweise bisher weder fundamentale Vorteile noch unlösbare Hürden für den publizierenden Wissenschaftler oder die Wissenschaftlerin erkennbar. Die gegebenenfalls positiven Folgen der offenen Publikation von Inhalten und Daten sowie deren Nachnutzung können im Rahmen dieser Arbeit nicht betrachtet werden. Es ist jedoch davon auszugehen, dass die Reichweite dieser Arbeit und der dazugehörigen Daten, die von Arbeiten im geschlossenen Raum übersteigt. Weitere Experimente mit der offenen Forschungsarbeit sind demnach notwendig, um abschließend zu evaluieren, ob eine solche Art des Verfassens von Forschungs- und Qualifizierungsarbeiten einen fundamentalen Vorteil für die Wissenschaft und die Öffentlichkeit bringt.

Ziel des Experiments war auch die Ableitung konkreter Handlungsempfehlungen aus den Erfahrungen beim offenen Verfassen wissenschaftlicher (Qualifikations-)Arbeiten. Folgende zehn Empfehlungen, die für das Verfassen einer offenen wissenschaftlichen Arbeit berücksichtigt werden sollten, resultieren aus den Erfahrungen des Experiments:

\begin{enumerate}
\item Bevor sich der Autor oder die Autorin für das offene Schreiben und das zeitnahe oder zeitgleiche Veröffentlichen des jeweils aktuellen Stands der Arbeit entscheidet, sollte mit der Universität geklärt werden, ob diese Art und Weise der Publikation mit den Richtlinien der Institution oder den Voraussetzungen des jeweiligen finalen Veröffentlichungskanals vereinbar sind. Falls Unklarheiten bestehen, sollte eine schriftliche Erlaubnis eingefordert werden. Das gilt insbesondere für wissenschaftliche Qualifikationsarbeiten.
\item Autoren und Autorinnen, die sich für die direkte Veröffentlichung im Internet entscheiden, sollten sich vorab mit den technischen Grundlagen vertraut machen. Ein Grundverständnis von Quellcodes und Software ist dabei von großem Vorteil, wenn nicht sogar Voraussetzung. Da es bisher nur wenige standardisierte Systeme und Formate für die Erstellung wissenschaftlicher Arbeiten gibt, helfen diese Kenntnisse, Probleme und Herausforderungen zu verstehen, zu lösen und gegebenenfalls zu umgehen.
\item Die gewissenhafte Auswahl der Software für die Texterstellung und Datenverarbeitung spielt eine wichtige Rolle für das Vorhaben. Die Autoren oder Autorinnen sollten von Beginn an eine Lösung wählen, die es ihnen einfach macht, den Text zu schreiben und zeitnah im Internet zu veröffentlichen. Der Aufwand für die Erstellung und die Motivation, die Arbeit voranzubringen, hängt auch mit dem Nutzungskomfort der Software zusammen. Es sollte zudem sichergestellt werden, dass die Stabilität für den gesamten Text gegeben ist.
\item Es empfiehlt sich, für die zeitnahe Veröffentlichung, Dokumentation und Anonymisierung der erhobenen Daten gesondert Zeit einzuplanen. Für die Veröffentlichung von Forschungsdaten sollte außerdem eine Plattform gewählt werden, die für die Gewährleistung eines hohen Qualitätsstandards ein Review durchführt und sämtliche Forschungsdaten vor der Veröffentlichung prüft. So kann sichergestellt werden, dass die nötige Anonymität gewahrt ist und die Daten nachhaltig verfügbar und auffindbar sind.
\item Die Erwartungen an die Reichweite und die Vorteile im Verlauf des offenen Verfassens der Arbeit sollten nicht zu hoch gesteckt werden. Wer das offene Verfassen nutzen will, um während des Schreibens zusätzliches Feedback oder weitere Ideen einzuholen, darf sich nicht darauf verlassen, dass das automatisch geschieht, nur weil die Arbeit jederzeit einsehbar ist. Das ist dadurch bedingt, dass es sich um eine relativ neue Form der wissenschaftlichen Arbeit handelt. Dennoch kann über diese Art der Veröffentlichung eine beachtliche Reichweite generiert werden.
\item Die Dokumentation des Vorhabens und die damit verbundenen Tätigkeiten sind wichtig und sollten ebenfalls mit eingeplant werden. Die umfassende Dokumentation ermöglicht eine bessere Darstellung des Forschungsvorhabens und der Beweggründe für das Vorhaben. Darüber hinaus bietet sie eine Möglichkeit, interessante Informationen (wie zum Beispiel Zeitplan und Ablauf) fortlaufend bei der Erstellung der Arbeit zu kommunizieren, Nutzer und Nutzerinnen stärker in den Erstellungsprozess mit einzubinden und den Erkenntnisprozess insgesamt transparenter und offener zu gestalten. Auch hier ist allerdings ein Mehraufwand gegenüber der geschlossenen Erstellung von wissenschaftlichen Arbeiten zu erkennen.
\item Es sollte unbedingt auf eine offene Lizenz zurückgegriffen werden, um den Ansprüchen für Offenheit und Transparenz gerecht zu werden sowie anderen die (Weiter-)Nutzung der Inhalte und Daten möglichst umfassend zu ermöglichen.
\item Bei der Erstellung, Erhebung und Darstellung sollte jederzeit berücksichtigt werden, dass alle Texte, Daten und Informationen unwiderruflich im Internet veröffentlicht sind und gegebenenfalls auch bleiben. Die offene wissenschaftliche Arbeit erfordert demnach sehr viel Sorgfalt und Disziplin.
\item Das soziale Umfeld des Autors oder der Autorin sollte auf die Dokumentation der Arbeit hingewiesen werden, da so positiver Druck im Rahmen des Zeitplans entstehen kann. Das motiviert und erhöht die Arbeitsmoral.
\item Es empfiehlt sich, an prominenter Stelle immer wieder darauf hinzuweisen, dass es sich um eine unvollendete und laufende Arbeit handelt. Außerdem sollten die eventuellen Einschränkungen der Funktionsvielfalt (zum Beispiel keine Kommentarfunktion) in Bezug auf die Selbständigkeit der Erstellung der Arbeit sauber und offen kommuniziert werden.
\end{enumerate}

---- TODO: Kapitel Ende umschreiben ----
