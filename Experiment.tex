\chapter{Das Experiment: Offenes Verfassen einer Dissertation}


\section{konzeptionelle und technische Rahmenbedingungen}

Konzeptionell war das Projekt so angelegt, dass die Arbeit unter allen Umständen jederzeit frei und offen Verfügbar einsehbar sein sollte. Inital sollte ein Blog auf der Grundlage der verbreiteten Open Source-Lösung Wordpress zum Einsatz kommen und nicht nur die Dokumentation rund um die Arbeit, sondern auch als technische Plattform für die gesamte Arbeit selbst zur Verfügung stehen.

\section{Herausforderungen der offenen Anfertigung der Dissertation}

Die Arbeit wurde, bis zur Klärung der Erlaubnis durch die Promotionskommission im Dezember 2013, in einem Google Dokument ohne Freigaben verfasst. Google Docs, ist ein kostenloser, webbasiertes Textverarbeitungssystem der Firma Google. Es ist angelehnt an die gängigen Programme von Microsoft Office oder Open Office, bietet aber einige Einschränkungen besonders für das wissenschaftliche Publizieren.

Nach der schriftlichen Erlaubnis durch die Promotionskommission vom 11. Dezember 2013, wurde in einem ersten Schritt das Google Dokument offen zur Verfügung gestellt beziehungsweise für jeden lesbar freigegeben und aus dem Blog verlinkt. Bei der Übertragung der bereits geschriebenen Inhalte in das Blogsystem, stellte sich schnell heraus, dass die Blog-Software für die Veröffentlichung in einzelnen Blogposts nicht geeignet war. Zwar ermöglichten zusätzliche Plugins die Veröffentlichung von Inhalten in wissenschaftlichen Formen und Formaten, aber an folgenden Aspekten scheiterte der Einsatz der Blogsoftware als Publikationplattform:
\begin{itemize}
\item Die Darstellung der einzelenen Kapitel als Blogposts hat sich als sehr aufwendig und insofern Schreibprozess als unpraktikabel herausgestellt
\item Eine einfache standardisierte Referenzierung von von Literaturverweisen ist nicht möglich
\item Fußnoten können nicht über mehrere Einträge hinweg zusammenhängend dargestellt werden
\item Der Export in ein lesbares Dokument ist immer mit Aufwand verbunden
\item Die strukturierte Eingabe der Inhalte ist ebenfalls nur unter zusätzlichem Aufwand möglich
\item Bei der Anzahl an unterschiedlichen Revisionen wäre das Revisionssystem sicher an seine Grenzen gestoßen
\item ...
\end{itemize}