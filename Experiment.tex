\chapter{Experimentelle Untersuchung: Offenes Verfassen einer Dissertation}

Um vor allem die Kriterien und Argumente (zum Beispiel Aufwand) für oder gegen das offene Publizieren prüfen zu können und Handlungsempfehlungen für das offene Schreiben von wissenschaftlichen Arbeiten am Beispiel von Dissertationen erstellen zu können, wurde für diese Arbeit als Selbstexperiment eine offene Schreibweise gewählt. Dabei wurde die Arbeit direkt und unmittelbar während der Erstellung für jeden, jederzeit frei zugänglich auf einer Webseite (http://offene-doktorarbeit.de) im Internet veröffentlicht.

\section{konzeptionelle und technische Rahmenbedingungen}

Konzeptionell war das Projekt so angelegt, dass die Arbeit unter allen Umständen jederzeit frei und offen im Internet einsehbar sein sollte. Initial sollte ein Blog auf der Grundlage der verbreiteten Open Source-Lösung Wordpress zum Einsatz kommen und nicht nur die Dokumentation rund um die Arbeit, sondern auch als technische Plattform für die gesamte Arbeit selbst zur Verfügung stehen.

In einem ersten Schritt wurden die Inhalte in einem Dokument auf Google Docs im Blog offen zur Verfügung gestellt. Google Docs, ist ein kostenloses, webbasiertes Textverarbeitungssystem der Firma Google. Es ist angelehnt an die gängigen Programme von Microsoft Office oder Open Office, bietet aber einige Einschränkungen besonders für das wissenschaftliche Publizieren. Vor der Erstellung der Arbeit wurden unterschiedliche technische Möglichkeiten getestet. In der damaligen Analyse wurde die Darstellung der Arbeit in einem Blogsystem präferiert. Blogs, auch Weblogs genannt, beschreiben eine Reihe von Softwarelösungen, die nach der Installation Internetnutzern ermöglichen Einträge im Internet zu veröffentlichen.

Bei der Übertragung der bereits geschriebenen Inhalte in das Blogsystem, stellte sich schnell heraus, dass die Blog-Software allein für die Veröffentlichung in einzelnen Blogposts nicht geeignet war. Zwar ermöglichten zusätzliche Anpassungen an dem System (Plugins) die Veröffentlichung von Inhalten in wissenschaftlichen Formen und Formaten, dennoch stieß das eingesetzte System schnell an seine Grenzen. Bei der Internetrecherche nach Alternativen Lösungswegen wurde das noch junge System Authorea ausgewählt und die Inhalte aus dem bisherigen System dahin übertragen.

Ziel des Experiments ist die Analyse ob und wie weit die offene Erstellung der Doktorarbeit möglich ist. Dieses Vorgehen soll helfen weitere Treiber und Bremser, sowie exemplarisch den Aufwand für offene Verfassen einer wissenschaftlichen Arbeit zu identifizieren.

\section{Herausforderungen der offenen Anfertigung der Dissertation}

Bei der Erstellung der Arbeit unter der Voraussetzung, dass diese jederzeit über das Internet abrufbar ist, stellten sich unterschiedliche Herausforderungen. Da es sich hierbei um den ersten Versuch eines offenen Promotionsverfahren handelte, konnte nicht auf Erfahrungswerte zurückgegriffen werden.

Damit die Kriterien und Argumente (zum Beispiel Aufwand) für oder gegen das offene Publizieren geprüft werden können und gegebenenfalls  Handlungsempfehlungen für das offene Schreiben von wissenschaftlichen Arbeiten am Beispiel von Dissertationen erstellt werden können, werden im Folgenden die rechtlichen und technischen Herausforderungen dokumentiert und strukturiert zusammengefasst.

\subsection{Rechtliche Herausforderungen}

Die Promotionsordnung der Fakultät Kulturwissenschaften der Leuphana Universität mit dem Stand vom 02.02.2011 untersagte nicht ausdrücklich das offene Verfassen einer Dissertationsarbeit, erlaubte das aber auch nicht explizit. Um sicherzustellen, dass die offene Schreibweise nicht gegen die Regeln der Promotionsordnung der Fakultät verstößt, wurde Anfang des Jahres 2013 ein Schreiben an Promotionskommission geschickt und um eine Erlaubnis der zeitgleichen Einsehbarkeit des aktuellen Stands meiner Arbeit gebeten.

Im Dezember 2013 sprach sich die Promotionskommission, nach der rechtlichen Prüfung durch das Justitiariat der Universität, mehrheitlich für eine direkte und unmittelbare Veröffentlichung des Schreibprozesses der Dissertation aus. Die Kommission, empfahl darüber hinaus der nachfolgenden Promotionskommission, Entstehungsform der Dissertation anzunehmen. Dennoch machte der Vorsitzenden eine Mitteilung zur Unsicherheit dieser Art der Veröffentlichung, da abschließend "voraussichtlich eine Promotionskommission unter anderer Zusammensetzung die Annahme der Dissertation zu prüfen und beschließen hat".

--- Todo: Vorveröffentlichung (Inhalte, Daten) durch Dritte ----

\subsection{Technische Umsetzung und Herausforderungen}

Die Arbeit wurde, bis zur Klärung der Erlaubnis durch die Promotionskommission im Dezember 2013, in einem geschlossenen Google Dokument ohne Freigaben verfasst.

An folgenden Aspekten scheiterte der Einsatz der Blogsoftware als Publikationsplattform:
\begin{itemize}
\item Die Darstellung der einzelnen Kapitel als Blogposts hat sich als sehr aufwendig und insofern Schreibprozess als unpraktikabel herausgestellt
\item Eine einfache standardisierte Referenzierung von Literaturverweisen ist nicht möglich
\item Fußnoten können nicht über mehrere Einträge hinweg zusammenhängend dargestellt werden, eine Zuordnung von Seitenzahlen stellte ebenfalls eine Herausforderung dar
\item Der Export in ein lesbares Dokument ist immer mit Aufwand verbunden
\item Die strukturierte Eingabe der Inhalte ist ebenfalls nur unter zusätzlichem Aufwand möglich
\item Bei der Anzahl an unterschiedlichen Revisionen wäre das Revisionssystem sicher an seine Grenzen gestoßen
\item ...
\end{itemize}

Da keine standardisierte Lösung für Verfassen offener Arbeiten vorherrscht und die gängigen Systeme nicht den Ansprüchen für wissenschaftliche Arbeit genügen, wurde mehrmals die Plattform gewechselt. Nachdem sich der Einsatz von GoogleDocs und Worpdress als unzureichend herausgestellt hatte, wurde die Arbeit im August 2014 auf Authorea übertragen. Authorea ist eine Webanwendung die im Browser die kollaborative Textverarbeitung ermöglicht und die Ablage der Dokumente in einem Repositorium ermöglicht. So können wissenschaftliche Texte webbasiert verfasst und öffentlich dargestellt werden. Die technische Infrastruktur hinter der Plattform ermöglicht es, bei der Erstellung der Arbeit wurden alle Zwischenstände und Veränderungen an dem Text abzulegen, zu dokumentieren und darzustellen. Dieses Revisionssystem kann optional auch an die Softwareentwicklungplattform GitHub angebunden werden, denn auch hier wird bei jedem Speichern eine Version der Arbeit als eigene Revisionen abgespeichert. Da in dieser Arbeit Parallelen zwischen der Öffnung der Softwareentwicklung im Rahmen der Open-Source-Bewegung und der Öffnung von wissenschaftlicher Kommunikation gezogen werden, war es naheliegend gängige Tools und Umgebungen für die Entwicklung von Software auch für die eigene wissenschaftliche Arbeit zu einzusetzen. Die Arbeit wurde also im Erstellungsprozess nach dem Umzug auf Authorea auch zusätzlich mit einem GitHub Repositorium verknüpft.

Diese Verknüpfung der beiden Anwendungen ermöglichte:
\begin{itemize}
\item die automatische, dezentrale Sicherung und Archivierung der Arbeit auch außerhalb von Authorea.
\item die Erstellung und Synchronisierung eines lokalen Abbilds der gesamten Arbeit und der Daten auf dem Rechner über den GitHub Desktop-Client.
\item die Bearbeitung und Synchronisation der lokalen Inhalte mit denen auf GitHub und Authorea, ohne auf die Vorteile des Revisionssystems (Darstellung der einzelnen Schritte der Erstellung der Arbeit) verzichten zu müssen.
\item die Arbeit auch ohne Internetzugang zu bearbeiten und die Veränderungen an der Arbeit trotzdem detailliert und transparent darzustellen.
\item ein Monitoring des Fortschritts durch die statistische Aufbereitung und Darstellung der Entwicklung und Veränderung im Erstellungsprozess der Promotionsarbeit
\end{itemize}

Für die Erstellung wurde das Textsatzsystem TeX mit der Makrosprache Lamport Tex (LaTeX) verwendet. LaTeX ist ein Art "Expertensystem" für Layout" \cite{suchen}, das besonders für wissenschaftliches Veröffentlichen geeignet ist. Im Gegensatz zu gängigen Textverarbeitungsprogrammen ermöglicht dieses System die Arbeit an Textdateien, die an bestimmten stellen so ausgezeichnet werden, dass sie später als strukturierter Datensatz in jede Mögliche Form übertragen werde kann. Während Textverarbeitungsprogrammen (wie zum Beispiel Microsoft Word) auf dem "What you see is what you get" (WYSIWYG), zählt man LaTeX zu den sogenannten Markup-­Sprachen, die nicht innerhalb einer bestimmten Umgebung verwendet werden müssen. Diese Art des Textsatzes ist vor allem dann sinnvoll, wenn das Ausgabemedium oder die Verwertung des Inhalts unbekannt oder variabel ist \cite{suchen}.

--- weiter ausarbeiten u.a.: Keine wirklich einfache Software zur Erstellung wissenschaftlicher Texte ----

\subsection{Vor- und Nachteile}


\section{Ergebnis}

Bisher kann der Aufwand für die Erstellung einer wissenschaftlichen Arbeit in einer geschlossenen Umgebung auf dem eigenen Rechner als geringer eingeschätzt werden, als die Texterstellung im Internet unter einer offenen Lizenz für jeden jederzeit einsehbar. Trotz mehrfachen Wechsels der Software, konnte keine einfache Lösung gefunden werden, die der Bedienbarkeit und Flexibilität der geschlossenen wissenschaftlichen Textbearbeitung auf dem Desktop entspricht. Die Anonymisierung und Veröffentlichung der Umfrage-Daten kurz nach Abschluss der Erhebung stellte ebenfalls einen Mehraufwand dar.

\section{Empfehlungen}
