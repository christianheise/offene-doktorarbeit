\chapter{Experimentelle Untersuchung: Offenes Verfassen einer Dissertation}

Um vor allem die Kriterien und Argumente (zum Beispiel Aufwand) für oder gegen das offene Publizieren prüfen zu können und Handlungsempfehlungen für das offene Schreiben von wissenschaftlichen Arbeiten am Beispiel von Dissertationen erstellen zu können, wurde für diese Arbeit als Selbstexperiment eine offene Schreibweise gewählt. Dabei wurde die Arbeit direkt und unmittelbar während der Erstellung für jeden, jederzeit frei zugänglich auf einer Webseite (http://offene-doktorarbeit.de) im Internet veröffentlicht.

Ziel des Experiments ist auch die Analyse ob und wie weit die offene Erstellung der Doktorarbeit unter den Kriterien und den Forderungen von Open Science möglich ist. Das Vorgehen soll helfen weitere Treiber und Bremser, sowie exemplarisch den Aufwand für offene Verfassen einer wissenschaftlichen Arbeit zu identifizieren und als Beispiel für ein konkretes Vorhaben der Öffnung von Wissenschaft und Forschung dienen. Darüber hinaus soll die Praxistauglichkeit der Forderung nach Öffnung an das wissenschaftliche Kommunikationsystem dargestellt und analysiert werden.

Für diese Herangehensweise wurde ein (auto-)ethnographischer Ansatz gewählt. Dieser ist primär durch drei Merkmale gekenntzeichnet, die bei der Darstellung im Folgenden berücksichtigt werden:  Erstens sucht er  "einen primär verstehenden Zugang", zweitens geht mit ihm (inzwischen) ein stark gebrochener Holismus einher und drittens sind "Ethnographien durch einen Methodenmix gekennzeichnet" \cite{bachmann_2011_ethnographie}. Dieser Methodenmix eignet sich für die Herangehensweise des Felds der Science and Technology Studies und ermöglicht die umfassende Darstellung und die erhöht die Reproduzierbarkeit des Vorhabens.

\section{konzeptionelle und technische Rahmenbedingungen}

Konzeptionell war das Projekt von Anfang an so angelegt, dass die Arbeit und alle Daten unter allen Umständen jederzeit frei und offen im Internet einsehbar sein sollte. Die Bedingungen untern denen diese Arbeit erstellt wird, sollten sich dabei so nahe wie möglich an den Forderungen von Open Science und den genannten Erklärungen zur Öffnung wissenschaftlicher Kommunikation orientieren. Nach der rechtlichen Klärung der offenen Schreibweise, sollte die Arbeit und alle damit verbundenen Tätigkeiten so schnell und umfassend wie möglich jederzeit frei und offen im Internet abrufbar sein.

Um die Eigenleistung und die Selbstständigkeit bei der Erstellung der wissenschaftlichen Qualifikationsarbeit zu gewährleisten, wurde technisch so sichergestellt, dass es für andere als den Autoren keine Möglichkeit gab, den erstellten Inhalt zu editieren oder direkt zu kommentieren. Die offene Darstellung ermöglichte sogar eine neue Form die Eigenständigkeit direkt während der wissenschaftlichen Arbeit und Erstellung des Inhalts jederzeit offen und transparent darzustellen.

Initial kam dafür ein Blog auf der Grundlage der Open Source-Lösung Wordpress zum Einsatz. Blogs, auch Weblogs genannt, beschreiben eine Reihe von Softwarelösungen, die nach der Installation Internetnutzern einfach ermöglichen Einträge im Internet zu veröffentlichen. Wordpress wurde zunächst als eine solche Blogging-Plattform entwickelt, in den letzten Jahren hat sich das System aber zu einem umfangreichen Content Management System weiterentwickelt \cite{Patel_2011_cms}. Content Management Systeme ermöglichen nicht nur die Darstellung von Texten in chronologischer Reihenfolge, sondern auch die Ablage und Organisation von Daten oder anderen Medien. Einer der wesentlichen Vorteile von Wordpress ist die große Anzahl von Plug-Ins \cite{Patel_2011_cms}. Über die Plug-Ins kann jeder Aspekt einer Wordpress Webseite in Bezug auf die Erstellung, Organisation und Optimierung von Inhalten mit dem Einsatz von Plug-Ins erweitert werden. Sie werden von unabhängigen Programmierern entwickelt und meist unter einer open source Lizenz freigegeben. Da die verwendete Software nicht nur für die Dokumentation rund um die Arbeit dienen, sondern auch als technische Plattform für die Veröffentlichung der gesamten wissenschaftlichen Arbeit selbst zur Verfügung stehen sollte, erschien Wordpress als die beste Lösung \cite{Jones_2013_CMS}.

In einem weiteren Schritt sollten die Inhalte in einem Dokument auf Google Docs im Blog eingebunden und offen zur Verfügung gestellt. Google Docs, ist ein kostenloses, webbasiertes Textverarbeitungssystem der Firma Google. Es ist angelehnt an die gängigen Programme von Microsoft Office oder Open Office, bietet aber einige Einschränkungen besonders für das wissenschaftliche Publizieren. So fehlen bei Google Docs die Möglichkeiten der strukturierten Ablage von Daten und die einfache Verwaltung von Quellen und referenzen. Vor der Erstellung der Arbeit wurden unterschiedliche technische Möglichkeiten für die einfache Darstellung getestet. In der ursprünglichen Analyse wurde wie oben dargestellt die Veröffentlichung der Arbeit in einem Blogsystem präferiert und später aus pragmatischen Gründen auf die Schreibplattform Authorea übertragen.

Insgesamt wurde der der Prozesse der Wissensschaffung öffentlich dokumentiert und jederzeit einsehbar veröffentlicht. Exemplarisch fand das in den folgenden fünf Phasen des wissenschaftlichen Erkenntnisprozesses statt:
\begin{enumerate}
\item Die \textit{Fragestellung und Planung} der Arbeit wurde im Blog seit August 2012 veröffentlicht. Auf den generellen Übersichtsseiten wurde dort das Vorhaben vorgestellt und regelmäßig zum Stand der Arbeit Beiträge veröffentlicht. Das Exposé für die Doktorarbeit wurde in einem Google Dokument verfasst und in dem Blog eingebunden.
\item Die \textit{Ausführung} der Befragung, des Schreibprozesses (seit Bestätigung durch die Promotionskommission Ende 2013) und des Experiments war zu jeder Zeit offen einsehbar. Auf der jeweiligen Schreibplattform wurde der Stand der Arbeit festgehalten und Blog wurden die Entwicklungen rund um die Arbeit regelmäßig dokumentiert. Die Befragung wurde ebenfalls auf dem Blog dokumentiert und die Umfragedaten anschließend als anonymisierte Rohdaten veröffentlicht.
\item Die \textit{Analyse} der Daten wurde auf Grundlage der Rohdaten durchgeführt und direkt in dem Text der Arbeit verarbeitet. Zwischenergebnisse wurden aufbereitet und vorab inklusove der jeweiligen Daten kommuniziert.
\item Das \textit{Auswertungsverfahren} wurde ebenfalls im Blog dokumentiert und direkt in der Arbeit inklusive der Daten veröffentlicht.
\item Die \textit{Verwendung und Kommunikation der Ergebnisse} fand ebenfalls unmittelbar und jederzeit auf der Schreibplattform und im Blog sowie direkt im Text der Arbeit statt.
\end{enumerate}

---- TODO: weiter ausarbeiten ----

\section{Herausforderungen der offenen Anfertigung der Dissertation}

Bei der Erstellung der Arbeit unter der Voraussetzung, dass diese jederzeit über das Internet abrufbar ist, stellten sich unterschiedliche Herausforderungen. Da es sich hierbei um den ersten Versuch eines offenen Promotionsverfahren handelte, konnte nicht auf Erfahrungswerte zurückgegriffen werden.

Damit die Kriterien und Argumente (zum Beispiel Aufwand) für oder gegen das offene Publizieren geprüft werden können und gegebenenfalls Handlungsempfehlungen für das offene Schreiben von wissenschaftlichen Arbeiten am Beispiel von Dissertationen erstellt werden können, werden im Folgenden die rechtlichen und technischen Herausforderungen dokumentiert und strukturiert zusammengefasst.

\subsection{Rechtliche Herausforderungen}

Die Promotionsordnung der Fakultät Kulturwissenschaften der Leuphana Universität mit dem Stand vom 02.02.2011 \cite{promotionsordnung_leuphana_kuwi_2011} untersagte nicht ausdrücklich das offene Verfassen einer Dissertationsarbeit, erlaubte das aber auch nicht explizit. Um sicherzustellen, dass die offene Schreibweise nicht gegen die Regeln der Promotionsordnung der Fakultät verstößt und möglicherweise zu einem Ausschluß der Arbeit aus dem Promotionsverfahren führt, wurde Anfang des Jahres 2013 ein Schreiben an Promotionskommission übermittelt und um eine Erlaubnis der zeitgleichen Veröffentlichung des aktuellen Stands der Arbeit gebeten.

Um den Anforderungen der aktuell geltenden Prüfungsordnung der Leuphana Universität vollends zu entsprechen, wurden in einem Schreiben an die Promotionskommission die Bedingungen für die offene Erstellung der abgestimmt und die Vereinbarkeit mit der Promotionsordnung geprüft. Nach einer rechtlichen Prüfung durch das Justitiariat der Universität entsprach die Promotionskommission am 12. Dezember 2013 dem Gesuch die Arbeit "offen" verfassen zu dürfen mehrheitlich. Sie stützte damit die Vermutung, dass die gewonnene Transparenz während des Erstellungsprozesses in diesem Fall keinen Widerspruch zu der Selbständigkeit bei der Ausarbeitung der Dissertation darstellt. Die Kommission, empfahl darüber hinaus der nachfolgenden Promotionskommission, die Entstehungsform der Dissertation anzunehmen. Dennoch machte der Vorsitzenden eine Mitteilung zur Unsicherheit dieser Art der Veröffentlichung, da "voraussichtlich eine Promotionskommission unter anderer Zusammensetzung die Annahme der Dissertation zu prüfen und zu beschließen hat".

Um den rechtlichen Vorraussetzungen der Open Definition und den Forderungen in den Erklärungen für die Offenheit im wissenschaftlichen Kommunikationsprozess von Budapest, Berlin und Bethesda gerecht zu werden, wurden die Inhalte der Arbeit unmittellbar, für alle frei und kostenlos unter einer Creative Commons Lizenz veröffentlicht. Im Rahmen dieser Arbeit kommt eine Creative Commons Lizenz unter den Bedingungen "Weitergabe unter gleichen Bedingungen 3.0 Unported" zum Einsatz, mit knapp einem Drittel Anteil die am häufigsten verwendete CC-Lizenz.

An dieser Stelle sei auch auf die rechtliche Möglichkeit hingewiesen, dass die Arbeit oder Derivate des Textes von Dritten vorab veröffentlicht werden könnte. Damit würde zwar nicht direkt gegen die Auflagen der Promotionsordnung verstoßen werden, dennoch bedürfte das einer erneuten Prüfung und damit voraussichtlich einer Verzögerung im Promotionsprozess. Ähnliches gilt für die erhobenen Daten.

\subsection{Technische Umsetzung und Herausforderungen}

Die Arbeit wurde, bis zur Klärung der Erlaubnis durch die Promotionskommission im Dezember 2013 zur Vorbereitung der öffentlichen Publikation in einem geschlossenen Google Dokument ohne Freigaben verfasst. Diese Veröffentlichungsform hatte sich bereits bei der Erstellung und Veröffentlichtung des Exposés für zum Promotionsvorhaben \cite{heise_2012_expose} als praktische Lösung herausgestellt und ermöglichte bei Freigabe durch die Promotionskommission eine unmittelbare Veröffentlichung in dem Blog.

Die Blogsoftware Wordpress wurde im Vorfeld des Erstellungsprozess der Arbeit auf einem Webserver des Autoren installiert und über die Domain http://offene-doktorarbeit.de im Internet für alle Nutzer verfügbar gemacht. Da keine Vorerfahrungen mit dem offenen Verfassen von Qualifikationsarbeiten vorlagen war es usprüngliches Ziel, nach Bestätigung der offenen Anfertigung durch die Kommission, die Arbeit aus dem Google Dokument in das bereits genutzte Blogsystem zu übertragen und den Schreibprozess dort weiterzuführen.

Bei der Übertragung der bereits geschriebenen Inhalte in das Blogsystem, stellte sich schnell heraus, dass die Blog-Software für die Veröffentlichung der gesamten Arbeit in einzelnen Blogposts unhandlich und unzureichend war. Zwar ermöglichten zusätzliche Anpassungen an dem System (Plug-Ins) die Veröffentlichung von Inhalten in wissenschaftlichen Formen und Formaten, dennoch stieß das eingesetzte System schnell an seine Grenzen. Folgenden Gründe verhinderten die einfache Überführung der Inhalte auf die Blogsoftware als primäre Publikationsplattform:
\begin{itemize}
\item Eine Blogsoftware ist für ein so umfassendes Vorhaben nicht konzipiert. Die nötigen Anpassungen stellten sich schnell als aufwendige Aufgabe dar.
\item Der Versuch der Überführung und Darstellung der bis Ende 2013 verfassten, einzelnen Kapitel aus dem unstrukturierten Google Dokument als Blogposts stellte sich ebenfalls als sehr aufwendig dar.
\item  Einzelne Inhalte und Formatierungen waren grundsätzlich nicht ohne Anpassungen übertragbar. Das betraf vor allem die Literaturangaben, die Struktur der Arbeit und die Formatierungen um Text.
\item Der Schreib- und Editierprozess stellte sich als unpraktikabel heraus, da die Software für eine schnelle Veröffentlichung von kurzen Beiträgen optimiert war.
\item Die einfache standardisierte Referenzierung von Literaturverweisen und die Sammlung von Quellen ist mit Wordpress (ohne umfassende Modifikationen oder der Verwendung von Softwarelösungen) nicht möglich.
\item Fußnoten konnten zwar über eine Erweiterung angezeigt, jedoch nicht ohne grundlegende Modfikationen über mehrere Einträge hinweg zusammenhängend dargestellt werden.
\item Die Zuordnung der Seitenzahlen stellte ebenfalls zum Zeitpunkt der Übertragung eine nicht einfach lösbare Herausforderung dar.
\item Der Export in ein lesbares Dokument war beim Testen immer mit zusätzlichen Aufwand verbunden.
\item Die strukturierte Eingabe der Inhalte war ebenfalls nur unter erhöhtem Aufwand möglich und veringerte die Produktivität beim Schreiben.
\item Erfahrungen aus anderen Projekten legten die Vermutung nah, dass die Anzahl an unterschiedlichen Revisionen der einzelenen Teile der Arbeit die Darstellung und Funktionsweise des Revisionssystems überfordert hätten.
\end{itemize}

Da im Jahr 2013 keine standardisierte Lösung für das offene Verfassen wissenschaftlicher Arbeiten verfügbar schien und die gängigen web-basierten Softwarelösungen zur Online-Textverarbeitung und -darstellung nicht den Ansprüchen für wissenschaftliche Arbeiten genügten, wurde unterschiedliche Plattformen evaluiert:
\begin{itemize}
\item Google Docs hatte sich bereits als einfache Lösung für kurze Texte herausgestellt, verfügte aber über keine Lösung umfangreiche Literaturangaben zu machen. Ausserdem war das System auf die Darstellung innerhalb von GoogleDocs beschränkt. Ebenfalls konnten Änderungen nur unzureichend dargestellt werden. Mit zunehmender Länge des Textes verringerte sich auch die Performanz der Web-Plattform und der Applikationen.
\item Fidus Writer ist eine Anwendung mit Webinterface für akademisches Schreiben und Publizieren. Ähnlich zu GoogleDocs aber angereichert mit Funktionen für das akademische Publizieren, wie Zitationsverwaltung und Formeleditor.
\item Authorea ist eine Webanwendung die im Browser die kollaborative Textverarbeitung ermöglicht und die Ablage wissenschaftlicher Dokumente in einem Repositorium ermöglicht. Sie ermöglicht das webbasierte Verfassen, Editieren und Dartstellen von wissenschaftlichen Texte. Authorea wurde 2013 von Alberto Pepe, Nathan Jenkins und Matteo Cantiello veröffentlicht. Die technische Infrastruktur hinter der Plattform ermöglicht es, bei der Erstellung der Arbeit alle Zwischenstände und Veränderungen abzulegen, sie zu dokumentieren und öffentlich darzustellen. Dieses Revisionssystem kann optional auch an die Softwareentwicklungplattform GitHub angebunden werden.
\item writeLaTeX (jetzt Overleaf) \cite{Perkel_2014}
\item shareLaTeX
\item Etherpad
\item andere Content-Mangement-Systeme
\end{itemize}

---- TODO: weiter ausarbeiten ----

Nachdem sich der Einsatz von Worpdress und GoogleDocs als unzureichend herausgestellt hatte, wurde im August 2014 die kollaborative, wissenschaftliche Schreibplattform Authorea ausgewählt und die Inhalte aus dem bisherigen System dahin konvertiert und übertragen.

Da in dieser Arbeit parallelen zwischen der Öffnung der Softwareentwicklung im Rahmen der Open-Source-Bewegung und der Öffnung von wissenschaftlicher Kommunikation gezogen wurde und beim Autor Vorkenntnisse über die Entwicklung von Open-Source-Software vorhanden waren ---- TODO: Kapitel angeben ----, war es naheliegend gängige Tools und Umgebungen für die Entwicklung von Software auch für die Erstellung der eigene wissenschaftliche Arbeit zu einzusetzen. Die Arbeit wurde deshalb im weiteren Erstellungsprozess und nach der Migration des Textes aus dem Google Dokument auf Authorea auch zusätzlich mit einem GitHub Repositorium als Ablage für den Quellcode hinter dem Text verknüpft.

Die Verknüpfung der Inhalte von der Schreibplattform mit dem Software-Repositium GitHub hatte folgende weitere Vorteile:
\begin{itemize}
\item die automatische, dezentrale Sicherung und Archivierung der Arbeit auch außerhalb von Authorea.
\item die Erstellung, Bearbeitung und Synchronisierung eines lokalen Abbilds der gesamten Arbeit und der dahinterliegenden Daten auf dem Rechner über den GitHub Desktop-Client.
\item die Bearbeitung und Synchronisation der lokalen Inhalte mit denen auf GitHub und Authorea, ohne auf die Vorteile des Revisionssystems (Darstellung der einzelnen Schritte der Erstellung der Arbeit) verzichten zu müssen.
\item sie ermöglichte die Bearbeitung der Arbeit über mobile Applikationen von unterwegs.
\item die Arbeit auch ohne Internetzugang zu bearbeiten und die Veränderungen an der Arbeit trotzdem detailliert und transparent darzustellen.
\item ein Monitoring des Fortschritts der Arbeit durch die statistische Aufbereitung und Darstellung der Entwicklung auf GitHub.
\item eine transparente Kontrolle der Verwendung der Daten durch Dritte innerhalb von GitHub
\item die Möglichkeit über eine Schnittstelle (API) auf die Inhalte von anderen Diensten und durch Applikationen zuzugreifen.
\end{itemize}

Der Text wurde mit dem Textsatzsystem TeX und der Makrosprache Lamport Tex (LaTeX) verfasst. LaTeX ist ein Layoutsystem, das besonders für wissenschaftliches Veröffentlichen geeignet ist. Im Gegensatz zu gängigen Textverarbeitungsprogrammen ermöglicht dieses System die Arbeit an Textdateien, die an bestimmten stellen so ausgezeichnet werden, dass sie später als strukturierter Datensatz in jede Mögliche Form übertragen werde kann. Während Textverarbeitungsprogrammen (wie zum Beispiel Microsoft Word) auf dem "What you see is what you get" (WYSIWYG), zählt man LaTeX zu den sogenannten Markup-­Sprachen beziehungsweise Auszeichnungssprachen, die nicht innerhalb einer bestimmten Umgebung verwendet werden müssen \cite{Sievers_2012}. Diese Art des Textsatzes ist vor allem dann sinnvoll, wenn die finale Verwertung oder Ausgabe des Inhalts unbekannt oder variabel ist  \cite{braune_2007_latex}.

Nachdem die Arbeit einen gewissen Umfang erreicht hat, kam es zu Problemen bei der Darstellung der Inhalte über Authorea. Nach Rücksprache mit den Entwicklern stellte sich heraus, dass die Arbeit auf Grund ihrer Komplexität nicht mehr beziehungsweise nur noch eingeschränkt geladen und dargestellt werden konnte. Grund hierfür waren unzureichende Ressourcen für die Umwandlung von LaTeX-Daten in eine browserkompatible HTML-Darstellung. Das hatte zur Folge, dass für die Darstellung des Textes auf offene-doktorbeit.de eine Alternative gewählt werden musste.

Zur Lösung wurde für diese alternative Live-Darstellung der aktuellen Arbeit durch den Author ein eigener Konverter (Reader) programmiert, der den LaTeX-Datensatz aus dem Repositorium live importiert und in einer einfachen HTML-Ansicht unter http://live.offene-doktorarbeit zur Verfügung stellt. Mit einer eigenen PHP Programmierung wurden dafür die einzelnen .tex-Dokumente über die Schnittstelle von GitHub aus dem Repositorium ausgelesen und in HTML konvertiert. Über diesen Konverter konnten die LaTeX-Auszeichnungen in HTML interpretiert und im Browser dargestellt werden.

Die Bibliographie und Literaturangaben wurde über die Roh-Daten-Schnittstelle von GitHub an eine bestehende Open Source Softwarelösung (Bibtexbrowser) dargestellt. Diese Lösung ermöglicht es direkt aus Texten auch die Literaturquellen direkt anzusteuern und die Metatinformationen im Frontend auszugeben. Die Forschungsdaten wurden wurden ebenfalls direkt aus der eigenen Readerlösung verlinkt. Die individuelle Programmierung ermöglichte das Darstellung der gesamten Arbeit, der Bibliographie und der Daten, auch ohne die direkten Verwendung von Authorea. Dennoch stellte diese Lösung sicher, dass die Inhalte zwischen GitHub und Authorea weiterhin automatisch syncronisiert wurden und stehts aktuell waren. Der Reader ermöglichte es jederzeit auf den aktuellen Stand der Texte und Daten zuzugreifen. Der Quellcode für die Reader-Applikation wurde wie die Daten und die gesammte Arbeit auf GitHub und dem Forschungsrepositorium zu dieser Arbeit veröffentlicht \cite{heise_2015_reader}.

Obwohl nach Evaluation aller webbasierten Schreibplattformen zunächst eine geeignete Lösung gefunden wurde, war es ohne eigene Programmierarbeit nicht möglich den gesamten Inhalt der Arbeit dauerhaft öffentlich verfügbar zu machen. Demnach wird davon ausgegangen, dass der Erstellungsprozess einer Promotion schon an den aktuellen technischen Möglichkeiten von Standardlösungen scheitern kann, wenn der Wissenschaftler nicht mit genügend (Programmier-)Kenntnissen zur Bewältigung der Herausforderungen ausgestattet ist.

\subsection{Vor- und Nachteile}

Vorteilhaft kann sich die offenen Schreibweise in den Fällen erweisen, in denen eine öffentliche Diskussion die Arbeit des Forschers oder der Forscherin positiv beeinflusst. Im Rahmen der Erstellung einer Promotion ist das nur begrenzt von Vorteil, da die Arbeit "selbstständig" erstellt werden muss. Es wird sich zukünftig herausstellen, in welchem Umfang Kommentare oder kollaborative Schreibweisen diese Selbstsändigkeit gefährden oder ein "unerlaubtes Hilfsmittel" \cite{promotionsordnung_leuphana_kuwi_2011} darstellen. Bei der Erstellung dieser Qualifikationsarbeit wurde die Kommentar- und kollaborative Schreibfunktionen, die bei Authorea zur Verfügung standen, proaktiv deaktiviert um die Vorraussetzungen der Promotionsordnung vollumfänglich zu erfüllen. Einzige Möglichkeit den Autor zu kontaktieren war per E-Mail und die Möglichkeit auf der Webseite http://offene-doktorarbeit.de Literaturempfehlungen per Mail einzusenden. Davon wurde zwei Mal gebrauch gemacht.

Ein weiterer Vorteil bezog sich auf die Möglichkeit die unterschiedlichen Versionen und Revisionen der Arbeit einfach und transparent zu durchsuchen. Somit war es möglich den Erstellungsprozess einzelner Sätze oder Absätze auch im Nachhinein transparenz nachvollziehbar zu machen. Für den Autoren hatte das den Vorteil, dass er auch nach 2 Jahren Schreibphase noch immer die Möglichkeit hatte, Satzkonstruktionen und Gedankengänge nachzuvollziehen.

Die Aufmerksamkeit für die offene Schreibweise der Doktorarbeit im direkten Umfeld des Autors war groß. Inhaltlich hatte das zwar kaum Effekte, dennoch stieg die Anzahl der regelmäßigen Nachfragen bezüglich des Bearbeitungstandes der Arbeit, nach dem der Text einsehbar war. Der soziale Druck die Arbeit voranzubringen stiegt ebenfalls. Hinweise auf Fehler inhaltlicher oder rechtschreibtechnischer Natur blieben weitestgehend aus. Im Schnitt wurde die Webseite http://offene-doktorarbeit.de von 300 Besuchern pro Monat aufgerufen. Die Anzahl der Nutzer stieg fast linear mit der Anzahl der Blogbeiträge und dem fortschreitenden Schreibprozess der Arbeit.

---- TODO: Grafik bauen ----

Nachteile ergaben sich aus der fehlenden Verfügbarkeit von einfachen technischen, rechtlichen und konzeptionellen Standards bei der offenen Erstellung wissenschaftlicher Qualifizierungstexte und bei der Veröffentlichung von Forschungsdaten. Es fehlt an einfachen und zugänglichen Diensten und Applikationen, die es dem Autor oder der Autoren einfach macht den Text zu erstellen und Daten zu verwalten.

Mit LaTeX ist zwar ein wissenschaftlicher Textsatz vorhanden, allerdings ist dieser kompiziert und fehlt es an einem einfachen Frontend mit dem Funktionsumfang eines gängigen Texrverarbeitungsprogramms um dieses System offen, einfach und zu jedem Zeitpunkt verfügbar zu nutzen. Ergänzend war mit BibTeX ein Literaturdatenbanksystem für Literaturangaben für LaTeX-Dokumente verfügbar, das ebenfalls sehr komplex, aber umfassend unterstützt wird. Auch für BibTex gibt es mit wenigen Ausnahmen kein einfaches Frontend und die Bearbeitung, Validierung und Darstellung ist eher aufwändig.

Die Arbeit jederzeit in jedem Zustand für jeden im Internet einsehbar zu halten scheint im am Anfang befremdlich, wird aber nach einiger Zeit normal. Die Befürchtung im Bearbeitungszustand falsche oder fehlerhafte Inhalte zu verbreiten, stellt eine weitere Hürde bei der offenen Schreibweise dar. Bei der Erstellung dieser Arbeit wurde deshalb möglichst auffällig immer wieder auf den Zustand der Arbeit hingewiesen.

Als weiteren Nachteil gegenüber der analogen und geschlossenen Arbeit, ist der Aufwand für die Anonymisierung und Veröffentlichung der Umfrage-Daten kurz nach Abschluss der Erhebung zu deklarieren. Auch die Abklärung der rechtlichen Rahmenbedingungen nimmt bisher viel Zeit in Anspruch, ist aber ebenfalls überwindbar.

Die Gefahr, dass Inhalte fehlinterpretiert werden oder vorab wissenschaftlich anerkannt "veröffentlicht" werden und die Arbeit somit nicht mehr als unveröffentlichte Leistung anerkannt wird, konnte während des Verfassens dieser Arbeit nicht bestätigt werden. Dennoch sollte dieser Umstand bei zukünftigen Vorhaben beachtet werden.

\section{Kritische Betrachtung und Alternative Vorgehensweisen}

Seit Beginn der Erstellung der Arbeit hat die technische Entwicklung einige Fortschritte gemacht. In den letzten zwei Jahren ist eine Vielzahl von Tools und Applikationen veröffentlicht worden, die bei Berücksichtigung seit Beginn der Arbeit Vorteile vermuten lassen. Vor allem in Bezug auf die Darstellung von wissenschaftlichen Inhalten in dem Blogsystem Wordpress hat es einige Interessante Entwicklungen und Modifikationen gegeben. Diese konnten in ihrem vollen Umfang nicht mehr berücksichtigt werden.

Darüber hinaus wurden aus dem Umfeld der wissenschaftlichen Forschung zunehmend Dienste und Tools entwickelt, die die digitale Erstellung von Inhalten, die Öffnung der wissenschaftlichen Informationen und Verwaltung von Forschungsdaten stark vereinfachen.

Kritisch muss auch angemerkt werden, das Offenheit bei sämtlicher Kommunikation im Rahmen dieser Arbeit nur begrenzt möglich war. Zwar wurden die Zwischenstände, Präsentation und Entwicklungen beim Erstellungsprozess der Arbeit im Blog unter http://offene-doktorarbeit.de dokumentiert, doch wären weitere Möglichkeiten der Öffnung denkbar gewesen.

Darüber hinaus sind die Erfahrungen der Erstellung einer offenen geisteswissenschaftlichen Arbeit nur in Grenzen mit denen anderer Fächer zu vergleichen. Der Aufwand bei der Datenbereitstellung, die Vorgehensweise, die Darstellungsformen und der wissenschaftliche Prozess unterscheidet zu stark von dem Aufwand in anderen Disziplinen.

---- TODO: Weiter ausarbeiten und Alternativen technisch, inhaltlich wie konzeptionell aufzeigen ----

\section{Ergebnis}

Bisher muss der Aufwand für die Erstellung einer wissenschaftlichen Arbeit in einer geschlossenen Umgebung auf dem eigenen Rechner als geringer eingeschätzt werden, als die Texterstellung im Internet unter einer offenen Lizenz für jeden jederzeit einsehbar. Trotz mehrfachen Wechsels der Software, konnte keine einfache Lösung gefunden werden, die der Bedienbarkeit und Flexibilität der geschlossenen wissenschaftlichen Textbearbeitung auf dem Desktop entspricht.

Es ist davon auszugehen, dass der Forderung ---- Todo: Budapest usw. ---- heute ohne erheblichen Mehraufwand noch nicht nachgekommen werden kann. Nur mit genügend Kenntnissen kann der offene Schreibprozess und die zeitnahe und umfassende Veröffentlichung von Daten nach wissenschaftlichen Maßstäben erfolgen. Demnach müssen Wissenschaftler und Wissenschaftlerinnen in der Lage sein, die technischen Begrenzungen zu überwinden und die Rahmenbedingungen für den offenen Schreibprozess selbst zu schaffen. Weder die Forschungsinstitutionen noch private Anbieter sind bisher in der Lage Platformen für die Öffnung des gesamten wissenschaftlichen Prozesses anzubieten. Zu unterschiedlich sind die Anforderungen in den verschiedenen Disziplinen, zu mannigfaltig die Funktionen in den vorhanden Applikationen, zu uneinheitlich die Standards für das digitale Publizieren und zu verschieden der Kenntnisstand bei der Verwendung digitaler Methoden und Tools.

Insgesamt müssen die Vorteile und Nachteile der offenen Schreibweise ausgewogen betrachtet werden. Die offene Erstellung dieser Arbeit hat gezeigt, dass der Forderung nach Öffnung der wissenschaftlichen Kommunikation im Rahmen einer Promotionsarbeit grundsätzlich entsprochen werden kann. Letztendlich, so das Ergebnis des eigenen Experiments, sind durch die offene Schreibweise bisher weder fundamentale Vorteile, noch unlösbare Hürden für den publizierenden Wissenschaftler absehbar. Weitere Experimente mit der offenen Forschungsarbeit sind notwendig um abschließend zu evaluieren, ob eine solche Art des Verfassens von Forschungs- und Qualifizierungsarbeiten einen fundamentalen Vorteil für die Wissenschaft und die Öffentlichkeit bringen.

---- TODO: ausarbeiten ----
