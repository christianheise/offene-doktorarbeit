\chapter{Inhaltsanalyse: Entwicklungen, Definitionen und Debatten um Open Access und Open Science}
Wie in den vorangegangnene Kapiteln deutlich wurde, sind die Entwicklungen, Definitionen und Debatten um die Öffnung von Wissenschaft und Forschung in der gängigen Literatur weder einheitlich dargestellt, noch unumstritten \cite{muller_2010_open} \cite{schulze_2013_open}. Von hervorzuhebendem Interesse ist im Rahmen der Analyse der aktuelle Forschungsstand, die Debatte zur Öffnung von Wissenschaft und die Treiber und Bremser dieser Entwicklung und dem damit einhergehenden Wandel mit Fokus auf den Bereich wissenschaftliche Reputation.

Die Inhaltsanalyse wird an die foucaultschen Diskursanalyse angelehnt. Dazu werden aus einem Korpus von ausgewählten Texten die Entwicklungen, Definitionen und Debatten rund um den Themenkomplex der Öffnung von Wissenschaft und Forschung extrahiert und zusammengefasst. Durch die Auswertung der Ergebnisse, werden weitere wissenschaftliche Fragestellungen für die Befragung von publizierender Wissenschaftler und Wissenschaftlerinnen verschiedener Fachbereiche entwickelt. Abschließend werden in diesem Kapitel aus den Texten die Treiber und Bremser für die Öffnung von Wissenschaft identifiziert, für die Befragung extrapoliert und in der Gesamtbetrachtung der Arbeit zusammengeführt und strukturiert ausgewertet.

\section{Beschreibung des Forschungsstandes}
Wie in den vorangegangenen Kapiteln erläutert, befindet sich das wissenschaftliche Kommunikationssystem in der Krise. Seitdem wesentliche Bestandteile des wissenschaftlichen Kommunikationsprozess privatisiert wurden, haben sich akademischen Ziele und die Marktinteressen der Verlage immer weiter voneinander entfernt. Das derzeitige Geschäftsmodell von Verlagen ermöglicht den Verlegern Betriebsgewinnmargen von über 35 Prozent \cite{russell_2008_business} \cite{cope2014future} und hohe jährliche Wachstumsraten \cite{Wellcome_Trust_2003}. Die drei größten Wissenschaftsverlage vereinen bereits 42 Prozent aller Journale und trotz der internationalen Finanzkrise stiegen die Umsätze ungebremst. In den Jahren zwischen 2008 und 2011 stiegen die Umsätze um 11,7 Prozent und die Gewinne von 1,6 Milliarden auf 1,9 Milliarden Dollar (17 Prozent) \cite{cope2014future}. Das entspricht einer Umsatzredite von 35,8 Prozent. Zum Vergleich, die durchschnittliche Umsatzrendite im Wirtschaftszweig "Verlagswesen" bei deutschen Firmen mit mehr als 50 Mitarbeitern lag laut der Bundesbank dem im Jahr 2011 bei 11,6 Prozent \cite{bundesbank_2014}.

Sucht man nach Gründen für die Beibehaltung des bisherigen Modells durch die Wissenschaftsgemeinschaft, wird deutlich, dass vor allem Unwissen über die wirtschaftlichen Entwicklungen und das etablierte wissenschaftliche Reputationssystem einen zentral extrinsischen Motivationsfaktor für die Unterstützung des bisherigen Systems durch die wissenschaftliche Gemeinschaft darstellt \cite{minssen_2012_arbeit}. Ein weiterer Grund ist die komfortable Situation der Wissenschaftler in diesem System, bei dem Wissenschaftler und Wissenschaftlerinnen selten auf Zugang zu wissenschaftlichen Publikationen verzichten müssen und von der Auseinandersetzung mit den finanziellen Aspekten weitestgehend befreit sind \cite{herb_2010}. Die Verschärfung der Krise und die spürbaren Auswirkungen auf die wissenschaftliche Gemeinschaft unterstützen die Forderung nach Veränderung des Systems.

Als Ausweg aus dieser Krise wird immer wieder die Öffnung der wissenschaftlichen Kommunikation durch Konzepte wie Open Access und Open Science genannt \cite{suchen}. Es gibt keine einheitliche Definitionen von Open Access und Open Science. Beide Begriffe umfassen eine Vielzahl von Annahmen über die Zukunft der Wissensbildung und Wissensverbreitung. Beide Begriffe funktionieren als Sammelbegriffe für unterschiedliche Aufassungen wie und wie weit Wissenschaft offener werden kann. Sie sind Bestandteil eines lebendige Diskurses in der wissenschaftlichen Gemeinschaft \cite{schulze_2013_open}. Kleinster gemeinsamer Nenner in dem Diskurs um die unterschiedlichen Konzepte ist, "dass wissenschaftliche Forschung sich irgendwie mehr öffnen muss" \cite{cite:9}.

\section{Forschungsfragen}

Mit Hilfe der Inhaltsanalyse und der Analyse der Definitionen und Debatten sollen folgende Fragen beantwortet werden:
\begin{itemize}
\item Warum kommt es zu der Forderung nach der Öffnung von Wissenschaft?
\item Welche Argumente für und gegen die Öffnung von Wissenschaftlicher Kommunikation gibt es?
\item Wie können Open Science und Open Access definiert und voneinander abegrenzt werden?
\item Warum ist die Öffnung von Wissenschaft und Forschung in den verschiedenen wissenschaftlichen Disziplinen unterschiedlich stark etabliert?
\item Was bedeutet Offenheit und freier Zugang im Rahmen des wissenschaftlichen Diskurs-, Reputations- und Machtbegriffs?
\end{itemize}

\section{Erhebungsmethode und Umfang}

Die Literaturanalyse ermöglicht es, existierenden Erkenntnisse über ein Thema darzulegen und aufzuzeigen in welchen Bereichen weitere Forschung angestrebt werden sollte \cite{webster2002analyzing}. Für die Analyse wurden xxx Quellen ausgewählt und analysiert. --- Todo ausführen ----

\section{Analyse der Debatte um die Öffnung wissenschaftlicher Kommunikation}

Friesike und Fecher unterscheiden drei Herangehensweisen wie der Prozess der Wissenserstellung, das Ergebnis dieser Wissensproduktion, der Forscher selbst und die Beziehung zwischen Wissenschaft und dem Rest der Gesellschaft im Rahmen von "Offenheit" definiert werden können \cite{cite:9}:
\begin{enumerate}
\item das demokratische Recht auf Zugang von Wissen (Open Access)
\item die Forderung nach Einbeziehung der Öffentlichkeit in Forschung (Citizen Science) bei der Verwendung kolaborativer Forschungsinstrumente und -mehthoden
\item --- TODO weiter ausarbeiten ----
\end{enumerate}

Schulze und Stockmann wählen einen anderen Ansatz, sie unterscheiden "zum einen die Offenheit des Zugangs zu Daten, Code oder Ergebnissen, zum anderen das Gebot der Transparenz, also der Offenlegung von Verfahren, Methoden und Zielen" \cite{schulze_2013_open}.

--- TODO weiter ausarbeiten ----

\subsection{Herausforderungen im bestehenden System wissenschaftlicher Kommunikation}

Über die Wirksamkeit und Zweckmäßigkeit des wissenschaftlichen Kommunikationssytems existieren seit Jahrzehnten Debatten in der Literatur \cite{suchen}. Die in der Literatur genannten Herausforderungen im bestehenden System formeller wissenschaftlicher Kommunikation beziehen sich auf zehn Bereiche:
\begin{enumerate}
\item Leistungsbewertung und Qualitätssicherung
\item Geschwindigkeit
\item Freiheit von Wissenschaft und Forschung
\item Kosten und Effizienz
\item Fehlerresistenz
\item Verbreitung und Zugänglichkeit
\item Digitalisierung
\item Reliabilität und Validität
\item Objektivität und Unabhängigkeit
\item Missbrauch
\end{enumerate}

--- TODO weiter ausarbeiten ----

\subsubsection{Leistungsbewertung und Qualitätssicherung in der Wissenschaft}

Die Verlage haben in den letzten Dekaden mit den wissenschaftlichen Journalen ein zentrales Steuerungs- und Bewertungssystem in der Wissenschaft etablieren können. In diesem System werden die Grundprinzipien der Wissenschaft für die verlegerischen Verwertungsinteressen (aus)genutzt und das, obwohl diese “wissenschaftlichen Grundprinzipien und Normen eigentlich ökonomischen Verwertungsinteressen zu widersprechen scheinen” \cite{hanekop_2006}. Darüber agieren die Forscherinnen und Forscher in einem Umfeld, in dem sie in vielen Fällen wenig oder keine Verantwortung für den Einkauf der wissenschaftlichen Informationen haben, die er oder sie im Rahmen der Veröffentlichung "verschenkt" \cite{steele_2006}.

Die Einführung der quantitativer Bewertungsindikatoren wie das Zitationsregister und die Impact Faktoren, sowie die Definition der Kernzeitschriften, führte zu einer weitgehenden Erstarrung des wissenschaftliche Zeitschriftenmarktes und gleichzeitig zu einem Anstieg der Kapazität der kommerziellen Verlagen, sowie deren Gewinnmargen \cite{CREATe_2014}. Die Steuerungsmechanismen werden über die Messbarkeit mittels Methoden direkt oder indirekt ausgeübt. Dabei stehen insbesondere die Methoden, die auf der quantitativen Grundlage der Zitationsraten wissenschaftlicher Publikationen gemessen werden in der Kritik \cite{Dong_2005} und auch andere Indikatoren für die Messung von Forschungsleistungen sind hoch umstritten \cite{Hornbostel_1997} \cite{Hicks_1996} \cite{Havemann_2002}. Die Verfahren, um die Wirkung von Wissenschaft und damit auch die Reputation von Wissenschaftlern zu messen, sind kein eigentliches wissenschaftliches Produkt \cite{suchen} und erfassen zum Beispiel die Tätigkeit einzelner Forschergruppen zu stark \cite{schmoch_2009}. Darüber hinaus sind "weder importance noch impact noch quality direkt meßbar" und man kann sich ihnen nur "nähern" \cite{Hornbostel_1997}. Das führt unter anderem dazu, dass der aus der "Zahl der Zitationen auch die Beiträge einer Zeitschrift ermittelte" \cite{weishaupt_2009_goldenOA} Impact Factor nicht  als perfektes Werkzeug betrachtet werden kann, um die Qualität der Artikel zu messen \cite{garfield_1999}. Trotzdem wird er zur Bewertung von Wissenschaft genutzt, denn “es gibt nichts Besseres" und er hat den Vorteil, dass er allein durch seine lange Existenz "eine gute Technik für die wissenschaftliche Bewertung” darstellt \cite{garfield_1999} \cite{weishaupt_2009_goldenOA}.

Die Kritik am Impact Faktor lässt sich laut der Bibliotheks- und Informationswissenschaftlerin Dr. Karin Weishaupt, am Beispiel des "Thomson Reuters Journal Citation Factors" in sechs Punkten zusammenfassen \cite{weishaupt_2009_goldenOA}:
\begin{enumerate}
\item Der Impact Factor bezieht sich immer auf die gesamte Zeitschrift und hat somit keine Aussagekraft über die "Rezeption oder Qualität des einzelnen Artikels" \cite{weishaupt_2009_goldenOA}.
\item Der Impact Factor berücksichtigt nur die Zeitschriften, die im eigenen Index gelistet sind und enthält weder Monographien, Tagungsbeiträge, sonstige Beiträge oder Internetquellen.
\item Durch Selbstzitierungen sind Manipulationen möglich.
\item Es werden nur Zitate aus den letzten beiden Jahren berücksichtigt und je nach Fachgebiet ist es von Vorteil wenn im eigenen Gebiet die Verwertungszyklen kürzer sind.
\item Publikationen, die nicht in englischer Sprache verfasst sind, weisen meist eine geringere Sichtbarkeit und Popolarität auf, da englische Journale überproportional vertreten sind
\item Spezialisierte Zeitschriften sind ebenfalls systematisch benachteiligt gegenüber Journalen großer Fach-Communities oder Journalen mit Übersichtsartikeln.
\end{enumerate}

Es bleibt festzuhalten, dass die im wissenschaftlichen System genutzten Indikatoren die komplexe Realität der Leistungsbewertung in der Wissenschaft nicht abbilden können und sie eine eigene Realität konstruieren \cite{Hornbostel_1997}. Versteht man Wissenschaft als soziales System, so stellen Reputation und nicht die Wahrheit der Beobachtungen und Erklärungen "nicht selten auch eingestandenes vorrangies Ziel wissenschaftlicher Tätigkeit" dar \cite{luhmann_1970_selbststeuerung}. Wie gering der Wirkungsgrad und die Methoden zur Messung “zur Reproduktion des traditionellen wissenschaftlichen Diskurses ausfall(en), wird von dem Moment an klar, an dem ein neues und offenes Kommunikationsmedium wie das Internet als alternativer Publikations- und Verbreitungskanal für Wissenschaft zur Verfügung steht" \cite{Rost_1998}.

\subsubsection{Geschwindigkeit}

Einen weiterer Aspekt der Debatte betrifft der zeitlichen Geschwindikeit zwischen der Fertigstellung einer wissenschaftlichen Arbeit durch den Autoren und der finalen Veröffentlichung der Ergebnisse.

Trotz der Beschleunigung der Prozesse bei der Qualitätssicherung und Bewertung von wissenschaftlichen Arbeiten durch die Digitalisierung der Kommunikation zwischen Wissenschaftlern, Gutachtern und Verlagen kann es bis zu mehrere Jahre dauern, bevor ein Text veröffentlicht wird \cite{suchen}. Diese Verzögerung beruht auf folgenden Umständen:

\begin{enumerate}
\item Gutachter/innen können aufgrund der Ausführung dieser Funktion als Nebentätigkeit meist Termine nicht einhalten \cite{suchen}.
\item Es gibt weder Anreiz- noch Sanktionsmöglichkeiten für Gutacher und Gutachterinnen.
\item Die wissenschaftlichen Zeitschriften erscheinen größtenteils noch immer als Periodika und wissenschaftlichen Bücher orientieren sich am Druck. Sie sind damit für einen bestimmten Zeitraum der Veröffentlichung terminiert.
\end{enumerate}

Eine Möglichkeit die wissenschaftlichen Inhalte schneller zugängnlich zu machen, ohne den sehr zeitaufwändigen Begutachtungsprozess strukturell oder inhaltlich zu verändern, ist die Veröffentlichung der wissenschaftlichen Arbeit als Pre-Print. Eine weitere Möglichkeit stellt die offene Begutachtung dar, bei der ein Text anonymisiert veröffentlicht wird und von der wissenschaftlichen Gemeinschaft kollaborativ bewertet wird.

--- Todo: weiter anhand von Literatur ausarbeiten ----

\subsubsection{Freiheit von Wissenschaft und Forschung}

Die Freiheit von Wissenschaft und Forschung, in diesem Fall insbesondere die Publikationsfreiheit, ist im aktuellen System des wissenschaftlichen Austauschs nicht direkt gefährdet, wird aber durch indirekte Faktoren stark beeinflusst \cite{suchen}. So fördert das System die Publikationsformen und -kanäle, die von der wissenschaftlichen Gemeinschaft der jeweiligen Fachdisziplin als etabliert betrachtet werden. Neue Formen und Kanäle hingegen werden nur selten im Rahmen der formellen Kommunikation berücksichtigt. Für sie ist es besonders schwer im Reputationssystem Fuß zu fassen.

--- Todo: weiter anhand von Literatur ausarbeiten ----

\subsubsection{Kosten und Effizienz}

An dem Kosten-Nutzen-Verhältnis des aktuellen wissenschaftlichen Kommunikationssystems gibt es seit Jahren detaillierte und grunsätzliche Zweifel. Für die Veröffentlichung einzelner Texte ergeben sich je nach Schätzungen bis zu xxxx Dollar pro veröffentlichten Text und xxxx Dollar pro veröffentlichtem Buch.

--- Todo: weiter anhand von Literatur ausarbeiten ----

\subsubsection{Fehlerresistenz}

Trotz des aufwändigen wissenschafltichen Qualitätssicherungsystems kommt es immer wieder zu Fehlern und falschen Aussagen bei der Veröffentlichung wissenschaftlicher Erkenntnisse und Ergebnisse.

Um der Problematik der unzureichenden Fehlerressistenz zu begegnen, werden in der Literatur folgende Herangehensweisen genannt:
\begin{enumerate}
\item offene Begutachtungsverfahren, bei der mehr als ein Gutachter die Möglichkeit bekommt die Inhalte auf Fehler zu prüfen \cite{suchen}
\item Öffnung der Daten und Methoden hinter der Publikation \cite{suchen}
\item umfassende Dokumentation des wissenschaftlichen Wertschöpfungsprozesses um Fehler bereits bei der Erstellung der Publikation für Forscherinnen und Forscher sichtbar und transparent nachvollziehbar zu machen \cite{suchen}
\end{enumerate}

\subsubsection{Verbreitung und Zugänglichkeit}

Ebenso wie die Frage nach der optimalen Geschwindikeit des aktuellen wissenschaftlichen Kommunikationssystems, stellt sich auch die Frage nach der optimalen Verbreitung und Zugänglichkeit wissenschaftlicher Informationen. Während die Geschwindigkeit auf die zeitliche Komponente von der Herstellung bis zum Vertrieb des Wissens abzielt, geht es bei der Frage nach Verbreitung um die Verfügbarkeit des Wissens für eine möglichst große Rezipientengruppe. Es gibt erhebliche Zweifel daran, ob es sich bei dem aktuellen System um ein System mit optimalen Vorraussetzungen für eine möglichst hohe Verbeitung von Wissen an die Gesamtgesellschaft handelt \cite{suchen}.

--- Todo: weiter anhand von Literatur ausarbeiten ----

\subsubsection{Digitalisierung}

Die Digitalisierung der wissenschaftlichen Kommunikation beschränkt sich bisher Wesentlichen darauf, dass die analog gedruckten und bewährten Journale, sowie andere Publikationsformen der großen wissenschaftlichen Verlage mit nahezu unverändertem Geschäftsmodell digital verbreitet werden \cite{Hanekop_2014}. Die digitale Distribution wird in diesem Zusammenhang als weiterer Kanal nach dem Drucken der Informationen verstanden. Die Möglichkeiten, die die Digitalisierung bietet, sind damit bei weitem nicht ausgeschöpft.

--- Todo: weiter anhand von Literatur ausarbeiten ----

\subsubsection{Reliabilität und Validität}

Die Zuverlässigkeit des Kommunikationssystems kann anhand dessen geprüft werden, ob die Einreichung einer Arbeit über unterschiedliche Wege den selben Erfolg hat beziehungsweise, wie stark Zufallsfaktoren den Erfolg der Einreichung beeinflussen. In dem aktuellen System wird die Replizierbarkeit und Zuverlässigkeit von Ergebnissen stark kritisiert und bezweifelt \cite{suchen}.

--- Todo: weiter anhand von Literatur ausarbeiten ----

\subsubsection{Objektivität und Unabhängigkeit}

Die Kenntnis von Eigenschaften der Autoren durch die Gutachter stellt eine der größten Herausforderungen für die Wahrung der Objektivität und Unabhängigkeit im wissenschaftlichen Qualitätssicherungsprozess dar. Aber auch bei anderen Formen der wissenschaftlichen Bewertung können Unabhängigkeit und Objektivität nicht gewährleistet werden. In der Literatur finden sich Beiträge, die sich mit Forschungsförderung beschäftigen und kommen mehrheitlich zu dem Ergebnis kommen, dass die Objektivität und Unabhängigkeit im bestehenden System nur schwer bis nicht gesichert werden können \cite{suchen}.

--- Todo: weiter anhand von Literatur ausarbeiten ----

\subsubsection{Missbrauch}

Die ethischen Grundsätze stellen in der wissenschaftlichen Debatte von Beginn an eine Besonderheit dar. Vertrauen, das Interesse aller Akteure an optimaler Kommunikation zwischen den Wissenschaftlern, Ehrlichkeit und der Ausschluss von Interessenskonflikten sind Grundpfeiler im wissenschaftlichen Wertschöpfungs- und Kommunikationsprozess.

Diesem wissenschaftlichem Ethos stehen die Beispiele gegenüber, bei denen bewusster Missbrauch durch Akteure des Kommunikationssystems zu Verwiklichung partikularer Interessen oder konkreten Einfluss auf wirtschaftliche Aspekte geführt haben \cite{suchen}.

Diesem Missbrauch im wissenschaflichen Kommunikations- und Reputationssystem kann nur durch ein größtmögliches Maß an Transparenz begegnet werden.

--- Todo: weiter anhand von Literatur ausarbeiten ----

\section{Analyse von Open Access: Zugang zu wissenschaftlicher Kommunikation}

Der etablierte Prozess wissenschaftlicher Kommunikation steht vor großen Herausforderungen. Die Zeitschriften- und Monographienkrise, der zunehmende finanzielle Druck im Rahmen der öffentlichen Finanzierung von Wissenschaft, die Veränderungen im wissenschaftlichen Kommunikationsprozess durch neue Arten und Möglichkeiten der Distribution, die steigenden Beschaffungskosten für wissenschaftliche Literatur \cite{cite:17} \cite{muller_2010_open}, sowie die Veränderungen in der Rezeption von Inhalten \cite{holub_2013_reception} zwingen zum Umdenken in der wissenschaftlichen Kommuinkationspraxis \cite{suchen}. Die anhaltende Forderung nach mehr Offenheit im wissenschaftlichen Kommunikationsprozess entwickelte sich zu einem konkreten Lösungsansatz für die Herausforderungen an das etablierte System. Nachfolgend wird die Debatte um das Modell des Offenen Zugangs zu Wissenschaft analysiert.

Der Schwerpunkt dieser Analyse beruht auf den Themenbereichen wissenschaftliche Reputation und (Effizienz der) Kommunikation. Dieser Zugang beruht auf der Annahme, dass die Öffnung der wissenschaftlichen Kommunikation eine große Chance für Veränderungen im wissenschaftlichen Qualitäts- und Reputationssystem darstellt. Diese Chancen beziehen sich auf die Aktivität der Wissenschaftler und die Qualtiät der Forschungsergebnisse. Die wissenschaftlichen Erkenntnisse werden bisher häufig erst nach langen intransparenten Verfahren bewertet, publiziert und nur an einen beschränkten Kreis von Rezipienten vermittelt. Diese intransparenze Praxis hat einen signifikant-negativen Einfluss für Allokation von Ressourcen durch die öffentliche Hand und die Kosten die im Rahmen öffentlich-finanzierter Forschung entstehen \cite{suchen}.

Als Auslöser für die Entwicklung von Open Access werden auch die infrastrukturellen Veränderungen angeführt, die "seit spätestens Mitte der 1990er-Jahre entscheidend Einfluss auch auf die Wissenschaftskommunikation und das wissenschaftliche Arbeiten genommen haben" \cite{schulze_2013_open}. Wissenschaftliche Informationen werden seither nicht nur in "digitaler Form konsumiert, sondern auch kollaborativ und kooperativ, zeitlich versetzt, durch teilweise räumlich weit verstreute Arbeitsgruppen und Forschungsverbünde, genutzt und weiterverarbeitet" \cite{schulze_2013_open}. Die Verbreitung und Akzeptanz von Open Access variiert dabei zwischen den einzelnen wissenschaftlichen Disziplinen erheblich \cite{cite:21a}.

Bei der Betrachtung der Fordernung nach Öffnung wissenschaftlicher Kommunikation muss allerdings klar zwisschen den Konzepten von Open Access und Open Science unterschieden werden. Bei Open Access geht es um einen möglichst uneingeschränkte Zugang zu finalen wissenschaftlichen Ergebnisspublikationen für die Gesamtgesellschaft. Open Science beschreibt hingegen den umfassenden Zugriff auf den gesammten wissenschaftlichen Wertschöpfungsprozess inklusive aller Daten und Informationen, die beider Erstellung, Bewertung und Kommunikation der wissenschaftlichen Erkenntnisse enstanden sind.

\subsection{Offener Zugang zur wissenschaftlichen Kommunikation: Definitionen von Open Access}

Open Access wird von Peter Suber als "digital, online, kostenlos, und frei von den meisten Urheber- und Lizenzbeschränkungen" \cite{suber_2012_open} definiert \cite{Adema_2014_open_access}. Open Access bedeutet den "Verzicht auf die finanzielle, technische und rechtliche Hindernisse, die dazu bestimmt sind, den Zugang zu wissenschaftlichen Forschungsartikel für zahlende Kunden zu begrenzen" und dass, "im Interesse der Beschleunigung der Forschung und den Austausch von Wissen, Verlage ihre Kosten aus anderen Quellen schöpfen" \cite{Suber_2002}. In der Literatur herrschen unterschiedliche Auffassungen über die Definition Open Access, wie es erreicht werden kann und welchen genauen Bezugsrahmen das Attribut "Open" aufweist \cite{Adema_2014_open_access}. Dies ist darauf zurückzuführen, dass es keine formelle Struktur, keine offizelle Organisation und keinen ernannter Führer innerhalb der Open Access Bewegung gibt \cite{poynder_2011_suber}.

Bei der Forderung nach Open Access geht es nicht um die Abschaffung oder die Entwertung materiellen geistigen Eigentums. Die Bewegung vereint das gemeinsame Ziel, "die Bedingungen zu verbessern, unter denen wissenschaftliche Arbeiten zirkulieren können"\cite{Adema_2014_open_access}. Die Propagierung der Öffnung der wissenschaftlichen Ergebnispräsentation ersteckt sich "auf solche Publikationen, die nicht darauf angelegt sind, Einnahmen aus Verkaufserlösen für ihre Urheber zu generieren" \cite{muller_2010_open}.

Exemplarisch gelten folgende Definitionen als Ansatzpunkt für das Verständnis von Open Access in dieser Arbeit. Im Rahmen der Inhaltsanalyse wurden sie auf Grundlage hoher Zitationsraten als die gängigen Einordnungen für Open Access identifiziert:
\begin{tabular}{lcccc}
\textbf{Autor (Jahr)} & \textbf{Typ der Publikation} & \textbf{Titel} & \textbf{Inhalt} & \textbf{BBB-Bezug} \\
Peter Suber (2004) & Webseite & A Very Brief Introduction to Open Access & "Open-access (OA) literature is digital, online, free of charge, and free of most copyright and licensing restrictions. What makes it possible is the internet and the consent of the author or copyright-holder." & ja \\
Gunther Eysenbach (2006)  & Artikel  & Citation Advantage of Open Access Articles & "Open access (OA) to the scientific literature means the removal of barriers (including price barriers) from accessing scholarly work." &  ja \\
Willinsky, J. (2006) & Artikel & The access principle: The case for open access to research and scholarship. & increasing access and improving access to the journal literature, largely through the use of the Internet &  ja \\
Harnad, Stevan, et al. (2008)  & Buch  & The access/impact problem and the green and gold roads to open access: An update. & "full texts are accessible online toll-free—let us call that “Open Access” (OA), in line with the definition provided in 2001 by the Budapest Open Access Initiative" &  ja \\
Uwe Müller (2010) & Buchkapitel & Open Access. Eine Bestandsaufnahme. & "Mit Open-Access wir der freie, unmittelbare und uneingechränkte Zugang zu wissenschaftlichen Publikationen und Forschungsergebnissen in elektronischer Form bezeichnet." &  ja \\
Laakso M., et al. (2011) & Artikel & The Development of Open Access Journal Publishing from 1993 to 2009 & "Open Access (OA), in the context of scholarly publishing, is a term widely used to refer to unrestricted online access to articles published in scholarly journals." &  ja \\
Herb, Ulrich (2012) & Buchkapitel & Offenheit und wissenschaftliche Werke: Open Access, Open Review, Open Metrics, Open Science und Open Knowledge & "Open Access bezeichnet demnach die Möglichkeit, wissenschaftliche Dokumente entgeltfrei nutzen zu können" & ja  \\
\end{tabular}

--- Todo: Tabelle weiter ausfüllen - Auflistung Open Access Definitionen in der Literatur &
Zusammenfassung der Definition ---

Das Attribut "Open" definiert den Bezugsrahmen für den Zugang zu wissenschaftlichen Publikationen. Eine der verbreitetsten Definition der Bedingungen von "Open" ist die Open Definition. Sie hat den Anspruch die Prinzipien und Bedingungen für die Offenheit von Daten und Inhalten zu definieren.

Gemäß der Open Definition gilt der Inhalt als "Open", der "für jeden Zweck von jedem kostenlos genutzt, modifiziert und geteilt werden" \cite{open_definition} kann. Ziel dieser Definition ist es, "die Bedeutung von offen in Bezug auf Wissen" zu präzisieren. Wissen erstreckt sich in diesem Zusammenhang auf Inhalte wie Musik, Filme, Bücher, jegliche Art von Daten, ob wissenschaftlicher, historischer, geographischer oder anderer Art und Regierungs- und andere Verwaltungsinformationen \cite{open_definition}.

Die Open Definition wurde von der Open Scource Definition abgeleitet und ist als synonym für "frei" oder "libre" im Rahmen der Definition für "freie kulturelle Werke" zu verstehen \cite{suchen}. Ein Werk oder Inhalt gilt nach dieser Definition als "offen", wenn es bei der Verbreitung folgenden Kriterien erfüllt:
\begin{enumerate}
\item Einhaltung der Prizipien von Zugang, Verteilung, Wiederverwendung und dem Fernbleiben von technologischen Restriktionen
\item Attribuierung, Integrität als maximale Einschränkung
\item Unterbindung der Diskiminierung von Personen, Gruppen oder bestimmten Bereichen/Gebieten
\item Einhaltung genannten Kriterieun im Rahmen der Lizensierung
\end{enumerate}

--- Todo: Grafik von http://de.slideshare.net/christianheise/gfm-open-whaaaaaaat102013bmch-27451852 ---

\subsection{Open Access Modelle und Formen}

In der Literatur wird Open Access in unterschiedliche Formen unterteilt und es existieren mehrere Definitionen \cite{CREATe_2014} \cite{albert_2006_open_implications}. Darüber hinaus bestehen unterschiedliche Auffassungen über die verschiedenen Modelle von Open Access \cite{CREATe_2014} \cite{cite:22b} \cite{lewis_2012_inevitability}. Sowohl die Definitionen, als auch die Modelle orientieren sich an den "three Bs", den derzeit geltenden Definitionen von Open Access \cite{Adema_2014_open_access}. Am Beispiel der Budapest Open Access Initiative werden zwei Wege für Open Access dargestellt \cite{albert_2006_open_implications}:
\begin{enumerate}
\item Die Etablierung "einer neuen Generation von Fachzeitschriften", die einen kostenfreien und unmittelbaren Zugang zu den Beiträgen ermöglichen ("goldener" Weg)
\item Die öffentlich zugängliche (Selbst-)Archivierung durch die Urheber ("grüner" Weg)
\end{enumerate}

Der "grüne Weg" beschreibt ein Modell, bei dem der Autor im Rahmen einer (Selbst-)Archivierung von Beiträgen in Repositorien (öffentlichen Dokumentenservern) die öffentliche Verfügbarkeit seiner Publikation anstrebt \cite{muller_2010_open}. Das vom Autor inital eingereichte Dokument (Manuskriptfassung) steht dabei als Pre-Print oder Post-Print-Version auf institutionellen oder disziplinären Dokumentenservern \cite{suchen} oder privaten Homepages \cite{suchen} jedem zur Verfügung. Im Unterschied zu Post-Prints, hat bei Pre-Print keine Peer Review stattgefunden \cite{suchen} und der Beitrag hat somit keine externe wissenschaftliche Qualitätssicherungsmaßnahme durchlaufen. Beim "grünen Weg" hat der publizierende Verlag darüber hinaus die Möglichkeit innerhalb einer Speerfrist von überlicherweise 6-12 Monaten \cite{suchen} den lektorierten und fertig-publizierten Beitrag unter einer eigenen Lizenz zu verkaufen \cite{suchen}. Erst nach Ablauf dieser Frist wird die finale und lektorierte Fassung des Beitrags frei und offen zur Verfügung gestellt. Es existieren je nach Verlag und Publikationsform verschiedenen Möglichkeiten der Ausgestaltung dieses Publikationsweges \cite{suchen}. Sie alle einigt die Möglichkeit für den Autor seinen eingereichten Beitrag unmittelbar, frei und kostenlos zu veröffentlichen und die freie und kostenlos Veröffentlichung der finalen Publikation durch den Verlag nach einer Speerfrist.

Beim "goldene Weg" stellt der Autor unmittelbar nach der Fertigstellung die finale und lektorierte Publikation über einen Verlag frei und offen zur Verfügung. Auch die Verlagsversion muss ohne Sperrfrist in einem Repositorium unmittelbar zur Verfügung gestellt werden. Der Verlag hat allerdings zusätzlich die Möglichkeit, den Beitrag kommerziell zu vertreiben und zu verkaufen, muss jedoch parallel eine freie und offene Version der Publikation zur Verfügung stellen. Alternativ ermöglicht es der verzögerte goldene Open Access-Weg dem Verlag, zeitverzögert für die Öffentlichkeit die finale Version der Publikation unter einer offenen Lizenz zur Verfügung zu stellen \cite{lewis_2012_inevitability}. Der Verlag hat bei diesem verzögerten Modell den Vorteil, einen bestimmten Zeitraum die Publikation vertreiben zu können, ohne zeitgleich eine offene und freie Version anbieten zu müssen. Der Autor hat im Gegensatz zum "grünen Modell" aber dennoch die Möglichkeit diese finale Publikation sofort kostenfrei anzubieten.

Im Rahmen anderer Modelle, meist gemischter Modelle, wird den Autoren im Nachhinein die Möglichkeit eingeräumt, durch zusätzliche Zahlung, die Publikation offen und frei zur Verfügung zu stellen\cite{lewis_2012_inevitability}. Das hat für den Autor den Nutzen, dass er von den Vorteilen bei der offenen Verbreitung von Publikationen unter den Bedingungen von Open Access profitiert. Macht der Autor davon erst nach einem gewissen Zeitraum gebrauch, generiert der Verlag neben den intialen Verkaufserlösen über diesen Weg zusätzliche Einnahmen. Diese alternativen Modelle ermöglichen es, das trotz des Open Access Publizierens parallel zu den kostenlosen und offenen elektronischen Veröffentlichungen weitere kostenpflichtige Publikation in gedruckter oder digitaler Form erfolgen können. Eine Grundvorraussetzung dafür ist nur, dass neben der kostenpflichtigen Version, auch eine kostenfreie Version der Publikation unter den in der Open Definition erklärten Bedingungen exisitiert.

Darüber hinaus findet seit kurzem in der Literatur die Segmentierung in gratis und libre Open Access statt. Mit gratis Open Access wird dabei die Möglichkeit bezeichnet, den Zugang zu Publikationen und Forschungsergebnisse zu erleichtern und die Kostenpflichtigkeit zu beenden. Libre Open Access bedeutet, dass weitere Barrieren, wie Urheber- und Lizenzbeschränkungen aufgehoben werden. \cite{Adema_2014_open_access} Diese Unterteilung wird von einigen Autoren kritisiert, da durch das Hinzufügen eines weiteren Attributs die eigentlich scharfe Abrenzung von "Close" und "Open" geschwächt wird, was sich auch auf andere Bereiche der Open-Bewegung (Open Data, Open Government, Open Spending uvm.) auswirken könnte \cite{suchen}.

Neben den dargestellten Modellen existieren weitere Veröffentlichungsmodelle für Open Access Publikationen. Die Einteilung in hybride, radikale und sonstige Formen von Open Access stellt dabei eine weitere Ebene der Unterteilung dar. Weitere, aber im Vergleich wenig genutzte Modelle sind hybride Modelle. Als hybrid werden diese deshalb bezeichnet, weil der Autor wählen kann, ob er den Verlag für den kostenlosen Zugriff auf seine Publikation finanziert oder der Leser über Subskriptionsmodell zahlt \cite{muller_2010_open}. Dieses Modell steht aber in der Kritik, da die rechtlichen Bedinungen nur selten eine Nachnutzung oder Weiterverbreitung erlauben und die Verlage nur selten auf das exklusive Verwertungsrecht verzichten \cite{muller_2010_open}. Diese Publikationsformen werden als Open Access bezeichnet, genügen aber nicht den gängigen Deklarationen \cite{boai_2012} oder verstoßen gegen die Open Definition. Entspricht eine Veröffentlichung nicht der Open Definition wird aber vom Verlag oder der herausgebenden Institution als "Open" bezeichnet, so wird auch von "Open Washing" gesprochen \cite{suchen}. Von einer weiteren Unterteilung der Open Access Modellen wird deshalb und aufgrund ihrer geringen Verbreitung und Praktikabilität in dieser Arbeit abgesehen.

\subsection{Open Access als Geschäftsmodell}
Der verzögerte goldene Weg und grüne Weg beeinträchtigen das klassische Geschäftsmodell der Verlage vorerst nicht direkt. Publikationen werden wie bisher angeboten und erst nach einer bestimmten Zeit auch kostenlos zur Verfügung gestellt. Im Gegensatz dazu kommt der goldene Weg, auf Grundlage unmittelbarer, freier und offener Veröffentlichungspflicht, ohne das tradierte Geschäftsmodell der Verlage aus \cite{lewis_2012_inevitability}.

Allerdings werden für Publikationen, die unter den Bedingungen von Open Access veröffentlicht werden, durch die Verlage vorab Veröffentlichungsgebühren von den Autoren erhoben \cite{suchen}. Diese sogenannten article processing charges (APC) werden damit gerechtfertigt, dass bei dieser Publikationsform weder auf den Peer-Review-Prozess, noch auf die Möglichkeit Umsatz zu generieren, Urheber zu schützen oder andere Stärken der traditionellen Publikationsformen verzichtet wird \cite{albert_2006_open_implications} \cite{Open_Access_net_2009}.

Somit ändert das Open Access Geschäftsmodell die Erlöstruktur der Verlage von nachgelagerten, verkaufsorientierten Einnahmen hin zu Vorabeinnahmen für die Erstellung und den Vertrieb der Publikation. Strukturell steht Open Access für Verlage damit vorerst in keinem Widerspruch zur Bewahrung der wissenschaftlichen Qualität oder den Vorteile des klassischen Publikationssystems \cite{Suber_2002}. Verlage nutzen zwar Open Access-Optionen, wollen damit aber die etablierten Verhältnisse möglichst fortschreiben und halten am Subskrikptionsmodell weiter fest \cite{schmidt_2007_goldenen}.

--- Todo: Tabelle - Auflistung Open Access  Modelle und Formen in der Literatur
Gegenstand / Zeitraum / Referenz
Zusammenfassung der Definition ---

\subsection{Open Access Kanäle und Formate}
In diesem Abschnitt wird auf unterschiedliche Open Access Kanäle und Publikationsformate eingegangen. Es wird unterschieden in: Open Access Aggregatoren, Open Access Repositorien, Open Access Journals, Open Access Bücher und Monografien. Diese Kanäle und Formate adressieren die unterschiedlichen Publikationsformen der wissenschaftlichen Kommunikation oder konkrete Herausforderungen in Bezug auf die Distribution und Archivierung im Rahmen der neuen Möglichkeiten von offenem und freien Publizierens.

Da es eine enge Verknüpfung zwischen Repositorien und der Entwicklung der Open-Access-Bewegung gibt \cite{offhaus_2012_institutionelle_repos}, soll in diesem Kapitel auf die Rolle der Repositorien als spezifischen Kanal für die Verbreitung von Publikationen eingegangen werden. Repositorien sind verwaltete Orte zur Aufbewahrung geordneter Dokumente, die im Unterschied zu Archiven ausschließlich historische Dokumente verwalten und öffentlich zugänglich sind \cite{suchen}. Institutionelle Repositorien gelten als ein Instrument für wissenschaftliche Einrichtungen, um Publikationen für einen institutionell abgegrenzten Bereich frei zugänglich zu machen \cite{dobratz_2007_open}.

Institutionelle Repositorien haben erhebliche Vorteile für die Institutionen, wenn sie in die ganzheitlichen Rahmenbedingungen der Universität integriert sind \cite{steele_2006}. Repositorien können neben der Kernufgabe der Archivierung und Verbreitung von Publikationen auf für die Lernumgebungen, den Forschungsservice und die Marketingaktivitäten einer Universität eine wichtige Rolle spielen. Sie ermöglichen zum Beispiel die Dokumentation des universitären Outputs und verbessern den institutionellen Austausch \cite{steele_2006}. Ökonomisch rentieren sie sich vor allem dann, wenn Skaleneffekte eintreten und Forschungseinrichtungen in Verbünden agieren \cite{blythe_2005value}. Neben den institutionellen sind auch fachliche oder andere Arten von Repositorien eng mit der Open Access Bewegung verknüpft. Repositiorien stehen  für die digitale Speicherung von Dokumenten und zunehmend auch Daten. Über die Repositorien wird der Zugang zu den unterschiedlichen Modellen von Open Access Publikationen ermöglicht.

--- Todo: Tabelle - Auflistung Open Access  Modelle und Formen in der Literatur Gegenstand / Zeitraum / Referenz Zusammenfassung der Definition ---

\subsection{Kritik an Open Access}

Im Rahmen der Literaturanalyse werden wesentliche Kritikpunkte an der Open Access Bewegung in Wissenschaft und Forschung dargestellt und erläutert. Die Auswahl der berücksichtigten Werke bezieht sich auf die für die Fragestellungen relevanten Beiträge und wird um die Betrachtung der Debatte von Open Access und Open Science ergänzt.

Aus der Perspektive der Leser gibt es wenig Kritik am Konzept von Open Access \cite{weishaupt_2009_goldenOA}. Sie bezieht sich zumeist auf die Befürchtung aus den Konsequenzen der Öffnung von Wissenschaft und Forschung.
--- weiter ausarbeiten: sinkende Forschungshetrogenität und Einflussnahme durch "Steuerzahler" und Unterwanderung der Steuerungsmechanismen ---


Von Seiten der Autoren herrscht eine geringe Akzeptanz von Open Access Publikationen und es existieren "viele Vorbehalte und Missverstädnisse" \cite{Suber_2002}. Diese fehlende Akzeptanz für Open Access in der wissenschaftlichen Gemeinschaft stellt noch immer eine der größten Herausforderungen für die Etablierung offener Kommunikation in der Wissenschaft und Forschung dar \cite{weishaupt_2009_goldenOA}. Die Vorurteile betreffen insbesondere das Author-Paymodell (APC) zur Refinanzierung der Publikation, bei dem Autoren für die Veröffentlichung der Texte selbst zahlen müssen, damit die Texte frei zugänglich sind \cite{suchen} und das "obwohl auch bei konventionellen (nicht Open Access) Veröffentlichungen oft genug die Druckkosten selbst aufgebracht werden mussten" \cite{weishaupt_2009_goldenOA}. Eine weitere Hürde für die Akzeptanz stellen Probleme bei der Sicherung der "Authentizität und Integrität der Texte" dar \cite{weishaupt_2009_goldenOA}. Darüber hinaus gibt es Herausfoderungen bei der Lanzeitarchivierung und der Einbettung offener Kommunikatione in das wissenschaftliche Reputationssystem \cite{weishaupt_2009_goldenOA} \cite{Suber_2002} \cite{Adema_2014_open_access}.

--- Todo: Tabelle - Auflistung Open Access Kritik in der Literatur
Gegenstand / Zeitraum / Referenz Zusammenfassung der Definition ---

\subsubsection{Kritik am ökonomischen Modell}

Ein Kritikpunkt an dem Open Access Modell bezieht sich auf das Kostenargument und die ursprüngliche Hoffnung, dass die technologischen Treiber gesteuert und organisiert von der Forschungs Community selbst, anstatt durch Fachverlage, die durchschnittlichen Kosten für einen publizierten Artikel signifikant senken könnten. So stellte sich die Frage, ob "aus der Sicht des individuellen Nutzenkalküls von Wissenschaftlern, Verlagen und weiteren Einrichtungen wie Bibliotheken als auch aus Sicht gesamtwirtschaftlicher Wohlfahrtsüberlegungen (...) der Markt der Wissenschaftskommunikation nicht effizienter organisiert werden könnte."\cite{Hess_2006} In einigen Beiträgen wurden schon sehr früh Kostensenkungen von bis zu 90 Prozent \cite{hilf_2004} \cite{suchen} prognostiziert. Folgende Punkte schürten darüber hinaus die Hoffung, das System leistungsfähiger zu machen und "von seinen durch den Papierdruck auferlegten Fesseln" zu befreien \cite{hilf_2004}:
\begin{itemize}
\item langer Zeitverzug vom Einreichen eines Manuskriptes bis zum finalen Bereitstellung des Wissens,
\item komplizierter Vertriebsweg vom Verlag über Grossisten zu Bibliotheken,
\item hohe Kosten (ca. 3.000,- Euro für die gesamte Verlagsarbeit je Artikel) mit den daraus folgenden horrenden Zeitschriftenpreisen,
\item und daraus folgend wenige, sowie ungleich in der Welt verteilte Leser (digital divide),
\item unvollständige Information (aus Platzmangel), was Nachnutzungen und das Nachprüfen erschwert und somit auch Fälschungen erleichtert,
\item nur anonymes Referieren vor der Veröffentlichung, was den Missbrauch erleichtert.
\end{itemize}

Verlage die Open Access publizieren stehen unter einer besonderen und neuen Herausforderung mit diesem Modell nachhaltig zu operieren und passen deshalb ihre Preise von Zeit zu Zeit an. "Aufällig ist jedoch, dass gerade die großen erfolgreichen Projekte wie BioMed Central und Public Library of Science nach ihrer Einführung am Markt deutlichen Gebrauch von Preissteigerungen gemacht haben"\cite{schmidt_2007_goldenen}. Diese Entwicklung hält, wenn auch verlangsamt, weiter an\cite{suchen}. Unter diesem Kostenaspekt unterscheiden sich subskripitonsbasierte und Open Access-Verlage nicht fundamental \cite{schmidt_2007_goldenen}.

\subsubsection{Gefahr der Einschränkung von Freiheit in Forschung, Lehre und Forschungsdiversität}

Die umfassende Öffnung der wissenschaftlichen Kommunikation hätte weitreichende Implikationen, nicht nur auf die Frage wie geforscht wird, sondern auch was geforscht wird \cite{suchen}. Die Vermischung von Interessen bei der Finanzierung von Forschung in Deutschland soll durch die Unabhängigkeit der Deutschen Froschungsgemeinschaft verhindert werden. Ziel dieser Trennung ist die unabhängige Verteilung der Mittel, völlig frei von politischer Couleur \cite{suchen}. Dennoch kann, so die Befürchtung einiger Autoren \cite{suchen}, nicht sichergestellt werden, dass eine die umfassende Einbeziehung und Information der Gesamtöffentlichkeit nicht doch mittelbar Einfluss auf die Mittelvergabe haben könnte. Ein Großteil der Wissenschaft wird durch Steuergelder finanziert, was nicht ausschließt, dass politische Interessen, die Steuerungsmechanismen von Wissenschaft und Forschungsförderung trotz unabhängiger Forschungsförderungsstrukturen beeinflussen können. Der Wissenschaftstheoretiker Michael Hagner formuliert seine Befürchtung in einem Beitrag für die Frankfurter Allgemeine Zeitung wie folgt: "Open Access als Traum der Verwaltungen". Er und andere sehen die Gefahr, dass die Wissenschaft bei der Verpflichtung zur elektronischen Veröffentlichung von Forschungsergebnissen für Wissenschaftler durch Universitäten auf eine vollends verwaltete Forschung hinaus laufen würde \cite{suchen}. Andere antizipieren eine weitere Gefährdung von Wissenschaft und Forschung, weil Grundlagenforschung, sowie andere komplexe oder explorative Forschungsbereiche in Zukunft weniger Berücksichtigung finden würden, wenn die Öffnung der wissenschaftlichen Forschungsprozesse weiter vorangetrieben wird \cite{suchen}.

Um diese Aspekte und Prognosen über die Implikationen von Open Access zu evaluieren, wird in diesem Teil der Arbeit auf Grundlage von Textbeispielen die Kritik an der Öffnung von Wissenschaft und der (forschungs-)politischen, rechtlichen und freiheitlichen Entwicklungen dargestellt.

Als ein konkretes Beispiel für die Kritik an der Einschränkung der Wissenschafts- und Publikationsfreiheit soll der "Heidelberger Appell" für Publikationsfreiheit und die Wahrung von Urheberrechten dienen. Am 22. März 2009 wurde auf der Webseite der „Frankfurter Allgemeinen Zeitung“ der Artikel "Geistiges Eigentum: Autor darf Freiheit über sein Werk nicht verlieren" \cite{faz_heidelberger_apell_2009} veröffentlicht. Vorangegangen war eine öffentlich ausgetragene Diskussion zwischen dem Literaturwissenschaftler Prof. Dr. Roland Reuß und weiteren Wissenschaftlern in einem Spezial der Onlineausgabe der Frankfurter Allgemeinen Zeitung: "Die Debatte über Open Access". Im Anhang zu diesem Artikel fand sich ein öffentlicher Aufruf, auch "Heidelberger Appell" genannt.

Der Appell richtete sich vor allem an "die Bundesregierung und die Regierungen der Länder, das bestehende Urheberrecht, die Publikationsfreiheit und die Freiheit von Forschung und Lehre entschlossen und mit allen zu Gebote stehenden Mitteln zu verteidigen" \cite{ITK_2009}. Die Autoren forderten, unter anderem in Bezug auf die Google Buchsuche, die Politik, Öffentlichkeit und Kreative auf, sich für die "Wahrung der Urheberrechte" und "gegen eine angebliche „Enteignung“ der Autoren durch das Vorgehen von Google einerseits sowie durch das Publikationsmodell Open Access andererseits" \cite{WD_bundestag_2009} zu engagieren.

Die Autoren des Appells unterscheiden zwei Ebenen: \textit{International} kritisieren sie "die nach deutschem Recht illegale Veröffentlichung urheberrechtlich geschützter Werke geistigen Eigentums auf Plattformen wie GoogleBooks und YouTube", sowie die Entwendung dieser "ohne strafrechtliche Konsequenzen". Im \textit{nationalen Rahmen}, so prangern die Autoren weiter an, werden diese "Eingriffe in die Presse- und Publikationsfreiheit, deren Folgen grundgesetzwidrig wären" durch die "Allianz der deutschen Wissenschaftsorganisationen (Mitglieder: Wissenschaftsrat, Deutsche Forschungsgemeinschaft, Leibniz-Gesellschaft, Max Planck-Institute u.a.)" sogar unterstützt.\cite{ITK_2009}

Die Kritik der Autoren des Heidelberger Apells an Open Access bezieht sich, laut einer Untersuchung des Wissenschafltichen Diensts des Bundestags im wesentlichen auf zwei Aspekte \cite{WD_bundestag_2009}:
\begin{enumerate}
\item Erzwungene Vertriebswege
"Eine Forschung, der man diktieren könnte, wo ihre Ergebnisse publiziert werden sollen, sei nicht mehr frei." Die Verpflichtung auf "bestimmte Publikationsform (...) dient nicht der Verbesserung der wissenschaftlichen Information" \cite{ITK_2009}.
\item Subventionierung von Vertriebswegen oder der Gefährdung von Fachzeitschriftenverlagen
\end{enumerate}

Der Appell "hat eine außergewöhnlich heftige Diskussion über die urheberrechtliche Problematik im Hinblick auf die aktuellen Entwicklungen im Internet ausgelöst. Viele Parlamentarier und Politiker sind für das Thema sensibilisiert" worden \cite{WD_bundestag_2009}. In Bezug auf Open Access widerlegt der Wissenschaftliche Dienst die Befürchtungen der Autoren des Heidelberger Apells. Dem Kritikpunkt der "Erzwungenen Vertriebswege" widerspricht der Wissenschaftliche Dienst mit einem Verweis auf Gudrun Gersmann, weil "auch (Anmerkung: unter Open Access) eine Veröffentlichung bei einem Verlag mit einfachem Nutzungsrecht weiterhin möglich sei". In Bezug auf das Modell und das Abhängigkeitsverhältnis halten die wissenschaftlichen Autoren des Bundestags Reuß entgegen, dass es im bisherigen System "zwischen Autor und Fachzeitschriftverlag oft ein einseitiges Abhängigkeitsverhältnis zu Lasten des Autors gibt" und Wissenschaftler "oftmals alle Rechte an ihren Beiträgen abtreten" \cite{WD_bundestag_2009} müssen. "Der Befürchtung im Heidelberger Appell, das Publikationsmodell Open Access gefährde Fachzeitschriftenverlage wird entgegengehalten, dass die digitale Plattform auf lange Sicht auch ein Ausweg aus der Zeitschriftenkrise sein könnte" \cite{WD_bundestag_2009}. Abschließend konstatiert der Wissenschaftliche Dienst des Bundestags, dass die "Kritik an Open Access kaum nachvollzogen werden" kann und "die hier gemachten Vorwürfe"... "eher auf die traditionellen Vertriebswege zu (teffen) als auf das neue Publikationsmodell" \cite{WD_bundestag_2009}.

Ein Teil der Kritik scheint an mindestens zwei Punkten berechtigt zu sein. Erstens, dass man seitens der Forschungsförderer nicht besonders bemüht war \cite{suchen}, sich "ein genaues Bild von den Nebenwirkungen (Anmerkung: von Open Access)" \cite{Reuss_2009} zu verschaffen und zweitens stellt die Sicherung von Freiheit von Forschung und Lehre sowie die Anpassung der Steuerungsmechanismen eine Herausforderung an die Bestrebungen zur Öffnung von Wissenschaft und Forschung dar \cite{suchen}.

Die Kritik am urheberrechtlichem Aspekt der Google Buchsuche soll in dieser Arbeit nicht berücksichtigt werden, da es sich dabei zwar um einen Aspekt der Digitalisierung von Büchern, nicht aber um die Öffnung von wissenschaftlicher Kommunikation nach den Kriterien der in dieser Arbeit gewählten Deklarationen handelt, sowie die Google Buchsuche als Dienst keinen Bezug zur Open Access-Bewegung aufweist.

\section{Analyse von Open Science}

\subsection{Offener Zugriff auf wissenschaftliche Kommunikation: Definitionen von Open Science}

Hinter dem Begriff Open Science oder Offene Wissenschaft verbirgt sich die Forderung, dass wissenschaftliche Erkenntnisse aller Art im Rahmen des wissenschaftlichen Erkenntnisprozesses schnellstmöglicht offen verbreitet und nutzbar gemacht werden sollen. \cite{https://lists.okfn.org/pipermail/open-science/2011-July/000907.html}. Open Science beschränkt sich dabei nicht nur auf den Zugang zur wissenschaftlichen Publikation am Ende des wissenschaftlichen Erkenntnisprozesses (Open Access) und auf die daraus resultierenden Veränderungen wissenschaftlicher Kommunikationsprozessen im Rahmen von Publikationen, sondern auf sämtliche Daten und Informationen die während des Prozesses anfallen. Aus technischer Sicht ist damit jeder Aspekt der wissenschaftlichen Arbeit gemeint, der digital auf einem Desktop-Computer stattfindet und somit auch öffentlich über das Web potenziell verfügbar gemacht werden kann \cite{mietchen2012wissenschaft}.

Open Science basiert somit auf der Anforderung, dass die Ausübung von wissenschaftlichen Tätigkeiten auf eine Art und Weise erfolgt, die es anderen ermöglicht zu den Forschungsbemühungen beizutragen, zusammenzuarbeiten und auf alle Daten, Ergebnisse und Protokolle in allen Phasen des Forschungsprozesses frei zuzugreifen. \cite{http://www.rin.ac.uk/our-work/data-management-and-curation/open-science-case-studies}. Der gesamte Forschungsprozess sollte demnach so transparent und zugänglich wie möglich gestaltet wird \cite{Scheliga_2014}.

Anhand der folgenden Einteilung werden die Charakteristika des wissenschaftlichen Erkenntnisprozesses erläutert und dargestellt, um zu verdeutlichen, was die Öffnung von Wissenschaft im Sinne von Open Science beinhaltet. Zur Verdeutlichung des Prozesses der Wissensschaffung wird in der vorliegenden Arbeit eine Einteilung extrapoliert in vier Phasen vorgenommen:
\begin{enumerate}
\item Fragestellung & Planung: Basis für den Prozess der Wissenschaffung ist die Definition einer Frage bezüglich einer spezifischen Beobachtung oder eine offene Frage \cite{suchen}. Für die wissenschaftliche Bearbeitung eines Themas ist es entscheidend, dass eine präzise Fragestellung im Zentrum steht \cite{suchen}. --- TODO: weiter beschreiben ---
\item Ausführung: Testen der Hypothese durch den Einsatz von geeigneten wissenschaftlichen Methoden und unter Minimierung der möglichen Fehler.
\item Verarbeitung und Analyse: Analyse der gewonnen Daten und Informationen im Hinblick auf die Verifikation und Falsifikation der Hypothese. --- TODO: weiter beschreiben ---
\item Auswertung ----TODO: Beschreiben-----
\item Kommunikation ----TODO: Beschreiben-----
\end{enumerate}

Die Forderung nach Öffnung des gesamten Prozesses der Wissensschaffung begründet sich dabei nicht (nur) durch Unzulänglichkeiten am bestehenden wissenschaftlichen Kommunikationssystem, sondern basiert auf den folgenden weiterführenden Annahmen:
\begin{enumerate}
\item Der offene Zugang zum gesamten Wissenschaftsprozess erhöht die Möglichkeiten der Validierung und Reproduzierbarkeit der gesamten Forschung(skette) und die Entwicklung neuer Qualitätskriterien. (enhanced Validation/Reputation-Argument)
\item Im Rahmen des Teilens (z.B. von Rohdaten) erhöht sich die Effizienz und Verwendbarkeit durch in der Forschung und Wissenschaft entstandenen Informationen. (Shared-Science-Argument)
\item Im klassischen wissenschaftlichen Kommunikationssystem gibt es keine Anreize negative, widerlegende oder unerfolgreiche wissenschaftliche Ergebnisse zu veröffentlichen. Eine vollumfängliche Öffnung des wissenschaftlichen Erkenntnisprozessess könnte dazu beitragen, dass Wissenschaft ihrem Anspruch an Falsifizierbarkeit gerecht wird. (negative-science/falsifiability-argument)
\end{enumerate}

--- TODO: Weiter Ausarbeiten
Fecher/Friesike 5 Schulen von Open Science http://blogs.lse.ac.uk/impactofsocialsciences/2013/06/20/open-science-new-perspectives-for-scholarly-communication/
Siehe "Open Science"-Teil @ https://docs.google.com/document/d/1qDkQV-M_2VazjWwncRq_udo9tQqrjuZZkdLeKFc3cpI/edit#heading=h.1ahb76xafkbm
Open science can be seen as a mechanism of cumulative knowledge production whereby scientists draw upon knowledge derived at by "prior researchers" and make their discoveries available to "future researchers". \cite{Mukherjee_2009}
Es gibt zahlreiche Open Science Initiativen \cite{Scheliga_2014}. Viele von Ihnen erreichen aber keine kritische Masse \cite{wrap_2010} und enden eher als "virtuelle Geisterstädte" \cite{Nielsen_2011}.

Tabelle - Auflistung Open Science Definitionen in der Literatur
Gegenstand / Zeitraum / Referenz
Zusammenfassung der Definition ---

\subsection{Open Science Modelle}
--- TODO: definieren ----

\subsection{Open Science Formate und Kanäle}

--- TODO: Data Repositorien, (offne) Forschungsanträge, offenes Publizieren (siehe OA), Laborbücher definieren ---

\subsection{Kritik an Open Science}

Während viele Wissenschaftler und Wissenschaftlerinnen Offenheit in der Forschung als wertvoll erachten \cite{suchen}, sind nur wenige tatsächlich bereit, die zusätzliche Zeit und Mühe dafür zu investieren und potenzielle nicht abgrenzbare Risiken einzugehen, Forschung offen und uneingeschränkt zugänglich zu machen \cite{Scheliga_2014} \cite{Procter_2010}. Forscherinnen und Forscher, die offene Wissenschaft praktizieren wollen, werden mit einer Reihe von Hindernissen konfrontiert \cite{Scheliga_2014}:
\begin{enumerate}
\item individuelle Hindernisse: Angst vor Trittbrettfahrern, gefürchteter Mehraufwand an Zeit und Mühe, Herausforderungen bei der Nutzung der digitaler Dienste, fehlender Anstoß, Angst negative Ergebnisse zu veröffentlichen, Herausforderung den Datenschutz sicherzustellen, Abneigung den Code zu teilen
\item systemische Hindernisse: Evaluationskriterien behindern Offenheit, kulturelle und institutionelle Einschränkungen, ineffektive (politische) Richtlinien, Mangel an Standards für das Teilen von Forschungsmaterialien, Mangel an rechtlicher Klarheit, finanzielle Aspekte der Offenheit
\end{enumerate}

Betrachtet man wie Scheliga und Friesike das Phänomen Open Science anhand des Konzepts des sozialen Dilemmas, wird deutlich, dass das was im kollektives Interesse der wissenschaftlichen Gemeinschaft ist, nicht unbedingt im Interesse des einzelnen Wissenschaftlers steht. "Wenn alle Wissenschaftler ihr Wissen nur in den Situationen teilen, in denen sie erwarten, dass sie selbst davon profitieren, ist der gemeinsame Wissenspool fragmentiert und alle Wissenschaftler stehen schlechter dar" \cite{Scheliga_2014}.

--- Todo: Tabelle - Auflistung Open Science Kritik in der Literatur
Gegenstand / Zeitraum / Referenz
Zusammenfassung der Definition ---

\section{Zusammenfassung und Ableitung von Anknüpfungspunkten für die empirische Untersuchung}
Viele der Erklärungsansätze die Forderung nach einem Wandel der wissenschaftlichen Kommunikation hin zur Öffnung der Wissenschaft basieren auf Annahmen, bei denen ein direkter Zusammenhang von technischen Entwicklungen auf (wissenschafts-)politische und kulturelle Bewegungen geschlossen wird. Diese Perspektive beschränkt sich dabei bisher auf den Zugang zum Ergebnis von Wissenschaft und weniger auf die Öffnung des gesamten Prozesses.

Die theoretische Auseinandersetzung mit der Geschlossenheit des wissenschaftlichen Diskurses auf der einen und den Treibern und Bremsern im realen wissenschaftlichen Prozess auf der anderen Seite werden in der Literatur bisher nur ungenügend berücksichtigt. Insbesondere wird die Verbindung zwischen wissenschaftlicher Reputation, die Motivation das etablierte System zu unterstützen und die Geschlossenheit des Wissensproduktionsprozesses nur selten erörtert. Als weiteres Manko kann angeführt werden, "dass die Deliberation und die Verbreitung von Wissen ein stabiles Set von Infrastrukturen braucht"\cite{kelty_2004}, nach denen man heute noch vergeblich sucht. Das Potenzial bei der Verwendung von digitalen Technologien um Wissenschaft offen zu teilen, ist nicht annährend ausgeschöpft und es "besteht eine erhebliche Diskrepanz zwischen der Idee der offenen Wissenschaft und wissenschaftliche Realität" \cite{Scheliga_2014}.

\subsection{Kritik der neoliberalen Rethorik von "Openness"}

Openness kann als "schwimmender Signifikant (...) ohne eindeutige Definition, adaptierbar von unterschiedlichen politischen Ideologien" verstanden werden \cite{Adema_2014_open_access}. Der Begriff Open Access wird in der neoliberalen Rethorik als effizientes Wettberwerbsmodell, verbunden mit den Ideen von Transparenz und Effizienz von Unternehmen und Regierung, eingesetzt \cite{tkacz_2012_open}. Über diesen Ansatz kann mittels Openness der wissenschaftlichen Prozess outputorientierter und seine Ergebnisse effektiver gestaltet, überwacht und gesteuert werden \cite{adema_2010_oaoverview} .

--- Todo: weiter ausarbeiten ---

\subsection{Treiber und Bremser für die Öffnung der wissenschaftlichen Kommunikation}

In den analysierten wissenschaftlichen Beiträgen zu Open Access und Open Science werden meist die positiven Auswirkungen der Forderungen auf das wissenschaftliche Kommunikationssystem dargestellt. Dafür ist eine Erarbeitung der Unzulänglichkeiten am bestehenden wissenschaftlichen Kommunikationssystem notwendig\cite{cite:17}. Folgende grundlegende Treiber für eine Veränderung und Öffnung des wissenschaftlichen Kommunikationssystems werden dabei besonders häufig genannt:

\begin{itemize}
\item Verbreitung und Nutzungsmöglichkeiten der digitalen Infrastrukturen
\item Vorteile des grenzüberschreitenden Austauschs im Rahmen der Globalisierung von Wissenschaft und Forschung
\item --- Todo: weiter ausarbeiten ---
\end{itemize}

Für Gruppierung der Argumente für die Öffnung von Wissen wurde die folgende Kategorisierung vorgenommen und der Beschreibung der Argumenten vorangestellt:
\begin{enumerate}
\item \textbf{Transition-Argument} - Die Nutzung der neuen Möglichkeiten für eine offene Wissensverbreitung neben den konventionellen Wegen der nicht-elektronischen Publikationen \cite{berliner_erklaerung_2003}. Voraussetzung für die Aufbereitung des Wissens als strukturierte Daten zur Wissensweiterverwendung und -verarbeitung über alle Kanäle.
\item \textbf{Speed & Circulation-Argument} - Offene Publikationsverfahren bieten die Chance wissenschaftliche Inhalte schneller und umfassender der wissenschaftlichen Community zur Verfügung zu stellen \cite{muller_2010_open}. Wenn das Wissen schneller zur Verfügung steht, kann es auch schneller zirkulieren und effizienter genutzt werden \cite{Woelfle_2011}. In den tradierten Verfahren wird die Wissensverbreitung künstlich durch Embargos und ineffiziente Validierungs- und Qualitätssicherungssysteme zurückgehalten. Die Digitalsierung und Verbreitung über elektronische Kanäle stellt einen Vorteil für die Wissensverbreitung und -verwertung dar. Eine offene Veröffentlichung erreicht potentiell eine größere Leserschaft als bei Subskriptionsmodellen \cite{cope2014future}.
\item \textbf{Higher Impact & Citation-Argument} - Die uneingeschränkte und globale Verfügbarkeit der offenen wissenschaftlichen Informationen führt zu einem wesentlich höheren Verbreitungsgrad \cite{muller_2010_open}. Der Verbreitungsgrad hängt unmittelbar mit der Zitierhäufigkeit zusammen \cite{muller_2010_open}, somit ist die Zitationsrate wissenschaftlicher Publikationen, die nach den Kriterien von Offenheit veröffentlicht werden potenziell höher \cite{cite:21a}. Diese Kausalität ist durch bedeutsame Untersuchungen bestätigt worden \cite{Lawrence_2001} \cite{Jeffrey_2008} \cite{Eysenbach_2006} \cite{Antelman_2004}.
\item \textbf{Tax-Payer-Agrument} - Die Kosten des traditionellen Publikationsverfahrens werden im Wesentlichen durch die öffentliche Hand getragen \cite{muller_2010_open}. Dem Steuerzahler ist die konventionelle wissenschaftliche Kommunikation jedoch nur selten unentgeldlich zugänglich, obwohl er defacto im Rahmen öffentlich geförderter Forschungsprogramme die Forschung bereits (mit-)finanziert hat \cite{suber_2003_taxpayer} \cite{Beverungen_2012} \cite{Adema_2014_open_access}. Da die Mittel nach intransparenten Kriterien verteilt werden ist im aktuellen Kommunikationssystem unklar, ob wissenschaftliche Kommunikation nach dem bestmöglichen Einsatz der monetären Ressourcen für Wissenschaft und Forschung abläuft \cite{Glasziou_2014} \cite{altman_1994_scandal}. Die Europäische Union und die Organisation für wirtschaftliche Zusammenarbeit und Entwicklung (OECD) kommen in diesem Zusammenhang zu dem Ergebnis, dass der volkswirtschaftliche Nutzen von Open Access die Kosten signifikant übersteigt \cite{WD_bundestag_2009}.
\item \textbf{Economic Promotion Argument} - Bisher profitieren wirtschaftliche Unternehmungen nur unzureichend von staatlich finanzierter wissenschaftlicher Kommunikation. Eine schnellere, kommerziell verwertbare und umfassendere Bereitstellung wissenschaftlicher Inhalte kann einen Beitrag zur non-monetären Wirtschaftsförderung und Innovation leisten \cite{heise_2012} \cite{suchen OECD EU}. Im Rahmen der offenen und schnelleren Verbreitung wissenschaftlicher Informationen sind darüberhinaus auch neue Geschäftmodelle denkbar \cite{suchen}.
\item \textbf{Digital Divide Argument} - Der offene Zugang zu Wissenschaft ermöglicht neue Chancen für die Überwindung sozialer, nationaler und globaler Wissenskluften \cite{suchen} zwischen bildungsfernereren und -affineren Bevölkerungsteilen und -schichten der Welt \cite{boai_2012}. Darüber hinaus ist der Mehrwert und die Chance von wissenschaftlichen Informationen für die schulische Bildung und für die Bewegung der offenen Bildungsmaterialien bisher ebenfalls noch nicht vollumfänglich ausgeschöpft \cite{heise_lernen_2013}.
\item \textbf{Validation & Reputation-Argument} - Offenheit in Wissenschaft und Forschung ermöglicht die Entwicklung neuer Verfahren, die die Aktivität und Qualität eines Forschers umfassender, transparenter und demokratischer messbar und kommunizierbar machen, als das es im bestehenden Reputations- und Förderungssystem möglich ist. \cite{chalmers_2009_avoidable_waste}. Es wird vermutet, dass Wissenschaftsevaluation dadurch effizienter wird, da Wissenschaft "per Definition die Bemühung um integre Information ist" \cite{umstatter_2007_qualitatssicherung}. Die Falsifikation ist nur dann umfassend und einfach möglich, wenn der Aufwand für die Falsifikation gering beziehungsweise der Zugriff auf die wissenschaftliche Informationen überhaupt gegeben ist \cite{umstatter_2007_qualitatssicherung}.
\item \textbf{Paradoxon of Information Argument} - Überwindung des bestehenden Informationsparadoxons bei der Verbreitung und Vermarktung wissenschaftlicher Inhalte. Hierbei handelt es sich um die Herausforderung im Rahmen kommerziell Be- und Verwertung von wissenschaftlichen Informationen ohne zu viel über Inhalt und Qualität auszusagen. Eine im Rahmen von Offenheit angestrebte Entkommerzialisierung des Zugangs zu Wissen würde dieses Informationsparadoxon aufheben.
\item \textbf{Science communication Crisis-Argument} - Durch die Öffnung wissenschaftlicher Kommunikations- und Reputationsprozesse entsteht die Möglichkeit, der vorherrschenden Zeitschriften- und Monographienkrise durch neue Geschäftsmodelle zu begegnen \cite{muller_2010_open}.
\item \textbf{Interdiscipline & International Exchange/Collaboration Argument} - Die Globalisierung führt auch in der Wissenschaft zunehmend zu internationalem Austausch und zur transnationalen Zusammenarbeit von Wissenschaftlern \cite{Waltman_2011}. Das gilt nicht nur für die grenzenüberschreitende Zusammenarbeit in Bezug auf die lokale Verortung, sondern auch für die Interdisziplinarität der Forschungsvorhaben. Die Öffnung der Wissenschaft ermöglicht auch fachfremden Wissenschaftlern Zugriff auf Publikationen und damit auf Wissensressourcen für die eigenen Arbeiten \cite{suchen}.
\item \textbf{Sustainable Access & Archiving Argument} - Nur Offenheit im Sinne von Verwertbarkeit ermöglicht es, in dezentralen Strukturen wie der des Internets alle Informationen nachhaltig und unabhängig voneinander zu speichern. Im Falle von Natur- oder anderen Katastrophen ermöglicht die digitale Ablage auf mehreren Kontinenten eine Präservierung von Wissen unabhängig von lokalen Gegebenheiten oder Bedingungen.
\item \textbf{Dataquality-Argument} - Die Veröffentlichung der Daten hinter den wissenschaftlichen Publikationen kann zu einer insgesamten Erhöhung der Datenqualität wissenschaftlichen Arbeitens führen. Ähnliche Erfahrungen wurden bereits im Bereich der Veröffentlichung von Daten der Verwaltung und bei der Entwicklungszusammenarbeit gemacht \cite{heise_2014_bundestag}.
\end{enumerate}

Bei der Etablierung von Offenheit in Wissenschaft und Forschung wird unter zwei Herangehensweisen unterschieden \cite{schulze_2013_open}:
\begin{enumerate}
\item "Top-down durch Förderstrategien, Vorgaben und Empfehlungen": Hierbei können durch die Bereitstellung zusätzlicher Mittel im Rahmen der Forschungsförderung konkrete Anreize für die offene Veröffentlichung und die Publikation von Forschungsergebnissen geschaffen werden \cite{suchen}. Eine weitere Möglichkeit der "Top-Down"-Etablierung von Offenheit in Wissenschaft und Forschung stellen Empfehlungen dar, bei denen Insitutionen, Organisationen oder Gruppen nicht bindende Empfehlungen aussprechen, anhand derer WissenschaftlerInnen überzeugt werden sollen, ihre wissenschaftlichen Ergebnisse offen zu veröffentlichen \cite{suchen}. Sind weder Anreize, noch Empfehlungen als Top-Down-Ansatz erfolgreich, können bindende Vorgabe etabliert werden um eine Verhaltensänderungen der Wissenschaftler und Wissenschaftlerinnen zu erzwingen \cite{suchen}.
\item "Bottom-up durch Graswurzelprojekte und den Einsatz von Evangelisten":
Im Gegensatz zur Strategie von "oben" gibt es auch Bestrebungen, die von einzelnen WissenschaftlerInnen oder Gruppen initiiert sind. Sie sind überwiegend informell und zielen die Verbreitung von Verhaltensänderungen oder die Etablierung von Richtlinien ab \cite{suchen}. Bottum-up-Projekte kommen aus dem wissenschaftlichen Alltag und erfahren meist keine politische oder monitäre Incentivierung für die Öffnung von Wissenschaft und Forschung. Der Einsatz von Evangelisten basiert auf der Idee einer konkreten Stelle oder Position um eine Änderung zu begleiten \cite{suchen} oder einen Mulitplikator innerhalb und außerhalb von Insitutionen oder Organisationen zu etablieren, der das gewünschte Ziel proaktiv kommuniziert und verbreitet \cite{suchen}. Evangelisten können helfen die Befindlichkeiten und Vorbehalte auszutarieren und die teils diffusen, teils realen Ängste bezüglich der Entwicklung von Offenheit und Transparenz der Wissenschaft innerhalb und ausserhalb der wissenschaftlichen Gemeinschaft zu beseitigen \cite{schulze_2013_open}.
\end{enumerate}

Grundsätzlich steht und fällt der Erfolg bei der Etablierung von Verhaltensänderungen damit, ob sich die jeweiligen Zielgruppe ein unmittelbarer Mehrwert und Nutzen erschließen wird \cite{schulze_2013_open}.

Neben den Gründen und Treibern die in den letzten Dekaden für die Öffnung wissenschaftlicher Kommunikation verantwortlich waren, gibt es Prozesse, die entweder zu einer Verlangsamung der Entwicklung geführt haben, oder sie in einigen Teilbereichen zum Erliegen gebracht haben. Dazu gehören:

\begin{itemize}
\item Fehlende Richtlinien auf regionaler, nationaler und internationaler Ebene
\item Führungslosigkeit der Open Access Bewegung
\item ...
\end{itemize}

Für eine ausgewogene Betrachtung darf nicht nur auf die Vorteile der Öffnung von Wissenschaft und Forschung betrachtet werden, auch wenn es sich "lohnt das Augenmerk  auf "diejenigen Vorteile zu legen, von denen Wissenschaftler selbst profitieren können" \cite{muller_2010_open} müssen auch die Argumente adressiert werden, die gegen die Öffnung der wissenschaftlichen Prozesse und Publikationen sprechen:
\begin{enumerate}
\item \textbf{Quality-Argument} - Das erste Argument umschreibt die Befürchtung, dass die Qualität von offener wissenschaftlicher Kommunikation aufgrund von schlechten oder nicht vorhandenen wissenschaftlichen Überprüfungsmechanismen leidet. Dabei wird argumentiert, dass ein durch ein Autorengebühren finanziertes Publikationsmodell keinen klaren Anreiz für Ablehnung bietet.
\item \textbf{Renomee-Argument} - Die Möglichkeit zur Erlangung von wissenschaftlicher Reputation ist ein grundlegender Motivationsfaktor für Wissenschaftler und Wissenschaftlerinnen die Ergebnisse ihrer Arbeit zu veröffentlichen. Eine Veröffentlichung zahlt nur dann auf die Reputation ein, wenn sie im Rahmen von renomierten Publikationskanälen stattfindet. Offene Publikationsplatformen und Journale können aufgrund des kurzen Zeitraums ihres Bestehens und aufgrund von Vorbehalten dieses Renomee nur selten vorweisen. Die Renomeefrage stellt eine der größten Hürden für die offene wissenschaftliche Kommunikation dar \cite{weishaupt_2009_goldenOA}.
\item \textbf{Archiving- & Sustainability-Argument} - Den grundsätzlichen Vorteilen des elektronischen Publizierens stehen Probleme und Zweifel an der langfristen Verfügbarkeit und Langzeitarchivierung \cite{weishaupt_2009_goldenOA} gegenüber. Einige Autoren kritisieren, dass die Sicherstellung der Langzeitarchivierung und die langfristige Auffindbarkeit, sowie Bereitstellung der Dokumente bisher nicht vollumfänglich durch digitale Strukturen gewährleistet werden kann \cite{umstatter_2007_qualitatssicherung}.
\item \textbf{Authenticity- or Integrity-Argument} - Ein weiteres Problem stellt die Sicherung der Authentizität der offen publizierten wissenschaftlichen Informationen dar \cite{umstatter_2007_qualitatssicherung} \cite{weishaupt_2009_goldenOA}. Weil elektronische Dokumente oft innerhalb weniger Tage oder Wochen in mehreren Versionen zugänglich sind wird befürchtet, dass Texte und Arbeiten, im Zeitablauf inhaltlich nicht mehr unverändert ihrem Autor zuordnenbar sind. Das gilt "solange sie nicht in Digitalen Bibliotheken mit gesicherter Authentizität abgeliefert" werden \cite{umstatter_2007_qualitatssicherung}.
\item \textbf{Rightsmanagement-Argument} - Die Verpflichtung für Mitarbeiter staatlich finanzierter Forschungsinsitutionen, alle Texte und Daten elektronisch frei und offen zu publizieren, wird von einigen Autoren als kritisch identifiziert \cite{suchen}. In dem 2009 veröffentlichten "Heidelberger Appell" \cite{faz_heidelberger_apell_2009} kritisieren zahlreiche Autoren, Wissenschaftler, Verleger und Publizisten, dass das “verfassungsmäßig verbürgte Grundrecht von Urhebern auf freie und selbstbestimmte Publikation” … “derzeit massiven Angriffen ausgesetzt und nachhaltig bedroht” ist. Weiter sehen die Unterzeichner „weitreichende Eingriffe in die Presse- und Publikationsfreiheit, deren Folgen grundgesetzwidrig wären“ \cite{ITK_2009}. Rechtliche Bedenken und die Befürchtung vor kostspieligen juristischen Fehltritten stellen einen weiteren Vorbehalt gegen die offene Veröffentlichung von Forschung- und Forschungsergebnissen dar \cite{weishaupt_2009_goldenOA}.
\item \textbf{(Re-)Financing-Argument} - Die unklare Refinanzierung der Kosten, die im Rahmen der offenen wissenschaftlichen Kommunikation vermutet werden, stellt ein weiteres Kernargumente gegen das offene Publizieren von Arbeiten und Daten dar. Die Befürchtung ist, dass die umfassende Öffnung des wissenschaftlichen Systems überhaupt nicht finanziert werden kann, konnte bisher nicht vollumfänglich ausgeräumt werden \cite{weishaupt_2009_goldenOA}.
\item \textbf{Ressource-Allocation-Argument} - Dieses Argument befasst sich mit der Annahme, dass der Vergabe von Fördermitteln und bei den Reputationsbildenen Maßnahmen durch offene System nicht Rechnung getragen werden kann. Das Argument ruht auf dem Verdacht, dass in Folge der Öffnung eine Mittelvergabe ausschließlich zugunsten populärer Forschung stattfindet und es zu einer Aushöhlung der wissenschaftlichen Fächer- und Facettenvielfalt kommt.
\item \textbf{Open-Caring-Argument} - Wissenschaftlerinnen und Wissenschaftler befürchten durch den Zwang zu umfassenden Bereitstellung ihrer Publikationen und gegebenenfalls soagar der Quelldaten einen nicht unwesentlichen zeitlichen Mehraufwand. Der notige Aufwand den die umfassende Öffnung der wissenschaftlichen im Alltag des Wissenschaftlers mit sich bringen würde, ist bisher kaum evaluiert \cite{osterloh2008anreize}.
\item \textbf{Scientific-Freedom/Loss of Idea-Diversity-Argument}
Dieses Argument betrifft zwei Ebenen: Die Sorge dass durch Offenheit und Transparenz Forschungsförderung und Öffentlichkeit die bestehenden Steuerungsmechanismen der Wissenschaft ausgehebelt werden und infolgedessen nur die wissenschaftlichen Projekte gefördern und unterstützt werden, die vom Souverän verstanden werden. Diese Befürchtung ruht auf der Annahme, dass die Gewinnung von Wissen zum Beispiel in der Grundlagenforschung ein "öffentliches Gut" darstellt, "dessen Wert von der Öffentlichkeit nur schwer beurteilt werden kann"\cite{osterloh2008anreize}. Darüber hinaus wird in der Literatur die Befürchtung geäussert, dass durch die Öffnung die Freiheit von Forschung und Lehre im Sinne der Publikations- und Veröffentlichungsfreiheit gefährdet wird \cite{Jochum_2009}. Infolgedessen wird an vielen Stellen die Befürchtung geäussert, dass im Rahmen von zunehmender Kollaboration über digitale Kanäle, sowie durch die Effizienz der elektronischen Suche die Diversität von wissenschaftlichen Meinungen und Projekten zu einem gleichen oder ähnlichem Thema eingeschränkt werden könnte \cite{Evans_2008}.
\item \textbf{Interpretations-Argument} - Eine weitere Sorge, die den Öffnungsprozess bremmst, ist die Angst der wissenschaftlichen Community vor  Fehlinterpretationen, sowie der Verlust der Kontrolle über die Informationen steuerung \cite{gibbons_1994}. Dabei steht vor allem die Befürchtung im Vordergrund, dass die frei verfügbaren veröffentlichten Arbeiten genutzt werden, um die Arbeit der Wissenschaft zu diskreditieren oder sie gezielt zur Falschinfromation der Öffentlichkeit genutzt werden.
\item \textbf{Transparent-Research-Intentions-Argument} - Die Forderung nach Offenlegung des gesamten Forschungsprozesses beinhaltet auch die Forderung nach "Transparenz der Interaktion zwischen Sponsoren (insbesondere kommerzielle Förderer wie die Pharma- und Medizinprodukteindustrie) und Auftragnehmern" \cite{Stengel_2013}
\end{enumerate}
