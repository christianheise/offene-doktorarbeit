\subsubsection{Chronologie der Bewegung}
Um Open Access zu verstehen und einordnen zu können ist eine historische Betrachtung unabdingbar. Dabei neigt die aktuelle Debatte über die Zukunft des wissenschaftliche Publizierens und Kommunizierens dazu, Konzepte um offene Wissenschaft als einen bisher beispiellosen und noch nie dagewesenen Paradigmenwechsel dar zu stellen. Allerdings wurden schon im antiken Griechenland, und in vielen anderen pre-modernen Zivilisationen, Wissen und Informationen als nicht besitzbare Ware angesehen .

Die Geschichte von Open Access ist aber auch eine Geschichte, die eng mit der Digitalisierung von Kommunikationsprozessen verknüpft. Open Access ist dabei kein Selbstzweck, sondern ein Symptom für tiefergehende Prozesse die mit der wachsenden Bedeutung der Digitalisierung in unserer Zivilisation sowie die damit einhergehenden Möglichkeiten für tiefgreifende Veränderungen im Machtgefüge zusammenhängen. 

Denn auch wenn es vorher schon vereinzelte Versuche gab komplett offen und frei zu kommunizieren war Open Access im Printzeitalter physisch und ökonomisch unmöglich . Trotzdem gehen die ersten Experimente mit offenem Zugang und freien Lizenzen in der Wissenschaft bis in die 60er Jahre des vorherigen Jahrhunderts und somit schon vor der Zeit der Erfindung des Internets, zurück.


Spätestens im Jahr 2001 erschien Open Access als eigenes und öffentlichkeitswirksames Thema im wissenschaftlichen Diskurs.  Die Public Library of Science (PLoS) foderte in einem offenen Brief  dazu auf, die in Zeitschriften erscheinenden Forschungsberichte spätestens sechs Monate nach ihrer Erstveröffentlichung für jedermann offen und unentgeltlich einsehbar im Internet zur Verfügung zu stellen. Nach eigenen Angaben unterzeichneten rund 38.000 Wissenschaftler aus 180 Nationen das Schreiben.

2002 wurden mit der “Budapest Open Access Initative”  erstmals die Bemühungen um Open Access in einer Erklärung zusammengefasst. In dem Zentrum steht die Forderung nach dem freien Zugang zu wissenschaftlichen Publikationen. Sie manifestiert erstmals, dass wissenschaftliche Peer-Review-Fachliteratur “kostenfrei und öffentlich im Internet zugänglich sein sollte, so dass Interessenten die Volltexte lesen, herunterladen, kopieren, verteilen, drucken, in ihnen suchen, auf sie verweisen und sie auch sonst auf jede denkbare legale Weise benutzen können, ohne finanzielle, gesetzliche oder technische Barrieren jenseits von denen, die mit dem Internet-Zugang selbst verbunden sind. In allen Fragen des Wiederabdrucks und der Verteilung und in allen Fragen des Copyrights überhaupt sollte die einzige Einschränkung darin bestehen, den Autoren Kontrolle über ihre Arbeit zu belassen und deren Recht zu sichern, dass ihre Arbeit angemessen anerkannt und zitiert wird.” 

Die Verfassser der Berliner Erklärung gehen darüber hinaus und forderten den kostenlosen und freien Zugang nicht nur zu wissenschaftlichem Wissen in Form von Publikationen sondern auch zu den Daten: „Open Access-Veröffentlichungen umfassen originäre wissenschaftliche Forschungsergebnisse ebenso wie Ursprungsdaten, Metadaten, Quellenmaterial, digitale Darstellungen von Bild- und Graphik-Material und wissenschaftliches Material in multimedialer Form.“  Sie symbolisiert damit auch ein erweitertes Verständnis von Open Access und bildet die Grundlage für Open Science. Dennoch konzentriert sie sich ausschließlich auf den geschlossenen wissenschaftlichen Prozess.
