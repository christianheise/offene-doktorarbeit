s\chapter{Methodik}
Die Verortung der Fragestellung dieser Arbeit, die von den Kulturwissenschaften über die Wirtschaftswissenschaften bis hin zu den Medienwissenschaften reicht, erfordert einen transdisziplinären Zugang zur wissenschaftlichen Bearbeitung. 
Drei wissenschaftliche Methodologien werden in dieser Arbeit angewandt: das Konzeptionelle/Theoretische im Rahmen der Literaturanalyse für die Begriffsbestimmung, das Ethnographische im Rahmen der Befragung zur Identifikation der Treiber und Bremser für die Öffnung von wissenschaftlicher Informationen und Prozesse sowie das Experimentelle. 
Die Herangehensweise folgt dabei der Auffassung des Medientheoretikers Geert Lovink, der diese dreifache Methodik für die Erforschung der digitalen Kultur für zwingend notwendig erachtet . 
Ziel ist es, letztlich zu einem vertieften theoretischen Verständnis der empirischen Ergebnisse zu gelangen. Im Rahmen der Arbeit am Inkubator bietet es sich hier an, weitere Hypothesen anhand von Experimenten zu gewinnen und diese mit neuen Geschäftsmodellen und politischen Prozessen forscherisch zu begleiten. So können mögliche Verallgemeinerungsmodelle im Rahmen der in Kapitel 3 definierten Fragestellungen theoretisch entwickelt und praktisch geprüft werden.
\section{Methode der Inhaltsanalyse}
Die unterschiedliche Verwendung der Begriffe Open Science und Open Access in der wissenschaftlichen Auseinandersetzung machen es notwendig die Begriffsbestimmungen für Open Science und Open Access im Rahmen dieser Arbeit vorzunehmen und zu konkretisieren. In Ergänzug zu der Literaturanalyse von Benedikt Fecher und Sascha Friesike für den Begriff "Open Science"\cite{cite:9} sowie der Litaraturanalyse von Giancarlo Frosio und Estelle Derclaye "Open Access Publishing" \cite{CREATe_2014} soll für diesen Zweck auch eine systematischen Literaturanalyse für die Begriffe "Open Access" und "Open Science" inklusive der Treiber und Bremser der Öffnung von Wissenschaft im Kontext des Begriffs "wissenschaftliche Reputation" durchgeführt werden. Neben der Berücksichtigung von Arbeiten aus den Medienwissenschaften im engeren Sinn sollen auch Arbeiten aus den Wirtschaftswissenschaften und den Kulturwissenschaften berücksichtigt werden.
\subsection{Forschungsfragen} 
Folgende Forschungsfragen sollen bei der Inhalsanalyse genauer analysiert werden:
\begin{itemize}
\item Warum kommt es zu der Bestrebungen hin zur Öffnung von Wissenschaft? 
\item Wie werden Open Science und Open Access definiert und voneinander abegrenzt? 
\item Welche Pro- und Contraargumente gibt es für die Öffnung von Wissenschaft - ist Offenheit in der Wissenschaft gut oder schlecht? 
\item Wo sind die Grenzen der Öffnung? 
\item Warum ist die Öffnung von Wissen in den verschiedenen wissenschaftlichen Disziplinen unterschiedlich weit verbreitet? 
\item Was bedeutet Offenheit und freier Zugang im Rahmen des wissenschaftlichen Diskurs-, Reputations- und Machtbegriffs?
\end{itemize}	

\subsection{Erhebungsmethode und Umfang} 
tbd

\subsection{Analyse der Definitionen von Open Access} 
tbd

\subsection{Analyse der Meinungen und Kritik an Open Access}

In diesem Teil der Arbeit soll im Rahmen der Literaturanalyse eine Auflistung der Kritikpunkte an der Open Access Bewegung in Wissenschaft und Forschung dokumentiert werden. Die Auswahl der berücksichtigten Werke bezieht sich auf dei genannten Fragestellungen und soll als verständlicher Überblick über den vorherrschenden Diskurs im Rahmen von Open Access und Open Science verstanden werden.

\subsubsection{Kritik am ökonomischen Modell}

Ein Kritikpunkt an dem Open Access Modell bezieht sich auf das Kostenargument und die frühe Hoffnung, dass die technologischen Treiber gesteuert und organisiert von der Forschungs Community selbst, anstatt durch Fachverlage, die durchschnittlichen Kosten für einen publizierten Artikel signifikant senke könnten. In einigen Beiträgen wurdne schon früh Kostensenkungen von bis zu 90 Prozent\cite{hilf_2004} prognostiziert. Grundlage dafür war die Fragestellung, dass "aus der Sicht des individuellen Nutzenkalküls von Wissenschaftlern, Verlagen und weiteren Einrichtungen wie Bibliotheken als auch aus Sicht gesamtwirtschaftlicher Wohlfahrtsüberlegunge (...) ob der Markt der Wissenschaftskommunikation nicht effizienter organisiert werden könnte."\cite{Hess_2006} Folgende Punkte schürten darüber hinaus die Hoffung, dass System leistungsfähiger zu machen und "von seinen durch den Papierdruck auferlegten Fesseln" zu befreien \cite{hilf_2004}:
\begin{itemize}
\item langer Zeitverzug vom Einreichen eines Manuskriptes bis zum Gelesen werden,
\item komplizierter Vertriebsweg vom Verlag über Grossisten zu Bibliotheken,
\item horrende Kosten (ca. 3.000,- Euro für die gesamte Verlagsarbeit je Artikel) mit den daraus folgenden horrenden Zeitschriftenpreisen,
\item und daraus folgend wenige Leser, auch noch ungleich in der Welt verteilt (digital divide),
\item unvollständige Information (aus Platzmangel), was Nachnutzungen und das Nachprüfen erschwert und somit auch Fälschungen erleichtert,
\item nur anonymes Referieren vor der Veröffentlichung, was den Missbrauch erleichtert. 
\end{itemize}

\subsubsection{Sicherung von Freiheit von Forschung und Lehre sowie Forschungsdiversität}

Eine Öffnung der wissenschaftlichen Kommunikation hat weitreichende Implikationen, nicht nur auf die Frage wie geforscht wird, sondern auch was geforscht wird \cite{suchen}. Da ein Großteil der Wissenschaft durch die öffentliche Hand finanziert wird, stecken hinter den Steuerungsmechanismen von Wissenschaft und Wissenschaftsförderung immer auch politische Interessen. Zwar soll die Vermischung dieser Interessen in Deutschland durch die Unabhängigkeit der Deutsche Froschungsgemeinschaft verhindert werden und Mittel völlig frei von politischer Couleur verteilt werden \cite{suchen}, dennoch kann, so die Befürchtung einiger Autoren \cite{suchen}, nicht sichergestellt werden, dass eine Einbeziehung der Öffentlichkeit nicht doch einen Einfluss auf die Mittelvergabe hätte. Drastischer ausgedrückt sieht Hagner in dem Beitrag "Open access als Traum der Verwaltungen" dass es im Rahmen des Öffnungsprozesses auch auf eine vollends verwaltete Forschung hinausläuft \cite{suchen}. Grundlagenforschung sowie andere komplexe oder explorative Forschungsbereiche würden in Zukunft weniger Berücksichtigung finden und die Freiheit von Wissenschaft und Forschung endgültig gefährdet \cite{suchen}, so der düstere Ausblick einiger Wissenschaftler \cite{suchen} \cite{suchen}. 

Um diese Aspekte beziehungsweise Prognosen über die Implikationen von Open Access zu evaluieren werden in diesem Teil der Arbeit auf Grundlage von Textbeispielen die Kritik an der Öffnung von Wissenschaft und der (forschungs-)politischen, rechtlichen und freiheitlichen Entwicklungen beleuchtet.

\subsubsection{Beispiel: Der "Heidelberger Apell" für Publikationsfreiheit und die Wahrung der Urheberrechte }

Am 22. März 2009 wurde auf der Webseite der „Frankfurter Allgemeinen Zeitung“ der Artikel "Geistiges Eigentum: Autor darf Freiheit über sein Werk nicht verlieren" \cite{faz_heidelberger_apell_2009} veröffentlicht. Im Anhang zu dem Artikel fand sich ein Aufruf, auch der "Heidelberger Appell" genannt. Vorangegangen war eine öffentlich ausgetragene Diskussion zwischen dem Literaturwissenschaftler Prof. Dr. Roland Reuß sowie weiteren Wissenschaftlern in einem Spezial der Onlineausgabe der Frankfurter Allgemeinen Zeitung: "Die Debatte über Open Access".

Der Appell richtete sich vor allem an "die Bundesregierung und die Regierungen der Länder, das bestehende Urheberrecht, die Publikationsfreiheit und die Freiheit von Forschung und Lehre entschlossen und mit allen zu Gebote stehenden Mitteln zu verteidigen" \cite{ITK_2009}. Die Autoren forderten Politik, Öffentlichkeit und weitere Kreative auf, sich für die "Wahrung der Urheberrechte", unter anderem in Bezug auf die Google Buchsuche "gegen eine angebliche „Enteignung“ der Autoren durch das Vorgehen von Google einerseits und durch das Publikationsmodell Open Access andererseits" \cite{WD_bundestag_2009} zu engagieren. 

Die Kritik am urheberrechtlichem Aspekt der Google Buchsuche soll in dieser Arbeit nicht berücksichtigt werden. Hier soll nur untersucht werden, inwiefern die Kritik am Publikationsmodell Open Access berechtigt ist. Der Apell unterscheidet dabei in zwei Ebenen: \textit{International} kritisieren die Autoren "die nach deutschem Recht illegale Veröffentlichung urheberrechtlich geschützter Werke geistiges Eigentum auf Plattformen wie GoogleBooks und YouTube" und die Entwendung dieser "ohne strafrechtliche Konsequenzen". \textit{National} werden die "Eingriffe in die Presse- und Publikationsfreiheit, deren Folgen grundgesetzwidrig wären" durch die "»Allianz der deutschen Wissenschaftsorganisationen« (Mitglieder: Wissenschaftsrat, Deutsche Forschungsgemeinschaft, Leibniz-Gesellschaft, Max Planck-Institute u. a.)" angeprangert.\cite{ITK_2009}

Die Kritik der Autoren des Heidelberger Apells bezieht laut einer Untersuchung des Wissenschafltichen Diensts des Bundestags insbesondere auf drei Aspekte \cite{WD_bundestag_2009}:
\begin{enumerate}
\item Erzwungene Vertriebswege
"Eine Forschung, der man diktieren könnte, wo ihre Ergebnisse publiziert werden sollen, sei nicht mehr frei." Die Verpflichtung auf "bestimmte Publikationsform (...) dient nicht der Verbesserung der wissenschaftlichen Information" \cite{ITK_2009}.
\item Abhängigkeitsverhältnis
\item Subventionierung von Vertriebswegen
\end{enumerate}

Der Appell "hat eine außergewöhnlich heftige Diskussion über die urheberrechtliche Problematik im Hinblick auf die aktuellen Entwicklungen im Internet ausgelöst. Er hat auch viele Parlamentarier und Politiker für das Thema sensibilisiert"\cite{WD_bundestag_2009}. An vielen Stellen widerlegt der Wissenschaftliche Dienst die Befürchtungen der Autoren des Heidelberger Apells. Beim Kritikpunkt der "Erzwungene Vertriebswege" widerspricht der Wissenschaftliche Dienst mit dem Verweis auf Gudrun Gersmann, weil "auch (Anmerkung: unter Open Access) eine Veröffentlichung bei einem Verlag mit einfachem Nutzungsrecht weiterhin möglich sei". In Bezug auf die im Apell erwähnte Kritik am neuen Abhängigkeitsverhältnis halten die wissenschaftlichen Autoren des Bundestags Reuß entgegen, dass es im bisherigen System "zwischen Autor und Fachzeitschriftverlag oft ein einseitiges Abhängigkeitsverhältnis zu Lasten des Autors gibt" und Wissenschaftler "oftmals alle Rechte an ihren Beiträgen abtreten" \cite{WD_bundestag_2009} müssen. "Der Befürchtung im Heidelberger Appell, das Publikationsmodell Open Access gefährde Fachzeitschriftenverlage", laut Autoren dritter Aspekt der Kritik an Open Access im Apell, "wird entgegengehalten, dass die digitale Plattform auf lange Sicht auch ein Ausweg aus der Zeitschriftenkrise sein könnte" \cite{WD_bundestag_2009}.

Dabei ist die Kritik im Rahmen des Apells mindestens an zwei Punkten berechtigt, so ist es erstens wahr, dass man seitens der Forschungsförderer nicht besonders bemüht war und ist \cite{suchen}, sich "ein genaues Bild von den Nebenwirkungen (Anmerkung: von Open Access)" \cite{Reuss_2009} zu verschaffen und zweitens stellt die Sicherung von Freiheit von Forschung und Lehre sowie die Anpassung der Steuerungsmechanismen eine Herausforderung an die Bestrebungen zur Öffnung von Wissenschaft und Forschung dar \cite{suchen}.

\subsubsection{Analyse der Definitionen Treiber und Bremser für Open Access} 

In den wissenschaftlichen Beiträgen zu Open Access werden viele positive Folgen aufgelistet. Folgende Treiber für eine Veränderung und Öffnung des wissenschaftlichen Kommunikationssystems werden dabei besonders häufig genannt:

\begin{itemize}
\item Verbreitung und Nutzungsmöglichkeiten der digitalen Infrastrukturen
\item Vorteile des grenzüberschreitenden Austauschs im Rahmen der Globalisierung von Wissenschaft und Forschung
\item ...
\end{itemize}

Neben den Aspekten die die Verbreitung von Open Access in den letzten Decaden unterstützt haben, gibt es aber auch einige Kriterien, die entweder zu einer Verlangsamung der Entwicklung geführt haben, oder sie in einigen Teilbereichen ganz zum erliegen gebracht haben. Dazu gehören:

\begin{itemize}
\item Fehlende Richtlinien auf regionaler, nationaler und internationaler Ebene
\item Führungslosigkeit der Open Access Bewegung
\item ...
\end{itemize}

\subsection{Analyse der Definitionen von Open Science} 
tbd

\subsection{Analyse der Definitionen Treiber und Bremser für Open Science} 
tbd

\subsection{Analyse der Meinungen und Kritik an Open Science}

\section{Methode der Onlinebefragung}
Um der Entwicklung der Öffnungvon Wissenschaft sowie deren Treiber und Bremser nachgehen zu können, soll eine Onlinebefragung unter den beteiligten Stakeholdern des akademischen Publizierens an wissenschaftlichen Institutionen explorativ durchgeführt werden. Dies ist nicht zuletzt deshalb für diese Arbeit relevant, weil theoretische Vorannahmen im Rahmen der Definition und Abgrenzung sowie der Literaturanalyse bestehen. Somit sollen die bestehenden Hypothesen getestet, beziehungsweise neue Hypothesen generiert werden. Durch einen Vergleich mit der Studie "Neue Formen des Wissenschaftlichen Publizierens" aus dem Jahr 2007 und 2008 vom Soziologisches Forschungsinstitut Göttingen (SOFI) soll darüber hinaus ein Einblick in die historische Entwicklung der Thematik im deutschsprachigen Raum ermöglicht werden. Die Befragung aus Göttingen bildet ausserdem die Grundlage für die Fragebogenkonstruktion dieser Erhebung. 

Die umfrangreiche Befragung aus den Jahren 2007 und 2008 entstand im Rahmen eines BMBF geförderten Verbundprojekts zwischen SOFI Göttingen und der Universitätsbibliothek Göttingen. Sie basierte auf einer "Vollerhebung der Wissenschaftler an den Instituten und Einrichtungen an fünf deutschen Standorten, die differenziert nach Fächern, Alters- und Statusgruppen (n=6500) erfasst wurden" \cite{Hanekop_2014}. Ziel der Befragung war es, die "Veränderungen beim Zugang zur Literatur wie auch bei den Veröffentlichungsstrategie" \cite{Hanekop_Wittke_2007_Fragebogen} zu untersuchen. In die Teilnehmer der Studie wurden anhand von Webseiten der Forschungseinrichtungen identifiziert und per Email zur Teilnahme aufgefordert. 

\subsection{Aufbau des Fragebogens}
Damls wurden 6500 Wissenschaftler und Wissenschaftlerinnen befragt, von denen 1803 geantwortet haben. Der 2007 verwendete Fragebogen bestand aus 51 Fragen. Im ersten Teil des Fragebogens wurden Fragen zu dem Fachgebiet und Tätigkeitsbereich zunächst als Leserin bzw. Leser wissenschaftlicher Publikationen erfasst. Im zweiten Teil wurden die Teilnehmer aus der Perspektive als Autorin beziehungsweise als Autor befragt. Abschließend wurden noch eineige personenbezogene Angaben abgefragt. \cite{Hanekop_Wittke_2007_Fragebogen} Die Skalen zur Beantwortung der Fragen waren unterschiedlich ausgewählt. 

Zu Beginn der Fragebogenkonstruktion wurde der Fragebogen und das Datenmaterial der Vorbefragung einer Itemanalyse zum Ausschluss unpassender Fragen (Items) unterzogen und Fragen bezglich der Fragestellung dieser Arbeit hinzugefügt. Dafür wurden die veröffentlichten Antworten analysiert. Fragen, die stark ungleich verteilt waren, wurden , wenn sie nicht inhaltlich interessant erschienen, ausgeschlossen.  Somit wurden auf der Basis der Analyse der Fragen der Fragepool auf 31 Fragen reduziert beziehungsweise verändert.

Die zentralen Forschungsfragen dieser Arbeit rückten in diesem Arbeitsschritt der Fragebogenerstellung in den Fokus und stellten die Grundlage für die Entwicklung des Fragepools dar. Die Formulierung der Fragen basierte, sofern nicht aus der Vorbefragung von SOFI unverändert übernommen, auf Handlungsmustern, Meinungen und Einstellungen zu folgenden Fragestellungen:
\begin{itemize}
\item Wie verändert die Digitalisierung, wie wir auf wissenschaftliche Daten und Informationen zugreifen?
\item In welchem Umfang herrscht unter den Wissenschaftlern und Wissenschaftlerinnen Wissen über die Öffnung von Wissenschaft vor? 
\item Welches Verständnis von Open Access besteht unter den Befragten? 
\item Wie stark ist das Interesse an Forschungsdaten? 
\item Welche Faktoren und Argumente begünstigen die Öffnung von Wissenschaft in einer wissenschaftlichen Disziplin? 
\item Welche Faktoren und Argumente sprechen gegen die Öffnung von Wissenschaft in einer wissenschaftlichen Disziplin? 
\item Wie wird der geschätzte Aufwand für die Öffnung von Wissenschaft in einer wissenschaftlichen Disziplin eingeschätzt?
\item Welche weiteren extrinsischen Faktoren unterstützen die Verbreitung von Offenheit in Wissenschaft und Forschung? 
\item Welche unterschiedlichen Auffassung bestehen zwischen den unterschiedlichen Fachdiziplinen, Alters- und Statusgruppen?
\item In welchem Umfang wird bereits heute im wissenschaftlichem Umfeld offen kommuniziert?
\item Welche Veränderungen beim Zugang zur Literatur wie auch bei den Veröffentlichungsstrategie sind im Vergleich zur der 2007 und 2008 durchgeführten Befragung des SOFI Göttingen zu erkennen?
\end{itemize}

Die Gliederung war ebenfalls an die Befragung aus den Jahren 2007 und 2008 angelehnt und beschränkte sich in der ersten Gruppe auf die Rahmenbedingungen der Teilnehmenden sowie deren wissenschaftlichen Tätigkeit. In der zweiten Fragegruppe wurden Aspekten aus der wissenschaftlichen Leserperspektive abgefragt. Die dritte Fragegruppe beschäftigte sich mit Fragen rund um den Zugang zu wissenschaftlichen Informationen, gefolgt von der vierten, die aus Fragen bezüglich des Zugangs zu wissenschaftlichen Informationen und des Zugriffs auf wissenschaftliche Infromationen bestand. In der fünften Fragegruppe wurden Fragen aus der Perspektive des Autors und der Autorin von wissenschaftlichen Inhalten gestellt. Abschließend wurden weitere personenbezogene Daten zur eindeutigen Segmentierung abgefragt. .

\subsubsection{Erhebungsmethode und Messinstrumente}

Auf Grund der zunehmenden Verbreitung und Nutzung des Internets, hat die Online-Befragung zunehmend Eingang in die empirische Sozialforschung gefunden \cite{Pannewitz_2002}. Auschlaggebend für die Auswahl dieser Befragungsform ist vor allem die Ökonomie, "die es einfach macht, große Stichproben in kurzer Zeit zu erheben" \cite{eichhorn_2004_online}. Darüber hinaus wurde in der Vorbefragung durch das SOFI ebenfalls auf das Internet als primäre Quelle für die Identifikation von Teilnehmern und Teilnehmerinnen und E-Mail als Kontaktaufnahmekanal zurückgegriffen.

\subsubsection{Technische Realisation des Fragebogens}

Als System für die Online-Befragung kam die Open-Source-Lösung "Limesurvey" zum Einsatz. Die Fragen wurden in 6 Gruppen geteilt. Auf der Startseite wurde der Fragebogen und das Vorgehen erklärt. Am Ende der Befragung wurde der Befragte auf die Studie des SOFI Göttingen hingewiesen und über Möglichkeiten zur weiteren Verbreitung des Fragebogens informiert.

\subsection{Ablauf der Befragung}
Vor der eigentlichen Befragung fand im Juni 2014 ein Vorversuch statt, in dem der Fragebogen auf seine Verständlichkeit, Logik und Eignung überprüft wurde. Insgesamt nahmen zwölf Wissenschaftler und Wissenschaftlerinnen, darunter auch die Wissenschaftlerin der Vorbefragung Heidemarie Hanekop, an einem Test teil. Die größte Kritik betraf die Fragen der Gruppe, die Tester kritisierten, dass Sie einige der Fragen nicht verstanden haben und sich teilweise zu einseitigen Antworten gezwungen fühlten. Darüber hinaus wurden einige der Matrixfragen vereinfacht und die Pflichtangaben reduziert. Die finale Befragung fand von August bis November 2014 statt.

\subsubsection{Rekrutierung der Teilnehmer}
Die Rekrutierung der Teilnehmer fand, wie bei der Befragung durch das SOFI, durch die Identifizierung der Teilnehmer auf den Webseiten der Forschungseinrichtungen für die unterschiedlichen Fachdiziplinen im deutschsprachigen Raum statt. Darüber hinaus wurde die Befragung in Sozialen Netzwerken (Twitter, Facebook, Google+ und Researchgate) und über Mailinglisten verbreitet. Es wurde darauf geachtet, dass keine Studenten, wenn möglich auch, an der Befragung teilnehmen, da sie noch über keinen ausreichenden Erfahrungsschatz in Bezug auf das wissenschaftliche Publikations- und Kommunikationssystem verfügen. In einer personalisierten E-Mail wurden die Angefragten gebeten an der Online-Befragung Teilzunehmen. Insgesammt wurde so xxxx Wissernschaftler und Wissenschaftlerinnen angefragt. Die Rekrutierung begann am 18.08.2014 und wurde am xx.xx.2014 abgeschlossen. Am Ende der Befragung hatten die Teilnehmer ausserdem die Möglichkeit die Umfrage in ihren Netzwerken zu verbreiten.

\subsubsection{Rücklaufquote}
Insgesamt wurden xxxx Wissenschaftlerinnen und Wissenschaftler per E-Mail angeschrieben. Davon haben xxxx an der Befragung teilgenommen, xxxx den Online-Fragebogen gestartet und xxxx den Online Fragebogen vollstädig beendet. xxx haben den Online-Fragebogen vor Beendigung abgebrochen. Die Rücklaufquote liegt somit bei xx,xx Prozent. 

\section{Das Experiment als wissenschaftliche Methode: Offenes Schreiben dieser Arbeit}
Zur weiteren Erkenntnisgewinnung und für das Ziel der Arbeit Handlungsempfehlungen für das offene Schreiben von Dissertationen erstellen zu können sowie die Kriterien und Argumente für oder gegen das offene Publizieren prüfen zu können, wurde für diese Arbeit selber eine offene Schreibweise gewählt. “Offen” bedeutet in diesem Fall, dass diese Arbeit direkt und unmittelbar bei der Erstellung für jeden, jederzeit frei zugänglich auf einer Webseite im Internet unter einer freien Lizenz (CC-BY-SA) veröffentlicht wurde. Der aktuelle Stand der Arbeit entsprach zu jedem Zeitpunkt dem Stand auf der Webseite. 

Um trotzdem den Anforderungen der Prüfungsordnung in allem Umfang gerecht zu werden, wurde in einem Schreiben an die Promotionskomission am 8. Januar 2013 alle betreffende Punkte in der Promotionsordung der Fakultät Kultutwissenschaften (Stand: 02.02.2011) hervorgehoben und versucht zu begründen, warum diese nicht im Widerspruch zur offenen Schreibweise meiner Arbeit stehen. Um die selbstständige, wissenschaftlicher Arbeit sicherzustellen, hatte kein anderer die Möglichkeit, den erstellten Inhalt zu editieren oder zu kommentieren. Die Transparenz während der Erstellung stellt in diesem Fall kein Widerspruch zu der Selbständigkeit bei der Ausarbeitung dar. Im Gegenteil, sie ermöglichte eine neue Form, die Eigenständigkeit direkt während der wissenschaftlichen Arbeit und Erstellung des Inhalts sicherzustellen. Dem Gesuch die Arbeit "offen" verfassen zu dürfen, wurde seitens der Promotionskommission am 12. Dezember 2013 mehrheitlich entsprochen.

\subsection{konzeptionelle und technische Rahmenbedingungen}

Konzeptionell war das Projekt so angelegt, dass die Arbeit unter allen Umständen jederzeit frei und offen Verfügbar einsehbar sein sollte. Inital sollte ein Blog auf der Grundlage der verbreiteten Open Source-Lösung Wordpress zum Einsatz kommen und nicht nur die Dokumentation rund um die Arbeit, sondern auch als technische Plattform für die gesamte Arbeit selbst zur Verfügung stehen.

\subsection{Herausforderungen während der offenen Anfertigung der Dissertation}

Die Arbeit wurde, bis zur Klärung der Erlaubnis durch die Promotionskommission im Dezember 2013, in einem Google Dokument ohne Freigaben verfasst. Google Docs, ist ein kostenloser, webbasiertes Textverarbeitungssystem der Firma Google. Es ist angelehnt an die gängigen Programme von Microsoft Office oder Open Office, bietet aber einige Einschränkungen besonders für das wissenschaftliche Publizieren.

Nach der schriftlichen Erlaubnis durch die Promotionskommission vom 11. Dezember 2013, wurde in einem ersten Schritt das Google Dokument offen zur Verfügung gestellt beziehungsweise für jeden lesbar freigegeben und aus dem Blog verlinkt. Bei der Übertragung der bereits geschriebenen Inhalte in das Blogsystem, stellte sich schnell heraus, dass die Blog-Software für die Veröffentlichung in einzelnen Blogposts nicht ohne geeignet war. Zwar ermöglichten zusätzliche Plugins die Veröffentlichung von Inhalten in wissenschaftlichen Formaten, aber an folgenden Aspekten scheiterte der Einsatz der Blogsoftware als Publikationplattform:
\begin{itemize}
\item Die Darstellung der einzelenen Kapitel als Blogposts hat sich als sehr aufwendig und im Schreibptozess als unpraktikabel herausgestellt
\item Eine einfache aber standardisierte Referenzierung von von Literaturverweisen ist nicht möglich
\item Fußnoten konnten nicht über mehrere Einträge hinweg zusammenhängend dargestellt werden
\item Der Export in ein lesbares Dokument ist immer mit Aufwand verbunden
\item Die strukturierte Eingabe der Inhalte ist ebenfalls nur unter Aufwand möglich
\item Bei der Anzahl an unterschiedlichen Revisionen wäre das Revisionssystem sicher an seine Grenzen gestoßen
\item ...
\end{itemize}

