\section{Methodik}
Die Verortung der Fragestellung dieser Arbeit, die von den Kulturwissenschaften über die Wirtschaftswissenschaften bis hin zu den Medienwissenschaften reicht, erfordert einen transdisziplinären Zugang zur wissenschaftlichen Bearbeitung. 
Drei wissenschaftliche Methodologien werden in dieser Arbeit angewandt: das Konzeptionelle/Theoretische im Rahmen der Literaturanalyse für die Begriffsbestimmung, das Ethnographische im Rahmen der Befragung zur Identifikation der Treiber und Bremser für die Öffnung von wissenschaftlicher Informationen und Prozesse sowie das Experimentelle. 
Die Herangehensweise folgt dabei der Auffassung des Medientheoretikers Geert Lovink, der diese dreifache Methodik für die Erforschung der digitalen Kultur für zwingend notwendig erachtet . 
Ziel ist es, letztlich zu einem vertieften theoretischen Verständnis der empirischen Ergebnisse zu gelangen. Im Rahmen der Arbeit am Inkubator bietet es sich hier an, weitere Hypothesen anhand von Experimenten zu gewinnen und diese mit neuen Geschäftsmodellen und politischen Prozessen forscherisch zu begleiten. So können mögliche Verallgemeinerungsmodelle im Rahmen der in Kapitel 3 definierten Fragestellungen theoretisch entwickelt und praktisch geprüft werden.