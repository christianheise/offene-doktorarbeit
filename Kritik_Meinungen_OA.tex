\subsubsection{Analyse der Meinungen und Kritik an Open Access}

In diesem Teil der Arbeit soll im Rahmen der Literaturanalyse eine Auflistung der Kritikpunkte an der Open Access Bewegung in Wissenschaft und Forschung dokumentiert werden. Die Auswahl der berücksichtigten Werke bezieht sich auf dei genannten Fragestellungen und soll als verständlicher Überblick über den vorherrschenden Diskurs im Rahmen von Open Access und Open Science verstanden werden.

\textbf{Kritik am ökonomischen Modell}

Ein Kritikpunkt an dem Open Access Modell bezieht sich auf das Kostenargument und die frühe Hoffnung, dass die technologischen Treiber gesteuert und organisiert von der Forschungs Community selbst, anstatt durch Fachverlage, die durchschnittlichen Kosten für einen publizierten Artikel signifikant senke könnten. In einigen Beiträgen wurdne schon früh Kostensenkungen von bis zu 90 Prozent\cite{hilf_2004} prognostiziert. Grundlage dafür war die Fragestellung, dass "aus der Sicht des individuellen Nutzenkalküls von Wissenschaftlern, Verlagen und weiteren Einrichtungen wie Bibliotheken als auch aus Sicht gesamtwirtschaftlicher Wohlfahrtsüberlegunge (...) ob der Markt der Wissenschaftskommunikation nicht effizienter organisiert werden könnte."\cite{Hess_2006} Folgende Punkte schürten darüber hinaus die Hoffung, dass System leistungsfähiger zu machen und "von seinen durch den Papierdruck auferlegten Fesseln" zu befreien \cite{hilf_2004}:
\begin{itemize}
\item langer Zeitverzug vom Einreichen eines Manuskriptes bis zum Gelesen werden,
\item komplizierter Vertriebsweg vom Verlag über Grossisten zu Bibliotheken,
\item horrende Kosten (ca. 3.000,- Euro für die gesamte Verlagsarbeit je Artikel) mit den daraus folgenden horrenden Zeitschriftenpreisen,
\item und daraus folgend wenige Leser, auch noch ungleich in der Welt verteilt (digital divide),
\item unvollständige Information (aus Platzmangel), was Nachnutzungen und das Nachprüfen erschwert und somit auch Fälschungen erleichtert,
\item nur anonymes Referieren vor der Veröffentlichung, was den Missbrauch erleichtert. 
\end{itemize}

\textbf{Der "Heidelberger Apell": Für Publikationsfreiheit und die Wahrung der Urheberrechte }

Am 22. März 2009 wurde auf der Webseite der „Frankfurter Allgemeinen Zeitung“ der Artikel "Geistiges Eigentum: Autor darf Freiheit über sein Werk nicht verlieren" \cite{faz_heidelberger_apell_2009} veröffentlicht. Im Anhang zu dem Artikel fand sich ein Aufruf, auch der "Heidelberger Appell" genannt. Vorangegangen war eine öffentlich ausgetragene Diskussion zwischen dem Literaturwissenschaftler Prof. Dr. Roland Reuß sowie weiteren Wissenschaftlern in einem Spezial der Onlineausgabe der Frankfurter Allgemeinen Zeitung: "Die Debatte über Open Access".

Der Appell richtete sich vor allem an "die Bundesregierung und die Regierungen der Länder, das bestehende Urheberrecht, die Publikationsfreiheit und die Freiheit von Forschung und Lehre entschlossen und mit allen zu Gebote stehenden Mitteln zu verteidigen" \cite{ITK_2009}. Die Autoren forderten Politik, Öffentlichkeit und weitere Kreative auf, sich für die "Wahrung der Urheberrechte", unter anderem in Bezug auf die Google Buchsuche "gegen eine angebliche „Enteignung“ der Autoren durch das Vorgehen von Google einerseits und durch das Publikationsmodell Open Access andererseits" \cite{WD_bundestag_2009} zu engagieren. 

Die Kritik am urheberrechtlichem Aspekt der Google Buchsuche soll in dieser Arbeit nicht berücksichtigt werden. Hier soll nur untersucht werden, inwiefern die Kritik am Publikationsmodell Open Access berechtigt ist. Der Apell unterscheidet dabei in zwei Ebenen: \textit{International} kritisieren die Autoren "die nach deutschem Recht illegale Veröffentlichung urheberrechtlich geschützter Werke geistiges Eigentum auf Plattformen wie GoogleBooks und YouTube" und die Entwendung dieser "ohne strafrechtliche Konsequenzen". \textit{National} werden die "Eingriffe in die Presse- und Publikationsfreiheit, deren Folgen grundgesetzwidrig wären" durch die "»Allianz der deutschen Wissenschaftsorganisationen« (Mitglieder: Wissenschaftsrat, Deutsche Forschungsgemeinschaft, Leibniz-Gesellschaft, Max Planck-Institute u. a.)" angeprangert.\cite{ITK_2009}

Die Kritik der Autoren des Heidelberger Apells bezieht laut einer Untersuchung des Wissenschafltichen Diensts des Bundestags insbesondere auf drei Aspekte \cite{WD_bundestag_2009}:
\begin{enumerate}
\item Erzwungene Vertriebswege
"Eine Forschung, der man diktieren könnte, wo ihre Ergebnisse publiziert werden sollen, sei nicht mehr frei." Die Verpflichtung auf "bestimmte Publikationsform (...) dient nicht der Verbesserung der wissenschaftlichen Information" \cite{ITK_2009}.
\item Abhängigkeitsverhältnis
\item Subventionierung von Vertriebswegen
\end{enumerate}

Der Appell "hat eine außergewöhnlich heftige Diskussion über die urheberrechtliche Problematik im Hinblick auf die aktuellen Entwicklungen im Internet ausgelöst. Er hat auch viele Parlamentarier und Politiker für das Thema sensibilisiert"\cite{WD_bundestag_2009}. An vielen Stellen widerlegt der Wissenschaftliche Dienst die Befürchtungen der Autoren des Heidelberger Apells. Beim Kritikpunkt der "Erzwungene Vertriebswege" widerspricht der Wissenschaftliche Dienst mit dem Verweis auf Gudrun Gersmann, weil "auch (Anmerkung: unter Open Access) eine Veröffentlichung bei einem Verlag mit einfachem Nutzungsrecht weiterhin möglich sei". In Bezug auf die im Apell erwähnte Kritik am neuen Abhängigkeitsverhältnis halten die wissenschaftlichen Autoren des Bundestags Reuß entgegen, dass es im bisherigen System "zwischen Autor und Fachzeitschriftverlag oft ein einseitiges Abhängigkeitsverhältnis zu Lasten des Autors gibt" und Wissenschaftler "oftmals alle Rechte an ihren Beiträgen abtreten" \cite{WD_bundestag_2009} müssen. "Der Befürchtung im Heidelberger Appell, das Publikationsmodell Open Access gefährde Fachzeitschriftenverlage", laut Autoren dritter Aspekt der Kritik an Open Access im Apell, "wird entgegengehalten, dass die digitale Plattform auf lange Sicht auch ein Ausweg aus der Zeitschriftenkrise sein könnte" \cite{WD_bundestag_2009}.

Dabei ist die Kritik im Rahmen des Apells mindestens an zwei Punkten berechtigt, so ist es erstens wahr, dass man seitens der Forschungsförderer nicht besonders bemüht war und ist \cite{suchen}, sich "ein genaues Bild von den Nebenwirkungen (Anmerkung: von Open Access)" \cite{Reuss_2009} zu verschaffen und zweitens stellt die Sicherung von Freiheit von Forschung und Lehre sowie die Anpassung der Steuerungsmechanismen eine Herausforderung an die Bestrebungen zur Öffnung von Wissenschaft und Forschung dar \cite{suchen}.
