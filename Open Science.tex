\subsection{Open Science}
Der Sammelbegriff Open Science geht über die Idee vom offenen Zugang (Open Access) zur Publikationen von wissenschaftlicher Forschung hinaus und beschäftigt sich mit dem offenen Zugriff auf die gesamte wissenschaftliche Wertschöpfungskette inklusive dem ganzen wissenschaftlichen Prozess. Open Science ist als Evolutionsschritt von Open Access zu verstehen. Ähnlich wie bei dem Konzept von Open Access erhoffen sich die Befürworter dabei erstmal grundsätzlich einen einfacheren und breiteren Weg Wissenschaft zu kommunizieren. Grundlegend lässt sich das Konzept von Open Science als die Forderung nach dem offenen Zugriff auf die öffentlich finanzierter Forschung verstehen.
In diesem Kapitel soll Open Science (medien)kulturwissenschaftlich in ihren technischen als auch in ihren gesellschaftlichen und politischen Aspekten sowie die kulturellen Auswirkungen der Medienbrüchen im Rahmen von hybridem Publizieren evaluiert und reflektiert werden.