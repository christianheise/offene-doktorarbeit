\subsection{Open Science}
Der Sammelbegriff Open Science geht über die Idee vom offenen Zugang (Open Access) zur Publikationen von wissenschaftlicher Forschung hinaus und beschäftigt sich mit dem offenen Zugriff auf die gesamte wissenschaftliche Wertschöpfungskette inklusive dem ganzen wissenschaftlichen Prozess. Open Science ist als Evolutionsschritt von Open Access zu verstehen. Ähnlich wie bei dem Konzept von Open Access erhoffen sich die Befürworter dabei erstmal grundsätzlich einen einfacheren und breiteren Weg Wissenschaft zu kommunizieren. Grundlegend lässt sich das Konzept von Open Science als die Forderung nach dem offenen Zugriff auf die öffentlich finanzierter Forschung verstehen.
In diesem Kapitel soll Open Science (medien)kulturwissenschaftlich in ihren technischen als auch in ihren gesellschaftlichen und politischen Aspekten sowie die kulturellen Auswirkungen der Medienbrüchen im Rahmen von hybridem Publizieren evaluiert und reflektiert werden.
\subsubsection{Offener Zugriff auf wissenschaftliche Kommunikation}
In Ergänzung zu 2.1.1. geht es bei Open Science dabei eben nicht nur um den offenen Zugang zu Wissenschaft und den daraus resultierenden Veränderungen von Kommunikationsprozessen im Rahmen von Publikationen, sondern auch um den generellen und offenen Zugriff auf den gesamten Prozess der Wissensschaffung. Dieser Ansatz folgt dabei der Annahme, dass aus technischer Sicht praktisch jeder Aspekt der Wissenschaftskommunikation, der digital auf einem Desktop-Computer stattfindet, auch öffentlich über das Web erfolgen kann . Dieser Wissenschafts-Prozess wird in dieser Arbeit grob in vier Phasen gegliedert:
\begin{enumerate}
\item Planung
\item Ausführung
\item Verarbeitung
\item Auswertung
\end{enumerate}
In diesem Kapitel sollen die Charakteristika des Wissenschafts-Prozesss erläutert werden und dargestellt werden, was eine Öffnung im Sinne des Zugriffs bedeutet. In diesem Zusammenhang soll Open Science nicht nur als Sammelbegriff, sondern auch als  weiteren Evolutionsschritt nach Open Access verstanden werden. Die Forderung nach Open Science begründet sich dabei nicht nur durch die in 2.1.1 genannten Unzulänglichkeiten am wissenschaftlichen Kommunikationssystem sondern basiert auf folgenden weiteren Annahmen:
\begin{enumerate}
\item Der offene Zugang zum gesamten Wissenschaftsprozess erhöht die Möglichkeiten der Validierung und Reproduzierbarkeit der gesamten Forschung(skette) und die Entwicklung neuer Qualitätskriterien. (enhanced Validation/Reputation-Argument)
\item Im Rahmen des Teilens (z.B. von Rohdaten) erhöht sich die Effizienz und Verwendbarkeit von Forschung und im Rahmen von Wissenschaft entstandenen Informationen (Shared-Science-Argument)
\item im klassischen wissenschaftlichen Kommunikationssystem gibt es kaum Anreize negative, widerlegende oder unerfolgreiche wissenschaftliche Ergebnisse zu veröffentlichen, eine grundsätzliche Öffnung könnte dazu beitragen, dass Wissenschaft ihrem Anspruch an Falsifizierbarkeit gerecht wird  z.B. in Pharmalogie (negative-science/falsifiability-argument)
\end{enumerate}
klassisches Geschäftsmodell vs. open Geschäftsmodell
\subsubsection{Wissenschaft als Open-Source-Prozeß}

Open Source ist ein Begriff aus der Softwareentwicklung der als Gegensatz zum “Verfahren der Wissenssicherung”  zugunsten einer quelloffenen Handhabe von Softwarecode verstanden werden will. Der Ende der 90iger Jahre des letzten Jahrhunderts eingeführte Begriff beschreibt, auch wenn es im Detail Unterschiede im Konzept gibt , das gleiche wie der Begriff “freie Software“ . Besonders der Entwicklungsprozess von Open-Source, in Ergänzung zum reinen Zugang und damit mit Open Science konvergent , unterscheidet sich von den klassisch-traditionellen closed-source Prozessen. Dabei folgt Open Source der Maxime, dass die Kernsteuerungsinformationen und -befehle (Quelltext) von Software öffentlich einsehbar und zugänglich und je nach gewähltem Lizenzmodell modifiziert, kopiert oder weitergegeben werden müssen . Es gibt aber auch unterschiede, so betont Steven Weber den Unterschied zwischen Open-Source-Software und dem traditionellen Modell des geistigen Eigentums mit der Feststellung, dass Open-Source-Software macht das Prinzip der Exklusivität des geistigen Eigentums auf den Kopf, weil diese Software 'um das Recht auf Vertrieb konfiguriert, nicht auszuschließen. "
Als Maurer und Scotchmer angemerkt haben, Open-Source-Software-Entwicklung Rechtsmittel ein Defekt der Schutz des geistigen Eigentums, die nicht allgemein zu fördern hat die Offenlegung des Quellcodes. 

Ebenso wie die Open Definition, gibt es festgelegte Kriterien für die Klassifizierung von Open Source Produkten. So heißt es in der Open Source Definition :
\begin{enumerate}
\item Freie Weitergabe
\item Quellcode, das Programm muss den Quellcode beinhalten, bzw. muss dieser offen zur Verfügung gestellt werden
\item Verwendete Lizenz muss Derivate zulassen
\item Unversehrtheit des Quellcodes des Autors muss garantiert werden
\item Auschluss von Diskriminierung von Personen oder Gruppen
\item Keine Enschränkung des Einsatzfeldes
\item Lizenz muss weitergegeben werden könnne
\item Lizenz muss auf das Produktpaket angewandt werden
\item Lizenz darf die Weitergabe zusammen mit anderer Software nicht einschränken
\end{enumerate}

Im Vergleich zum klassischen wissenschaftlichen Entwicklungsprozess gelten dabei folgende charakteristische Merkmale :
\begin{enumerate}
\item “Anzahl der beteiligten Entwickler: Im Vergleich zu traditioneller Softwareentwicklung ist eine weitaus größere Anzahl von Entwicklern beteiligt. Zudem gibt es keine klare Grenze zwischen Entwicklern und Anwendern, da die Hürden für eine Partizipation im Entwicklungsprozess sehr gering sind. Auch wenn ein großer Teil der Entwicklungsarbeit von Freiwilligen übernommen wird, gibt es dennoch den Trend zum Einsatz bezahlter Entwickler.
\item Zuteilung der Arbeit: Im OSP wird die Entwicklungsarbeit nicht länger von einer definierten Instanz zugeteilt, sondern die Teilnehmer wählen ihre Arbeitspakete selbst aus.
\item Architektur: In der Regel orientierten sich die Teilnehmer eines OSP nicht an einer vorgegebenen System-Architektur. Die Gestaltung der Architektur geschieht dezentral und ist oftmals häufigen Richtungswechseln unterworfen.
\item Koordination: Es gibt wenig oder keine institutionalisierten traditionellen Koordinationsmechanismen, wie z.B. Projekt- und Zeitpläne, Lasten- und Pflichtenhefte u.ä.”
\end{enumerate}

Vergleicht man diese mit dem traditionellen Wissenschaftsprozess (siehe 2.2.1.), ergeben sich gewisse Parallelen. Adaptiert man also den Open-Source-Prozess auf Wissenschaft und versteht wissenschaftliche Publikationen als Quellcode, ist das Konzept übertragbar. Der deutsche Literaturwissenschaftler und Medientheoretiker Friedrich Kittler sieht den Gedanken hinter Open-Source fest verankert und äussert in seinem Beitrag “Wissenschaft als Open-Source-Prozeß” die Sorge, “daß mit der Freiheit von Quellcode auch die Freiheit der Wissenschaft steht und fällt” . Wie Wissenschaft als Open-Source-Prozess verstanden werden kann soll in diesem Kapitel genauer erläutert werden.  
\subsubsection{Entwicklung der Bewegung}
Wissenschaft und Offenheit sind seit jeher zwei stark verbundene Elemente. “Open Science” ist dabei ein Begriff der historisch sehr eng mit der Entwicklung von kollaborativen Arbeiten durch neue Kommunikationstechniken verbunden ist. Open Science ist im Rahmen der Open Movements als Evolution zur reinen Öffnung des Zugangs (Open Access) zu wissenschaftlichen Publikationen zu verstehen. Eine klare Definition für den Sammelbergiff steht jedoch noch aus. Dabei spielt insbesondere, die Entwicklung der Tradition für eine "offenen Wissenschaft" im siebzehnten Jahrhundert einen Ansatzpunkt, da dieser historische Übergang noch nicht erforscht ist.  Dieser Strang der Forschung ist eine sinnvolle Analyse, um grundlegende Argumente für Open Science in wissenschaftliche Forschung zu untersuchen. 

In diesem Kapitel werden die historischen Aspekte der Veröffentlichung und Verbreitung von wissenschaftlichen Informationen chronologisch unter der Berücksichtigung der Frage, wie Wissen der Allgemeinheit zur Verfügung gestellt wird und wurde, erfasst und erläutert. 

Die Verschlüsselungs- und Patentwut zur Wahrung eines möglichen kommerziellen Vorteils durch Wissenschaft im Rahmen öffentlich-finanzierter Forschung, geht dabei bis auf die xxxx Jahre zurück. ### Beispiel Galileo, Kepler, Newton ### Das Ergebnis dieser Wut war eine Debatte über die Verfügbarkeit der wissenschaftlichen Arbeit und die Entlohnung der “Erfinder“ im wissenschaftlichen System. 
\subsubsection{Open Science Modelle}
\subsubsection{Open Science Formate}
Data Repositorien, (offne) Forschungsanträge, offenes Publizieren (siehe OA), Laborbücher