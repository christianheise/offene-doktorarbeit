\subsubsection{Offener Zugang zu Wissenschaft}
Wie in Kapitel 1. beschrieben steht der bisherige Prozess der wissenschaftlichen Kommunikation vor großen Herausforderungen. Die Zeitschriften- und Monographienkrise sowie zunehmender finanzieller Druck, die steigenden Beschaffungskosten für wissenschaftliche Literatur  und die Veränderungen durch die digitale Revolution zwingen zum Umdenken. Deshalb soll in diesem Teil der Arbeit das Modell des Offenen Zugangs zu Wissenschaft erläutert werden.
Offenheit stellt dabei eine große Chance für dringend notwendige Veränderungen im wissenschaftlichen Qualitäts- und Reputationssystem (siehe Kapitel 2.3) dar. Bisher wird die Aktivität und Qualität eines Forschers mit intransparenten Mitteln durch private Unternehmungen durchgeführt, die im Zweifelsfall sogar Geld damit verdienen. Die Erkenntnisse der Forschung werden dabei häufig erst nach langen Verfahren publiziert und gelangen dann auf Grund von closed-Access-Modellen nicht an die breite Öffentlichkeit.