\chapter{Grundlagen, Definitionen und Abgrenzungen}

Der theoretische Bezugsrahmen wissenschaftlich gesicherter Modelle, Theorien und Ansätze ermöglicht es, Erklärungen und Handlungsempfehlungen abzuleiten \cite{martin_2007_wissenschaftstheorie}. Er trägt dazu bei, die Fragestellungen in einen Zusammenhang zu stellen, legitimiert die Erforschung dieser Fragen und bildet den Rahmen für die Auswertung gesammelter Erkenntnisse \cite{suchen}. Ziel dieses Kapitels ist es, die theoretischen Grundlagen für die im nächsten Abschnitt folgende Inhaltsanalyse und die empirischen und experimentellen Ergebnisse zu erarbeiten, sowie die Einführung der Begriffe, Definitionen und Konzepte, die für das Thema der vorliegenden Arbeit grundlegend sind.

Wenn es im aktuellen öffentlichen Diskurs um wissenschaftliche Informationen, Infrastruktur und Arbeiten geht, werden immer öfter Schlagworte mit dem Attribut „Open“, wie Open Access, Open Research und Open Science, verwendet \cite{bunz_2014} \cite{schulze_2013_open}. "Offen" bezieht sich dabei üblicherweise auf zwei Kernaspekte: Zum einen die Offenheit des Zugangs zu Daten, Quellcode oder Ergebnissen und zum anderen auf das Gebot der Transparenz, also die Offenlegung, beziehungsweise der Zugriff auf Verfahren, Methoden und Ziele \cite{schulze_2013_open}. Während "Openness" vielfach mit den Entwicklungen rund um offene Software assoziiert wird, gibt es andererseits Anknüpfungspunkte von "Offenheit" als Begriff in der wissenschaftlichen Auseinandersetzung die schon früher anzusetzen sind \cite{Tkacz_2014}. So sieht Christopher Kel­ty die ersten Anfänge bereits in den 1980ern \cite{kelty_2008_two_bits}. Andrew Russell sieht die ideologischen Ursprünge von "Offenheit" als Standard schon in der Entwicklung des Telegraphs und weiteren Ingenieurleistungen seit 1860 \cite{Russell_2014}.

In der gegenwärtigen Literatur sind die Begriffe „Open Access“ und „Open Science“ nicht eindeutig voneinander abgegrenzt und sie finden in der wissenschaftlichen Auseinandersetzung auf unterschiedlichste Art und Weise Verwendung \cite{cite:9}. Infolgedessen werden Open Science und Open Access in dieser Arbeit im Kontext wissenschaftlicher Reputation in Bezug auf ihre technischen, gesellschaftlichen und politischen Aspekte beschrieben und die Betrachtung wird auf die kulturellen Auswirkungen der Medienbrüche im Rahmen wissenschaftlichen Publizierens erweitert. Der historische und gesellschaftliche Kontext ihrer Anwendung wird dargestellt und mittels der Analyse wissenschaftlicher Literatur abgegrenzt. Es wird erläutert, welche Bedeutung sie in der Forschung, der Gesellschaft und der Politik haben. Die Entstehung und Entwicklung der Begriffe wird beschrieben. Um ein möglichst umfassendes Bild zu erhalten, wird "Entwicklung" hier in den drei folgenden Dimensionen erfasst: erstens, als "analytische Kategorie", zweitens als "Forschungsgegenstand" und drittens als "politische Praxis in der moralischen Auseinandersetzung über die Wünschbarkeit von Zuständen" \cite{cite:10}. Die Analysen in dieser Arbeit werden aus der Perspektive des Produzenten (Wissenschaftler als Autoren) sowie aus der, damit nicht immer harmonisierenden, Perspektive des Rezipienten, beziehungsweise Medienkonsumenten (Wissenschaftler als Leser) stattfinden. Es wird auch adressiert, inwiefern Macht, regulierende Prinzipien wie die Verknappung, sowie die Ein- und Ausgrenzung im Rahmen wissenschaftlicher Diskurse (nach dem Diskurs- und Machtbegriff) mit den Modellen Open Access, Open Science und wissenschaftlicher Reputation in der Kommunikation vereinbar sind oder diesen gegenüberstehen.

Die Unterscheidung von "Zugang" und "Zugriff" ist in dieser Arbeit wesentlich und stellt eine zentrale Grundlage für die Definition und Abgrenzung der Begriffe "Open Access" und "Open Science" dar. "Zugang" bezieht sich in diesem Zusammenhang auf einen unbeschränkten Zugang zur finalen wissenschaftlichen Publikation. "Unbeschränkt" meint hier vor allem das ausschließliche Lesen der finalen Ergebnispublikation \cite{cite:9a}, Verarbeitung und Weiternutzung. "Der Open-Access-Ansatz bezieht sich zunächst lediglich auf die Zugangsbedingungen zu den wissenschaftlichen Arbeiten" \cite{muller_2010_open}. "Zugriff" soll als erweiterte Nutzung der jeweiligen Wissensressourcen verstanden werden und schließt neben dem "Zugang" zur Publikation sämtliche Informationen und Daten, sowie die komplette Kommunikation hinter der finalen Veröffentlichung \cite{cite:9b} ein. "Zugriff" beschränkt sich hier also nicht nur auf den reinen Zugang zu wissenschaftlicher Information im Rahmen des Publikationsprozesses, sondern schließt auch den Zugriff auf sämtliche Forschungsdaten, Methoden und alle weiteren Informationen, die während der wissenschaftliche Arbeit auf dem Weg zur finalen Publikation entstehen \cite{cite:9c}, ein.

Die Themenbereiche kollaboratives Arbeiten, Social Media in Wissenschaft und Forschung, Citizen Science und aktuelle Diskurse zu Tools und Diensten werden in dieser Arbeit bewusst ausgelassen und nur am Rande, beziehungsweise nur wenn sie die Beantwortung der Forschungsfragen tangieren, eingeschlossen.

\section{Wissenschaftliche Kommunikation}
Bevor die Grundlagen für Offenheit in Wissenschaft und Forschung definiert werden, wird eine grundlegende Einordnung und Abgrenzung von wissenschaftlicher Kommunikation vorgenommen sowie deren Wandel im Rahmen der Digitalisierung beschrieben.

Kommunikation ist einer wesentlicher Bestandteil der Wissenschaft \cite{garvey_2014_communication}. Sie ist eng mit der Wissensproduktion verknüpft und basiert auf dem Austausch zwischen Wissenschaftlern, die auf einem "gemeinsamen Wissensbestand" zugreifen, "den sie testen, verändern und erweitern" \cite{Gl_ser_2007}. Sinn und Zweck der Kommunikation besteht auf dem bestmöglichen Austausch zwischen den Mitgliedern der Wissenschaftsgemeinschaft. Jede kommunizierte Erkenntnis trägt theoretisch zur Produktion von Wissen bei \cite{kaden_2009_library}. Grundvoraussetzung dafür ist, dass Wissenschaftler und Wissenschaftlerinnen den Willen zu optimalen Kommunikation untereinander haben.

Es existieren verschiedene Arten wissenschaftlicher Kommunikation und "vielfältige Erscheinungsformen" \cite{graefen2007_wissenschaftliche_artikel}: die \textit{formelle} und die \textit{informelle wissenschaftliche Kommunikation} sowie Unterschiede zwischen den Disziplinen. Was genau als formell oder informell gilt, hängt demnach von der jeweiligen Fachdisziplin ab und "ist historisch gewachsen und damit durchaus unterschiedlich, ebenso unterschiedlich wie die Bedeutung der verschiedenen Arten von Publikationen (Journale, internationale Journale, Monografien, Handbücher und Proceedings, etc.)" \cite{Hanekop_2014}.

Eine wesentliche Plattform für wissenschaftlichen Fortschritt, Forschungsförderung und traditionelles Publizieren bilden Journale und Monographien \cite{cope2014future}. Das wissenschaftliche Journal ist (in den meisten wissenschaftlichen Disziplinen) ein wichtiger formeller Kanal der wissenschaftlichen Kommunikation und essenziell für Wissenschaftler und Wissenschaftlerinnen um auf dem Laufenden zu bleiben \cite{cope2014future}.

Die formelle Kommunikation wird an bestimmte Bedingungen der wissenschaftlichen Gemeinschaft geknüpft und hat einen direkten Einfluss auf die Reputation der einzelnen Mitglieder der wissenschaftlichen Community. Diese Art der Kommunikation beinhaltet die Einbeziehung Dritter, die die Funktion der Einordnung und Bewertung der Kommunikation übernehmen. Die Praxis dieser Kommunikation ist die Publikation. Ziel dieser Art der Kommunikation ist die Sicherung des Verbleibs und Positionierung des einzelnen Wissenschaftlers innerhalb der wissenschaftlichen Gemeinschaft. Diese Formalisierung der Kommunikation ist wichtig um das Wissenschaftssystem strukturell zu sichern \cite{kaden_2009_library}. Sie macht Erkenntnisprozesse nachweisbar \cite{kaden_2009_library}.

Formelle wissenschaftliche Kommunikation beruht nach dem Bibiliothekswissenschaftler Ben Kaden auf drei Faktoren \cite{kaden_2009_library}:
\begin{enumerate}
\item \textit{Publizität} meint die Veröffentlichung der Erkenntnisse in einem wissenschaftlichen Fachmedium. Eine Erkenntnis wird durch die Veröffentlichung bekannt gegeben und so für die Community "registriert".  Sie muss dabei "zeitnah" in einer "wahrnehmbaren" Form vorliegen \cite{Schimank_2012}, damit sie intersubjektiv vermittelbar ist.
\item \textit{Vertrauenswürdigkeit} meint das Vertrauen auf die Einhaltung der Regeln im Kommunikationssystem durch alle Teilnehmer. Das Vertrauen wird bei einer Publikation durch die Überprüfung (Peer-Review) bestätigt und durch Bezugnahme (Zitationen) anderer Wissenschaftler auf die Publikation zu Reputation.
\item \textit{Zugänglichkeit} bezieht sich auf die dauerhafte Sicherung und Zugänglichkeit in einer allgemein verfügbaren Form für die Fachöffentlichkeit.
\end{enumerate}

Die Möglichkeiten der informelle Wissenschaftskommunikation sind demgegenüber höchst vielfältig und reichen "vom persönlichen Gespräch über Vorträge, Konferenzen, Zwischen- oder Abschlussberichte aus Projekten, Working Papers und vieles andere mehr"\cite{Hanekop_2014}. Informelle Kommunikation umfasst alle Arten der Kommunikation, die dem individuellem Wissenschaftler einen schnellen und direkten Austausch mit Kollegen ermöglichen und die keinen direkten Einfluss auf die wissenschaftliche Reputation des einzelnen Wissenschaftlers haben \cite{suchen}. Diese Art der Kommunikation stand im klassischen wissenschaftlichen Wertschöpfungsprozess meist am Anfang. Sie umfasst zum Beispiel die Ideenfindung, die Entwicklung von Fragestellungen oder Konkretisierung des Forschungsvorhabens und hilft Wissenschaftlern dabei relevante Ideen für formelle Kommunikation "herauszukristallisieren" \cite{Hanekop_2014}. Informelle Kommunikation ist auf Grund ihrer Heterogenität und impliziten Verankerung weniger präzise differenzierbar und erfassbar \cite{kaden_2009_library}. Die Abgrenzung informeller Kommunikation zu "nicht-wissenschaftlicher Kommunikation" resultiert daraus, dass sie auf "die Erzeugung formeller Kommunikation hinarbeitet" \cite{kaden_2009_library}.

\subsection{Digitalisierung der wissenschaftlichen Kommunikation}

Wie bereits in der Einleitung dieser Arbeit dargestellt wurde, üben die Digitalisierung und die dahinterstehenden Technologien einen tiefgreifenden Einfluss auf die wissenschaftlichen Kommunikation und die wissenschaftliche Arbeit in allen Fachdisziplinen aus.

Dieser Einfluss ergibt sich aus einer der wichtigsten Unterschiede der digitalen Kommunikation im Vergleich zu analogen Kommunikation. Digital kommunizierten Inhalte sind weder endgültig noch endlich und weder im Kern noch in Form fixiert, denn sie können leicht geändert werden und das ohne Spur von Löschung oder Korrektur \cite{http://files.eric.ed.gov/fulltext/ED427787.pdf}. Aus digitalen Informationen können eine endlose Anzahl von identischen Kopien erstellt werden, ohne dass ein Zerfallsprozess eintritt \cite{http://files.eric.ed.gov/fulltext/ED427787.pdf}. Ergänzt durch die Möglichkeit diese Informationen in einem weltumspannenden Netzwerk in nahezu Echtzeit unabhängig von Lokation und Zeit zu transportieren, haben diese fundamentalen Veränderungen am Medium für die Informationsspeicherung, -kommunikation und -verbreitung einen direkten Einfluss auf die wissenschaftliche Kommunikation, die bis zu diesem Zeitpunkt auf dem Austausch analoger Medien und Kommunikation basierte. Mittlerweile sind über 90 Prozent der englischsprachigen Journale online verfügbar und es gibt einen ansteigenden Trend zu Journalen die nur im Internet digital verfügbar sind \cite{cope2014future}.

Mit diesem digitalen Wandel in der wissenschaftlichen Kommunikation wird auch die Chance für eine umfassende “Beschleunigung des Wissensumschlages” verbunden \cite{Wenzel_2003}, sowie mit der Hoffnung verknüpft, dass offene Innovation und offene wissenschaftliche Kommunikation den privaten und staatlichen Forschungsbereich offener, integrativer und effizienter machen \cite{suchen}.

Konkret erfolgte bisher mit der Etablierung der digitalen Kommunikation eine Veränderung der Kategorisierung wissenschaftlicher Kommunikation. Während im Druckzeitalter die formelle wissenschaftliche Kommunikation eng an die bibliometrischen Indikatoren geknüpft war und eindeutig von der informellen abgegrenzt werden konnte, scheint diese klare Grenze im Rahmen der Digitalisierung zu verschwimmen. Vor allem der Bereich der informellen Kommunikation verändert sich durch das Internet, was "jedoch nur vermittelt und mit zeitlicher Verzögerung Wirkungen auf das formelle Publikationssystem zeigt" \cite{Hanekop_2014}. Die Sozialwirtin Heidemarie Hanekop definiert diesbezüglich den folgenden Zusammenhang: "Je größer die Abkopplung zwischen informellen und formellen Aspekten der wissenschaftlichen Kommunikation in einem disziplinären, thematischen oder nationalen Wissenschaftsbereich, um so geringer, vermittelter oder langwieriger kann auch die Wirkung des Internets auf diesen Teilbereich des Publikationssystems sein" \cite{Hanekop_2014}. Ben Kaden hingegen definiert die Veränderungen im Kommunikationssystem als \textit{kanalerweiterte Wissenschaftskommunikation} und erklärt diese als "Form der Wissenschaftskommunikation, die die informelle und formelle ergänzt" und die "individuell affirmativ" als "als eine Art informelles offenes Post Review" verstanden werde kann \cite{kaden_2009_library}.

In dieser Arbeit beschränkt sich der Begriff wissenschaftliche Kommunikation auf die Kommunikation, die formelle Bezugspunkte aufweist und vor allem einen Einfluss auf die wissenschaftliche Reputation des Wissenschaftlers oder der Wissenschaftlerin hat.

\subsection{Wissenschaftliche Kommunikation als Open-Source-Prozeß}
Im Rahmen der Forderung nach der Öffnung der wissenschaftlichen Kommunikation und wissenschaftlichen Publikationen werden in der Literatur häufig Vergleiche zur Open Source-Bewegung gezogen \cite{suchen}. Diese Vergleiche können beispielhaft dem Verständnis theoretischer Grundlagen im Rahmen der Öffnung von Wissenschaft und Forschung dienen.

"Open Source" ist ein Begriff aus der Softwareentwicklung der als Gegensatz zum “Verfahren der Wissenssicherung” \cite{stallman2002} eine quelloffenen Handhabe von Programmcode beschreibt und der Ende der 1990iger erstmals eingeführt wurde  \cite{suchen}. Dieser Begriff wird, auch wenn es im Detail Unterschiede im Konzept gibt \cite{suchen}, häufig synonym mit “freier Software“ (nicht Freeware) verwendet \cite{suchen}. Dabei folgt die Open Source-Entwicklung der Maxime, dass die Kernsteuerungsinformationen und -befehle (Quelltext) von Software öffentlich einsehbar und zugänglich sind, sowie je nach gewähltem Lizenzmodell modifiziert, kopiert oder weitergegeben werden können \cite{suchen}.

Bei der Open Source-Entwicklung veröffentlichen Programmierer den Code einer Software offen im Internet. Andere Programmierer haben die Möglichkeit diesen Code so weiterzuentwickeln und anzupassen, wie es ihnen beliebt. Dadurch entsteht ein offenes Ökosystem an Software, bei dem nicht mehr der Zugriff die Hürde darstellt sondern die Adaption oder der Einsatz der vorhandenen Lösungen.

Die Entwicklungsmethode unterscheidet zwischen Open Source-Software und dem traditionellen Modell des geistigen Eigentums bei der Entwicklung von Software mit der Feststellung, dass Open Source-Software das Prinzip der Exklusivität des geistigen Eigentums auf den Kopf stellt, weil diese Software "um das Recht auf Vertrieb konfiguriert, nicht auszuschließen ist" \cite{suchen}. Auch wenn noch immer nicht vollständig geklärt ist, ob Open Source Software wirklich "schneller, besser oder günstiger" ist, hat sich Open Source in den letzten Jahren stark verbreitet hat \cite{Lerner_2001} und an Bedeutung gewonnen.

Die Definition von Open Source beinhaltet festgelegte Kriterien für die Klassifizierung von Open Source Produkten \cite{suchen}: Freie Weitergabe ohne zusätzliche Kosten, das Programm muss den Quellcode beinhalten und den Code offen zur Verfügung stellen, die verwendete Lizenz muss Derivate zulassen, die Unversehrtheit des Quellcodes des Autors muss garantiert werden, die Diskriminierung von Personen oder Gruppen muss ausgeschlossen sein, es darf keine Einschränkung des Einsatzfeldes geben, die Lizenz muss weitergegeben werden können und auf das Produktpaket anwendbar sein und die Lizenz darf die Weitergabe des Programmcodes zusammen mit anderer Software nicht einschränken.

Im Vergleich zum klassischen Softwareentwicklungsprozess gelten folgende charakteristische Merkmale \cite{suchen}:
\begin{enumerate}
\item “Anzahl der beteiligten Entwickler: Im Vergleich zu traditioneller Softwareentwicklung ist eine weitaus größere Anzahl von Entwicklern beteiligt. Es gibt es keine klare Grenze zwischen Entwicklern und Anwendern, da die Hürden für eine Partizipation im Entwicklungsprozess sehr gering sind. Auch wenn ein großer Teil der Entwicklungsarbeit von Freiwilligen übernommen wird, gibt es dennoch den Trend zum Einsatz bezahlter Entwickler.
\item Zuteilung der Arbeit: Im Open Source Programming (OSP) wird die Entwicklungsarbeit nicht länger von einer definierten Instanz zugeteilt, sondern die Teilnehmer wählen ihre Arbeitspakete selbst aus.
\item Architektur: In der Regel orientierten sich die Teilnehmer eines OSP nicht an einer vorgegebenen System-Architektur. Die Gestaltung der Architektur geschieht dezentral und ist oftmals häufigen Richtungswechseln unterworfen.
\item Koordination: Es gibt wenig oder keine institutionalisierten traditionellen Koordinationsmechanismen, wie z.B. Projekt- und Zeitpläne, Lasten- und Pflichtenhefte u.ä.” \cite{suchen}
\end{enumerate}

---- TODO: konkretisieren und neu ausarbeiten ----

Die Verknüpfung der Open-Source Entwicklungsmethode mit dem Wissenschaftsprozess wurde zuerst von dem Literaturwissenschaftler und Medientheoretiker Friedrich Kittler manifestiert \cite{suchen}. Open Source Entwicklungsprozesse unterscheiden sich von den klassisch-traditionellen (closed-source) Softwareentwicklungsprozessen insbesondere durch ihre transparente Präsenz und permanente öffentliche Einsehbarkeit. Open Source weist diesbezüglich mit der Forderung nach der umfassenden Öffnung wissenschaftlicher Kommunikation Konvergenzen auf, da es in beiden Fällen nicht nur um den freien und offenen Zugang zum finalen Ergebnis geht, sondern um die Möglichkeit des Zugriffs im gesamten Verlauf des Erstellungsprozesses \cite{kelty_2004_prüfen!}. Adaptiert man den Open Source-Prozesse an wissenschaftliche Wertschöpfungsprozesse und definiert in diesem Zusammenhang wissenschaftliche Publikationen als Quellcode, ist das Konzept auf die wissenschaftliche Kommunikation übertragbar \cite{Singh_2008} \cite{Bradley_2008} \cite{Bradley_2007} \cite{Willinsky_2005}.

Die Annahme, dass das System der offenen Softwareentwicklung dem System der Wissensproduktion in der Wissenschaft ähnelt, beruht unter anderem auf der Parallele, dass bei der Wissensproduktion neues Wissen auf der Grundlage von bereits vorhandenem und verfügbaren Wissen entsteht. Die Ähnlichkeiten bei der Motivation der Ersteller von offener Software und die mögliche Motivation von Wissenschaftlern zu Publizieren bietet die Möglichkeit der Verknüpfung der Debatte um Open Source mit der um die Öffnung von Wissenschaft.

Im Konkreten erstreckt sich das auf folgende Aspekte:
\begin{enumerate}
\item Die Kontributoren von Open Source Projekten versprechen sich neue "Karrieremöglichkeiten oder eine Ego-Genugtuung" \cite{Lerner_2001}, Selbstverwirklichung oder Befriedigung der intellektuellen Neugier \cite{Willinsky_2005}, sowie gegenseitige Beurteilung und Anerkennung (non-monetäres Kapital).
\item "Free Software (im Sinne von Open Source), Open Access und Creative Commons sind alles Rechts- und Infrastrukturexperimente"\cite{kelty_2004}. Open Source-Software sollte dabei nicht mit "Shareware" verglichen werden, die zwar kostenlos verbreitet wird, aber deren Quellcode proprietär bleibt \cite{Lerner_2001}
\item Wie bei der wissenschaftlichen Kommunikation, geht es bei der Mitarbeit an Open Source Projekten nicht ausschließlich um altruistische Motive \cite{Lerner_2001}.
\item Der Stand der Forderung nach Öffnung der wissenschaftlichen Kommunikation kann aus technologisch-entwicklungsmethodischer Sicht mit der Debatte um kostenloser Software (Freeware) versus Open Source verglichen werden. Der Vergleich: Freeware und Open Access Publikationen sind zwar kostenlos verfügbar, ihr Erstellungsprozess wird jedoch nicht offen und transparent kommuniziert.
\end{enumerate}
---- TODO: weiter ausarbeiten & neu nach prio sortieren ----

Dieser Vergleich der Öffnung von Wissenschaft mit der Open-Source Bewegung deutet somit ein mögliches Szenario an \cite{Kuhlen_2002_universalaccess}, wie in Zukunft die Wissensproduktion frei und öffentlich gestaltet werden kann.

\subsection{Die Forderung nach Öffnung der wissenschaftlichen Kommunikation}

Der Fortschritt der Wissenschaft ist maßgeblich durch den freien Austausch und der Verbreitung von Informationen bedingt \cite{cite:11}. Das System der wissenschaftlichen Kommunikation, das in der derzeitigen Form seit mehreren hundert Jahren besteht, basiert auf der Forschung, der Begutachtung, dem Druck, der Kommunikation der Ergebnisse in wissenschaftlichen Publikationen, der Verbreitung sowie dem Verkauf an Bibliotheken und andere wissenschaftliche Institutionen \cite{cite:11a} und dem anschließenden Diskurs in der wissenschaftlichen Fachöffentlichkeit \cite{suchen}.

Das aktuell vorherrschende System ist in den 1960er Jahre entwickelt worden und funktionierte am Besten, als die akademischen Ziele und die Marktinteressen noch vereinbar waren. Doch die Rahmenbedingungen wissenschaftlicher Kommunikation haben sich fundamental verändert \cite{epaa_Weiner_2001}. Infolge des weltweit steigenden Haushaltsdrucks der Bibliotheken und wissenschaftlichen Institutionen, des "ungewöhnlichen Geschäftsmodells" \cite{cite:12} der Wissenschaftsverlage mit immer höheren Margen \cite{albert_2006_open_implications} und des Umstandes, dass private Wissenschaftsverlage durch das wissenschaftlichen Reputationssystem über öffentlich finanzierte Wissenschaftlerkarrieren entscheiden \cite{heise_2012}, befindet sich das wissenschaftliche Kommunikationssystem in einer Krise \cite{cite:14}.

Im Rahmen der technologischen Entwicklungen bei der Digitalisierung und des elektronischen Publizierens kann die Öffnung der wissenschaftlichen Kommunikation als eine mögliche Antwort auf diese Krise verstanden werden und setzt bei der Öffnung (Open) und dem freien Zugang (Access) zu wissenschaftlichen Publikationen an und könnte perspektivisch zu einer Öffnung (Open) des Zugriffs auf den Prozess des Forschens (Science) führen.

Diese Antwort birgt aber auch Herausforderungen, die der Philosoph Jean-François Lyotard als "Kommerzialisierung des Wissens" \cite{Das_postmoderne_Wissen_Ein_Bericht} bezeichnet. Infolgedessen der Konsequenzen besteht die Gefahr, dass "Wissen immer weniger der Bildung dient, sondern für den Verkauf geschaffen und konsumiert wird" \cite{hagner_2015_sache_buches}.

\subsubsection{Offener Zugang zur wissenschaftlichen Publikation}

\begin{quote}
Der offene Zugang, auch Open Access, bedeutet, dass Peer-Review-Fachliteratur kostenfrei und öffentlich im Internet zugänglich sein sollte, so dass Interessenten die Volltexte lesen, herunterladen, kopieren, verteilen, drucken, in ihnen suchen, auf sie verweisen und sie auch sonst auf jede denkbare legale Weise benutzen können, ohne finanzielle, gesetzliche oder technische Barrieren jenseits von denen, die mit dem Internet-Zugang selbst verbunden sind. In allen Fragen des Wiederabdrucks und der Verteilung und in allen Fragen des Copyrights sollte die einzige Einschränkung darin bestehen, den Autoren Kontrolle über ihre Arbeit zu belassen und deren Recht zu sichern, dass ihre Arbeit angemessen anerkannt und zitiert wird.
\cite{boai_2012}
\end{quote}

Der offene Zugang zu wissenschaftlicher Kommunikation ist seit der Entwicklung des gedruckten Wortes eng mit der Frage nach Urheberrechten für wissenschaftliche Informationen verknüpft \cite{Case_2000}. Open Access beschreibt ein wissenschaftliches Kommunikationssystem, in dem der Zugang zu den unterschiedlichsten Formen wissenschaftlicher Publikationen, im Gegensatz zum bestehenden System, unter freien, kostenlosen Bedingungen und ohne finanzielle, gesetzliche oder technische Hürden möglich ist \cite{WD_bundestag_2009}. Dieses System ermöglicht darüber hinaus ein "alternatives Geschäftsmodell"\cite{lewis_2012_inevitability} für wissenschaftliche Publikationen. Was auf Maßgabe beruht, dass der Autor die "Eigentumsrechte an den Artikeln, die bisher für die Publikation in wissenschaftlichen Journals an die jeweiligen Fachverlage abgetreten wurden, (...) nun bei den Autoren der Artikel selbst verbleiben" \cite{Hess_2006}.

"Geringere Kostenbarrieren und damit eine einfachere Verbreitung ihrer eigenen Werke" \cite{WD_bundestag_2009} stellen dabei die Wünsche der wissenschaftlichen Autoren und Urheber an Open Access dar. Der Einsatz (offener) Lizenzen ist dafür ein weiterer Haupteinflussfaktor \cite{cite:16}. Open Access hat damit den Zweck, die durch Copyright generierten Barrieren zu überwinden und möglichst schnell und umfassend Zugriff auf neue wissenschaftliche Erkenntnisse zu haben.

\subsubsection{Offener Zugriff auf den wissenschaftlichen Prozess}

Die Entwicklung von “Open Science” knüpft an die Entwicklung der Open Access-Bewegung an und kann als Folge der neuen Möglichkeiten für kollaboratives Arbeiten im Rahmen der Digitalisierung und neuer Kommunikationstechniken verstanden werden. Open Science wird im Folgenden darüber definiert, wie der gesamte wissenschaftliche Wertschöpfungsprozess der Allgemeinheit zur Verfügung gestellt werden kann.

Der Sammelbegriff Open Science erstreckt sich über die gesamte wissenschaftliche Wertschöpfungskette \cite{Scheliga_2014}: Vom offenen Zugang zu Publikationen wissenschaftlicher Forschung (Open Access), sowie den ganzheitlichen wissenschaftlichen Erkenntnisprozess. Unter diesem Gesichtspunkt kann Open Science als eine Weiterentwicklung von Open Access bezeichnet werden. Die diesbezügliche Evolution des Konzepts von Open Access führt zu einem direkten und breiten Weg, Wissenschaft an jedem Schritt der wissenschaftlichen Wertschöpfungskette zu kommunizieren und zu transferieren. Open Science ist die Reaktion auf die Forderung nach offenem Zugriff auf Wissenschaft und Forschung und kann dazu führen, "dass sich die Bedeutung von Forschungsergebnissen zukünftig nicht mehr auf sogenannte klassische wissenschaftliche Publikationen (im Format von Einleitung – Methoden – Ergebnisse – Diskussion), sondern die globale Echtzeitpublikation von Originaldaten stützen wird" \cite{Stengel_2013}.

Wie Open Access hat die Bewegung für Open Science ihre Dynamik der zunehmenden Verbreitung des Internets Anfang der 1990er zu verdanken \cite{Lievrouw_2010} und der neuen Möglichkeiten des kollaborativen Arbeitens sowie des Teilens von Daten und Informationen über das globale Netzwerk \cite{Meyer_2013}. Diese technologischen Entwicklungen ermöglichten jedoch nicht nur das kollaborative Arbeiten zwischen Wissenschaftlern in aller Welt, sondern schafften auch die Möglichkeit die ausgetauschten Informationen nicht nur unter Wissenschaftlern zu teilen, sondern die Verbreitung wissenschaftlicher Informationen an die Gesamtgesellschaft. Befürworter von Open Science sehen hier eine Möglichkeit die gesamten wissenschaftlichen Prozesse, von der Idee bis zur Abschlusspublikation, transparenter, effizienter, nachvollziehbarer und offener zu gestalten.

Diese Vision einer offenen Wissenschaft steht der Verschlüsselungs- und Patentwut zur Wahrung der Geschlossenheit der wissenschaftlichen Informationen und eines möglichen kommerziellen Vorteils durch Wissenschaft im Rahmen öffentlich-finanzierter Forschung gegenüber und führt zu einer Debatte über die Verfügbarkeit der wissenschaftlichen Arbeit und die Entlohnung der "Erfinder" im wissenschaftlichen System \cite{suchen}.

Insbesondere die Entwicklung der Tradition für eine "offenen Wissenschaft" im siebzehnten Jahrhundert bietet einen ersten Ansatzpunkt zur Erforschung der Entwicklung von Open Science \cite{Scheliga_2014}, da dieser historische Übergang noch nicht erforscht ist \cite{CREATe_2014}.

\subsection{Chronologie der Forderung nach Öffnung der wissenschaftlichen Kommunikation}
Für ein erweitertes Verständnis der Prozesse, die zu der Öffnung von Wissenschaft und Forschung führen, ist eine historische Betrachtung der Entwicklung wissenschaftlicher Kommunikation sowie der Forderung nach Offenheit in Wissenschaft und Forschung unabdingbar. Angelehnt an die Arbeiten des kanadischen Philosophen McLuhan und des Germanisten Wenzel können drei bedeutende Umbrüche der MedienentwickAngelehnt lung im Rahmen der Kommunikation von Wissen \cite{wunderlich_2008_buchdruck} genannt werden  \cite{wenzel_mediengeschichte_2007}:
\begin{enumerate}
\item der Übergang vom Körpergedächtnis (brain memory) zum Schriftgedächtnis (script memory)
\item der Übergang von der Handschriftenkultur zur Druckkultur (print memory)
\item und der Übergang vom Buch zum Bildschirm (electronic memory)
\end{enumerate}

\subsubsection{Wissenschaft und wissenschaftliche Kommunikation in pre-modernen Zivilisationen}

Die pre-modernen Zivilisationen bezog sich "Wissenschaft" unmittelbar auf die täglichen Bedürfnissen und das Wissen und Informationen sind als nicht besitzbare Ware angesehen worden\cite{cite:18} \cite{steiner_1998_autorenhonorar}. Im Vergleich zu den heutigen Möglichkeiten war in den vormodernen Zivilisationen der Wissensaustausch stark beschränkt \cite{cite:17c} und es gab keine "scharfe Grenze zwischen dem vorhandenen und dem aktuell benutzten Wissen"\cite{Luhmann1998}. Die Produktion von Literatur beschränkte sich in den vorwissenschaftlichen Gesellschaften vornehmlich auf "auf die Überlieferung und Kommentierung des althergebrachten Wissens, insbesondere des theologischen" Wissens \cite{steiner_1998_autorenhonorar}. Was die Gelehrte "zu sagen und zu schreiben hatten, war nicht als Beitrag zum Fortschritt von Wissenschaft als einem kollektiven Unternehmen zu verstehen, sondern eher als Dokumentation ihrer persönlichen Erkenntnisfortschritte" \cite{graefen2007_wissenschaftliche_artikel}, somit stellten sich vor allem der Aufgabe "das Wissen zu verbessern und vor allem zu erhalten und zu tradieren" \cite{Luhmann1998}. "Eine Textart, die dem Wissenschaftlichen Artikel entspricht oder mit ihm ver- gleichbar ist, existierte im Mittelalter nicht." \cite{graefen2007_wissenschaftliche_artikel}

---- TODO: weiter ausführen ----

\subsubsection{Die Einführung des Buchdrucks als Grundlagen der modernen Wissenschaft}

Die Geschichte des gedruckten Buchs beginnt maßgeblich mit Johannes Gensfleischs, auch Gutenberg genannt, Beiträgen zur Buchdruckerkunst \cite{wittmann_1999_geschichte} Mitte des 15. Jahrhunderts \cite{suchen}. Die Einführung des Buchdrucks führte nicht nur zu neuen Möglichkeiten der Kommunikation, sondern zu einer Veränderung der generellen Aufgabe von Wissenschaft, insbesondere ihrer der Orientierung auf den täglichen Bedarf \cite{Luhmann1998}. Durch die neue Möglichkeiten der Vervielfältigung und Massenverbreitung hat das Selbstverständnis der europäischen Kultur in bis dahin unbekannter \cite{giesecke_1991_buchdruck} und revolutionärer Weise verändert \cite{wunderlich_2008_buchdruck}. Der Buchdruck stellte die "Grundlagen und Meilensteine sowohl für die Kommunikation der Menschheit insgesamt als auch für den wissenschaftlichen Gedankenaustausch im Besonderen dar" \cite{schirmbacher_2009_wisspub}, er war ein "Bestandteil des Übergangs vom Mittelalter in die frühe Neuzeit" \cite{lange2008medienwettbewerb} und leitete die "Moderne" ein \cite{luhmann_1997_gesellschaft}.

---- TODO: "Moderne" Zitat prüfen wenn es nicht stimmt weg, weil angreifbar ----

Diese neue Technologie führte zu einem bis dahin unbekannten, explodierenden Informationsangebot. Infolgedessen sich eine neue Denkstruktur entwickelte \cite{eisenstein_1997_druckerpresse}, bei der das "mittelalterliche Denken in Bildern und Metaphern" von der "wissenschaftlich-systematischen Methodik" abgelöst wurde \cite{wunderlich_2008_buchdruck}. Sie führte zur Befreiung des Autors aus der weitgehenden Anonymität mittelalterlicher Manuskriptkultur und zur Entkopplung der "Herstellung und Verbreitung vom singulären Interesse eines Autors, Kopisten oder Auftraggebers"\cite{wunderlich_2008_buchdruck} \cite{schirmbacher_2009_wisspub}.

Mit der Entwicklung der Buchdrucktechnologie folgte im 16. Jahrhundert die Verbreitung eines "freien Marktes als Vertriebsnetz für typographische Informationen"\cite{giesecke_1991_buchdruck} und die "Kapitalisierung der Buchproduktion" \cite{steiner_1998_autorenhonorar}. Das gedruckte Wort führte somit zu einem "Verlust an Macht und Herrschaft über das geschriebene Wort" \cite{wunderlich_2008_buchdruck}. Anfangs handelte es sich bei der Technologie nur um ein "elitäres und teures Medium für die gebildete Klasse" \cite{hartmann_2008_medien}. Sie führte weder von Beginn an zum zeitlich unmittelbaren Zugang zu Wissen noch war sie sofort für die Allgemeinheit zugänglich \cite{hartmann_2008_medien}. Die wissenschaftliche Elite der damaligen Zeit forderte deshalb, dass Werke ohne Rücksicht auf Profitgier erscheinen sollten und appellierte an eine "obrigkeitliche Lenkung", damit der Buchhandel "seiner Aufgabe der Verbreitung von nützlichem Wissen gerecht würde" \cite{wittmann_1999_geschichte}. Gutenbergs Druckinnovation sollte als sogenannte "Schlüsseltechnologie" \cite{jager_1993_theoretische} eine neue Dimension der Informations- und Wissensverbreitung für die Gesamtgesellschaft ermöglichen.

In der Übergangszeit von der primären Kommunikation zwischen den Gelehrten anhand von Briefen und der Verbreitung des Buchdrucks kam es zu einer Vielzahl sogenannter Prioritätsstreits in der Wissenschaft und Forschung \cite{schirmbacher_2009_wisspub}. Denn die meisten wissenschaftliche Erkenntnisse waren zwar im direkten Briefwechsel, aber noch nicht öffentlich verbreitet worden und deshalb konnte zu dieser Zeit selten ein für alle nachvollziehbarer Bezug zum jeweiligen Entdecker hergestellt werden. Als Beispielhaft für einen solchen Prioritätsstreit kann die Auseinandersetzung zwischen Isaac Newton und Gottfried Wilhelm Leibniz um eine Veröffentlichung zur Fluxionsrechnung im 17. Jahrhundert genannt werden. Leibniz rezensierte eine von Newton verfasste Veröffentlichung anonym und stellte sich selbst namentlich als Erfinder dieser dar \cite{2013_leibniz}, ohne auf eine öffentliche Publikation seiner deutlich länger vorhandenen Erkenntnisse hinweisen zu können \cite{schirmbacher_2009_wisspub}. Auf Grund des fehlenden öffentlichen Nachweises wurde Leibniz infolgedessen durch die Royal Society, ---- TODO: RS genauer erklären ---- des Plagiats für schuldig befunden und Entdeckung Newton zugesprochen. Der Buchdruck, wie auch die ersten wissenschaftlichen Zeitungen, wurden für die wissenschaftlichen Autoren somit nicht nur zu einem neuen "Kommunikationsinstrument", einem Instrument zur "Erlangung von Reputation" oder zu einem Instrument "zur Generierung finanzieller Erträge" sondern auch zu einem "Nachweisinstrument" \cite{wunderlich_2008_buchdruck} \cite{schirmbacher_2009_wisspub} für die Vermeidung solcher Prioritätskonflikte.

Die Verbreitung des Buchdrucks fand aber nicht ungebremst und nicht ohne umfassende Kritik in der damaligen Gesellschaft statt. Vor allem kirchliche Instanzen waren über eine "wachsende theologische Begriffsverwirrung" und die Verbreitung der Schriften in Volkssprachen besorgt \cite{giesecke_1991_buchdruck}. Sie stellten die größten Gruppe an Kritikern des Buchdrucks dar und versuchten die neue "Bücherflut" zu unterbinden\cite{giesecke_1991_buchdruck}. Zwischenzeitlich führte die Einführung des Buchdrucks zu einer neuen Bedeutung der Zensur, als "prohibitives Instrument für die Überwachung der Lektüren" und als "Kampfmittel" \cite{sprachgeschichte_1998_besch} gegen zu viel Wissen \cite{suchen} und "unerwünschte Literatur" \cite{suchen}. Beispielhaft für diese Art der Zensur, zitiert der Kommunikations- und Medientheoretiker Michael Giesecke aus einem Gutachten dieser Zeit: "In den Anfängen muß man Widerstand (gegen das Übel des Drucks von Büchern, die aus den heiligen Schriften in die Volkssprache übersetzt sind), damit nicht durch die Vermehrung der deutschsprachigen Bücher der Funke des Irrtums endlich sich zu einem großen Feuer entwickle" \cite{giesecke_1991_buchdruck}.

Zusammenfassend nennt Giesecke vor allem folgende grundlegenden Einwände gegen den Buchdruck als unregulierte, "freie" Kunst \cite{giesecke_1991_buchdruck} für die Verbreitung von Wissen und Informationen:
\begin{itemize}
\item Die Einführung des Buchdrucks wurden von vielen Warnungen vor Missbrauch der Technologie begleitet \cite{lange2008medienwettbewerb}. Im Mittelpunkt der Warnungen standen der antireligöser Missbrauch durch die Verbreitung gefährlichen Gedankenguts \cite{kruse2003multimedia}, die bewusste Falschinformation und Verfälschung von Inhalten \cite{sprachgeschichte_1998_besch}, die willkürliche Informationsverbreitung über Bücher, ohne Zustimmung der geistlichen und weltlichen Regenten \cite{rother_2002_siebenbuergen} und die Angst der Traditionalisten, die ihre Herrschaft durch das Monopol auf die Interpretation der Bibel gefährdet sahen \cite{lange2008medienwettbewerb}.
\item Ein weitere Einwand adressierte die Befürchtung, dass die Qualität und Reinhaltung der besten Texterzeugnisse beim Buchdruck nicht sichergestellt werden kann \cite{giesecke_1991_buchdruck}.
\item Auch die Nachlässigkeit und Unachtsamkeit von Buchdruckern und Setzern wurde früh kritisiert. Sie spielten im Buchdruckprozess eine entscheidende Rolle, da sie großen Einfluss auf die Qualität der Nachdrucke hatten. Nachlässigkeit oder ungenaues Arbeiten führten zu erheblichen strukturellen und inhaltlichen Qualitätsverlusten, was von Autoren wie Martin Luther schon früh beklagt wurde \cite{sprachgeschichte_1998_besch}.
\item Die Multiplikation von Fehlern, da in den gedruckten Exemplaren auch die Fehler völlig Übereinstimmen und nicht behoben werden können, schließt an die Kritik der Qualität der gedruckten Bücher an. Die Befürchtung begründete auf der Irreversibilität der Verbreitung fehlerhafter Inhalte beim Buchdruck, die bei der geringeren Anzahl handschriftlichen Kopien bisher weniger Einfluss hatte.
\item Die staatlichen und geistigen Obrigkeiten befürchteten durch die Demokratisierung der Vervielfältigung und Verbreitung von Wissen die Verwirrung der "Laien" (der Glaubensgemeinschaft) und damit einen Kontrollverlust für die bestehende gesellschaftliche Ordnung. \cite{giesecke_1991_buchdruck}.
\item Demzufolge befürchtete die Obrigkeit, die Auflösung der ständischen Ordnung da der "Zugang zu den Speichern des Wissens nicht länger bestimmten Schichten vorbehalten bleibt" und das "Schreiben und Lesen wird von einer ständischen zu einer gemeinen Tätigkeit". Aus heutiger Sicht mag diese Sicht auf Grund der sehr geringen Alphabetisierungsrate und der noch immer sehr geringen Anzahl an Büchern Ende des 15. Jahrhunderts als unbegründet erscheinen, dennoch wurde die soziale Umwälzung durch den Buchdruck beschleunigt und unumkehrbar gemacht. \cite{giesecke_1991_buchdruck}
\item Auflösung des "Amts" des Bücherschreibers als eigenes Handwerk
\item Die Angst vor dem Überfluss an Büchern und Wissen stellte einen weiteren Einwand dar. Die Kritiker der Buchdrucktechnologie befürchteten  durch die massenhafte Verbreitung ein Chaos an Informationen \cite{giesecke_1991_buchdruck}.
\item Sogar physische Konsequenzen wurden befürchtet: "Augen schmerzen, vom Lesen, unsere Finger vom Blättern" \cite{giesecke_1991_buchdruck}
\item Auch "psychische Bedenken" wurden eingebracht, so gab es im 15. Jahrhundert bei den Menschen die Angst vor dem Anhäufen von Informationen. Sie galt im Mittelalter als "gefährliches und verwirrendes Unterfangen" und führte zu Annahmen wie "je gelehrter, je verkehrter". \cite{giesecke_1991_buchdruck}
\end{itemize}

Die genannten Einwände fußten alle auf Ängsten oder Befürchtungen vor den Veränderungen der etablierten Machtstrukturen, die die Informationsverbreitung bis Ende des Mittelalters beeinflusst hatten. Vor der Einführung des Buchdrucks wurde vorab entschieden, was veröffentlicht und verbreitet wurde und es gab klare Instanzen, die die Weitergabe von Wissen (meist Auftragsarbeiten) organisierten. Der Buchdruck kehrte dieses System um, da nun Texte erstmals verbreitet wurden und man es dem "Markt und dem nachträglichen Meinungsstreit überließ, welche Information zum Gemeingut wurden" \cite{giesecke_1991_buchdruck}. Niklas Luhmann fasste diese Veränderung hin zu einem Prozess wie folgt zusammen: "Wer für den Druck schreibt, gibt die Situationskontrolle auf" und "produziert für das Gedächtnis des Systems" bei dem weder "Kommunikationsvorgang" noch der "Wissenszuwachs" abgeschlossen sind \cite{Luhmann1998}.

Die Etablierung des Drucks führte, zunächst "unbemerkt und naturwüchsig", zu einer Veränderung der Sozialisierung von Informationen, der Veröffentlichung \cite{giesecke_1991_buchdruck}. Das Medium der Schrift wurde demnach unter den Buchdruckbedingungen als eine "Verbreitungstechnologie" für Informationen genutzt, die zwar die unmittelbare Interaktion zwischen Sender und Empfänger (weiterhin) ausschloss, aber mittelbar nur mit Hilfe von Empfängern zu Wissen werden konnte \cite{Luhmann1998}.

Die Einführung des Buchdrucks stellte somit einen Bestandteil des "Übergangs vom Mittelalter in die frühe Neuzeit dar"\cite{lange2008medienwettbewerb}, da zwischen Buchdruck und demokratischen Freiheiten "sowohl faktisch als auch ideologisch" \cite{suchen} ein Zusammenhang hergestellt werden kann. Dieser Zusammenhang wird darin deutlich, dass im Gegensatz zum Mittelalter, in dem jede breitere Sozialisierung und Verbreitung privater Gedanken "legitimationsbedürftig" war, nun jeder Eingriff in die "Freiheit, Meinungen oder Informationen" zu drucken einer politischen Legitimation \cite{giesecke_1991_buchdruck}. Der Buchdruck kann demnach im Rahmen der "fundamentalen Umbrüche in Politik und Verwaltung, Ökonomie und Handel, Religion, Bildung und nicht zuletzt in den Prozessen der kognitiven Welterkenntnis" \cite{pscheida_2010_wikipedia} als "Katalysator des kulturellen Wandels"\cite{giesecke_1991_buchdruck} verstanden werden.

Um den Arbeitsaufwand der Drucker zu honorieren und die verlegerische Leistung zu würdigen\cite{szilagyi_2011_leistungsschutzrecht}, wurden mit der Entstehung des Druckerwesens auch erste Privilegien vergeben \cite{gieseke_1995_privileg}, die es den Druckern erlaubte, die Buchdruckkunst für einen bestimmten Zeitraum allein oder in einem bestimmten Gebiet auszuüben \cite{martin2008publizistische} \cite{koller_1995_Urheberrecht}. Diese Privilegien ermöglichen dem Begünstigen Sonderberechtigungen oder Rechte gegenüber den allgemeinen Rechtsregeln \cite{jaenich_2002_geistiges}. Im Zuge der Verbreitung der Drucktechnologie und des steigenden Wettbewerbs kam es auch zu den ersten Privilegien für Urheber und Erstverleger, die sich damit versuchten gegen das Nachdrucken und Raubdrucken zu erwehren. Die erfolgreiche Einforderung dieser Privilegien führte schon früh zu einer Art Monopolstellung bestimmter Druckereien und zu einem generellen Nachdruckverbot für bestimmte Werke in einem bestimmten Gebiet oder für einen bestimmten Zeitraum \cite{szilagyi_2011_leistungsschutzrecht}. Später wurden auch erste Autorenprivilegien gewährt.

\subsubsection{Wissenschaftliche Journale als Medium der wissenschaftlichen Kommunikation}

Noch zu Beginn des 17. Jahrhunderts stellten das Schreiben von Briefen oder Büchern die häufigsten Formen des wissenschaftlichen Austauschs dar \cite{porter_1964_scientific}. Der Brief, als besonders exklusive Form der Kommunikation stand dem Buch als sehr zeitaufwändige Form gegenüber \cite{fecher_hiig_2014}.

Erst die "drucktechnische Möglichkeit der schnellen Produktion, Vervielfältigungund Verbreitung von Texten" und "die Loslösung der Wissenschaft(en) von Religion und schöner Literatur" machten eine "Umorientierung von sporadischer individueller wissenschaftlicher Betätigung hin zu gesellschaftlich anerkannter und zur Kenntnis genommener, kollektiv bzw. arbeitsteilig betriebener Wissenschaft" möglich \cite{graefen2007_wissenschaftliche_artikel}. Ergänzt durch die Gründung von Akademien als einer Art von nationalen Gelehrtengesellschaften im 17. und 18. Jahrhundert führte das zu Veränderungen der wissenschaftlichen Literatur \cite{graefen2007_wissenschaftliche_artikel}. Diese fungierten als Vereingung einzelner Gelehrter und "durch sie fand eine Konzentration vereinzelter wissenschaftlicher Anstrengungen und Leistungen statt"\cite{graefen2007_wissenschaftliche_artikel}. Mitte des 17. Jahrhunderts kam es in Folge der Gründung der "Royal Society", als eine Akademie zur Förderung naturwissenschaftlicher Experimente, zu einer wissenschaftlichen Diskussion über die Etablierung einer neuen "Philosophie für die Förderung von Wissen". Die Mitglieder der Royal Society hegten den Wunsch nach einer Verbesserung bei der Verbreitung wissenschaftlicher Erkenntnisse und eine "wissenschaftlichen Revolution" mit Hilfe der Drucktechnologie voranzutreiben \cite{Dear_1985}. Als ein Ergebnis der 1660 gegründeten Akademie erschienen 1662 die ersten beiden Bücher, John Evelyn's "Sylva" und "Micrographia" von Robert Hooke \cite{hall_1992_library_rsol}. Am 6. März 1665 mit "Philosophical Transactions" eine der ersten wissenschaftliche Fachzeitschriften \cite{suchen}, "die bis ins 20. Jahrhundert hinein eine der angesehensten Fachzeitschriften blieb" \cite{graefen2007_wissenschaftliche_artikel}. Im gleichen Jahr erschien auch das "Journal des sçavans" in Frankreich, das zu Beginn über aktuelle Entdeckungen berichtete \cite{epaa_Weiner_2001}. Bis zum 17. Jahrhunderts folgten circa 30 weitere Journalgründungen diesem Beispielen. Die Journals unterschieden sich in ihrer Struktur stark von den heutigen und wiesen bis Ende des 18. Jahrhunderts kaum eine fachliche Spezialisierung auf und beinhalteten "auch anwendungs- und praxisbezogene Beiträge" \cite{graefen2007_wissenschaftliche_artikel}. Sie enthielten im Vergleich zu den heutigen Fachzeitschriften jeweils eine nur sehr geringe Anzahl von Beiträgen \cite{suchen} und waren an wissenschaftlichen Briefe (meist in der Ich-Form) angelehnt, die Wissenschaftler vor der Entwicklung der Journale noch direkt aneinander verschickt hatten \cite{epaa_Weiner_2001}. "Oft handelte es sich gar nicht um Originalbeiträge, sondern die Herausgeber teilten der gelehrten und gebil- deten Menschheit mit, was sie aus ihren Briefwechseln mit Gelehrten Interessantes entnahmen" \cite{graefen2007_wissenschaftliche_artikel}.

Die wissenschaftliche Fachzeitschrift oder das wissenschaftliche Journal, wie wir es heute kennen, geht strukturell auf das 19. Jahrhundert zurück, als die forscherischen Aktivitäten und das öffentliche Interesse an der Wissenschaft generell anstieg. In dieser Zeit kam es zu den meisten Gründungen der großen Fachzeitschriften von heute \cite{porter_1964_scientific}. Bis zur Etablierung des Peer-Review-Verfahrens als Qualitätssicherungsverfahren in der zweiten Hälfte des 20. Jahrhunderts gab es sehr unterschiedliche oder keine Verfahren zur Sicherung der Qualität von Inhalten in den Journalen. Im 20. Jahrhundert folgte auf die weltweite Intensivierung wissenschaftlicher Aktivitäten ein weiterer rasanter Anstieg der wissenschaftlichen Journale. Im Jahr 1961 wurde die erste quantitative Studie an Hand der Anzahl von wissenschaftlichen Journalen durchgeführt. Im Rahmen dieser Erhebung wurde von 50.000 wissenschaftliche Zeitschriften und von einer Verdopplung der Anzahl aller wissenschaftlichen Journale alle 15 Jahre ausgegangen \cite{de_1982_little}.

\subsubsection{Die Rolle der Verlage und die Publikationskrise}

Noch bis in das 19. Jahrhundert wurden Bücher unter dem Eindruck einer "Unsterblichkeitsnorm geschrieben, die darauf baute, dass erst die Nachwelt das eigentliche Anliegen eines Buches verstehen würde" \cite{hagner_2015_sache_buches}. Darüber hinaus wurden Entdeckungen manchmal in Form eines Anagramms veröffentlicht, so etwa Galileis Entdeckungen der Jupitermonde \cite{miner2007discovery} und Hookes Elastizitätsgesetz \cite{szabo_2013_geschichte}. Auf diese Weise konnten Prioritätsrechte gesichert werden, ohne dass die Entdeckung selbst veröffentlicht werden mussten \cite{miner2007discovery}. Erst ab Mitte des 19. Jahrhunderts "verlagerte sich die Produktion immer mehr auf das Hier und Jetzt" \cite{hagner_2015_sache_buches}.

Ursprünglich wurden die wissenschaftlichen Journale und Bücher an Universitäten gedruckt, waren in gleicher Weise verbreitet \cite{hagner_2015_sache_buches} und Eigentum derer, die dafür schrieben oder sie lasen \cite{epaa_Weiner_2001}. Sie wurden durch die wissenschaftlichen Akademien oder akademischen Fachgesellschaften, die auch die inhaltliche Ausrichtung verantworteten sowie die finanzielle Trägerschaft übernahmen \cite{suchen}, als Kommunikationsmedium organsiert. Erst im 20. Jahrhundert kam es zu einer unterschiedlichen Verbreitung der Veröffentlichungsformate in und zwischen den unterschiedlichen Disziplinen und im späten 20. Jahrhundert kam der Sammelband als neue Form dazu \cite{hagner_2015_sache_buches}.

Mit dem weltweiten Anstieg der wissenschaftlichen Forschung Mitte des 20. Jahrhunderts und der stetig steigenden Anzahl an wissenschaftlichen Publikationen nach dem zweiten Weltkrieg stieß das universitätseigene Journalsystem an seine Grenzen und es entwickelte sich zu einem "Flaschenhals" \cite{epaa_Weiner_2001} im Kommunikationssystem. Dem Anstieg an wissenschaftlicher Forschung und dem zunehmenden Publikationsdruck der Wissenschaftler konnte das System nicht mehr gerecht werden. Kommerzielle Verlage entdeckten diese Lücke und begannen den Markt mit Unterstützung der überforderten Universitäten zu absorbieren \cite{suchen}.

Nachdem die Kommerzialisierung des System anfangs gut funktionierte, kam es zunehmend zu einem Bruch, als die Anforderungen des Marktes nicht mehr denen der akademischen Gemeinschaft entsprachen \cite{epaa_Weiner_2001}. Dennoch verharrte die wissenschaftliche Gemeinschaft in einem "weltfremden" Zustand, in dem der Druck zu veröffentlichen, dazu führte, dass sie ein System unterstützten, das sie ausnutzte \cite{epaa_Weiner_2001}. Auch in Deutschland nahmen Anfang der 1990er Jahre die wissenschaftlichen Verlage eine marktbeherrschende Stellung ein und agierten exklusiver Distributor bei der Veröffentlichung wissenschaftlicher Informationen \cite{schloegl_2005} \cite{offhaus_2012_institutionelle_repos}.

Diese Entwicklung basiert auf dem in der Welt des geistigen Eigentums ungewöhnlichen Umstand, dass seit dem Beginn des wissenschaftlichen Journals im Jahr 1665, wissenschaftliche Autoren nicht vordergründig finanzieller Belohnung profitierten, sondern maßgeblich durch die weite Verbreitung und Hinweise auf ihre Arbeit, sowie die wissenschaftlichen Erkenntnisse ihrer Forschung \cite{albert_2006_open_implications}. Darüber hinaus ist es eine Besonderheit des Systems, dass Wissenschaftler sowohl Produzenten als auch Konsumenten der Wissenschaftskommunikation sind und damit Ihre eigene Zielgruppe darstellen \cite{Hess_2006}. Die kommerziellen Verlage haben sich dieses System zu nutze gemacht.

Im Laufe der Zeit erlangten die Verlage eine Vormachtstellung im wissenschaftlichen Publikations- und Distributionssystem. Diese stützt sich bis heute auf drei Säulen \cite{offhaus_2012_institutionelle_repos} \cite{bargheer_2006_open}:
\begin{enumerate}
\item "Urheberrecht, wonach Verlage [...] weitgehende Ansprüche an dem veröffentlichten Werk erwerben“;
\item "redaktionelle Themenbündelung (bundling)“;
\item Organisation der "Qualitätssicherung durch Begutachtung (Peer Review)"
\end{enumerate}

Die marktbeherrschende Stellung der Verlage führte zu einer Situation, in der die Verlage die Preise für wissenschaftliche Publikationen weitgehend diktieren und Preiserhöhungen unlimitiert durchgesetzt werden konnten. Als Folge der ungebremsten Ausnutzung dieser Marktmacht kam es kurz vor der Jahrtausendwende zur sogenannten "Zeitschriftenkrise" \cite{schirmbacher_2009_wisspub} \cite{muller_2010_open}. Die Zeitschriftenkrise, "die richtigerweise Zeitschriftenpreiskrise oder Zeitschriftenpreisexplosion genannt werden müsste"\cite {Brintzinger_2010}, kam als Begriff das erste Mal in den 1990er Jahren auf \cite{Boni_2010}. Diese Krise war das Ergebnis folgender Entwicklungen auf der Angebots- und Nachfragenseite \cite{Brintzinger_2010}: Auf der Angebotsseite wurden durch einen "Konzentrationsprozess" "innerhalb von etwas mehr als einem Jahrzehnt im Bereich der Zeitschriften mittelständische Verlage nahezu vollkommen durch internationale Kapitalgesellschaften substituiert". \cite{Brintzinger_2010} Unterstützt von der Nachfrageseite resultierte daraus eine "monopolistische Preispolitik" der Verlage \cite{Brintzinger_2010}. Ein zeitgleicher Anstieg der Titelvielfalt, bei der aus "einer mehr generalistischen Zeitschrift drei oder vier Spezialzeitschriften" entstanden, "die dann allesamt wieder von Bibliotheken abonniert werden mussten"\cite{Brintzinger_2010}, verschärfte das Problem. Eine weitere Ursache für die krisenhafte Zuspitzung der Situation besteht in der institutionellen Organisation der Literaturbeschaffung an den Hochschulen und wissenschaftlichen Einrichtungen. Bei der Arbeitsteilung von Bibliothekaren und Wissenschaftlern war und ist es für das Ansehen des einzelnen Faches durchaus rational, mit einem möglichst hohen Anteil am Gesamtetat der Literaturbeschaffung zu partizipieren. Es gibt für individuelle Einsparungen von allen Seiten nur wenig Anlass, da beide Systeme unabhängig voneinander funktionieren. \cite{Brintzinger_2010}.

Die Preisexplosion konnte auch durch die Bildung von Bibliothekskonsortien, "deren Aufgabe es war, für Bibliotheken kostengünstige Rahmenbedingungen auszuhandeln", nicht gebändigt werden \cite{Fladung_2003} \cite{Brintzinger_2010}. Gleichzeitig standen die Wissenschaftler unter einem starken Publikationszwang, der mit "Publish or Perish" \cite{CLAPHAM_2005} beziehungsweise "impact factor fever" \cite{Cherubini_2008} und "impact factor race" \cite{Brischoux_2009} beschrieben wurde \cite{offhaus_2012_institutionelle_repos}. "Publish or Perish" beschreibt das Problem, dass im Rahmen der "wachsenden Konkurrenz um Forschungsförderung und akademische Positionen (...) kombiniert mit dem zunehmenden Einsatz bibliometrischer Parameter für Evaluation" \cite{Fanelli_2010} junge Akademiker viel und vornehmlich mit positiven wissenschaftlichen Ergebnissen publizieren müssen um Anerkennung und gegebenenfalls eine Anstellung im Wissenschaftsbetrieb zu erreichen \cite{pscheida_2010_wikipedia} \cite{Beasley_2005}. Das erzeugte viel "nutzlose Forschung und Artikel"\cite{smith1990killing}.

\subsubsection{Das Internet als neues Medium wissenschaftlicher Kommunikation}

Das Ende des letzten Jahrtausends eröffnete das Medium Internet "neue Nutzungsmöglichkeiten, durch welche die Schrift als ein Medium einsetzbar wird, das den permanenten Wechsel zwischen Sender- und Empfängerposition ähnlich flexibel zu gestalten erlaubt, wie es im gesprochenen Gespräch der Fall ist" \cite{sandbothe_2000_pragmatische}. Der Webbrowser hat sich als Kreuzung zwischen Buch und Fernseher entwickelt bei dem das multimediale Dokument von der Buchkultur als zentrales Wahrnehmungsobjekt übernommen wird, zugleich aber darüber hinaus greift \cite{Warnke_2011}. Als weiteres Veränderung in Abgrenzung zur Technologie Buchdrucks revidierte das Internet "die Vorstellung von einem geschlossenen Sinngehalt" \cite{sandbothe_2000_pragmatische} mit einem Anfang und Ende wie zum Beispiel in einem Buch.

Diese Entwicklung folgt der Annahme, dass die Buchkultuur von einer Dialogkultur abgelöst, aber nicht vollständig verdrängt wird und das Gedruckte als eine Art Rückzugs- oder Entlastungsmedium zum Einsatz kommt \cite{hagner_2015_sache_buches}.

---- TODO: weiter ausarbeiten siehe Text Warnke ----

\subsubsection{Die ersten Experimente mit dem offenen Zugang zu wissenschaftlichen Publikationen}

Die Zeitschriftenkrise und der gestiegene Publikationsdruck stellen zwei fundamentale Gründe für das Aufkommen der Forderungen nach "Open Access" dar \cite{Brintzinger_2010} \cite{suchen}. Als Reaktion auf die Herausforderungen und auf Basis der Digitalisierung gründete Anfang der 1990er der Physiker Paul Ginsparg mit arXiv den ersten wissenschaftliche Preprint-Dienst des Internets \cite{suchen}, der es Wissenschaftlern ermöglichen sollte, Ideen vor der gedrukten Veröffentlichung zu teilen.

Ein Ausgangspunkt dafür waren die ersten Experimente mit offenem Zugang und freien Lizenzen für Publikationen in der Wissenschaft aus den 1960er Jahren und somit schon vor der Zeit der Erfindung des Internets \cite{cite:18b}. Noch bevor die digitalen Nutzungsmöglichkeiten verfügbar waren und bevor an das "globalen Dorf"\cite{mcluhan_1962_gutenberg} zu denken war, wurde vor allem in den Technik- und Naturwissenschaften eine “pre-print Kultur” entwickelt bei der die Autoren ihre zur Begutachtung eingereichten Artikel zeitgleich oder bevor diese veröffentlicht wurden unter Kollegen über den Postweg zirkulieren ließen, um den Kommunikationsprozess zu beschleunigen \cite{suchen-Hoffmann-Zugang-undVerwertung-öffentlicher-Informationen}.

Mitte der 1990er forderte Steven Harnad die wissenschaftliche Community dazu auf, sofort mit der digitalen Selbstarchivierung und öffentlichen Zurverfügungstellung ihrer Beiträge zu beginnen \cite{albert_2006_open_implications}, um "den Barrieren, die zwischen ihrer Arbeit und ihrer (kleinen) Leserschaft aufgestellt werden, zu entkommen" \cite{harnad_1995_subversive_proposal}.

Durch die zunehmende Verbreitung und Nutzung der dieser digitalen Pre-Print Dienste, gründete sich im Oktober 1999 im Rahmen der "Santa Fe Convention" die "Open Archives Initiative", die sich maßgeblich mit den technischen und organisatorischen Aspekten der Transformation der wissenschaftlichen Kommunikation beschäftigte. ---- TODO: Bitex The Santa Fe Convention of the Open Archives Initiative ----

2001 wurde der europäische Ableger von der Scholarly Publishing and Academic Resources Coalition (SPARC) einer der späteren "major player" der Open Access Bewegung \cite{russell2008business} \cite{Herb_2012} gegründet. Als Konsequenz aus der Zeitschriftenkrise sollte diese 1998 in den USA gegründete Allianz zwischen Universitäten und wissenschaftlichen Bibliotheken dafür sorge tragen, dass die Kosten für wissenschaftliche Publikationen reduziert werden und durch die Bereitstellung kostengünstiger oder freier, nicht-kommerzieller, Peer-Review-Fachzeitschriften zu ersetzen sind. Durch Weiterbildung, politische Arbeit und die Förderung alternativer Geschäftsmodelle, war es Ziel von SPARC, Initiativen für offenes wissenschaftliches Publizieren zu stimulieren \cite{suchen}.

\subsubsection{Die Manifestierung der Forderung nach offenem Zugang}

In 2001 erschien Open Access erstmals im wissenschaftlichen Diskurs als eigenes und öffentlichkeitswirksames Thema \cite{cite:19}. Die Public Library of Science (PLoS), gegründet im Oktober 2000, forderte die gesamte wissenschaftliche Gemeinschaft in einem offenen Brief im Mai 2001 dazu auf, ab September 2001 nur noch in Zeitschriften zu veröffentlichen, nur noch die Zeitschriften zu reviewen, zu editieren und zu abonnieren, deren Beiträge spätestens sechs Monate nach ihrer Erstveröffentlichung für jedermann im Internet kostenlos und unentgeltlich einsehbar sind \cite{cite:20}. Schon nach kurzer Zeit unterzeichneten nach eigenen Angaben \cite{cite:19a} rund 38.000 Wissenschaftler und Wissenschaftlerinnen aus 180 Nationen das Schreiben. Auf diesen Brief folgte ein 20-monatige sehr aktive und öffentlichkeitswirksame Phase der Forderung nach Öffnung der wissenschaftlichen Kommunikation. In diesen 20 Monaten wird neben PLoS der britische Verlag Biomed Central als weiterer "Wegbereiter in der von OA" \cite{suchen-Hoffmann-Zugang-undVerwertung-öffentlicher-Informationen} gegründet und es entstehen drei der bis heute wichtigsten Erklärungen im Bereich der Öffnung des Zugangs zu wissenschaftlicher Kommunikation \cite{CREATe_2014}:
\begin{enumerate}
\item Erklärung der Budapest Open Access Initiative (Dezember 2001 und 2012)

Im gleichen Jahr wie der PLoS-Brief, wurden im Rahmen einer Konferenz des Open Society Institutes in Budapest, mit der “Budapest Open Access Initiative" (BOAI)\cite{boai_2012} erstmals die Bemühungen um Open Access in einer eigenen Erklärung zusammengefasst \cite{cite:21a}. Im Fokus dieser Erklärung steht die Forderung nach freiem Zugang zu wissenschaftlichen Publikationen. In der BOAI wird erstmals manifestiert, dass wissenschaftliche Peer-Review-Fachliteratur “kostenfrei und öffentlich im Internet zugänglich sein sollte, so dass Interessenten die Volltexte lesen, herunterladen, kopieren, verteilen, drucken, in ihnen suchen, auf sie verweisen und sie auch sonst auf jede denkbare legale Weise benutzen können, ohne finanzielle, gesetzliche oder technische Barrieren jenseits von denen, die mit dem Internet-Zugang selbst verbunden sind". \cite{boai_2012} Die Erklärung manifestiert auch, dass in "allen Fragen des Wiederabdrucks und der Verteilung und in allen Fragen des Copyrights überhaupt, sollte die einzige Einschränkung darin bestehen, den Autoren Kontrolle über ihre Arbeit zu belassen und deren Recht zu sichern, dass ihre Arbeit angemessen anerkannt und zitiert wird." \cite{boai_2012}

Anlässlich des zehnten Jahrestages der BOAI, wurde von der Open Society Foundation mit der BOAI 10 (2012) die ursprüngliche Erklärung um weitere Richtlinien und Empfehlungen für die Entwicklungen und Herausforderungen bei der Öffnung wissenschaftlicher Kommunikation ergänzt. Die Initiatoren kommen unverändert zu dem Schluss, dass "noch immer Zugangsbeschränkungen zu Peer-Review-Forschungsliteratur, meist eher zugunsten der Verlage, als zugunsten der Autoren, Reviewer oder Redakteure und damit auch auf Kosten der Forschung, Forscher und Forschungseinrichtungen" \cite{boai_2012} bestehen. Dazu heißt es in der überarbeiteten Erklärung: "Nichts aus den letzten zehn Jahren lässt darauf schließen, dass das ursprüngliche Ziel von Open Access weniger sinnvoll oder erstrebenswert erscheint. Im Gegenteil, die Notwendigkeit, dass Wissen für jeden, der es nutzen, anwenden oder darauf aufbauen kann, offen verfügbar sein sollte, ist dringlicher als je zuvor" \cite{boai_2012}.

\item Die Bethesda Erklärung (Juni 2003)

Zwei Jahre nach Veröffentlichung der initialen Version der BOAI-Erklärung, im Juni 2003, verabschiedete eine Gruppe von Forschungsförderern, wissenschaftlicher Gesellschaften, Verlegern, Bibliothekaren, Forschungseinrichtungen und einzelner Wissenschaftler im US-Bundesstaat Maryland das "Bethesda Statement on Open Access Publishing" \cite{suchen}. Ziel der Erklärung war die Stimulation der Diskussion in der biomedizinischen Forschung, "wie man schnellstmöglich den offenen Zugang zu der primären wissenschaftlichen Literatur in der Biomedizin erreichen könnte"\cite{suchen}. Ähnlich wie in der BOAI benannten die Autoren des "Bethesda Statements on Open Access Publishing" die Bedingungen für den offenen Zugang zu wissenschaftlichen Publikationen \cite{suchen}:

Erstens werden Autor(en) und Urheberrechts-Inhaber aufgefordert, für alle Benutzer ein freies, unwiderrufliches, weltweites und unbefristetes Recht auf den Zugang zu genehmigen, sowie eine Lizenz zu verwenden, die das Kopieren, Nutzen, Verbreiten, Übertragen und öffentliches Darstellen der Publikation ermöglicht. Darüber hinaus soll es erlaubt sein, abgeleitete Werke zu verteilen und in jedem digitalen Medium für jeden Zweck zu veröffentlichen, vorbehaltlich einer angemessenen Zuordnung der Urheberschaft. Das beinhaltet auch das Recht auf eine kleine Anzahl gedruckter Kopien für den persönlichen Gebrauch.

Zweitens, muss eine vollständige Version der Arbeit und aller ergänzender Materialien, einschließlich einer Kopie der Genehmigung, wie oben erwähnt, in einem geeigneten elektronischen Standardformat unmittelbar bei der ersten Veröffentlichung in mindestens einem Online-Repositorium, das von einer wissenschaftlichen Einrichtung unterstützt wird, hinterlegt werden. Dieses Repositorium muss von einer wissenschaftlichen Gesellschaft, Regierungsbehörde oder einer anderen etablierten Organisation akzeptiert sein. Diese muss sich für einen offenen Zugang, uneingeschränkte Verbreitung sowie Interoperabilität und Langzeitarchivierung (für die biomedizinischen Wissenschaften, PubMed Central ist ein solches Repository) verpflichtend einsetzen.

\item Die Berliner Erklärung (Oktober 2003)

Einen weiteren Meilenstein für die Verbreitung der Idee von Open Access auf dem europäischen Kontinent stellten die "Berlin Konferenzen"\cite{CREATe_2014} dar. Die erste Tagung wurde 2003 von der Max-Planck-Gesellschaft und dem Projekt European Cultural Heritage Online (ECHO) organisiert, um über "Zugangsmöglichkeiten zu Forschungsergebnissen" zu diskutieren. In diesem Rahmen entstand 2003 auch die "Berliner Erklärung über den offenen Zugang zu wissenschaftlichem Wissen" \cite{berliner_erklaerung_2003}, in der die Verfasser über die Budapester Erklärung hinaus gehen und neben dem kostenlosen und freien Zugang zu wissenschaftlichen Endergebnissen in Form von Publikationen auch den freien und offenen Zugang zu wissenschaftlichen Daten fordern. „Open Access-Veröffentlichungen umfassen originäre wissenschaftliche Forschungsergebnisse ebenso wie Ursprungsdaten, Metadaten, Quellenmaterial, digitale Darstellungen von Bild- und Graphik-Material und wissenschaftliches Material in multimedialer Form.“ \cite{berliner_erklaerung_2003}

Mit dieser Ausweitung der Erklärung auf die Daten hinter den Publikationen, formiert sich erstmals ein erweitertes Verständnis von Open Access. Damit entsteht die Grundlage für ein erste Ansatzpunkte zur Definition von Open Science. Die Diskussion um die Berliner Konferenzen konzentrieren sich in diesem Stadium aber dennoch hauptsächlich auf den bereits abgeschlossenen wissenschaftlichen Prozess.
\end{enumerate}

Alle drei Erklärungen, auch die "three B's" \cite{suber_2004_praising_oa} genannt, gelten als die anerkanntesten Definitionen von Open Access und stimmen in den wesentlichsten Merkmalen überein\cite{albert_2006_open_implications}. Sie alle eint vor allem die Kernforderung nach der Beseitigung der preislichen und in Teilen der rechtlichen Barrieren bezüglich des freien Zugangs zu den wissenschaftlichen Publikationen überein. Trotz einiger Unterschiede im Detail ähneln sich die Definitionen auch bei geforderten Beseitigung der Barrieren für die kommerzielle Nutzung und die Erstellung von Derivaten \cite{CREATe_2014}.

Im Jahr 2003 entstand das Portal Directory of Open Access Journals (DOAJ), das bis zum Jahr 2013 von der schwedischen Universität Lund betrieben wurde. Das Portal stellt eine zentrale Anlaufstelle für Open Access-Journale dar \cite{suchen-Hoffmann-Zugang-undVerwertung-öffentlicher-Informationen}. Dem DOAJ folgte 9 Jahre später mit dem Directory of Open Access Books (DOAB) ein Portal für Open Access Bücher und Monografien.

Vorangegangen war im Jahr 2002 die Entwicklung und Veröffentlichung der Creative Commons Lizenzen \cite{suchen-Hoffmann-Zugang-undVerwertung-öffentlicher-Informationen}. Diese Lizenzen waren inspiriert von den Lizenzen der freien Softwarebewegung und wurden kostenlos zur Verfügung gestellt. Sie ermöglichten das freie Lizenzieren von Werke für bestimmte Verwendungen, unter bestimmten Bedingungen; oder ermöglichten die gemeinfreie Nutzung ohne Einschränkungen. Die Creative Commons Lizenzen bilden bis heute die urheberrechtliche Grundlage für eine Vielzahl der Open Access Publikationen weltweit \cite{suchen}. Die modularen Lizenzen sind im Kontext von Open Access besonders wichtig, "um (Nach-)Nutzungsmöglichkeiten für Texte, Daten und andere wissenschaftliche Erzeugnisse festlegen zu können" \cite{suchen-Hoffmann-Zugang-undVerwertung-öffentlicher-Informationen}.

Die Deutsche Forschungsgemeinschaft (DFG) reagierte Anfang 2006 und verabschiedete eine Richtlinie nach der sie zwar nicht voraussetzt, aber "erwartet", dass Publikationen aus DFG-geförderten Projekten "möglichst" als Open Access veröffentlicht werden \cite{suchen:dfg-richtlinie}. Eine ähnliche Erklärung verabschiedete auch die größte amerikanische Förderinstitution National Institutes of Health (NIH) und "stellte mit PubMed Central (PMC) eine entsprechende Plattform bereit \cite{muller_2010_open}. In 2008 wurde die Veröffentlichung NIH-geförderter Publikationen verpflichtend \cite{Hanekop_2014}. Aktuell gibt es in Deutschland ist keine zentrale Plattform wie PMC und die Veröffentlichug der geförderten Ergbnisse als Open Access ist weiterhin nicht bindend.

In der Debatte über die Zukunft des wissenschaftlichen Publizierens und Kommunizierens besteht die Tendenz, Konzepte der offenen Wissenschaft als einen bisher beispiellosen und noch nie dagewesenen Wandel darzustellen \cite{cite:17a} \cite{cite:17b}. Diese Haltung basiert auf "verschiedenen Gründungsmythen", die auf "unterschiedliche Zielsetzungen und Lösungspfade" verweisen \cite{suchen-Hoffmann-Zugang-undVerwertung-öffentlicher-Informationen}. Die Geschichte von Open Access ist eine Entwicklung, die eng mit der Digitalisierung von Kommunikationsprozessen \cite{albert_2006_open_implications} auf der einen, sowie mit der Zeitschriftenkrise auf der anderen Seite verknüpft ist \cite{suchen-Hoffmann-Zugang-undVerwertung-öffentlicher-Informationen}. Open Access ist kein Selbstzweck\cite{cite:17d}, sondern ein Attribut tiefergehender Prozesse, die mit der wachsenden Bedeutung der Digitalisierung in unserer Zivilisation und dem damit einhergehenden Wandlungsprozessen im Machtgefüge zusammenhängen\cite{cite:17e}. Es bleibt jedoch herauszuheben, dass es trotz vereinzelter Versuche, wissenschaftliche Informationen und Publikationen offen und frei zu kommunizieren, Open Access im Printzeitalter physisch und ökonomisch über lokale Grenzen hinaus schwer möglich war\cite{cite:18a}.

Die Forderung nach der Öffnung von Wissenschaft und Forschung ist in diesem Zusammenhang nicht nur eine "politische Reaktion" oder "technische Alternative", sondern eine "alternative Formatierungen einer wissenschaftlichen Infrastruktur im technischen, rechtlichen und zeitlichen Sinne" \cite{kelty_2004}. Sie betrifft "Wissenschaftler, politische Entscheidungsträger und die Öffentlichkeit" \cite{Scheliga_2014}.

\subsubsection{Von Open Access zu Open Science}

Im April 2012 wurde die Erklärung "Open Science for the 21st century" vom Zusammenschluss der Europäischen Akademien (ALLEA) verabschiedet \cite{ALLEA_2012}. Sie war nur eine von mehreren Erklärungen und Positionspapieren für die Öffnung von Wissenschaft durch international angesehenen Einrichtungen, durch die verdeutlich wurde, dass die Forderung nach offenem Umgang mit Wissen und Information im wissenschaftlichen Bereich zunehmend an Relevanz gewinnt \cite{schulze_2013_open}.

---- TODO: Weitere Entwicklung darstellen ----

\section{Wissenschaftliche Reputation}

Wissenschaftliche Reputation kann als eine "Art von Kredit" \cite{luhmann_1970_selbststeuerung}. Dieser Kredit basiert auf der "gegenseitigen Beurteilung und Anerkennung der jeweils neuen Ergebnisse der Fachkollegen (Peers) durch die Wissenschaftler selbst"\cite{Hanekop_2014} \cite{suchen_Hornbostel_2006}, teils auf der "Generalisierung von Einzelleistungen", auf "gegenseitiger Ansteckung" und teils "auf der bloßen Häufigkeit der Publikationen oder der Anwesenheit an renommierten Plätzen" \cite{luhmann_1970_selbststeuerung}. Dabei gesteht auch Luhmann die Existenz von "Nebencodes der Reputation" zu \cite{schmoch_2003_hochschulforschung}. Die Reputation steuert die wissenschaftliche Aufmerksamkeit und die Verteilung motivierender Effekten, die sich durch das reine Streben nach Erkenntnis nicht erzeugen lassen \cite{suchen_luhmann}. Die akademische Reputation „ist die zentrale Kommunikationsform, die das Wissenschaftssystem charakterisiert“ \cite{Rutenfranz_1997}. Die Ergebnisse aus wissenschaftlicher Forschung werden dabei als Publikationen vor allen Mitgliedern der Wissenschaft präsentiert, „um diese intern von der Wissenschaftsgemeinde als wissenschaftlich beziehungsweise unwissenschaftlich zertifizieren zu lassen" \cite{Rutenfranz_1997}.

Wissenschaftliche Reputation wird in diesem Zusammenhang auch als Währung bezeichnet, mittels derer “Status und Ressourcen verteilt werden” \cite{hanekop_2006}. Sie verteilt sich auf Einrichtungen und einzelne Personen, die wissenschaftlich tätig sind \cite{suchen}. Die Evaluation wissenschaftlicher Einrichtungen findet dabei über “Beobachtungen und Gespräche mit den Wissenschaftlern vor Ort sowie über den Austausch über die Eindrücke innerhalb der Begehungsgruppe und die gemeinsame Verständigung”\cite{Barl_sius_2008} statt.

Als "guter akademischer Forscher" gilt nur "wer viel und in möglichst angesehenen Journalen" \cite{Frey_2005} oder wissenschaftlichen Buchverlagen veröffentlicht. Dabei spielt der Peer-Review-Prozess eine zentrale Rolle und ist das Kernelement der Selbststeuerung von Wissenschaft \cite{Neidhardt_2010}. In dem Peer-Review Prozess "werden eingereichte Beiträge von fachlich versierten Wissenschaftlern (...) beurteilt und gemäß der qualitativen Anforderungen der Forschungs-Community zur Veröffentlichung angenommen oder abgelehnt" \cite{Hess_2006}. Der Prozess gilt als "Herzstück einer autonomen, selbstverwalteten Wissenschaft" \cite{suchen_Hornbostel_2006} und beschränkt sich nicht nur auf den Prozess der Publikation von Texten, sondern deckt ein breites Spektrum von Aktivitäten über die Fachdisziplinen hinaus ab \cite{Lee_2012}:
\begin{itemize}
\item die Beobachtung der klinischen Praxis (z.B. in der Medizin)
\item Beurteilung des Lehrenden oder der Fähigkeiten der Kollegen
\item Bewertung bei der Forschungsförderung und Stipendien bei Einreichung von Anträgen an staatliche und anderen Förderorganisationen
\item Begutachtung bei Artikeleinreichungen für wissenschaftlichen Zeitschriften
\item Bewertung von Papieren und Plakaten für Konferenzen
\item Bewertung von Buchvorschlägen für Universitätsverlage oder andere Verlage
\itemEinschätzungen der Qualität, Anwendbarkeit und Interpretierbarkeit von Datensätzen und wissenschaftlichen Organisationen
\end{itemize}

Publikationen bilden im Hinblick auf die Funktion der Reputationsverteilung "eine Art Telos wissenschaftlicher Kommunikation" \cite{hirschauer2004peer}. In Bezug auf die Erlangung von Reputation ist wissenschaftliche Arbeit besonders auf ein funktionierendes Peer-Review-System angewiesen \cite{suchen}. Das Verfahren hat zwei Funktionen: Erstens die Selektionsfunktion, in deren Rahmen die Auswahl von Personen, Projekten und Texten stattfindet und zweitens eine Konstruktionsfunktion, in der Gutachter "produktiv in den Wissenschaftsprozess eingreifen" \cite{Neidhardt_2010} und die eigenen Fachstandards durchzusetzen. Der Peer Review-Prozess sichert aber nicht nur "Vertrauen" und die Grundlage für die "Anschlusskommunikation" innerhalb der wissenschaftlichen Community, sondern "wirkt überdies auch nach außen und gewährleistet die gesellschaftliche Legitimation des wissenschaftlichen Wissens" \cite{pscheida_2010_wikipedia}.

Das Peer-Review Verfahren ist in der Wissenschaft etabliert und weit verbreitet. Dennoch haben aktuelle qualitative Peer Review-Systeme und quantitative bibliometrische Verfahren objektiv viele Mängel \cite{osterloh2008anreize} \cite{Lee_2012} \cite{Jansen_2007}. Die Mängel lassen sich laut Osterloh und Frey wie folgt zusammenfassen \cite{osterloh2008anreize}: Erstens, die "geringe Reliabilität der Gutachter-Urteile", zweitens die "geringe prognostische Qualität von Gutachten" und drittens das "opportunistisches Verhalten der Gutachter und Editoren, sowie das "opportunistisches Verhalten der Autoren". Zusammenfassend kommen die Autoren zu dem Schluss, dass "die Annahme eines Manuskriptes einem Zufallsprozess gleicht" und das "System der qualitativen Peer Reviews (...) auf einer erstaunlich fragwürdigen wissenschaftlichen Grundlage" beruht \cite{osterloh2008anreize}.

---- TODO: Grafik aus Kritik am Peer-Review bauen http://www.zalf.de/de/forschung/services/pubman/service/Documents/Manuskriptfluss/Mueller_2008_peer_review.pdf ----

Folgende weitere Indikatoren werden in diesem Zusammenhang für die Erlangung von wissenschaftlicher Reputation für wissenschaftliche Institutionen und Personen genannt \cite{hanekop_2008}:
\begin{enumerate}
\item \textbf{Anzahl der wissenschaftlichen Aufsätze / Beiträge}: Die Anzahl der Texte die Wissenschaftler im Rahmen ihrer Tätigkeit publizieren ist ein wesentlicher Faktor der Bewertung wissenschaftlicher Reputation \cite{luhmann_1970_selbststeuerung} \cite{CLAPHAM_2005}. Zum Beispiel erhöht die Anzahl an Texten die Chance durch die wissenschaftliche Community zitiert zu werden und damit die Möglichkeit auf die Erlangung von Reputation. Durch den zunehmenden Wettbewerb in der Wissenschaft muss sich der einzelne Wissenschaftler entscheiden, "zu publizieren oder im wissenschaftlichen System zu scheitern" \cite{Suess_2006}. Dadurch entsteht im wissenschaftlichen Kommunikationssystem ein konstanter Publikationsdruck, bei dem die Relevanz der publizierten Ergebnisse nicht immer im Vordergrund steht \cite{suchen}. Die Anzahl der veröffentlichten Artikel hat einen Einfluss auf die Vergabe von Ressourcen und finanziellen Mittel für weitere Forschung an Institutionen und Individuen\cite{suchen}.
\item \textbf{Relevanz der publizierten Ergebnisse}: Die Relevanz der publizierten Ergebnisse ist für das Wissenschaftssystem ein wesentlicher Treiber für den Prozess der Wissensgewinnung. Relevante Erkenntnisse sind die Grundlage für die Produktion von neuem Wissen und damit Grundlage für den gesellschaftlichen Auftrag des Wissenschaftssystems \cite{suchen}. Das wissenschaftliche System beruht auf der Annahme, dass die Relevanz der publizierten Ergebnisse einen direkten Einfluss auf die wissenschaftliche Reputation hat.
\item \textbf{Anzahl Monografien}: Die Anzahl der veröffentlichten Monographien ist ein wesentlicher Reputationsfaktor. Das gilt für die Disziplinen, in denen diese Publikationsform wichtig ist, wie den Geistes- und Sozialwissenschaften. In den anderen wissenschaftlichen Fachrichtungen spielt die Anzahl der Veröffentlichungen von Artikeln in wissenschaftlichen Journalen eine wichtige Rolle.
\item \textbf{Drittmittelprojekte}: Drittmittel sind, so der deutsche Wissenschaftsrat, "solche Mittel, die zur Förderung der Forschung und Entwicklung sowie des wissenschaftlichen Nachwuchses und der Lehre zusätzlich zum regulären Hochschulhaushalt (Grundausstattung) von öffentlichen oder privaten Stellen eingeworben werden" \cite{wr_2014}. Die Drittmitteleinwerbung hat sich in Deutschland als "meist gebrauchter Maßstab der Messung von Forschungsqualität durchgesetzt" \cite{M_nch_2006}. Diese Entwicklung geht mit einer zunehmenden Finanzierung der Forschung über Drittmittel einher \cite{Neidhardt_2010} \cite{Jansen_2007} \cite{simon_2009_wissenschaft_governance}. Durch die zunehmende Knappheit öffentlicher Ressourcen für Wissenschaft und Forschung, ist die Akquise von Drittmitteln zu einem kritisch zu betrachtenden Kernziel geworden \cite{Jansen_2007}. Das führt, dass zunehmend direkte finanzielle und administrative Kontrolle der Forschung eine Rolle spielen \cite{Barl_sius_2008}. Dabei spielt die Frage eine Rolle, ob die Publikationen, die im Rahmen der Drittmittelfinanzierung als wissenschaftliche Erkenntnisse veröffentlicht werden und ob der Antrag um Drittmitteleinwerbung selbst, "zum Erkenntnisfortschritt in der wissenschaftlichen Gemeinschaft beiträgt" \cite{M_nch_2006}. Die wissenschaftliche Community befürchtet durch die zunehmende Relevanz der Anzahl von Drittmittelprojekten bei der Erlangung von wissenschaftlicher Reputation eine Einschränkung der Freiheit von Wissenschaft und Forschung.
\item \textbf{Patente}: "Unter einem Patent versteht man das vom Staat verliehene Schutzrecht für eine technische Erfindung, welches dem Patentinhaber für eine bestimmte Zeit die ausschließliche wirtschaftliche Nutzung der Erfindung vorbehält." \cite{greif_2003_patente} Die Anzahl dieser Schutzrechte im Hochschulbereich nimmt seit den 1970er konstant zu. \cite{schmoch_2003_hochschulforschung}. Vor allem in den technischen Fachdisziplinen wird eine Patentschrift "als funktionales Äquivalent zur wissenschaftlichen Publikation begriffen" und bewertet \cite{mersch_2014_patente}. Die deutsche Hochschulrektorenkonferenz hält fasst die Rolle des Patentwesen an den Hochschulen wie folgt zusammen:  "Patente leisten einen Beitrag zur Förderung der Wissenschaft, die Grundlagen des Patentwesens sind daher dem wissenschaftlichen Nachwuchs über entsprechende Lehrangebote zu vermitteln." \cite{suchen-Position-HRK}
\item \textbf{Vorträge}: Vorträge dienen der Verbreitung der Forschungserkenntnisse, sowie Zwischenständen und ermöglichen das Vermitteln des Wissens an andere \cite{rassenhoevel_2010_performancemessung}. Vorträge stellen eine informelle und schnelle Form für die Verbreitung neuer wissenschaftlicher Erkenntnisse und Ergebnisse dar. Die in einem Vortrag vermittelten Inhalten müssen meist nicht genauer belegt werden und die kommunizierten Inhalte lassen gegebenenfalls später schriftlich konkretisieren oder korrigieren \cite{haberle_2002_jahrbuch}. Vorträge bieten die Möglichkeit bereits vor der eigentlichen Publikation von wissenschaftlichen Erkenntnissen Anregungen und Reaktion einzuholen.
\item\textbf{Anwendungsrelevanz bzw. Verwertbarkeit}: Ein vergleichsweise neuer Indikator die Reputation von Hochschulen und außeruniversitäre Forschungsinstitute ist die Anwendungsrelevanz von Wissenschaft und Forschung \cite{simon_2009_wissenschaft_governance}. Sie bezieht sich auf einen Outputfaktor, der sich primär auf den Einsatz der gewonnenen wissenschaftlichen Erkenntnisse und auf die Verwertbarkeit für wirtschaftliche Produkte oder Patente als auf die eigentliche wissenschaftliche Veröffentlichung abzielt \cite{suchen}.
\item \textbf{Netzwerke und Kontakte}: Netzwerke beschreiben formelle und informelle Verbundsysteme zwischen Wissenschaftlern. Sie erlauben den schnellen Austausch und können Grundlage für Aktivitäten zur Steigerung der wissenschaftlichen Reputation darstellen. Diese Aktivitäten umfassen zum Beispiel gemeinsame Publikationsvorhaben und den Austausch wissenschaftlicher Erkenntnisse. Kontakte und Netzwerke schaffen soziale Beziehungen, die für eine erfolgreiche Integration an der Hochschule und der Fachcommunity sorgen, Zugang zu wissenschaftlicher Kommunikation ermöglichen und somit einen Einfluss auf die Anerkennung eines Wissenschaftler oder einer Wissenschaftlerin haben können.
\item \textbf{Öffentliche Aufmerksamkeit}: Die öffentliche Aufmerksamkeit stellt zum einen eine Möglichkeit des Wissenstransfers außerhalb der wissenschaftlichen (Fach-)Community dar, zum Anderen ermöglicht sie die Einflussnahme auf die politische Relevanz wissenschaftlicher Forschungsthemen. Die Veröffentlichung von wissenschaftlichen Informationen zu einem bestimmten Thema des öffentlichen Interesses stellt eine Möglichkeit dar, dieses Thema öffentlichkeitswirksam zu katalysieren. Öffentliche Aufmerksamkeit im Rahmen von wissenschaftlicher Tätigkeit stellen eine kritisch zu hinterfragende Möglichkeit für die alternative Ressourcengewinnung dar. \cite{suche}
\item \textbf{Politische Relevanz}: Die wissenschaftliche Tätigkeit mit politischer Relevanz stellt eine weitere Möglichkeit dar, wissenschaftliche Inhalte ausserhalb der Wissenschaft anwendbar zu machen und führt zu Anerkennung der wissenschaftlichen Arbeit. Daraus ergeben sich allerdings grundsätzliche "Verständigungsprobleme und Interessenkonflikte", da  "Wissenschaft und Politik aufgrund unterschiedlicher Rationalitäten handeln, einander aber zugleich brauchen" \cite{Mayntz_1996}. Während es im Wissenschaftssystem "um Erwerb und Erhalt von Wissen" geht, zielt die Politik auf "Erwerb und Erhalt von Macht" \cite{Mayntz_1996} ab. Die daraus resultierenden Interessenkonflikte führen zu "gegenseitigen Enttäuschungen", vor allem in der "forschungspolitischen Beziehung" \cite{Mayntz_1996}.
\item \textbf{Renommee der Forschungseinrichtung}: Das Renommee einer Forschungseinrichtung ist die Wahrnehmung der Einrichtung innerhalb und außerhalb der wissenschaftlichen (Fach-)Community. Sie hat für Wissenschaftler und die Wissenschaftlerin eine besondere Bedeutung \cite{mayntz_2008_wissensproduktion}. Sie basiert auf dem Konzept der "Ansteckung" \cite{luhmann_1970_selbststeuerung}. Diese Ansteckung führt zum Beispiel dazu, dass renommierte Professoren den Ruf einer Fakultät und eine renommierte Fakultät auch den Ruf von Professoren aufbessern können. Übertragen auf das wissenschaftliche Publizieren profitiert ein Autor oder eine Autorin bei der "Ansteckung" von dem Renommee einer Einrichtung, wenn er durch die Publikationsorgane der renommierten Institution veröffentlicht \cite{lutz_2012_zugang}.
\item \textbf{Renommee von Herausgebern oder Mitautoren} Der Herausgeber organisiert den Begutachtungsprozess und sichert bestimmte Qualitätskriterien mit seiner Reputation und seinem Namen \cite{mueller_2009_peerreview}. Auch hier kommt es im Rahmen des symbolischen wissenschaftlichen Kapitals zu einer Übertragung der Reputation der Herausgeber oder Mitautoren auf die anderen veröffentlichenden Autoren.
\item \textbf{personelle und materielle Ausstattung}: Die materielle Ausstattung beschreibt die Rahmenbedingungen, in der ein Wissenschaftler arbeitet. Diese Rahmenbedingungen haben eine herausragende Bedeutung bei der Entscheidung über einen Wirkungsort von Wissenschaftlern \cite{mayntz_2008_wissensproduktion}. Insbesondere die materielle und personelle Ausstattung sind bei traditionellen Berufungsverfahren deutscher Professorinnen und Professoren von besonderem Belang \cite{himpele_2011_job}, da sie die Arbeitsfähigkeit und die Anerkennung direkt beeinflussen \cite{suche}. Wie die materielle Ausstattung gilt auch die personelle Ausstattung als ein reputationstiftendes Merkmal für Wissenschaftler und die Institution, an denen sie arbeiten \cite{mayntz_2008_wissensproduktion}. Bei der Ausstattung handelt es sich um einen bilateralen Indikator, der zum einen aus der Bewertung der wissenschaftlichen Arbeit (im Rahmen der Forschungsförderung) resultiert \cite{Herb_vermessung_2008} und  zum anderen Reputation innerhalb der Community schafft \cite{mayntz_2008_wissensproduktion}.
\item \textbf{Gutachtertätigkeit und Herausgeberschaft}: Gutachter werden zum Beispiel in Peer-Review-Verfahren Autoren des entsprechenden Fachgebietes zugeordnet und entscheiden über die Veröffentlichung des Textes \cite{Frey_2005}. Bei manchen Publikationen wird ein Text mehrmals abgelehnt und eine Überarbeitung durch den Autoren eingefordert, bevor der Artikel final akzeptiert und daraufhin publiziert wird \cite{Frey_2005}. In diesem Zusammenhang wirkt sich die Reputation der mit diesem Verfahren betrauten Gutachter auch auf das Image des Verlages aus und umgekehrt. Die Gutachtertätigkeit ist aber nicht nur Kernbestandteil des wissenschaftlichen Qualitätssicherungs- und interdependenten Reputationssystems, sondern stellt auch einen informellen Weg der Kommunikation dar. Er ermöglicht den Gutachtern die Vorabsichtung neuster wissenschaftlicher Informationen und Erkenntnisse. Ähnlich wie die Gutachtertätigkeit ist auch die Herausgeberschaft fester Bestandteil des interdependenten wissenschaftlichen Reputationssystems \cite{Frey_2005}: Herausgeber profitieren von den publizierten Inhalten und Erkenntnissen der Autoren, Autoren von der Reputation Herausgebern und der Verlag von beiden \cite{suchen}.
\item \textbf{Funktion}: Die jeweilige Funktion oder die (universitäre) Stellenbezeichnung ist ein weiter Faktor für wissenschaftliche Reputation. Zum wissenschaftlichen Personal zählen Professoren, Juniorprofessoren, wissenschaftliche und künstlerische Mitarbeiter, sowie Lehrkräfte \cite{erhardt_2011_hochschulen}. Eine Weiterentwicklung und der "Aufstieg" in der wissenschaftlichen Hierarchie zielt auf das akademische Streben nach einer Professur \cite{Klecha_2008}.
\item \textbf{Awards und Preise}: Preise sind ein weitere Indikator für das wissenschaftliche Belohnungs- und Bewertungssystem. "Die Praxis der Award-Verleihung beruht auf dem Konzept, dass Ressourcen von unabhängigen Dritten auf Qualität geprüft und (...) zertifiziert werden" \cite{bargheer_2002_qualitatskriterien}. Wissenschaftler und Wissenschaftlerinnen, die Preise oder Awards gewinnen, erfahren Anerkennung. Diese Anerkennung können jedoch nicht automatisch als "Garant für wissenschaftsrelevante Qualität"\cite{bargheer_2002_qualitatskriterien} verstanden werden. Die Ehrung mit einem Preis weckt große Erwartungen und führt zu dem Anspruch eines stetigen Nachschubs an Anerkennung für den Wissenschftler oder die Wissenschaftlerin \cite{suchen}.
\end{enumerate}

Die Reputation einzelner Wissenschaftler steht in enger Abhängigkeit zum bestehenden wissenschaftlichen Kommunikationssystem \cite{suchen}. Anstatt finanzieller Entlohnung, wird in der Wissenschaft primär mit Aufmerksamkeit "bezahlt" \cite{suchen}. Vereinfacht lässt sich das System der Wechselbeziehungen der Reputationsverteilung im Rahmen von Publikationen wie folgt darstellen \cite{cite:21a}:

Grafik aus Text von Bernius
http://www.eap-journal.com/archive/v39_i1_8_bernius.pdf

Bernius et al. unterscheiden drei aufeinandertreffende koordinierende Marktmechanismen: die Reputation, die Nutzung wissenschaftlicher Publikationen, sowie den Preis für den Erwerb der Publikation \cite{suchen}. Während die Reputation ein non-monetärer Aushandlungsmechanismus zwischen wissenschaftlichen Verlagen und wissenschaftlichen Autoren ist, findet die monetäre Preisdefinition zwischen Bibliotheken und Verlagen statt. Der monetäre Aushandlungsprozess zwischen Wissenschaftlern und Bibliotheken wird duch die Bedeutung und Nutzung der jeweiligen Publikation bestimmt \cite{cite:21a}. Nicht jede Publikation hat diesbezüglich die gleiche Wertigkeit \cite{suchen} und damit den gleichen Einfluss auf die Reputation eines Autors.

Die allgemeine Verfügbarkeitmachung von Forschungsergebnissen stellt dabei einen integralen Bestandteil des wissenschaftlichen Ethos dar \cite{Fangerau_2014}. Der US-amerikanische Soziologe Robert K. Merton stellte diesen und weitere Grundprinzipien als normative Grundstruktur des Ethos von Wissenschaft vor \cite{Merton_1985}. Er hält "Anspruchlosigkeit und Bescheideheit für Grundtugenden" \cite{hagner_2015_sache_buches} des modernen Wissenschaftlers. Der Ethos wird in diesem Zusammenhang als "Komplex von Werten und Normen"\cite{suchen} beziehungsweise "Verhaltensmaßregeln"\cite{suchen} verstanden. Merton unterteilt die Kriterien in die Kategorien:
\begin{itemize}
\item Universalismus: Die sozialen Merkmale eines Wissenschaftlers, wie zum Beispiel Nationalität, Geschlecht, Religion, Klasse usw. dürfen nicht in die Evaluation wissenschaftlicher Ergebnisse einfließen \cite{suchen}.
\item Kommunismus (Kommunalität) - Es gibt eine Pflicht zur Veröffentlichung der Ergebnisse von Wissenschaft und Forschung und sie sind als Allgemeingut zu betrachten. Die wissenschaftliche Anerkennung und das Ansehen sind einziges damit verbundenes "Besitzrecht"\cite{suchen}.
\item Uneigennützigkeit: Intrinsische "Neugier"\cite{suchen}, "selbstloses Eintreten für das Wohl der Menschheit"\cite{suchen} und der Wissensdurst müssen die vornehmlichen Motivatoren für Wissenschaftler darstellen \cite{suchen}.
\item Objektivität und Desinteresse: Eines der weiteren Kriterien erfordert "Objektivität und Desinteresse" an den Ergebnissen der eigenen Forschung \cite{suchen} unabhängig von finanziellem Erfolg und Prestige \cite{suchen}.
\item Organisierter Skeptizismus: Zweifel muss als "grundsätzliches Denkprinzip der Wissenschaft" \cite{suchen} und die "unvoreingenommene Prüfung und Kritik an Wissenschaft, Forschung und Autorität" \cite{suchen} verstanden werden. Dabei gilt es auch den "Matthew Effect" zu vermeiden. Der Matthäus-Effekt ist ein Phänomen auf der Makroebene der Wissenschaft \cite{bonitz_1998_matthaus}. Der „Matthäus-Effekt" ("Wer hat, dem wird gegeben" Mt. 25,29) beschreibt den Umstand, dass Autoren oder Publikationen, die bereits eine hohe Zitationsrate vorweisen können, meist noch häufiger zitiert werden als die Autoren oder Beiträge mit einer geringeren Zitationsrate. Überproportional profitieren in diesem System also die, die besonders häufig zitiert wurden \cite{Merton_1968} \cite{meier_2009_matthaus}.
\end{itemize}

Als Folge dieser Kriterien erkannte Merton das Urheberrecht an wissenschaftlichen Ideen und Beiträgen an, allerdings nur insofern, dass das Urheberrecht allein auf die Ermöglichung der Anerkennung durch Kollegen und die Achtung der Priorität beschränkt bleibt \cite{Fangerau_2014}. Zusammenfassend lassen die neuen Möglichkeiten der Verbreitung von Informationen einen vergleichbaren Veränderungsprozess der wissenschaftlichen Reputation und damit auch Anerkennung vermuten, wie er durch die Entwicklung des Buchdrucks ausgelöst worden war \cite{hanekop_2006}.

\subsection{Messbarkeit wissenschaftlicher Qualität und Publikationsquantität}
Wissenschaft ist ein Prozess, bei dem aus “unterschiedlichen Inputfaktoren, mittels verschiedener Transformationen Beiträge zur Schaffung neuer wissenschaftlicher Erkenntnisse als Output entstehen” \cite{Jansen_2007}. Die Bewertungen des jeweiligen Outputs führt “zur Aussage über die Forschungsperformanz” \cite{suchen}. Neben den Indikatoren für den Output wissenschaftlicher Perfomanz, müssen aber auch intermediäre Aspekte berücksichtigt werden \cite{schmoch_2009}.

Mit Beginn des 20. Jahrhunderts wurden in der Wissenschaftsforschung Indikatoren überwiegend zur Beschreibung der exponentiellen Wachstumsverläufe von Wissenschaft entwickelt und eingesetzt \cite{Hornbostel_1997}. Nach dem zweiten Weltkrieg etablierten sich erstmals Indikatoren für die Effizienzmessung wissenschaftlicher Wissensproduktion und -verbreitung, die aber "ebenso wie Sozial- und Wirtschaftsindikatoren keine neutralen Realitätsbeschreibungen" \cite{Hornbostel_1997} darstellten. Spätestens seit den 1970er Jahren werden diese Messungen, die die Forschungsleistung quantifizieren sollen, flächendeckend durchgeführt \cite{Hornbostel_1997} um Forschungsqualität und Quantität messbar zu machen.

Seit den 1990er Jahren ist diese Bewertung von Wissenschaft in Gestalt von Zahlen als unkontrollierte Nebenprodukte digitaler Wissenskommunikation erweitert worden \cite{angermueller_2010}. Heute zählen in der Wissenschaft vor allem die wissenschaftliche Reputation und die als "Impact" bezeichnete Wirkung wissenschaftlicher Publikationen\cite{herb_open_2013} \cite{Hornbostel_1997}. Die Wirkung wird anhand der quantitativen Betrachtung der Zitationen der jeweiligen Publikation ermittelt \cite{suchen}.

Diese Art der Betrachtung basiert auf der Grundannahme, dass Kommunikation die "Essenz der Wissenschaft"\cite{bonitz_1998_matthaus} ist und "Zitierungen in ihrer Gesamtheit so etwas, wie die Grundelemente eines weltweiten Expertensystems" \cite{bonitz_1990_sci}. Nach dieser Sichtweise stellt eine häufige Zitation einen wesentlichen Indikator für die Wirkung der wissenschaftlichen Arbeit dar. Ein generalisierter und überzeitlicher Begriff von Qualität wissenschaftlicher Arbeit scheint nicht möglich, weil eine grundlegende Definition der Wissenschaftsindikatoren sowie ihrem Ziel der "Abbildung eines Konstruktes, das die Bewertungen einzelner Wissenschaftler oder Experten transzendiert" nicht möglich erscheint \cite{Hornbostel_1997}.

In den letzten Jahren haben sich neue Möglichkeiten für die Qualitätssicherung und -bewertung herausgebildet. Die "Anforderungen an Verfügbarkeit von Dokumenten und Transparenz der Begutachtungen" der Open Access Bewegung haben die Frage aufgebracht, "ob möglicherweise Veränderungen der Review-Praktiken notwendig sind, um exzellente Wissenschaft zu identifizieren und vor allem zu fördern" \cite{suchen_Hornbostel_2006}. Des Weiteren stellt sich die Frage, ob die Berücksichtigung neuer Metriken für die Bewertung von wissenschaftlicher Qualität eine Antwort auf die Herausforderungen mit den etablierten Messsystem von wissenschaftlicher Qualität und Publikationsquantität sein können.

Bestand die klassische Wirkungsmessung von Wissenschaft in der Ermittlung der Anzahl von Zitationen, ermöglichen die veränderten Bedingungen von wissenschaftlicher Kommunikation im Rahmen der Digitalisierung alternative Erhebungsmöglichkeiten der Wirkung von formeller wissenschaftlicher Kommunikation und damit auch für die Erlangung wissenschaftlicher Reputation. Ergänzend zu den etablierten zitationsbasierten Metriken spielen zunehmend detailliertere Analyse von nutzungsbasierten Metriken bei der Bewertung von Publikationen eine Rolle. Die Befürworter solcher alternativer Metriken erhoffen sich von diesen neuen Verfahren eine unmittelbare und detailliertere Impact-Messung wissenschaftlicher Kommunikation und eine gerechtere Verteilung von wissenschaftlicher Reputation \cite{suchen_ggf_herb}.

\subsection{Wissenschaftliches Kapital}

Die Wissenschaft ist ein soziales Feld, dessen Strukturen und Praktiken das bestimmen, was als Wissenschaft und als wissenschaftliches Ergebnis gilt \cite{mikl_2010_soziologie}. Im Rahmen der Betrachtung von Steuerungs- und Reputationsmethoden für die Wissenschaft ist der Begriff "wissenschaftliches Kapital" von herausragender Bedeutung \cite{suchen}. Wissenschaftliches Kapital kann als eine Ausprägung des kulturellen Kapitals und als symbolisches, beziehungsweise non-monetäres Kapital \cite{irmer2011} verstanden werden. Symbolisches Kapitel wird von der Sozilogin Mikl-Horke als Besitz an symbolischen Gütern beschrieben, "der besonders in einer Gesellschaft, die auf die Kooperation aller angewiesen ist, sehr kostbar ist"\cite{mikl_2010_soziologie}. Eine genauere Betrachtung des wissenschaftlichen Kapitals ist für das Verständnis der Motivation von Wissenschaftlern zu publizieren und zu kommunizieren, sowie für die Herausarbeitung der Treiber und Bremser für die Öffnung wissenschaftlicher Kommunikation unabdingbar.

Die "Gewährung wissenschaftlichen Kapitals" basiert heute auf der Kooperation zwischen publizierenden Wissenschaftlern und Verlagen \cite{herb_2006}. Die Wissenschaftler befinden sich in einer Abhängigkeit zu den Verlagen. Diese Abhängigkeit wird auch als "Faustischer Pakt" bezeichnet und hinterfragt \cite{hagner_2015_sache_buches}. Diesen Pakt sind Wissenschaftler notgedrungen eingangen, "um den Preis, dass Barrieren zwischen Autoren und Lesern aufgebaut wurden" \cite{hagner_2015_sache_buches}. "Wissenschaftliches Kapital" kann in diesem Zusammenhang als “Ergebnis einer Investition (...), die sich auszahlen muss” \cite{herb_2006} definiert werden. “Diejenigen, die diese Berechtigungsscheine in der Hand halten, verteidigen ihr 'Kapital' und ihre 'Profite', indem sie diejenigen Institutionen verteidigen, die ihnen dieses 'Kapital' garantieren.” \cite{Bourdieu_1992}

Der Soziologe Bordieu unterscheidet zwei Typen von wissenschaftlichen Kapital \cite{Bourdieu_1998}. Das Kapital, das auf der politischen und institutionellen Macht beruht und das andere, dass aus der rein wissenschaftliche Anerkennung resultiert \cite{mikl_2010_soziologie}. Zitationsindexe sind Indikatoren für das wissenschaftliche Kapital, das durch Anerkennung entsteht \cite{Bourdieu_1998}. Die wissenschaftliche Reputation, die aus dem wissenschaftlichen Kapital resultiert, basiert auf der Liste der Publikationen in hoch gerankten Journalen und angesehenen Verlagen \cite{herb_2010}. Diese Bewertung ist symbolischer Natur und basiert "auf der Anerkennung und dem Kredit (...), den die Gesamtheit der Wettbewerber innerhalb des wissenschaftlichen Feldes gewähren" \cite{Bourdieu_1998} \cite{herb_2010}.

Das wissenschaftliches Kapitel ist zunehmend der Kapitalisierung von Wissenschaft ausgesetzt, bei der um den Einfluss der Ökonomie und den "wissenschaftswidrigen Verwertungsdruck". \cite{suchen_Hornbostel_2006} Als ein Indikator dafür ist die Kopplung des wissenschaftliches Kapitals und an die output-orientierte Anreizsysteme zu verstehen. Ein Beispiel ist die zunehmende Relevanz des Performanzindikators "Drittmittel" \cite{Jansen_2007}, bei dem neben der Sicherung der Qualität von Forschung und Lehre zunehmend direkte finanzielle und administrative Kontrolle eine Rolle spielt \cite{Barl_sius_2008}. Dem Drittmitteleinkommen wird als Indikator für Forschungsleistung eine hohe Bedeutung zugemessen \cite{Jansen_2007}. Daraus entsteht die Tendenz, das nicht nur die Erwartungen an die Bewertung von Wissenschaft sehr ambitioniert sind, sondern auch, dass die Interessen privater und öffentlicher Drittmittel-Auftraggeber in den Vordergrund rücken und die Unabhängigkeit von Wissenschaft und Forschung gefährden. Ähnliches ist im Rahmen der leistungsbezogenen Mittelzuweisungen an die Universitäten zu beobachten \cite{suchen_Hornbostel_2006}. Vor allem die Verknüpfung von wissenschaftlicher Reputation mit der damit einhergehenden Verteilung von Mitteln und Stellen stellt eine neuartige Herausforderung an das Wissenschaftssystem dar, dessen Währung [ursprünglich] nicht Geld war \cite{hanekop_2006} \cite{suchen_Hornbostel_2006}.

Die Öffnung wissenschaftlicher Kommunikation folgt bisher nicht der wissenschaftlichen Logik, sondern basiert auf einer "feldunabhängigen Logik der Akkumulation von Kapital" \cite{herb_2006}. Insbesondere das deutsche Wissenschaftssystem ist dabei zunehmend von der Einführung an output-orientierter Anreizsysteme \cite{osterloh2008anreize} und einem Ungleichgewicht in der Kooperation zwischen wissenschaftlicher Kommunikation und wissenschaftlichen Kapital gekennzeichnet. Diese Entwicklung wird bei der weiteren Betrachtung der Motivationsfaktoren für Prozesse wissenschaftlicher Kommunikation eine wichtige Rolle spielen.

\subsection{Freiheit von Wissenschaft, Lehre und Forschung}

"Die Autonomie der Wissenschaft wird nach außen durch die Abhängigkeit der Universität vom Staat und universitätsintern durch die Einheit von Wissenschaft und Forschung gesichert" \cite{Huber_2005}. Diese Wahrung ist im Artikel 5 Absatz 3 Grundgesetz als garantiertes Grundrecht wie folgt festgehalten: "Wissenschaft, Forschung und Lehre sind frei" \cite{suchen_GG}. Dieses Recht ist nicht nur ein Grundrecht auf wissenschaftliche Meinungsfreiheit, sondern auch eine rechtliche Garantie. "Jeder, der in Wissenschaft, Forschung und Lehre tätig ist, hat - vorbehaltlich der Treuepflicht gemäß Art. 5 Absatz 3 Satz 2 GG - ein Recht auf Abwehr jeder staatlichen Einwirkung auf den Prozess der Gewinnung und Vermittlung wissenschaftlicher Erkenntnisse" \cite{suchen_BVG}."Das garantiert einerseits die Einrichtung wissenschaftlicher Hochschulen mit Anspruch auf Selbstverwaltung, die staatliche Finanzierung und Sicherung ihrer Arbeit"\cite{suchen_BVG}, andererseits richtet es sich als "Abwehrrecht auf die Abwehr von Eingriffen in die wissenschaftliche Betätigung" gegen staatliche Eingriffe \cite{mayen_grundrechte_forscher} \cite{spindler_2006_rechtloa}. Jede Form der wissenschaftlichen Betätigung ist durch dieses Abwehrrecht geschützt. Dazu zählen laut Urteil des Bundesverfassungsgerichts "vor allem die auf wissenschaftlichen Eigengesetzlichkeiten beruhenden Prozesse, Verhaltensweisen und Entscheidungen bei dem Auffinden von Erkenntnissen, ihrer Deutung und Weitergabe" \cite{suchen}. Das betrifft auch die freie Entscheidung über die Art und Weise der Veröffentlichung von Forschungsergebnissen (Publikationsfreiheit) \cite{Fangerau_2014}.

In der praktischen Auslegung dieser Freiheit, wird allerdings von einer "Entmythologisierung" der Humboldt’schen Idee der "Einheit von Forschung und Lehre" in der Universität gesprochen. Diese hat jedoch nicht erst mit dem steigenden Kosten- und Effizienzdruck, der "Verwertbarkeit" von Wissenschaft und Forschung, sowie der Modernisierung der Steuerungsmechanismen stattgefunden. Schon Imanuel Kant und Friedrich Nietzsche kritisierten eine Ausrichtung der Universität auf die Verwertbarkeit wissenschaftlichen Wissens \cite{Huber_2005}. Die Idee der Einheit von Forschung und Lehre, auf Grundlage eines völligen Verzichts auf Differenzierung, lässt sich grundsätzlich nur in Ausnahmefällen realisieren \cite{Schimank_2001}. Als realistische Lesart kann nur eine situative Differenzierung stattfinden bei der die Mittel der Grundausstattung nicht nach beiden Aufgaben separiert sind \cite{Schimank_2001}.

Diese Lesart der Humboldt’schen Idee ist noch immer hegemonialer Rahmen der aktuellen Hochschulreformen \cite{Huber_2005}. Das Recht auf Freiheit von Lehre und Forschung und die humboldtsche Idee der Universität wird und wurde immer für die Erhaltung des "organisationellen Status Quo", die Absicherung der "Institution Universität" und die Wahrung der "Staatsunabhängigkeit" angebracht \cite{Huber_2005}. Diese Autonomie der Wissenschaft und Forschung gilt auch heute als "hohes Gut, das es gegen externe Anforderungen zu verteidigen gilt"\cite{kaldewey_2010}.

In Hinblick auf die wissenschaftliche Publikation kann also festgehalten werden, dass Hochschullehrer nicht von der Hochschule oder anderer staatlicher Institutionen gezwungen werden können, über einen bestimmten Weg oder Kanal zu veröffentlichen \cite{spindler_2006_rechtloa}. Ausnahme stellen hier nur die privatfinanzierten Drittmittelprojekte dar, da sich der Hochschullehrer hier nicht auf die Wissenschaftsfreiheit als Abwehrrecht gegen den Staat berufen kann \cite{spindler_2006_rechtloa}. Wissenschaftlichen Mitarbeiter und Mitarbeiterinnen "müssen ihrer Hochschule die Nutzungsrechte an ihrer Publikation einräumen", es sein denn, sie haben sie nicht nach Weisung des Lehrstuhl- oder Institutsleiters erarbeitet oder es handelt sich um eine Dissertation oder Habilitation \cite{spindler_2006_rechtloa}. Ein direkter staatlicher Eingriff im Rahmen einer Richtlinie zum Publikationszwang über einen bestimmten Weg scheint mit der Wissenschafts- und Publikationsfreiheit nicht vereinbar. Dennoch kann der Staat Anreizsysteme oder Rahmenbedingungen zu schaffen, die die Öffnung des wissenschaftlichen Kommunikations- und Publikationssystems befördern.

\subsection{Ökonomie der wissenschaftlichen Kommunikation}
Die klassische Ökonomie der wissenschaftlichen Kommunikation beruht auf der Durchsetzung von Urheberrechten. Diese beschränken den Zugang und Zugriff auf sowie die Wieder- und Weiterverwendung von urheberrechtlich geschützten Inhalten. Leserinnen und Leser können nur gegen die Zahlung einer Gebühr Zugang zu der Veröffentlichung erhalten \cite{CREATe_2014}. Das gilt vor allem für die Veröffentlichung wissenschaftlicher Erkenntnisse. Bislang werden dafür "in der Regel wissenschaftliche Arbeiten zwar mit öffentlichen Mitteln finanziert, aber von privaten Verlagen in Fachzeitschriften herausgegeben" \cite{WD_bundestag_2009}. Diese Ökonomie der Wissenschaftsverlage ist nicht neu und hat sich im Laufe der Zeit weiter ausdifferenziert. Dieses Modell der wissenschaftlichen Publikation basiert auf einer "sozial ineffizienten" System \cite{mueller-langer_2010}. Die Wahrnehmung der Unverhältnismäßigkeit dieses Systems, insbesondere der Preisgestaltung für wissenschaftliche Publikationen \cite{King_2008} findet allerdings erst seit kurzem statt\cite{CREATe_2014}.

Wissenschaftliche Inhalte werden über drei grundlegende Vertriebsarten zur Verfügung gestellt \cite{cope2014future}:
\begin{enumerate}
\item Wissen als Inhalt zum Verkauf: Der größte Anteil wissenschaftlicher Publikation wird über diese Art vertrieben. Allein für die STM-Fächer (Science, Technology, Medicine) wird in der Literatur von einem Markt von 6 Milliarden Dollar für wissenschaftliche Zeitschriften ausgegangen \cite{cope2014future}.
\item Wissen als kostenlose Ressource ---- TODO: weiter ausarbeiten ----
\item Wissen als bei der Produktion bezahlte Ressource ---- TODO: weiter ausarbeiten ----
\end{enumerate}

Das wissenschaftliche Publizieren kann als "gesellschaftlich bedingter Kreislauf" \cite{schirmbacher_2009_wisspub} betrachtet werden. Eine weitere wesentliche Besonderheit der Ökonomie wissenschaftlicher Kommunikation ist die Organisation des Marktes, die von spezifischen Akteuren und Prozessen geprägt wird \cite{Hess_2006}. Diese Ökonomie, ihre Akteure und Prozesse können wie folgt unterteilt werden \cite{cite:11b} \cite{Hess_2006}:
\begin{enumerate}
\item Erstellung von Inhalten durch Wissenschaftler und Wissenschaftlerinnen (Erstellung): Der Kreislauf beginnt mit der Anfertigung der geistigen Werke durch die Autoren \cite{schirmbacher_2009_wisspub}. Nach der Entwicklung eines konkreten Forschungsvorhabens sowie einer wissenschaftlichen Fragestellung entstehen im Rahmen der wissenschaftlichen Forschung oder der jeweiligen Untersuchung Daten \cite{cite:11c}, die im Forschungsprozess gesammelt, analysiert, ausgewertet, aufbereitet und verschriftlicht werden \cite{cite:11d}. Die Ergebnisse werden abschließend strukturiert zusammengefasst und niedergeschrieben \cite{Hess_2006}.
\item Qualitätskontrolle und die Bewertung von Inhalten (Bewertung):
Die Qualitätskontrolle ist einer der wesentlichen Bestandteile der wissenschaftlichen Kommunikation. Sie sichert die im ersten Schritt gewonnen Erkenntnisse \cite{cite:11e} und stellt einen klaren Abgrenzungsaspekt zu nicht-wissenschaftlichen Informationen und Erkenntnissen dar\cite{cite:11f}. Sie findet im Kommunikationsprozess an zwei Stellen des Prozesses statt. Bei der initialen Bewertung wird die Publikation der Erkenntnisse vom Verlag organisiert \cite{schirmbacher_2009_wisspub} und von anderen Wissenschaftlern überprüft und gesichert (Peer-Review) \cite{Hess_2006}.
\item Auswahl der Inhalte durch Verlage auswählen (Bündelung):
Die Verlage kuratieren in Zusammenarbeit mit anderen Wissenschaftlern die wissenschaftlichen Inhalte für die Publikation. Bei wissenschaftlichen Journalen werden zum Beispiel die eingereichten Beiträge gebündelt und in einer Ausgabe mit anderen Beiträgen zusammengefasst.
\item Publikation der Inhalte durch Verlage (Druck):
Nach Erstellung und Erkenntnissicherung findet die "eigentlichen Publikation" \cite{schirmbacher_2009_wisspub} der Informationen statt. Bis zur Digitalisierung bestand dieser Schritt ausschließlich aus dem Druck der Inhalte auf Papier.\cite{cite:11h} Im Rahmen der Digitalisierung besteht der Prozess in der Aufbereitung der Beiträge für die digitale Verbreitung.
\item Distribution der Inhalte durch die Verlage (Verbreitung):
Der Vertrieb und die Verbreitung von Forschungsergebnissen an die wissenschaftliche Community ermöglicht den Zugriff auf die Informationen durch andere Wissenschaftler. Dieser Schritt stellt einen essenziellen Teil der Zirkulation und Kommunikation des neu gewonnen Wissens dar\cite{cite:11i}. Er sichert die Verfügbarkeit, die Möglichkeit des Zugriffs auf die Informationen und ist Teil des Selektionsprozesses für die Erschaffung neuen Wissens.\cite{cite:11l}
\item Support und Archivierung (Archivierung): Dieser Schritt beinhaltet die Erschließung, Aufbewahrung und Bereitstellung der Publikation durch Bibliotheken \cite{schirmbacher_2009_wisspub}. Die Bibliotheken unterstützen den Wissenschaftler und die Institution bei der Bewahrung und die Archivierung von Wissen \cite{K_lbel_2002}.
\item Konsum und Rezeption der Inhalte (Aufnahme von Wissen): Die Rezeption der publizierten Inhalte durch die wissenschaftliche Gemeinschaft ist der  Schritt des wissenschaftlichen Kommunikationsprozesses, bevor der ganze Prozess von neuem beginnt \cite{schirmbacher_2009_wisspub}. In diesem Schritt wird durch den Vergleich neuer Ergebnisse mit bereits publizierten Inhalten erneut die wissenschaftliche Qualität gesichert \cite{umstatter_2007_qualitatssicherung}. Aus der Mitte der wissenschaftliche Gemeinschaft entsteht durch die Verschriftlichung der wissenschaftliche Kommunikation der publizierten Ergebnissen neues Wissen \cite{cite:11k} \cite{schirmbacher_2009_wisspub} und der Kommunikationsprozess beginnt von vorn.
\end{enumerate}

An diesem Prozess des wissenschaftlichen Publizierens sind vor allem drei Gruppen beteiligt: erstens die Wissenschaftler, als Produzenten und Konsumenten der Informationen, zweitens die Verleger, die als Intermediäre wissenschaftliche Informationen sammeln, bündeln und verkaufen, sowie drittens die Bibliotheken, die die Informationen wieder den Wissenschaftlern zur Verfügung stellen \cite{Odlyzko_1997}. Wissenschaftler stehen dabei an einer komfortablen Stelle des wissenschaftlichen Produktions- und Distributionssystems \cite{herb_2010}, da sie ausschließlich mit der Verarbeitung und Neuerstellung von Wissens beschäftigt sind. Den Erwerb der Publikationen übernehmen die Bibliotheken, mit der Distribution sind die Verlage befasst. Wissenschaftler und Wissenschaftlerinnen verfügen häufig über sehr gute Zugangsmöglichkeiten zu wissenschaftlichen Informationen durch ihre Forschungsinsititutionen \cite{cope2014future}. Aus dieser Position sind sie als Autoren und als Leser mit den finanziellen Herausforderungen beim Vertrieb von Wissen nicht konfrontiert. Sie werden an staatlichen, wissenschaftlichen Institutionen größtenteils durch öffentliche Gelder finanziert und erhalten durch die Bibliotheken ihrer Institution Zugang zu wissenschaftlichen Publikationen. Sie schreiben Texte für die Publikation in wissenschaftlichen Verlagen, und werden mit im Rahmen der Veröffentlichung mit Reputation "belohnt". In diesem Publikationskreislauf sind die Verlage die einzige voll-privatwirtschaftliche Gruppe, die Ressourcen aus dem System herauszieht, ohne dass diese Ressourcen vollumfänglich dem Kreislauf der Wissenschaftskommunikation wieder zugeführt werden \cite{kiley_2006_open}.

---- TODO: Grafik bauen ----

\subsection{Wissenschaftlicher Diskurs nach dem Diskurs- und Machtbegriff}

\begin{quote}Wissenschaftliche Kommunikation vollzieht sich in Behauptungen, Erklärungen, Prognosen; sie ist nicht nur ein Informationsaustausch. Vielmehr vollzieht sich im wissenschaftlichen Diskurs der kollektive Prozeß des wissenschaftlichen Begreifens. Deshalb ist die wissenschaftliche Sprache als Diskurs nicht bloß ein Medium der Kommunikation, sondern der Ort, an dem sich ein wesentlicher Teil der wissenschaftlichen Arbeit vollzieht, der kollektive Darstellungsraum der Wissenschaft. \cite{bohme_1978_wissenschaftssprachen}\end{quote}

Dennoch muss auch die Erforschung von wissenschaftlichen Fragestellungen als ein Bestandteil des wissenschaftlichen Diskurses\cite{suchen} betrachtet werden. Die Verarbeitung von Forschungsergebnissen, die Anwendung und Neuinterpretation von Ergebnissen und das Verfassen von Gegenentwürfen und synthetischer Gesamtdarstellungen stellen Faktoren für den wissenschaftlichen Diskurs dar \cite{suchen}. Jürgen Habermas unterschied das kommunikatives Handeln von strategischem Handeln. Im dem "rationalen Diskurs" findet dabei vor allem eine Verständigung über problematische Geltungsansprüche statt \cite{suchen}. Der Beobachter entwickelt Methoden und Verfahren um zu einer Verständigung mit seiner Zielgruppe zu kommen \cite{suchen}. Der wissenschaftliche Diskurs operiert in diesem Verständigungsprozess funktional eigenständig und alles, was durch Wissenschaft kommuniziert wird, ist “entweder wahr oder unwahr” \cite{Luhmann1998}.

Michel Foucault versteht unter einem Diskurs "eine Menge von Aussagen, die einem gleichen Formationssystem zugehören"\cite{foucault_archaologie_1981}. Der wissenschaftliche Diskurs gründet sich demnach nur zum Teil auf die Forschung und kann auch nicht nur als “Kontaktglied zwischen dem Denken und dem Sprechen” \cite{foucault_ordnung_2003} definiert werden. Er wird getrieben vom Wille zur Wahrheit der sich durch "die Pädagogik, dem System der Bücher, der Verlage und Bibliotheken, den gelehrten Gesellschaften einstmals und den Laboratorien heute" ständig erneuert \cite{foucault_ordnung_2003}. Abgesichert wird der Diskurs "durch die Art und Weise, in der das Wissen in einer Gesellschaft eingesetzt wird, in der es gewertet und sortiert, verteilt und zugewiesen wird"\cite{foucault_ordnung_2003}. In der Foucault'schen Diskursanalyse wird der Diskurs als die Fähigkeit definiert, die “Beziehungen” zwischen “Institutionen, ökonomischen und gesellschaftlichen Prozessen, Verhaltensformen, Normsystemen, Techniken, Klassifikationstypen und Charakterisierungsweisen herzustellen”\cite{foucault_archaologie_1981}.

Im Rahmen des wissenschaftlichen Diskurses versuchen Menschen mit diversen "Machtprozeduren", die "ungeordnete und wuchernde Masse aller Äußerungen" zu reglementieren und zu kontrollieren \cite{Neymeyer_diskurs_2010}. Resultierend daraus entstehen Diskurse, die sich über einen gemeinsamen Gegenstand definieren. Sie gehorchen "impliziten wie expliziten Regeln", unterliegen "spezifischen Funktionen", nehmen bestimmte Formen an und sind von "Machtmechanismen gekennzeichnet". \cite{Neymeyer_diskurs_2010}

Der Begriff Diskurs wird dabei im Allgemeinen als die mündliche, unmittelbare Interaktion aufgefasst, da es (wie hier) um Publikationen, also um eine schriftliche kommunikative Betätigung geht wird in dieser Arbeit, wie in anderen \cite{graefen2007_wissenschaftliche_artikel}, "wissenschaftliche Kommunikation" als übergreifender Ausdruck verwendet.
