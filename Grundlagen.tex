\chapter{Grundlagen: Chronologie, Begriffsbestimmungen und Debatten}

Das System der wissenschaftlichen Kommunikation, das in der derzeitigen Form seit mehreren hundert Jahren besteht, basiert auf der Forschung, der Begutachtung, dem Druck, der Kommunikation der Ergebnisse in wissenschaftlichen Publikationen, der Verbreitung sowie dem Verkauf an Bibliotheken und andere wissenschaftliche Institutionen \cite{cite:11a} und dem anschließenden Diskurs in der wissenschaftlichen Fachöffentlichkeit \cite{bbaw_publizieren_2015}. Der Fortschritt in diesem System ist demnach maßgeblich durch den offenen und freien Austausch sowie der Verbreitung von Informationen bedingt \cite{cite:11}.

Die Grundlagen, Annäherungsversuche an Definitionen und Debatten um die Öffnung von Kommunikation in Wissenschaft und Forschung sind in der gängigen Literatur weder einheitlich dargestellt und abgegrenzt, noch unumstritten \cite{muller_2010_open} \cite{schulze_2013_open}. Von hervorzuhebendem Interesse sind im Rahmen dieser Arbeit sind die chronologische Entwicklung wissenschaftlicher Kommunikation, die Darstellung der Ökonomie des Kommunikationssystems, die Herausforderungen im aktuellen Kommunikationssystem, die Debatte und Anknüpfungspunkte zur Öffnung von Wissenschaft, die Katalysatoren und Hindernisse dieser Entwicklung und der damit einhergehende Wandel mit Fokus auf den Bereich wissenschaftlicher Reputation, Ethos und Diskurs.

Im Folgenden werden wesentliche Anknüpfungspunkte an der Open-Access und Open-Science-Bewegung in Wissenschaft und Forschung dargestellt und erläutert. Die Auswahl der berücksichtigten Werke bezieht sich auf die für die Fragestellungen relevanten Beiträge und wird um die Betrachtung der Debatten von Open Access und Open Science ergänzt. Dabei gibt es nicht die eine Debatte um die Öffnung von wissenschaftlicher Kommunikation, es sind viel mehr eine Vielzahl von Auseinandersetzungen bezüglich unterschiedlicher Bedenken und Interessen von einer fluktuierenden Gruppe von Akteuren \cite{Beals_2013} mit nicht selten polemisch geführte Diskussionen \cite{Lossau_oa_2007} \cite{naeder_2010_open}.

Die Öffnung von Wissenschaft und Forschung wird in dieser Arbeit im Kontext wissenschaftlicher Reputation in Bezug auf ihre technischen, gesellschaftlichen, sozialen und politischen Aspekte beschrieben und die Betrachtung wird auf die kulturellen Auswirkungen der Medienbrüche im Rahmen wissenschaftlichen Publizierens erweitert. Der historische und gesellschaftliche Kontext ihrer Anwendung wird dargestellt und mittels der Analyse wissenschaftlicher Literatur abgegrenzt. Es wird erläutert, welche Bedeutung sie in der Forschung, der Gesellschaft und der Politik haben. Die Entstehung und Entwicklung der Begriffe wird im Verlauf der Arbeit dargestellt. Um ein möglichst umfassendes Bild zu erhalten, wird "Entwicklung" hier in den drei folgenden Dimensionen erfasst: erstens, als "analytische Kategorie", zweitens als "Forschungsgegenstand" und drittens als "politische Praxis in der moralischen Auseinandersetzung über die Wünschbarkeit von Zuständen" \cite{cite:10}.

Die Betrachtungen in dieser Arbeit werden aus der Perspektive des Produzenten (Wissenschaftler als Autoren) sowie aus der, damit nicht immer harmonisierenden, Perspektive des Rezipienten, beziehungsweise Medienkonsumenten (Wissenschaftler als Leser) stattfinden. Es wird auch adressiert, inwiefern Macht, regulierende Prinzipien wie die Verknappung, sowie die Ein- und Ausgrenzung im Rahmen wissenschaftlicher Diskurse mit den Modellen Open Access, Open Science und wissenschaftlicher Reputation in der Kommunikation vereinbar sind oder diesen entgegenstehen.

Ziel ist es, existierende Erkenntnisse über die Begriffe und deren Entwicklung darzulegen sowie aufzuzeigen in welchen Bereichen weitere Forschung angestrebt werden sollte \cite{webster2002analyzing}. Für die Analyse wurden eine Vielzahl von Quellen mit thematischen Bezug zur Öffnung von Wissenschaft und Forschung ausgewählt und analysiert. Ziel dieses Kapitels ist es, die Debatten rund um die Begriffen und Forschungsfragen darzustellen um zu einer ausgewogenen Basis für dessen Betrachtung, soiwe zur Beantwortung der vorab definierten Forschungsfragen zu gelangen. Dafür werden die chronologische Entwicklung wissenschaftlicher Kommunikation sowie Herausforderungen im bestehenden System wissenschaftlicher Kommunikation dargestellt. In einem weiteren Schritt werden Anknüpfungspunkte zur Forderung nach Öffnung der wissenschaftlichen Kommunikation identifiziert und abschließend die Ergebnisse zusammengefasst und Anknüpfungspunkte für die empirische Untersuchung abgeleitet.

Folgenden Fragestellungen sollen mit Hilfe dieses Kapitels adressiert werden:
\begin{itemize}
\item Welche historischen Entwicklungen haben die Entwicklung der wissenschaftlichen Kommunikation und die Forderung nach Öffnung beeinflusst?
\item Wie funktioniert die Ökonomie der wissenschaftlichen Kommunikation?
\item Was bedeutet die Digitalisierung für das wissenschaftliche Kommunikationssystem?
\item Welche Rolle spielen die wissenschaftliche Reputation, wissenschaftliches Ethos und der wissenschaftliche Diskurs im Rahmen des Kommunikationssystems?
\item Welche Herausforderungen bestehen im System wissenschaftlicher Kommunikation?
\item Welche Indikatoren für Reputationsverteilung im wissenschaftlichen Kommunikationssystem werden in der Literatur genannt?
\item Welche Anknüpfungspunkte werden zur Forderung nach Öffnung wissenschaftlicher Kommunikation genannt?
\item Welche Ableitungen können für die empirische Untersuchung gemacht werden?
\end{itemize}

Zur Auswertung der vorliegenden Literatur werden aus einem Korpus von ausgewählten Texten die Entwicklungen, unterschiedlichen Definitionen und Debatten rund um den Themenkomplex der Öffnung von Wissenschaft und Forschung extrahiert und mit dem Ziel zusammengefasst weitere wissenschaftliche Fragestellungen für die Befragung von publizierender Wissenschaftler und Wissenschaftlerinnen verschiedener Fachbereiche zu entwickeln. Abschließend werden in diesem Kapitel aus der Literatur die Katalysatoren und Hindernisse für die Öffnung von Wissenschaft identifiziert, für die Befragung extrapoliert und in der Gesamtbetrachtung der Arbeit zusammengeführt und strukturiert ausgewertet.

Der theoretische Bezugsrahmen wissenschaftlich gesicherter Modelle, Theorien und Ansätze macht es möglich, Erklärungen und Handlungsempfehlungen abzuleiten \cite{martin_2007_wissenschaftstheorie}. Er trägt dazu bei, die Fragestellungen in einen Zusammenhang zu stellen, legitimiert die Erforschung dieser Fragen und bildet den Rahmen für die Auswertung gesammelter Erkenntnisse \cite{suchen}. Ziel dieses Kapitels ist es, die theoretischen Grundlagen für die im späteren Verlauf der Arbeit folgende empirischen und experimentellen Untersuchungen zu erarbeiten, sowie die Begriffe,  Definitions- und Konzeptvielfalt, die für das Thema der vorliegenden Arbeit grundlegend sind, einzuführen.

\section{Wissenschaftliche Kommunikation}

Bevor die Grundlagen für Offenheit in Wissenschaft und Forschung dargestellt werden, werden eine grundlegende Einordnung von wissenschaftlicher Kommunikation vorgenommen, die Entwicklung chronologisch dargestellt und deren Wandel im Rahmen der Digitalisierung beschrieben. Darauf folgt eine Beschreibung der Ökonomie der Kommunikation in der Wissenschaft, Ausführungen zum Verhältnis von wissenschaftlicher Reputation, Ethos und wissenschaftlichem Diskurs sowie die Darstellung der Herausforderungen im aktuellen Kommunikationssystem und Anknüpfungspunkte zu der Forderung nach Öffnung der wissenschaftlichen Kommunikation.

Kommunikation stellt einen wesentlichen Bestandteil des wissenschaftlichen Systems und der wissenschaftlichen Arbeit dar \cite{garvey_2014_communication} \cite[:63]{Luhmann1998}. Sie basiert auf dem Austausch zwischen Wissenschaftlern, die auf einem "gemeinsamen Wissensbestand" zugreifen, "den sie testen, verändern und erweitern" \cite{Gl_ser_2007} und ist eng mit dem "Prozess des Veröffentlichens wissenschaftlichen Publikationen" \cite{weller2011twitter} verknüpft. Sinn und Zweck der Kommunikation beruht auf dem bestmöglichen Austausch zwischen den Mitgliedern der Wissenschaftsgemeinschaft. Er dient der Überprüfung der Zuverlässigkeit von Informationen und ermöglicht die kritische Auseinandersetzung innerhalb der Gemeinschaft \cite{fox_1983_publication}. Jede kommunizierte Erkenntnis trägt dabei theoretisch zur Produktion von Wissen bei \cite{kaden_2009_library}. Grundvoraussetzung dafür ist, dass Wissenschaftler und Wissenschaftlerinnen den Willen zu optimalen Kommunikation untereinander haben.

In der Theorie existieren verschiedene Arten der Organisation wissenschaftlicher Kommunikation und "vielfältige Erscheinungsformen" \cite{graefen2007_wissenschaftliche_artikel}, die sich im Laufe der Zeit immer wieder verändert haben \cite{Konneker_2013} \cite{hagner_2015_sache_buches}. Grundsätzlich ist die Unterscheidung in \textit{formelle} und \textit{informelle}, sowie die \textit{interne} und \textit{externe} wissenschaftliche Kommunikation zwar etabliert, aber nicht unstrittig \cite{seidenfaden_2005_kommunikation}.

Was nach dieser Betrachtung genau als \textit{formell} oder \textit{informell} gilt, hängt unter anderem von der jeweiligen Fachdisziplin ab, "ist historisch gewachsen und damit durchaus unterschiedlich" \cite{Hanekop_2014}. Eine wesentliche Plattform für die wissenschaftliche Kommunikation, Fortschritt und Forschungsförderung bilden Publikationen in Journalen und Monographien \cite{cope2014future} \cite{fox_1983_publication}. Das wissenschaftliche Journal sowie die Monographie sind (in den meisten wissenschaftlichen Disziplinen) sind wichtige Kanäle für die \textit{formelle} wissenschaftlichen Kommunikation und essenziell für Wissenschaftler und Wissenschaftlerinnen um auf dem Laufenden zu bleiben \cite{cope2014future}.

\begin{figure}[h!]
\includegraphics{smalltableid:o9wBj}
\caption{Traditionelle Trennung von informaler und formaler Kommunikation}
\end{figure}

Die \textit{formelle} Kommunikation wird an bestimmte Bedingungen der wissenschaftlichen Gemeinschaft geknüpft und hat einen direkten Einfluss auf die Reputation der einzelnen Mitglieder der wissenschaftlichen Community. Diese Art der Kommunikation beinhaltet die Einbeziehung Dritter, die die Funktion der Einordnung und Bewertung der Kommunikation übernehmen. Der bisherige Outputkanal für diese Kommunikation ist die gedruckte Publikation \cite{winkler_2011_anforderungen}, denn "es wird für den Druck geforscht" \cite{luhmann_1997_gesellschaft}. Durch sie "wird festgeschrieben, was nach den Kriterien des jeweiligen Fachs als geprüftes Wissen gelten kann" \cite{bbaw_publizieren_2015}. Ziel dieser Art der Kommunikation ist die Sicherung des Verbleibs und die Positionierung des einzelnen Wissenschaftlers innerhalb der wissenschaftlichen Gemeinschaft. Diese Formalisierung der Kommunikation ist wichtig um das Wissenschaftssystem und das Wissen strukturell sowie nachhaltig zu sichern und sie macht Erkenntnisprozesse nachweisbar \cite{kaden_2009_library}. Erst mit einer formell begutachteten Publikation wird eine wissenschaftliche Entdeckung als solche erkennbar \cite{brembs2015open}.

\textit{Formelle} wissenschaftliche Kommunikation beruht nach dem Bibliothekswissenschaftler Ben Kaden auf folgenden drei abstrakten Faktoren \cite{kaden_2009_library}:
\begin{enumerate}
\item \textit{Publizität} meint die Veröffentlichung der Erkenntnisse in einem wissenschaftlichen Fachmedium. Eine Erkenntnis wird durch die Veröffentlichung bekannt gegeben und so für die Community "registriert" \cite{kaden_2009_library} \cite[:5]{seidenfaden_2005_kommunikation}. Sie muss dabei "zeitnah" in einer "wahrnehmbaren" Form vorliegen \cite{Schimank_2012}, damit sie intersubjektiv vermittelbar ist.
\item \textit{Vertrauenswürdigkeit} meint das Vertrauen auf die Einhaltung der Regeln und die Möglichkeit der Zertifizierung \cite[:6]{seidenfaden_2005_kommunikation} im wissenschaftlichen Kommunikationssystem durch alle Teilnehmer und Teilnehmerinnen des Systems. Das Vertrauen wird bei einer Publikation durch die Überprüfung (Peer-Review) bestätigt und durch Bezugnahme (Zitationen) anderer Wissenschaftler auf die Publikation zu Reputation. Eine Zitation ist - aus Sicht der zitierten Arbeit - eine formelle Erwähnung der Arbeit innerhalb in einer anderen wissenschaftlichen Publikation \cite{weller2011twitter}.
\item \textit{Zugänglichkeit} bezieht sich auf die dauerhafte Sicherung, Archivierung \cite[:6]{seidenfaden_2005_kommunikation} und Zugänglichkeit in einer allgemein verfügbaren Form für die (Fach)Öffentlichkeit \cite{naeder_2010_open} um anderen Wissenschaftlern und Wissenschaftlerinnen zu ermöglichen, auf die Erkenntnisse, die für ihre eigene Tätigkeit von Relevanz sind, für die eigene Forschung zu nutzen \cite[:6]{seidenfaden_2005_kommunikation}.
\end{enumerate}

Die Möglichkeiten der \textit{informelle} Wissenschaftskommunikation sind höchst vielfältig und reichen "vom persönlichen Gespräch über Vorträge, Konferenzen, Zwischen- oder Abschlussberichte aus Projekten, Working Papers und vieles andere mehr"\cite{Hanekop_2014}. \textit{Informelle} Kommunikation umfasst alle Arten der Kommunikation, die dem individuellem Wissenschaftler oder der individuellen Wissenschaftlerin einen schnellen und direkten Austausch mit Kollegen ermöglichen und die keinen direkten Einfluss auf die wissenschaftliche Reputation des einzelnen Wissenschaftlers haben.

Die informelle Kommunikation findet überlicherweise am Beginn und nach Abschluss des wissenschaftlichen Erkenntnisprozesses statt. Sie umfasst zum Beispiel die Ideenfindung, die Entwicklung von Fragestellungen oder Konkretisierung des Forschungsvorhabens und hilft Wissenschaftlern dabei relevante Ideen für formelle Kommunikation "herauszukristallisieren" \cite{Hanekop_2014}. Informelle Kommunikation ist auf Grund ihrer Heterogenität und impliziten Verankerung weniger präzise differenzierbar und erfassbar \cite{kaden_2009_library}. Die Abgrenzung informeller Kommunikation zu "nicht-wissenschaftlicher Kommunikation" resultiert daraus, dass diese meist auf "die Erzeugung formeller Kommunikation hinarbeitet" \cite{kaden_2009_library}.

Im Gegensatz zur Segmentierung von \textit{formeller} und \textit{informeller} Kommunikation, zielt die Unterscheidung zwischen \textit{interner} und \textit{externer} Kommunikation auf die jeweilige Zielgruppe des Austauschs ab. \textit{Interne} Kommunikation beschreibt alle Prozesse die der Kommunikation innerhalb der wissenschaftlichen Gemeinschaft dienen. \textit{Externe} Kommunikation beschreibt die Kommunikation, die an Akteure außerhalb der wissenschaftlichen Gemeinschaft gerichtet ist \cite{Konneker_2013}.

---- TODO: Grafik bauen nach seidenfaden_2005_kommunikation bauen ----

In der vorliegenden Arbeit bezieht sich der Begriff "wissenschaftliche Kommunikation" vornehmlich auf jene Kommunikation, die in der Theorie formelle sowie interne organisatorische Bezugspunkte aufweist als auch einen Einfluss auf die wissenschaftliche Reputation des Wissenschaftlers oder der Wissenschaftlerin hat. Im Rahmen des Öffnungsprozesses der wissenschaftlichen Kommunikation wird von einem Aufbrechen dieser Trennung ausgegangen, deshalb wird diese eindeutige Unterscheidung im Laufe der Arbeit hinterfragt.

\subsection{Chronologie der Entwicklung wissenschaftlichen Kommunikation}

Für ein erweitertes Verständnis für die Prozesse, die zu der Öffnung von Wissenschaft und Forschung führen, sowie für die Darstellung der Beziehung neuer digitaler Kommunikationssysteme zu ihren analogen Vorläufern, ist eine historische Betrachtung der Entwicklung wissenschaftlicher Kommunikation sowie der Forderung nach Offenheit in Wissenschaft und Forschung unabdingbar. Diese stellt zum einen die Grundlagen für die Analyse von Offenheit dar und ebnet die weitere Grundlage für die Darstellung  des "Forschungsgegenstands" \cite{cite:10}. Diese historische Darstellung bietet einen ersten Ansatzpunkt für die Erforschung der unterschiedlichen Definitionen und Debatten um Open Access und Science \cite{Scheliga_2014}, da diese historischen Übergange bisher noch immer nur unzureichend dargestellt wurden \cite{CREATe_2014}.

Angelehnt an die Arbeiten des kanadischen Philosophen McLuhan und des Germanisten Wenzel können dabei drei bedeutende Umbrüche der Medienentwicklung im Rahmen der Kommunikation von Wissen genannt werden \cite{wunderlich_2008_buchdruck} \cite{wenzel_mediengeschichte_2007}:
\begin{enumerate}
\item der Übergang vom Körpergedächtnis (brain memory) zum Schriftgedächtnis (script memory)
\item der Übergang von der Handschriftenkultur zur Druckkultur (print memory)
\item und der Übergang vom Buch zum Bildschirm (electronic memory)
\end{enumerate}

\subsubsection{Wissenschaft und wissenschaftliche Kommunikation in pre-modernen Zivilisationen}

In der Antike stellten der orale Dialog und Disput, Vortrag und die Lehrstunde die Formen "wissenschaftlicher Kommunikation" dar \cite{hollricher_wandel_2009}. Dabei bezog sich "Wissenschaft" in pre-modernen Zivilisationen unmittelbar auf die täglichen Bedürfnissen. Wissen und Informationen wurden als nicht besitzbare Ware angesehen \cite{cite:18} \cite{steiner_1998_autorenhonorar} und im Vergleich zu den heutigen Möglichkeiten war in den vormodernen Zivilisationen der Wissensaustausch stark beschränkt \cite{cite:17c}. Es gab keine "scharfe Grenze zwischen dem vorhandenen und dem aktuell benutzten Wissen" \cite{Luhmann1998}. Die Produktion von Literatur beschränkte sich in den vorwissenschaftlichen Gesellschaften vornehmlich auf "auf die Überlieferung und Kommentierung des althergebrachten Wissens, insbesondere des theologischen" Wissens \cite{steiner_1998_autorenhonorar}. Was die Gelehrten "zu sagen und zu schreiben hatten, war nicht als Beitrag zum Fortschritt von Wissenschaft als einem kollektiven Unternehmen zu verstehen, sondern eher als Dokumentation ihrer persönlichen Erkenntnisfortschritte" \cite{graefen2007_wissenschaftliche_artikel}. Sie hatten vor allem die Aufgabe das Wissen "zu erhalten und zu tradieren" \cite{Luhmann1998}. Eine Textart, die dem heutigen wissenschaftlichen Artikel entspricht oder mit ihm vergleichbar ist, existierte bis zum Mittelalter nicht. Noch im 15. und 16. Jahrhundert sind nur wenige Texte "fachinterner Kommunikation" also schriftlicher Kommunikation unter Vertretern eines Faches über fachliche Inhalte, nachgewiesen" \cite{graefen2007_wissenschaftliche_artikel}. Texte die wir heute als wissenschaftlich bezeichnen würden, wurden im Mittelalter nur dann akzeptiert, wenn sie den Namen eines (anderen) Autors trugen \cite{foucault_2000_autor}.

Die Sprachwissenschaftlerin Graefen hat exemplarisch die Entwicklung zum wissenschaftlichen Text wie folgt zusammengefasst: "Erst wenn ein gesamtgesellschaftlicher Bedarf an Wissen und an ständiger Wissenserweiterung allgemein erkennbar wird und entsprechende Leistungen von Individuen auch persönliche Vorteile versprechen, findet eine Umorientierung von sporadischer individueller wissenschaftlicher Betätigung hin zu gesellschaftlich anerkannter und zur Kenntnis genommener, kollektiv bzw. arbeitsteilig betriebener Wissenschaft statt" \cite{graefen2007_wissenschaftliche_artikel}.

\subsubsection{Einführung des Buchdrucks als Grundlagen der modernen Wissenschaft}

Die Geschichte der modernen Wissenschaft ist eng mit der Geschichte des Buchdrucks verbunden. Diese beginnt maßgeblich mit Johannes Gensfleischs, auch Gutenberg genannt, Beiträgen zur Buchdruckerkunst \cite{wittmann_1999_geschichte} in der Mitte des 15. Jahrhunderts \cite{suchen}. Gutenberg führte um 1460 die Druckerpresse ein, "die er von den Weinpressen der rheinischen Winzer abgeschaut und dann verbessert haben dürfte" \cite{stober_2014_pressegeschichte}. Die Einführung des Buchdrucks führte nicht nur zu neuen Möglichkeiten der Kommunikation, sondern zu einer Veränderung der generellen Aufgabe der Wissenschaft, insbesondere ihrer der Orientierung auf den täglichen Bedarf \cite{Luhmann1998}. Durch die neue Möglichkeiten der Vervielfältigung und Massenverbreitung hat das Selbstverständnis der europäischen Kultur in bis dahin unbekannter \cite{giesecke_1991_buchdruck} und revolutionärer Weise verändert \cite{wunderlich_2008_buchdruck} \cite{stober_2014_pressegeschichte}.

Der Buchdruck stellte somit die "Grundlagen und Meilenstein sowohl für die Kommunikation der Menschheit insgesamt als auch für den wissenschaftlichen Gedankenaustausch im Besonderen dar" \cite{schirmbacher_2009_wisspub}, er war ein "Bestandteil des Übergangs vom Mittelalter in die frühe Neuzeit" \cite{lange2008medienwettbewerb} und die Druckerpresse, nahm die "entscheidende Schwelle für das Entstehen der neuzeitlichen Wissenschaften" \cite{luhmann_1997_gesellschaft}.

Diese neue Technologie führte zu einem bis dahin unbekannten, explodierenden Informationsangebot. Infolgedessen entwickelte sich eine neue Denkstruktur \cite{eisenstein_1997_druckerpresse}, bei der das "mittelalterliche Denken in Bildern und Metaphern" von der "wissenschaftlich-systematischen Methodik" abgelöst wurde \cite{wunderlich_2008_buchdruck}. Sie führte zur Befreiung des jeweiligen Autors aus der weitgehenden Anonymität mittelalterlicher Manuskriptkultur und zur Entkoppelung der "Herstellung und Verbreitung vom singulären Interesse eines Autors, Kopisten oder Auftraggebers"\cite{wunderlich_2008_buchdruck} \cite{schirmbacher_2009_wisspub}.

Mit der Entwicklung der Buchdrucktechnologie folgte im 16. Jahrhundert die Verbreitung eines "freien Marktes als Vertriebsnetz für typographische Informationen"\cite{giesecke_1991_buchdruck} und die "Kapitalisierung der Buchproduktion" \cite{steiner_1998_autorenhonorar}. Das gedruckte Wort führte somit zu einem "Verlust an Macht und Herrschaft über das geschriebene Wort" \cite{wunderlich_2008_buchdruck}. Anfangs handelte es sich bei der Technologie nur um ein "elitäres und teures Medium für die gebildete Klasse" \cite{hartmann_2008_medien}, Bücher waren "Luxusgegenstände" und die Gewinnspannen der Buchdrucker und -händler waren "enorm" \cite{stober_2014_pressegeschichte}. Sie führte weder von Beginn an zum zeitlich unmittelbaren Zugang zu Wissen noch war sie sofort für die Allgemeinheit zugänglich \cite{hartmann_2008_medien}. Die wissenschaftliche Elite der damaligen Zeit forderte deshalb, dass Werke ohne Rücksicht auf Profitgier und "Geiz" \cite{luther_1876} erscheinen sollten und appellierte an eine "obrigkeitliche Lenkung", damit der Buchhandel "seiner Aufgabe der Verbreitung von nützlichem Wissen gerecht würde" \cite{wittmann_1999_geschichte}. Gutenbergs Druckinnovation sollte als sogenannte "Schlüsseltechnologie" \cite{jager_1993_theoretische} eine neue Dimension der Informations- und Wissensverbreitung für die Gesamtgesellschaft ermöglichen.

In der Übergangszeit von der primären Kommunikation zwischen den Gelehrten anhand von Briefen und der Verbreitung des Buchdrucks kam es zu einer Vielzahl sogenannter Prioritätsstreits \cite{schirmbacher_2009_wisspub}, denn die meisten wissenschaftliche Erkenntnisse waren zuvor zwar im direkten Briefwechsel, aber noch nicht öffentlich verbreitet worden. Deshalb konnte zu dieser Zeit selten ein für alle nachvollziehbarer Bezug zum jeweiligen Entdecker hergestellt werden. Als Beispielhaft für einen solchen Prioritätsstreit kann die Auseinandersetzung zwischen Isaac Newton und Gottfried Wilhelm Leibniz um eine Veröffentlichung zur Fluxionsrechnung im 17. Jahrhundert genannt werden. Leibniz rezensierte eine von Newton verfasste Veröffentlichung anonym und stellte sich selbst namentlich als Erfinder dieser dar \cite{2013_leibniz}, ohne auf eine öffentliche Publikation seiner deutlich länger vorhandenen Erkenntnisse hinweisen zu können \cite{schirmbacher_2009_wisspub}. Auf Grund des fehlenden öffentlichen Nachweises wurde Leibniz infolgedessen durch die Royal Society, einer der ersten Gelehrtenvereinigungen, des Plagiats für schuldig befunden und Entdeckung Newton zugesprochen. Doch selbst wenn Newton seine Fluxionsrechnung "früher entwickelt hat, geht die algorithmische Eleganz von Differentialen und Integralen doch auf Leibniz zurück" \cite{kittler_faz_1996}.

Der Buchdruck, wie auch die ersten wissenschaftlichen Zeitungen, wurden für die wissenschaftlichen Autoren somit nicht nur zu einem neuen "Kommunikationsinstrument", einem Instrument zur "Erlangung von Reputation" oder zu einem Instrument "zur Generierung finanzieller Erträge" sondern auch zu einem "Nachweisinstrument" \cite{wunderlich_2008_buchdruck} \cite{schirmbacher_2009_wisspub} für die Vermeidung solcher Prioritätskonflikte. Darüber hinaus "waren gedruckte Meinungen schwerer zu widerrufen oder umzuinterpretieren als nur mündlich geäußerte oder nur wenigen zugängliche (etwa Briefe)" \cite{luhmann_1997_gesellschaft}.

Die Verbreitung des Buchdrucks fand aber nicht ungebremst und nicht ohne umfassende Kritik in der damaligen Gesellschaft statt. Vor allem kirchliche Instanzen waren über eine "wachsende theologische Begriffsverwirrung" und die Verbreitung der Schriften in Volkssprachen besorgt \cite{giesecke_1991_buchdruck}. Sie stellten die größten Gruppe an Kritikern des Buchdrucks dar und versuchten die neue "Bücherflut" zu unterbinden\cite{giesecke_1991_buchdruck}. Additiv führte die Einführung des Buchdrucks zu einer neuen Bedeutung der Zensur, als "prohibitives Instrument für die Überwachung der Lektüren" und als "Kampfmittel" \cite{sprachgeschichte_1998_besch} gegen zu viel Wissen \cite{suchen} und "unerwünschte Literatur" \cite{suchen}. Beispielhaft für diese Art der Zensur, zitiert der Kommunikations- und Medientheoretiker Michael Giesecke aus einem Gutachten dieser Zeit: "In den Anfängen muß man Widerstand (gegen das Übel des Drucks von Büchern, die aus den heiligen Schriften in die Volkssprache übersetzt sind), damit nicht durch die Vermehrung der deutschsprachigen Bücher der Funke des Irrtums endlich sich zu einem großen Feuer entwickle" \cite{giesecke_1991_buchdruck}.

Zusammenfassend nennt Giesecke vor allem folgende grundlegenden Einwände gegen den Buchdruck als unregulierte, "freie" Kunst \cite{giesecke_1991_buchdruck} für die Verbreitung von Wissen und Informationen:
\begin{itemize}
\item Die Einführung des Buchdrucks wurden von vielen Warnungen vor Missbrauch der Technologie begleitet \cite{lange2008medienwettbewerb}. Im Mittelpunkt der Warnungen standen der anti-religiöser Missbrauch durch die Verbreitung gefährlichen Gedankenguts \cite{kruse2003multimedia}, die bewusste Falschinformation und Verfälschung von Inhalten \cite{sprachgeschichte_1998_besch}, die willkürliche Informationsverbreitung über Bücher, ohne Zustimmung der geistlichen und weltlichen Regenten \cite{rother_2002_siebenbuergen} sowie die Angst der Traditionalisten, die ihre Herrschaft durch das Monopol auf die Interpretation der Bibel gefährdet sahen \cite{lange2008medienwettbewerb}.
\item Ein weitere Einwand adressierte die Befürchtung, dass die Qualität und Reinhaltung der besten Texterzeugnisse beim Buchdruck nicht sichergestellt werden kann \cite{giesecke_1991_buchdruck}.
\item Auch die Nachlässigkeit und Unachtsamkeit von Buchdruckern und Setzern wurde früh kritisiert. Sie spielten im Buchdruckprozess eine entscheidende Rolle, da sie großen Einfluss auf die Qualität der Nachdrucke hatten. Nachlässigkeit oder ungenaues Arbeiten führten zu erheblichen strukturellen und inhaltlichen Qualitätsverlusten, was von Autoren wie Martin Luther schon früh beklagt worden war \cite{sprachgeschichte_1998_besch} \cite{stober_2014_pressegeschichte} \cite{luther_1876}.
\item Die Multiplikation von Fehlern, da in den gedruckten Exemplaren auch die Fehler völlig übereinstimmen und nicht behoben werden können, schließt an die Kritik der Qualität der gedruckten Bücher an. Die Befürchtung begründete auf der Irreversibilität der Verbreitung fehlerhafter Inhalte beim Buchdruck, die bei der geringeren Anzahl handschriftlichen Kopien bisher weniger Einfluss hatte \cite{kittler_2004}.
\item Die staatlichen und geistigen Obrigkeiten befürchteten durch die Demokratisierung der Vervielfältigung und Verbreitung von Wissen die Verwirrung der "Laien" (der Glaubensgemeinschaft) und damit einen Kontrollverlust für die bestehende gesellschaftliche Ordnung. \cite{giesecke_1991_buchdruck}.
\item Demzufolge befürchtete die Obrigkeit, die Auflösung der ständischen Ordnung da der "Zugang zu den Speichern des Wissens nicht länger bestimmten Schichten vorbehalten bleibt" und das "Schreiben und Lesen wird von einer ständischen zu einer gemeinen Tätigkeit". Aus heutiger Sicht mag diese Sicht auf Grund der sehr geringen Alphabetisierungsrate und der noch immer sehr geringen Anzahl an Büchern Ende des 15. Jahrhunderts als unbegründet erscheinen, dennoch wurden andererseits die soziale Umwälzungen durch den Buchdruck beschleunigt und unumkehrbar gemacht. \cite{giesecke_1991_buchdruck}
\item Auflösung des "Amts" des Bücherschreibers als eigenes Handwerk
\item Die Angst vor dem Überfluss an Büchern und Wissen stellte einen weiteren Einwand dar. Die Kritiker der Buchdrucktechnologie befürchteten  durch die massenhafte Verbreitung ein Chaos an Informationen \cite{giesecke_1991_buchdruck}.
\item Sogar physische Konsequenzen wurden befürchtet: "Augen schmerzen, vom Lesen, unsere Finger vom Blättern" \cite{giesecke_1991_buchdruck}
\item Auch "psychische Bedenken" wurden eingebracht, so gab es im 15. Jahrhundert bei den Menschen die Angst vor dem Anhäufen von Informationen. Sie galt im Mittelalter als "gefährliches und verwirrendes Unterfangen" und führte zu Annahmen wie "je gelehrter, je verkehrter". \cite{giesecke_1991_buchdruck}
\end{itemize}

Die genannten Einwände fußten allesamt auf den Ängsten oder Befürchtungen vor den Veränderungen und deren Auswirkungen auf die etablierten Machtstrukturen, die ihrerseits die Informationsverbreitung bis Ende des Mittelalters beeinflusst hatten und weisen punktuell Parallelen zu den Debatten der heutigen Veränderungsprozessen auf \cite{hagner_2015_sache_buches}. Vor der Einführung des Buchdrucks wurde vorab entschieden, was veröffentlicht und verbreitet wurde und es gab klare Instanzen, die die Weitergabe von Wissen (meist Auftragsarbeiten) organisierten. Der Buchdruck kehrte dieses System um, da nun Texte erstmals verbreitet wurden und man es dem "Markt und dem nachträglichen Meinungsstreit überließ, welche Information zum Gemeingut wurden" \cite{giesecke_1991_buchdruck}. Niklas Luhmann fasste diese Veränderung wie folgt zusammen: "Wer für den Druck schreibt, gibt die Situationskontrolle auf" und "produziert für das Gedächtnis des Systems" bei dem weder "Kommunikationsvorgang" noch der "Wissenszuwachs" abgeschlossen sind \cite{Luhmann1998}.

Die Etablierung und schnelle Verbreitung \cite{stober_2014_pressegeschichte} des Drucks führte, zunächst "unbemerkt und naturwüchsig", zu einer Veränderung der Sozialisierung von Informationen, der Veröffentlichung \cite{giesecke_1991_buchdruck}. Das Medium der Schrift wurde demnach unter den Buchdruckbedingungen als eine "Verbreitungstechnologie" für Informationen genutzt, die zwar die unmittelbare Interaktion zwischen Sender und Empfänger (weiterhin) ausschloss, aber mittelbar nur mit Hilfe von Empfängern zu Wissen werden konnte \cite{Luhmann1998}.

Die Einführung des Buchdrucks stellte somit einen Bestandteil des "Übergangs vom Mittelalter in die frühe Neuzeit dar"\cite{lange2008medienwettbewerb}, da zwischen Buchdruck und demokratischen Freiheiten "sowohl faktisch als auch ideologisch" \cite{suchen} ein Zusammenhang hergestellt werden kann. Dieser Zusammenhang wird darin deutlich, dass im Gegensatz zum Mittelalter, in dem jede breitere Sozialisierung und Verbreitung privater Gedanken "legitimationsbedürftig" war, nun jeder Eingriff in die "Freiheit, Meinungen oder Informationen" zu drucken einer politischen Legitimation \cite{giesecke_1991_buchdruck}. Der Buchdruck kann als "Katalysator des kulturellen Wandels"\cite{giesecke_1991_buchdruck} im Rahmen der "fundamentalen Umbrüche in Politik und Verwaltung, Ökonomie und Handel, Religion, Bildung und nicht zuletzt in den Prozessen der kognitiven Welterkenntnis" \cite{pscheida_2010_wikipedia} verstanden werden.

Um den Arbeitsaufwand der Drucker zu honorieren und die verlegerische Leistung zu würdigen\cite{szilagyi_2011_leistungsschutzrecht}, wurden mit der Entstehung des Druckerwesens auch erste Privilegien vergeben \cite{gieseke_1995_privileg}, die es den Druckern erlaubte, die Buchdruckkunst für einen bestimmten Zeitraum allein oder in einem bestimmten Gebiet auszuüben \cite{martin2008publizistische} \cite{koller_1995_Urheberrecht}. Diese Privilegien ermöglichten den Begünstigen Sonderberechtigungen oder -rechte gegenüber den derzeit üblichen allgemeinen Rechtsregeln \cite{jaenich_2002_geistiges}. Im Zuge der Verbreitung der Drucktechnologie und des steigenden Wettbewerbs kam es auch zu ersten Privilegien für Urheber, die bereits im 15. Jahrhundert damit begannen ihre Manuskripte zu verkaufen \cite{hesse2002rise}. Ebenso Erstverleger, die damit versuchten sich gegen das Nachdrucken und Raubdrucke zu erwehren. Die erfolgreiche Einforderung dieser Privilegien führte schon früh zu einer Art Monopolstellung bestimmter Druckereien und zu einem generellen Nachdruckverbot für bestimmte Werke in einem bestimmten Gebiet oder für einen bestimmten Zeitraum \cite{szilagyi_2011_leistungsschutzrecht} \cite{hesse2002rise}. Später wurden auch erste Autorenprivilegien gewährt, welche als die ersten Ursprünge für das heutige Verwertungs- und Urheberrecht im Publikationssystem darstellen \cite{koller_1995_Urheberrecht}.

\subsubsection{Wissenschaftliche Journale als Medium der wissenschaftlichen Kommunikation}

Noch zu Beginn des 17. Jahrhunderts stellten das Schreiben von Briefen oder Büchern die häufigsten Formen des wissenschaftlichen Austauschs dar \cite{porter_1964_scientific}. Der Brief, als besonders exklusive Form der Kommunikation stand dem Buch als sehr zeitaufwändige Form gegenüber \cite{fecher_hiig_2014}.

Erst die "drucktechnische Möglichkeit der schnellen Produktion, Vervielfältigung und Verbreitung von Texten" und "die Loslösung der Wissenschaft(en) von Religion und schöner Literatur" machten eine "Umorientierung von sporadischer individueller wissenschaftlicher Betätigung hin zu gesellschaftlich anerkannter und zur Kenntnis genommener, kollektiv bzw. arbeitsteilig betriebener Wissenschaft" möglich \cite{graefen2007_wissenschaftliche_artikel}. Die Gründung von Akademien als einer Art von nationalen Gelehrtengesellschaften im 17. und 18. Jahrhundert führte zu Veränderungen der wissenschaftlichen Literatur \cite{graefen2007_wissenschaftliche_artikel} und Verschiebung der Darstellung wissenschaftlicher Praxis in separater Experimentierräume \cite{weingart_2005_wissenschaft}. Die Akademien fungierten als Vereinigung einzelner Gelehrter und "durch sie fand eine Konzentration vereinzelter wissenschaftlicher Anstrengungen und Leistungen statt" \cite{graefen2007_wissenschaftliche_artikel}. Die "Einführung von Präzisionsmessungen als Teil der experimentellen Praxis", sowie "die Einrichtung separater Experimentierräume, um der Sensibilität der Präzisionsinstrumente gerecht zu werden" ging mit einer "Veränderung der Umgangsformen in der Akademie einher". Damit verlagerte sich "das Problem, andere zu überzeugen, von der unmittelbaren Demonstration von Evidenz auf die mittelbare Darstellung in Texten" \cite{weingart_2005_wissenschaft}.

Mitte des 17. Jahrhunderts kam es in Folge der Gründung der "Royal Society", als eine Akademie zur Förderung naturwissenschaftlicher Experimente, zu einer wissenschaftlichen Diskussion über die Etablierung einer neuen "Philosophie für die Förderung von Wissen". Die Mitglieder der Royal Society hegten den Wunsch nach einer Verbesserung bei der Verbreitung ihrer wissenschaftlichen Erkenntnisse und eine "wissenschaftlichen Revolution" mit Hilfe der Drucktechnologie voranzutreiben \cite{Dear_1985}. Als ein Ergebnis der 1660 gegründeten Akademie erschienen 1662 die ersten beiden Bücher, John Evelyn's "Sylva" und "Micrographia" von Robert Hooke \cite{hall_1992_library_rsol}. Am 6. März 1665 wurde mit "Philosophical Transactions" eine der ersten wissenschaftliche Fachzeitschriften veröffentlicht \cite{Peters_2014}, "die bis ins 20. Jahrhundert hinein eine der angesehensten Fachzeitschriften blieb" \cite{graefen2007_wissenschaftliche_artikel}. Im gleichen Jahr, bereits am  5. Januar 1665, erschien das "Journal des sçavans" in Frankreich \cite{ball_2011_zeitalter}, \cite{hollricher_wandel_2009} das zu Beginn über aktuelle Entdeckungen berichtete \cite{epaa_Weiner_2001}. Bis zum 17. Jahrhunderts folgten circa 30 weitere Journalgründungen. Die Journals unterschieden sich in ihrer Struktur stark von den heutigen und wiesen bis Ende des 18. Jahrhunderts kaum eine fachliche Spezialisierung auf. Sie beinhalteten "auch anwendungs- und praxisbezogene Beiträge" \cite{graefen2007_wissenschaftliche_artikel}. Sie enthielten im Vergleich zu den heutigen Fachzeitschriften jeweils eine nur sehr geringe Anzahl von Beiträgen \cite{suchen} und waren an wissenschaftlichen Briefe (meist in der Ich-Form) angelehnt, die Wissenschaftler vor der Entwicklung der Journale noch direkt aneinander verschickt hatten \cite{epaa_Weiner_2001}. "Oft handelte es sich gar nicht um Originalbeiträge, sondern die Herausgeber teilten der gelehrten und gebildeten Menschheit mit, was sie aus ihren Briefwechseln mit Gelehrten Interessantes entnahmen" \cite{graefen2007_wissenschaftliche_artikel}.

Mit dieser Veränderung änderte sich auch die Rolle des Autors und es wurden, im Gegensatz zum Mittelalter, auch solche Texte als wissenschaftliche Texte akzeptiert, deren "Garantie" in der Zugehörigkeit zu einem systematischen Ganzen - der Wissenschaft - bestand und nicht mehr nur aus dem Verweis auf das Individuum (Autoren) \cite{foucault_2000_autor}. Infolge dessen wurden Entdeckungen manchmal in Form eines Anagramms veröffentlicht, so etwa Galileis Entdeckungen der Jupitermonde \cite{miner2007discovery} und Hookes Elastizitätsgesetz \cite{szabo_2013_geschichte}. Auf diese Weise konnten Prioritätsrechte gesichert werden, ohne dass die Entdeckung selbst veröffentlicht werden mussten \cite{miner2007discovery}, Geheimnisse vor Diebstahl geschützt und religiösen Verfolgung vermeiden werden \cite{resnik_2005_ethics}. Erst ab Mitte des 19. Jahrhunderts "verlagerte sich die Produktion immer mehr auf das Hier und Jetzt" \cite{hagner_2015_sache_buches}.

Noch bis in das 19. Jahrhundert wurden dabei Bücher unter dem Eindruck einer "Unsterblichkeitsnorm geschrieben, die darauf baute, dass erst die Nachwelt das eigentliche Anliegen eines Buches verstehen würde" \cite{hagner_2015_sache_buches}.

Die wissenschaftliche Fachzeitschrift oder das wissenschaftliche Journal, wie wir es heute kennen, geht strukturell auf das 19. Jahrhundert in Zusammenhang mit der Konstruktion der moderne deutsche Universität zurück \cite{Paletschek_2002}, als die forscherischen Aktivitäten und das öffentliche Interesse an der Wissenschaft generell anstieg. In dieser Zeit kam es zu den meisten Gründungen der großen Fachzeitschriften von heute \cite{porter_1964_scientific}. Bis zur Etablierung des Peer-Review-Verfahrens als Qualitätssicherungsverfahren in der zweiten Hälfte des 20. Jahrhunderts gab es sehr unterschiedliche oder keine Verfahren zur Sicherung der Qualität von Inhalten in den Journalen. Im 20. Jahrhundert folgte auf die weltweite Intensivierung wissenschaftlicher Aktivitäten ein weiterer rasanter Anstieg der wissenschaftlichen Journale \cite[:23]{haustein_2012_multidimensional}. Im Jahr 1961 wurde die erste quantitative Studie an Hand der Anzahl von wissenschaftlichen Journalen durchgeführt. Im Rahmen dieser Erhebung wurden von 50.000 wissenschaftliche Zeitschriften und von einer Verdopplung der Anzahl aller wissenschaftlichen Journale alle 15 Jahre ausgegangen \cite{de_1982_little}.

\subsubsection{Rolle der Verlage und die Publikationskrise}

Ursprünglich wurde Wissen an Universitäten gespeichert, übertragen, verarbeitet, aufgezeichnet und später in wissenschaftlichen Journalen und Büchern gedruckt \cite{kittler_2004}. Dieses Wissen wurde in gleicher Weise verbreitet \cite{hagner_2015_sache_buches} und war Eigentum derer, die dafür schrieben oder es lasen \cite{epaa_Weiner_2001}. Sie wurden durch die wissenschaftlichen Akademien oder akademischen Fachgesellschaften, die die inhaltliche Ausrichtung verantworteten und die finanzielle Trägerschaft übernahmen \cite{suchen}, als Kommunikationsmedium organsiert. Erst im 20. Jahrhundert kam es zu einem Unterschied bei der Verbreitung verschiedener Veröffentlichungsformate innerhalb und zwischen den Fachdisziplinen \cite{hagner_2015_sache_buches}.

Mit dem weltweiten Anstieg der wissenschaftlichen Forschung Mitte des 20. Jahrhunderts und der stetig steigenden Anzahl wissenschaftlicher Publikationen nach dem zweiten Weltkrieg stieß das universitätseigene Journalsystem an seine Grenzen und es entwickelte sich zu einem "Flaschenhals" \cite{epaa_Weiner_2001} im Kommunikationssystem der Wissenschaft. Dem Anstieg an wissenschaftlicher Forschung und dem zunehmenden Publikationsdruck der Wissenschaftler konnte das System nicht mehr gerecht werden. Kommerzielle Verlage entdeckten diese Lücke und begannen den Markt mit Unterstützung der überforderten Universitäten zu absorbieren \cite{Hirschi_2015_buch_oa}.

Nachdem die Privatisierung- und Kommerzialisierung des System anfangs gut funktionierte, kam es zunehmend zu einem Bruch. Die Anforderungen des Marktes entsprachen nicht mehr denen der akademischen Gemeinschaft \cite{epaa_Weiner_2001}. Dennoch verharrte die wissenschaftliche Gemeinschaft in einem "weltfremden" Zustand, in dem der Druck zu veröffentlichen, dazu führte, dass sie ein System unterstützten, das sie ausnutzte \cite{epaa_Weiner_2001}. Sie sahen sprachlos mit an, wie die "Zeitschriften immer größere Anteile der Bibliotheksetats verschlangen" \cite{hagner_2015_sache_buches}. Auch in Deutschland nahmen Anfang der 1990er Jahre die wissenschaftlichen Verlage eine marktbeherrschende Stellung ein und agierten als exklusiver Distributor bei der Veröffentlichung wissenschaftlicher Informationen \cite{schloegl_2005} \cite{offhaus_2012_institutionelle_repos}.

Diese Entwicklung basiert auf dem in der Welt des geistigen Eigentums ungewöhnlichen Umstand, dass seit dem Beginn des wissenschaftlichen Journals im Jahr 1665, wissenschaftliche Autoren nicht vordergründig finanzieller Belohnung profitierten, sondern maßgeblich von der weiten Verbreitung und Hinweise auf ihre Arbeit, sowie die wissenschaftlichen Erkenntnisse ihrer Forschung \cite{albert_2006_open_implications}. Darüber hinaus ist es eine Besonderheit des Systems, dass Wissenschaftler sowohl Produzenten als auch Konsumenten der Wissenschaftskommunikation sind und damit Ihre eigene Zielgruppe darstellen \cite{Hess_2006}. Die kommerziellen Verlage haben sich dieses System zu nutze gemacht.

Zunehmend erlangten die Verlage eine Vormachtstellung im wissenschaftlichen Publikations- und Distributionssystem. Diese stützt sich bis heute auf drei Säulen \cite{offhaus_2012_institutionelle_repos} \cite{bargheer_2006_open}:
\begin{enumerate}
\item "Urheberrecht, wonach Verlage [...] weitgehende Ansprüche an dem veröffentlichten Werk erwerben“;
\item "redaktionelle Themenbündelung (bundling)“;
\item Organisation der "Qualitätssicherung durch Begutachtung (Peer Review)"
\end{enumerate}

Die marktbeherrschende Stellung der Verlage führte zu einer Situation, in der die Verlage vorerst im englischsprachigen Raum die Preise für wissenschaftliche Publikationen weitgehend diktieren und Preiserhöhungen unlimitiert durchgesetzt werden konnten. Als Folge der ungebremsten Ausnutzung dieser Marktmacht kam es kurz vor der Jahrtausendwende zur sogenannten "Zeitschriftenkrise"  \cite{Martin_2013} \cite{muller_2010_open} \cite{schirmbacher_2009_wisspub} \cite{Parks_2002_acadamic_faust}. Die Zeitschriftenkrise, "die richtigerweise Zeitschriftenpreiskrise oder Zeitschriftenpreisexplosion genannt werden müsste" \cite {Brintzinger_2010}, kam als Begriff das erste Mal in den 1990er Jahren auf \cite{Boni_2010}. Diese Krise war das Ergebnis folgender Entwicklungen auf der Angebots- und Nachfragenseite \cite{Brintzinger_2010}: Auf der Angebotsseite wurden durch einen "Konzentrationsprozess" "innerhalb von etwas mehr als einem Jahrzehnt im Bereich der Zeitschriften mittelständische Verlage nahezu vollkommen durch internationale Kapitalgesellschaften substituiert". \cite{Brintzinger_2010} Unterstützt von der Nachfrageseite resultierte daraus eine "monopolistische Preispolitik" der Verlage \cite{Brintzinger_2010}. Ein zeitgleicher Anstieg der Titelvielfalt, bei der aus "einer mehr generalistischen Zeitschrift drei oder vier Spezialzeitschriften" entstanden, "die dann allesamt wieder von Bibliotheken abonniert werden mussten" \cite{Brintzinger_2010}, verschärfte das Problem. Eine weitere Ursache für die krisenhafte Zuspitzung der Situation besteht in der institutionellen Organisation der Literaturbeschaffung an den Hochschulen und wissenschaftlichen Einrichtungen. Bei der Arbeitsteilung von Bibliothekaren und Wissenschaftlern war und ist es für das Ansehen des einzelnen Faches durchaus rational, mit einem möglichst hohen Anteil am Gesamtetat der Literaturbeschaffung zu partizipieren. Es gibt für individuelle Einsparungen von allen Seiten nur wenig Anlass, da beide Systeme unabhängig voneinander funktionieren. \cite{Brintzinger_2010}.

Die Preisexplosion konnte auch durch die Bildung von Bibliothekskonsortien, "deren Aufgabe es war, für Bibliotheken kostengünstige Rahmenbedingungen auszuhandeln", nicht gebändigt werden \cite{Fladung_2003} \cite{Brintzinger_2010}. Gleichzeitig standen die Wissenschaftler unter einem starken Publikationszwang, der mit "Publish or Perish" \cite{CLAPHAM_2005} beziehungsweise "impact factor fever" \cite{Cherubini_2008} und "impact factor race" \cite{Brischoux_2009} beschrieben wurde \cite{offhaus_2012_institutionelle_repos}. "Publish or Perish" beschreibt das Problem, dass im Rahmen der "wachsenden Konkurrenz um Forschungsförderung und akademische Positionen (...) kombiniert mit dem zunehmenden Einsatz bibliometrischer Parameter für Evaluation" \cite{Fanelli_2010} junge Akademiker viel und vornehmlich mit positiven wissenschaftlichen Ergebnissen publizieren müssen um Anerkennung und gegebenenfalls eine Anstellung im Wissenschaftsbetrieb zu erreichen \cite{pscheida_2010_wikipedia} \cite{Beasley_2005} \cite{hamilton_1990_publishing}. Das führte zu einer "beinahe explosionsartige Entwicklung der Anzahl wissenschaftlicher Publikationen" \cite{bortz_Doering_2006_fragestellung} und zu der damit einhergehenden Vermutung von viel "nutzlose Forschung und Artikel"\cite{smith1990killing}, einem „leeren Größenwachstum" \cite{bbaw_publizieren_2015} und viele wissenschaftliche Arbeiten mit "vernachlässigbare Beiträge zum Wissen" \cite{hamilton_1990_publishing}. Inwieweit diese Entwicklungen allein zu einer "Lawine von niedriger Qualität der Forschung" \cite{Bauerlein_2010} in dem beschriebenen Umfang geführt haben oder ob die neuen (digitalen) Möglichkeiten die schon immer bestehenden Qualitätsunterschiede wissenschaftlicher Publikationen einfach nur sichtbar gemacht haben, ist umstritten \cite{rekdal_2014_academic}.

Die genannten Entwicklungen machten dennoch mehrere der problematischen Effekten im Publikationssystem sichtbar: Erstens, die vermehrte Einreichung von Manuskripten bei begutachteten Publikationsmedien führte zu einer "schädlichen und vermeidbaren zusätzlichen Belastung der Begutachtung", zweitens erhöhen das "Größenwachstum" auf Seiten des Lesers "den Aufwand für Auswahl, Beschaffung und Lektüre von Publikationen" und drittens "steigen (...) die Kosten für das Publikationssystem insgesamt" \cite{bbaw_publizieren_2015}.

\subsubsection{Computer und Internet als neue Medien wissenschaftlicher Kommunikation}

Mit dem Aufkommen des Computerzeitalters in der zweiten Hälfte des 20. Jahrhunderts entwickelte sich der Begriff "Medien" zu einem Sammelbegriff, der in der älteren Medientheorie entweder als neutrale technische Infrastrukturen oder als Kommunikations-, Wahrnehmungs- oder kulturdeterminierende Techniken betrachtet wurde \cite{beck2005_Kommunikation}. Bei genauerer Betrachtung des Begriffs in den unterschiedlichen Disziplinen, die sich mit Medien beschäftigen, "sind die Gebrauchsweisen und Bestimmungen des Begriffs Medium äußerst heterogen" und "es hat den Anschein, als könnte die Frage, was Medien sind, zu keiner befriedigenden Antwort führen" \cite{Burkhardt_2015}.

Der Begriff digitale Medien hat in dieser Zeit das Denken über Medien nachhaltig beeinflusst  \cite{Burkhardt_2015}. Digitale Medien können als Medientechnologien bezeichnet werden, die durch Computer verarbeitet werden \cite{nunning_2013_metzler}. Durch die zunehmende Verbreitung des Computers und des Internets Ende der 1980er Jahre wurde dem Medienbegriff eine weitere Unbekannte hinzugefügt \cite{Burkhardt_2015}. Das Internet gilt dennoch als Paradigma für digitale Medien, da hier unterschiedliche Medien mehrfach vernetzt werden: Zum einen werden miteinander vernetzte Computer lokal und global über Telekommunikationskanäle miteinander verbunden, zum Anderen konvergieren in diesem globalen Netz Schriftlichkeit, Bild und Ton \cite{nunning_2013_metzler}.

Der Bestand der Rechenkapazitäten an Universitäten hat sich seit den 1989 konstant weiter verdichtet \cite{Rutenfranz_1997}. Ende des letzten Jahrtausends eröffnete das Internet "neue Nutzungsmöglichkeiten, durch welche die Schrift als ein Medium einsetzbar wird, das den permanenten Wechsel zwischen Sender- und Empfängerposition ähnlich flexibel zu gestalten erlaubt, wie es im gesprochenen Gespräch der Fall ist" \cite{sandbothe_2000_pragmatische}. Die Vernetzung schaffte auch in der Wissenschaft eine mediale Schnittstelle zwischen Autoren und Rezipienten, die keiner menschlichen Vermittlung durch Dritte (wie z.b. Verlage) mehr bedarf \cite{naeder_2010_open}. Mit der Etablierung eines globalen Kommunikationsnetzes ging auch die Vermutung einher, "dass im Internet als einem frei zugänglichen Medium mit geringen Zugangsbarrieren (...) Zugang zur Öffentlichkeit erhalten können, der ihnen bei den alten Medien verwehrt bleibt" \cite{Gerhards_2007}. Auch wenn sich im Internet bisher keine direkte demokratischere Kommunikation finden lässt \cite{Gerhards_2007}, so herrscht weiterhin große Euphorie bezügliches der verminderten Zugangsbarrieren, der umfassenden Möglichkeiten für die Vermittlung von Inhalten und die Transformation klassischer Kommunikationsmedien und Kanäle \cite{suchen}.

Digitale Souveränität und die Nutzung des Internets wird in Deutschland strukturell durch das Bildungsniveau und die erworbene Medienkompetenz bestimmt. Bei einer repräsentativen Befragung gaben 2014 92 Prozent der Teilnehmer mit abgeschlossenem Hochschulstudium an "Online" \cite{nonliner_2014}. Ganz pragmatisch ausgedrückt, gehören zum Einsatz digitaler Medien in den Geisteswissenschaften "die Nutzung von Textverarbeitungssoftware genauso wie die Recherche im Bibliothekskatalog mittels OPAC und die Informationsbeschaffung und Kommunikation mittels World Wide Web und E-Mail" \cite{naeder_2010_open}.

Im Zusammenhang mit der zunehmenden Verbreitung von Computer und Internet entwicklte sich der Webbrowser zu einer Kreuzung zwischen Buch und Fernseher, bei dem das multimediale Dokument von der Buchkultur als zentrales Wahrnehmungsobjekt übernommen wurde, zugleich aber darüber hinaus greift \cite{Warnke_2011}. Als weitere Veränderung in Abgrenzung zur Technologie Buchdrucks revidierte das Internet "die Vorstellung von einem geschlossenen Sinngehalt" \cite{sandbothe_2000_pragmatische} mit einem Anfang und Ende wie zum Beispiel in einem Buch.

Die Buchkultur wird von einer Dialogkultur abgelöst, aber nicht vollständig verdrängt. Das Gedruckte kommt demnach als eine Art Rückzugs- oder Entlastungsmedium zum Einsatz \cite{hagner_2015_sache_buches}. Dabei sind "wechselseitige Steigerungen, funktionale Kopplungen und vielfältige Kombinationen" zu erwarten und der damit einhergehende Medienwandel verändert vor allem "die bereits verbreiteten Medien und damit die medialen Verhältnisse einer Gesellschaft" \cite{Koenen_1997}, er verdrängt sie aber nicht zwangsläufig.

Die Entwicklung des Internets Ende des 20. Jahrhunderts waren eng mit der Idee verbunden, dass sie "Freiheit" sichert, bietet, verbessert, oder verstärkt. Der Computer, als Zugangsgerät zu digitalen Informationen, ermöglichte eine neue Form der Zusammenarbeit unterschiedlicher wissenschaftlicher Richtungen an einer gemeinsamen Arbeitsstation, eröffnet die Perspektive einer methodischen Integration unterschiedlicher wissenschaftlicher Betrachtungen und bietet die Chance der Vereinigung von bisher getrennten Notationssystemen von Alphabeten und mathematische Symbolen \cite{kittler_2004}. Doch auch nach 25 Jahren kann nicht abschließend evaluiert werden, inwieweit die Freiheit diesen Technologien innewohnt - und mit "free" nicht nur der Preis gemeint ist \cite{stallman2002} - und wie sie gestaltet werden kann \cite{kelty_2014_freedom}.

\subsubsection{Erste Experimente mit offenem Zugang zu wissenschaftlichen Publikationen}

Die Zeitschriftenkrise und der gestiegene Publikationsdruck stellen zwei fundamentale Gründe für das Aufkommen der Forderungen nach Öffnung des Zugangs zu wissenschaftlicher Literatur dar \cite{Brintzinger_2010} \cite{wein_2010_erwerbung}. Als Reaktion auf die Herausforderungen und auf Basis der Digitalisierung gründete Anfang der 1990er der Physiker Paul Ginsparg mit der Internetseite arXiv.org den ersten wissenschaftliche Preprint-Dienst des Internets \cite{cite:5} \cite{bjork_2004_open}, der es Wissenschaftlern ermöglichen sollte, Ideen vor der gedruckten Veröffentlichung zu teilen.

Ein Ausgangspunkt dafür waren die ersten Experimente mit offenem Zugang und freien Lizenzen für Publikationen in der Wissenschaft aus den 1960er Jahren und somit schon vor der Zeit der Erfindung des Internets \cite{cite:18b}. Noch bevor die digitalen Nutzungsmöglichkeiten verfügbar waren und bevor an das "globalen Dorf" \cite{mcluhan_1962_gutenberg} zu denken war, wurde vor allem in den Technik- und Naturwissenschaften eine “pre-print Kultur” entwickelt bei der die Autoren ihre zur Begutachtung eingereichten Artikel zeitgleich oder bevor diese veröffentlicht wurden unter Kollegen über den Postweg zirkulieren ließen, um den Kommunikationsprozess zu beschleunigen \cite{suchen-Hoffmann-Zugang-undVerwertung-oeffentlicher-Informationen}. Darüber hinaus gab und gibt es "informellen Wege des Zugangs" zu wissenschaftlichen Publikationen: zum Beispiel durch Kollegen an Institutionen die auf die Publikation zugreifen können oder durch die direkte Anfrage einer Kopie beim Autor \cite{davis_2011_open}.

Mitte der 1990er Jahre forderte Steven Harnad die wissenschaftliche Community dazu auf, sofort mit der digitalen Selbstarchivierung und öffentlichen Zurverfügungstellung ihrer Beiträge zu beginnen \cite{albert_2006_open_implications}, um "den Barrieren, die zwischen ihrer Arbeit und ihrer (kleinen) Leserschaft aufgestellt werden, zu entkommen" \cite{harnad_1995_subversive_proposal}.

Durch die zunehmende Verbreitung und Nutzung dieser digitalen Pre-Print Dienste, gründete sich im Oktober 1999 im Rahmen der "Santa Fe Convention" die "Open Archives Initiative", die sich maßgeblich mit den technischen und organisatorischen Aspekten der Transformation der wissenschaftlichen Kommunikation beschäftigte \cite{van_2000_santa_fe}.

2001 wurde der europäische Ableger von der Scholarly Publishing and Academic Resources Coalition (SPARC) einer der späteren "major player" der Open Access Bewegung \cite{russell2008business} \cite{Herb_2012} gegründet. Als Konsequenz aus der Zeitschriftenkrise sollte diese 1998 in den USA gegründete Allianz zwischen Universitäten und wissenschaftlichen Bibliotheken dafür Sorge tragen, dass die Kosten für wissenschaftliche Publikationen reduziert werden und durch die Bereitstellung kostengünstiger oder freier, nicht-kommerzieller, Peer-Review-Fachzeitschriften zu ersetzen sind. Durch Weiterbildung, politische Arbeit und die Förderung alternativer Geschäftsmodelle, war es Ziel von SPARC, Initiativen für offenes wissenschaftliches Publizieren zu stimulieren \cite{suchen}.

\subsubsection{Die Manifestierung der Forderung nach offenem Zugang}

In 2001 erschien Open Access erstmals im wissenschaftlichen Diskurs als öffentlichkeitswirksames Thema \cite{cite:19}. Die Public Library of Science (PLoS), gegründet im Oktober 2000, forderte die gesamte wissenschaftliche Gemeinschaft in einem offenen Brief im Mai 2001 dazu auf, ab September 2001 nur noch in Zeitschriften zu veröffentlichen, nur noch die Zeitschriften zu begutachten, zu editieren und zu abonnieren, deren Beiträge spätestens sechs Monate nach ihrer Erstveröffentlichung für jedermann im Internet kostenlos und unentgeltlich einsehbar sind \cite{cite:20}. Schon nach kurzer Zeit unterzeichneten nach eigenen Angaben \cite{cite:19a} rund 38.000 Wissenschaftler und Wissenschaftlerinnen aus 180 Nationen das Schreiben. Auf diesen Brief folgte ein 20-monatige sehr aktive und öffentlichkeitswirksame Phase der Forderung nach Öffnung der wissenschaftlichen Kommunikation. In diesen 20 Monaten wird neben PLoS der britische Verlag Biomed Central als weiterer "Wegbereiter in der von OA" \cite{suchen-Hoffmann-Zugang-undVerwertung-oeffentlicher-Informationen} gegründet und es entstehen drei der bis heute wichtigsten Erklärungen im Bereich der Öffnung des Zugangs zu wissenschaftlicher Kommunikation \cite{CREATe_2014}:
\begin{enumerate}

\item Erklärung der Budapest Open Access Initiative (Dezember 2002 und 2012)

Im gleichen Jahr wie der PLoS-Brief, wurden im Rahmen einer Konferenz des Open Society Institutes in Budapest, mit der “Budapest Open Access Initiative" (BOAI)\cite{boai_2002} erstmals die Bemühungen um Open Access in einer eigenen Erklärung zusammengefasst \cite{yiotis_2013_open} \cite{garcia_2010_open} \cite{cite:21a}. Im Fokus dieser Erklärung steht die Forderung nach freiem Zugang (ausschließlich) zu wissenschaftlichen Zeitschriftenpublikationen, "die zuvor einen Peer-Review-Prozess durchlaufen haben und anschließend, parallel zur Veröffentlichung in der Zeitschrift, im Netz frei zur Verfügung gestellt werden sollten" \cite{Schirmbacher_oa_2007}. In der BOAI wird erstmals manifestiert, dass wissenschaftliche Peer-Review-Fachliteratur "kostenfrei und öffentlich im Internet zugänglich sein sollte, so dass Interessenten die Volltexte lesen, herunterladen, kopieren, verteilen, drucken, in ihnen suchen, auf die Volltexte verweisen, sie indexieren, sie als Daten weiterverarbeiten und sie auch sonst auf jede denkbare legale Weise benutzen können, ohne finanzielle, gesetzliche oder technische Barrieren jenseits von denen, die mit dem Internet-Zugang selbst verbunden sind". \cite{boai_2002} Die Erklärung manifestiert auch, dass in "allen Fragen des Wiederabdrucks und der Verteilung und in allen Fragen des Copyrights überhaupt, sollte die einzige Einschränkung darin bestehen, den Autoren Kontrolle über ihre Arbeit zu belassen und deren Recht zu sichern, dass ihre Arbeit angemessen anerkannt und zitiert wird." \cite{boai_2002}

Die Erklärung manifestierte erstmals ein Bild davon, was eine Open Access Publikation von einer Veröffentlichung in einer herkömmlichen Fachzeitschrift und von einer kostenlosen, aber nur sehr eingeschränkt nutzbaren Digitalversion eines Artikels unterscheidet und eignet sich demnach als Anknüpfungspunkt für die Open-Access-Bewegung \cite{naeder_2010_open}. Sie bezog sich dabei explizit erst mal nur auf wissenschaftliche Zeitschriftenliteratur \cite{boai_2002}.

Anlässlich des zehnten Jahrestages der BOAI, wurde von der Open Society Foundation mit der BOAI 10 (2012) die ursprüngliche Erklärung um weitere Richtlinien und Empfehlungen für die Entwicklungen und Herausforderungen bei der Öffnung wissenschaftlicher Kommunikation ergänzt. Die Initiatoren kommen unverändert zu dem Schluss, dass "noch immer Zugangsbeschränkungen zu Peer-Review-Forschungsliteratur, meist eher zugunsten der Verlage, als zugunsten der Autoren, Reviewer oder Redakteure und damit auch auf Kosten der Forschung, Forscher und Forschungseinrichtungen" \cite{boai_2012} bestehen. Dazu heißt es in der überarbeiteten Erklärung: "Nichts aus den letzten zehn Jahren lässt darauf schließen, dass das ursprüngliche Ziel von Open Access weniger sinnvoll oder erstrebenswert erscheint. Im Gegenteil, die Notwendigkeit, dass Wissen für jeden, der es nutzen, anwenden oder darauf aufbauen kann, offen verfügbar sein sollte, ist dringlicher als je zuvor" \cite{boai_2012}. Darüber hinaus erfolgte auch eine Adaption der weiterführenden Aspekte der Stellungnahme von Bethesda und der Berliner Erklärung.

\item Die Bethesda Stellungnahme (Juni 2003)

Ein Jahre nach Veröffentlichung der initialen Version der BOAI-Erklärung, im Juni 2003, verabschiedete eine Gruppe von Forschungsförderern, wissenschaftlicher Gesellschaften, Verlegern, Bibliothekaren, Forschungseinrichtungen und einzelner Wissenschaftler im US-Bundesstaat Maryland das "Bethesda Statement on Open Access Publishing" \cite{suber_2003_bethesda}. Ziel der Erklärung war die Stimulation der Diskussion in der biomedizinischen Forschung, "wie man schnellstmöglich den offenen Zugang zu der primären wissenschaftlichen Literatur in der Biomedizin erreichen könnte" \cite{suber_2003_bethesda}. Ähnlich wie in der BOAI benannten die Autoren des "Bethesda Statements on Open Access Publishing" die Bedingungen für den offenen Zugang zu wissenschaftlichen Publikationen \cite{suber_2003_bethesda}:

Erstens werden Autor(en) und Urheberrechts-Inhaber aufgefordert, für alle Benutzer ein freies, unwiderrufliches, weltweites und unbefristetes Recht auf den Zugang zu genehmigen, sowie eine Lizenz zu verwenden, die das Kopieren, Nutzen, Verbreiten, Übertragen und öffentliches Darstellen der Publikation ermöglicht. Darüber hinaus soll es erlaubt sein, abgeleitete Werke zu verteilen und in jedem digitalen Medium für jeden Zweck zu veröffentlichen, vorbehaltlich einer angemessenen Zuordnung der Urheberschaft. Das beinhaltet auch das Recht auf eine kleine Anzahl gedruckter Kopien für den persönlichen Gebrauch.

Zweitens, muss eine vollständige Version der Arbeit und aller ergänzender Materialien, einschließlich einer Kopie der Genehmigung, wie oben erwähnt, in einem geeigneten elektronischen Standardformat unmittelbar bei der ersten Veröffentlichung in mindestens einem Online-Repositorium, das von einer wissenschaftlichen Einrichtung unterstützt wird, hinterlegt werden. Dieses Repositorium muss von einer wissenschaftlichen Gesellschaft, Regierungsbehörde oder einer anderen etablierten Organisation akzeptiert sein. Diese muss sich für einen offenen Zugang, uneingeschränkte Verbreitung sowie Interoperabilität und Langzeitarchivierung (für die biomedizinischen Wissenschaften, PubMed Central ist ein solches Repository) verpflichtend einsetzen.

Die Bethesda Stellungnahme ist in einigen Punkten präziser als die Budapester Erklärung, öffnet aber ihren Wirkungsraum auch auf Monografien und nicht-wissenschaftliche Publikationen. So enthält die Stellungnahme und der damit einhergehende Definitionsversuch Erweiterungen, die später in der Berliner Erklärung ebenfalls aufgegriffen werden, adressiert die Zugänglichkeit von im Rahmen der Publikationen erarbeiteten Zusatzmaterialien wie Mess- und statistische Daten, fordert das "Recht zur Erstellung und Publikation abgeleiteter Werke" (Derivate), "bindet Open Access unmittelbar an digitale Medien", schreibt sofort nach der Erstveröffentlichung die frei zugänglichen Veröffentlichung vor und rückt Open Access in die Nähe offener und freier Inhalte im weiteren Sinne \cite{naeder_2010_open}.

\item Die Berliner Erklärung (Oktober 2003)

Einen weiteren Meilenstein für die Verbreitung der Idee von Open Access auf dem europäischen Kontinent stellten die "Berlin Konferenzen" \cite{CREATe_2014} dar. Die erste Tagung wurde 2003 von der Max-Planck-Gesellschaft und dem Projekt European Cultural Heritage Online (ECHO) organisiert, um über "Zugangsmöglichkeiten zu Forschungsergebnissen" zu diskutieren. In diesem Rahmen entstand 2003 auch die "Berliner Erklärung über den offenen Zugang zu wissenschaftlichem Wissen" \cite{berliner_erklaerung_2003}, in der die Verfasser über die Budapester und die Bethesda Erklärung hinaus gehen und neben dem kostenlosen und freien Zugang zu wissenschaftlichen Endergebnissen in Form von Publikationen auch den freien und offenen Zugang zu wissenschaftlichen Daten fordern. „Open Access-Veröffentlichungen umfassen originäre wissenschaftliche Forschungsergebnisse ebenso wie Ursprungsdaten, Metadaten, Quellenmaterial, digitale Darstellungen von Bild- und Graphik-Material und wissenschaftliches Material in multimedialer Form.“ \cite{berliner_erklaerung_2003}

Mit dieser Ausweitung der Erklärung auf die Daten hinter den Publikationen, formiert sich erstmals ein klares erweitertes Verständnis von Open Access. Damit entsteht auch die erste Grundlage für ein erste Ansatzpunkte zur Eingrenzung des Open Science Begrifs, da hier der offene Zugang als eine "umfassende Quelle menschlichen Wissens und kulturellen Erbes, die von der Wissensgemeinschaft bestätigt wurden" \cite{berliner_erklaerung_2003} verstanden wird. Die Erklärung schließt damit jegliche wissenschaftlichen und nicht-wissenschaftliche Arbeiten ein, "unabhängig von Disziplin und Art der Publikation" und "jedweder Herkunft" \cite{naeder_2010_open}. Die Diskussionen um die Berliner Konferenzen konzentrieren sich in diesem Stadium aber dennoch hauptsächlich auf den bereits abgeschlossenen wissenschaftlichen Prozess und die finale wissenschaftliche Publikation.

Die Autoren der Berliner Erklärung erahnten die Bedeutung und möglichen Konsequenzen ihrer umfassenden Forderungen, sowie den Herausforderungen bei der Umsetzung. Nur so erklärt sich die "Diskrepanz zwischen der kompromisslosen Proklamation der Prinzipien und der durch vorsichtige Wortwahl geprägten "Unterstützung des Übergangs zum ‚Prinzip des offenen Zugangs’"" in der Praxis" \cite{Lossau_oa_2007}.
\end{enumerate}

Alle drei Erklärungen, auch die "three B's" genannt \cite{suber_2004_praising_oa}, gelten als die anerkanntesten Erklärungen von Open Access und stimmen in den wesentlichsten Merkmalen überein \cite{albert_2006_open_implications}, divergieren aber in Detailfragen \cite{naeder_2010_open}. Sie alle eint vor allem die Kernforderung nach der Beseitigung der preislichen und partiell der rechtlichen Barrieren bezüglich des freien Zugangs zu den wissenschaftlichen Publikationen. Sie alle haben keine zwar rechtlich bindende Interventionen und keine Sanktionsmechanismen, nutzen aber Anreizelemente für die Durchsetzung der definierten Forderungen. Weiterhin eint sie, dass alle drei Erklärungen ihre Ursprünge in den STM-Fächern haben und vornehmlich auf den Erfahrungen mit der Zeitschriftenkrise in diesen Fächern basieren \cite{naeder_2010_open}. Trotz der Unterschiede im Detail ähneln sich die Erklärungen auch bei geforderten Beseitigung der Barrieren für die kommerzielle Nutzung und die Erstellung von Derivaten \cite{CREATe_2014}. Die drei Erklärungen wurden darüber hinaus "von unterschiedlicher Seite vielfach präzisiert, interpretiert, eingeschränkt und erweitert" \cite{naeder_2010_open}, woraufhin sich eine "BBB-
Definition (Budapest-Bethesda-Berlin) von Open Access etabliert hat" \cite{Schirmbacher_oa_2007}. Diese wird in dieser Arbeit jedoch nur als ein weiterer grundsätzlicher Bezugsrahmen für die Annäherung an die Begrifflichkeiten von Open Access und Open Science betrachtet.

Schon ein Jahr vor der ersten Open Access Erklärung, in 2001, folgte die Entwicklung und 2002 die Veröffentlichung der ersten Creative Commons Lizenzen \cite{garcia_2010_open}. Diese Lizenzen waren inspiriert von den Lizenzen der freien Softwarebewegung und wurden kostenlos zur Verfügung gestellt \cite{Minjeong_2007}. Sie ermöglichten das freie Lizenzieren von Werken für bestimmte Verwendungen, unter bestimmten Bedingungen; oder ermöglichten die gemeinfreie Nutzung ohne Einschränkungen. Die Creative Commons Lizenzen bilden bis heute die urheberrechtliche Grundlage für eine Vielzahl der Open Access Publikationen weltweit \cite{suchen}. Im Oktober 2004 waren 5 Millionen Werke unter einer CC-Lizenz verfügbar \cite{Suchen_Forbes_Movement_Seeks_Copyright_Alternatives}. Nach eigenen Angaben von Creative Commons (https://stateof.creativecommons.org/) stieg die Anzahl der unter CC-lizensierten Werke auf 50 Millionen im Jahr 2006, 400 Millionen in 2010 und 882 Millionen in 2014. Seit 2010 ist auch ein Shift hin zu offenen Lizenzmodellen innerhalb der CC-Lizenzen ersichtlich. Waren 2010 noch 60 Prozent der 400 Millionen Werke unter den restriktiven CC-Lizenzen veröffentlicht, sank der Anteil in 2014 auf 44 Prozent. Die modularen Lizenzen sind im Kontext von Open Access besonders wichtig, "um (Nach-)Nutzungsmöglichkeiten für Texte, Daten und andere wissenschaftliche Erzeugnisse festlegen zu können" \cite{suchen-Hoffmann-Zugang-undVerwertung-oeffentlicher-Informationen}.

\subsubsection{Weitere Etablierung von Offenheit}

Im Jahr 2003 entstand das Portal Directory of Open Access Journals (DOAJ), das bis zum Jahr 2013 von der schwedischen Universität Lund betrieben wurde \cite{doaj_2015_about}. Das Portal ist eine zentrale Anlaufstelle für Open Access-Journale \cite{suber_2015} und "zielt darauf ab, Ausgangspunkt für qualitative und peer-reviewte open access Materialien zu sein" \cite{doaj_2015_about}. 2012 folgte dem DOAJ-Modell mit dem Directory of Open Access Books (DOAB) ein Portal für qualitätsgeprüfte Open Access Bücher und Monografien \cite{adema_2013_political}.

Anfang 2006 reagierte die Deutsche Forschungsgemeinschaft (DFG) auf die Entwicklungen und verabschiedete eine Richtlinie nach der sie zwar nicht voraussetzt, aber "erwartet", dass Publikationen aus DFG-geförderten Projekten "möglichst" als Open Access veröffentlicht werden \cite{suchen:dfg-richtlinie}. Eine ähnliche Erklärung verabschiedete auch die größte amerikanische Förderinstitution National Institutes of Health (NIH) und "stellte mit PubMed Central (PMC) eine entsprechende Plattform bereit \cite{muller_2010_open}. Anfangs wurde die "offene" Veröffentlichung unter den Kriterien und Bedingungen der Erklärungen, der Publikationen auf Grund eines Aufschreis der Verlage nur "empfohlen". Die Verlage sahen in der Richtlinie einen Untergang der wissenschaftlichen Qualitätssicherungsprozesse vorher \cite{Baggs_2006}. In 2008 wurde die Veröffentlichung NIH-geförderter Publikationen nach einer Embargozeit dennoch verpflichtend \cite{Hanekop_2014}. Aktuell gibt es in Deutschland keine zentrale Plattform wie PMC und die Veröffentlichung der geförderten Ergebnisse als Open Access ist weiterhin nicht bindend.

Auf die Entwicklungen folgten viele weitere "anerkenndende" weiche Erklärungen von unterschiedlicher Gruppen mit "Bekenntnissen", "Empfehlungen" und "Einladungen" das Ziel die Öffnung wissenschaftlicher Kommunikation zu fördern. Hier eine Auswahl der Dokumente:
\begin{itemize}
\item 2004 die "Declaration on Access to Research Data" \cite{oecd_2004} der Organisation für wirtschaftliche Zusammenarbeit und Entwicklung (OECD). Die Regierungen "erkennen in dieser Erklärung die Forderung zum Zugang zu wissenschaftlichen Publikationen aus steuerfinanzierter Forschung an" und "bekennen sich" zu der Notwendigkeit eines Zugang zu wissenschaftlichen Daten. In der Erklärung bekennen sich die OECD-Staaten (darunter auch Deutschland) darüber hinaus gemeinsame Regelungen für den Zugang zu digitalen Forschungsdaten aus öffentlicher Mittel unter Berücksichtigung sozialer, wissenschaftlicher und ökonomischer Interessen zu schaffen.
\item 2007 die "Kronberg Declaration on the Future of Knowledge Acquisition and Sharing" \cite{unesco_2007} der Organisation der Vereinten Nationen für Bildung, Wissenschaft und Kultur (UNESCO) adressiert generell das Thema Wissen, dessen Zukuft und sieht es als Schlüssel zu sozialer und wirtschaftlicher Entwicklung sowie die Veränderungen bei der Erstellung, Aneignung und Verbreitung des Wissens im Rahmen neuer Informationstechnologien. Sie entstand im Rahmen eines Treffens von einer Expertengruppe am 22 und 23 Juni 2007. Wie die vorherigen Erklärungen beinhaltet sie zwar weder konkrete Ziele noch Anreiz- oder Zwangmechanismen, setzt aber konkrete Anknüpfungspunkte und kommuniziert "Empfehlungen" an die Weltgemeinschaft für den Umgang mit Wissen in den nächsten 25 Jahren. Im Gegensatz zu den three B's fokusiert sie sich dabei nicht nur auf wissenschaftliche Kommunikation, fordert aber die weiter Unterstützung für Open Access und ebenfalls die Öffnung von Daten. Darüber hinaus hatten die Teilnehmer und Teilnehmerinnen der Arbeitsgruppe mehrheitlich einen wissenschaftlichen Hintergrund \cite{unesco_2007_list}.
\item 2007 veröffentlicht der Rat der Europäischen Union die "Council Conclusions on scientific information in the digital age: access, dissemination and preservation" \cite{eu_council_2007} und "lädt" die Mitgliedsländer und die europäischen Institutionen ein neue Strategien und Strukturen für die Verbesserung des Zugangs zu und die Sicherung und Verbreitung von wissenschaftlichen Informationen zu entwickeln.
\item 2012 setzt sich das "The Cost of Knowledge Manifesto" \cite{Gowers_2012} im Gegensatz zu den bisher genannten Erklärungen nicht direkt für die Verbesserung des Zugangs zu wissenschaftlichen Informationen ein, sondern gegen die Praxis eines konkreten Wissenschaftsverlages: Elseviers. Dazu startete der Mathematiker Gowers einen Boykottaufruf gegen die überhöhten Subskriptionspreise des Verlages, die Form der Bündelung von wissenschaftlichen Inhalten sowie gegen die politische Arbeit von Elsevier gegen die Verbreitung von Open Access. Über 15.000 Wissenschaftlerinnen und Wissenschaftler (Stand August 2015) haben das Manifest bisher unterzeichnet und öffentlich auf der Webseite angegeben, in welchen sie in Zunkuft die Zusammenarbeit mit dem Verlag einstellen wollen. Elsevier antwortete in einem Februar 2012 auf den Boykott \cite{elsevier_2012}, konnte aber die Vorwürfe nicht restlos ausräumen.
\end{itemize}

\subsubsection{Von Open Access zu Open Science}

Die zunehmenden Verbreitung des Internets, die zunehmenden Digitalisierung wissenschaftlicher Abläufe und die Möglichkeiten des kollaborativen Arbeitens über digitale Infrastrukturen haben die "praktischen und wirtschaftlichen Bedingungen für die Verbreitung von wissenschaftlichem Wissen und kulturellem Erbe grundlegend verändert" \cite{berliner_erklaerung_2003}. Diese Veränderungen ermöglichen erstmals nicht nur die Öffnung der wissenschaftlichen Erkenntnisse in Form von Publikationen, sondern auch die Daten und Informationen der gesamten wissenschaftlicher Kommunikation. In Ergänzung zu dem Konzept von offenem Zugang (Open Access) zu wissenschaftlichen Publikationen, erstreckt sich das Konzept der offenen Wissenschaft (Open Science) auf sämtliche Prozesse im Verlauf des wissenschaftlichen Erkenntnisprozesses.

Die 2010 veröffentlichten Panton Principles \cite{Mounce_2015} greifen einen Teil dieser Erweiterung des offenen Zugangs zu wissenschaftlichen Publikationen auf und ergänzen diesen um den offenen Zugang zu den (Roh-)Daten der jeweiligen Publikation. Sie folgen unter anderem der Annahme, dass andere Wissenschaftler und Wissenschaftlerinnen, sowie die Gesamtgesellschaft nur dann vollumfänglich von wissenschaftlicher Forschung profitieren kann, wenn auch der Kern der Forschung, die Daten auf der sie basiert, unter den Kriterien der Open Definition \cite{open_definition} zur Verfügung stehen.

Die Mehrzahl der bis dahin veröffentlichten Open Access Erklärungen bezogen sich auf die Öffnung der finalen wissenschaftlichen Publikationen (mit Ausnahme der Berliner Erklärung die auch auf die Öffnung von Daten eingeht), mit maximal geringfügiger Änderung des Kommunikationssystems. Open Science wiederum hat Zielt auf eine Transformation des gesamten wissenschaftlichen Erkenntnissgprozess. Die Europäische Kommission sieht in diesen Transformationsprozess vor allem in Hinblick auf die Demokratisierung von Forschung, neue Disziplinen und Forschungsthemen, die Symbiose aus Wissenschaft, Gesellschaft und Leitlinien und transparenter, reproduzierbarer Forschung.

---- TODO: Bild aus Europäische Kommission: Digital Agenda for Europe - Open Science Dreieck https://ec.europa.eu/digital-agenda/sites/digital-agenda/files/DS_6_0.png einbauen ----

Im Jahr 2013 konsultierte die Europäische Kommission in diesem Zusammenhang über 130 Vertreterinnen und Vertreter aus Forschung, Industrie, Forschungsförderung, Bibliotheken, Verlagen und Anbietern von Forschungsinfrastrukturen, um die Implikationen aus diesem raschen technologischen Wandel zusammenzufassen sowie Grundlagen für die kommende europäische Forschungsförderungsprogramm (Horizont 2020) zu definieren. Dabei stehen vor allem Forschungsdaten und die Digitalisierung wissenschaftlicher Kommunikation im Vordergrund. Aus Sicht der Forscher, umfassen Forschungsdaten alle Daten aus einem Experiment, Analyse oder Messung, einschließlich Metadaten und Details über die Verarbeitung der Daten \cite{eu_consult_data_2013}. Für Verlage, handelt es sich dabei eher ausschließlich um Daten, die mit der finalen Publikationen verknüpft sind \cite{eu_consult_data_2013}.

Soziale Medien, die technologischen (Weiter)Entwicklungen im letzten Jahrzehnt in Bezug auf Geschwindigkeit der Verbreitung von Informationen und Speicherkapazität für Daten ermöglichen erstmals die digitale Bereitstellung sämtlicher Erkenntnisse und Informationen die in der Wissenschaft gewonnen werden. Die Berliner Erklärung (siehe Kapitel xxx) nimmt diese Gedanken schon 2003 auf und ergänzt die Forderung nach offenem Zugang zu originäre wissenschaftliche Forschungsergebnisse um "Ursprungsdaten, Metadaten, Quellenmaterial, digitale Darstellungen von Bild- und Graphik-Material und wissenschaftliches Material in multimedialer Form" \cite{berliner_erklaerung_2003}.

Im April 2012 wurde die Erklärung "Open Science for the 21st century" vom Zusammenschluss der Europäischen Akademien (ALLEA) verabschiedet \cite{ALLEA_2012}. Sie war nur eine von mehreren Erklärungen und Positionspapieren für die Öffnung von Wissenschaft durch international angesehenen Einrichtungen, durch die verdeutlich wurde, dass die Forderung nach offenem Umgang mit Wissen und Information im wissenschaftlichen Bereich zunehmend an Relevanz gewinnt \cite{schulze_2013_open}.

2013 folgte mit der "San Francisco Declarationon Research Assessment" (DORA) \cite{DORA_2013} der öffentlicher Aufruf, nicht länger auf journal-basierte Metriken als Maß für die Messung der Qualität einzelner Forschungsartikel, die Beiträge eines einzelnen Wissenschaftlers, oder bei der Einstellung, Beförderung, oder Forschungsförderungsentscheidungen zu setzen. Die Erklärung fordert zudem Forschungsförderer auf, die gesamte Forschungsleistung und die Wirkung von Wissenschaftlern zu berücksichtigen. Dazu gehören neben der Publikation, auch die Datensätze und die Software sowie der Quellcode.

Beide Erklärungen adressieren auf unterschiedliche Art und Weise eindrücklich die Notwendigkeit für die Öffnung des wissenschaftlichen Erkenntnisprozesses weit über den reinen Zugang zu wissenschaftlichen Publikationen hinaus. Nur durch eine Öffnung des gesamten Prozesses wissenschaftlicher Forschung, so die Annahme, könne die Wissenschaft dem gesellschaftlichen Auftrag des Wissenschaftssystems im digitalen Zeitalter vollumfänglich gerecht werden und die Herausforderungen an das wissenschaftliche Kommunikationssystem gelöst werden.

\subsection{Ökonomie der wissenschaftlichen Kommunikation}

Die klassische Ökonomie der wissenschaftlichen Kommunikation beruht auf der Durchsetzung von Urheberrechten. Diese beschränken den Zugang und Zugriff auf, sowie die Wieder- und Weiterverwendung urheberrechtlich geschützter Inhalte. Leserinnen und Leser können nur gegen die Zahlung einer Gebühr Zugang zu der Veröffentlichung erhalten \cite{CREATe_2014}. Das gilt vor allem für die Veröffentlichung wissenschaftlicher Erkenntnisse.

Das wissenschaftliche Publizieren kann dabei als "gesellschaftlich bedingter Kreislauf" \cite{schirmbacher_2009_wisspub} betrachtet werden. Eine Besonderheit der Ökonomie wissenschaftlicher Kommunikation ist die Organisation des Marktes, die von spezifischen Akteuren und Prozessen geprägt wird \cite{Hess_2006}. Im Rahmen der formellen wissenschaftlichen Kommunikation und des wissenschaftlichen Verlagsgeschäfts, "ist es der Staat, der diesen Markt schafft" \cite{Hirschi_2015_buch_oa}. Die Ökonomie der wissenschaftlichen Kommunikation, ihre Akteure und Prozesse können wie folgt unterteilt werden \cite{cite:11b} \cite{Hess_2006}:
\begin{enumerate}
\item Erstellung von Inhalten durch Wissenschaftler und Wissenschaftlerinnen (Erstellung): Der Kreislauf beginnt mit der Anfertigung der geistigen Werke durch die Autoren \cite{schirmbacher_2009_wisspub}. Nach der Entwicklung eines konkreten Forschungsvorhabens sowie einer wissenschaftlichen Fragestellung entstehen im Rahmen der wissenschaftlichen Forschung oder der jeweiligen Untersuchung Daten \cite{cite:11c}, die im Forschungsprozess gesammelt, analysiert, ausgewertet, aufbereitet und verschriftlicht werden \cite{cite:11d}. Die Ergebnisse werden abschließend strukturiert zusammengefasst und niedergeschrieben \cite{Hess_2006}.
\item Qualitätskontrolle und die Bewertung von Inhalten (Bewertung):
Die Qualitätskontrolle ist wesentlicher Bestandteil der wissenschaftlichen Kommunikation. Sie sichert die gewonnen Erkenntnisse \cite{cite:11e} und stellt einen klaren Abgrenzungsaspekt zu nicht-wissenschaftlichen Informationen und Erkenntnissen dar \cite{cite:11f}. Sie findet im Kommunikationsprozess an zwei Stellen des Prozesses statt (siehe auch abschließende Aufnahme von Wissen). Bei der initialen Bewertung wird die Publikation der Erkenntnisse vom Verlag organisiert \cite{schirmbacher_2009_wisspub} und von anderen Wissenschaftlern überprüft und gesichert (Peer-Review) \cite{Hess_2006}.
\item Auswahl der Inhalte durch Verlage (Bündelung):
Die Verlage kuratieren in Zusammenarbeit mit anderen Wissenschaftlern die wissenschaftlichen Inhalte für die Publikation. Bei wissenschaftlichen Journalen werden zum Beispiel die eingereichten Beiträge gebündelt und in einer Ausgabe mit anderen Beiträgen zusammengefasst.
\item Publikation der Inhalte durch Verlage (Druck):
Nach Erstellung und Erkenntnissicherung findet die "eigentlichen Publikation" \cite{schirmbacher_2009_wisspub} der Informationen statt. Bis zur Digitalisierung bestand dieser Schritt ausschließlich aus dem Druck der Inhalte auf Papier.\cite{cite:11h} Im Rahmen der Digitalisierung besteht der Prozess in der Aufbereitung der Beiträge für die digitale Verbreitung.
\item Distribution der Inhalte durch die Verlage (Verbreitung):
Der Vertrieb und die Verbreitung von Forschungsergebnissen an die wissenschaftliche Community ermöglicht den Zugriff auf die Informationen durch andere Wissenschaftler. Dieser Schritt stellt einen essenziellen Teil der Zirkulation und Kommunikation des neu gewonnen Wissens dar \cite{cite:11i}. Er sichert die Verfügbarkeit, die Möglichkeit des Zugriffs auf die Informationen und ist Teil des Selektionsprozesses für die Erschaffung neuen Wissens \cite{cite:11l}.
\item Support und Archivierung (Archivierung): Erschließung, Aufbewahrung und Bereitstellung der Publikation durch Bibliotheken \cite{schirmbacher_2009_wisspub}. Die Bibliotheken unterstützen den Wissenschaftler und die Institution bei der Bewahrung und die Archivierung von Wissen \cite{K_lbel_2002}.
\item Konsum und Rezeption der Inhalte (Aufnahme von Wissen): In diesem Schritt wird durch den Vergleich neuer Ergebnisse mit bereits publizierten Inhalten, sowie die Diskussion der Ergebnisse in der Gemeinschaft, erneut die wissenschaftliche Qualität gesichert \cite{umstatter_2007_qualitatssicherung}. Die Rezeption der veröffentlichten Inhalte durch die wissenschaftliche (Fach-)Gemeinschaft ist damit der letzte Schritt des wissenschaftlichen Kommunikationsprozesses. Aus der Mitte der wissenschaftliche Gemeinschaft entsteht durch diese Verschriftlichung der wissenschaftliche Kommunikation und das Aufgreifen durch die Gemeinschaft neues Wissen \cite{cite:11k} \cite{schirmbacher_2009_wisspub} und der Kommunikationsprozess beginnt von vorn.
\end{enumerate}

An dem System der Wissenschaftskommunikation und dem Prozess des wissenschaftlichen Publizierens sind neben Fachgesellschaften, dem Buchhandel, Zeitschriftenagenturen und der Öffentlichkeit \cite[:6]{seidenfaden_2005_kommunikation} vor allem drei Gruppen beteiligt: erstens die Wissenschaftler, als Produzenten und Konsumenten der Informationen, zweitens die kommerziellen Verleger, die als Intermediäre wissenschaftliche Informationen sammeln, bündeln und verkaufen, sowie drittens die Bibliotheken, die die Informationen wieder den Wissenschaftlern zur Verfügung stellen \cite{Odlyzko_1997}.

Wissenschaftler stehen dabei an einer komfortablen Stelle des wissenschaftlichen Produktions- und Distributionssystems \cite{herb_2010}, da sie ausschließlich mit der Verarbeitung und Neuerstellung von Wissens beschäftigt sind. Den Erwerb der Publikationen übernehmen die Bibliotheken und mit der Distribution sind die Verlage befasst. Wissenschaftler und Wissenschaftlerinnen verfügen häufig über sehr gute Zugangsmöglichkeiten zu wissenschaftlichen Informationen durch ihre Forschungsinstitutionen \cite{cope2014future}. Aus dieser Position sind sie als Autoren und als Leser mit den finanziellen Herausforderungen beim Vertrieb von Wissen nicht konfrontiert. Sie werden an staatlichen, wissenschaftlichen Institutionen größtenteils durch öffentliche Gelder finanziert und erhalten durch die Bibliotheken ihrer Institution Zugang zu wissenschaftlichen Publikationen. Sie schreiben Texte für die Publikation in wissenschaftlichen Verlagen, und werden mit im Rahmen der Veröffentlichung mit Reputation "belohnt". In diesem Publikationskreislauf sind die Verlage die einzige voll-privatwirtschaftliche Gruppe, die Ressourcen aus dem System herauszieht, ohne dass diese Ressourcen vollumfänglich dem Kreislauf der Wissenschaftskommunikation wieder zugeführt werden \cite{kiley_2006_open}.

Wissenschaftliche Inhalte werden bisher vor allem über drei grundlegende Vertriebsarten zur Verfügung gestellt \cite{cope2014future}:
\begin{enumerate}
\item Wissen als Inhalt zum Verkauf: Der größte Anteil wissenschaftlicher Publikation wird über den Verkauf vertrieben. Allein für die STM-Fächer (Science, Technology, Medicine) wird von einem Markt von 10 Milliarden Dollar für wissenschaftliche, englischsprachige Zeitschriften und weitere 5 Milliarden Dollar für Bücher ausgegangen \cite[:6]{ware_2015_stm}.
\item Wissen als vollkommen "kostenlose" Ressource: Diese Art des Vertriebs folgt der Maxime, dass die Wissenschaft theoretisch die Verantwortung mit sich trägt, die größtmögliche Verbreitung zu erreichen. Damit sind in der Masse (bis auf einige Ausnahmen) meist noch gering verbreitete Vertriebsmodelle gemeint, bei denen der Leser kostenlos auf Inhalte zugreifen kann und auch dem Autor keine Kosten entstehen. Notwendige Erlöse für die Bereitstellung der Plattformen können hier über Werbung oder Zusatzdienste erzielt werden.
\item Wissen als bei der Produktion bezahlte Ressource: Ein wachsendes Modell für die kostenlose und offene Bereitstellung wissenschaftlicher Inhalte, bei dem der Autor oder die Autorin (oder die Förderinstitution) die Kosten für die Veröffentlichung und Verbreitung übernimmt.
\end{enumerate}

---- TODO: Grafik bauen siehe Terry & Kiley, 2006 oder http://www.uni-bremen.de/fileadmin/user_upload/forschung/Uploads_ProUB/Newsletter/Herb_Publikationsstrategien.pdf und http://southernlibrarianship.icaap.org/content/v09n03/mcguigan_g01.html ----

Für die Betrachtung der ökonomischen PRozesse im Rahmen dieser Arbeit steht die Diskrepanz im Vordergrund, dass "in der Regel wissenschaftliche Arbeiten zwar mit öffentlichen Mitteln finanziert, aber von privaten Verlagen in Fachzeitschriften herausgegeben" werden \cite[:9]{WD_bundestag_2009} \cite[:2]{Reichert_2009} und die zentrale Rolle der wissenschaftlichen Akteure in diesem System. Kritisch betrachtet basiert dieses System demnach auf einer "sozial ineffizienten" Grundlage \cite[:47]{mueller-langer_2010} bei der durch öffentlichen Gelder geförderte wissenschaftliche Arbeit exklusiv von privatwirtschaftlichen Verlagen vertrieben wird. Diese Ökonomie der Wissenschaftsverlage ist zwar nicht neu und hat sich im Laufe der Zeit, spätestens seit den 1960er Jahren weiter ausdifferenziert. Diese Wahrnehmung einer Unverhältnismäßigkeit in diesem System, insbesondere bei der Preisgestaltung für wissenschaftliche Publikationen \cite{King_2008} findet allerdings erst seit kurzem statt \cite{CREATe_2014} und wird als ein Grund für die Forderung nach Öffnung wissenschaftlicher Kommunikation erachtet \cite{yiotis_2013_open} \cite{herb_2010}.

\subsection{Digitalisierung der wissenschaftlichen Kommunikation}

Als Digitalisierung werden folgend Fortschritte im Kommunikationssystem bezeichnet, die durch die Entwicklung elektronischer Informations- und Kommunikationstechnologien angestoßen wurden \cite{bbaw_publizieren_2015}. Diese Fortschritte lassen das wissenschaftliche Publikationswesen und die wissenschaftliche Kommunikation nicht unberührt \cite{naeder_2010_open}. Wie bereits in der Einleitung dieser Arbeit dargestellt, üben die Digitalisierung und die dahinterstehenden Technologien einen tiefgreifenden Einfluss auf die wissenschaftliche Prozesse in allen Fachdisziplinen aus, die in Verlauf dieser Arbeit genauer untersucht werden.

Dieser Einfluss ergibt sich aus einer der wichtigsten Unterschiede der digitalen Kommunikation im Vergleich zu analogen Kommunikation. Digital kommunizierten Inhalte sind im Vergleich zu analogen Inhalten weder endgültig noch endlich und weder im Kern noch in Form fixiert, denn sie können leicht geändert werden und das ohne Spur von Löschung oder Korrektur \cite{smith_1999_digitize}. Aus digitalen Informationen können eine endlose Anzahl von identischen Kopien erstellt werden, ohne dass ein Zerfallsprozess eintritt \cite{smith_1999_digitize}. Ergänzt durch die Möglichkeit diese Informationen in einem weltumspannenden Netzwerk in nahezu Echtzeit unabhängig von Lokation und Zeit zu transportieren, haben diese fundamentalen Veränderungen für die Informationsspeicherung, -kommunikation und -verbreitung auch einen direkten Einfluss auf die wissenschaftliche Kommunikation, die bis zu diesem Zeitpunkt ausschließlich auf dem Austausch analoger Medien und Kommunikation basierte \cite{seidenfaden_2005_kommunikation}. Die weitgehende Verlagerung des wissenschaftlichen Kommunikationsprozesses in die digitale Welt führt dazu, dass mittlerweile über 90 Prozent der englischsprachigen Journale online verfügbar sind und es einen ansteigenden Trend zu Journalen gibt, die nur im digital abrufbar sind \cite{cope2014future} \cite{cite:5}.

Mit diesem digitalen Wandel in der wissenschaftlichen Kommunikation wird die Chance für eine umfassende “Beschleunigung des Wissensumschlages” \cite{Wenzel_2003} und die Möglichkeit einer im Prinzip unbegrenzten Verbreitung aller wissenschaftlichen Publikationen \cite{bbaw_publizieren_2015} \cite{yiotis_2013_open} auch an nicht-wissenschaftliche Zielgruppen \cite{Konneker_2013} verbunden. Durch die Digitalisierung und die neuen Möglichkeiten der Dissemination befindet sich bisher vor allem die \textit{externe} und \textit{informelle} Kommunikation im Wandel. Als Konsequenz dieses Wandels sind Wissenschaftler und Wissenschaftlerinnen heute in der Lage ihre Arbeiten öffentlich auf diversen digitalen Plattformen darzustellen, sich so von der Kommunikation über professionelle, wissenschaftliche Fachmedien zu "befreien" und direkt mit Teilen innerhalb und außerhalb der wissenschaftlichen Gemeinschaft zu interagieren \cite{Konneker_2013}.

Diese ersten Veränderungen sind mit der Hoffnung verknüpft, dass offene Innovation und offene wissenschaftliche Kommunikation, wie auch veränderte Zugriffsmöglichkeiten auf wissenschaftliches Wissen \cite[:109]{naeder_2010_open} den privaten und staatlichen Forschungsbereich offener, integrativer und effizienter machen können \cite{harmon_2012_commercialization}. Für die \textit{formelle} wissenschaftliche Kommunikation und das Publikationssystem fasst Johannes Nader das weitere Potenzial der Digitalisierung in folgenden vier Punkten zusammen \cite[:66-76]{naeder_2010_open}:
\begin{enumerate}
\item Ökonomische Effizienzsteigerung und Kostenersparnis: Platzersparnis und, abgesehen von der initialen Digitalisierung analoger Bestände, fallende Kosten für die Bestandserhaltung; verbesserte Verfügbarkeit
\item "Paradigmenwechsel bei der Archivierung": Effizienzsteigerung bei der Bestandserhaltung inklusive besserer Nutzung von Skaleneffekten und Dezentralisierung; identische Kopierbarkeit; Aufhebung der Nutzbestände und der Archivbestände; Trennung der Information von ihrem Trägermedium: nicht mehr die Langzeithaltbarkeit eines physischen Trägermediums ist ausschlaggebend, sondern die Erstellung von identischen Kopien
\item Veränderte und verbesserte Produktions- und Publikationsabläufe: neue Möglichkeiten der Textproduktion, -verarbeitung-, -überarbeitung und -transmission; Anreicherung von Inhalten; Autor zunehmend mit Gestaltung und Schriftsatz beschäftigt
\item Stabilisierung des wissenschaftlichen Kommunikationssystems: Kosteneinsparungen bei Produktion, Distribution, Zugänglichmachung und Archivierung; Funktionsverschiebungen vom Verlag hin zum Autor und Rezipienten lockern starre Publikations- und Erkenntnisketten
\end{enumerate}

Die Vernetzung im Rahmen der Digitalisierung ermöglicht erstmals eine direkten Schnittstelle zwischen Autoren und Rezipienten, die grundsätzlich keiner Vermittlung durch Dritte mehr bedarf. Als Konsequenz dieser Veränderungen obliegt es mehr denn je dem Leser und der Leserin aus einer größeren Menge an theoretisch verfügbaren Werke, die für ihn wichtigen Informationen zu identifizieren \cite{hagner_2015_sache_buches}, denn "verlagliche Mittler- und Selektionsinstanzen werden dadurch aus ihrer medialen Bindung gelöst und stehen zumindest in ihrer traditionellen Rolle zur Disposition." \cite[:109]{naeder_2010_open}.

Als ganz konkrete Veränderung erfolgte bisher mit der Etablierung der digitalen Kommunikation eine Veränderung der Kategorisierung wissenschaftlicher Kommunikation. Während im Druckzeitalter die formelle und \textit{interne} wissenschaftliche Kommunikation eng an die bibliometrischen Indikatoren geknüpft war und eindeutig von der informellen und externen abgegrenzt werden konnte, scheinen diese klaren Grenzen im Rahmen der Digitalisierung zu verschwimmen, auch wenn das "jedoch nur vermittelt und mit zeitlicher Verzögerung Wirkungen auf das formelle Publikationssystem zeigt" \cite{Hanekop_2014}. Hanekop definiert diesbezüglich den folgenden Zusammenhang: "Je größer die Abkopplung zwischen \textit{informellen} und \textit{formellen} Aspekten der wissenschaftlichen Kommunikation in einem disziplinären, thematischen oder nationalen Wissenschaftsbereich, um so geringer, vermittelter oder langwieriger kann auch die Wirkung des Internets auf diesen Teilbereich des Publikationssystems sein" \cite{Hanekop_2014}. Ben Kaden fasst diese Veränderungen im Kommunikationssystem als \textit{kanalerweiterte Wissenschaftskommunikation} zusammen und erklärt diese als "Form der Wissenschaftskommunikation, die die \textit{informelle} und \textit{formelle} ergänzt" und die "individuell affirmativ" als "als eine Art informelles offenes Post Review" verstanden werde kann \cite{kaden_2009_library}.

Diese neuen Formen und Kulturen der Kommunikation führen jedoch auch zu neue Fragen in Bezug auf mögliche Ungleichgewichte und Verzerrungen innerhalb des Wissenschaftssystems und erhöhen damit auch die Herausforderung, die Auswirkungen wissenschaftlicher Kommunikation zu standardisieren und zu messen \cite{gerber_2014_science}. Diese Herausforderungen werden im weiteren Verlauf der Arbeit weiter ausgeführt und vertieft betrachtet.

Der digitale Wandel bezieht sich folglich vor allem auf die folgenden drei Bereiche der formellen, internen wissenschaftlichen Kommunikation: die digitale Erstellung von Beiträgen und Texten, das Trägermedium der wissenschaftlichen Information und die Verbreitung, Vermittlung und Rezeption des Wissens \cite{bbaw_publizieren_2015}.

\subsection{Wissenschaftliche Kommunikation als Open-Source-Prozeß}

Wenn es im aktuellen öffentlichen Diskurs um die Öffnung wissenschaftliche Informationen, Infrastruktur und Arbeiten geht, werden immer öfter Schlagworte mit dem Attribut „Open“, wie Open Access, Open Research und Open Science, verwendet \cite{bunz_2014} \cite{schulze_2013_open}. "Offen" ist dabei nicht mit "kostenlos" gleichzusetzen \cite{grand_2012_open} und bezieht sich üblicherweise auf zwei Kernaspekte: Zum einen die Offenheit des Zugangs zu wissenschaftlichen Text, Daten, Quellcode oder Ergebnissen und zum anderen auf das Gebot der Transparenz, also die Offenlegung, beziehungsweise der Zugriff auf Verfahren, Methoden und Ziele \cite{schulze_2013_open}. "Offenheit" (Openness) wird im Rahmen dieser Arbeit multidimensional adressiert. Sie hat eine rechtliche, wirtschaftliche, technische, politische, sowie eine kulturelle Dimension.

Im Rahmen der Forderung nach der Öffnung der wissenschaftlichen Kommunikation und wissenschaftlichen Publikationen werden in der Literatur immer wieder Vergleiche zur Open Source-Bewegung gezogen  \cite{cite:9} \cite{Peters_2014} \cite{RIN_2010_open_research} \cite[:423]{mantz_2007_open} \cite{cite:1}. Diese Vergleiche dienen dabei beispielhaft dem Verständnis theoretischer Grundlagen im Rahmen der Öffnung von Wissenschaft und Forschung.

Open Source ist ein Begriff aus der Softwareentwicklung, der als Gegensatz zum Verfahren der Wissenssicherung \cite{stallman2002} eine quelloffenen Handhabe von Programmcode beschreibt und in den 1990iger erstmals eingeführt wurde \cite[:5]{hippel_2003_open}. Dieser Begriff wird praktisch, auch wenn es philosophisch enorme Meinungsunterschiede gibt \cite[:5]{hippel_2003_open}  \cite[:169]{stallman2002}, synonym mit “freier Software“ (nicht Freeware) verwendet \cite{naeder_2010_open} \cite[:414]{mantz_2007_open}. Dabei folgt die Open Source-Entwicklung der Maxime, dass die Kernsteuerungsinformationen und -befehle (Quelltext) von Software öffentlich einsehbar und zugänglich sind, sowie je nach gewähltem Lizenzmodell modifiziert, kopiert oder weitergegeben werden können. Der Unterschied zu Stallmans "Free Software" besteht hauptäschlich darin, dass Open Source Software Produktion nicht zwangsläufig ausschließt das Produkt kommerziell gegen Bezahlung mit properitären Erweiterungen zu vertreiben, während Free Software prinzipiell immer frei verbreitet werden muss \cite{stallman2002}.

Bei der Open Source-Entwicklung veröffentlichen Programmierer den Code einer Software offen im Internet. Andere Programmierer haben jeweils die Möglichkeit, diesen Code so weiterzuentwickeln und anzupassen, wie es ihnen beliebt. Dadurch entsteht ein offenes Ökosystem an Software - womit nicht zwangsläufig ein fertiges Programm gemeint sein muss - , bei dem nicht mehr der Zugriff die Hürde darstellt sondern die Adaption oder der Einsatz der vorhandenen Lösungen. Diese Entwicklungsmethode unterscheidet sich zum traditionellen Modell der Entwicklung von Software mit der Feststellung, dass Open Source-Software das Prinzip der Exklusivität des geistigen Eigentums auf den Kopf stellt, weil diese Software "um das Recht auf Vertrieb konfiguriert, nicht auszuschließen ist" \cite{suchen}. Auch wenn noch immer nicht vollständig geklärt ist, ob Open Source Software wirklich "schneller, besser oder günstiger" ist, hat sich Open Source in den letzten Jahren stark verbreitet \cite{Lerner_2001} und an Bedeutung gewonnen.

Die Definition von Open Source beinhaltet festgelegte Kriterien für die Klassifizierung \cite{osd_2003}: Freie Weitergabe ohne zusätzliche Kosten, das Programm muss den Quellcode beinhalten und den Code offen zur Verfügung stellen, die verwendete Lizenz muss Derivate zulassen, die Unversehrtheit des Quellcodes des Autors muss garantiert werden, die Diskriminierung von Personen oder Gruppen muss ausgeschlossen sein, es darf keine Einschränkung des Einsatzfeldes geben, die Lizenz muss weitergegeben werden können und auf das Produktpaket anwendbar sein und die Lizenz darf die Weitergabe des Programmcodes zusammen mit anderer Software nicht einschränken.

Im Vergleich zum klassischen Softwareentwicklungsprozess definiert der Hamburger Wirtschaftsinformatiker Markus Nüttgens folgende charakteristische Merkmale \cite{nuttgens_2014}:
\begin{enumerate}
\item Anzahl der beteiligten Entwickler: Im Vergleich zu traditioneller Softwareentwicklung ist eine weitaus größere Anzahl von Entwicklern beteiligt. Es gibt es keine klare Grenze zwischen Entwicklern und Anwendern, da die Hürden für eine Partizipation im Entwicklungsprozess sehr gering sind. Auch wenn ein großer Teil der Entwicklungsarbeit von Freiwilligen übernommen wird, gibt es dennoch den Trend zum Einsatz bezahlter Entwickler.
\item Zuteilung der Arbeit: Im Open Source Programming (OSP) wird die Entwicklungsarbeit nicht länger von einer definierten Instanz zugeteilt, sondern die Teilnehmer wählen ihre Arbeitspakete selbst aus.
\item Architektur: In der Regel orientierten sich die Teilnehmer eines OSP nicht an einer vorgegebenen System-Architektur. Die Gestaltung der Architektur geschieht dezentral und ist oftmals häufigen Richtungswechseln unterworfen.
\item Koordination: Es gibt wenig oder keine institutionalisierten traditionellen Koordinationsmechanismen, wie z.B. Projekt- und Zeitpläne, Lasten- und Pflichtenhefte u.ä.” \cite{suchen}
\end{enumerate}

Um die Logik, Sprache, Begriffe, Kategorien und Operationen die die (neuen) Medien charakterisieren zu können, bedarf es eine Verknüpfung und tiefere Auseinandersetzung mit Informationstechnolgie \cite[:65]{manovich_2001_language}. Die Verknüpfung der Open-Source Entwicklungsmethode mit der Forderung nach Öffnung von Technologie, Bildung und wissenschaftlichen Kommunikation wurde unter anderem von dem Literaturwissenschaftler und Medientheoretiker Friedrich Kittler manifestiert \cite{cite:1}, der darin eine Chance für den anhaltenden Überlebenskampf der Universität sieht \cite[:7]{chun_2006_new}.

Open Source Entwicklungsprozesse weisen auch insofern Konvergenzen mit der Forderung nach der umfassenden Öffnung wissenschaftlicher Kommunikation auf, als dass es in beiden Fällen nicht nur um den freien und offenen Zugang zum finalen Ergebnis geht, sondern um die Möglichkeit des Zugriffs im gesamten Verlauf des Erstellungsprozesses \cite{kelty_2004}. Die Open Source Entwicklungsprozesse unterscheiden sich von den klassisch-traditionellen (closed-source) Softwareentwicklungsprozessen insbesondere durch die transparente Präsenz und permanente öffentliche Einsehbarkeit. Adaptiert man diese Open Source-Prozesse auf wissenschaftliche Erkenntnisprozesse und definiert in diesem Zusammenhang wissenschaftliche Publikationen als Quellcode, ist das Konzept auf den wissenschaftliche Erkenntnisprozess mindestens teilweise übertragbar \cite{garcia_2010_open} \cite{Singh_2008} \cite{Bradley_2008} \cite{mantz_2007_open} \cite{dorschel_2006_open} \cite{Bradley_2007} \cite{Willinsky_2005}. Dass das System der offenen Softwareentwicklung dem System der Erkenntnisgewinnung in der Wissenschaft ähnelt, beruht auch auf der Parallele, dass in der Wissenschaft neues Wissen auf der Grundlage von bereits vorhandenem und verfügbaren Wissen entsteht. Das gilt ebenso für Open-Source Entwicklungen, bei denen Entwickler und Entwicklerinnen häufig auf Softwareteile anderer zurückgreifen.

Ähnlichkeiten bestehen ebenso bei der Motivation für die Erstellung offener Software und für den wissenschaftlichen Erkenntnisgewinn. Zusammenfassend sind beide Prozesse in nachfolgend aufgelisteten Aspekten einander ähnlich:
\begin{enumerate}
\item Wie bei der wissenschaftlichen Kommunikation, baut die Entwicklung vieler Open Source Projekten auf den Inhalten, Steuerungsinformationen und Erfahrungen anderer Projekte auf. Die Projekte profitieren dabei von einem ständigen Austausch von Informationen, gegenseitiger Optimierung und Verbesserung. Wie bei Open Source Software streben auch Wissenschaftler nach der größtmöglichen Verbreitung ihrer Inhalte.
\item "Free Software (im Sinne von Open Source), Open Access und Creative Commons sind alles Rechts- und Infrastrukturexperimente"\cite{kelty_2004}. Open Source-Software sollte dabei nicht mit "Shareware" verglichen werden, die zwar kostenlos verbreitet wird, aber deren Quellcode proprietär bleibt \cite{Lerner_2001}
\item Die Kontributoren von Open Source Projekten versprechen sich neue "Karrieremöglichkeiten oder eine Ego-Genugtuung" \cite{Lerner_2001}, Selbstverwirklichung oder Befriedigung der intellektuellen Neugier \cite{Willinsky_2005}, sowie gegenseitige Beurteilung und Anerkennung (non-monetäres Kapital). Das wissenschaftliche System basiert auf ähnlichen Mechanismen beim Karriere- und Reputationssystem.
\item Parallelen ergeben sich auch auf der Nutzerseite: "Denn hier wie dort gilt es, das Spannungsfeld zwischen dem Prinzip des „offenen Zugangs“ auf der einen Seite und dem Wunsch mancher Urheber, die Nutzung seines Werkes – teils aus ideellen, teils aus ökonomischen Motiven – auf bestimmte „gewünschte“ Nutzungsformen zu beschränken" \cite{dorschel_2006_open}.
\item Wie bei der wissenschaftlichen Kommunikation, geht es bei der Mitarbeit an Open Source Projekten nicht ausschließlich um altruistische Motive \cite{Lerner_2001} und um kollektive bzw. arbeitsteilige Prozesse zur Wissensproduktion.
\item Die Debatte um die Forderung nach Öffnung der wissenschaftlichen Kommunikation kann aus technologisch-entwicklungsmethodischer Sicht mit der Debatte um kostenloser Software (Freeware) versus Open Source Software verglichen werden. Der Vergleich: Freeware und Open Access Publikationen sind zwar kostenlos verfügbar, ihr Erstellungsprozess wird jedoch nicht offen und transparent kommuniziert. Bei Open Science geht es wie bei Open Source um die Offenlegung des gesamten Erstellungsprozesses inklusive der Daten \cite{grand_2012_open} sowie auch den benutzten wissenschaftlichen Code \cite{hey_2015_open}.
\item Auch die häufig genutzten Lizenzmodelle und Definitionen von Offenheit im Rahmen wissenschaftlicher Kommunikation haben ihren Ursprung in der Open-Source-Bewegung \cite{suchen}.
\end{enumerate}

Der Vergleich der Öffnung von Wissenschaft mit der Open-Source Bewegung wird im Rahmen dieser Arbeit als ein Ansatzpunkt erachtet, um ein mögliches Szenario aufzuzeigen, wie in Zukunft die Wissensproduktion frei und öffentlich gestaltet werden kann. Dabei gilt es zu berücksichtigen, dass bei der Nutzung Open Access publizierter Werke und Publikationen die Erfahrungen aus dem Bereich der Open Source-Software dienlich sein können, wobei sich die die rechtlichen Fragestellungen und Lösungsansätze auf Anbieterseite doch zum Teil erheblich unterscheiden \cite{dorschel_2006_open}. Diese Einschränkung resultiert aus einer partiell differenten Interessenlage: OpenSource-Software basiert in hohem Maße auf dem Community-Gedanken und ist letztlich altruistischen Motiven geprägt, während bei Open Access zum Teil die Ressourcenknappheit der öffentlichen Hand sowie die individuellen Renommeeinteressen des Wissenschaftlers im Vordergrund stehen \cite{dorschel_2006_open}. Im Rahmen der Veränderungsprozesse und Ausweitung der Öffnung auf den gesamten wissenschaftlichen Erkenntnisprozesses muss diese Einschränkung der Vergleichbarkeit allerdings hinterfragt werden, was im weiteren Verlauf dieser Arbeit dargestellt wird.

\subsection{Die Forderung nach Öffnung der wissenschaftlichen Kommunikation}

Während "Openness" bisher vielfach mit den Entwicklungen im Zusammenhang mit offener Software assoziiert wird, gibt es Anknüpfungspunkte von "Offenheit" als Begriff in der wissenschaftlichen Auseinandersetzung, die schon zeitlich früher anzusetzen sind \cite{Tkacz_2014}. So sieht Christopher Kel­ty die ersten Anfänge bereits in den 1980er Jahren \cite{kelty_2008_two_bits}. Andrew Russell sieht die ideologischen Ursprünge von "Offenheit" als Standard schon in der Entwicklung des Telegraphs und weiteren Ingenieurleistungen seit 1860 \cite{Russell_2014}. Könneker und Lugger sehen erste Beispiele einer offenen Wissenschaft bereits im 17. Jahrhundert \cite{Konneker_2013}. Zu dieser Zeit herrschte noch keine strikte Trennung zwischen Wissenschaftlern und nicht-Wissenschaftlern und die "öffentliche Demonstrationen von Experimenten mit großem Überraschungs-und Unterhaltungswert beziehen das Publikum ein" \cite{weingart_2005_wissenschaft}.

Das aktuell vorherrschende System der wissenschaftlichen Kommunikation hat sich seit den 1960er Jahre etabliert und funktionierte am Besten, als die akademischen Ziele und mit den Marktinteressen vereinbar waren. Doch die Rahmenbedingungen wissenschaftlicher Kommunikation haben sich seitdem fundamental verändert \cite{epaa_Weiner_2001}. Infolge des weltweit steigenden Haushaltsdrucks der Bibliotheken und wissenschaftlichen Institutionen, des "ungewöhnlichen Geschäftsmodells" \cite{cite:12} der Wissenschaftsverlage mit immer höheren Margen \cite{albert_2006_open_implications}, der Massifizierung der Universität \cite{binswanger_2014_excellence}, des konstanten Anstiegs des wissenschaftliche Outputs \cite[:23]{haustein_2012_multidimensional} und des Umstandes, dass private Wissenschaftsverlage durch das wissenschaftlichen Reputationssystem über öffentlich finanzierte Wissenschaftlerkarrieren entscheiden \cite{heise_2012}, befindet sich das wissenschaftliche Kommunikationssystem in einer Krise \cite{cite:14}.

Im Rahmen der technologischen Entwicklungen bei der Digitalisierung des wissenschaftlichen Arbeitens und elektronischen Publizierens kann die Öffnung der wissenschaftlichen Kommunikation als eine mögliche Antwort auf diese Krise verstanden werden, setzt bei der Öffnung (Open) und dem freien Zugang (Access) zu wissenschaftlichen Publikationen an und könnte perspektivisch zu einer Öffnung (Open) des Zugriffs auf den Prozess des Forschens (Science) führen. Darüber hinaus werden wissenschaftliche Ergebnisse zunehmend zum Thema massenmedialer Berichterstattung, wodurch sich der ursprüngliche Publikumsbezug der wissenschaftlichen Kommunikation zur jeweiligen Fachgemeinschaft um den Bezug zur allgemeinen Öffentlichkeit ergänzt" \cite{bbaw_publizieren_2015} .

Dieser Wandel im "Zeitalter der Informatik", birgt aber auch Herausforderungen, die der Philosoph Jean-François Lyotard als "Ökonomisierung des Wissens" \cite{lyotard_1993_postmoderne} auch im Rahmen der zunehmenden Quantifizierbarkeit bezeichnet. Die Produktion von Wissen erfolgt demnach nicht mehr (nur) zum Zweck der Erweiterung des Wissens, sondern mit dem Ziel es zu verkaufen \cite[:156]{troy_2012_wissen}. Wie im Kapitel "Wissenschaftliches Kapital" beschrieben beruht die Produktion von wissenschaftlichem Wissen aber eben ursprünglich nicht auf den Marktmechanismen, sondern auf einem nicht-kommerzielle Anreizsystem und wird durch symbolisches Kapital angetrieben \cite[:157]{troy_2012_wissen}.

Aus der Öffnung der Kommunikation, so wird befürchtet, entsteht die Gefahr, dass wissenschaftliches "Wissen immer weniger der Bildung dient, sondern für den Verkauf geschaffen und konsumiert wird" \cite{hagner_2015_sache_buches}. Diese Gefahr beschreibt das Spannungsverhältnis in dem Wissenschaftler und Wissenschaftlerinnen auf der einen Seite zunehmend angehalten sind, die Forschung gemeinsam mit der Industrie schnell in Produkte zu übersetzen und auf der anderen Seite das Wissen so schnell wie in der wissenschaftlichen Gemeinschaft verbreitet werden soll, um den wissenschaftlichen Fortschritt zu fördern sowie um die gesellschaftlichen und humanitären Ziele von Wissenschaft zu erfüllen \cite{harmon_2012_commercialization} \cite{Woelfle_2011}. Diese Gefahr wird auch in der Debatte um die Öffnung wissenschaftlicher Kommunikation genannt \cite{Kansa_2014_open_neoliberalism} \cite{bunz_2013_open} \cite{tkacz_2012_open} \cite{mirowski_2005_contract} und wird im weiteren Verlauf der Arbeit aufgegriffen. Eine weitere Herausforderung stellt die Frage dar, ob und in wie weit durch die Öffnung der wissenschaftlichen Kommunikation massenmediale Selektionskriterien als Steuerungsmechanismen für Wissenschaft wirksam gemacht werden \cite{bbaw_publizieren_2015}.

Diese Veränderungen in der Kommunikation von Forschung und Wissenschaft sind keine völlig neue Phänomene, denn in gewisser Weise ist die Öffnung der wissenschaftlichen Kommunikationsprozesse eine Rückkehr zu der ursprünglichen Beziehung zwischen Wissenschaft und Öffentlichkeit \cite{Konneker_2013} \cite{weingart_2005_wissenschaft}.

In der gegenwärtigen Literatur kommen die Begriffe um "Offenheit" in der wissenschaftlichen Auseinandersetzung auf unterschiedlichste Art und Weise zur Anwendung \cite{cite:9}. Die Unterscheidung von "Zugang" und "Zugriff" erscheint dabei wesentlich und stellt eine der zentralen Grundlagen für die Abgrenzung der hier verwendeten unterschiedlichen "Open"-Begriffe dar:
\begin{itemize}
\item Offener "Zugang" bezieht sich auf einen unbeschränkten Zugang zur finalen wissenschaftlichen Publikation. Zugang meint das "das freie, unwiderrufliche und
weltweite Zugangsrecht" \cite{berliner_erklaerung_2003}. "Unbeschränkt" meint hier vor allem das ausschließliche Lesen der finalen Ergebnispublikation aber auch die Erstellung von Kopien, sowie Verarbeitung und Benutzung dieser \cite{Lossau_oa_2007} bei Nennung der Urheberschaft. Dieser Open-Access-Ansatz bezieht sich zunächst lediglich auf die Zugangsbedingungen zu den wissenschaftlichen Arbeiten \cite{muller_2010_open}. Dabei bezieht sich dieser Zugang auf einen Zeitpunkt, nach der Entwicklung und Veröffentlichung des bereits abgeschlossenen wissenschaftlichen Erkenntnisprozesses und die Publikation bereits eingereicht oder veröffentlicht wurde.
\item Offener "Zugriff" soll als erweiterte Nutzung der jeweiligen Wissensressourcen verstanden werden und schließt neben dem "Zugang" zur Publikation sämtliche Informationen und Daten, Quellcode, sowie die Kommunikation hinter und vor der finalen Veröffentlichung \cite{hey_2015_open} ein. Dieser Zugriff bezieht sich als Erweiterung zu den ersten Forderungen nach "Open Access" auch auf "Daten" als "Gesamtheit der binär codierten, maschinenlesbaren Inskriptionen" und "all das, was auf digitalen Datenträgern gespeichert vorliegt" \cite{Burkhardt_2015}. "Zugriff" beschränkt sich hier also nicht nur auf den reinen Zugang zu wissenschaftlicher Information im Rahmen des Publikationsprozesses, sondern schließt auch den Zugriff auf sämtliche Forschungsdaten, Methoden und wissenschaftlichen Begleitinformationen, die während der wissenschaftliche Arbeit auf dem Weg zur finalen Publikation entstehen ein und ermöglicht die Weiternutzung, Weiterverarbeitung sowie die Erstellung von Derivaten durch Dritte. Im Unterschied zum "Zugang" geht es dabei auch um einen "Zugriff" auf die Informationen der weit vor den Zeitpunkt der finalen Einreichung oder Publikation liegt und unmittelbar mit Beginn des wissenschaftlichen Erkenntnisprozesses beginnt.
\end{itemize}

\subsubsection{Offener Zugang zur wissenschaftlichen Publikation: Open Access}

\begin{quote}
Der offene Zugang, auch Open Access, bedeutet, dass Peer-Review-Fachliteratur kostenfrei und öffentlich im Internet zugänglich sein sollte, so dass Interessenten die Volltexte lesen, herunterladen, kopieren, verteilen, drucken, in ihnen suchen, auf sie verweisen und sie auch sonst auf jede denkbare legale Weise benutzen können, ohne finanzielle, gesetzliche oder technische Barrieren jenseits derer, die mit dem Internet-Zugang selbst verbunden sind. In allen Fragen des Wiederabdrucks und der Verteilung sowie in allen Fragen des Copyrights sollte die einzige Einschränkung darin bestehen, den Autoren Kontrolle über ihre Arbeit zu belassen und deren Recht zu sichern, dass ihre Arbeit angemessen anerkannt und zitiert wird.
\end{quote} \cite{boai_2012}

Der offene Zugang zu wissenschaftlicher Kommunikation ist seit der Entwicklung des gedruckten Wortes eng mit der Frage nach Urheberrechten für wissenschaftliche Informationen verknüpft \cite{Case_2000}. Open Access beschreibt ein wissenschaftliches Kommunikationssystem, in dem der Zugang zu den unterschiedlichsten Formen wissenschaftlicher Publikationen, im Gegensatz zum bestehenden System, unter freien, kostenlosen Bedingungen und ohne finanzielle, gesetzliche oder technische Hürden möglich ist \cite{WD_bundestag_2009}. Dieses System ermöglicht darüber hinaus ein "alternatives Geschäftsmodell"\cite{lewis_2012_inevitability} für wissenschaftliche Publikationen. Was auf Maßgabe beruht, dass der Autor die "Eigentumsrechte an den Artikeln, die bisher für die Publikation in wissenschaftlichen Journals an die jeweiligen Fachverlage abgetreten wurden, (...) nun bei den Autoren der Artikel selbst verbleiben" \cite{Hess_2006}.

"Geringere Kostenbarrieren und damit eine einfachere Verbreitung ihrer eigenen Werke" \cite{WD_bundestag_2009} stellen dabei die Wünsche der wissenschaftlichen Autoren und Urheber an Open Access dar. Der Einsatz (offener) Lizenzen ist dafür ein Haupteinflussfaktor \cite{cite:16}. Open Access hat damit den Zweck, die durch Copyright generierten Barrieren zu überwinden und möglichst schnell und umfassend Zugriff auf neue wissenschaftliche Erkenntnisse zu ermöglichen.

\subsubsection{Offener Zugriff auf den wissenschaftlichen Prozess: Open Science}

"Open Science" knüpft an die Entwicklung der Ideen der Open Access-Bewegung an \cite{garcia_2010_open}. Beschränkte sich die Idee von Open Access vorerst auf den offene Zugang zur finalen wissenschaftlichen Publikation, wird das Ziel von Open Science im Folgenden darüber definiert, wie der gesamte wissenschaftliche Erkenntnisprozess der Allgemeinheit offen zur Verfügung gestellt werden kann \cite{grand_2012_open}. Open Science kann folglich zum einen als Folge der neuen Möglichkeiten für kollaboratives Arbeiten im Rahmen der Digitalisierung und neuer Kommunikationstechniken und zum Anderen als Schritt hin zu einer "geistigen Allmende" \cite{naeder_2010_open} verstanden werden.

Der Sammelbegriff Open Science erstreckt sich dabei über den gesamte wissenschaftlichen Forschungsprozess \cite{Scheliga_2014}: Vom offenen Zugang zu Publikationen wissenschaftlicher Forschung (Open Access), sowie den ganzheitlichen wissenschaftlichen Erkenntnisprozess umfassend. Unter diesem Gesichtspunkt kann Open Science als eine Weiterentwicklung von Open Access bezeichnet werden. Die diesbezügliche Evolution des Konzepts von Open Access führt zu einem direkten und breiten Weg, Wissenschaft an jedem Schritt des wissenschaftlichen Erkenntnisprozess zu kommunizieren und zu transferieren. Open Science ist die Reaktion auf die Forderung nach offenem Zugriff auf Wissenschaft und Forschung und kann dazu führen, "dass sich die Bedeutung von Forschungsergebnissen zukünftig nicht mehr auf sogenannte klassische wissenschaftliche Publikationen (im Format von Einleitung – Methoden – Ergebnisse – Diskussion), sondern die globale Echtzeitpublikation von Originaldaten stützen wird" \cite{Stengel_2013}.

Wie Open Access hat die Bewegung für Open Science ihre Dynamik der zunehmenden Verbreitung des Internets Anfang der 1990er Jahren zu verdanken \cite{Lievrouw_2010} sowie der neuen Möglichkeiten des kollaborativen Arbeitens sowie des Teilens von Daten und Informationen über das globale Netzwerk \cite{Meyer_2013}. Diese technologischen Entwicklungen ermöglichen jedoch nicht nur das kollaborative Arbeiten zwischen Wissenschaftlern in aller Welt, sondern schaffen auch die Möglichkeit die ausgetauschten Informationen nicht nur unter Wissenschaftlern zu teilen, sondern die Verbreitung wissenschaftlicher Informationen an die Gesamtgesellschaft. Befürworter von Open Science sehen hier eine Möglichkeit die gesamten wissenschaftlichen Prozesse, von der Idee bis zur Abschlusspublikation, transparenter, effizienter, nachvollziehbarer und offener zu gestalten \cite{Woelfle_2011} und diese weltweit auch in unterentwickelten Regionen zu verbreiten \cite{yiotis_2013_open}.

Diese Vision einer offenen Wissenschaft steht der Verschlüsselungs- und Patentwut zur Wahrung der Geschlossenheit der wissenschaftlichen Informationen und eines möglichen kommerziellen Vorteils durch Wissenschaft im Rahmen öffentlich-finanzierter Forschung entgegen und führt zu einer Debatte über die Verfügbarkeit der wissenschaftlichen Arbeit und die Entlohnung der "Erfinder" im wissenschaftlichen System \cite{suchen}.

\section{Wissenschaftliche Reputation, das Ethos und der Diskurs}

Wissenschaftliche Reputation, wissenschaftliches Kapital, das wissenschaftliche Ethos und der Diskurs sind sich bedingende Pfeiler des wissenschaftlichen Kommunikationssystem. Sie vereinen stukturelle Grundlagen, Verhaltensrichtlinien und Anreizmechanismen und für die Produktion von Wissen. Die genauere Betrachtung dieser Aspekte ist eine Vorraussertung für das Verständnis von Veränderungsprozessen sowie deren Treiber und Bremser.

\subsection{Wissenschaftliche Reputation}

Wissenschaftliche Reputation kann als eine "Art von Kredit" \cite{luhmann_1970_selbststeuerung} verstanden werden, mittels derer “Status und Ressourcen verteilt werden” \cite{hanekop_2006}. Diese Währung basiert auf der "gegenseitigen Beurteilung und Anerkennung der jeweils neuen Ergebnisse der Fachkollegen (Peers) durch die Wissenschaftler selbst" \cite{Hanekop_2014} \cite{suchen_Hornbostel_2006}, teils auf der "Generalisierung von Einzelleistungen", auf "gegenseitiger Ansteckung" und teils "auf der bloßen Häufigkeit der Publikationen oder der Anwesenheit an renommierten Plätzen" \cite{luhmann_1970_selbststeuerung}. Dabei gesteht auch Luhmann die Existenz von "Nebencodes der Reputation" zu \cite{schmoch_2003_hochschulforschung}.

Die Reputation steuert die wissenschaftliche Aufmerksamkeit und die Verteilung motivierender Effekte, die sich durch das alleinige Streben nach Erkenntnis nicht erzeugen lassen \cite{suchen_luhmann}. Die akademische Reputation „ist die zentrale Kommunikationsform, die das Wissenschaftssystem charakterisiert“ \cite{Rutenfranz_1997}. Die Ergebnisse aus wissenschaftlicher Forschung werden dabei als Publikationen vor allen Mitgliedern der Wissenschaft präsentiert, „um diese intern von der Wissenschaftsgemeinde als wissenschaftlich beziehungsweise unwissenschaftlich zertifizieren zu lassen" \cite{Rutenfranz_1997} und durch einen kontinuierlichen Prozess der Selbstprüfung, wird die Korrektheit der wissenschaftlichen Erkenntnisse sichergestellt \cite{edsall_1976_scientific}. Das Reputationssystem ist infolge dessen gleichzeitig ein ausschlaggebender Treiber und Bremser für die Verhaltensweisen der Akteure im wissenschaftlichen System.

Wissenschaftliche Reputation verteilt sich nicht nur auf einzelne Personen sonder auch auf Einrichtungen, die wissenschaftlich tätig sind. Die Evaluation wissenschaftlicher Einrichtungen findet dabei über “Beobachtungen und Gespräche mit den Wissenschaftlern vor Ort sowie über den Austausch über die Eindrücke innerhalb der Begehungsgruppe und die gemeinsame Verständigung”\cite{Barl_sius_2008} statt.

Publikationen bilden im Hinblick auf die Funktion der Reputationsverteilung "eine Art Telos wissenschaftlicher Kommunikation" \cite{hirschauer2004peer}. In Bezug auf die Erlangung von Reputation ist wissenschaftliche Arbeit besonders auf ein funktionierendes Peer-Review-System angewiesen \cite{Luescher_2014}. Das Verfahren hat zwei Funktionen: Erstens die Selektionsfunktion, in deren Rahmen die Auswahl von Personen, Projekten und Texten stattfindet und zweitens eine Konstruktionsfunktion, in der Gutachter "produktiv in den Wissenschaftsprozess eingreifen" \cite{Neidhardt_2010} und die eigenen Fachstandards durchzusetzen. Der Peer Review-Prozess sichert aber nicht nur "Vertrauen" und die Grundlage für die "Anschlusskommunikation" innerhalb der wissenschaftlichen Community, sondern "wirkt überdies auch nach außen und gewährleistet die gesellschaftliche Legitimation des wissenschaftlichen Wissens" \cite{pscheida_2010_wikipedia}.

Als "guter akademischer Forscher" oder gute wissenschaftliche Institution gilt nur "wer viel und in möglichst angesehenen Journalen" \cite{Frey_2005} oder wissenschaftlichen Buchverlagen veröffentlicht. Dabei spielt der so genannte Peer-Review-Prozess eine zentrale Rolle im wissenschaftlichen Prozess \cite{smith_1999_opening} und ist Kernelement der Selbststeuerung von Wissenschaft \cite{Neidhardt_2010}. In dem Peer-Review Prozess "werden eingereichte Beiträge von fachlich versierten Wissenschaftlern (...) beurteilt und gemäß der qualitativen Anforderungen der Forschungs-Community zur Veröffentlichung angenommen oder abgelehnt" \cite{Hess_2006}.

Die Geschichte des Peer Review Verfahrens geht auf das 17. Jahrhundert zurück \cite{Kronick_1978}, etablierte sich aber erst Mitte des 18. Jahrhunderts, als die Royal Society of London ein "Committee of Papers" gründete, dass die Bewertung von Artikeln in seiner Zeitschrift Philosophical Transactions beaufsichtigen sollte \cite{Kronick_1990}. Das Verfahren unterschied sich damals grundlegend von dem, was heute im Einzelfall unter "Peer Review" verstanden wird und auch heute unterscheiden sich die Verfahren und deren Verbreitung in Abhängigkeit vom jeweiligen Fachgebiet. Drei gängige Verfahren der Peer-Review sind heute besonders stark verbreitet \cite{mueller_2009_peerreview}:
\begin{enumerate}
\item Bei der Double Blind Peer Review (DBPR) kennen sich Autoren und Gutachter eines eingereichten Manuskripts nicht
\item Bei der Single Blind Peer Review (SBPR) kennen die Gutachter die Autoren, die Autoren wissen jedoch nicht, wer ihr Manuskript bewertet
\item Open Peer Commentary (OPC) wird einer vergleichsweise großen Anzahl von Wissenschaftlern die Möglichkeit eingeräumt wird, an der Bewertung wissenschaftlicher Arbeiten teilzuhaben
\end{enumerate}

\begin{figure}[h!]
\includegraphics{smalltableid:MTDOg}
\caption{Vergleich Peer Review und Open Peer Commentary}
\end{figure}

Obwohl die meisten Gutachter für ihre Tätigkeit nicht bezahlt werden \cite{yiotis_2013_open}, steckt hinter dem Prozess ein komplexes System bestehend aus Redakteuren, Redaktionen, und der Verwaltung des Peer-Review-Prozess, der meist von Verlagen gesteuert und bezahlt wird \cite{Bargheer_2015} \cite{mueller_2009_peerreview} \cite{Baggs_2006}. Als Herzstück einer autonomen, selbstverwalteten Wissenschaft" \cite{suchen_Hornbostel_2006} beschränkt er sich nicht nur auf den Prozess der Publikation von Texten \cite{mueller_2009_peerreview}, sondern deckt ein breites Spektrum von Aktivitäten über die Fachdisziplinen hinaus ab \cite{Lee_2012}:
\begin{itemize}
\item die Beobachtung der klinischen Praxis (z.B. in der Medizin)
\item Beurteilung des Lehrenden oder der Fähigkeiten der Kollegen
\item Bewertung bei der Forschungsförderung und Stipendien bei Einreichung von Anträgen an staatliche und anderen Förderorganisationen
\item Begutachtung bei Artikeleinreichungen für wissenschaftlichen Zeitschriften
\item Bewertung von Papieren und Plakaten für Konferenzen
\item Bewertung von Buchvorschlägen für Universitätsverlage oder andere Verlage
\item Einschätzungen der Qualität, Anwendbarkeit und Interpretierbarkeit von Datensätzen und wissenschaftlichen Organisationen
\end{itemize}

Das Peer-Review Verfahren ist zwar innerhalb in der Wissenschaft weit verbreitet, bleibt aber der Öffentlichkeit weitgehend verborgen \cite{Konneker_2013}. Obwohl diese Verfahren den Kern der wissenschaftlichen Qualitätssicherung darstellen, werden den qualitativen Peer Review-Systeme und quantitativen bibliometrische Qualitätssicherungsverfahren zunehmend Mängel zugeschrieben \cite{Peters_2014} \cite{Lee_2012} \cite{bar_2009_wissenschaftliche} \cite{osterloh2008anreize} \cite{ware_2008_peer} \cite{Jansen_2007} \cite{smith_1999_opening}. Die Mängel lassen sich laut Osterloh und Frey wie folgt zusammenfassen \cite{osterloh2008anreize}: Erstens, die "geringe Reliabilität der Gutachter-Urteile", zweitens die "geringe prognostische Qualität von Gutachten" und drittens das "opportunistisches Verhalten der Gutachter und Editoren, sowie das "opportunistisches Verhalten der Autoren". Zusammenfassend kommen die Autoren zu dem Schluss, dass "die Annahme eines Manuskriptes einem Zufallsprozess gleicht" und das "System der qualitativen Peer Reviews (...) auf einer erstaunlich fragwürdigen wissenschaftlichen Grundlage" beruht \cite{osterloh2008anreize}.

---- TODO: Grafik aus Kritik am Peer-Review bauen \cite{mueller_2009_peerreview} ----

Das Belohnungssystem für Wissenschaftler und Wissenschaftlerinnen bietet folglich Anreize für die, die als erstes neues Wissen entdecken und veröffentlichen. Das System der wissenschaftlichen Reputation baut demnach auf der Verbreitung dieser Ergebnisse in der wissenschaftliche Gemeinschaft auf \cite{Fabrizio_2008}. Die Reputation einzelner Wissenschaftler befindet sich damit in enger Abhängigkeit vom bestehenden wissenschaftlichen Kommunikationssystem. Anstatt finanzieller Entlohnung, wird in der Wissenschaft primär mit Aufmerksamkeit bezahlt. Vereinfacht lässt sich das System der Wechselbeziehungen der Reputationsverteilung im Rahmen von Publikationen wie folgt darstellen \cite{cite:21a}:

---- TODO: Grafik aus Text von Bernius ----

Bernius et al. unterscheiden drei aufeinandertreffende koordinierende Marktmechanismen: die Reputation, die Nutzung wissenschaftlicher Publikationen, sowie den Preis für den Erwerb der Publikation \cite{cite:21a}. Während die Reputation ein non-monetärer Aushandlungsmechanismus zwischen wissenschaftlichen Verlagen und wissenschaftlichen Autoren ist, findet die monetäre Preisdefinition direkt zwischen Bibliotheken und Verlagen statt \cite{EuropeanCommission_sciencepub_2006}. Der monetäre Aushandlungsprozess zwischen Wissenschaftlern und Bibliotheken wird durch die Bedeutung und Nutzung der jeweiligen Publikation bestimmt \cite{cite:21a}. Nicht jede Publikation hat diesbezüglich die gleiche Wertigkeit \cite{suchen} und damit den gleichen Einfluss auf die Reputation eines Autors.

Zusammenfassend lassen die neuen Möglichkeiten der Verbreitung von Informationen einen vergleichbaren Veränderungsprozess der wissenschaftlichen Reputation und damit auch Anerkennung vermuten, wie er durch die Entwicklung des Buchdrucks ausgelöst worden war \cite{hanekop_2006}.

\subsubsection{Messbarkeit wissenschaftlicher Qualität und Publikationsquantität}

Wissenschaft ist ein Prozess, bei dem aus “unterschiedlichen Inputfaktoren, mittels verschiedener Transformationen Beiträge zur Schaffung neuer wissenschaftlicher Erkenntnisse als Output entstehen” \cite{Jansen_2007}. Die Messung und Bewertung des jeweiligen Outputs führt zur Aussage über die Performanz des jeweiligen Forschungsprozesses. Neben den Indikatoren für den Output wissenschaftlicher Performanz, müssen aber auch intermediäre Aspekte berücksichtigt werden \cite{schmoch_2009}.

Mit Beginn des 20. Jahrhunderts wurden in der Wissenschaftsforschung Indikatoren überwiegend zur Beschreibung der exponentiellen Wachstumsverläufe von Wissenschaft entwickelt und eingesetzt \cite{Hornbostel_1997}. In der Zeit nach dem zweiten Weltkrieg etablierten sich erstmals Indikatoren für die Effizienzmessung wissenschaftlicher Wissensproduktion und -verbreitung, die aber "ebenso wie Sozial- und Wirtschaftsindikatoren keine neutralen Realitätsbeschreibungen" \cite{Hornbostel_1997} darstellten. Spätestens seit den 1970er Jahren werden diese Messungen, die die Forschungsleistung quantifizieren sollen, flächendeckend durchgeführt \cite{Hornbostel_1997} um Forschungsqualität und Quantität quantifizierbar zu machen.

Seit den 1990er Jahren ist diese Bewertung von Wissenschaft in Gestalt von Zahlen als unkontrollierte Nebenprodukte digitaler Wissenskommunikation erweitert worden \cite{angermueller_2010}. Heute zählen in der Wissenschaft vor allem die wissenschaftliche Reputation und die als "Impact" bezeichnete Wirkung wissenschaftlicher Publikationen \cite{herb_open_2013} \cite{Hornbostel_1997}. Die Wirkung der wissenschaftlichen Kommunikation wird , wie im Kapitel "Wissenschaftliche Kommunikation" ausgeführt, anhand der quantitativen Betrachtung der Zitationen der jeweiligen Publikation ermittelt \cite{Brembs_2013} \cite[:16]{haustein_2012_multidimensional} \cite{weller2011twitter}. Diese rein quantitative Betrachtung muss allerdings auch als Proxy für die Bewertung von Wirkung in der "publish or perish" community verstanden werden \cite{peters_2015_research}.

Diese Art der Betrachtung basiert auf der Grundannahme, dass Kommunikation die "Essenz der Wissenschaft" \cite{bonitz_1998_matthaus} ist und "Zitierungen in ihrer Gesamtheit so etwas, wie die Grundelemente eines weltweiten Expertensystems" \cite{bonitz_1990_sci}. Nach dieser Sichtweise stellt eine häufige Zitation einen wesentlichen Indikator für die Wirkung der wissenschaftlichen Arbeit dar \cite{hamilton_1990_publishing}. Ein generalisierter und überzeitlicher Begriff von Qualität wissenschaftlicher Arbeit scheint nicht möglich, weil eine grundlegende Definition der Wissenschaftsindikatoren sowie ihrem Ziel der "Abbildung eines Konstrukts, das die Bewertungen einzelner Wissenschaftler oder Experten transzendiert" nicht möglich erscheint \cite{Hornbostel_1997}.

In den letzten Jahren haben sich neue Möglichkeiten für die Qualitätssicherung und -bewertung herausgebildet \cite{rekdal_2014_academic}. Die "Anforderungen an Verfügbarkeit von Dokumenten und Transparenz der Begutachtungen" der Open Access Bewegung haben die Frage aufgebracht, "ob möglicherweise Veränderungen der Review-Praktiken notwendig sind, um exzellente Wissenschaft zu identifizieren und vor allem zu fördern" \cite{suchen_Hornbostel_2006}. Des Weiteren stellt sich die Frage, ob die Berücksichtigung neuer Metriken für die Bewertung wissenschaftlicher Qualität eine Antwort auf die Herausforderungen in dem etablierten Messsystem von wissenschaftlicher Qualität und Publikationsquantität sein können.

Bestand die klassische Wirkungsmessung von Wissenschaft in der Ermittlung der Anzahl von Zitationen, ermöglichen die veränderten Bedingungen von wissenschaftlicher Kommunikation im Rahmen der Digitalisierung alternative Erhebungsmöglichkeiten der Wirkung formeller wissenschaftlicher Kommunikation und damit auch für die Erlangung wissenschaftlicher Qualität und Reputation. In den letzten Jahren wurde es viel einfacher, Fälle von Plagiaten und wissenschaftliches Fehlverhalten zu identifizieren, und auch andere Arten akademische Abkürzungen zu entdecken und zu sehen, wie erschreckend häufig sie auftreten \cite{rekdal_2014_academic}.

Ergänzend zu den etablierten zitationsbasierten Metriken spielen zunehmend detailliertere Analyse von nutzungsbasierten Metriken und Metriken Basis von Social Indikatoren \cite{peters_2015_research} bei der Bewertung von wissenschaftlicher Kommunikation eine Rolle. Die Befürworter solcher alternativer Metriken erhoffen sich von diesen neuen Verfahren eine unmittelbare, umfassendere und detailliertere Wirkungsmessung wissenschaftlicher Kommunikation und eine gerechtere Verteilung von wissenschaftlicher Reputation \cite{peters_2015_research} \cite{cite:17} \cite{dora_2013}.

\subsubsection{Wissenschaftliches Kapital}

Die Wissenschaft ist ein soziales Feld, dessen Strukturen und Praktiken das bestimmen, was in dem Kommunikations- und Publikationssystem als Wissenschaft und als wissenschaftliches Ergebnis gilt \cite{mikl_2010_soziologie}. Im Rahmen der Betrachtung von Steuerungs- und Reputationsmethoden für die Wissenschaft ist der Begriff "wissenschaftliches Kapital" von herausragender Bedeutung \cite{Barl_sius_2008}. Wissenschaftliches Kapital kann als eine Ausprägung des kulturellen Kapitals und als symbolisches, beziehungsweise non-monetäres Kapital \cite{irmer2011} \cite{hagner_2015_sache_buches} \cite{bourdieu_1988_homo} verstanden werden.

Piere Bourdieu sieht in der Wissenschaft ein soziales System und beschreibt es als angetrieben von dem ständigen Machtkampf um die Erlangung und Erhaltung von symbolischem Kapital \cite{bourdieu_1988_homo}. Dieser von Piere Bourdieu beschriebene Homo Academicus ist durch Selbstdisziplin, eine sehr ausgeprägte Neugier und die Fähigkeit Forschung und Lehre zu betreiben, charakterisiert \cite{bourdieu_1988_homo}. Das symbolische Kapitel als Antreiber seines Handelns wird in diesem Zusammenhang von der Sozilogin Mikl-Horke als Besitz an symbolischen Gütern beschrieben, "der besonders in einer Gesellschaft, die auf die Kooperation aller angewiesen ist, sehr kostbar ist" \cite{mikl_2010_soziologie}. Eine genauere Betrachtung dieses wissenschaftlichen Kapitals ist für das Verständnis der Motivation von Wissenschaftlern zu publizieren und zu kommunizieren, sowie für die Herausarbeitung der Katalysatoren und Hindernisse für die Öffnung wissenschaftlicher Kommunikation demnach unabdingbar.

Die Gewährung wissenschaftlichen Kapitals basiert heute auf der Kooperation zwischen publizierenden Wissenschaftlern und Verlagen \cite{herb_2006}. Die Wissenschaftler befinden sich in einer Abhängigkeit von den Verlagen. Diese Abhängigkeit wird auch als "Faustischer Pakt" bezeichnet und hinterfragt \cite{hagner_2015_sache_buches} \cite{Parks_2002_acadamic_faust}. Den Pakt sind Wissenschaftler notgedrungen eingegangen, "um den Preis, dass Barrieren zwischen Autoren und Lesern aufgebaut wurden" \cite{hagner_2015_sache_buches}. "Wissenschaftliches Kapital" kann in diesem Zusammenhang als “Ergebnis einer Investition (...), die sich auszahlen muss” \cite{herb_2006} definiert werden. “Diejenigen, die diese Berechtigungsscheine in der Hand halten, verteidigen ihr 'Kapital' und ihre 'Profite', indem sie diejenigen Institutionen verteidigen, die ihnen dieses 'Kapital' garantieren.” \cite{Bourdieu_1992}

Der Soziologe Bordieu unterscheidet zwei Typen wissenschaftlichen Kapitals \cite{Bourdieu_1998}. Das Kapital, das auf der politischen und institutionellen Macht beruht und das andere, dass aus der rein wissenschaftliche Anerkennung resultiert \cite{mikl_2010_soziologie}. Das Institutionelles wissenschaftliches Kapital weist die "Macht und die Erwartung zu, auf Institutionen und Organisationen der Wissenschaft einzuwirken und über die Produktionsmittel der Wissenschaft zu disponieren" \cite[:257]{Barl_sius_2008}. Es ist disziplinunabhängig und fachübergreifend. Das reine wissenschaftliche Kapital muss disziplinspezifisch erarbeitet werden und wird durch die Publikation von Inhalten in der jeweiligen Fachdisziplin hoch angesehenen Zeitschriften, bei besonderen Verlagen oder durch die Arbeit in reputierlichen wissenschaftlichen Einrichtungen erlangt \cite[:257]{Barl_sius_2008}.

Zitationsindexe sind Indikatoren für das wissenschaftliche Kapital, das durch Anerkennung entsteht \cite{Bourdieu_1998}. Die wissenschaftliche Reputation, die aus dem wissenschaftlichen Kapital resultiert, basiert auf der Liste der Publikationen in hoch gerankten Journalen und angesehenen Verlagen \cite{herb_2010}. Diese Bewertung ist klar symbolischer Natur und basiert "auf der Anerkennung und dem Kredit (...), den die Gesamtheit der Wettbewerber innerhalb des wissenschaftlichen Feldes gewähren" \cite{Bourdieu_1998} \cite{Barl_sius_2008} \cite{herb_2010}.

Das wissenschaftliches Kapitel ist dabei zunehmend der Kapitalisierung von Wissenschaft ausgesetzt, bei der um den Einfluss der Ökonomie und den "wissenschaftswidrigen Verwertungsdruck"  \cite[:12]{Neidhardt_2006} gerungen wird. Als ein Indikator dafür ist die Kopplung des wissenschaftliches Kapitals und an die output-orientierte Anreizsysteme zu verstehen. Diese Fokusierung auf die "Kenngrößen führt dazu, dass Wissenschaftlerinnen und Wissenschaftler einen Anreiz haben, sich weniger als Homo academicus, sondern eher als Homo strategicus zu verhalten und sich auf die gut messbaren Aufgabenbestandteile zu konzentrieren" \cite{Frost_2014}. Ein Beispiel ist die zunehmende Relevanz des Performanzindikators "Drittmittel" \cite{Fabrizio_2008} \cite{Jansen_2007}, bei dem neben der Sicherung der Qualität von Forschung und Lehre zunehmend direkte finanzielle und administrative Kontrolle eine Rolle spielt \cite{Barl_sius_2008}. Dem Drittmitteleinkommen wird als Indikator für Forschungsleistung eine hohe Bedeutung zugemessen \cite{Jansen_2007}. Daraus entsteht die Tendenz, das nicht nur die Erwartungen an die Bewertung von Wissenschaft sehr ambitioniert sind, sondern auch, dass die Interessen privater und öffentlicher Drittmittel-Auftraggeber in den Vordergrund rücken und die Unabhängigkeit von Wissenschaft und Forschung gefährden.

Ähnliches ist im Rahmen der stetigen Ökonomisierung des internationalen Universitätsbetriebes \cite{brembs2015open} und bei der leistungsbezogenen Mittelzuweisungen an die Universitäten zu beobachten \cite[:12]{Neidhardt_2006}. Vor allem die Verknüpfung von wissenschaftlicher Reputation mit der damit einhergehenden Verteilung von Mitteln und Stellen stellt eine neuartige Herausforderung an das Wissenschaftssystem dar, dessen Währung ursprünglich nicht Geld war \cite{hanekop_2006}.

Die Auseinandersetzung mit dem wissenschaftlichen Kapital im Rahmen der Forderung nach Öffnung wissenschaftlicher Kommunikation kann auch deshalb als wichtig erachtet werden, weil diese bisher nur begrenzt der wissenschaftlichen Logik folgt und eher auf einer "feldunabhängigen Logik der Akkumulation von Kapital" basiert \cite{herb_2006}. Mit der Zunahme an output-orientierten Anreizsystemen im deutschen Wissenschaftssystem \cite{osterloh2008anreize} und einem Ungleichgewicht in der Kooperation zwischen wissenschaftlicher Kommunikation und wissenschaftlichem Kapital wird diese Entwicklung bei der weiteren Betrachtung der Motivationsfaktoren für Veränderungsprozesse in der wissenschaftlichen Kommunikation eine wichtige Rolle spielen.

\subsection{Wissenschaftliches Ethos}

Das wissenschaftlichen Ethos hat die Funktion der Wissenschaft "eine soziale und politische Legitimationsbasis zu verschaffen" \cite{descher_2012_ethos}. Generell stellt die Verfügbarmachung von Forschungsergebnissen einen integralen Bestandteil des wissenschaftlichen Ethos dar \cite{Fangerau_2014}. Dabei wird "die moderne Wissenschaft als Methode und Praxis der Wissensbildung wird durch ein Ethos epistemischer Rationalität geleitet" \cite{Oezmen_2015}. Als "selbstregulatives und nach eigenen Regeln operierendes System muss sie ihren Ethos jeder neuen Generation vermitteln" und "Verantwortungsstrukturen und Rahmenbedingungen" schaffen, "die langfristig eine verlässliche Kultur wissenschaftlicher Integrität stärken" \cite[:7]{wr_2015_wissenschaft_integritaet}.

Der US-amerikanische Soziologe Robert K. Merton stellte diesen und weitere Grundprinzipien als normative Grundstruktur des Ethos von Wissenschaft vor \cite{Merton_1985}. Diesem Ethos liegt auch den Annahmen zu Grunde, dass es Vorteile für die wissenschaftliche Gemeinschaft bringt, wenn Daten zweitverwertet werden und dass Daten ein wirtschaftliches Gemeingut sind, deren Wert durch breitere Nutzung verbesser wird \cite{RIN_2010_open_research}. Darüber hinaus sind "systematische Widerspruchsfreiheit, interne Kohärenz, Klarheit, aber auch Sparsamkeit und Eleganz, Genauigkeit und Überprüfbarkeit" weitere integrale Bestandteile des Ethos \cite{Oezmen_2015}. Dabei gilt es auch dem Umstand Rechnung zu tragen, dass Wahrheit relativ ist und sich die Richtigkeit einer wahren Aussage nur mit Hilfe der genauen Bedingungen beschreiben lässt, unter denen sie wahr ist. Karl Raimund Popper attestiert der Wissenschaft demnach eine grundsätzliche "Fehlbarkeit" und leitet daraus ab, dass jedwede wissenschaftliche Erkenntnis möglichst offen für Kritik sein muss \cite{popper_2005_logic}.

Darüber hinaus sind "Anspruchslosigkeit und Bescheidenheit" sind weitere Grundtugenden des modernen Wissenschaftlers \cite{hagner_2015_sache_buches}. Das Ethos wird in diesem Zusammenhang als "Komplex von Werten und Normen" \cite{suchen} beziehungsweise "Verhaltensmaßregeln" \cite{suchen} verstanden. Merton unterteilt die Kriterien in die Kategorien, die alle auf das wissenschaftliche Kommunikationssystem anwendbar sind  \cite{Fröhlich_oa_2009}:
\begin{itemize}
\item \textit{Universalismus}: Die Geltungsansprüche der Wissenschaft sind allgemein und objektiv \cite{Oezmen_2015}. Die sozialen Merkmale eines Wissenschaftlers, wie zum Beispiel Nationalität, Geschlecht, Religion, Klasse usw. dürfen nicht in die Evaluation wissenschaftlicher Ergebnisse einfließen \cite{suchen}.
\item \textit{Kommunismus (Kommunalität)}: Es gibt eine Pflicht zur Veröffentlichung der Ergebnisse von Wissenschaft und Forschung und sie sind als Allgemeingut zu betrachten. Die wissenschaftliche Anerkennung und das Ansehen sind einziges damit verbundenes "Besitzrecht" \cite{suchen}.
\item \textit{Uneigennützigkeit}: Intrinsische "Neugier"\cite{suchen}, "selbstloses Eintreten für das Wohl der Menschheit"\cite{suchen} und der Wissensdurst müssen die vornehmlichen Motivatoren für Wissenschaftler darstellen \cite{suchen}.
\item \textit{Objektivität und Desinteresse}: Wissenschaft erfordert "Objektivität und Desinteresse" an den Ergebnissen der eigenen Forschung \cite{suchen} unabhängig von finanziellem Erfolg und Prestige \cite{suchen}. Wissenschaft nicht durch die persönlichen Präferenzen oder eigennützigen Motive und subjektiven Meinungen geleitet, sondern durch reines Erkenntnisinteresse und durch Wahrheit \cite{Oezmen_2015}.
\item \textit{Organisierter Skeptizismus}: Zweifel muss als "grundsätzliches Denkprinzip der Wissenschaft" \cite{suchen} und die "unvoreingenommene Prüfung und Kritik an Wissenschaft, Forschung und Autorität" \cite{suchen} als Kernbestandteil des Systems verstanden werden. Zum Beispiel gilt es den "Matthew Effect" zu vermeiden. Der Matthäus-Effekt ("Wer hat, dem wird gegeben" Mt. 25,29) ist ein Phänomen auf der Makroebene der Wissenschaft \cite{bonitz_1998_matthaus} und  beschreibt den Umstand, dass Autoren oder Publikationen, die bereits eine hohe Zitationsrate vorweisen können, meist noch häufiger zitiert werden als die Autoren oder Beiträge mit einer geringeren Zitationsrate. Überproportional profitieren in diesem System also die, die besonders häufig zitiert wurden \cite{Merton_1968} \cite{meier_2009_matthaus}.
\end{itemize}

Als Folge dieser Kriterien erkannte Merton das Urheberrecht an wissenschaftlichen Ideen und Beiträgen an, allerdings nur insofern, als dass das Urheberrecht allein auf die Ermöglichung der Anerkennung durch Kollegen und die Achtung der Priorität beschränkt bleibt \cite{Fangerau_2014}. Damit kritisiert er implizit das System der wissenschaftlichen Kommunikation. In Folgeschriften kritisierter er darüber hinaus die neueren Konzepte und Praktiken, die "die Werte des klassischen Wissenschaftsethos korrumpieren" \cite{Fröhlich_oa_2009}.

Neben der internen wissenschaftlichen Verantwortung, die eng mit der der Regel guter wissenschaftlicher Praxis und offener wissenschaftlicher Kommunikation zusammenhängt, kann eine Ergänzung dieses Zusammenhangs um eine externe Verantwortung "im Sinne der Rechenschaftspflicht für die möglichen Anwendungen und Folgen seiner Forschung" \cite{Oezmen_2015} stattfinden. Sie ist zwar kein konstitutiver Bestandteil des Ethos der Wissenschaft, aber ein Teil einer Ethik der Wissenschaft \cite{Oezmen_2015}. In Verbindung mit der Forderung nach Öffnung der wissenschaftlichen Kommunikation ist aber dennoch zu berücksichtigen, dass die Verteidigung der Autonomie und Freiheit der Wissenschaft "nicht mit der ganz andersgelagerten These der Autarkie der Wissenschaft verwechselt" werden darf und mit "ethischen Zwecksetzungen eine Verletzung des epistemischen Ethos sowie eine Gefährdung der Autonomie der Wissenschaft befürchtet werden" kann \cite{Oezmen_2015}.

Der Umstand dar, dass die zunehmende Berücksichtigung bibliometrische Indikatoren anstatt der tatsächlichen Qualität der Wissenschaft im Rahmen der Steuerung von Wissenschaft zu negativen Auswirkungen auf normative Grundstruktur des Ethos der Wissenschaft haben kann, stellt einen weiteren Anknüpfungspunkt für die Herausforderungen im aktuellen wissenschaftlichen Kommunikationssystem dar.

\subsection{Wissenschaftlicher Diskurs}

\begin{quote}Wissenschaftliche Kommunikation vollzieht sich in Behauptungen, Erklärungen, Prognosen; sie ist nicht nur ein Informationsaustausch. Vielmehr vollzieht sich im wissenschaftlichen Diskurs der kollektive Prozess des wissenschaftlichen Begreifens. Deshalb ist die wissenschaftliche Sprache als Diskurs nicht bloß ein Medium der Kommunikation, sondern der Ort, an dem sich ein wesentlicher Teil der wissenschaftlichen Arbeit vollzieht, der kollektive Darstellungsraum der Wissenschaft. \cite{bohme_1978_wissenschaftssprachen}\end{quote}

Ebenso muss auch die Erforschung wissenschaftlicher Fragestellungen als ein zentraler Bestandteil des wissenschaftlichen Diskurses \cite{suchen} betrachtet werden. Die Verarbeitung von Forschungsergebnissen, die Anwendung und Neuinterpretation von Ergebnissen sowie das Verfassen von Gegenentwürfen und synthetischer Gesamtdarstellungen stellen Faktoren für den wissenschaftlichen Diskurs dar \cite{suchen}. Jürgen Habermas unterscheidet hier das kommunikatives Handeln von strategischem Handeln. Im dem "rationalen Diskurs" findet dabei vor allem eine Verständigung über problematische Geltungsansprüche statt \cite{suchen}. Der Beobachter entwickelt Methoden und Verfahren um zu einer Verständigung mit seiner Zielgruppe zu kommen \cite{suchen}. Der wissenschaftliche Diskurs operiert in diesem Verständigungsprozess funktional eigenständig und alles, was durch Wissenschaft kommuniziert wird, ist “entweder wahr oder unwahr” \cite{Luhmann1998}.

Michel Foucault versteht unter einem Diskurs "eine Menge von Aussagen, die einem gleichen Formationssystem zugehören"\cite{foucault_archaologie_1981}. Der wissenschaftliche Diskurs gründet sich demnach nur zum Teil auf die Forschung und kann auch nicht nur als "Kontaktglied zwischen dem Denken und dem Sprechen" \cite{foucault_ordnung_2003} definiert werden. Er wird getrieben vom Willen zur Wahrheit, der sich durch "die Pädagogik, dem System der Bücher, der Verlage und Bibliotheken, den gelehrten Gesellschaften einstmals und den Laboratorien heute" ständig erneuert \cite{foucault_ordnung_2003}. Abgesichert wird der Diskurs "durch die Art und Weise, in der das Wissen in einer Gesellschaft eingesetzt wird, in der es gewertet und sortiert, verteilt und zugewiesen wird" \cite{foucault_ordnung_2003}. In der foucault'schen Diskursanalyse wird der Diskurs als die Fähigkeit definiert, die "Beziehungen" zwischen "Institutionen, ökonomischen und gesellschaftlichen Prozessen, Verhaltensformen, Normsystemen, Techniken, Klassifikationstypen und Charakterisierungsweisen herzustellen" \cite{foucault_archaologie_1981}.

Im Rahmen des wissenschaftlichen Diskurses versuchen Menschen mit diversen "Machtprozeduren", die "ungeordnete und wuchernde Masse aller Äußerungen" zu reglementieren und zu kontrollieren \cite{Neymeyer_diskurs_2010}. Resultierend daraus entstehen Diskurse, die sich über einen gemeinsamen Gegenstand definieren. Sie gehorchen "impliziten wie expliziten Regeln", unterliegen "spezifischen Funktionen", nehmen bestimmte Formen an und sind von "Machtmechanismen gekennzeichnet" \cite{Neymeyer_diskurs_2010}. Diese grundlegende Definition des Diskurses ist für die weitere Betrachtung der Veränderungsprozesse und die Forderung nach Öffnung der wissenschaftlichen Kommunikation unabdingbar.

Die Forderungen um die Öffnung wissenschaftlicher Kommunikation beruhen auf der Annahme, dass sie die Grundlage dafür schaffen, dass wissenschaftliche Diskurse besser und umfassender geführt werden können, als im aktuell bestehenden System. Nach der klassischen Einordnung ist damit aber nicht eine Öffnung des Diskurses außerhalb der wissenschaftlichen Gemeinschaft gemeint, sondern nur die Senkung der Zugangsbarrieren zu wissenschaftlicher Kommunikation. Das Ziel des Ausschlusses der Öffentlichkeit aus den wissenschaftlichen Diskursen liegt darin, die Macht sowie Auswüchse des Diskurses einzugrenzen, seine Ergebnisse zu bändigen und das Wissen durch den "Willen zur Wahrheit" \cite[:15]{foucault_1991_ordnung} zu kanalisieren.

\section{Herausforderungen im bestehenden System wissenschaftlicher Kommunikation}

Diese Kanalisierung des Wissens im Rahmen der wissenschaftlichen Kommunikation und die Wirksamkeit, wie auch Zweckmäßigkeit dieses wissenschaftlichen Kommunikationssystems sind seit Jahrzehnten Bestandeil von Debatten in der Literatur \cite{suchen} in denen diese immer wieder als suboptimal bezeichnet werden \cite{suchen}. Die folgende stukturierte Einteilung in verschiedene Bereiche der Kritik und Darstellung dieser dient der Einordung der Herausforderungen und zur Eingrenzung der Debatten um das aktuelle System der wissenschaftlichen Kommunikation. Sie werden im Verlaug der Arbeit bei der Betrachtung von Hindernissen und Katalysatoren sowie im Rahmen der Befragung an späterer Stelle erneut aufgegriffen.

Die Herausforderungen im bestehenden System formeller wissenschaftlicher Kommunikation beziehen sich vor allem auf neun Bereiche:
\begin{enumerate}
\item Leistungsbewertung der wissenschaftlichen Arbeit
\item Geschwindigkeit im Kommunikationsprozess
\item Wahrung der Freiheit von Wissenschaft und Forschung
\item Kosten und Effizienz
\item Fehlerresistenz und Qualitätssicherung
\item Verbreitung und Zugänglichkeit
\item Digitalisierung
\item Möglichkeiten der Überprüfbarkeit des Wissens/der wissenschaftlichen Güte
\item Verhinderung von Missbrauch und wissenschaftliches Fehlverhalten
\end{enumerate}

\subsection{Leistungsbewertung der wissenschaftlichen Arbeit}

Die Verlage haben in den letzten Dekaden mit den wissenschaftlichen Journalen und Monographien ein zentrales Steuerungs- und Bewertungssystem in der Wissenschaft etablieren können. In diesem System werden die Grundprinzipien der Wissenschaft für die verlegerischen Verwertungsinteressen (aus)genutzt und das, obwohl diese "wissenschaftlichen Grundprinzipien und Normen eigentlich ökonomischen Verwertungsinteressen zu widersprechen scheinen" \cite{hanekop_2006}. Darüber agieren die Forscherinnen und Forscher in einem Umfeld, in dem sie in vielen Fällen wenig oder keine Verantwortung für den Einkauf der wissenschaftlichen Informationen haben, die er oder sie im Rahmen der Veröffentlichung "verschenkt" \cite{steele_2006}.

Die Einführung der quantitativer Bewertungsindikatoren wie das Zitationsregister und die Impact Faktoren, sowie die Definition der Kernzeitschriften, führte zu einer weitgehenden Erstarrung des wissenschaftliche Zeitschriftenmarktes und gleichzeitig zu einem Anstieg der Kapazität der kommerziellen Verlagen, sowie deren Gewinnmargen \cite{CREATe_2014}. Die Steuerungsmechanismen werden über die Messbarkeit mittels Methoden direkt oder indirekt ausgeübt. Dabei stehen insbesondere die Methoden, die auf der quantitativen Grundlage der Zitationsraten wissenschaftlicher Publikationen gemessen werden in der Kritik \cite{Brembs_2013} \cite{Dong_2005} und auch andere Indikatoren für die Messung von Forschungsleistungen sind hoch umstritten \cite{Hornbostel_1997} \cite{Hicks_1996} \cite{Havemann_2002} \cite{Warnke_2012}. Die Verfahren, um die Wirkung von Wissenschaft und damit auch die Reputation von Wissenschaftlern zu messen, sind kein eigentliches wissenschaftliches Produkt \cite{suchen} und erfassen zum Beispiel die Tätigkeit einzelner Forschergruppen zu stark \cite{schmoch_2009}. Darüber hinaus sind "weder importance noch impact noch quality direkt messbar" und man kann sich ihnen nur "nähern" \cite{Hornbostel_1997}. Das führt unter anderem dazu, dass der aus der "Zahl der Zitationen auch die Beiträge einer Zeitschrift ermittelte" \cite{weishaupt_2009_goldenOA} Impact Factor nicht als perfektes Werkzeug betrachtet werden kann, um die Qualität der Artikel zu messen \cite{garfield_1999} und "selbst die grundlegendsten wissenschaftlichen Standards verletzt" \cite{Brembs_20013}. Trotzdem wird er zur Bewertung von Wissenschaft genutzt, denn “es gibt nichts Besseres" und er hat den Vorteil, dass er allein durch seine lange Existenz "eine gute Technik für die wissenschaftliche Bewertung” darstellt \cite{garfield_1999} \cite{weishaupt_2009_goldenOA}.

Die Kritik am Impact Faktor lässt sich laut der Bibliotheks- und Informationswissenschaftlerin Dr. Karin Weishaupt, am Beispiel des "Thomson Reuters Journal Citation Factors" in sechs Punkten zusammenfassen \cite{weishaupt_2009_goldenOA}:
\begin{enumerate}
\item Der Impact Factor bezieht sich immer auf die gesamte Zeitschrift und hat somit keine Aussagekraft über die "Rezeption oder Qualität des einzelnen Artikels" \cite{weishaupt_2009_goldenOA}.
\item Der Impact Factor berücksichtigt nur die Zeitschriften, die im eigenen Index gelistet sind und enthält weder Monographien, Tagungsbeiträge, sonstige Beiträge oder Internetquellen.
\item Durch Selbstzitierungen sind Manipulationen möglich.
\item Es werden nur Zitate aus den letzten beiden Jahren berücksichtigt und je nach Fachgebiet ist es von Vorteil wenn im eigenen Gebiet die Verwertungszyklen kürzer sind.
\item Publikationen, die nicht in englischer Sprache verfasst sind, weisen überwiegend eine geringere Sichtbarkeit und Popularität auf, da englische Journale überproportional vertreten sind
\item Spezialisierte Zeitschriften sind ebenfalls systematisch benachteiligt gegenüber Journalen großer Fach-Communities oder Journalen mit Übersichtsartikeln.
\end{enumerate}

Der neue Managerialismus an Universitäten setzen dabei auf die quantitative Leistungsmessung und die wissenschaftlichen Kommunikation wird zunehmend anhand quantitativer bibliometrischer Methoden evaluiert \cite[:40]{Frost_2014}. Seit der Entwicklung des Science Citation Index (SCI) sowie des Aufkommens systematischer Wissenschaftsevaluation in Form von Rankings wird das zunehmend von Autoren, Wissenschaftlern, Lesern, Verlagen und Herausgebern für die Evaluation der Wirkung der Kommunikation akzeptiert und adoptiert \cite[:2]{haustein_2012_multidimensional}. Diese rein quantitative Betrachtungen können eine Tendenz zu Fehlanreizen darstellen \cite{wr_2015_wissenschaft_integritaet} die dazu führt, dass zunehmend messbarer Output ein wichtigeres Ziel für Wissenschaftler und Wissenschaftlerinnen darstellt, als die eigentliche Kreation und Produktion von originellen und innovativen Wissen nach den Kriterien der guten wissenschaftlichen Praxis \cite[:41]{Frost_2014}.

Hier offenbart sich ein "Generaldilemma wettbewerblicher Wissenschaft" \cite[:37]{wr_2015_wissenschaft_integritaet}. Die Idee, dass Wettbewerb in der Wissenschaft zu mehr Qualität führt steht dem Überdruck und der Beschleunigung im System gegenüber, "was Qualitätsverluste und eine Gefährdung wissenschaftlicher Integrität zufolge haben kann" \cite[:37]{wr_2015_wissenschaft_integritaet}.

Es bleibt festzuhalten, dass die im wissenschaftlichen System genutzten Indikatoren die komplexe Realität der Leistungsbewertung in der Wissenschaft nicht abbilden können und sie eine eigene Realität konstruieren \cite{Hornbostel_1997}. Versteht man Wissenschaft als soziales System, so stellen Reputation und nicht die Wahrheit der Beobachtungen und Erklärungen "nicht selten auch eingestandenes vorrangiges Ziel wissenschaftlicher Tätigkeit" dar \cite{luhmann_1970_selbststeuerung}. Wie gering der Wirkungsgrad und die Methoden zur Messung “zur Reproduktion des traditionellen wissenschaftlichen Diskurses ausfallen, wird von dem Moment an klar, an dem ein neues und offenes Kommunikationsmedium wie das Internet als alternativer Publikations- und Verbreitungskanal für Wissenschaft zur Verfügung steht" \cite{Rost_1998}.

\subsection{Geschwindigkeit im Kommunikationsprozess}

Einen weiterer Aspekt der Debatte betrifft die Kritik an der Geschwindigkeit zwischen der Fertigstellung einer wissenschaftlichen Arbeit durch den Autoren, der Einreichung zur Veröffentlichung und der finalen Veröffentlichung der Ergebnisse. Trotz der Beschleunigung der Prozesse bei der Qualitätssicherung und Bewertung wissenschaftlicher Arbeiten durch die Digitalisierung der Kommunikation zwischen Wissenschaftlern, Gutachtern und Verlagen, kann es bis zu mehrere Jahre dauern, bevor ein Text veröffentlicht wird \cite{nosek_2012_scientific}. Diese Verzögerung beruht unter anderem auf folgenden Umständen:

\begin{enumerate}
\item Gutachter/innen können aufgrund der Ausführung dieser Funktion als Nebentätigkeit meist Termine nicht einhalten \cite{bar_2009_wissenschaftliche}.
\item Es gibt weder Anreiz- noch Sanktionsmöglichkeiten für Gutachter und Gutachterinnen.
\item Die wissenschaftlichen Zeitschriften erscheinen größtenteils noch immer als Periodika und wissenschaftlichen Bücher orientieren sich am Druck. Sie sind damit für einen bestimmten Zeitraum der Veröffentlichung terminiert.
\end{enumerate}

Eine Möglichkeit, die wissenschaftlichen Inhalte schneller zugänglich zu machen, ohne den sehr zeitaufwändigen Begutachtungsprozess strukturell oder inhaltlich zu verändern, ist die Veröffentlichung der wissenschaftlichen Arbeit als digitalen Pre-Print, sowie eine umfangreichere Kommunikation des Erkenntnisprozesses vor der finalen Veröffentlichung. Ergänzend stellt die offene Begutachtung (Open Peer Commentary) eine Möglichkeit dar, bei der ein Text anonymisiert (vorab) veröffentlicht und kommuniziert, sowie von der wissenschaftlichen Gemeinschaft kollaborativ bewertet wird \cite{mueller_2009_peerreview}. Dabei darf der Wunsch nach einer erhöhten Geschwindigkeit nicht über dem Wunsch nach einem ausgewogenen Qualitätssicherungsprozess gestellt werden \cite{Beall_2012}.

\begin{figure}[h!]
\includegraphics{smalltableid:HW9dy}
\caption{Eigenschaften und Ausprägungen von OPC-Verfahren mit entsprechenden Beispielen}
\end{figure}

Insgesamt behindert die Trägheit des tradierten Systems der wissenschaftlichen Kommunikation den wissenschaftlichen Fortschritt und wird den Möglichkeiten für die digitale Informationsversorgung nicht gerecht. Dabei ist die schnelle und umfassende Verbreitung von wissenschaftlichen Inforamtionen und Daten im Rahmen des kumulativ orientieren wissenschaftlichen Erkenntnisprozesses von grundlegender Bedeutung. Forscherinnen und Forscher würden in vielfacher Hinsicht davon profitieren, wenn sie gegenseitig schneller auf die Ergebnisse ihrer Arbeit zugreifen könnten \cite{nosek_2012_scientific} \cite{winkler_2011_anforderungen}.

\subsection{Wahrung der Freiheit von Wissenschaft und Forschung}

Die freie Verbreitung von Informationen und offene Diskussion ist ein wesentlicher Teil des wissenschaftlichen Prozesses \cite{edsall_1976_scientific}. Das Recht auf Wissenschaftsfreiheit ist ein "Erbe der deutschen Achtundvierzigerrevolution \cite{kempny_2013_wissfreiheit}. Neben der Freiheit der Lehre bildet die Freiheit der Forschung den zweiten Pfeiler der Wissenschaftsfreiheit \cite{thurnherr_2014_pubfreiheit}. Die Forschungsfreiheit ist in Deutschland grundrechtlich nach Artikel 5 Absatz 3 GG  geschützt und ist auch europäisches Verfassungsgut \cite{kempny_2013_wissfreiheit}. Sie ist eine „Freiheit schlechthin, nicht Freiheit zu bestimmten Zielen oder Zwecken“ \cite{Boeckenfoerde_1974} und ihr Schutzbereich umfasst auch die Bewertung der Forschungsergebnisse sowie ihre Verbreitung \cite{Pfeiffer_2013_forschungsfreiheit}.

Die Wissenschaft unterliegt mannigfaltigen externen Einflüssen, operiert aber dennoch autonom \cite{Luhmann1998}. So greifen "andere Funktionssysteme (...) in die Wissenschaft zwar ein, wenn sie in Erfüllung ihrer eigenen Funktionen operieren und ihren eigenen Codes folgen. Aber sie können, jedenfalls unter den Bedingungen der modernen Gesellschaft, nicht selbst festlegen, was wahr und was unwahr ist."  \cite{Luhmann1998}. Dabei ist die vorgabenfreie Erarbeitung und Veröffentlichung neuer Erkenntnisse die Grundlage für wissenschaftlichen Fortschritt. "Die Autonomie der Wissenschaft wird nach außen durch die Abhängigkeit der Universität vom Staat und universitätsintern durch die Einheit von Wissenschaft und Forschung gesichert" \cite{Huber_2005}. Diese Wahrung ist im Artikel 5 Absatz 3 Grundgesetz als garantiertes Grundrecht wie folgt festgehalten: "Wissenschaft, Forschung und Lehre sind frei" \cite{suchen_GG}.

Dieses Recht ist nicht nur ein Grundrecht auf wissenschaftliche Meinungsfreiheit, sondern auch eine rechtliche Garantie. "Jeder, der in Wissenschaft, Forschung und Lehre tätig ist, hat - vorbehaltlich der Treuepflicht gemäß Art. 5 Absatz 3 Satz 2 GG - ein Recht auf Abwehr jeder staatlichen Einwirkung auf den Prozess der Gewinnung und Vermittlung wissenschaftlicher Erkenntnisse" \cite{suchen_BVG}. "Das garantiert einerseits die Einrichtung wissenschaftlicher Hochschulen mit Anspruch auf Selbstverwaltung, die staatliche Finanzierung und Sicherung ihrer Arbeit"\cite{suchen_BVG}, andererseits richtet es sich als "Abwehrrecht auf die Abwehr von Eingriffen in die wissenschaftliche Betätigung" gegen staatliche Eingriffe \cite{mayen_grundrechte_forscher} \cite{spindler_2006_rechtloa}. Jede Form der wissenschaftlichen Betätigung ist durch dieses Abwehrrecht geschützt. Dazu zählen laut Urteil des Bundesverfassungsgerichts "vor allem die auf wissenschaftlichen Eigengesetzlichkeiten beruhenden Prozesse, Verhaltensweisen und Entscheidungen bei dem Auffinden von Erkenntnissen, ihrer Deutung und Weitergabe" \cite{suchen}.

Christopher Kelty bedient sich bei der grundlegenden Einordnung von Freiheit des Konzepts der positiven und negativen Freiheit \cite{kelty_2014_freedom}. Die positive Freiheit definiert dabei die Freiheit, die es aktiv erlaubt etwas zu tun. Die negative Freiheit beschreibt demgegenüber die Freiheit von einer bestimmten (meist ungewünschten) Einflüssen. Damit eignet sich dieses Konzept von Freiheit als ein analytisches Werkzeug für die Erforschung der Auswirkungen von neuen Technologien \cite{kelty_2014_freedom}. Das betrifft auch die freie Entscheidung über die Art und Weise der Veröffentlichung von Forschungsergebnissen (positive Publikationsfreiheit) \cite{Fangerau_2014} \cite[:190]{Fehling_2014} oder eben die Freiheit der nicht-Veröffentlichung von Inhalten (negative Publikationsfreiheit).

Somit steht es allen an öffentlichen Forschungseinrichtungen tätigen Wissenschaftlerinnen und Wissenschaftlern es frei "zu entscheiden, ob und in welcher Form sie ihre dort erbrachten wissenschaftlichen Leistungen veröffentlichen" \cite{Schmidt_2009}. Auch die Wahl zwischen einer Veröffentlichung in einem kostenpflichtigen Journal oder in einem Open-Access Journal fällt damit unter die positive Publikationsfreiheit \cite[:190]{Fehling_2014}. Diese Publikationsfreiheit
im Rahmen der individuellen Wissenschaftsfreiheit ist zwar im aktuellen System des wissenschaftlichen Austauschs nicht direkt gefährdet, wird aber durch indirekte Faktoren und Anreize stark eingeschränkt \cite{binswanger_2014_excellence}. So fördert das System insbesondere die Publikationsformen und -kanäle, die von der wissenschaftlichen Gemeinschaft der jeweiligen Fachdisziplin als etabliert und als förderungsfähig betrachtet werden. Neue Formen und Kanäle hingegen werden nur selten im Rahmen der formellen Kommunikation berücksichtigt. Für sie ist es besonders schwer im bestehenden Reputationssystem Fuß zu fassen.

Wissenschaftliche Freiheit bezieht sich demnach auf der einen Seite auf die selbstbestimmte und unabhängige Wahl der Themen, Methodik, die freie Wahl Verbreitungs- und Publikationskanal sowie des Zeitpunkts und betrifft die Selbstorganisation bei der Durchführung und Steuerung der wissenschaftlichen Arbeit \cite{Fehling_2014}. Auf der anderen Seite beschreibt sie die Freiheit von inhaltlichen und methodischen Richtlinien und Vorgaben \cite{Goetting_2015}. Diese beiden Garantien beziehen sich in abgeleiteter Form auch auf die unterschiedlichen Organisationen und Institutionen von Wissenschaft. Wer "diese Freiheit der Wissenschaft beschneidet, behindert das Bemühen um Wahrheit und damit den Zweck der Wissenschaft selbst" \cite{Oezmen_2015}.

In Hinblick auf die wissenschaftliche Publikation kann festgehalten werden, dass Hochschullehrer nicht von der Hochschule oder anderen staatlichen Institutionen gezwungen werden können, über einen bestimmten Weg oder Kanal zu veröffentlichen \cite{spindler_2006_rechtloa} \cite{dorschel_2006_open}. Ausnahme stellen hier nur die privatfinanzierten Drittmittelprojekte dar, da sich der Hochschullehrer hier nicht auf die Wissenschaftsfreiheit als Abwehrrecht gegen den Staat berufen kann \cite{spindler_2006_rechtloa}. Wissenschaftlichen Mitarbeiter und Mitarbeiterinnen "müssen ihrer Hochschule die Nutzungsrechte an ihrer Publikation einräumen", es sein denn, sie haben sie nicht nach Weisung des Lehrstuhl- oder Institutsleiters erarbeitet oder es handelt sich um eine Dissertation oder Habilitation \cite{spindler_2006_rechtloa}. Ein direkter staatlicher Eingriff im Rahmen einer Richtlinie zum Publikationszwang über einen bestimmten Weg scheint demnach mit der Wissenschafts- und Publikationsfreiheit nicht vereinbar.

Dennoch kann der Staat Anreizsysteme oder Rahmenbedingungen zu schaffen, die die Öffnung des wissenschaftlichen Kommunikations- und Publikationssystems befördern. In der rechtlichen Auseinandersetzung mit dem Thema zielen die diskutierten Ansätze meist darauf ab, "den Autor eines öffentlich finanzierten wissenschaftlichen Werkes zu zwingen, die Allgemeinheit in gewissem Umfang an diesem partizipieren zu lassen und den Verlagen die Möglichkeit zu nehmen, durch einseitige Vertragsgestaltungen eine solche (kostenlose) Partizipation zu verhindern" \cite{dorschel_2006_open}.

Im bestehenden System kann auch eine Art Nötigung zur Veröffentlichung auf dem tradiertem Weg vermutet werden, die den Wissenschaftler und die Wissenschaftlerin indirekt in seiner Freiheit indirekt einschränkt, den Publikationsweg frei zu wählen, den er oder sie für richtig halten. Die Forderung der International Association of STM Publishers, dass "Autoren sollten in einem gesunden, unverzerrten freien Markt frei wählen können, wo sie publizieren" \cite{Brussels_Declaration_2007}, zeigt deutlich diesen Bias in der Argumentation im Rahmen des bestehenden Systems.

Diese grundsätzliche Darstellung, dass die Wissenschaft als Prozess der Wissensbildung und Wissensvermittlung in Deutschland durch das Grundgesetz abgesichert ist, zeigt, dass Freiheit von Wissenschaft und Forschung eine Bedingung für die Wahrheitssuche der Wissenschaft ist \cite{Oezmen_2015}. Neben diesem rechtlichem Schutz sichern auf der einen Seite das wissenschaftliche Ethos und die Regeln des wissenschaftlichen Diskurses, auf die im Verlaufe der Arbeit eingegangen wurde, die Autonomie und die Unabhängigkeit der Wissenschaft von politischen und gesellschaftlichen Interessenslagen \cite{Oezmen_2015}: "Politik gehört nicht in den Hörsaal" \cite[:494]{Weber_1992}. Weitere Anknüpfungspunkte zur Forderung nach Öffnung wissenschaftlicher Kommunikation im Spannungsfeld der Freiheit von Wissenschaft und Forschung stellen in diesem Zusammenhang auch die Dual-Use-Problematik und der Umgang mit Datenschutz dar \cite{Fritsch_2015}.

\subsection{Kosten und Effizienz}

An dem Kosten-Nutzen-Verhältnis des aktuellen wissenschaftlichen Kommunikationssystems gibt es seit Jahren detaillierte und grundsätzliche Zweifel. Für die Veröffentlichungen einzelner Texte ergeben sich je nach Schätzungen unterschiedliche hohe Kosten. Berechnungen des Wissenschaftsjournalisten Richard Van Noorden ergaben Kosten von 4.871 Dollar pro veröffentlichtem Text im tradierten Print und Online Subskriptionsmodell ohne freien Zugang, 3.509 Dollar bei der reinen Onlineveröffentlichung im Subskriptionsmodell ohne freien Zugang und 2.289 Dollar unter den Bedinungen von Open Access frei zugänglich. \cite{van_2013_true}. Wissenschaftliches Wissen kann für das wissenschaftliche System allerdings nur dann als umfassend effizient betrachtet werden, wenn das neue Wissen frei und offen für andere Forscherinnen und Forscher zur Verfügung steht. Im analogen System war dies auf Grund der Bindung des Wissens an das Speichermedium Druckerzeugnisse nur durch hohe Kosten für die Erstellung, den Vertrieb, die Sicherung und Verbreitung möglich.

Mit Beginn der Verbreitung elektronischer Publikationen kam es zu einer Umkehr des Bring- zum Holprinzip bei der Verbreitung wissenschaftlicher Publikationen. Die Erwartungen an die neuen Kanäle richten sich vor allem darauf, mit elektronischen Publikationen die Publikations- und Vertriebszyklen kostengünstiger und effizienter zu machen \cite{Brueggemann-Klein_1995}. Die Vermutung Ende der 1990er Jahre: "Einsparungen in Zeit, Raum und Kosten werden erheblich sein, wenn zunehmend Schreib- und Publikationstätigkeiten in den elektronischen Raum verlegt werden" \cite{roberts_1999_scholarly}. Doch nach mehreren Dekaden der Verfügbarkeit dieser "elektronischen Räume" hat sich herausgestellt, dass es sich beim wissenschaftlichen Kommunikationssystem um ein "sozial ineffizientes" System \cite[:47]{mueller-langer_2010}, bei dem die Publikations- und Vertriebszyklen weder kostengünstiger noch merklich effizienter geworden sind, handelt.

Obwohl die zunehmende Verbreitung digitaler Systeme im wissenschaftlichen Alltag die Möglichkeit eröffnet haben, nicht nur Publikationen schnell und umfassend zu veröffentlichen, sondern auch Daten und Informationen hinter Publikationen zu veröffentlichen, stehen Publikationen und Daten selten für die digitalen Informationsversorgung offen für die Gesamtgesellschaft zur Verfügung. Dennoch wird eine Effizienzsteigerung durch die Möglichkeit der Zweitnutzung und Weiterverwendung von Daten, die während des wissenschaftlichen Erkenntnisprozesses entstehen, vermutet \cite{RIN_2010_open_research}.

Der restriktiven und geschlossene Umgang mit Publikationen, Daten und wissenschaftlichen Informationen im aktuelle System verhindert nicht nur die wissenschaftsinterne, sonder auch die gesamtgesellschaftliche Nutzung der neuen Möglichkeiten für kollaborative Arbeit und den umfassenden Zugriff auf zusätzliche Forschungsergebnisse, bessere Bildung, neue Möglichkeiten und Nutzungsszenarien und eine umfassendere Aufzeichnung, Evaluation und Darstellung von Wissen.

Weder die Kosten für das System der wissenschaftlichen Kommunikation noch die Effizienz im Rahmen der Produktion von neuem Wissen aus bestehendem Wissen werden im gegenwärtigen Kommunikationspraktiken optimal genutzt. Die Auswirkungen dieser Ineffizienz führen zu einem erhöhtem (Zeit)Aufwand seitens der am Kommunikationssystem beteiligten Akteure und zur Verschwendung von Ressourcen \cite{nosek_2012_scientific}.

\subsection{Fehlerresistenz und Qualitätssicherung}

Damit der Erkenntnisfortschritt im Kommunikationsprozess gelingt braucht es Verlässlichkeit bei der Vermeidung von Fehlern im wissenschaftlichen Erkenntnisprozess \cite{Bargheer_2015}. Trotz des aufwändigen wissenschaftlichen Qualitätssicherungssystems kommt es immer wieder zu Fehlern und falschen Aussagen bei der Veröffentlichung wissenschaftlicher Erkenntnisse und Ergebnisse \cite{brembs2015open} \cite{Luescher_2014}. Die Gründe für diese Fehler sind vielfältig und erstrecken sich von Nachlässigkeit über Fahrlässigkeit bis hin zu Vorsatz.

In der Literatur werden folgende Faktoren als Herausforderungen für die Absicherung der Fehlerresistenz genannt:
\begin{enumerate}
\item Geschlossene Begutachtungsverfahren ermöglichen nur eine kleinen Anzahl an Gutachtern wissenschaftliche Inhalte auf Fehler zu prüfen \cite{suchen}
\item Nichtverfügbare Methoden und Daten hinter den Publikationen behindern die Qualitätssicherung und Reproduzierbarkeit von Wissen \cite{suchen}
\item Nicht dokumentierte und veröffentlichte Kommunikation während des wissenschaftlichen Erkenntnisprozesses, macht es unmöglich Fehler bereits bei der Erstellung der Publikation sichtbar und transparent nachvollziehbar zu machen \cite{suchen}
\end{enumerate}

Die Fehlerresistenz des wissenschaftlichen Kommunikationssystems ist demnach durch seine Geschlossenheit beeinträchtigt. Hier gibt es einen weiteren Anknüpfungspunkt zu Open-Source-Bewegung im Rahmen der Softwareentwicklung, bei der die Öffnung des Quellcodes von Software eine Möglichkeit der Sicherung der gewünschten Funktionstüchtigkeit und Sicherheit darstellt \cite{hoepman_2007_increased}. Darüber hinaus werden durch die Öffnung auch langfristig die Fehler einseh, reproduzier- und nachverfolgbar, die durch Nachlässigkeit oder Fahrlässigkeit aber auch durch Vorsatz entstanden sind. Das ermöglicht eine bisher nicht mögliche Berücksichtigung durch andere Wissenschaftler und Wissenschaftlerinnen im kummulativen Prozess der Generierung von neuem Wissen und stellt einen neuen Ansatz zum Erkenntnisgewinnung dar, der im geschlossenen Kommunikationssystem nicht möglich ist.

Wenn die Quelldokumente und Daten auch zum Zeitpunkt der Erstellung geöffnet sind, können interessierten Akteure die Informationen auf Fehler testen und gegebenenfalls Fehler schnell und umfassend bereinigen. Dadurch ist nicht nur eine Erhöhung der Qualität von wissenschaftlichen Inhalten sondern auch eine Erhhöhung der Fehlerresistenz bei Abschluss des jeweiligen wissenschaftlichen Erkenntnisprozesses zu erwarten.

\subsection{Verbreitung und Zugänglichkeit}

Ebenso, wie die Frage nach der optimalen Geschwindigkeit des aktuellen wissenschaftlichen Kommunikationssystems, stellt sich die Frage nach der optimalen Verbreitung und Zugänglichkeit wissenschaftlicher Informationen. Während die Geschwindigkeit auf die zeitliche Komponente von der Herstellung bis zum Vertrieb des Wissens abzielt, geht es bei der Frage nach der Verbreitung um die Verfügbarkeit des Wissens für eine möglichst große Rezipientengruppe. Es gibt erhebliche Zweifel daran, dass es sich bei dem aktuellen System um ein System mit optimalen Voraussetzungen für eine möglichst hohe Verbreitung von Wissen an die Gesamtgesellschaft handelt \cite{suchen}.

Noch heute ist das gedruckte Werk neben dem persönlichen Austausch auf Konferenzen oder Kongressen \cite{winkler_2011_anforderungen} für die Wissenschaftler und Wissenschaftlerinnen eine der maßgeblichen Informationsquellen, . Analoge Publikationen und Verbreitungswege sind allerdings beim ortsübergreifenden Austausch stark beschränkt. Und selbst bei der Verbreitung der Informationen die bereits digitalisiert worden sind, oder bereits bei Erstellung digital vorlagen werden sie im aktuellen System noch immer häufig durch Zugangsbarrieren wie Bezahlschranken gehemmt und damit die Zirkulation von Wissen eingeschränkt.

Die Herausforderung im aktuellen System besteht zum Einen aus der Bereitstellung der wissenschaftlichen Informationen über die unterschiedlichen Kommunikationskanäle hinweg und zum anderen in der langfristigen Sicherung und Bereitstellung dieser Informationen. Der digitale Transformationsprozess stellt in diesem Zusammenhang eine weitere Herausforderung und einen Ausweg zugleich dar, denn obwohl die Verarbeitung digitaler Daten heute ein wesentlicher Bestandteil der allermeisten wissenschaftlichen Vorhaben ist \cite{winkler_2011_anforderungen}, müssen die Informationen meist auf dem gedruckten und digitalen Speichermedium vorgehalten werden. Auch die vornehmlich durch Verlage praktizierte reine Digitalisierung des analogen Subskriptionsmodells für den Zugriff auf wissenschaftliche Inhalte \cite{Hanekop_2014} \cite{boai_2012} stellt eine Barriere für den Zugang zu den Informationen auch außerhalb der wissenschaftlichen Institutionen dar, da digitalisiertes Wissen weiterhin auf den Ort des analogen Wissens beschränkt bleibt.

\subsection{Digitalisierung}

Wie im Kapitel "Wissenschaftliche Kommunikation" beschrieben, ist die Verarbeitung digitaler Daten heute ein wesentlicher Bestandteil der meisten wissenschaftlichen Vorhaben. Obwohl die wissenschaftliche Arbeit und die wissenschaftliche Kommunikation überwiegend an digitalen Geräten stattfindet, wird noch immer für den Druck produziert. Während Wissenschaftler und Wissenschaftlerinnen schon seit Ende des letzten Jahrhunderts überwiegend mit Hilfe von Textsystemen schreiben \cite{Brueggemann-Klein_1995} \cite{bjork_2004_open} haben Verlage erst mit großer Verzögerung auf die elektronische Produktion von Wissen reagiert.

Auch die wissenschaftlichen Rohdaten und Informationen werden bei Abschluss des Erkenntnisprozesses (Publikation der Ergebnisse) umkodiert um analog publiziert zu werden und auch die rein digitalen Versionen der Publikationen entstehen überwiegend noch immer aus Informationen die für die analoge Publikation kodiert worden sind. In diesem Prozess kann ein Großteil der erzeugten Daten nicht weiter genutzt werden und viele der Informationen gehen verloren, beziehungsweise stehen nur selten für die Nachnutzung zur Verfügung.

Auch im Rahmen des Vertriebs beschränkt sich die Digitalisierung der wissenschaftlichen Kommunikation bisher in vielen Fällen noch immer darauf, dass die analog gedruckten und bewährten Journale, sowie andere Publikationsformen der großen wissenschaftlichen Verlage mit nahezu unverändertem Geschäftsmodell digital verbreitet werden \cite{Hanekop_2014} \cite[:179]{Fehling_2014}. Die digitale Distribution wird in diesem Zusammenhang als weiterer Kanal nach dem Drucken der Informationen verstanden.

Die Möglichkeiten, die die Digitalisierung für die wissenschaftliche Informationsversorgung bietet, sind damit bei weitem nicht ausgeschöpft. Es stehen zwar zunehmend nicht nur digitalisierte Informationen ehemals analoger Veröffentlichungen orts- und zeitunabhängig zur Verfügung, sondern auch wissenschaftliche Sammlungen stehen inzwischen im Fokus. Ebenso wird den Metadaten über oder Digitalisate relevanter Objekte ein großes Potenzial für die Wissenschaft zugesprochen \cite{winkler_2011_anforderungen}. Die Anzahl dieser Daten ist aktuell jedoch noch stark begrenzt.

Die Herausforderungen im bestehenden System formeller wissenschaftlicher Kommunikation bezieht sich bei der Digitalisierung vor allem auf die ungenutzten Potenziale einer umfassenderen Verbreitung und Kommunikation wissenschaftlicher Erkenntnisse. Diese werden im aktuellen System bei der Veröffentlichung nur selten genutzt und das derzeitige System der wissenschaftlichen Veröffentlichungen arbeitet noch immer gegen die maximale Verbreitung der wissenschaftlicher Informationen und Daten hinter den eigentlichen Publikationen \cite{Molloy_2011}.

\subssection{Überprüfbarkeit der wissenschaftlichen Güte: Objektivität, Reliabilität und Validität}

Wissenschaftliches Wissen zeichnet sich gegenüber anderen Formen des Wissens dadurch aus dass es Prüfprozeduren gibt, mit denen das spezifisch wissenschaftliche Wissen geprüft wird \cite{Luhmann1998}. Bisher wurde durch die formelle Publikation festgeschrieben, was nach den Kriterien des jeweiligen Fachs beziehungsweise der jeweiligen Disziplin als geprüftes Wissen gelten kann \cite[:11]{bbaw_publizieren_2015}. Hier werden beispielhaft die Herausforderungen an die Prüfbarkeit der Gütekiterien Objektivität, Reliabilität und Validität im wissenschaftlichen  Kommunikationssystem dargestellt.

Unabhängigkeit (Objektivität) in der Wissenschaft gilt für die Sammlung, Aufzeichnung, Analyse, Interpretation, gemeinsame Nutzung und Speicherung von Daten, sowie andere wichtige Verfahren in der Wissenschaft, wie zum Beispiel die Veröffentlichungspraxis und Peer-Review \cite{resnik_2005_ethics}. Die Kenntnis von Eigenschaften der Autoren durch die Gutachter stellt eine der größten Herausforderungen für die Wahrung der Objektivität und Unabhängigkeit im wissenschaftlichen Qualitätssicherungsprozess dar. Aber auch bei anderen Formen der wissenschaftlichen Bewertung können Unabhängigkeit und Objektivität nicht immer uneingeschränkt gewährleistet werden. In der Literatur finden sich Beiträge, die mehrheitlich zu dem Ergebnis kommen, dass die Objektivität und Unabhängigkeit im bestehenden System nur schwer bis nicht gesichert werden können \cite{binswanger_2014_excellence}.

Resnik beschreibt diesbezüglich folgende Herausforderungen an das bestehende geschlossenen System der wissenschaftlichen Kommunikation und an die Wahrung der Objektivität und an das selbstkorrigierende System der Wissenschaft \cite{resnik_2005_ethics}:
\begin{enumerate}
\item Präzision der wissenschaftlichen Arbeit
\item Ehrlichkeit bei der Datenerhebung und Darstellung der Ergebnisse
\item Vermeidung von Fehlverhalten
\item Vermeidung von Fehlern und Selbsttäuschung
\item Offenlegung Interessenskonflikte
\item Offenheit bezüglich Daten, Ideen, Theorien und  Ergebnissen
\item bewusstes Datenmanagement und Dokumentation
\end{enumerate}

Die Zuverlässigkeit (Reliabilität) des Kommunikationssystems kann anhand dessen geprüft werden, ob die Einreichung einer Arbeit über unterschiedliche Wege den selben Erfolg hat beziehungsweise, wie stark Zufallsfaktoren den Erfolg der Veröffentlichung wissenschaftlicher Erkenntnisse beeinflussen. Hier bestehen im aktuellen System wenig Möglichkeiten der Überprüfung. Die Verbreitung der Informationstechnologien ermöglicht zwar ein umfassenderes Monitoring der Aktivitäten von Wissenschaftlerinnen und Wissenschaftlern, eindeutige Sicherheit kann jedoch nicht gewährleistet werden.

Im Gegenteil, die umfassende Replizierbarkeit und Zuverlässigkeit von Ergebnissen kann aktuell stark kritisiert und bezweifelt werden \cite{Luescher_2014}. Das liegt zum einen an den Herausforderungen im Zusammenhang mit der meist nicht praktizierten Veröffentlichung von (Roh-)Daten, zum anderen an der Verwendung von geschlossenen Systemen und Formaten sowie fehlender Transparenz im Rahmen der genutzten Methoden und Verfahren. Die Transparenz muss dabei nicht zwangsläufig ein Widerspruch zur Notwendigkeit von Unabhängigkeit und Objektivität verstanden werden, da offene Verfahren auch anonym stattfinden können. Als weitere kritische Faktoren für die Wahrung der Zuverlässigkeit im Kommunikationssystem werden in der Literatur unter anderem Lücken im Qualitätssicherungsprozess (siehe auch "Fehlerresistenz") \cite{bar_2009_wissenschaftliche} und der zunehmende zeitliche Druck im Rahmen der Qualitätssicherung \cite{Luescher_2014} genannt.

Die Herausforderungen an die Überprüfbarkeit der Gültigkeit (Validität) der für den Druck bestimmten wissenschaftlichen Arbeiten und deren Ergebnisse schließen nahtlos an die anderen genannten Kriterien der Güte an. Im Unterschied zur Zuverlässigkeit ermöglicht die Überprüfung der Validität die Eignung der eingesetzten Meßverfahren zur Beantwortung der wissenschaftlichen Fragestellungen und Zielsetzungen. Auch hier sind die Möglichkeiten bei gedruckten Publikationen durch den fehlenden Zugang zu Daten und Informationen, die während des wissenschaftlichen Erkenntnisprozess entstehen bisher eingeschränkt.

\subsection{Verhinderung von Missbrauch und wissenschaftliches Fehlverhalten}

Neben der Notwendigkeit für eine umfassenden Überprüfbarkeit des Wissens, stellen die ethischen Grundsätze in der wissenschaftlichen Debatte von Beginn an eine Besonderheit dar. Vertrauen, das Interesse aller Akteure an optimaler Kommunikation zwischen den Wissenschaftlern, Ehrlichkeit und der Ausschluss von Interessenskonflikten sind Grundpfeiler im wissenschaftlichen Forschungs- und Kommunikationsprozess \cite{Bargheer_2015} \cite{wr_2015_wissenschaft_integritaet}. "Betrug ist dabei zwingend an die Absicht zu täuschen gebunden" \cite{Luescher_2014}.

Es muss das Anliegen jedes Forschers sein, "die Wahrheit und nichts als die Wahrheit zu suchen und zu berichten" \cite{Luescher_2014}. Drüber hinaus gilt: "Ohne Vertrauen in die Ehrlichkeit von Forschern gäbe es keine Wissenschaft mehr" \cite{hagner_2015_sache_buches}. Vertrauen und Redlichkeit bilden die Grundlage der Wissenschaft \cite{Bargheer_2015} auch wenn diese auf einer "delikaten Struktur weitgehend ungeschriebenen Regeln" \cite{grand_2012_open} beruhen.

Auch wenn die Wissenschaft "eine besondere ethische Verantwortung" hat, sind Formen von "Fehlverhalten, Betrugsfälle und Nachlässigkeiten, die in anderen Lebensbereichen geschehen können, auch in der Wissenschaft möglich" \cite{wr_2015_wissenschaft_integritaet}. Diesem wissenschaftlichem Ethos stehen die Beispiele gegenüber, bei denen bewusster Missbrauch durch Akteure des Kommunikationssystems zu Verwirklichung partikularer Interessen oder konkreten Einfluss auf wirtschaftliche Aspekte geführt haben \cite{Luescher_2014}  \cite{binswanger_2014_excellence} \cite{Beall_2012}.

Margo Bargheer und Birgit Schmidt klassifizieren wissenschaftliches Fehlverhalten wie folgt \cite{Bargheer_2015}:
\begin{enumerate}
\item Unlauterer Umgang mit Ergebnissen (z.B. erfundene Ergebnisse)
\item Unlauteres Forschungsverhalten (z.B. Unzulässige Forschungsmethoden)
\item Fehlverhalten im Datenmanagement (z.B. Zurückhalten von Daten wider besseres Wissen )
\item Fehlverhalten im Publikationsprozess (z.B. Unangemessene Partitionierung von Ergebnissen „Salamitaktik“  \cite{binswanger_2014_excellence})
\item Soziales Fehlverhalten (z.B. Sabotage oder Behinderung der Arbeit Anderer )
\item Administratives Fehlverhalten (z.B. Verstoß gegen Verwendungsrichtlinien)
\end{enumerate}

Gegen ein solches Fehlverhalten im Rahmen der wissenschaftlichen Kommunikation wurden die internationalen Leitlinien "Principles of Transparency and Best Practice in Scholarly Publishing" \cite{oaspa_principles_2013} veröffentlicht, "sie sollen die Qualitätsstandards im Publikationswesen und zugleich die Filterfunktion der initiierenden Mitgliedsorganisationen stärken" \cite{Bargheer_2015}. Bisher kommen die wenig vorhandenen Studien zu dem Ergebnis, dass abgelehnte Manuskripte, sofern sie andernorts veröffentlicht wurden, deutlich weniger zitiert wurden \cite{Hornbostel_1997}. Mit Blick auf die neuen Möglichkeiten der ergänzenden Veröffentlichung von Meta-Informationen und Daten zusätzlich zur finalen Publikation, ist zu vermuten, dass die Möglichkeiten zur Sicherung der Qualität im bestehnden System nicht ausreichen.

Auch wenn noch nie zuvor über Betrug in der Wissenschaft so intensiv berichtet worden ist \cite{brembs2015open}, wie in den letzten Jahren, ist es "keineswegs ausgemacht, dass die Intensität der Berichterstattung allein auf die tatsächlich gestiegene Inzidenz von Betrug" \cite{weingart_2005_wissenschaft}, sondern eher auf den Anstieg medialer Beobachtung zurückzuführen ist. Dennoch stehen die intransparenten Verfahren, die bisher mangelhafte Veröffentlichung von Supplementen und (Roh-)Daten der Verhinderung von Missbrauch und wissenschaftliches Fehlverhalten entgegen. Demnach ist zu vermuten, dass die Bereitschaft der Forschenden zu stärken, positive wie negative Daten zu teilen und zurückgezogene Artikel sichtbar zu machen, helfen können die notwendigen effektiven Mechanismen zur Verfolgung wissenschaftlichen Fehlverhaltens \cite[:14]{wr_2015_wissenschaft_integritaet} zu installieren.

\section{Anknüpfungspunkte zur Forderung nach Öffnung der wissenschaftlichen Kommunikation}

Wie dargestellt, steht das wissenschaftliche Kommunikationssystem vor mannigfaltigen Herausforderungen. In folge der zunehmenden Privatisierung wesentlicher Bestandteile des wissenschaftlichen Kommunikationsprozess, haben sich die akademischen Ziele und die Marktinteressen der privatwirtschaftlichen Anbieter immer weiter voneinander entfernt. Zum Beispiel sind im Zeitraum von 1986 bis 2012 die Ausgaben für Bibliotheksbestände in den Vereinigten Staaten um 322 Prozent gestiegen \cite{lewis_2015_future}. Den Verlegern werden Betriebsgewinnmargen von über 35 Prozent \cite{russell_2008_business} \cite{cope2014future} sowie hohe jährliche Wachstumsraten \cite{Martin_2013} \cite{Wellcome_Trust_2003}. Die drei größten Wissenschaftsverlage vereinen bereits 42 Prozent aller Journale und trotz der internationalen Finanzkrise stiegen die Umsätze ungebremst weiter. In den Jahren zwischen 2008 und 2011 stiegen die Umsätze um 11,7 Prozent und die Gewinne von 1,6 Milliarden auf 1,9 Milliarden Dollar (17 Prozent) \cite{cope2014future}. Das entspricht einer Umsatzrendite von 35,8 Prozent. Im Vergleich dazu lagen die durchschnittlichen Umsatzrenditen im Wirtschaftszweig "Verlagswesen" bei deutschen Firmen mit mehr als 50 Mitarbeitern lag laut der Bundesbank dem im Jahr 2011 bei 11,6 Prozent \cite{bundesbank_2014}. Es sind nur wenig Beispiele bekannt, "in denen das symbolische Kapital in so außerordentlichem Maße zu ökonomischen Kapital verdinglicht worden ist", wie bei dem Geschäftsmodell des wissenschaftlichen Publizierens \cite{hagner_2015_sache_buches}.

Als möglicher Ausweg aus dieser Krise wird immer wieder die Digitalisierung und Öffnung der wissenschaftlichen Kommunikation durch Konzepte wie Open Access und Open Science genannt \cite{lewis_2015_future}. Beide Begriffe umfassen eine Vielzahl von Möglichkeiten für die Zukunft der Wissensbildung und Wissensverbreitung. Sie funktionieren als Sammelbegriffe für unterschiedliche Auffassungen wie weit und in welcher Form Wissenschaft offener werden kann. Sie sind Bestandteil eines notwendigen Diskurses in der wissenschaftlichen Gemeinschaft \cite{schulze_2013_open}. Kleinster gemeinsamer Nenner in diesem Diskurs um die unterschiedlichen Konzepte ist, "dass wissenschaftliche Forschung sich irgendwie mehr öffnen muss" \cite{cite:9}.

Die etablierten Prozesse wissenschaftlicher Kommunikation stehen vor umfangreichen Herausforderungen. Die Zeitschriften- und Monographienkrise, der zunehmende finanzielle Druck im Rahmen der öffentlichen Finanzierung von Wissenschaft, die Veränderungen im wissenschaftlichen Kommunikationsprozess durch neue Arten und Möglichkeiten der Distribution, die steigenden Beschaffungskosten für wissenschaftliche Literatur \cite{cite:17} \cite{muller_2010_open}, sowie die Veränderungen in der Rezeption von Inhalten \cite{holub_2013_reception} zwingen zum Umdenken in der wissenschaftlichen Kommunikationspraxis \cite{suchen}. Die anhaltende Forderung nach mehr Offenheit im wissenschaftlichen Kommunikationsprozess entwickelte sich zu einem konkreten Lösungsansatz für die Herausforderungen an das etablierte System.

Die Öffnung der wissenschaftlichen Kommunikation ist eine große Chance für Veränderungen im wissenschaftlichen Qualitäts- und Reputationssystem zu erwirken. Bisher werden wissenschaftlichen Erkenntnisse häufig erst nach langen intransparenten Verfahren bewertet, publiziert und nur an einen beschränkten Kreis von Rezipienten vermittelt. Diese intransparente Praxis hat einen signifikant-negativen Einfluss für Allokation von Ressourcen durch die öffentliche Hand und die Kosten die im Rahmen öffentlich-finanzierter Forschung entstehen. Die Öffnung der wissenschaftlichen Kommunikation würde demnach entgegen der bisherigen Praxis eine stärkere Berücksichtigung der Aspekte Aktivität der Wissenschaftler und die Qualität der Forschungsergebnisse ermöglichen \cite{suchen}.

Als Auslöser für die Entwicklung von Open Access werden auch die infrastrukturellen Veränderungen angeführt, die "seit spätestens Mitte der 1990er-Jahre entscheidend Einfluss auch auf die Wissenschaftskommunikation und das wissenschaftliche Arbeiten genommen haben" \cite{schulze_2013_open}. Open Access entwickelte sich vorerst primär in den STM-Fächern, in denen viel Aufmerksamkeit auf der Selbstarchivierung der Wissenschaftler und Wissenschaftlerinnen vor der finalen Publikation (Pre-Print) in privaten, zentralen oder institutionellen Repositorien lag \cite{adema_2013_political} und bei denen die Auswirkungen der Zeitschriftenkrise am stärksten zum Tragen kam \cite{naeder_2010_open}. Wissenschaftliche Informationen werden seither nicht nur in "digitaler Form konsumiert, sondern auch kollaborativ und kooperativ, zeitlich versetzt, durch teilweise räumlich weit verstreute Arbeitsgruppen und Forschungsverbünde, genutzt und weiterverarbeitet" \cite{schulze_2013_open}. Die Verbreitung und Akzeptanz von Open Access variiert dabei zwischen den einzelnen wissenschaftlichen Disziplinen erheblich \cite{cite:21a}.

In der Debatte über die Zukunft des wissenschaftlichen Publizierens und Kommunizierens besteht die Tendenz, Konzepte der offenen Wissenschaft als einen bisher beispiellosen und noch nie dagewesenen Wandel darzustellen \cite{cite:17a} \cite{cite:17b}. Diese Haltung basiert auf "verschiedenen Gründungsmythen", die auf "unterschiedliche Zielsetzungen und Lösungspfade" verweisen \cite{suchen-Hoffmann-Zugang-undVerwertung-oeffentlicher-Informationen}. Die Geschichte von Open Access ist eine Entwicklung, die unter anderem eng mit der Digitalisierung von Kommunikationsprozessen auf der einen, sowie mit der Zeitschriftenkrise auf der anderen Seite verknüpft ist \cite{suchen-Hoffmann-Zugang-undVerwertung-oeffentlicher-Informationen} \cite{yiotis_2013_open} \cite{wein_2010_erwerbung}. Open Access ist kein Selbstzweck \cite{cite:17d}, sondern ein Attribut tiefergehender Prozesse, die mit der wachsenden Bedeutung der Digitalisierung in unserer Zivilisation und dem damit einhergehenden Wandlungsprozessen im Machtgefüge zusammenhängen \cite{cite:17e}. Es bleibt jedoch herauszuheben, dass es trotz vereinzelter Versuche, wissenschaftliche Informationen und Publikationen offen und frei zu kommunizieren, Open Access im Printzeitalter physisch und ökonomisch über lokale Grenzen hinaus schwer möglich war \cite{cite:18a}.

Die Forderung nach der Öffnung von Wissenschaft und Forschung ist in diesem Zusammenhang nicht nur eine "politische Reaktion" oder "technische Alternative", sondern eine "alternative Formatierungen einer wissenschaftlichen Infrastruktur im technischen, rechtlichen und zeitlichen Sinne" \cite{kelty_2004}. Sie betrifft "Wissenschaftler, politische Entscheidungsträger und die Öffentlichkeit" \cite{Scheliga_2014}. Bei der genauen Betrachtung dieser Forderungen nach Öffnung wissenschaftlicher Kommunikation ist es zwingend erforderlich die Konzepte von Open Access und Open Science zu unterscheiden. Open Access bezieht sich einen möglichst uneingeschränkten Zugang zu finalen wissenschaftlichen Ergebnispublikationen für die Gesamtgesellschaft. Open Science beschreibt hingegen den umfassenden Zugriff auf den gesamten wissenschaftlichen Erkenntnisprozess inklusive aller Daten und Informationen, die bereits bei der Erstellung, Bewertung und Kommunikation der wissenschaftlichen Erkenntnisse entstanden sind.

\subsection{Offener Zugang zur wissenschaftlichen Kommunikation: Open Access}

Open Access wird von Peter Suber "zugespitzt" \cite{naeder_2010_open} als "digital, online, kostenlos, und frei von den meisten Urheber- und Lizenzbeschränkungen" \cite{suber_2012_open} eingegrenzt \cite{Adema_2014_open_access}. Open Access bedeutet den "Verzicht auf die finanzielle, technische und rechtliche Hindernisse, die dazu bestimmt sind, den Zugang zu wissenschaftlichen Forschungsartikel für zahlende Kunden zu begrenzen" und dass, "im Interesse der Beschleunigung der Forschung und den Austausch von Wissen, Verlage ihre Kosten aus anderen Quellen schöpfen" \cite{Suber_2002}. Die meisten programmatischen Erklärungsversuche sehen Open Access demnach "als adäquate Selbsthilfe wissenschaftlicher Autoren und Institutionen gegen die diskurshemmenden Auswirkungen der 'Zeitschriftenkrise'” \cite{naeder_2010_open}. In der Literatur werden unterschiedliche Auffassungen über die Bestimmung von Open Access, wie es erreicht werden kann und welchen genauen Bezugsrahmen das Attribut "Open" beschreibt \cite{Adema_2014_open_access} \cite{cite:17}. Dies ist darauf zurückzuführen, dass es keine formelle Struktur, keine offizielle Organisation und keinen ernannter Führer innerhalb der Open Access Bewegung gibt \cite{poynder_2011_suber}. Darüber hinaus sind die existierenden "Definitionen" meist interessengeleitet und führen dazu, "Kriterien, Methoden, Ziele und Folgenabschätzungen ineinander zu verflechten" \cite{naeder_2010_open}.

Einigkeit besteht darin, dass es der Forderung nach Open Access nicht um die Abschaffung oder die Entwertung materiellen geistigen Eigentums geht. Kaum jemand bestreitet, dass Open Access mit dem Urheberrecht, mit dem Peer-Review System, mit Einnahmen (auch Gewinn), dem Drucken, der Erhaltung, Reputation, Qualität, wissenschaftlichen Karriere-Fortschritt, der Indexierung, und andere Merkmale und unterstützende Aspekte die mit dem herkömmlichen wissenschaftlichen Publikationssystems assoziiert werden kann \cite{suber_2015}. Die Unterstützer dieser Idee vereint das gemeinsame Ziel, "die Bedingungen zu verbessern, unter denen wissenschaftliche Arbeiten zirkulieren können"\cite{Adema_2014_open_access}. Die Propagierung der Öffnung der wissenschaftlichen Ergebnisse erstreckt sich dabei vor allem auf "Publikationen, die nicht darauf angelegt sind, Einnahmen aus Verkaufserlösen für ihre Urheber zu generieren" \cite{muller_2010_open}.

Nachfolgende Zitate dienen exemplarisch der Darstellung des Terminus von Open Access in dieser Arbeit. Im Rahmen der Literaturrecherche wurden sie auf Grundlage hoher Zitationsraten als die gängigen Einordnungen für Open Access identifiziert:
\begin{figure}[h!]
\includegraphics{tableid:fnV4I}
\caption{Übersiche "Open Access" in der Literatur}
\end{figure}

---- TODO: Tabelle weiter ausfüllen - Auflistung Open Access Definitionen in der Literatur und Zusammenfassung der Definition ----

Im Unterschied zu den verschiedenen Interpretationen und Wegen von "Open Access" bietet das Attribut "Open" einen vielleicht eindeutigeren Bezugsrahmen für die Beschreibung des offenen Zugangs zu wissenschaftlichen Publikationen. Eine der Definitionen der Bedingungen von "Open" ist die Open Definition der Open Knowledge Foundation. Sie hat den Anspruch die Prinzipien und Bedingungen für die Offenheit von Daten und Inhalten zu definieren. Diese Definition von Offenheit setzt voraus, dass Daten und Publikationen als Ganzes und für nicht mehr als angemessene Wiederherstellungskosten (vorzugsweise als Download) und in einer bequemen und modifizierbaren Form verfügbar sein sollten \cite{Molloy_2011}.

Gemäß der Open Definition gilt der Inhalt als "Open", der "für jeden Zweck von jedem kostenlos genutzt, modifiziert und geteilt werden" \cite{open_definition} kann. Ziel dieser Definition ist es, "die Bedeutung von offen in Bezug auf Wissen" zu präzisieren. Wissen erstreckt sich in diesem Zusammenhang auf Inhalte wie Musik, Filme, Bücher, jegliche Art von Daten, ob wissenschaftlicher, historischer, geographischer oder anderer Art sowie Regierungs- und andere Verwaltungsinformationen \cite{open_definition}.

Die Open Definition wurde von der Open Source Definition abgeleitet und ist als synonym für "frei" oder "libre" im Rahmen der Definition für "freie kulturelle Werke" zu verstehen \cite{suchen}. Ein Werk oder Inhalt gilt nach dieser Definition als "offen", wenn es bei der Verbreitung folgenden Kriterien erfüllt:
\begin{enumerate}
\item Einhaltung der Prinzipien von Zugang, Verteilung, Wiederverwendung und dem Fernbleiben von technologischen Restriktionen
\item Attribuierung, Integrität als maximale Einschränkung
\item Unterbindung der Diskriminierung von Personen, Gruppen oder bestimmten Bereichen/Gebieten
\item Einhaltung genannten Kriterien  im Rahmen der Lizensierung
\end{enumerate}

Die Definition setzt eindeutige Kriterien, deren Erfüllung notwendig sind, um das Attribut "Open" zu verwenden. Wie bei den three B's kann ein Verstoß gegen diese Kriterien nicht sanktioniert werden, aber die öffentliche die Verwendung von von "Open" erschweren. Kommt es zu einem Verstoß gegen diese wird von "Openwashing" gesprochen. Die Definition dient damit vor allem den zivilgesellschaftlichen Akteuren zur Bloßstellung von Bestrebungen ohne die gewünsche ideelle Wirkung.

\subsubsection{Wege des Open Access Publizierens}

In der Literatur wird Open Access in unterschiedliche Formen unterteilt \cite{CREATe_2014} \cite{albert_2006_open_implications} und es bestehen unterschiedliche Auffassungen über die verschiedenen Erklärungsversuche von Open Access \cite{CREATe_2014} \cite{cite:22b} \cite{lewis_2012_inevitability}. Die meisten Begriffsbestimmungen von Open Access wie auch die Modelle orientieren sich an den "three Bs", den derzeit meist verwendeten Erklärungsversuche von Open Access \cite{Adema_2014_open_access}. Am Beispiel der Budapest Open Access Initiative werden zwei Wege für Open Access dargestellt \cite{albert_2006_open_implications}:
\begin{enumerate}
\item Die Etablierung "einer neuen Generation von Fachzeitschriften", die einen kostenfreien und unmittelbaren Zugang zu den Beiträgen ermöglichen ("goldener" Weg)
\item Die öffentlich zugängliche (Selbst-)Archivierung durch die Urheber ("grüner" Weg) \cite{adema_2013_political} \cite{hall_2008_digitize}
\end{enumerate}

Der "grüne Weg" beschreibt ein Modell, bei dem der Autor im Rahmen einer (Selbst-)Archivierung von Beiträgen in Repositorien (teilweise öffentlichen Dokumentenservern) die Verfügbarkeit seiner Publikation anstrebt \cite{brembs2015open} \cite{muller_2010_open} \cite{grand_2012_open}. Das vom Autor initial eingereichte Dokument (Manuskriptfassung) steht dabei als Pre-Print oder Post-Print-Version auf institutionellen oder disziplinären Dokumentenservern \cite{suchen} oder privaten Homepages \cite{suchen} jedem zur Verfügung. Im Unterschied zu Post-Prints, hat bei Pre-Print keine Peer Review stattgefunden \cite{suchen} und der Beitrag hat somit keine externe wissenschaftliche Qualitätssicherungsmaßnahme durchlaufen. Beim "grünen Weg" hat der publizierende Verlag darüber hinaus die Möglichkeit innerhalb einer Speerfrist von üblicherweise 6-12 Monaten \cite{suchen} den lektorierten und fertig-publizierten Beitrag unter einer eigenen Lizenz zu verkaufen \cite{suchen}. Erst nach Ablauf dieser Frist wird die finale und lektorierte Fassung des Beitrags frei und offen zur Verfügung gestellt. Es existieren je nach Verlag und Publikationsform verschiedenen Möglichkeiten der Ausgestaltung dieses Publikationsweges \cite{suchen}. Sie alle einigt die Möglichkeit für den Autor seinen eingereichten Beitrag unmittelbar, frei und kostenlos zu veröffentlichen und die freie und kostenlos Veröffentlichung der finalen Publikation durch den Verlag nach einer Speerfrist \cite{dorschel_2006_open}. Die vertragsrechtliche Ausgestaltung des grünen Weges ist vielfältig und reicht von einer tatsächlichen Beschränkung der Rechtseinräumung auf das für den Vertragszweck erforderliche Maß bis zu einer für Autoren und Archivaren ungünstigen "vollständigen Übertragung, gepaart mit einer schuldrechtlichen Gestattung einzelner Nutzungshandlungen nach Ablauf einer gewissen Schutzfrist" \cite{dorschel_2006_open}. Der grüne Weg ist demnach als Kompromiss für ein Open Access auf Grundlage der Interessen der Verlage anzusehen \cite{Mussell_2013}.

Beim "goldene Weg" stellt der Autor unmittelbar nach der Fertigstellung die finale und lektorierte Publikation über einen Verlag frei und offen zur Verfügung. Auch die Verlagsversion muss ohne Sperrfrist in einem Repositorium unmittelbar zur Verfügung gestellt werden. Der Verlag hat allerdings zusätzlich die Möglichkeit, den Beitrag kommerziell zu vertreiben und zu verkaufen, muss jedoch parallel eine freie und offene Version der Publikation zur Verfügung stellen.

Alternativ ermöglicht es der verzögerte goldene Open Access-Weg dem Verlag, zeitverzögert für die Öffentlichkeit die finale Version der Publikation unter einer offenen Lizenz zur Verfügung zu stellen \cite{lewis_2012_inevitability}. Der Verlag hat bei diesem verzögerten Modell den Vorteil, einen bestimmten Zeitraum die Publikation vertreiben zu können, ohne zeitgleich eine offene und freie Version anbieten zu müssen. Der Autor hat im Gegensatz zum "grünen Modell" aber dennoch die Möglichkeit diese finale Publikation vollumfassend sofort kostenfrei anzubieten.

Im Rahmen anderer Modelle, meist gemischter Modelle, wird den Autoren im Nachhinein die Möglichkeit eingeräumt, durch zusätzliche finanzielle Zahlung, die Publikation offen und frei zur Verfügung zu stellen\cite{lewis_2012_inevitability}. Das hat für den Autor den Nutzen, dass er von den Vorteilen bei der offenen Verbreitung von Publikationen unter den Bedingungen von Open Access profitiert. Macht der Autor davon erst nach einem gewissen Zeitraum gebrauch, generiert der Verlag neben den initialen Verkaufserlösen über diesen Weg zusätzliche Einnahmen. Diese alternativen Modelle ermöglichen es, dass parallel zu den kostenlosen und offenen elektronischen Veröffentlichungen, weitere kostenpflichtige Publikation in gedruckter oder digitaler Form erscheinen können. Eine Grundvoraussetzung dafür ist, dass neben der kostenpflichtigen Version, auch eine kostenfreie Version der Publikation unter den in der Open Definition erklärten Bedingungen existiert.

Darüber hinaus findet in der Literatur die Segmentierung in gratis und libre Open Access statt \cite{Martin_2013} \cite{naeder_2010_open} \cite{Mounce_2015}. Mit gratis Open Access wird dabei die Möglichkeit bezeichnet, den Zugang zu Publikationen und Forschungsergebnisse zu erleichtern und die Kostenpflichtigkeit zu beenden. Libre Open Access bedeutet, dass weitere Barrieren, wie Urheber- und Lizenzbeschränkungen aufgehoben werden. \cite{Adema_2014_open_access} Diese Unterteilung wird von einigen Autoren kritisiert, da durch das Hinzufügen eines weiteren Attributs die eigentlich scharfe Abgrenzung von "Close" und "Open" geschwächt wird, was sich auch auf andere Bereiche der Open-Bewegung (Open Government Data, Open Hardware, Open Educational Resources‎ uvm.) auswirken könnte \cite{suchen}. Diese Kritik kann auch auf die Modelle von Green und Golden Open Access ausgeweitet werden und die Differenzierung der Begriffe steht unter dem Verdacht grundsätzlich wenig oder falsch verstanden zu werden \cite{Mounce_2015}.

Neben den dargestellten Modellen existieren weitere Veröffentlichungsmodelle für Open Access Publikationen. Die Einteilung in hybride, radikale und sonstige Formen von Open Access stellt dabei eine weitere entwertete Ebene der Unterteilung dar \cite{Mounce_2015}. Weitere, aber im Vergleich wenig genutzte Modelle sind hybride Modelle. Als hybrid werden diese deshalb bezeichnet, weil der Autor wählen kann, ob er den Verlag für den kostenlosen Zugriff auf seine Publikation finanziert oder der Leser über das Subskriptionsmodell zahlt \cite{muller_2010_open}. Dieses Modell steht aber in der Kritik, da die rechtlichen Bedingungen nur selten eine Nachnutzung oder Weiterverbreitung erlauben und die Verlage nur selten auf das exklusive Verwertungsrecht verzichten \cite{muller_2010_open}. Diese Publikationsformen werden als Open Access bezeichnet, genügen aber nicht den gängigen Deklarationen \cite{boai_2012} oder verstoßen gegen die Open Definition. Entspricht eine Veröffentlichung nicht der Open Definition wird aber vom Verlag oder der herausgebenden Institution als "Open" bezeichnet, so wird auch von "Open Washing" gesprochen \cite{suchen}. Von einer weiteren Unterteilung der Open Access Modellen wird deshalb und aufgrund ihrer geringen Verbreitung und Praktikabilität in dieser Arbeit abgesehen.

Der verzögerte goldene Weg und grüne Weg beeinträchtigen das klassische Geschäftsmodell der Verlage vorerst nicht direkt. Publikationen werden wie bisher angeboten und erst nach einer bestimmten Zeit auch kostenlos zur Verfügung gestellt. Im Gegensatz dazu kommt der goldene Weg, auf Grundlage unmittelbarer, freier und offener Veröffentlichungspflicht, ohne das tradierte Geschäftsmodell der Verlage aus \cite{lewis_2012_inevitability}.

Allerdings werden für Publikationen, die unter den Bedingungen von Open Access veröffentlicht werden, durch die Verlage vorab Veröffentlichungsgebühren von den Autoren erhoben \cite{suchen}. Diese sogenannten Article Processing Charges (APC) werden damit gerechtfertigt, dass bei dieser Publikationsform weder auf den Peer-Review-Prozess, noch auf die Möglichkeit Umsatz zu generieren, Urheber zu schützen oder andere Stärken der traditionellen Publikationsformen verzichtet wird \cite{albert_2006_open_implications} \cite{Open_Access_net_2009}.

Somit ändert das Open Access Geschäftsmodell die Erlösstruktur der Verlage von nachgelagerten, verkaufsorientierten Einnahmen hin zu Vorabeinnahmen für die Erstellung und den Vertrieb der Publikationen. Strukturell steht Open Access für Verlage damit vorerst in keinem Widerspruch zur Bewahrung der wissenschaftlichen Qualität oder den Vorteile des klassischen Publikationssystems \cite{Suber_2002}. Verlage nutzen zwar Open Access-Optionen, wollen damit aber die etablierten Verhältnisse möglichst fortschreiben und halten am Subskriptionsmodell weiter fest \cite{schmidt_2007_goldenen}.

---- TODO: Tabelle - Auflistung Open Access  Modelle und Formen in der Literatur
Gegenstand / Zeitraum / Referenz Zusammenfassung der unterschiedlichen Definitionsmodelle & Darstellen wie die vermuteten Unterschiede in Ausschnitte zeigen deutlich zeigen, wie vielfältig allein die Definitionen der verschiedenen Wege des Open Access Publizierens sind ----

\subsubsection{Open Access Kanäle und Publikationsformate}

In diesem Abschnitt wird auf unterschiedliche Open Access Kanäle und Publikationsformate eingegangen. Es wird unterschieden in: Open Access Aggregatoren, Open Access Repositorien, Open Access Journals, Open Access Bücher und Monografien. Diese Kanäle und Formate adressieren die unterschiedlichen Publikationsformen der wissenschaftlichen Kommunikation oder konkrete Herausforderungen in Bezug auf die Distribution und Archivierung im Rahmen der neuen Möglichkeiten von offenem und freien Publizierens.

Da es eine enge Verknüpfung zwischen Repositorien und der Entwicklung der Open-Access-Bewegung gibt \cite{adema_2013_political} \cite{offhaus_2012_institutionelle_repos}, soll in diesem Kapitel auf die Rolle der Repositorien als spezifischen Kanal für die Verbreitung von Publikationen eingegangen werden. Repositorien sind verwaltete Orte zur Aufbewahrung geordneter Dokumente. Institutionelle Repositorien gelten als ein Instrument für wissenschaftliche Einrichtungen oder eine Gruppe von Einrichtungen, um Publikationen für einen institutionell in einem meist abgegrenzten Bereich frei zugänglich zu machen \cite{dobratz_2007_open} \cite{Baggs_2006}. Über die Hälfte der forschungsorientierten deutschen Universitäten betreiben ein solches institutionelles Repositorium \cite{Schmidt_2009}.

Institutionelle Repositorien haben erhebliche Vorteile für die Institutionen, wenn sie in die ganzheitlichen Rahmenbedingungen der Universität integriert sind \cite{steele_2006}. Repositorien können neben der Kernaufgabe der Archivierung und Verbreitung von Publikationen auf für die Lernumgebungen, den Forschungsservice und die Marketingaktivitäten einer Universität eine wichtige Rolle spielen. Sie ermöglichen zum Beispiel die Dokumentation des universitären Outputs und verbessern den institutionellen Austausch \cite{steele_2006}. Ökonomisch rentieren sie sich vor allem dann, wenn Skaleneffekte eintreten und Forschungseinrichtungen in Verbünden agieren \cite{blythe_2005value}. Neben den institutionellen sind auch fachliche oder andere Arten von Repositorien eng mit der Open Access Bewegung verknüpft. Repositorien stehen für die digitale Speicherung von Dokumenten und zunehmend auch Daten zur Verfügung. Sie entwickeln sich von "bloßen Repositorien für Literatur in Richtung digitaler Forschungsportale und -umgebungen", die "verschiedenste Materialien integrieren und damit nutzbar und zitierfähig machen" \cite{Schmidt_2009}. Digitale Repositorien bieten Mehrwertdienste, insbesondere die Erhebung von Nutzungsstatistiken, Zitationsanalysen und webometrischer Daten \cite{jahn_2011_personliche} \cite{mayr_2005_webometrie}. Über diese Repositorien wird der Zugang zu den unterschiedlichen Modellen und Publikationswegen von Open Access Publikationen ermöglicht \cite{suber_2015}.

---- TODO: Tabelle - Auflistung Open Access  Modelle und Formen in der Literatur Gegenstand / Zeitraum / Referenz Zusammenfassung der unterschiedlichen Definitionsmodelle & Darstellen, dass die Auflistung zeigt, wie unterschiedlich die Modelle des Open Access Publizierens sind & wie die verschiedenen Interpretationen zur Begriffsverwirrung beitragen ----

\subsubsection{Kritische Betrachtungen von Open Access}

Open Access ist nicht unumstritten. Kritik an Open Access kommt vor allem von den "etablierten Wissenschaftsverlagen, aber auch von Autoren, die um Einnahmen aus Autorenverträgen" \cite{Schirmbacher_oa_2007} und die Einschränkung der Wissenschaftsfreiheit fürchten. Neben der Kritik am ökonomischen Modell, sowie der Angst vor der Einschränkung von Freiheit in Forschung, Lehre und Forschungsheterogenität wird gar befürchtet \cite{Szczesny_2014}, dass Open Access es "tatsächlich in den Händen" hat ganze Publikationsformen, wie "dem geisteswissenschaftliche Buch ein Ende zu bereiten" \cite{Hirschi_2015_buch_oa}.

Aus der Perspektive der Leser gibt es demgegenüber wenig Kritik am Konzept von Open Access \cite{wein_2010_erwerbung} \cite{weishaupt_2009_goldenOA}. Sie bezieht sich, wenn überhaupt, vorwiegend auf die Befürchtungen aus den Konsequenzen der Öffnung für die Wissenschaft und Forschung sowie der Verknüpfung der Öffnung mit der Digitalisierung. Dabei werden vor allem die sinkende Forschungsheterogenität \cite{Hirschi_2015_buch_oa}, die eventuell steigende Einflussnahme durch die "Steuerzahler", die Gefahr der Medialisierung der Wissenschaft \cite{weingart_2005_wissenschaft} sowie die Konsequenzen einer Unterwanderung der Steuerungsmechanismen von Wissenschaft und Forschung genannt. Als theoretische Gefahr wird diesbezüglich beispielhaft die Gefahr der Au­ßer­kraft­set­zung des Wahrheitsmonopols der Wissenschaft durch das Aufmerksamkeitsmonopol der Medien genannt \cite{weingart_2005_wissenschaft} sowie die Angst des Verlustes der Möglichkeit der analogen Informationsversorgung durch "konzentriertes Lesen in einem Lesesaal" \cite{winkler_2011_anforderungen}.

Seitens der Autoren besteht bisher eine vergleichsweise geringe Akzeptanz für die Öffnung wissenschaftlicher Kommunikation, wenig Interesse an Open Access Publikationen. Es existieren noch immer "viele Vorbehalte und Missverständnisse" \cite{Suber_2002}. Diese fehlende Akzeptanz für Open Access in der wissenschaftlichen Gemeinschaft stellt die größten Herausforderungen für die Etablierung offener Kommunikation in der Wissenschaft und Forschung dar \cite{weishaupt_2009_goldenOA}. Die Vorurteile betreffen insbesondere die Verschiebung des Leser/Bibliotheken-Bezahl-System zum Autoren-Bezahl-System zur Refinanzierung des Publikationsprozesses \cite{EuropeanCommission_sciencepub_2006} \cite{Chibnik_2015}. Die meisten Autoren und Autorinnen vermuten, dass sie selbst im Rahmen des Systemwandels zukünftig für die Veröffentlichung der Texte zahlen müssen um die freie und offene Zugänglichkeit zu gewährleisten \cite{Mussell_2013} und das, obwohl schon bei konventionellen (nicht Open Access) Veröffentlichungen oft genug die Druckkosten selbst aufgebracht werden müssen \cite{weishaupt_2009_goldenOA}. Darüber hinaus ermöglicht diese Modell im Rahmen der Verschiebung der Erlösquelle von der Bibliothek zum Autor oder der Autorin, die zunehmend die Entwicklung von "falsche" Open Access Journale und Publikationen durch betrügerische Verleger (Predatory Publishers) \cite{Beall_Predatory_2015}, die eine ernsthafte Bedrohung für die Zukunft der Wissenschaftskommunikation im Rahmen der Öffnung der wissenschaftlichen Kommunikation darstellen \cite{Beall_2012}.

Eine weitere Hürde für die Akzeptanz stellen Herausforderungen bei der Sicherung der "Authentizität und Integrität der Texte"  \cite{weishaupt_2009_goldenOA} \cite[:191]{Fehling_2014}, bei der Langzeitarchivierung \cite{hagner_2015_sache_buches} \cite{Martin_2013} und der Einbettung offener Kommunikation in das wissenschaftliche Reputationssystem \cite{weishaupt_2009_goldenOA} \cite{Suber_2002} \cite{Adema_2014_open_access} dar. Darüber hinaus gerät das bisher von den Verlagen organisierte Bewertungssystem zunehmend ins Wanken, wenn Wissenschaftler und Wissenschaftlerinnen anfangen einfach ihre Publikationen frei und offen im Internet zu veröffentlichen und "die Auszeichnung, eine Veröffentlichung in einem so genannten renommierten wissenschaftlichen Journal zu platzieren, nichts mehr gelten soll" \cite{Schirmbacher_oa_2007}.

Die Vertriebsarten, die auf einem Modell beruhen, bei denen der Autor die Kosten für die Publikation trägt, wird auch als "Sozialismus für die Reichen" \cite{cope2014future} bezeichnet. Denn das Modell nährt die Befürchtung, dass zum einen der beim tradierten Publikationssystem kritisierte Matthäus-Effekt wieder in Kraft tritt und nur gut ausgestattete und damit meist bereits renommierte Universitäten, Institutionen oder Lehrstühle in der Lage sind die Ressourcen aufzubrigen um Publikationen zu veröffentlichen. Dieser Effekt verstärkt die Vermutung, dass in sozial Schwächeren Umgebungen auch unter Open Access die Wissensproduktion und -verbreitung weiter behindert wird und die erhoffte Schaffung gleicher Bedingungen im wissenschaftlichen Kommunikationssystem ausbleiben werden.

---- TODO: Tabelle - Auflistung Open Access Kritik in der Literatur Gegenstand / Zeitraum / Referenz Zusammenfassung der Kritik / Anzahl der Zitationen ----

Zusammenfassend bezieht sich die Kritik vornehmlich auf die Gefahren in einem System, in dem die Öffnung erzwungen wird oder die wissenschaftliche Gemeinschaft ohne Einbeziehung in die Ausgestaltung dazu verpflichtet wird. Der Umstand, dass die wissenschaftlichen Akteure selbst entscheiden können, welchen Weg sie wählen, wird dabei bisher nur unzureichend berücksichtigt und kommuniziert. In diesem Zusammenhang werden im Folgenden exemplarisch zwei Bereiche der Kritik werden genauer dargestellt um einen tieferen Einblick in die Themen und Akteure der Debatten um die Kritik an der Öffnung wissenschaftlicher Kommunikation zu ermöglichen: Erstens die Kritik am ökonomischen Modell und zweitens die Kritik an der Einschränkung von Freiheit in Forschung, Lehre sowie der Forschungsdiversität.

\textbf{1. Kritik am ökonomischen Modell}

Ein Kritikpunkt am Open Access Modell bezieht sich vor allem auf das Kostenargument und die ursprüngliche Hoffnung, dass die technologischen Treiber gesteuert und organisiert von der Forschungscommunity selbst, anstatt durch Fachverlage, die durchschnittlichen Kosten für einen publizierten Artikel signifikant senken könnten. So stellte sich die Frage, ob "aus der Sicht des individuellen Nutzenkalküls von Wissenschaftlern, Verlagen und weiteren Einrichtungen wie Bibliotheken als auch aus Sicht gesamtwirtschaftlicher Wohlfahrtsüberlegungen (...) der Markt der Wissenschaftskommunikation nicht effizienter organisiert werden könnte" \cite{Hess_2006}. In einigen Beiträgen wurden schon sehr früh Kostensenkungen von zwischen 50 bis zu 90 Prozent \cite{hilf_2004} \cite[:64]{cite:5} prognostiziert.

Folgende Punkte schürten zunächst die Hoffnung, das System leistungsfähiger zu machen und "von seinen durch den Papierdruck auferlegten Fesseln" zu befreien \cite{hilf_2004}:
\begin{itemize}
\item langer Zeitverzug vom Einreichen eines Manuskriptes bis zum finalen Bereitstellung des Wissens,
\item komplizierter Vertriebsweg vom Verlag über Grossisten zu Bibliotheken,
\item hohe Kosten (ca. 3.000,- Euro für die gesamte Verlagsarbeit je Artikel) mit den daraus folgenden horrenden Zeitschriftenpreisen,
\item und daraus folgend wenige, sowie ungleich in der Welt verteilte Leser (digital divide),
\item unvollständige Information (aus Platzmangel), was Nachnutzungen und das Nachprüfen erschwert und somit auch Fälschungen erleichtert,
\item nur anonymes Referieren vor der Veröffentlichung, was den Missbrauch erleichtert.
\end{itemize}

Verlage die Open Access publizieren stehen dabei allerdings unter einer besonderen und neuen Herausforderung mit diesem Modell nachhaltig zu operieren und passen deshalb ihre Preise von Zeit zu Zeit an. "Auffällig ist jedoch, dass gerade die großen erfolgreichen Projekte wie BioMed Central und Public Library of Science nach ihrer Einführung am Markt deutlichen Gebrauch von Preissteigerungen gemacht haben" \cite{schmidt_2007_goldenen}. Diese Entwicklung hält, wenn auch verlangsamt, weiter an \cite{suchen}. Unter diesem Kostenaspekt wird befürchtet, dass sich subskriptionsbasierte und Open Access-Verlage nicht fundamental unterscheiden \cite{schmidt_2007_goldenen}. Diese Betrachtung basiert auf der Annahme, das die Gesamtpublikationskosten unter den Forderungen von Open Access für Institutionen mit relativ hohe Publikationsoutput höher sein könnten, als die eingesparten Gebühren für die Subskriptionen von Publikationen nach dem aktuellen Modell \cite{mueller-langer_2010}.

Neben der Refinanzierung über Modelle, im Rahmen derer die Autoren vorab die Kosten für die Veröffentlichung übernehmen, werden in der Literatur auch andere Möglichkeiten genannt. Erstens, die Refinanzierung über Werbung. Diese eignet sich allerdings nur für einige Disziplinen \cite{bjork_2004_open} und birgt die Gefahr der Medialisierung von Wissenschaft. Zweitens die Finanzierung über hybride Modelle, bei denen Open Access Text mit Texten nach dem klassischen Erlösmodell gemixt werden und die Autoren gegen zusätzliche Zahlung den Text unter den Bedingungne von Open Access "freikaufen" können \cite{bjork_2012_hybrid}. Oder drittens, Modelle auf dem Wirtschaftsmodell von Versicherungen, bei dem wissenschaftliche Institutionen ex ante für die Publikation aller mit ihr assoziierten Autoren bezahlen \cite[:63]{mueller-langer_2010}.

Auch wenn die ersten Open Access Verlage wie PLOS ONE seit 2010 ohne Verlust opperieren \cite{Jerram_2010} sind die meisten Modelle (vor allem im Vergleich zu den non-open Verlagen) bisher nur mäßig erfolgreich \cite{bjork_2012_hybrid}. Nach den ersten Dekade von Experimenten rund um die Refinanzierung von Open Access bleibt die Kritik an der Nachhaltigkeit von Open Access in Bezug auf das ökonomische Modell bestehen. Somit verbleibt die Frage nach der Refinanzierung weiterhin von zentraler Bedeutung für die weitere Verbreitung von Open Access.

\textbf{2. Gefahr der Einschränkung von Freiheit in Forschung, Lehre sowie der Forschungsdiversität}

Würden Forschungsförderer eine Erstveröffentlichung als Open-Access (golder Weg) verlangen, so wäre zweifelsohne der Schutzbereich der positiven Publikationsfreiheit und damit ein integraler Bestandteil der Wissenschaftsfreiheit berührt \cite[:191]{Fehling_2014}. Eine Veröffentlichungspflicht unter den Bedingungen von Open Access würde allerdings auch klar die negative Publikationsfreiheit im Sinne der Freiheit, Forschungsergebnisse nicht zu publizieren wiedersprechen \cite[:192]{Fehling_2014}. Wobei die Schutzwürdigkeit dieser Freiheit gelegentlich hinterfragt wird \cite[:192]{Fehling_2014}.

Darüber hinaus wird vermutet, dass die umfassende Öffnung der wissenschaftlichen Kommunikation weitreichende Konsequenzen, auf das "wie" und "was geforscht" wird \cite{suchen}, hat. Bisher wird in Deutschland die Vermischung von politischen und forscherischen Interessen bei der öffentlichen Finanzierung von Forschung durch die Unabhängigkeit der Deutschen Forschungsgemeinschaft (DFG) sichergestellt. Ziel dieser Trennung ist es, dass Wissenschaftler unabhängig von unmittelbaren politischen Interessen und Verwertungskriterien forschen können. Als privatrechtlicher Verein sieht sich die DFG als "wissenschaftliche Selbstverwaltung" und steht für "Autonomie gegenüber der Politik"  \cite{DFG_2011}. Dennoch kann, so die Befürchtung einiger Autoren \cite{suchen}, nicht sichergestellt werden, dass eine die umfassende Einbeziehung und Information der Gesamtöffentlichkeit nicht ohne Einfluss auf die Mittelvergabe haben könnte \cite{weingart_2005_wissenschaft}. Ein Großteil der Wissenschaft wird durch Steuergelder finanziert, was trotz der Autonomie nicht gänzlich ausschließen kann, dass politische Interessen, die Steuerungsmechanismen von Wissenschaft und Forschungsförderung trotz unabhängiger Forschungsförderungsstrukturen beeinflussen können.

Der Mediziner und Wissenschaftshistoriker Michael Hagner formuliert seine Befürchtung in einem Beitrag für die Frankfurter Allgemeine Zeitung wie folgt: "Open Access als Traum der Verwaltungen". Er und andere beschreiben die Gefahr, dass die Wissenschaft bei der Verpflichtung zur elektronischen Veröffentlichung von Forschungsergebnissen für Wissenschaftler durch Universitäten auf eine vollends verwaltete Forschung hinaus laufen würde \cite{hagner_faz_2009}. Andere antizipieren eine weitere Gefährdung von Wissenschaft und Forschung, weil Grundlagenforschung, sowie andere komplexe oder explorative Forschungsbereiche in Zukunft weniger Berücksichtigung finden würden, wenn die Öffnung der wissenschaftlichen Forschungsprozesse unter rein kommerziellen Aspekten weiter vorangetrieben wird \cite{suchen} und das obwohl die  zweckfreien Grundlagenforschung heute schon bezweifelt wird \cite{suchen}.

Obwohl die Normen von Offenheit schon immer eine entscheidende Rolle bei der Aufrechterhaltung der systemischen Wirksamkeit der modernen wissenschaftlichen Forschung gespielt haben, sind sie sehr anfällig für die Legitimation des Rückzugs der staatlichen Schirmherrschaft und den öffentlichen Schutz zur Sicherung der Rahmenbedingungen für die eigentliche Aufgabe von Wissenschaft \cite{david1998_common}.

Um diese Aspekte und Prognosen über die Implikationen von Open Access zu evaluieren, wird in diesem Teil der Arbeit auf Grundlage von Textbeispielen die Kritik an der Öffnung von Wissenschaft und der (forschungs-)politischen, rechtlichen und freiheitlichen Entwicklungen dargestellt.

Als ein konkretes Beispiel für die "Kontroversen um die Zukunft des Buches, um Autorenschaft und geistiges Eigentum, die Rolle von Verlagen und die für Leser kostenlose Bereitstellung aller wissenschaftlichen Literatur" \cite{hagner_2015_sache_buches}, die Einschränkung der Wissenschafts- und Publikationsfreiheit soll der "Heidelberger Appell" für Publikationsfreiheit und die Wahrung von Urheberrechten dienen. Am 22. März 2009 wurde auf der Webseite der „Frankfurter Allgemeinen Zeitung“ der Artikel "Geistiges Eigentum: Autor darf Freiheit über sein Werk nicht verlieren" \cite{faz_heidelberger_apell_2009} veröffentlicht. Vorangegangen war eine öffentlich ausgetragene Diskussion zwischen dem Literaturwissenschaftler Prof. Dr. Roland Reuß und weiteren Wissenschaftlern in einem Spezial der Onlineausgabe der Frankfurter Allgemeinen Zeitung: "Die Debatte über Open Access". Im Anhang zu diesem Artikel fand sich ein öffentlicher Aufruf, auch "Heidelberger Appell" genannt.

Der Appell richtete sich vor allem an "die Bundesregierung und die Regierungen der Länder, das bestehende Urheberrecht, die Publikationsfreiheit und die Freiheit von Forschung und Lehre entschlossen und mit allen zu Gebote stehenden Mitteln zu verteidigen" \cite{ITK_2009}. Die Autoren forderten, unter anderem in Bezug auf die Google Buchsuche (Google Books), die Politik, Öffentlichkeit und Kreative auf, sich für die "Wahrung der Urheberrechte" und "gegen eine angebliche „Enteignung“ der Autoren durch das Vorgehen von Google einerseits sowie durch das Publikationsmodell Open Access andererseits" \cite{WD_bundestag_2009} zu engagieren.

Die Autoren des Appells unterscheiden zwei Ebenen: \textit{International} kritisieren sie "die nach deutschem Recht illegale Veröffentlichung urheberrechtlich geschützter Werke geistigen Eigentums auf Plattformen wie GoogleBooks und YouTube", sowie die Entwendung dieser "ohne strafrechtliche Konsequenzen". Im \textit{nationalen Rahmen}, so prangern die Autoren weiter an, werden diese "Eingriffe in die Presse- und Publikationsfreiheit, deren Folgen grundgesetzwidrig wären" durch die "Allianz der deutschen Wissenschaftsorganisationen (Mitglieder: Wissenschaftsrat, Deutsche Forschungsgemeinschaft, Leibniz-Gesellschaft, Max Planck-Institute u.a.)" sogar unterstützt \cite{ITK_2009}.

Die Kritik der Autoren des Heidelberger Apells an Open Access bezieht sich, laut einer Untersuchung des Wissenschaftlichen Diensts des Bundestags im wesentlichen auf die folgenden Aspekte \cite{WD_bundestag_2009}:
\begin{enumerate}
\item Erzwungene Vertriebswege
"Eine Forschung, der man diktieren könnte, wo ihre Ergebnisse publiziert werden sollen, sei nicht mehr frei." Die Verpflichtung auf "bestimmte Publikationsform (...) dient nicht der Verbesserung der wissenschaftlichen Information" \cite{ITK_2009}.
\item Subventionierung von Vertriebswegen oder der Gefährdung von Fachzeitschriftenverlagen \cite{ITK_2009}
\end{enumerate}

Der Appell "hat eine außergewöhnlich heftige Diskussion über die urheberrechtliche Problematik im Hinblick auf die aktuellen Entwicklungen im Internet ausgelöst. Viele Parlamentarier und Politiker sind für das Thema sensibilisiert" worden \cite{WD_bundestag_2009}. In Bezug auf Open Access widerlegt der Wissenschaftliche Dienst die Befürchtungen der Autoren des Heidelberger Apells. Dem Kritikpunkt der "Erzwungenen Vertriebswege" widerspricht der Wissenschaftliche Dienst mit einem Verweis auf Gudrun Gersmann, weil "auch (Anmerkung: unter Open Access) eine Veröffentlichung bei einem Verlag mit einfachem Nutzungsrecht weiterhin möglich sei". In Bezug auf das Modell und das Abhängigkeitsverhältnis halten die wissenschaftlichen Autoren des Bundestags Reuß entgegen, dass es im bisherigen System "zwischen Autor und Fachzeitschriftverlag oft ein einseitiges Abhängigkeitsverhältnis zu Lasten des Autors gibt" und Wissenschaftler "oftmals alle Rechte an ihren Beiträgen abtreten" \cite{WD_bundestag_2009} müssen. "Der Befürchtung im Heidelberger Appell, das Publikationsmodell Open Access gefährde Fachzeitschriftenverlage wird entgegengehalten, dass die digitale Plattform auf lange Sicht auch ein Ausweg aus der Zeitschriftenkrise sein könnte" \cite{WD_bundestag_2009}. Abschließend konstatiert der Wissenschaftliche Dienst des Bundestags, dass die "Kritik an Open Access kaum nachvollzogen werden" kann und "die hier gemachten Vorwürfe" "eher auf die traditionellen Vertriebswege zu treffen, als auf das neue Publikationsmodell" \cite{WD_bundestag_2009}.

Obwohl der Heidelberger Apell unter dem Verdacht stand, eine "eine an Informationsdefiziten und Fehlinterpretationen reiche Kampagne" \cite{Schmidt_2009} darzustellen, scheint ein Teil der Kritik  mindestens zwei Punkten berechtigt zu sein. Erstens, dass man seitens der Forschungsförderer nicht besonders bemüht war \cite{suchen}, sich "ein genaues Bild von den Nebenwirkungen (Anmerkung: von Open Access)" \cite{Reuss_2009} zu verschaffen und zweitens wurde die Sicherung von Freiheit von Forschung und Lehre sowie die Anpassung der Steuerungsmechanismen bei den Bestrebungen zur Öffnung von Wissenschaft und Forschung nur ungenügend berücksichtigt \cite{hagner_2015_sache_buches}.

Die Kritik am urheberrechtlichem Aspekt der Google Buchsuche soll in dieser Arbeit nicht berücksichtigt werden, da es sich dabei zwar um einen Aspekt der Digitalisierung von Büchern, nicht aber um die Öffnung von wissenschaftlicher Kommunikation nach den Kriterien der in dieser Arbeit gewählten Deklarationen handelt, sowie die Google Buchsuche als Dienst keinen Bezug zur Open Access-Bewegung aufweist \cite{hagner_2015_sache_buches}. Dennoch sei auf den Umstand verwiesen, dass die Fixierung auf das Urheberrecht einem idealisierten Verständnis des wissenschaftlichen Verlagswesens entspringt und von den wirklichen Gefahren für die Buchkultur ablenkt \cite{Hirschi_2015_buch_oa}.

\subsection{Offener Zugriff auf wissenschaftliche Kommunikation: Open Science}

Trotz aufsehenerregender öffentlicher Debatten und der Kritik an der Öffnung wissenschaftlicher Kommunikation, führten die Digitalisierung des wissenschaftlichen Alltags und die zunehmende digitale Vernetzung der wissenschaftlichen Gemeinschaft zu Weiterentwicklungen der Idee von der Öffnung wissenschaftlicher Kommunikation und zur Inanspruchnahme der schnellen und digitalen Informationsversorgung durch Wissenschaftler und Wissenschaftlerinnen \cite{winkler_2011_anforderungen}. Mit dem Anstieg der Übertragungsgeschwindigkeiten im Internet, neuen Speichertechniken und dem Aufkommen neuer kollaborativer Arbeitsweisen und Handlungsoptionen in der Wissenschaft, entwickelten sich neue Gegebenheiten für den Austausch wissenschaftlicher Informationen. Die flächendeckende Verfügbarkeit von Breitbandzugängen an Universitäten Anfang der 2000er Jahre, und später auch in Privathaushalten, machte es nicht nur Möglich die formelle wissenschaftliche  Kommunikation, sondern auch die informelle Kommunikation und wissenschaftliche Daten zu teilen. Das führte zu einer theoretischen Ausweitung des möglichen Umfangs digital verfügbarer wissenschaftlicher Kommunikation, die Möglichkeit diese über das Internet anderen Forscherinnen und Forschern zur Verfügung zu stellen und die Forderung nach der weiteren Öffnung des wissenschaftlichen Kommunikationssystems.

Unter dem Begriff Open Science oder Offene Wissenschaft verbirgt sich demnach die Forderung, die technischen Entwicklungen zu nutzen, um wissenschaftliche Erkenntnisse aller Art im Rahmen des wissenschaftlichen Erkenntnisprozesses schnellstmöglich offen zu verbreiten und für andere nutzbar zu machen. \cite{stafford_2010_science}. Open Science beschränkt sich dabei nicht nur auf den Zugang zur wissenschaftlichen Publikation am Ende des wissenschaftlichen Erkenntnisprozesses (Open Access) und auf die daraus resultierenden Veränderungen wissenschaftlicher Kommunikationsprozessen im Rahmen von Publikationen, sondern auf sämtliche Daten und Informationen die während des Prozesses anfallen. Aus technischer Sicht ist damit jeder Aspekt der wissenschaftlichen Arbeit gemeint, der digital auf einem Desktop-Computer stattfindet und somit auch öffentlich über das Web potenziell verfügbar gemacht werden kann \cite{mietchen2012wissenschaft}.

Offene Wissenschaft kann als Sammelbegriff für eine Vielzahl an Aktivitäten und Mechanismen der kumulativen Wissensproduktion verstanden werden \cite{Mukherjee_2009}. Sie alle tragen dazu bei, dass die sämtliche Inhalte der Kommunikation während und nach der Wissensproduktion durch andere innerhalb und außerhalb der wissenschaftlichen Gemeinschaft weiterverwendet werden können. Open Science resultiert auch aus der zunehmenden Anwendung von Diensten und Applikationen des sozialen Webs auf die Arbeit von Wissenschaftlern und umfasst die "Zugänglichkeit des gesamten Forschungsprozesses, vom Sammeln der Daten an, über die Begutachtung hin zur fertigen Publikation" \cite{brembs2015open}.

Open Science basiert zudem auf der in der ureigenen wissenschaftlichen Anforderung, dass die Ausübung von wissenschaftlichen Tätigkeiten auf eine Art und Weise erfolgt, die es anderen ermöglicht zu den Forschungsbemühungen beizutragen, zusammenzuarbeiten und auf alle Daten, Ergebnisse und Protokolle in allen Phasen des Forschungsprozesses frei zuzugreifen \cite{RIN_2010_open_research}. Der gesamte Forschungsprozess sollte demnach so transparent und so zugänglich wie möglich gestaltet werden \cite{Scheliga_2014}.

Anhand der folgenden Einteilung werden die Charakteristika des wissenschaftlichen Erkenntnisprozesses erläutert und dargestellt, um zu argumentieren, was die Öffnung von Wissenschaft im Sinne von Open Science beinhaltet. Zur Verdeutlichung des Prozesses der Wissensschaffung wird in der vorliegenden Arbeit eine Einteilung extrapoliert in vier Phasen vorgenommen:
\begin{enumerate}
\item Fragestellung und Planung
\item Ausführung
\item Verarbeitung und Analyse
\item Auswertungsverfahren
\item Verwendung und Kommunikation der Ergebnisse
\end{enumerate}

---- TODO: Grafik aus http://de.slideshare.net/petermurrayrust/osbrazil Slide 14 bauen und an Phasen anpassen ----

Die Forderung nach Öffnung des gesamten Prozesses der Wissensschaffung begründet sich dabei nicht (nur) durch die technologische Entwicklung und die Herausforderungen im bestehenden wissenschaftlichen Kommunikationssystem, sondern basiert auf den folgenden weiterführenden Annahmen:
\begin{enumerate}
\item Der offene Zugang zum gesamten Wissenschaftsprozess erhöht die Möglichkeiten der Validierung und Reproduzierbarkeit der gesamten Forschung(-skette) \cite{Aleksic_2014} \cite{Krumholz_2014} \cite{hey_2015_open} und die Entwicklung neuer Qualitätskriterien. (enhanced Validation/Reputation-Argument)
\item Im Rahmen des Teilens (z.B. von Rohdaten) erhöht sich die Effizienz und Verwendbarkeit durch in der Forschung und Wissenschaft entstandenen Informationen \cite{Fecher_2015}. (Shared-Science-Argument)
\item Im klassischen wissenschaftlichen Kommunikationssystem gibt es keine Anreize negative, widerlegende oder nicht-erfolgreiche wissenschaftliche Ergebnisse zu veröffentlichen. Eine vollumfängliche Öffnung des wissenschaftlichen Erkenntnisprozesses könnte dazu beitragen, dass Wissenschaft ihrem Anspruch an Falsifizierbarkeit gerecht wird. (negative-science/falsifiability-Argument)
\item In Ergänzung zu den bestehenden Mechanismen unter denen Vertrauen unter Wissenschaftlern und von der Öffentlichkeit in Wissenschaft besteht \cite{weingart_2005_wissenschaft}, bietet die vollständige Veröffentlichung der Informationen, die von offenen Wissenschaft ausgeht, als Ersatz für oder Ergänzung zu älteren Vertrauenssystem betrachtet werden, zum Nutzen der wissenschaftlichen Gemeinschaft und der Gesamtgesellschaft. \cite{grand_2012_open}.  (Trust-Technology-Argument)
\end{enumerate}

Beispielhaft für diese Annahme kann hier genannt werden, dass die Hälfte der klinischen Forschungsstudien nie veröffentlicht werden, diese selektive Veröffentlichung den medizinischen Fortschritt und mögliche Erkenntnisse verzerrt sowie den Fluss von Informationen hemmt, die wichtig sind, um die Entscheidungsfindung durch die Patienten und ihre Ärzte zu unterstützen \cite{Ross_2013}. Werden Ergebnisse nicht veröffentlicht beeinträchtigt das auch andere Forschung, da auch negative Ergebnisse einen Beitrag zum Falsifikationsprozess liefern \cite{nosek_2012_scientific}. Die Replikation ein wesentlicher Teil der wissenschaftlichen Arbeit, Qualitätskontrolle und Methode.

Die Kultur der Forschung braucht diesen Grad an Offenheit in der wissenschaftlichen Kommunikation, sowie den Zugang zu Daten anderer Wissenschaftler und Wissenschaftlerinnen um im wissenschaftlichen Erkenntnisprozess erfolgreich zu sein \cite{Fecher_2015} \cite{Krumholz_2014} \cite{patlak_2010_open}.

Bestrebung der Öffnung des wissenschaftlichen Arbeitsprozesses sollte es demnach sein, erfolgreiche Wege zu finden, um Daten und benutzten Programmcode unter Berücksichtigung der Interessen aller Beteiligten möglichst umfangreich im besten Interesse der Gesellschaft zu teilen \cite{naeder_2010_open} \cite{Ross_2013} \cite{hey_2015_open}. Offene Wissenschaft hat das Potenzial durch Transparenz und die Möglichkeit des Eröffnung des Zugriffs auf wissenschaftliche Informationen und Daten einen notwendigen Beitrag zu dem Vertrauen der Menschen in Wissenschaft und das Vertrauen von Wissenschaft in Menschen zu leisten \cite{grand_2012_open}.

---- TODO: Tabelle - Auflistung Open Science Definitionsversuche in der Literatur Gegenstand / Zeitraum / Referenz Zusammenfassung der Definitionen + Grafik aus https://www.fosteropenscience.eu/foster-taxonomy/open-science-definition -> View the taxonomy tree ----

\subsubsection{Open Science: Modelle, Formate und Kanäle}

Open Science vereint als Sammelbegriff viele Modelle, Formate und Kanäle. Ein maßgeblicher Katalysator für Open Science war die technologische Entwicklung und die neue Möglichkeiten für Wissenschaft aus methodologischer Sicht und für die Dissemination von Forschungsinformationen \cite{garcia_2010_open}. Ermöglichte das Internet zunächst die einfache Darstellung von Inhalten in einem globalen Netzwerk, führte die Entwicklung des sozialen Webs zu der Möglichkeit eines umfassenden Austauschs in nahezu Echtzeit und die offenen Kommunikation in Wissenschaft und Forschung. Weder die einfache Darstellung von Inhalten in einem globalen Netzwerk, noch die Möglichkeiten zum direkten und offenen Austausch zwischen Wissenschaftlern waren im analogen Zeitalter rein technisch nicht möglich und die Frage eines freien und umfassenden Zugriffs auf wissenschaftliche Information stellte sich erst gar nicht \cite{Schirmbacher_oa_2007}.

---- TODO: Grafik bauen aus: e-Science -> Open Access -> Science 2.0 -> Open Science ----

Open Science umfasst alle Charakteristika des wissenschaftlichen Erkenntnisprozesses. Exemplarisch werden hier Möglichkeiten der Öffnung des Prozesses dargestellt:
\begin{itemize}
\item Veröffentlichung in Datenrepositorien - Diese Repositorien ermöglichen die Ablage und die Verbreitung von wissenschaftlichen Daten, die im Rahmen des wissenschaftlichen Erkenntnisprozesses anfallen. Dabei können hier nicht nur die Daten abgelegt werde, die im Rahmen der endgültigen Publikation genutzt wurden, sondern auch die, die im Vorfeld erhoben wurden oder negative Ergebnisse enthalten. Grundsätzlich können diese Repositorien in begutachtete und nicht-begutachtete Repositorien unterteilt werden.
\item Offene Erstellung von Forschungsanträgen - Forschungsförderung ermöglicht die Einwerbung und Allokation von Ressourcen für ein wissenschaftliches Vorhaben. Die öffentliche Erstellung eines solchen Antrags bietet zwar die Gefahr der Kopie durch andere, ermöglicht aber auch die Einbeziehung externen Wissens und somit die Möglichkeit eines besseren Antrags. Eventuelle Herausforderungen können so früh erkannt werden und die Möglichkeit der positiven Begutachtung steigt. Zudem schafft diese Art der Beantragung mehr Transparenz bei der Mittelvergabe und (fach-)öffentliches Interesse an dem Projekt.
\item Arbeit mit offenen Laborbüchern - Stellen eine Möglichkeit für die offene Ablage von Informationen und die Dokumentation rund um die wissenschaftliche Arbeit dar. Ziel ist es, ein möglichst umfassendes Bild von der Materie und den eingesetzten Methoden und Applikationen frühestmöglich und so umfangreich wie ausführbar zu dokumentieren. Das verbessert die Voraussetzungen für die Replizierbarkeit zu und ermöglicht gegebenenfalls Fehler früh zu erkennen.
\item Erweitertes offenes Publizieren - Neben dem offenen Publizieren von fertigen Texten (Open Access), ist es grundsätzlich realisierbar die digitalen Publikationen auch mit den Daten anzureichern. Demnach hätten Leserinnen und Leser von Literatur nicht nur die Möglichkeit eines Zugangs zum wissenschaftlichen Text, sondern könnten beim Lesen auch auf die Daten auf der die Ergebnisse beruhen zugreifen.
\end{itemize}

\subsubsection{Kritik an Open Science}

Während viele Wissenschaftler und Wissenschaftlerinnen Offenheit in der Forschung als wertvoll erachten, sind nur wenige tatsächlich bereit, die zusätzliche Zeit und Mühe dafür zu investieren und potenzielle nicht abgrenzbare Risiken einzugehen, Forschung offen und uneingeschränkt zugänglich zu machen \cite{Scheliga_2014} \cite{Tenopir_2011} \cite{Procter_2010}. In vorhergehenden Studien waren es vor allem jüngere Forscherinnen und Forscher, die ein spezielles Interesse hatten, ihre Daten nicht zu ohne Einschränkungen veröffentlichen \cite{Tenopir_2011}. Forscherinnen und Forscher, die offene Wissenschaft praktizieren wollen, werden mit einer Reihe von Hindernissen konfrontiert \cite{Scheliga_2014}:
\begin{enumerate}
\item individuelle Hindernisse: Angst vor Trittbrettfahrern, gefürchteter Mehraufwand an Zeit und Mühe, Herausforderungen bei der Nutzung der digitaler Dienste, fehlender Anstoß, Angst negative Ergebnisse zu veröffentlichen, Herausforderung den Datenschutz sicherzustellen, Abneigung den Code zu teilen
\item systemische Hindernisse: Evaluationskriterien behindern Offenheit, kulturelle und institutionelle Einschränkungen, ineffektive (politische) Richtlinien, Mangel an Standards für das Teilen von Forschungsmaterialien, Mangel an rechtlicher Klarheit, finanzielle Aspekte der Offenheit
\end{enumerate}

Betrachtet man wie Scheliga und Friesike das Phänomen Open Science anhand des Konzepts des sozialen Dilemmas, wird deutlich, dass das was im kollektives Interesse der wissenschaftlichen Gemeinschaft ist, nicht unbedingt im Interesse des einzelnen Wissenschaftlers steht. In der wissenschaftlichen Gemeinschaft besteht dabei ein Spannungsverhältnis zwischen dem tun was das Beste für die Gemeinschaft ist, gegenüber dem, was am besten für den einzelnen Wissenschaftler ist \cite{Ekins_2014} \cite{patlak_2010_open} \cite{wein_2010_erwerbung}. "Wenn alle Wissenschaftler ihr Wissen nur in den Situationen teilen, in denen sie erwarten, dass sie selbst davon profitieren, ist der gemeinsame Wissenspool fragmentiert und alle Wissenschaftler stehen schlechter dar" \cite{Scheliga_2014}.

Kritisch wird auch angemerkt, dass Wissenschaftler, die vorläufige Ergebnisse veröffentlichen, ein unkalkulierbares Risiko eingehen, dass andere die Arbeit kopieren und die Anerkennung dafür erlangen, oder die Ergebnisse sogar patentieren lassen \cite{Peters_2014}. In einigen Disziplinen wäre eine an Echtzeit angelehnte Veröffentlichung der Laborbücher oder Ergebnisse gar kontraproduktiv für den Erkenntnisprozess, in anderen Disziplinen technisch (noch) nicht möglich.

Neben diesen ganz pragmatischen Aspekten, gibt es auch ein institutionelles Dilemma der Balance zwischen dem prinzipiell offenen Zugang zu Wissen und der Einschränkung des Zugangs zu Wissenschaft. So wird die Trennung von nicht-Wissenschaft und Wissenschaft als wichtig erachtet um die systemische Distanz zu wahren, die Spezialisten vorab von Laien trennt und die eine Grenze darstellt, "die nicht beliebig überschreitbar ist" \cite{weingart_2005_wissenschaft}. Würde diese Grenze aufgehoben werden müsste die Wissenschaft das mit dem "Preis des Verlusts ihrer besonderen Leistungsfähigkeit" und mit einer Medialisierung der Wissenschaft bezahlen \cite{weingart_2005_wissenschaft}.

---- TODO: Auflistung Open Science Kritik in der Literatur Gegenstand / Zeitraum / Referenz Zusammenfassung der Definition + Grundlage selbe Kritik wie bei Open Access ----

\section{Zusammenfassung und Ableitungen für die empirische Untersuchung}

Viele der unterschiedlichen Erklärungsansätze für die Forderung nach einem Wandel der wissenschaftlichen Kommunikation hin zur Öffnung der Wissenschaft basieren auf Annahmen, bei denen ein direkter Zusammenhang von technischen Entwicklungen auf (wissenschafts-)politische und kulturelle Bewegungen geschlossen wird. Diese Perspektive ist in ihren Wegen und Kanälen sehr fragmentiert und beschränkt sich in seiner Klarheit bisher nur auf das gemeinsame Ziel den Zugang zu Ergebnissen von Wissenschaft offener zu gestalten und weniger auf die Öffnung des gesamten Prozesses sowie die Konsequenzen auf das Wissenschaftssystem.

Die theoretische Auseinandersetzung mit der Geschlossenheit des wissenschaftlichen Diskurses auf der einen und den Treibern und Bremsern im realen wissenschaftlichen Prozess auf der anderen Seite werden in der Literatur bisher nur ungenügend berücksichtigt. Insbesondere wird die Verbindung zwischen wissenschaftlicher Reputation, die Motivation das etablierte System zu unterstützen und die Geschlossenheit des Wissensproduktionsprozesses nur selten erörtert und die Debatten über die Veränderungen des wissenschaftlichen Publikationswesens werden von beiden Seiten mit teilweise "heftiger Polemik" \cite[:12]{naeder_2010_open} geführt. Als weiteres Manko kann angeführt werden, "die Deliberation und die Verbreitung von Wissen ein stabiles Set von Infrastrukturen braucht" \cite{kelty_2004}, nach denen man heute noch vergeblich sucht. Das Potenzial bei der Verwendung von digitalen Technologien um Wissenschaft offen zu teilen, ist nicht annährend ausgeschöpft und es "besteht eine erhebliche Diskrepanz zwischen der Idee der offenen Wissenschaft und wissenschaftliche Realität" \cite{Scheliga_2014}. Demgegenüber ist die "(geistes-)wissenschaftliche Alltagspraxis längst von digitalen Recherche- und Kommunikationsformen durchsetzt" \cite{hagner_2015_sache_buches}.

Openness kann als "schwimmender Signifikant (...) ohne eindeutige Definition, adaptierbar von unterschiedlichen politischen Ideologien" verstanden werden \cite{Adema_2014_open_access}. Der Begriff Open Access wird in der neoliberalen Rhetorik als effizientes Wettbewerbsmodell, verbunden mit den Ideen von Transparenz und Effizienz von Unternehmen und Regierung, eingesetzt \cite{tkacz_2012_open}. Darüber hinaus muss die Öffnung von wissenschaftlicher Kommunikation auch im Rahmen des Versuchs betrachtet werden, einen Martkmodus als dominante Governanceform der Gesellschaft auch in der Wissenschaft zu verankern \cite[:152]{troy_2012_wissen}. Über diesen Ansatz wird mittels Openness der wissenschaftlichen Prozess outputorientierter und seine Ergebnisse effektiver zu Gunsten des Marktes gestaltet, überwacht und gesteuert \cite{adema_2010_oaoverview}. Dabei stehen diese neoliberalen Ansätze den Ideealen der Öffnung des gesamten wissenschaftlichen Prozesses gegenüber, "denn die Position funktioniert nur dann ökonomisch effizient, wenn innovatives technisches Wissen nicht nur patentrechtlich sondern auch marktmäßig gehandelt wird" \cite[:179]{troy_2012_wissen}.

Diese Entwicklung bedroht dabei auch das System der Universität als Produzent, Archivar und bei der Dissemination von Wissen. Demnach muss die Veränderung hin zur Öffnung von Wissenschaft und Forschung als Möglichkeit genutzt werden, dass die Universität selbst wieder zu dem (primären) Ort der Wissensproduktion, -speicherung und -vermittlung wird, der sie mal gewesen ist \cite{kittler_2004}.

Um diese Veränderungen voranzutreiben, werden in der Literatur zwei Herangehensweisen für die Etablierung von Offenheit in Wissenschaft und Forschung unterschieden \cite{schulze_2013_open}:
\begin{enumerate}
\item "Top-down durch Förderstrategien, Vorgaben und Empfehlungen": Hierbei können durch die Bereitstellung zusätzlicher Mittel im Rahmen der Forschungsförderung konkrete Anreize für die offene Veröffentlichung und die Publikation von Forschungsergebnissen geschaffen werden \cite{suchen}. Eine weitere Möglichkeit der "Top-Down"-Etablierung von Offenheit in Wissenschaft und Forschung stellen Empfehlungen dar, bei denen Institutionen, Organisationen oder Gruppen nicht bindende Empfehlungen aussprechen, anhand derer Wissenschaftler und Wissenschaftlerinnen überzeugt werden sollen, ihre wissenschaftlichen Ergebnisse offen zu veröffentlichen \cite{suchen}. Sind weder Anreize, noch Empfehlungen als Top-Down-Ansatz erfolgreich, können bindende Vorgabe etabliert werden um eine Verhaltensänderungen der Wissenschaftler und Wissenschaftlerinnen zu erzwingen \cite{suchen}.
\item "Bottom-up durch Graswurzelprojekte und den Einsatz von Evangelisten":
Im Gegensatz zur Strategie von "oben" gibt es auch Bestrebungen, die von einzelnen Wissenschaftlern, Wissenschaftlerinnen oder Gruppen initiiert sind. Sie sind überwiegend informell und zielen die Verbreitung von Verhaltensänderungen oder die Etablierung von Richtlinien ab \cite{suchen}. Bottum-up-Projekte kommen aus dem wissenschaftlichen Alltag und erfahren meist keine politische oder monetäre Incentivierung für die Öffnung von Wissenschaft und Forschung. Der Einsatz von Evangelisten basiert auf der Idee einer konkreten Stelle oder Position um eine Änderung zu begleiten \cite{suchen} oder einen Multiplikator innerhalb und außerhalb von Institutionen oder Organisationen zu etablieren, der das gewünschte Ziel pro aktiv kommuniziert und verbreitet \cite{suchen}. Evangelisten können helfen die Befindlichkeiten und Vorbehalte auszutarieren und die teils diffusen, teils realen Ängste bezüglich der Entwicklung von Offenheit und Transparenz der Wissenschaft innerhalb und außerhalb der wissenschaftlichen Gemeinschaft zu beseitigen \cite{schulze_2013_open}.
\end{enumerate}

Ergänzend dazu sehen die Rechtwissenschaftler Götting und Lauber-Rönsberg vier konkrete, rechtliche und faktische Maßnahmen zur Förderung der Öffnung wissenschaftlicher Kommunikation \cite{Goetting_2015}:
\begin{enumerate}
\item Verpflichtungen durch das Hochschulrecht - z.B. eine rechtliche Verpflichtung steuerfinanzierte wissenschaftliche Werke unter einer offenen Lizenz veröffentlichen
\item Maßnahmen der Hochschulen - z.B. durch institutionelle Selbstverpflichtungen oder finanzielle und andere faktische Anreizsysteme
\item Maßnahmen der öffentlichen Forschungsförderung - z.B. Verpflichtung im Rahmen der Drittmittelfinanzierung von Forschungsvorhaben oder direkte Förderungsinstrumente für den Aufbau oder die Refinanzierung von offenen Publikation
\item Urheberrechtliche Maßnahmen - z.B.  Vorhaben steuerfinanzierte wissenschaftliche Werke vom urheberrechtlichen Schutz auszunehmen oder Schrankenregelungen bzw. Zwangslizenzen für öffentlich-finanzierte Werke einzuführen
\end{enumerate}

Im Folgenden werden die Katalysatoren und Hindernisse für die Etablierung der Öffnung von wissenschaftlicher Kommunikation und die Indikatoren für die Reputationsverteilung im wissenschaftlichen System gegenübergestellt. Diese Ausarbeitung zielt auf die Beantwortung der Forschungsfragen ab und stellt eine Grundlage für die darauffolgende Befragung der wissenschaftlichen Akteure im Publikations- und Kommunikationssystem dar.

\subsection{Katalysatoren für die Öffnung der wissenschaftlichen Kommunikation}

In den analysierten wissenschaftlichen Beiträgen zu Open Access und Open Science wurden umfassend positiven Auswirkungen der Forderungen nach Offenheit im wissenschaftlichen Kommunikationssystem, aber auch Vermutungen über die negativen Effekte der Öffnung auf das wissenschaftliche Kommunikationssystem dargestellt. Grundlage für die Darstellung der Vorteile war die vorhergehend durchgeführte Erarbeitung der Herausforderungen und Unzulänglichkeiten im bestehenden wissenschaftlichen Kommunikationssystem \cite{cite:17}.

Grundsätzlich steht und fällt der Erfolg bei der Etablierung von Verhaltensänderungen damit, ob sich die jeweiligen Zielgruppe ein unmittelbarer Mehrwert und Nutzen erschließen wird \cite{schulze_2013_open}. Bisher scheint dieser eher gering, denn rechtlich steht es bereits nach der heutigen Rechtslage Wissenschaftlerinnen und Wissenschaftlern frei, "sich für eine Erstveröffentlichung ihrer Werke im Wege des Open Access zu entscheiden" \cite[:146]{Goetting_2015}, auch wenn "Möglichkeiten und Grenzen von Open-Access-Publikationsverpflichtungen wesentlich durch die urheberrechtlichen Rahmenbedingungen beeinflusst werden" \cite[:211]{Fehling_2014}.

Für die weitere Gruppierung der Argumente für die Öffnung von Wissen wurde die folgende Kategorisierung vorgenommen. Auf sie folgt die Beschreibung der grundlegende Katalysatoren und Argumente für die Öffnung des wissenschaftlichen Kommunikationssystems:
\begin{enumerate}
\item \textbf{Transition} - Die Nutzung der neuen Möglichkeiten für eine offene Wissensverbreitung neben den konventionellen Wegen der nicht-elektronischen Publikationen \cite{hall_2008_digitize} \cite{berliner_erklaerung_2003}. Voraussetzung für die Aufbereitung des Wissens als strukturierte Daten zur Wissensweiterverwendung und -verarbeitung über alle Kanäle.
\item \textbf{Speed and Circulation} - Offene Publikationsverfahren bieten die Chance wissenschaftliche Inhalte schneller und umfassender der wissenschaftlichen Community zur Verfügung zu stellen \cite{muller_2010_open}\cite{RIN_2010_open_research} \cite{hall_2008_digitize} \cite{EuropeanCommission_sciencepub_2006}. Wenn das Wissen schneller zur Verfügung steht, kann es auch schneller zirkulieren und effizienter genutzt werden \cite{Woelfle_2011}. In den tradierten Verfahren wird die Wissensverbreitung künstlich durch Embargos und ineffiziente Validierungs- und Qualitätssicherungssysteme zurückgehalten. Die Digitalisierung und Verbreitung über elektronische Kanäle stellt einen Vorteil für die Wissensverbreitung und -verwertung dar. Eine offene Veröffentlichung erreicht potentiell eine größere Leserschaft als bei Subskriptionsmodellen \cite{cope2014future}.
\item \textbf{Higher Impact and Citation} - Die uneingeschränkte und globale Verfügbarkeit der offenen wissenschaftlichen Informationen führt zu einem wesentlich höheren Verbreitungsgrad und Einfluss von Wissenschaft \cite{davis_2011_open} \cite{muller_2010_open} \cite{Baggs_2006} \cite{cite:5} \cite{Kurtz2005_oa_citation}. Der Verbreitungsgrad kann einen positiven Einfluss auf die Zitierhäufigkeit haben \cite{muller_2010_open} \cite{EuropeanCommission_sciencepub_2006} \cite{Hajjem_2005}. Die Zitationsrate wissenschaftlicher Publikationen, die nach den Kriterien von Offenheit veröffentlicht werden ist damit potenziell höher \cite{cite:21a}. Diese Kausalität wird "access-citation effect"\cite{davis_2011_open} genannt und ist durch bedeutsame Untersuchungen bestätigt worden \cite{Lawrence_2001} \cite{Jeffrey_2008} \cite{Hajjem_2005} \cite{Eysenbach_2006} \cite{Antelman_2004}. Dennoch gibt Gründe diesen Effekt genau zu hinterfragen und im Detail mögliche Abschwächungseffekte zu berücksichtigen \cite{davis_2011_open} \cite{davis_2008_open}.
\item \textbf{Tax-Payer} - Die Kosten des traditionellen Publikationsverfahrens werden im Wesentlichen durch die öffentliche Hand getragen \cite{muller_2010_open}. Dem Steuerzahler ist die konventionelle wissenschaftliche Kommunikation jedoch nur selten unentgeltlich zugänglich, obwohl er de facto im Rahmen öffentlich geförderter Forschungsprogramme die Forschung bereits (mit-)finanziert hat \cite{suber_2003_taxpayer} \cite{resnik_2005_ethics} \cite{Baggs_2006} \cite{Woelfle_2011} \cite{Beverungen_2012} \cite{Adema_2014_open_access}. Da die Mittel nach intransparenten Kriterien verteilt werden ist im aktuellen Kommunikationssystem unklar, ob wissenschaftliche Kommunikation nach dem bestmöglichen Einsatz der monetären Ressourcen für Wissenschaft und Forschung abläuft \cite{Glasziou_2014} \cite{altman_1994_scandal}. Die Europäische Union und die Organisation für wirtschaftliche Zusammenarbeit und Entwicklung (OECD) kommen in diesem Zusammenhang zu dem Ergebnis, dass der volkswirtschaftliche Nutzen von Open Access die Kosten signifikant übersteigt \cite{WD_bundestag_2009}.
\item \textbf{Economic Promotion} - Bisher profitieren wirtschaftliche Unternehmungen nur unzureichend von staatlich finanzierter wissenschaftlicher Kommunikation. Eine schnellere, kommerziell verwertbare und umfassendere Bereitstellung wissenschaftlicher Inhalte kann einen Beitrag zur non-monetären Wirtschaftsförderung und Innovation leisten \cite{heise_2012} \cite{suchen OECD EU}. Im Rahmen der offenen und schnelleren Verbreitung wissenschaftlicher Informationen sind darüberhinaus auch neue Geschäftsmodelle denkbar \cite{suchen}.
\item \textbf{Digital Divide} - Der offene Zugang zu Wissenschaft ermöglicht neue Chancen für die Überwindung sozialer, nationaler und globaler Wissenskluften \cite{suchen} zwischen bildungsferneren und -affineren Bevölkerungsteilen und -schichten der Welt \cite{boai_2012}. Darüber hinaus ist der Mehrwert und die Chance von wissenschaftlichen Informationen für die schulische Bildung und für die Bewegung der offenen Bildungsmaterialien bisher ebenfalls noch nicht vollumfänglich ausgeschöpft \cite{heise_lernen_2013}.
\item \textbf{Validation, Quality and Reputation} - Offenheit in Wissenschaft und Forschung ermöglicht die Entwicklung neuer Verfahren, die die Aktivität und Qualität eines Forschers umfassender, transparenter und demokratischer messbar und kommunizierbar machen, als das es im bestehenden Reputations- und Förderungssystem möglich ist \cite{grand_2012_open}. \cite{chalmers_2009_avoidable_waste}. Es wird vermutet, dass Wissenschaftsevaluation dadurch effizienter wird, da Wissenschaft "per Definition die Bemühung um integre Information ist" \cite{umstatter_2007_qualitatssicherung}. Die Falsifikation ist nur dann umfassend und einfach möglich, wenn der Aufwand für die Falsifikation gering beziehungsweise der Zugriff auf die wissenschaftlichen Informationen überhaupt gegeben \cite{umstatter_2007_qualitatssicherung} und offen ist \cite{Peters_2014}. "Offenheit verhindert, dass Wissenschaft dogmatisch, unkritisch und voreingenommen wird" \cite{resnik_2005_ethics}.
\item \textbf{Paradoxon of Information} - Überwindung des bestehenden Informationsparadoxons bei der Verbreitung und Vermarktung wissenschaftlicher Inhalte. Hierbei handelt es sich um die Herausforderung im Rahmen kommerziell Be- und Verwertung von wissenschaftlichen Informationen ohne zu viel über Inhalt und Qualität auszusagen. Eine im Rahmen von Offenheit angestrebte Entkommerzialisierung des Zugangs zu Wissen würde dieses Informationsparadoxon aufheben.
\item \textbf{Science communication Crisis} - Durch die Öffnung wissenschaftlicher Kommunikations- und Reputationsprozesse entsteht die Möglichkeit, der vorherrschenden Zeitschriften- und Monographienkrise durch neue Geschäftsmodelle zu begegnen \cite{muller_2010_open} \cite{naeder_2010_open}.
\item \textbf{Interdiscipline and International Exchange/Collaboration} - Die Globalisierung führt auch in der Wissenschaft zunehmend zu internationalem Austausch und zur transnationalen Zusammenarbeit von Wissenschaftlern \cite{Waltman_2011}. Das gilt nicht nur für die grenzüberschreitende Zusammenarbeit in Bezug auf die lokale Verortung, sondern auch für die Interdisziplinarität der Forschungsvorhaben. Die Öffnung der Wissenschaft ermöglicht auch fachfremden Wissenschaftlern Zugriff auf Publikationen und damit auf Wissensressourcen für die eigenen Arbeiten \cite{suchen}.
\item \textbf{Sustainable Access and Archiving} - Nur Offenheit im Sinne von Verwertbarkeit ermöglicht es, in dezentralen Strukturen wie der des Internets alle Informationen nachhaltig und unabhängig voneinander zu speichern. Im Falle von Natur- oder anderen Katastrophen ermöglicht die digitale Ablage auf mehreren Kontinenten eine Präservierung von Wissen unabhängig von lokalen Gegebenheiten oder Bedingungen.
\item \textbf{Dataquality} - Die Veröffentlichung der Daten hinter den wissenschaftlichen Publikationen kann zu einer insgesamten Erhöhung der Datenqualität wissenschaftlichen Arbeitens führen. Ähnliche Erfahrungen wurden bereits im Bereich der Veröffentlichung von Daten der Verwaltung und bei der Entwicklungszusammenarbeit gemacht \cite{heise_2014_bundestag}.
\end{enumerate}

\subsection{Hindernisse für die Öffnung der wissenschaftlichen Kommunikation}

Differenzierte Ansätze für den Umgang mit den Fragestellungen rund um die Öffnungsprozesse von Wissenschaft und Forschung sind wichtig, um einen "weniger ideologisch-aufgeregten Umgang mit dem Sujet" \cite[:13]{naeder_2010_open} bei der Ausarbeitung der Arbeit zu erreichen. Im Folgenden werden die Prozesse dargestellt, die entweder zu einer Verlangsamung der Entwicklung führen, oder sie in einigen Teilbereichen sogar ganz zum Erliegen bringen können. Dabei soll explizit keine Position für oder gegen die Veränderung des bestehenden Publikationssystems bezogen werden.

Grundsätzlich lässt sich dabei in strukturelle Hindernisse und individuelle Hindernisse unterscheiden \cite{Scheliga_2014}. Strukturelle Hindernisse beziehen sich dabei auf generelle Herausforderungen bei der Etablierung einer Verhaltensänderung im Rahmen der wissenschaftlichen Kommunikation. Dazu gehören zum Beispiel:
\begin{itemize}
\item Fehlende Anreizsysteme für Wissenschaftlerinnen und Wissenschaftler auf regionaler, nationaler und internationaler Ebene
\item Führungs- und Planlosigkeit der Bewegung für Offenheit in Wissenschaft und Forschung
\item Mangelhafte Infrastrukturen und nicht disponible Applikationen für die Durchführung offener wissenschaftlicher Kommunikation
\end{itemize}

Der Fokus dieser Arbeit liegt auf den wissenschaftlichen Akteuren des Kommunikationssystems. Im Folgenden werden, auch wenn es sich "lohnt das Augenmerk auf "diejenigen Vorteile zu legen, von denen Wissenschaftler selbst profitieren können" \cite{muller_2010_open}, die individuellen Hindernisse für die Öffnung von Wissenschaft und Forschung betrachtet. Folgende individuelle Hindernisse und Argumente für die Öffnung der wissenschaftlichen Prozesse und Publikationen wurden identifiziert:
\begin{enumerate}
\item \textbf{Quality} - Der erste Hindernisbereich umschreibt die Befürchtung, dass die Qualität von offener wissenschaftlicher Kommunikation, durch schlechtee oder nicht vorhandenen wissenschaftlichen Überprüfungsmechanismen, leidet \cite{Chibnik_2015} \cite{Beall_2012}. Dabei wird argumentiert, dass ein durch ein Autorengebühren finanziertes Publikationsmodell keinen klaren Anreiz für Ablehnung bietet \cite{suchen}.
\item \textbf{Renommee} - Die Möglichkeit zur Erlangung von wissenschaftlicher Reputation ist ein grundlegender Motivationsfaktor für Wissenschaftler und Wissenschaftlerinnen die Ergebnisse ihrer Arbeit zu veröffentlichen. Eine Veröffentlichung zahlt nur dann auf die Reputation ein, wenn sie im Rahmen von renommierten Publikationskanälen stattfindet. Offene Publikationsplattformen und Journale können aufgrund des kurzen Zeitraums ihres Bestehens und aufgrund von Vorbehalten dieses Renommee nur selten vorweisen. Die Renommeefrage stellt eine der größten Hürden für die offene wissenschaftliche Kommunikation dar \cite{weishaupt_2009_goldenOA} \cite{Woelfle_2011}.
\item \textbf{Archiving- and Sustainability} - Den grundsätzlichen Vorteilen des elektronischen Publizierens stehen Probleme und Zweifel an der langfristen Verfügbarkeit und Langzeitarchivierung \cite{weishaupt_2009_goldenOA} gegenüber. Einige Autoren kritisieren, dass die Sicherstellung der Langzeitarchivierung und die langfristige Auffindbarkeit, sowie Bereitstellung der Dokumente bisher nicht vollumfänglich durch digitale Strukturen gewährleistet werden kann \cite{umstatter_2007_qualitatssicherung} \cite{Gersmann_2007}.
\item \textbf{Authenticity- or Integrity} - Ein weiteres Problem stellt die Sicherung der Authentizität der offen publizierten wissenschaftlichen Informationen dar \cite{umstatter_2007_qualitatssicherung} \cite{weishaupt_2009_goldenOA} \cite{grand_2012_open}. Weil elektronische Dokumente oft innerhalb weniger Tage oder Wochen in mehreren Versionen zugänglich sind wird befürchtet, dass Texte und Arbeiten, im Zeitablauf inhaltlich nicht mehr unverändert ihrem Autor zuordenbar sind. Das gilt "solange sie nicht in Digitalen Bibliotheken mit gesicherter Authentizität abgeliefert" werden \cite{umstatter_2007_qualitatssicherung}.
\item \textbf{Rightsmanagement} - Die Verpflichtung für Mitarbeiter staatlich finanzierter Forschungsinstitutionen, alle Texte und Daten elektronisch frei und offen zu publizieren, wird von einigen Autoren als kritisch bewertet \cite{suchen}. In dem 2009 veröffentlichten "Heidelberger Appell" \cite{faz_heidelberger_apell_2009} kritisieren zahlreiche Autoren, Wissenschaftler, Verleger und Publizisten, dass das “verfassungsmäßig verbürgte Grundrecht von Urhebern auf freie und selbstbestimmte Publikation” … “derzeit massiven Angriffen ausgesetzt und nachhaltig bedroht” ist. Weiter sehen die Unterzeichner „weitreichende Eingriffe in die Presse- und Publikationsfreiheit, deren Folgen grundgesetzwidrig wären“ \cite{ITK_2009}. Rechtliche Bedenken und die Befürchtung vor kostspieligen juristischen Fehltritten stellen einen weiteren Vorbehalt gegen die offene Veröffentlichung von Forschung- und Forschungsergebnissen dar \cite{weishaupt_2009_goldenOA}.
\item \textbf{(Re-)Financing} - Die unklare Refinanzierung der Kosten, die im Rahmen der offenen wissenschaftlichen Kommunikation vermutet werden, werden als weiteres Kernargumente gegen das offene Publizieren von Arbeiten und Daten angeführt \cite{Chibnik_2015}. Die Befürchtung ist, dass die umfassende Öffnung des wissenschaftlichen Systems überhaupt nicht finanziert werden kann, konnte bisher nicht vollumfänglich ausgeräumt werden \cite{weishaupt_2009_goldenOA}.
\item \textbf{Ressource-Allocation} - Dieses Hindernis befasst sich mit der Annahme, dass der Vergabe von Fördermitteln und bei den reputationsbildenden Maßnahmen durch offene System nicht Rechnung getragen werden kann. Das Argument ruht auf dem Verdacht, dass die Öffnung des wissenschaftlichen Prozesses Einfluss auf Mittelvergabe hat \cite{grand_2012_open} und ausschließlich zugunsten populärer Forschung stattfindet und es zu einer Aushöhlung der wissenschaftlichen Fächer- und Facettenvielfalt kommt.
\item \textbf{Open-Caring} - Wissenschaftlerinnen und Wissenschaftler befürchten durch den Zwang zu umfassenden Bereitstellung ihrer Publikationen und gegebenenfalls sogar der Quelldaten sowie des genutzten Softwarecodes einen nicht unwesentlichen zeitlichen und finanziellen Mehraufwand \cite{bbaw_publizieren_2015} \cite{mennes_2013_making_os} \cite{grand_2012_open}. Der nötige Aufwand den die umfassende Öffnung der wissenschaftlichen im Alltag des Wissenschaftlers mit sich bringen würde, ist bisher kaum evaluiert \cite{osterloh2008anreize}.
\item \textbf{Scientific-Freedom/Loss of Idea-Diversity}
Dieses Argument betrifft zwei Ebenen: Die Sorge dass durch Offenheit und Transparenz Forschungsförderung und Öffentlichkeit die bestehenden Steuerungsmechanismen der Wissenschaft ausgehebelt werden und infolgedessen nur die wissenschaftlichen Projekte gefördert und unterstützt werden, die vom Souverän verstanden werden. Diese Befürchtung ruht auf der Annahme, dass die Gewinnung von Wissen zum Beispiel in der Grundlagenforschung ein "öffentliches Gut" darstellt, "dessen Wert von der Öffentlichkeit nur schwer beurteilt werden kann"\cite{osterloh2008anreize}. Darüber hinaus wird in der Literatur die Befürchtung geäußert, dass durch die Öffnung die Freiheit von Forschung und Lehre im Sinne der Publikations- und Veröffentlichungsfreiheit gefährdet wird \cite{Jochum_2009}. Damit ist die Wahl des Publikationsmediums gemeint, die bei den Wissenschaftlerinnen und Wissenschaftlern liegen sollten \cite{bbaw_publizieren_2015} . Infolgedessen wird an vielen Stellen die Befürchtung geäußert, dass im Rahmen von zunehmender Kollaboration über digitale Kanäle, sowie durch die Effizienz der elektronischen Suche die Diversität von wissenschaftlichen Meinungen und Projekten zu einem gleichen oder ähnlichem Thema eingeschränkt werden könnte \cite{Evans_2008}. Diese Betrachtung ist allerdings nicht unumstritten \cite{lariviere2009decline}.
\item \textbf{Fehlinterpretation} - Eine weitere Sorge, die den Öffnungsprozess bremst, ist die Angst der wissenschaftlichen Community vor Fehlinterpretationen \cite{grand_2012_open}, sowie der Verlust der Kontrolle über die Informationssteuerung \cite{gibbons_1994}. Dabei steht vor allem die Befürchtung im Vordergrund, dass die frei verfügbaren veröffentlichten Arbeiten genutzt werden, um die Arbeit der Wissenschaft zu diskreditieren oder sie gezielt zur Falschinformation der Öffentlichkeit genutzt werden.
\item \textbf{Transparent-Research-Intentions} - Die Forderung nach Offenlegung des gesamten Forschungsprozesses beinhaltet auch die Forderung nach "Transparenz der Interaktion zwischen Sponsoren (insbesondere kommerzielle Förderer wie die Pharma- und Medizinprodukteindustrie) und Auftragnehmern" \cite{Stengel_2013}
\end{enumerate}

---- TODO: Liste überarbeiten  bzw. abgleichen ----

Die erarbeiteten Hindernisse für die Verbreitung der Öffnung wissenschaftlicher Kommunikation werden im Verlauf der Arbeit im Rahmen der Befragung aufgegriffen. Die möglichen Irritationspotenziale durch die Ausweitung des Zugangs zu oder Zugriffs auf wissenschaftliche Kommunikation sowie die Kritik an digitalen Medien werden nur in Zusammenhang mit den Forschungsfragen berücksichtigt.

\subsection{Indikatoren für Reputationsverteilung im wissenschaftlichen Kommunikationsystem}

Um die Anreize für das Verhalten der wissenschaftlichen Akteure im Kommunikationssystem besser zu verstehen werden im folgenden die Indikatoren für die Reputationsverteilung herausgearbeitet, wobei die Publikation von Erkenntnissen nur einer von vielen Indikator für die Reputationsverteilung ist \cite{hirschauer2004peer}. Im Gegensatz zu den Modellen die eine Verpflichtung von oben für ein bestimmtes Verhalten beinhalten und die wissenschaftliche Selbstständigkeit beeinflussen könnten, werden hier vor allem die Indikatoren betrachtet, die Anreize für ein bestimmtes Verhalten darstellen.

Aus der Literatur wurden demnach folgende Indikatoren für die Verteilung von Reputation herausgearbeitet. Die vorgenommene Kategorisierung ist dabei an Heidemarie Hanekop \cite{hanekop_2008} und die Befragung durch das SOFI 2007 \cite{Hanekop_Wittke_2007_Fragebogen} angelehnt:
\begin{enumerate}
\item \textbf{Anzahl der wissenschaftlichen Aufsätze / Beiträge}: Die Anzahl der Texte die Wissenschaftler im Rahmen ihrer Tätigkeit publizieren ist ein wesentlicher Faktor der Bewertung wissenschaftlicher Reputation \cite{Warnke_2012} \cite{CLAPHAM_2005} \cite{luhmann_1970_selbststeuerung}. Zum Beispiel erhöht die Anzahl an Texten die Chance durch die wissenschaftliche Community zitiert zu werden und damit die Möglichkeit auf die Erlangung von Reputation. Durch den zunehmenden Wettbewerb in der Wissenschaft muss sich der einzelne Wissenschaftler entscheiden, "zu publizieren oder im wissenschaftlichen System zu scheitern" \cite{Suess_2006}. Dadurch entsteht im wissenschaftlichen Kommunikationssystem ein konstanter Publikationsdruck, bei dem die Relevanz der publizierten Ergebnisse nicht immer im Vordergrund steht \cite{hamilton_1990_publishing}. Die Anzahl der veröffentlichten Artikel hat einen Einfluss auf die Vergabe von Ressourcen und finanziellen Mittel für weitere Forschung an Institutionen und Individuen \cite{Warnke_2012} \cite{hamilton_1990_publishing}.
\item \textbf{Relevanz der publizierten Ergebnisse}: Die Relevanz der publizierten Ergebnisse ist für das Wissenschaftssystem ein wesentlicher Katalysator für den Prozess der Wissensgewinnung. Relevante Erkenntnisse sind die Grundlage für die Produktion von neuem Wissen und damit Grundlage für den gesellschaftlichen Auftrag des Wissenschaftssystems \cite{hanekop_2008}. Das wissenschaftliche System beruht auf der Annahme, dass die Relevanz der publizierten Ergebnisse einen direkten Einfluss auf die wissenschaftliche Reputation hat.
\item \textbf{Anzahl Monografien}: Die Anzahl der veröffentlichten Monographien ist ein wesentlicher Reputationsfaktor. Das gilt für die Disziplinen, in denen diese Publikationsform wichtig ist, wie den Geistes- und Sozialwissenschaften. In den anderen wissenschaftlichen Fachrichtungen spielt die Anzahl der Veröffentlichungen von Artikeln in wissenschaftlichen Journalen eine wichtige Rolle.
\item \textbf{Drittmittelprojekte}: Drittmittel sind, so der deutsche Wissenschaftsrat, "solche Mittel, die zur Förderung der Forschung und Entwicklung sowie des wissenschaftlichen Nachwuchses und der Lehre zusätzlich zum regulären Hochschulhaushalt (Grundausstattung) von öffentlichen oder privaten Stellen eingeworben werden" \cite{wr_2014}. Die Drittmitteleinwerbung hat sich in Deutschland als "meist gebrauchter Maßstab der Messung von Forschungsqualität durchgesetzt" \cite{M_nch_2006}. Diese Entwicklung geht mit einer zunehmenden Finanzierung der Forschung über Drittmittel einher \cite{Neidhardt_2010} \cite{Jansen_2007} \cite{simon_2009_wissenschaft_governance}. Durch die zunehmende Knappheit öffentlicher Ressourcen für Wissenschaft und Forschung, ist die Akquise von Drittmitteln zu einem kritisch zu betrachtenden Kernziel geworden \cite{Jansen_2007}. Das führt, dass zunehmend direkte finanzielle und administrative Kontrolle der Forschung eine Rolle spielen \cite{Barl_sius_2008}. Dabei spielt die Frage eine Rolle, ob die Publikationen, die im Rahmen der Drittmittelfinanzierung als wissenschaftliche Erkenntnisse veröffentlicht werden und ob der Antrag um Drittmitteleinwerbung selbst, "zum Erkenntnisfortschritt in der wissenschaftlichen Gemeinschaft beiträgt" \cite{M_nch_2006}. Die wissenschaftliche Community befürchtet durch die zunehmende Relevanz der Anzahl von Drittmittelprojekten bei der Erlangung von wissenschaftlicher Reputation eine Einschränkung der Freiheit von Wissenschaft und Forschung.
\item \textbf{Patente}: Im Gegensatz zu Urheberrechten, werden Patente nur auf Antrag und nach Prüfung staatlich erteilt \cite[:152]{troy_2012_wissen}. Es handelt sich dabei ausl um ein "vom Staat verliehene Schutzrecht für eine technische Erfindung, welches dem Patentinhaber für eine bestimmte Zeit die ausschließliche wirtschaftliche Nutzung der Erfindung vorbehält" \cite{greif_2003_patente}. Diese Kommodifizierung von Wissen in Form von Patenten ist dabei exemplarisch für die Privatisierung von Wissen \cite[:152]{troy_2012_wissen}. Die Anzahl dieser Schutzrechte im Hochschulbereich nimmt seit den 1970er konstant zu \cite[:168]{troy_2012_wissen}. \cite{schmoch_2003_hochschulforschung} \cite{Fabrizio_2008}. Vor allem in den technischen Fachdisziplinen wird eine Patentschrift "als funktionales Äquivalent zur wissenschaftlichen Publikation begriffen" und bewertet \cite{mersch_2014_patente}. Die deutsche Hochschulrektorenkonferenz hält fasst die Rolle des Patentwesen an den Hochschulen wie folgt zusammen: "Patente leisten einen Beitrag zur Förderung der Wissenschaft, die Grundlagen des Patentwesens sind daher dem wissenschaftlichen Nachwuchs über entsprechende Lehrangebote zu vermitteln." \cite{suchen-Position-HRK} Die Befürchtung, dass Patente einen negativen Effekt auf die Erstellung und Veröffentlichung von fundamentaler Forschungsergebnisse hat, konnte nicht abschließend bestätigt werden \cite{Fabrizio_2008}.
\item \textbf{Vorträge}: Vorträge dienen der Verbreitung der Forschungserkenntnisse, sowie Zwischenständen und ermöglichen das Vermitteln des Wissens an andere \cite{rassenhoevel_2010_performancemessung}. Vorträge stellen eine informelle und schnelle Form für die Verbreitung neuer wissenschaftlicher Erkenntnisse und Ergebnisse dar. Die in einem Vortrag vermittelten Inhalten müssen meist nicht genauer belegt werden und die kommunizierten Inhalte lassen gegebenenfalls später schriftlich konkretisieren oder korrigieren \cite{haberle_2002_jahrbuch}. Vorträge bieten die Möglichkeit bereits vor der eigentlichen Publikation von wissenschaftlichen Erkenntnissen Anregungen und Reaktion einzuholen.
\item\textbf{Anwendungsrelevanz bzw. Verwertbarkeit}: Ein vergleichsweise neuer Indikator die Reputation von Hochschulen und außeruniversitäre Forschungsinstitute ist die Anwendungsrelevanz von Wissenschaft und Forschung \cite{simon_2009_wissenschaft_governance}. Sie bezieht sich auf einen Outputfaktor, der sich primär auf den Einsatz der gewonnenen wissenschaftlichen Erkenntnisse und auf die Verwertbarkeit für wirtschaftliche Produkte oder Patente als auf die eigentliche wissenschaftliche Veröffentlichung abzielt \cite{suchen}.
\item \textbf{Netzwerke und Kontakte}: Netzwerke beschreiben formelle und informelle Verbundsysteme zwischen Wissenschaftlern. Sie erlauben den schnellen Austausch und können Grundlage für Aktivitäten zur Steigerung der wissenschaftlichen Reputation darstellen. Diese Aktivitäten umfassen zum Beispiel gemeinsame Publikationsvorhaben und den Austausch wissenschaftlicher Erkenntnisse. Kontakte und Netzwerke schaffen soziale Beziehungen, die für eine erfolgreiche Integration an der Hochschule und der Fachcommunity sorgen, Zugang zu wissenschaftlicher Kommunikation ermöglichen und somit einen Einfluss auf die Anerkennung eines Wissenschaftler oder einer Wissenschaftlerin haben können.
\item \textbf{Öffentliche Aufmerksamkeit}: Die öffentliche Aufmerksamkeit stellt zum einen eine Möglichkeit des Wissenstransfers außerhalb der wissenschaftlichen (Fach-)Community dar, zum Anderen ermöglicht sie die Einflussnahme auf die politische Relevanz wissenschaftlicher Forschungsthemen. Die Veröffentlichung von wissenschaftlichen Informationen zu einem bestimmten Thema des öffentlichen Interesses stellt eine Möglichkeit dar, dieses Thema öffentlichkeitswirksam zu katalysieren. Öffentliche Aufmerksamkeit im Rahmen von wissenschaftlicher Tätigkeit stellt eine kritisch zu hinterfragende Möglichkeit für die alternative Ressourcengewinnung dar. \cite{suche}
\item \textbf{Politische Relevanz}: Die wissenschaftliche Tätigkeit mit politischer Relevanz stellt eine weitere Möglichkeit dar, wissenschaftliche Inhalte außerhalb der Wissenschaft anwendbar zu machen und führt zu Anerkennung der wissenschaftlichen Arbeit. Daraus ergeben sich allerdings grundsätzliche "Verständigungsprobleme und Interessenkonflikte", da  "Wissenschaft und Politik aufgrund unterschiedlicher Rationalitäten handeln, einander aber zugleich brauchen" \cite{Mayntz_1996}. Während es im Wissenschaftssystem "um Erwerb und Erhalt von Wissen" geht, zielt die Politik auf "Erwerb und Erhalt von Macht" \cite{Mayntz_1996} ab. Dennoch wirkt Wissenschaft durch wissenschaftliche Beratung auf Politik und  Politik beeinflusst Wissenschaft durch Wissenschaftspolitik \cite[:10]{Brown_2014}. Die daraus resultierenden Interessenkonflikte können jedoch die Legitimität der Wissenschaft beeinträchtigen \cite{weingart_2005_wissenschaft} \cite[:494]{Weber_1992} führen gegebenenfalls zu "gegenseitigen Enttäuschungen", vor allem in der "forschungspolitischen Beziehung" \cite{Mayntz_1996}.
\item \textbf{Renommee der Forschungseinrichtung}: Das Renommee einer Forschungseinrichtung ist die Wahrnehmung der Einrichtung innerhalb und außerhalb der wissenschaftlichen (Fach-)Community. Sie hat für Wissenschaftler und die Wissenschaftlerin eine besondere Bedeutung \cite{mayntz_2008_wissensproduktion}. Sie basiert auf dem Konzept der "Ansteckung" \cite{luhmann_1970_selbststeuerung}. Diese Ansteckung führt zum Beispiel dazu, dass renommierte Professoren den Ruf einer Fakultät und eine renommierte Fakultät auch den Ruf von Professoren aufbessern können. Übertragen auf das wissenschaftliche Publizieren profitiert ein Autor oder eine Autorin bei der "Ansteckung" von dem Renommee einer Einrichtung, wenn er durch die Publikationsorgane der renommierten Institution veröffentlicht \cite{lutz_2012_zugang}.
\item \textbf{Renommee von Herausgebern oder Mitautoren} Der Herausgeber organisiert den Begutachtungsprozess und sichert bestimmte Qualitätskriterien mit seiner Reputation und seinem Namen \cite{mueller_2009_peerreview}. Auch hier kommt es im Rahmen des symbolischen wissenschaftlichen Kapitals zu einer Übertragung der Reputation der Herausgeber oder Mitautoren auf die anderen veröffentlichenden Autoren.
\item \textbf{personelle und materielle Ausstattung}: Die materielle Ausstattung beschreibt die Rahmenbedingungen, in der ein Wissenschaftler arbeitet. Diese Rahmenbedingungen haben eine herausragende Bedeutung bei der Entscheidung über einen Wirkungsort von Wissenschaftlern \cite{mayntz_2008_wissensproduktion}. Insbesondere die materielle und personelle Ausstattung sind bei traditionellen Berufungsverfahren deutscher Professorinnen und Professoren von besonderem Belang \cite{himpele_2011_job}, da sie die Arbeitsfähigkeit und die Anerkennung direkt beeinflussen \cite{suche}. Wie die materielle Ausstattung gilt auch die personelle Ausstattung als ein reputationsstiftendes Merkmal für Wissenschaftler und die Institution, an denen sie arbeiten \cite{mayntz_2008_wissensproduktion}. Bei der Ausstattung handelt es sich um einen bilateralen Indikator, der zum einen aus der Bewertung der wissenschaftlichen Arbeit (im Rahmen der Forschungsförderung) resultiert \cite{Herb_vermessung_2008} und  zum anderen Reputation innerhalb der Community schafft \cite{mayntz_2008_wissensproduktion}.
\item \textbf{Gutachtertätigkeit und Herausgeberschaft}: Gutachter werden zum Beispiel in Peer-Review-Verfahren Autoren des entsprechenden Fachgebietes zugeordnet und entscheiden über die Veröffentlichung des Textes \cite{Frey_2005}. Bei manchen Publikationen wird ein Text mehrmals abgelehnt und eine Überarbeitung durch den Autoren eingefordert, bevor der Artikel final akzeptiert und daraufhin publiziert wird \cite{Frey_2005}. In diesem Zusammenhang wirkt sich die Reputation der mit diesem Verfahren betrauten Gutachter auch auf das Image des Verlages aus und umgekehrt. Die Gutachtertätigkeit ist aber nicht nur Kernbestandteil des wissenschaftlichen Qualitätssicherungs- und interdependenten Reputationssystems, sondern stellt auch einen informellen Weg der Kommunikation dar. Er ermöglicht den Gutachtern die Vorabsichtung neuster wissenschaftlicher Informationen und Erkenntnisse. Ähnlich wie die Gutachtertätigkeit ist auch die Herausgeberschaft fester Bestandteil des interdependenten wissenschaftlichen Reputationssystems \cite{Frey_2005}: Herausgeber profitieren von den publizierten Inhalten und Erkenntnissen der Autoren, Autoren von der Reputation Herausgebern und der Verlag von beiden \cite{suchen}.
\item \textbf{Funktion}: Die jeweilige Funktion oder die (universitäre) Stellenbezeichnung ist ein weiter Faktor für wissenschaftliche Reputation. Zum wissenschaftlichen Personal zählen Professoren, Juniorprofessoren, wissenschaftliche und künstlerische Mitarbeiter, sowie Lehrkräfte \cite{erhardt_2011_hochschulen}. Eine Weiterentwicklung und der "Aufstieg" in der wissenschaftlichen Hierarchie zielt auf das akademische Streben nach einer Professur \cite{Klecha_2008}.
\item \textbf{Awards und Preise}: Preise sind ein weitere Indikator für das wissenschaftliche Belohnungs- und Bewertungssystem. "Die Praxis der Award-Verleihung beruht auf dem Konzept, dass Ressourcen von unabhängigen Dritten auf Qualität geprüft und (...) zertifiziert werden" \cite{bargheer_2002_qualitatskriterien}. Wissenschaftler und Wissenschaftlerinnen, die Preise oder Awards gewinnen, erfahren Anerkennung. Diese Anerkennung können jedoch nicht automatisch als "Garant für wissenschaftsrelevante Qualität"\cite{bargheer_2002_qualitatskriterien} verstanden werden. Die Ehrung mit einem Preis weckt große Erwartungen und führt zu dem Anspruch eines stetigen Nachschubs an Anerkennung für den Wissenschaftler oder die Wissenschaftlerin \cite{suchen}.
\end{enumerate}
