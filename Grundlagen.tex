\chapter{Grundlagen, Definitionen und Abgrenzungen}

Der theoretische Bezugsrahmen wissenschaftlich gesicherter Modelle, Theorien und Ansätze ermöglicht es, Erklärungen und Handlungsempfehlungen abzuleiten \cite{martin_2007_wissenschaftstheorie}. Er trägt dazu bei, die Fragestellungen in einen Zusammenhang zu stellen, legitimiert die Erforschung dieser Fragen und bildet den Rahmen für die Auswertung gesammelter Erkenntnisse \cite{suchen}. Ziel dieses Kapitels ist es, die theoretischen Grundlagen für die im nächsten Abschnitt folgende Inhaltsanalyse und die empirischen und experimentellen Ergebnisse zu erarbeiten, sowie die Einführung der Begriffe, Definitionen und Konzepte, die für das Thema der vorliegenden Arbeit grundlegend sind.

Wenn es im aktuellen öffentlichen Diskurs um wissenschaftliche Informationen, Infrastruktur und Arbeiten geht, werden immer öfter Schlagworte mit dem Attribut „Open“, wie Open Access, Open Research und Open Science, verwendet \cite{bunz_2014} \cite{schulze_2013_open}. "Offen" bezieht sich dabei üblicherweise auf zwei Kernaspekte: Zum einen die Offenheit des Zugangs zu Daten, Quellcode oder Ergebnissen und zum anderen auf das Gebot der Transparenz, also die Offenlegung, beziehungsweise der Zugriff auf Verfahren, Methoden und Ziele \cite{schulze_2013_open}. Während "Openness" vielfach mit den Entwicklungen rund um offene Software assoziiert wird, gibt es andererseits Anknüpfungspunkte von "Offenheit" als Begriff in der wissenschaftlichen Auseinandersetzung die schon früher anzusetzen sind \cite{Tkacz_2014}. So sieht Christopher Kel­ty die ersten Anfänge bereits in den 1980ern \cite{kelty_2008_two_bits}. Andrew Russell sieht die ideologischen Ursprünge von "Offenheit" als Standard schon in der Entwicklung des Telegraphs und weiteren Ingenieurleistungen seit 1860 \cite{Russell_2014}.

In der gegenwärtigen Literatur sind die Begriffe „Open Access“ und „Open Science“ nicht eindeutig voneinander abgegrenzt und sie finden in der wissenschaftlichen Auseinandersetzung auf unterschiedlichste Art und Weise Verwendung \cite{cite:9}. Infolgedessen werden Open Science und Open Access in dieser Arbeit im Kontext wissenschaftlicher Reputation in Bezug auf ihre technischen, gesellschaftlichen und politischen Aspekte beschrieben und die Betrachtung wird auf die kulturellen Auswirkungen der Medienbrüche im Rahmen wissenschaftlichen Publizierens erweitert. Der historische und gesellschaftliche Kontext ihrer Anwendung wird dargestellt und mittels der Analyse wissenschaftlicher Literatur abgegrenzt. Es wird erläutert, welche Bedeutung sie in der Forschung, der Gesellschaft und der Politik haben. Die Entstehung und Entwicklung der Begriffe wird beschrieben. Um ein möglichst umfassendes Bild zu erhalten, wird "Entwicklung" hier in den drei folgenden Dimensionen erfasst: erstens, als "analytische Kategorie", zweitens als "Forschungsgegenstand" und drittens als "politische Praxis in der moralischen Auseinandersetzung über die Wünschbarkeit von Zuständen" \cite{cite:10}. Die Analysen in dieser Arbeit werden aus der Perspektive des Produzenten (Wissenschaftler als Autoren) sowie aus der, damit nicht immer harmonisierenden, Perspektive des Rezipienten, beziehungsweise Medienkonsumenten (Wissenschaftler als Leser) stattfinden. Es wird auch adressiert, inwiefern Macht, regulierende Prinzipien wie die Verknappung, sowie die Ein- und Ausgrenzung im Rahmen wissenschaftlicher Diskurse (nach dem Diskurs- und Machtbegriff) mit den Modellen Open Access, Open Science und wissenschaftlicher Reputation in der Kommunikation vereinbar sind oder diesen gegenüberstehen.

Die Unterscheidung von "Zugang" und "Zugriff" ist in dieser Arbeit wesentlich und stellt eine zentrale Grundlage für die Definition und Abgrenzung der Begriffe "Open Access" und "Open Science" dar. "Zugang" bezieht sich in diesem Zusammenhang auf einen unbeschränkten Zugang zur finalen wissenschaftlichen Publikation. "Unbeschränkt" meint hier vor allem das ausschliessliche Lesen der finalen Ergebnispublikation \cite{cite:9a}, Verarbeitung und Weiternutzung. "Der Open-Access-Ansatz bezieht sich zunächst lediglich auf die Zugangsbedingungen zu den wissenschatlichen Arbeiten" \cite{muller_2010_open}. "Zugriff" soll als erweiterte Nutzung der jeweiligen Wissensressourcen verstanden werden und schließt neben dem "Zugang" zur Publikation sämtliche Informationen und Daten, sowie die komplette Kommunikation hinter der finalen Veröffentlichung \cite{cite:9b} ein. "Zugriff" beschränkt sich hier also nicht nur auf den reinen Zugang zu wissenschaftlicher Information im Rahmen des Publikationsprozesses, sondern schließt auch den Zugriff auf sämtliche Forschungsdaten, Methoden und alle weiteren Informationen, die während der wissenschaftliche Arbeit auf dem Weg zur finalen Publikation entstehen \cite{cite:9c}, ein.

Die Themenbereiche kollaboratives Arbeiten, Social Media in Wissenschaft und Forschung, Citizen Science und aktuelle Diskurse zu Tools und Diensten werden in dieser Arbeit bewusst ausgelassen und nur am Rande, beziehungsweise nur wenn sie die Beantwortung der Forschungsfragen tangieren, eingeschlossen.

\section{Wissenschaftliche Kommunikation}
Bevor die Grundlagen für Offenheit in Wissenschaft und Forschung definiert werden, wird eine grundlegende Einordnung und Abgrenzung von wissenschaftlicher Kommunikation vorgenommen sowie deren Wandel im Rahmen der Digitalisierung beschrieben und mit anderen Bewegungen verglichen.

Kommunikation kann als die "Essenz der Wissenschaft" verstanden werden \cite{garvey_2014_communication} und ist eng mit der Wissensproduktion verknüpft. Generell basiert Wissenschaft auf der Kommunikation zwischen den Wissenschaftlern, die auf einem "gemeinsamen Wissensbestand, den sie testen, verändern und erweitern" \cite{Gl_ser_2007}. Sinn und Zweck der wissenschaftlichen Kommunikation besteht auf dem bestmöglichen Austausch zwischen den Mitgliedern der Wissenschaftsgemeinschaft. Jede kommunizierte Erkenntnis trägt theoretisch zur Produktion von Wissen bei \cite{kaden_2009_library}. Grundvoraussetzung dafür ist, dass Wissenschaftler und Wissenschaftlerinnen einen Willen zu optimalen Kommunikation untereinander haben.

Es existieren verschiedene Arten der wissenschaftlichen Kommunikation: die \textit{formelle} und die \textit{informelle wissenschaftliche Kommunikation}. Was genau als formell oder informell gilt, hängt von der jeweiligen Fachdisziplin ab und "ist historisch gewachsen und damit durchaus unterschiedlich, ebenso unterschiedlich wie die Bedeutung der verschiedenen Arten von Publikationen (Journale, internationale Journale, Monografien, Handbücher und Proceedings, etc.)" \cite{Hanekop_2014}.

Die Basis für wissenschaftlichen Fortschritt, Forschungsförderung und traditionelles Publizieren bilden Journale und Monographien \cite{cope2014future}. Das wissenschaftliche Journal ist (in den meisten wissenschaftlichen Disziplinen) ein wichtiger formeller Kanal der wissenschaftlichen Kommunikation und essenziell für Wissenschaftler und Wissenschaftlerinnen um auf dem Laufenden zu bleiben \cite{cope2014future}.

Die formelle Kommunikation wird an bestimmte Bedingungen der wissenschaftlichen Gemeinschaft geknüpft und hat einen direkten Einfluss auf die wissenschaftliche Reputation der einzelnen Mitglieder der wissenschaftlichen Community. Diese Art der Kommunikation beinhaltet meist auch die Einbeziehung Dritter, die meist die Funktion der Einordnung und Bewertung der Kommunikation übernehmen. Die Praxis dieser Kommunikation ist die Publikation. Ziel dieser Art der Kommnikation ist die Sicherung des Verbleibs und Positionierung des einzelnen Wissenschaftlers innerhalb der wissenschaftlichen Gemeinschaft. Diese Formalisierung der Kommunikation ist wichtig um das Wissenschaftssystem strukturell zu sichern \cite{kaden_2009_library}. Sie macht Erkenntnisprozesse nachweisbar \cite{kaden_2009_library}.

Formelle wissenschaftliche Kommunikation beruht nach dem Bibiliothekswissenschaftler Ben Kaden auf drei Faktoren \cite{kaden_2009_library}:
\begin{enumerate}
\item \textit{Publizität} meint die Veröffentlichung der Erkenntnisse in einem wissenschaftlichen Fachmedium. Eine Erkenntnis wird durch die Veröffentlichung bekannt gegeben und so für die Community "registriert".  Sie muss dabei "zeitnah" in einer "wahrnehmbaren" Form vorliegen \cite{Schimank_2012}, damit sie intersubjektiv vermittelbar ist.
\item \textit{Vertrauenswürdigkeit} meint das Vertrauen auf die Einhaltung der Regeln im Kommunikationssystem durch alle Teilnehmer. Das Vertrauen wird bei einer Publikation durch die Überprüfung (Peer-Review) bestätigt und durch Bezugnahme (Zitationen) anderer Wissenschaftler auf die Publikation zu Reputation.
\item \textit{Zugänglichkeit} bezieht sich auf die dauerhafte Sicherung und Zugänglichkeit in einer allgemein verfügbaren Form für die Fachöffentlichkeit.
\end{enumerate}

Die Möglichkeiten der informelle Wissenschaftskommunikation sind höchst vielfälti und reichen "vom persönlichen Gespräch über Vorträge, Konferenzen, Zwischen- oder Abschlussberichte aus Projekten, Working Papers und vieles andere mehr"\cite{Hanekop_2014}. Informelle Kommunikation umfasst somit alle Arten der Kommunikation, die dem individuellem Wissenschaftler einen schnellen und direkten Austausch mit Kollegen ermöglichen und die keinen direkten Einfluss auf die wissenschaftliche Reputation des einzelnen Wissenschaftlers haben. Diese Art der Kommunikation stand im klassischen wissenschaftlichen Wertschöpfungsprozess meist am Anfang. Sie umfasst zum Beispiel die Ideenfindung, die Entwicklung von Fragestellungen oder Konkretisierung des Forschungsvorhabens und hilft Wissenschaftlern dabei relevante Ideen für formelle Kommunikation "herauszukristallisieren" \cite{Hanekop_2014}. Die Abgrenzung von informeller Kommunikation zu "nicht-wissenschaftlichen Kommunikation" resultiert daraus, dass sie auf "die Erzeugung formeller Kommunikation hinarbeitet" \cite{kaden_2009_library}. Informelle Kommunikation ist auf Grund ihrer Heterogenität und impliziten Verankerung weniger präzise differenzierbar und erfassbar \cite{kaden_2009_library}.

\subsection{Digitalisierung der wissenschaftlichen Kommunikation}

Wie bereits in der Einleitung dieser Arbeit dargestellt wurde, haben die Digitalisierung und die dahinterstehenden Technologien einen tiefgreifenden Einfluss auf die wissenschaftlichen Kommunikation und die wissenschaftliche Arbeit in allen Fachdisziplinen. Infolgedessen sind mittlerweile über 90 Prozent der englischsprachigen Journale online verfügbar und es gibt einen steigenden Trend zu Journalen die nur im Internet verfügbar sind \cite{cope2014future}. Mit diesem digitalen Wandel in der wissenschaftlichen Kommunikation wird auch die Chance für eine umfassende “Beschleunigung des Wissensumschlages” verbunden \cite{Wenzel_2003}, sowie mit der Hoffnung verknüpft, dass offene Innovation und offene wissenschaftliche Kommunikation den privaten und staatlichen Forschungsbereich offener, integrativer und effizienter machen \cite{suchen}.

Konkret erfolgte bisher mit der Etablierung digitaler Kommunikation eine Veränderung der Kategorisierung von wissenschaftlicher Kommunikation. Während im Druckzeitalter die formelle wissenschaftliche Kommunikation eng an die bibliometrischen Indikatoren geknüpft war und eindeutig von der informellen abgegrenzt werden konnte, scheint diese klare Grenze im Rahmen der Digitalisierung zu verschwimmen. Vor allem der Bereich der informellen Kommunikation verändert sich durch das Internet, was "jedoch nur vermittelt und mit zeitlicher Verzögerung Wirkungen auf das formelle Publikationssystem zeigt" \cite{Hanekop_2014}. Die Sozialwirtin Heidemarie Hanekop definiert diesbezüglich den folgenden Zusammenhang: "Je größer die Abkopplung zwischen informellen und formellen Aspekten der wissenschaftlichen Kommunikation in einem disziplinären, thematischen oder nationalen Wissenschaftsbereich, um so geringer, vermittelter oder langwieriger kann auch die Wirkung des Internets auf diesen Teilbereich des Publikationssystems sein" \cite{Hanekop_2014}. Ben Kaden hingegen definiert die Veränderungen im Kommunikationssystem als \textit{kanalerweiterte Wissenschaftskommunikation} und erklärt diese als "Form der Wissenschaftskommunikation, die die informelle und formelle ergänzt" und die "indiviuell affirmativ" als "als eine Art informelles offenes Post Review" verstanden werde kann \cite{kaden_2009_library}.

Im folgenden Verlauf dieser Arbeit wird wissenschaftliche Kommunikation als die Kommunikation verstanden, die formelle Bezugspunkte aufweist und vor allem einen Einfluss auf die wissenschaftliche Reputation des Wissenschaftlers oder der Wissenschaftlerin hat.

\subsection{Wissenschaftliche Kommunikation als Open-Source-Prozeß}
Im Rahmen der Forderung nach der Öffnung der wissenschaftlichen Kommunikation und wissenschaftlichen Publikationen werden in der Literatur häufig Vergleiche zur Open Source-Bewegung gezogen \cite{suchen}. Diese Vergleiche können beispielhaft dem Verständnis theoretischer Grundlagen im Rahmen der Öffnung von Wissenschaft und Forschung dienen.

"Open Source" ist ein Begriff aus der Softwareentwicklung der als Gegensatz zum “Verfahren der Wissenssicherung” \cite{stallman2002} eine quelloffenen Handhabe von Programmcode beschreibt und der Ende der 1990iger erstmals eingeführt wurde  \cite{suchen}. Dieser Begriff wird, auch wenn es im Detail Unterschiede im Konzept gibt \cite{suchen}, häufig synonym mit “freier Software“ (nicht Freeware) verwendet \cite{suchen}. Dabei folgt die Open Source-Entwicklung der Maxime, dass die Kernsteuerungsinformationen und -befehle (Quelltext) von Software öffentlich einsehbar und zugänglich, sowie je nach gewähltem Lizenzmodell modifiziert, kopiert oder weitergegeben werden können \cite{suchen}.

Bei der Open Source-Entwicklung veröffentlichen Programmierer den Code einer Software offen im Internet. Andere Programmierer haben die Möglichkeit diesen Code so weiterzuentwicklen und anzupassen, wie es ihnen beliebt. Dadurch entsteht ein offenes Ökosystem an Software, bei dem nicht mehr der Zugriff die Hürde darstellt sondern die Adaption oder der Einsatz der vorhandenen Lösungen.

Die Entwicklungsmethode unterscheidet zwischen Open Source-Software und dem traditionellen Modell des geistigen Eigentums bei der Entwicklung von Software mit der Feststellung, dass Open-Source-Software das Prinzip der Exklusivität des geistigen Eigentums auf den Kopf stellt, weil diese Software "um das Recht auf Vertrieb konfiguriert, nicht auszuschließen ist" \cite{suchen}. Auch wenn letztendlich noch immer unklar ist, ob Open Source Software wirklich "schneller, besser oder günstiger" ist, gibt es wenig Zweifel daran, dass sich Open Source in den letzten Jahren stark verbreitet hat \cite{Lerner_2001}.

Die Definition von Open Source beinhaltet festgelegte Kriterien für die Klassifizierung von Open Source Produkten \cite{suchen}: Freie Weitergabe ohne zuästliche Kosten, das Programm muss den Quellcode beinhalten und den Code offen zur Verfügung stellen, die verwendete Lizenz muss Derivate zulassen, die Unversehrtheit des Quellcodes des Autors muss garantiert werden, die Diskriminierung von Personen oder Gruppen muss ausgeschlossen sein, es darf keine Enschränkung des Einsatzfeldes geben, die Lizenz muss weitergegeben werden können und auf das Produktpaket anwendbar sein und die Lizenz darf die Weitergabe des Programmcodes zusammen mit anderer Software nicht einschränken.

Im Vergleich zum klassischen Softwareentwicklungsprozess gelten folgende charakteristische Merkmale \cite{suchen}:
\begin{enumerate}
\item “Anzahl der beteiligten Entwickler: Im Vergleich zu traditioneller Softwareentwicklung ist eine weitaus größere Anzahl von Entwicklern beteiligt. Es gibt es keine klare Grenze zwischen Entwicklern und Anwendern, da die Hürden für eine Partizipation im Entwicklungsprozess sehr gering sind. Auch wenn ein großer Teil der Entwicklungsarbeit von Freiwilligen übernommen wird, gibt es dennoch den Trend zum Einsatz bezahlter Entwickler.
\item Zuteilung der Arbeit: Im Open Source Programming (OSP) wird die Entwicklungsarbeit nicht länger von einer definierten Instanz zugeteilt, sondern die Teilnehmer wählen ihre Arbeitspakete selbst aus.
\item Architektur: In der Regel orientierten sich die Teilnehmer eines OSP nicht an einer vorgegebenen System-Architektur. Die Gestaltung der Architektur geschieht dezentral und ist oftmals häufigen Richtungswechseln unterworfen.
\item Koordination: Es gibt wenig oder keine institutionalisierten traditionellen Koordinationsmechanismen, wie z.B. Projekt- und Zeitpläne, Lasten- und Pflichtenhefte u.ä.” \cite{suchen}
\end{enumerate}

--- TODO: konkretisieren und neu ausarbeiten ----

Die Verknüpfung der Open-Source Entwicklungsmethode mit dem Wissenschaftsprozess wurde zuerst von dem Literaturwissenschaftler und Medientheoretiker Friedrich Kittler manifestiert \cite{suchen}. Open Source Entwicklungsprozesse unterscheiden sich von den klassisch-traditionellen (closed-source) Softwareentwicklungsprozessen insbesondere dadurch, dass sie jederzeit öffentlich einsehbar und transparent nachvollziehbar sind und zeigt diesbezüglich mit der Öffnung wissenschaftlicher Kommunikation Konvergenzen, als dass es dabei nicht nur den freien und offenen Zugang zum finalen Ergebnis betrifft, sondern auch den Zugriff auf den gesammten Prozess zur Erlangung der Informationen sowie die Daten offenlegt und den Erstellungsprozess transparent nachvollziehbar macht \cite{kelty_2004}. Adaptiert man den Open-Source-Prozess auf wissenschaftliche Wertschöpfungsprozesse und definiert in diesem Zusammenhang wissenschaftliche Publikationen als Quellcode, ist das Konzept auf die wissenschaftliche Kommunikation übertragbar \cite{Singh_2008} \cite{Bradley_2008} \cite{Bradley_2007} \cite{Willinsky_2005}.

Die Annahme, dass das System der offenen Softwareentwicklung dem System der Wissensproduktion in der Wissenschaft ähnelt, beruht unter anderem auch auf der Parallele, dass auch bei der Wissensproduktion neues Wissen auf der Grundlage von bereits vorhandenem (und verfügbaren) Wissen ensteht. Darüber hinaus wird die Verknüpfung der Debatte um Open Source mit der um die Öffnung von Wissenschaft wird auch als sinnvoll erachtet, weil Ähnlichkeiten bei der Motivation der Ersteller von Software und die möglichen Motivation von Wissenschaftlern zu Publizieren vermutet werden können.

Folgende weitere Annahmen spielen für diese Betrachtung von Open Source und Offenheit in Wissenschaft und Forschung eine Rolle:
\begin{enumerate}
\item Wie bei der wissenschaftlichen Kommunikation, geht es bei der Mitarbeit an Open Source Projekten nicht ausschließlich um altruistische Motive \cite{Lerner_2001}.
\item Die Kontributoren von Open Source Projekten versprechen sich neue "Karrieremöglichkeiten oder eine Ego-Genugtuung" \cite{Lerner_2001}, Selbstverwirklichung oder Befriedigung der intellektuellen Neugier \cite{Willinsky_2005}, sowie gegenseitige Beurteilung und Annerkennung (non-monetäres Kapital).
\item "Free Software (im Sinne von Open Source), Open Access und Creative Commons sind alles Rechts- und Infrastrukturexperimente"\cite{kelty_2004}. Open Source-Software sollte dabei nicht mit "Shareware" verglichen werden, die zwar kostenlos verbreitet wird, aber deren Quellcode proprietär bleibt \cite{Lerner_2001}
\item Der Stand der Forderung nach Öffnung der wissenschaftlichen Kommunikation kann aus technologisch-entwicklungsmethodischer Sicht mit der Debatte um kostenloser Software (Freeware) versus Open Source verglichen werden. Der Vergleich: Freeware und Open Access Publikationen sind zwar kostenlos verfügbar, ihr Erstellungprozess wird jedoch nicht offen und transparent kommuniziert.
\end{enumerate}
--- Todo weiter ausarbeiten ---

Dieser Vergleich der Öffnung von Wissenschaft mit der Open-Source Bewegung deutet somit ein mögliches Szenario an \cite{Kuhlen_2002_universalaccess}, wie in Zukunft die Wissensproduktion frei und öffentlich gestaltet werden kann.

\subsection{Die Forderung nach Öffnung der wissenschaftlichen Kommunikation}

Wie bereits dargestellt, hängt der Fortschritt der Wissenschaft maßgeblich von dem freien Austausch und der Verbreitung von Informationen ab \cite{cite:11}. Das System der wissenschaftlichen Kommunikation, das in der derzeitigen Form seit mehreren hundert Jahren besteht, basiert auf der Forschung, der Begutachtung, dem Druck, der Kommunikation der Ergebnisse in wissenschaftlichen Publikationen, der Verbreitung sowie dem Verkauf an Bibliotheken und andere wissenschaftliche Institutionen \cite{cite:11a} und dem anschließenden Diskurs in der wissenschaftlichen Fachöffentlichkeit \cite{suchen}.

Infolge des weltweit steigenden Haushaltsdrucks der Bibliotheken und wissenschaftlichen Insitutionen, des “ungewöhnlichen Geschäftsmodells” \cite{cite:12} der Wissenschaftsverlage mit immer höheren Margen \cite{albert_2006_open_implications} und des Umstandes, dass private Wissenschaftsverlage durch das wissenschaftlichen Reputationssystem über öffentlich finanzierte Wissenschaftlerkarrieren entscheiden \cite{heise_2012}, befindet sich das wissenschaftliche Kommunikationssystem in einer Krise \cite{cite:14}.

Durch die Digitalisierung kann die Öffnung der wissenschaftlichen Kommunikation als eine mögliche Antwort auf diese Krise verstanden werden und setzt bei der Öffnung (Open) und dem freien Zugang (Access) zu wissenschaftlichen Publikationen an und könnte perspektivisch zu einer Öffnung (Open) des Zugriffs auf den Prozess des Forschens (Science) führen.

\subsubsection{Offener Zugang zur wissenschaftlichen Publikation}

\begin{quote}
Der offene Zugang, auch Open Access, bedeutet, dass Peer-Review-Fachliteratur kostenfrei und öffentlich im Internet zugänglich sein sollte, so dass Interessenten die Volltexte lesen, herunterladen, kopieren, verteilen, drucken, in ihnen suchen, auf sie verweisen und sie auch sonst auf jede denkbare legale Weise benutzen können, ohne finanzielle, gesetzliche oder technische Barrieren jenseits von denen, die mit dem Internet-Zugang selbst verbunden sind. In allen Fragen des Wiederabdrucks und der Verteilung und in allen Fragen des Copyrights sollte die einzige Einschränkung darin bestehen, den Autoren Kontrolle über ihre Arbeit zu belassen und deren Recht zu sichern, dass ihre Arbeit angemessen anerkannt und zitiert wird.
\cite{boai_2012}
\end{quote}

Der offene Zugang zu wissenschaftlicher Kommunikation ist seit der Entwicklung des gedruckten Wortes eng mit der Frage nach Urheberrechten für wissenschaftliche Informationen verknüpft \cite{Case_2000}. Open Access beschreibt ein wissenschaftliches Kommunikationssystem, in dem der Zugang zu den unterschiedlichsten Formen wissenschaftlicher Publikationen, im Gegensatz zum bestehenden System, unter freien, kostenlosen Bedingungen und ohne finanzielle, gesetzliche oder technische Hürden möglich ist \cite{WD_bundestag_2009}. Dieses System ermöglicht darüber hinaus ein "alternatives Geschäftsmodell"\cite{lewis_2012_inevitability} für wissenschaftliche Publikationen. Was auf Maßgabe beruht, dass der Autor die "Eigentumsrechte an den Artikeln, die bisher für die Publikation in wissenschaftlichen Journals an die jeweiligen Fachverlage abgetreten wurden, (...) nun bei den Autoren der Artikel selbst verbleiben" \cite{Hess_2006}.

"Geringere Kostenbarrieren und damit eine einfachere Verbreitung ihrer eigenen Werke" \cite{WD_bundestag_2009} stellen dabei die Wünsche der wissenschaftlichen Autoren und Urheber an Open Access dar. Der Einsatz (offener) Lizenzen ist dafür ein weiterer Haupteinflussfaktor \cite{cite:16}.

\subsubsection{Offener Zugriff auf den wissenschaftlichen Prozess}

Die Entwicklung von “Open Science” knüpft an die Entwicklung der Open Access-Bewegung an und kann als Folge der neuen Möglichkeiten für kollaboratives Arbeiten im Rahmen der Digitalisierung und neuer Kommunikationstechniken verstanden werden. Open Science wird im Folgenden darüber defniniert, wie der gesamte wissenschaftliche Wertschöpfungsprozess der Allgemeinheit zur Verfügung gestellt werden kann.

Der Sammelbegriff Open Science erstreckt sich über die gesamte wissenschaftliche Wertschöpfungskette \cite{Scheliga_2014}: Vom offenen Zugang zu Publikationen wissenschaftlicher Forschung (Open Access), sowie den ganzheitlichen wissenschaftlichen Erkenntnisprozess. Unter diesem Gesichtspunkt kann Open Science als eine Weiterentwicklung von Open Access bezeichnet werden. Die diesbezügliche Evolution des Konzepts von Open Access führt zu einem direkten und breiten Weg, Wissenschaft an jedem Schritt der wissenschaftlichen Wertschöpfungskette zu kommunizieren und zu transferieren. Open Science ist die Reaktion auf die Forderung nach offenem Zugriff auf Wissenschaft und Forschung und kann dazu führen, "dass sich die Bedeutung von Forschungsergebnissen zukünftig nicht mehr auf sogenannte klassische wissenschaftliche Publikationen (im Format von Einleitung – Methoden – Ergebnisse – Diskussion), sondern die globale Echtzeitpublikation von Originaldaten stützen wird" \cite{Stengel_2013}.

Wie Open Access hat die Bewegung für Open Science ihre Dynamik der zunehmenden Verbreitung des Internets Anfang der 1990er zu verdanken \cite{Lievrouw_2010} und der neuen Möglichkeiten des kollaborativen Arbeitens sowie des Teilens von Daten und Informationen über das globale Netzwerk \cite{Meyer_2013}. Diese technologischen Entwicklungen ermöglichten jedoch nicht nur das kollaborative Arbeiten zwischen Wissenschaftlern in aller Welt, sondern schafften auch die Möglichkeit die ausgetauschten Informationen nicht nur unter Wissenschaftlern zu teilen, sondern die Verbreitung wissenschaftlicher Informationen an die Gesamtgesellschaft. Befürworter von Open Science sehen hier eine Möglichkeit die gesamten wissenschaftlichen Prozesse, von der Idee bis zur Abschlusspublikation, transparenter, effizienter, nachvollziehbarer und offener zu gestalten.

Diese Vision einer offenen Wissenschaft steht der Verschlüsselungs- und Patentwut zur Wahrung der Geschlossenheit der wissenschaftlichen Informationen und eines möglichen kommerziellen Vorteils durch Wissenschaft im Rahmen öffentlich-finanzierter Forschung gegenüber und führt zu einer Debatte über die Verfügbarkeit der wissenschaftlichen Arbeit und die Entlohnung der "Erfinder" im wissenschaftlichen System \cite{suchen}.

Insbesondere die Entwicklung der Tradition für eine "offenen Wissenschaft" im siebzehnten Jahrhundert bietet einen ersten Ansatzpunkt zur Erforschung der Entwicklung von Open Science \cite{Scheliga_2014}, da dieser historische Übergang noch nicht erforscht ist \cite{CREATe_2014}.

\subsection{Chronologie der Forderung nach Öffnung der wissenschaftlichen Kommunikation}
Für das erweiterte Verständnis der Prozesse, die zu der Öffnung von Wissenschaft und Forschung führen, ist eine historische Betrachtung der Entwicklung wissenschaftlicher Kommunikation, sowie der Forderung nach Offenheit in Wissenschaft und Forschung unabdingbar. Angeleht an die Arbeiten des kanadischen Philosophen McLuhan und des Germanisten Wenzel können werden drei bedeutende Umbrüche der Medienentwicklung im Rahmen der Kommunikation von Wissen \cite{wunderlich_2008_buchdruck} genannt werden  \cite{wenzel_mediengeschichte_2007}:
\begin{enumerate}
\item den Übergang vom Körpergedächtnis (brain memory) zum Schriftgedächtnis (script memory)
\item den Übergang von der Handschriftenkultur zur Druckkultur (print memory)
\item und den Übergang vom Buch zum Bildschirm (electronic memory)
\end{enumerate}

\subsubsection{Die Einführung des Buchdrucks als Grundlagen für die moderne Wissenschaft}

In pre-modernen Zivilisationen sind Wissen und Informationen als nicht besitzbare Ware angesehen worden\cite{cite:18} \cite{steiner_1998_autorenhonorar}. Der Wissensaustausch im Vergleich zu den heutigen Möglichkeiten war stark beschränkt \cite{cite:17c}. "In vorwissenschaftlichen Gesellschaften gibt es keine scharfe Grenze zwischen dem vorhandenen und dem aktuell benutzten Wissen"\cite{Luhmann1998}. Die vormoderne Wissenschaft bezog sich unmittelbar auf die täglichen Bedürfnissen. Sie stellte sich die Aufgabe "das Wissen zu verbessern und vor allem zu erhalten und zu tradieren" \cite{Luhmann1998}. Die Produktion von Literatur beschränkte sich zu dieser Zeit auf "auf die Überlieferung und Kommentierung des althergebrachten Wissens, insbesondere des theologischen" Wissens \cite{steiner_1998_autorenhonorar}.

Die Einführung des Buchdrucks führte zur Veränderung der Aufgabe von Wissenschaft, sowie ihrer Unvermitteltheit der Orientierung auf den täglichen Bedarf \cite{Luhmann1998}. Die Geschichte des gedruckten Buchs beginnt mit Johannes Gensfleisch, auch Gutenberg gennant, Beiträgen zur Buchdruckerkunst \cite{wittmann_1999_geschichte}. Durch die Entwicklung des Buchdrucks hat das Selbstverständnis der europäischen Kultur durch die neue Möglichkeiten der Vervielfältigbarkeit und Massenverbreitung in bis dahin unbekannter \cite{giesecke_1991_buchdruck} und revolutionärer Weise verändert \cite{wunderlich_2008_buchdruck}. Der Buchdruck stellte die "Grundlagen und Meilensteine sowohl für die Kommunikation der Menschheit insgesamt als auch für den wissenschaftlichen Gedankenaustausch im Besonderen dar" \cite{schirmbacher_2009_wisspub}, er war ein "Bestandteil des Übergangs vom Mittelalter in die frühe Neuzeit" \cite{lange2008medienwettbewerb} und leitete die "Moderne" ein \cite{luhmann_1997_gesellschaft}.

--- TODO: "Moderne" Zitat prüfen wenn es nicht stimmt weg, weil angreifbar ----

Die Buchdrucktechnologie führte auch zu einem bis dahin unbekannten, explodierenden Informationsangebot. Es entwickelte sich eine neue Denkstruktur \cite{eisenstein_1997_druckerpresse}, in der das " mitteralterliche Denken in Bildern und Methaphern" von der "wissensschaftlich-systematischen Methodik" abgelöst wurde \cite{wunderlich_2008_buchdruck}. Sie führte zur Befreiung des Autors aus der weitgehenden Anonymität mitteralterlicher Manuskriptkultur und zur Entkopplung von "Herstellung und Verbreitung vom singulären Interesse eines Autors, Kopisten oder Auftraggebers"\cite{wunderlich_2008_buchdruck} \cite{schirmbacher_2009_wisspub}.

In der Zeit der Verbreitung des Buchdrucks in der Wissenschaft und Forschung kam es zu einer Vielzahl sogenannter Prioritätsstreits \cite{schirmbacher_2009_wisspub}, bei denen bereits vorhandene wissenschaftliche Erkenntnisse nicht öffentlich verbreitet worden waren und deshalb kein für alle nachvollziehbarer Bezug zum Entdecker hergestellt werden konnte. Als Beispielhaft für einen solchen Prioritätsstreit kann die Auseinandersetzung zwischen Isaac Newton und Gottfried Wilhelm Leibniz um eine Veröffentlichung zur Fluxionsrechnung im 17. Jahrhundert durch Newton, genannt werden. Leibniz rezensierte diese Veröffentlichung anonym und stellte sich selbst namentlich als Erfinder dieser dar \cite{2013_leibniz}. Leibniz wurde in diesem Zusammenhang durch die Royal Society des Plagiats für schuldig befunden, weil er seine bereits deutlich länger vorhandenen Erkenntnisse nicht veröffentlicht hatte" \cite{schirmbacher_2009_wisspub}.

Der Buchdruck, wie auch die ersten wissenschaftlichen Zeitungen, wurden für die wissenschaftlichen Autoren somit nicht nur zu einem neuen "Kommunikationsinstrument", einem Instrument zur "Erlangungvon von Reputation" oder zu einem Instrument "zur Generierung finanzieller Erträge" sondern auch zu einem "Nachweisinstrument" \cite{wunderlich_2008_buchdruck} \cite{schirmbacher_2009_wisspub} zur Vermeidung solcher Prioritätskonflikte.

Die Entwicklung eines "freien Marktes als Vertriebsnetz für typographische Informationen"\cite{giesecke_1991_buchdruck} und die "Kapitalisierung der Buchproduktion" \cite{steiner_1998_autorenhonorar} im 16. Jahrhundert für das gedruckte Wort führte langfristig zu einem "Verlust an Macht und Herrschaft über das geschriebene Wort" \cite{wunderlich_2008_buchdruck}. Gutenbergs Druckinnovation ermöglichte als sogenannte "Schlüsseltechnologie" \cite{jager_1993_theoretische} eine neue Dimension der Wissensverbreitung. Anfangs handelte es sich bei der Technologie nur um ein "elitäres und teures Medium für die gebildete Klasse" \cite{hartmann_2008_medien}. Sie führte weder von Beginn an zum zeitlich unmittelbaren Zugang zu Wissen noch war sie sofort  für die Allgemeinheit zugänglich \cite{hartmann_2008_medien}. Leibniz forderte deshalb, dass Werke ohne Rücksicht auf Profitgier erscheinen sollten und appellierte an eine "obrigkeitliche Lenkung", damit der Buchhandel "seiner Aufgabe der Verbreitung von nützlichem Wissen gerecht würde" \cite{wittmann_1999_geschichte}.

Vor allem kirchliche Instanzen waren über eine "wachsende theologiesche Begriffsverwirrung" und die Verbreitung der Schriften in Volkssprachen besorgt \cite{giesecke_1991_buchdruck}. Sie stellten die größten Gruppe an Kritikern des Buchdrucks dar und versuchten die neue "Bücherflut" zu unterbinden\cite{giesecke_1991_buchdruck}. Die Einführung des Buchrdurcks "gab der Zensur eine neue Bedeutung" \cite{sprachgeschichte_1998_besch}, sie entwickelte sich als "prohibitives Instrument für die Überwachung der Lektüren und zur Eindämung" von unerwünschter Literatur \cite{suchen} und als "Kampfmittel" \cite{sprachgeschichte_1998_besch}. Der Kommunikations- und Medientheoretiker Michael Giesecke zitiert aus einem Gutachten dieser Zeit: "In den Anfängen muß man Widerstand (gegen das Übel des Drucks von Büchern, die aus den heiligen Schriften in die Volkssprache übersetzt sind), damit nicht durch die Vermehrung der deutschsprachigen Bücher der Funke des Irrtums endlich sich zu einem großen Feuer entwickle" \cite{giesecke_1991_buchdruck}.

Zusammenfassend nennt Giesecke grundlegenden Einwände gegen den Buchdruck als "freie" Kunst \cite{giesecke_1991_buchdruck}:
\begin{itemize}
\item Missbrauch - Die Einführung des Buchdrucks wurden von vielen Warnungen vor Mißbrauch der Technologie begleitet \cite{lange2008medienwettbewerb}: antireligöser Missbrauch durch die Verbreitung gefährlichen Gedankenguts \cite{kruse2003multimedia}, bewusste Falschinformation und Verfälschung von Inhalten \cite{sprachgeschichte_1998_besch},  willkürliche Informationsverbreitung über Bücher, ohne Zustimmung der geistlichen und weltlichen Regenten \cite{rother_2002_siebenbuergen} und die Angst der Traditionalisten, die ihre Herschaft durch das Monopol auf die Interpretaion der Bibel gefährdet sahen \cite{lange2008medienwettbewerb}.
\item Die Qualität und Reinhaltung der der besten Texterzeugnisse \cite{giesecke_1991_buchdruck} stellte ein weiterer Kritikpunkt am Buchdruck dar.
\item Auch die Nachlässigkeit und Unachtsamkeit von Buchdruckern und Setzern wurde früh kritisiert. Sie spielten im Buchdruckprozess eine entscheidende Rolle, da sie großen Einfluss auf die Qualität der Nachdrucke hatten. Nachlässigkeit oder ungenaues Arbeiten führten zu erheblichen Verschlimmerungen, was von Autoren wie Martin Luther schon früh beklagt wurde \cite{sprachgeschichte_1998_besch}.
\item Die Mulitplikation von Fehlern, da in den gedruckten Exemplaren auch die Fehler völlig Übereinstimmen und nicht behohben werden können, schließt an die Kritik der Qualität der gedruckten Bücher an. Die Berüfchtung begründete auf der Irreversibilität der Verbreitung fehlerhafter Inhalte beim Buchdruck, die bei der geringeren Anzahl handschriftlichen Kopien bisher weniger Einfluss hatte.
\item Vor allem bei den staatlichen und geistigen Organen machte sich die Kritik der Demokratisierung der Vervielfältigung und Verbreitung als Grund für die Verwirrung der "Laien" (der Glaubensgemeinschaft) breit \cite{giesecke_1991_buchdruck}.
\item Demzufolge befürchtete die Obrigkeit, die Auflösung der ständischen Ordnung da der "Zugang zu den Speichern des Wissens nicht länger bestimmten Schichten vorbehalten bleibt" und das "Schreiben und Lesen wird von einer ständischen zu einer gemeinen Tätigkeit". Aus heutiger Sicht mag diese Sicht auf Grund der sehr geringen Alphabetisierungsrate und der noch immer sehr geringen Anzahl an Büchern Ende des 15. Jahrhunderts als unbegründet erscheinen, dennoch wurde die soziale Umwälzung durch den Buchdruck beschleunigt und unumkehrbar gemacht. \cite{giesecke_1991_buchdruck}
\item Auflösung des "Amts" des Bücherschreibers als eigenes Handwerk
\item Die Angst vor Überfluss an Büchern sowie das Chaos an Büchern stellte eine weitere Befürchtung bei der Etablierung des Buchdrucks dar. Sogar physische Konsequenzen werden vermutet: "Augen schmerzen, vom Lesen, unsere Finger vom Blättern" \cite{giesecke_1991_buchdruck}
\item Zudem gab es auch "psychische Bedenken", so gab es im 15. Jahrhundert die Angst vor dem Anhäufen von Informationen. Sie galt im Mittelalter als "gefährliches und verwirrendes Unterfangen" und fürte zu Annahmen wie "je gelehrter, je verkehrter". \cite{giesecke_1991_buchdruck}
\end{itemize}

"Wer für den Druck schreibt, gibt die Situationskontrolle auf" und produziert für das Gedächtnis des Systems" bei dem weder "Kommunikationsvorgang" noch der "Wissenszuwachs" abgeschlossen sind \cite{Luhmann1998}. Vor der Einführung des Buchdrucks wurde vorab entschieden, was veröffentlicht und verbreitet wird und es gab Instanzen, die die Weitergabe von Wissen (meist Auftragsarbeiten) organisierten. Der Buchdruck kehrte das System um, da nun Texte verbreitet wurden und man es dem "Markt und dem nachträglichen Meinungsstreit überließ, welche Information zum Gemeingut wurden" \cite{giesecke_1991_buchdruck}. Die Etablierung des Drucks führte, zunächst "unbemerkt und naturwüchsig", zu einer Veränderung der Sozialisierung von Informationen, der Veröffentlichung \cite{giesecke_1991_buchdruck}. "Das Medium der Schrift" unter Buchdruckbedingungen wurde "als eine Verbreitungstechnologie genutzt, welche die unmittelbare Interaktion zwischen Sender und Empfänger (weiterhin) ausschließt". Soll der Text "Wissen werden (...) muss er Leser finden" \cite{Luhmann1998}. "Die Einführung des Buchdrucks stellt damit einen Bestandteil des Übergangs vom Mittelalter in die frühe Neuzeit dar"\cite{lange2008medienwettbewerb}. Schon in der frühen Neuzeit konnte so ein Zusammenhang zwischen Buchdruck und demokratischen Freiheiten "sowohl faktisch als auch ideologisch" \cite{suchen} hergestellt werden. Im Gegensatz zum Mittelalter, in dem jede breitere Sozialisierung privater Gedanken "Legitimationsbedürftig" war, bedurfte jetzt jeder Eingriff in die "Freiheit, Meinungen oder Informationen" zu drucken einer politischen Legitimation \cite{giesecke_1991_buchdruck}. Der Buchdruck kann im Rahmen der "fundamentalen Umbrüche in Politik und Verwaltung, Ökonomie und Handel, Religion, Bildung und nicht zuletzt in den Prozessen der kognitiven Welterkenntnis" \cite{pscheida_2010_wikipedia} als "Katalysator des kulturellen Wandels"\cite{giesecke_1991_buchdruck} verstanden werden.

Um den Arbeitsaufwand der Drucker zu honorieren und die verlegerischen Leistung zu würdigen\cite{szilagyi_2011_leistungsschutzrecht}, wurden mit der Entstehung des Druckerwesens auch erste Privilegien vergeben \cite{gieseke_1995_privileg}, die es ihnen zum Beispiel erlaubte, die Buchdruckkunst für einen bestimmten Zeitraum allein oder in einem bestimmten Gebiet auszuüben \cite{martin2008publizistische} \cite{koller_1995_Urheberrecht}. Privilegien ermöglichen dem Begünstigen Sonderberechtigungen oder Rechte gegenüber den allgemeinen Rechtsregeln \cite{jaenich_2002_geistiges}. Im Zuge der Verbreitung der Drucktechnologie und des steigenden Wettbewerbs kamm es auch zu Privilegien für Urheber und Erstverleger, die sich damit versuchten gegen das Nachdrucken zu erwehren. Die erfolgreiche Einforderung dieser Privilegien führte schon früh zu einer Art Monopolstellung  und zu einem generellen Nachdruckverbot \cite{szilagyi_2011_leistungsschutzrecht}.

---- TODO: Entwicklung Urheberrecht? ----

\subsubsection{Wissenschaftliche Journale als Medium der wissenschaftlichen Kommunikation}

Noch zu Beginn des 17. Jahrhunderts stellte das Schreiben von Briefen oder das Buch die häufigste Form des wissenschaftlichen Austauschs dar \cite{porter_1964_scientific}. Der Brief, als besonders exklusive Form der Kommunikatiin stand dem Buch als zeitaufwändige Form gegenüber \cite{fecher_hiig_2014}. Erst Mitte des 17. Jahrhunderts kam es in Folge der Gründung der "Royal Society" of London zu einer wissenschaftlichen Diskussion über die Etablierung einer neuen Philosophie für die Förderung von Wissen, den Wunsch nach Veränderung bei der Verbreitung wissenschaftlicher Erkenntnisse und den Wunsch eine "wissenschaftlichen Revolution" voranzutreiben \cite{Dear_1985}. Als ein Ergebnis der 1640 gegründeten Vereinigung erschien am 6. März 1665 mit "Philosophical Transactions" eine der ersten wissenschaftliche Fachzeitrschiften \cite{suchen}. Im gleichen Jahr entstand erschien das wissenschaftliche "Journal des sçavans" in Frankreich. Bis des 17. Jahrhunderts folgten ca. 30 Journals diesen Beispielen. Die Journals unterschieden sich in Ihrer Struktur stark von denen heute. Sie enthielten im Vergleich zu den heutigen Fachzeitschriften jeweils eine nur sehr eng umschriebene Anzahl von Beiträgen \cite{suchen} und waren wie wissenschaftlichen Briefe (meist in der Ich-Form) verfasst, die Wissenschaftler vor der Entwicklung des Journals direkt aneinander verschickt hatten \cite{suchen}.

--- Todo: Im 17. Jahrhundert, ja noch im 19., wurden Entdeckungen manchmal in Form eines Anagramms (Buchstabenversetzrätsel, G.F.) bekannt gemacht - so etwa Galileis ‘Dreistern’ Saturn und Hookes Elastizitätsgesetz -, um gleichzeitig die Priorität zu sichern und die Rivalen nicht auf Fährten zu locken, ehe der Gedanke weiter ausgebaut war. <- siehe: http://sammelpunkt.philo.at:8080/2435/1/Optimale_Informationsvorenthaltung.pdf---

Die wissenschaftliche Fachzeitschrift oder das wissenschaftliche Journal, wie wir es heute kennen, geht strukturell auf das 19. Jahrhundert zurück, als die wachsende Aktivität und das öffentliche Interesse an der Wissenschaft anstieg. Zu dieser Zeit gab es die meisten Gründungen der großen Fachzeitrschriften von heute \cite{porter_1964_scientific}. Dabei fand über die Jahrhunderte ein rasanter Anstieg der wissenschaftlichen Journale statt. In 1961 wurde die erste quantitative Studie an Hand der Anzahl von wissenschaftlichen Journalen durchgeführt \cite{de_1982_little}.

--- Todo: weiter ausarbeiten ---

\subsubsection{Die Rolle der Verlage, die Zeitschriftenkrise und erste Experimente mit dem offenen Zugang zu wissenschaftlichen Publikationen}

Mitte des 20. Jahrhunderts entdeckten Verlage die kommerziellen Möglichkeiten der Distribution wissenschaftlicher Inhalte. \cite{suchen}

Die ersten Experimente mit offenem Zugang und freien Lizenzen für Publikationen in der Wissenschaft findet man in den 1960er Jahren und somit schon vor der Zeit der Erfindung des Internets, zurück \cite{cite:18b}. Noch bevor die digitalen Nutzungsmöglichkeiten verfügbar waren und bevor an das "globalen Dorf"\cite{mcluhan_1962_gutenberg} zu denken war, wurde vor allem in den Technik- und Naturwissenschaften eine “pre-print Kultur” entwickelt bei der die Autoren ihre zur Begutachtung eingereichten Artikel zeitgleich unter Kollegen (über den Postweg) zirkulieren ließen, um den Kommunikationsprozess zu beschleunigen \cite{suchen-Hoffmann-Zugang-undVerwertung-öffentlicher-Informationen}.

In Deutschland nahmen bis Anfang der 1990er Jahre die wissenschaftlichen Verlage eine marktbeherrschende Stellung ein und waren exklusiver Distributor bei der Veröffentlichung wissenschaftlicher Informationen \cite{schloegl_2005} \cite{offhaus_2012_institutionelle_repos}. Erst das Medium Internet eröffnete "Nutzungsmöglichkeiten, durch welche die Schrift als ein Medium einsetzbar wird, das den permanenten Wechsel zwischen Sender- und Empfängerposition ähnlich flexibel zu gestalten erlaubt, wie es im gesprochenen Gespräch der Fall ist" \cite{sandbothe_2000_pragmatische}. Als weiteres Veränderung in Abgrenzung zur Technologie Buchdrucks revidierte das Internet "die Vorstellung von einem geschlossenen Sinngehalt" \cite{sandbothe_2000_pragmatische} mit einem Anfang und Ende wie zum Beispiel in einem Buch. Diese Entwicklung basiert auf dem in der Welt des geistigen Eigentums ungewöhnlichen Umstand, dass seit dem Beginn des wissenschaftlichen Journals im Jahr 1665, wissenschaftliche Autoren nicht vordergründig finanzieller Belohnung profitierten, sondern maßgeblich durch die weite Verbreitung und Hinweise auf ihre Arbeit, sowie die wissenenschaftlichen Erkenntnisse ihrer Forschung \cite{albert_2006_open_implications}. Darüber hinaus ist es eine Besonderheit des Systems, dass Wissenschaftler sowohl Produzenten als auch Konsumenten der Wissenschaftskommunikation sind und damit Ihre eigene Zielgruppe darstellen \cite{Hess_2006}.

Im Laufe der Zeit erlangten Verlage in diesem System eine Vormachtstellung im wissenschaftlichen Publikations- und Distributionssystem. Diese stüzt sich bis heute auf drei Säulen \cite{offhaus_2012_institutionelle_repos} \cite{bargheer_2006_open}:
\begin{enumerate}
\item "Urheberrecht, wonach Verlage [...] weitgehende Ansprüche an dem veröffentlichten Werk erwerben“;
\item "redaktionelle Themenbündelung (bundling)“;
\item Organisation der "Qualitätssicherung durch Begutachtung (Peer Review)"
\end{enumerate}

Die marktbeehrschende Stellung der Verlage führte zu einer Situation, in der die Verlage die Preise weitgehend diktieren und Preiserhöhungen unlimitiert durchgesetzt werden konnten. Als Folge der ungebremsten Ausnutzung dieser Marktmacht kam kurz vor der Jahrtausendwende zur sogenannten "Zeitschriftenkrise" \cite{schirmbacher_2009_wisspub} \cite{muller_2010_open}. Die Zeitschriftenkrise, "die richtigerweise Zeitschriftenpreiskrise oder Zeitschriftenpreisexplosion genannt werden müsste"\cite {Brintzinger_2010}, kam als Begriff das erste Mal in den 1990er Jahren auf \cite{Boni_2010}. Die Krise war das Ergbnis folgender Entwicklungen auf der Angebots- und Nachfragenseite \cite{Brintzinger_2010}: Auf der Angebotsseite wurden durch einen "Konzentrationsprozess" "innerhalb von etwas mehr als einem Jahrzehnt im Bereich der STM-Zeitschriften mittelständische Verlage nahezu vollkommen durch internationale Kapitalgesellschaften substituiert". \cite{Brintzinger_2010} Gestützt von der Nachfrageseite resultierte daraus eine "monopolistische Preispolitik" der Verlage \cite{Brintzinger_2010}. Ein zeitgleicher Anstieg der Titelvielfalt, bei der aus "einer mehr generalistischen Zeitschrift drei oder vier Spezialzeitschriften" entstanden, "die dann allesamt wieder von Bibliotheken abonniert werden mussten"\cite{Brintzinger_2010}, verschärfte das Problem. Eine weitere Ursache für die krisenhafte Zuspitzung bestand in der institutionellen Organisation der Literaturbeschaffung an den Hochschulen. Bei der Arbeitsteilung von Bibliothekaren und Wissenschaftlern war und ist es für das Ansehen des einzelnen Faches durchaus rational, mit einem möglichst hohen Anteil am Gesamtetat der Bibliothek zu partizipieren. Folglich gibt es für individuelle Einsparungen von allen Seiten nur wenig Anlass. \cite{Brintzinger_2010}.

Die Preisexplosion konnte auch durch die Bildung von Bibliothekskonsortien, "deren Aufgabe es war, für Bibliotheken kostengünstige Rahmenbedingungen auszuhandeln", nicht gebändigt werden \cite{Fladung_2003} \cite{Brintzinger_2010}. Gleichzeitig standen die Wissenschaftler unter einem starken Publikationszwang, der mit "Publish or Perish" \cite{CLAPHAM_2005} beziehungsweise "impact factor fever" \cite{Cherubini_2008} und "impact factor race" \cite{Brischoux_2009} beschrieben wurde \cite{offhaus_2012_institutionelle_repos}.

"Publish or Perish" beschreibt das Problem, dass im Rahmen der "wachsenden Konkurrenz um Forschungsförderung und akademische Positionen (...) kombiniert mit dem zunehmenden Einsatz bibliometrischer Parameter für Evaluation" \cite{Fanelli_2010} junge Akademiker viel (mit positiven Ergbnissen) publizieren müssen um Anerkennung und gegebenenfalls eine Anstellung im Wissenschaftsbetrieb zu erreichen \cite{pscheida_2010_wikipedia} \cite{Beasley_2005}. Das erzeugte viel "nutzlose Forschung und Artikel"\cite{smith1990killing}.

Die "Zeitschriftenkrise" und der gestiegene Publikationsdruck stellen zwei fundamentale Aspekte für das Aufkommen der Forderungen nach "Open Access" dar \cite{Brintzinger_2010} \cite{suchen} . Als Reaktion darauf und auf der Grundlage der ersten Früchte der Digitalisierung gründete Anfang der 1990er der Physiker Paul Ginsparg mit arXiv den ersten wissenschaftliche Preprint-Dienst des Internets \cite{suchen}, der es Wissenschaftlern ermöglichen sollte, Ideen vor der gedrukten Veröffentlichung zu teilen. Vier Jahre später forderte Steven Harnad die wissenschaftliche Community dazu auf, sofort mit der digitalen Selbstarchivierung und öffentlichen Zurverfügungstellung ihrer Beiträge zu beginnen \cite{albert_2006_open_implications}, um "den Barrieren, die zwischen ihrer Arbeit und ihrer (kleinen) Leserschaft aufgestellt werden, zu entkommen" \cite{harnad_1995_subversive_proposal}.

Durch die zunehmende Verbreitung und Nutzung der dieser digitalen Pre-Print Dienste, gründete sich im Oktober 1999 im Rahmen der "Santa Fe Convention" die "Open Archives Initiative", die sich maßgeblich mit den technischen und organisatorischen Aspekten der Transformation der wissenschaftlichen Kommunikation beschäftigte. ---- TODO Bitex: The Santa Fe Convention of the Open Archives Initiative ---

2001 wurde der europäische Arm von der Scholarly Publishing and Academic Resources Coalition (SPARC) einer der späteren "major player" der Open Access Bewegung \cite{russell2008business} \cite{Herb_2012} gegründet. Als Konsequenz aus der Zeitschriftenkrise sollte diese 1998 in den USA gegründete Allianz zwischen Universitäten und wissenschaftlichen Bibliotheken dafür sorgen, dass die Kosten für wissenschaftliche Zeitschriften reduziert oder durch die Bereitstellung kostengünstiger oder freier, nicht-kommerzieller, Peer-Review-Fachzeitschriften ersetzt werden. Durch Weiterbildung, politische Arbeit und die Förderung alternativer Geschäftsmodelle, war es Ziel von SPARC, Initiativen für offenes wissenschaftliches Publizieren zu stimulieren \cite{suchen}.

\subsubsection{Die Manifestierung der Forderung nach offenem Zugang}

In 2001 erschien Open Access erstmals im wissenschaftlichen Diskurs als eigenes und öffentlichkeitswirksames Thema \cite{cite:19}. Die Public Library of Science (PLoS), gegründet im Oktober 2000, forderte Wissenschaftler in einem offenen Brief im Mai 2001 dazu auf, ab September 2001 nur noch in den Zeitschriften zu veröffentlichen, beziehungsweise nur noch die Zeitschriften zu reviewen, zu editieren und zu abonnieren, deren Beiträge spätestens sechs Monate nach ihrer Erstveröffentlichung für jedermann im Internet kostenlos und unentgeltlich einsehbar sind \cite{cite:20}. Schon nach kurzer Zeit unterzeichneten (nach eigenen Angaben \cite{cite:19a}) rund 38.000 Wissenschaftler aus 180 Nationen das Schreiben. Dieser Brief kann als Auftakt zu einem 20-monatigen theoretischen Schub gesehen werden. Neben PLoS wird der im Jahr 2000 gegründete britische Verlag Biomed Central als "Wegbereiter in der von OA" \cite{suchen-Hoffmann-Zugang-undVerwertung-öffentlicher-Informationen}. In diesem Zeitraum entstanden drei der bis heute wichtigsten Erklärungen im Bereich der Öffnung des Zugangs zu wissenschaftlicher Kommunikation \cite{CREATe_2014}:
\begin{enumerate}
\item Erklärung der Budapest Open Access Initiative (Dezember 2001 und 2012)

Im gleichen Jahr wie der PLoS-Brief, wurden im Rahmen einer Konferenz des Open Society Institutes in Budapest, mit der “Budapest Open Access Initative” (BOAI)\cite{boai_2012} erstmals die Bemühungen um Open Access in einer eigenen Erklärung zusammengefasst\cite{cite:21a}. Im Fokus dieser steht die Forderung nach freiem Zugang zu wissenschaftlichen Publikationen. In der BOAI wird erstmals manifestiert, dass wissenschaftliche Peer-Review-Fachliteratur “kostenfrei und öffentlich im Internet zugänglich sein sollte, so dass Interessenten die Volltexte lesen, herunterladen, kopieren, verteilen, drucken, in ihnen suchen, auf sie verweisen und sie auch sonst auf jede denkbare legale Weise benutzen können, ohne finanzielle, gesetzliche oder technische Barrieren jenseits von denen, die mit dem Internet-Zugang selbst verbunden sind. In allen Fragen des Wiederabdrucks und der Verteilung und in allen Fragen des Copyrights überhaupt sollte die einzige Einschränkung darin bestehen, den Autoren Kontrolle über ihre Arbeit zu belassen und deren Recht zu sichern, dass ihre Arbeit angemessen anerkannt und zitiert wird."\cite{boai_2012}

Anlässlich des zehnten Jahrestages der BOAI, wurde von der Open Society Foundation mit der BOAI 10 (2012) die usrprüngliche Erklärung bestärkt und anhand von weitere Richtlinien und Empfehlungen die Entwicklungen und Herausforderungen in seiner zehnjährigem Bestehen adressiert. Die Initiatoren kommen unverändert zu dem Schluss, dass "noch immer Zugangsbeschränkungen zu Peer-Review-Forschungsliteratur, meist eher zugunsten der Verlage, als zugunsten der Autoren, Reviewer oder Redakteure und damit auch auf Kosten der Forschung, Forscher und Forschungseinrichtungen" \cite{boai_2012} bestehen. "Nichts aus den letzten zehn Jahren" lässt "darauf schließen, dass das ursprüngliche Ziel von OA weniger sinnvoll oder erstrebenswert erscheint. Im Gegenteil, die Notwendigkeit, dass Wissen für jeden, der es nutzen, anwenden oder darauf aufbauen kann, offen verfügbar sein sollte, ist dringlicher als je zuvor" \cite{boai_2012}.

\item Die Bethesda Erklärung (Juni 2003)

Zwei Jahre nach Veröffentlichung der initalen Version der BOAI-Erklärung, im Juni 2003, verabschiedete eine Gruppe von Forschungsförderern, wissenschaftlicher Gesellschaften, Verlegern, Bibliothekaren, Forschungseinrichtungen und einzelner Wissenschaftler im US-Bundesstaat Maryland das "Bethesda Statement on Open Access Publishing".\cite{suchen} Ziel der Erklärung war die Stimulation der Diskussion in der biomedizinischen Forschung, "wie man schnellstmöglich den offenen Zugang zu der primären wissenschaftlichen Literatur in der Biomedizin erreichen könnte"\cite{suchen}. Wie bereits in der BOAI erklärten die Autoren des "Bethesda Statement on Open Access Publishing" Bedingungen für den offenen Zugang zu wissenschaftlichen Publikationen \cite{suchen}:

Erstens werden Autor(en) und Urheberrechts-Inhaber aufgefordert für alle Benutzer eine freies, unwiderrufliches, weltweites und unbefristetes Recht auf den Zugang zulassen, sowie eine Lizenz zu verwenden, die das Kopieren, Nutzen, Verbreiten, Übertragen und öffentliches Darstellen der Publikation ermöglichen. Darüber hinaus muss es erlaubt sein, abgeleitete Werke zu verteilen, in jedem digitalen Medium für jeden Zweck zu veröffentlichen, vorbehaltlich einer angemessenen Zuordnung der Urheberschaft. Das beinhaltet auch das Recht auf eine kleine Anzahl gedruckter Kopien für den persönlichen Gebrauch.

Zweitens, muss eine vollständige Version der Arbeit und aller ergänzender Materialien, einschließlich einer Kopie der Genehmigung, wie oben erwähnt, in einem geeigneten elektronischen Standardformat sofort bei der ersten Veröffentlichung in mindestens einem Online-Repositorium, das von einer wissenschaftlichen Einrichtung unterstützt wird, hinterlegt werden. Dieses Repositorium muss von einer wissenschaftlichen Gesellschaft, Regierungsbehörde oder einer anderen etablierten Organisation akzeptiert sein. Diese muss sich für einen offenen Zugang, uneingeschränkte Verbreitung sowie Interoperabilität und Langzeitarchivierung (für die biomedizinischen Wissenschaften, PubMed Central ist ein solches Repository) verpflichtend einsetzen.

\item Die Berliner Erklärung (Oktober 2003)

Ein weiterer Meilenstein für die Verbreitung von Open Access auf dem europäischen Kontinent waren die "Berlin Konferenzen"\cite{CREATe_2014}. Die erste Tagung wurde 2003 von der Max-Planck-Gesellschaft und dem Projekt European Cultural Heritage Online (ECHO) organisiert, um über "Zugangsmöglichkeiten zu Forschungsergebnissen" zu diskutieren. In diesem Rahmen entstand 2003 auch die "Berliner Erklärung über den offenen Zugang zu wissenschaftlichem Wissen" \cite{berliner_erklaerung_2003}, in der die Verfasser über die Budapester Erklärung hinaus gehen und neben dem kostenlosen und freien Zugang zu wissenschaftlichem Wissen in Form von Publikationen auch den freien und offenen Zugang zu den Daten fordern. „Open Access-Veröffentlichungen umfassen originäre wissenschaftliche Forschungsergebnisse ebenso wie Ursprungsdaten, Metadaten, Quellenmaterial, digitale Darstellungen von Bild- und Graphik-Material und wissenschaftliches Material in multimedialer Form.“ \cite{berliner_erklaerung_2003} Es formiert sich ein erweitertes Verständnis von Open Access und es entsteht damit die Grundlage für ein erste Ansatzpunkte zur Definition von Open Science. Dennoch konzentriert sich die Diskussion in diesem Stadium noch ausschließlich auf den bereits abgeschlossenen wissenschaftlichen Prozess.
\end{enumerate}

Alle drei Erklärungen, auch die "three B's"\cite{suber_2004_praising_oa} genannt, gelten als die angesehensten Definitionen von Open Access und sind in wesentlichen Merkmalen in sich stimmig\cite{albert_2006_open_implications}. Sie stimmen vor allem in den Kernforderung nach der Entfernung preislichen und teil der rechtlichen Barrieren bezüglich des freien Zugangs überein. Trotz einiger Unterschiede ähneln sich die Definitionen auch bei der Fragen nach der Enfernung der Behinderungen für die kommerzielle Nutzung und die Erstellung von Derivaten  \cite{CREATe_2014}.

Im Jahr 2003 entstand das Directory of Open Access Journals (DOAJ), das bis zum Jahr 2013 wurde das Portal von der schwedischen Universität Lund betrieben. Das Portal stellt eine zentrale Anlaufstelle für OA-Journale dar \cite{suchen-Hoffmann-Zugang-undVerwertung-öffentlicher-Informationen}. Vorangegangen war im Jahr 2002 die Entwicklung und veröffentlichung der Creative Commons Initiative \cite{suchen-Hoffmann-Zugang-undVerwertung-öffentlicher-Informationen} und ihrer ersten Version der Lizenzen, die bis heute als rechtliche Grundlage für eine Vielzahl der Open Access Publikationen dienen\cite{suchen}. Die modularen Lizenzen sind im Kontext von Open Access wichtig, "um (Nach)nutzungsmöglichkeiten für Texte, Daten und andere wissenschaftliche Erzeugnisse festlegen zu können" \cite{suchen-Hoffmann-Zugang-undVerwertung-öffentlicher-Informationen}.

Die Deutsche Forschungsgemeinschaft (DFG) regierte Anfang 2006 und verabschiedete eine Richtlinie nach denen sie zwar nicht vorraussetzt, aber "erwartet", dass Publikationen aus DFG-geförderten Projekten "möglichst" als Open Access veröffentlicht werden \cite{suchen:dfg-richtlinie}. Eine ähnliche Erklärung verabschiedete auch die größte amerikanische Förderinstitution National Institutes of Health (NIH) aus und "stellte mit PubMed Central (PMC) eine entsprechende Plattform bereit \cite{muller_2010_open}. In 2008 wurde die Veröffentlichung NIH-geförderter Publikationen verpflichtend \cite{Hanekop_2014}. In Deutschland gibt es keine zentrale Plattform wie PMC und die Veröffentlichug der geförderten Ergbnisse als Open Access ist weiterhin nicht bindend.

In der Debatte über die Zukunft des wissenschaftlichen Publizierens und Kommunizierens besteht die Tendenz, Konzepte der offenen Wissenschaft als einen bisher beispiellosen und noch nie dagewesenen Wandel darzustellen \cite{cite:17a} \cite{cite:17b}. Diese Darstellung basiert auf "verschiedenen Gründungsmythen", die auf "unterschiedliche Zielsetzungen und Lösungspfade" verweisen \cite{suchen-Hoffmann-Zugang-undVerwertung-öffentlicher-Informationen}. Die Geschichte von Open Access ist eine Geschichte, die mit der Digitalisierung von Kommunikationsprozessen \cite{albert_2006_open_implications} auf der einen, sowie mit der Zeitschriftenkrise auf der anderen Seite verknüpft ist \cite{suchen-Hoffmann-Zugang-undVerwertung-öffentlicher-Informationen}. Open Access ist kein Selbstzweck\cite{cite:17d}, sondern ein Attribut tiefergehender Prozesse, die mit der wachsenden Bedeutung der Digitalisierung in unserer Zivilisation und dem damit einhergehenden Wandlungsprozessen im Machtgefüge zusammenhängen\cite{cite:17e}. Dennoch, obwohl es davor schon vereinzelt Versuche gab, komplett Informationen und Publikationen offen und frei zu kommunizieren, war Open Access im Printzeitalter physisch und ökonomisch über lokale Grenzen hinaus schwer möglich \cite{cite:18a}.

Die Forderung nach Öffnung von Wissenschaft und Forschung ist in diesem Zusammenhang nicht nur eine "politische Reaktion" oder "technische Alternative", sondern eine "alternative Formatierungen einer wissenschaftlichen Infrastruktur im technischen, rechtlichen und zeitlichen Sinne" \cite{kelty_2004}. Sie betrifft "Wissenschaftler, politische Entscheidungsträger und die Öffentlichkeit" \cite{Scheliga_2014}.

Im April 2012 wurde die Erklärung "Open Science for the 21st century", vom Zusammenschluss der Europäischen Akademien (ALLEA) verabschiedet \cite{ALLEA_2012}. Sie war nur eine von mehreren Erklärungen und Positionspapiere für die Öffnung von Wissenschaft durch international angesehenen Einrichtungen, durch die verdeutlich wurde, dass die Forderung nach offenem Umgang mit Wissen und Information im wissenschaftlichen Bereich zunehmend an Relevanz gewinnt \cite{schulze_2013_open}.

--- TODO: Weitere Entwicklung darstellen ----

\section{Wissenschaftliche Reputation}

Wissenschaftliche Reputation ist eine "Art von Kredit" \cite{luhmann_1970_selbststeuerung}. Ein wesentliches Prinzip des Wissenschaftssystems basiert auf der "gegenseitigen Beurteilung und Anerkennung der jeweils neuen Ergebnisse der Fachkollegen (Peers) durch die Wissenschaftler selbst"\cite{Hanekop_2014} \cite{suchen_Hornbostel_2006}, also teils auf der "Generalisierung von Einzelleistungen", auf "gegenseitiger Ansteckung" und teils "auf der bloßen Häufigkeit der Publikationen oder der Anwesenheit an renomierten Plätzen" \cite{luhmann_1970_selbststeuerung}. Dabei gesteht auch Luhmann die Existenz von "Nebencodes der Reputation" zu \cite{schmoch_2003_hochschulforschung}. Die Reputation steuert die wissenschaftliche Aufmerksamkeit und die Verteilung motivierender Effekten, die sich durch das reine Streben nach Erkenntnis nicht erzeugen lassen \cite{suchen_luhmann}.

Als "guter akademischer Forscher" gilt nur der, "wer viel und in möglichst angesehenen Journalen" veröffentlicht \cite{Frey_2005}. Dabei spielt der Peer-Review-Prozess eine zentrale Rolle und ist Kernelement der Selbststeuerung von Wissenschaft \cite{Neidhardt_2010}. In dem Peer-Review Prozess "werden eingereichte Beiträge von fachlich versierten Wissenschaftlern (...) beurteilt und gemäß der qualitativen Anforderungen der Forschungs-Community zur Veröffentlichung angenommen oder abgelehnt" \cite{Hess_2006}. Der Prozess gilt als "Herzstück einer autonomen, selbstverwalteten Wissenschaft" \cite{suchen_Hornbostel_2006}. Peer-Review beschränkt sich dabei aber nicht nur auf den Prozess der Publikation von Texten, sondern deckt ein breites Spektrum von Aktivitäten über Fachdisziplinen hinaus ab \cite{Lee_2012}:
\begin{itemize}
\item die Beobachtung der klinischen Praxis (z.B. in der Medizin)
\item Beurteilung des Lehrenden
\item Fähigkeiten der Kollegen
\item Bewertung durch Experten bei der Forschungsförderung und Stipendien bei Einreichung von Anträgen an staatliche und anderen Förderorganisationen
\item Begutachtung von Redakteuren und externen Gutachtern bei Artikeleinreichungen für wissenschaftlichen Zeitschriften
\item Bewertung von Papieren und Plakate für Konferenzen
\item Bewertung von Buchvorschlägen für Universitätverlagen oder andere Verlagen
\itemEinschätzungen der Qualität, Anwendbarkeit und Interpretierbarkeit von Datensätzen und wissenschaftlichen Organisationen
\end{itemize} Dennoch bilden "Publikationen im Hinblick auf die Funktion der Reputationsverteilung eine Art Telos wissenschaftlicher Kommunikation" \cite{hirschauer2004peer}. Im Rahmen von Reputation ist wissenschaftliche Arbeit besonders auf ein funktionierendes Peer-Review-System angewiesen \cite{suchen}. Das Verfahren hat zwei Funktionen: 1. die Selektionsfunktion, in deren Rahmen die Auswahl von Personen, Projekten und Texten stattfindet und 2. eine Konstruktionsfunktion, in der Gutachter "produktiv in den Wissenschaftsprozess eingreifen" und die eigenen Fachstandards durchzusetzen \cite{Neidhardt_2010}. Der Peer Review-Prozess sichert aber nicht nur "Vertrauen" und die Grundlage für die "Anschlusskommunikation" innerhalb der wissenschaftlichen Community, sondern "wirkt überdies auch nach außen und gewährleistet die gesellschafliche Legitimation des wissenschaftlichen Wissens" \cite{pscheida_2010_wikipedia}.

Das Peer-Review Verfahten ist in der Wissenschaft etabliert und weit verbreitet. Dennoch haben aktuelle qualitatives Peer Review-Systeme und quantitative bibliometrischen Verfahren objektiv viele Mängel \cite{osterloh2008anreize} \cite{Lee_2012} \cite{Jansen_2007}. Die Mängel lassen sich laut Osterloh und Frey wie folgt zusammenfassen \cite{osterloh2008anreize}:
\begin{itemize}
\item "Geringe Reliabilität der Gutachter-Urteile."
\item "Geringe prognostische Qualität von Gutachten"
\item "Opportunistisches Verhalten der Gutachter und Editoren"
\item "Opportunistisches Verhalten der Autoren"
\end{itemize}
Die beiden Autoren kommen zu dem Schluss, dass "die Annahme eines Manuskriptes einem Zufallprozess gleicht"  und das "System der qualitativen Peer Reviews (...) auf einer erstaunlich fragwürdigen wissenschaftlichen Grundlage" beruht \cite{osterloh2008anreize}.

--- Todo: Grafik aus Kritik am Peer-Review bauen http://www.zalf.de/de/forschung/services/pubman/service/Documents/Manuskriptfluss/Mueller_2008_peer_review.pdf ----

Folgende weitere Indikatoren werden in diesem Zusammenhang für wissenschaftliche Reputation für wissenschaftliche Institutionen und Personen genannt \cite{hanekop_2008}:
\begin{enumerate}
\item \textbf{Anzahl der wissenschaftlichen Aufsätze / Beiträge} - Die reine Anzahl der Texte die Wissenschaftler im Rahmen ihrer Tätigkeit publizieren ist ein wesentlicher Faktor der Bewertung wissenschaftlicher Reputation \cite{luhmann_1970_selbststeuerung} \cite{CLAPHAM_2005}. Zum Beispiel erhöht die reine Anzahl an Texten schon die Chance auf mehr Zitationen durch die wissenschaftliche Community und damit auf Reputation. Im zunehmenden Wettbewerb in der Wissenschaft muss sich der einzelne Wissenschaftler dazu entscheiden, "zu publizieren oder im wissenschaftlichen System zu scheitern" \cite{Suess_2006}. Dadurch entsteht im wissenschaftlichen Kommunikationssystem ein imenser Publikationsdruck, bei dem die Relevanz der publizierten Ergebnisse nicht immer im Vordergrund steht\cite{suchen}. Auch Rahmen der Drittmittel spielt die Anzahl der Artikel bei der Vergabe von Ressourcen für weitere Forschung eine Rolle. \cite{suchen}
\item \textbf{Relevanz der publizierten Ergebnisse} - Die Relevanz der publizierten Ergebnisse ist für das Wissenschaftssystem ein wesentlicher Treiber im wissenschaftlichen Kreislauf. Relevante Erkenntnisse sind die Grundlage für die Produktion von neuem Wissen und damit Grundlage für den gesellschaftlichen Auftrag des Wissenschaftssystems. \cite{suchen} Es besteht die Annahme, dass eine höhre Relevanz auch zu mehr Reputation führt.
\item \textbf{Anzahl Monografien} - Die Anzahl der Monographien ist ein wesentlicher Reputationsfaktor. Das gilt aber nur für die Disziplinen, in diese Publikationsform wichtig sind, wie den Geistes- und Sozialwissenschaften.
\item \textbf{Drittmittelprojekte} - Drittmittel sind, so der Wissenschaftsrat bereits 1993, "solche Mittel, die zur Förderung der Forschung und Entwicklung sowie des wissenschaftlichen Nachwuchses und der Lehre zusätzlich zum regulären Hochschulhaushalt(Grundausstattung) von öffentlichen oder privaten Stellen eingeworben werden" \cite{wr_2014}. Die Drittmitteleinwerbung hat sich in Deutschland als "meist gebrauchter Maßstab der Messung von Forschungsqualtität durchgesetzt" \cite{M_nch_2006}. Es geht mit einer zunehmenden Finanzierung der Forschung über Drittmittel einher \cite{Neidhardt_2010} \cite{Jansen_2007} \cite{simon_2009_wissenschaft_governance}. Dabei spielt die Frage eine Rolle, ob die Publikationen, die im Rahmen der Drittmittelfinanzierung veröffentlicht werden oder auch der Antrag um Drittmitteleinwerbung selbst, "zum Erkenntnisfortschrit in der wissenschaftlichen Gemeinschaft beiträgt" \cite{M_nch_2006}. Die Erfordernis der Aquise von Drittmitteln, durch die zunehmende Knappheit öffentlicher Ressourcen für Wissenschaft und Forschung, ist zu einem kritisch zu betrachtenden Kernziel geworden \cite{Jansen_2007} was dazu führt, dass zunehmend direkte finanzielle und administrative Kontrolle der Forschung eine Rolle spielen \cite{Barl_sius_2008}.
\item \textbf{Patente} - "Unter einem Patent versteht man das vom Staat verliehene Schutzrecht für eine technische Erfindung, welches dem Patentinhaber für eine bestimmte Zeit die ausschließliche wirtschaftliche Nutzung der Erfindung vorbehält." \cite{greif_2003_patente} Seit dem 1970er Jahen wird eine steigende Anzahl von Patenten aus dem Hochschulbereich verzeichnet \cite{schmoch_2003_hochschulforschung}. Dabei wird die Patentschrift "als funktionales Äquivalent zur wissenschaftlichen Publikation begriffen" und bewertet \cite{mersch_2014_patente}. "Patente leisten einen Beitrag zur Förderung der Wissenschaft, die Grundlagen des Patentwesens sind daher dem wissenschaftlichen Nachwuchs über entsprechende Lehrangebote zu vermitteln." \cite{suchen}
\item \textbf{Vorträge} - Vorträge dienen der Verbreitung der Forschungserkenntnisse und ermöglichen das Aufgreifen des Wissens durch andere \cite{rassenhoevel_2010_performancemessung}. Vorträge stellen eine schnelle Form für die Verbreitung neuer wissenschaftliche Erkenntnisse und Ergebnisse dar, ohne dass jeder Gedanke genauer belegt werden muss und die sich gegebenenfalls später wieder schriftlich korrigieren lassen \cite{haberle_2002_jahrbuch}.
\item\textbf{Anwendungsrelevanz bzw. Verwertbarkeit} - Ein vergleichsweise neuer Indikator für Hochschulen und ausseruniversitäte Forschungsinstitute ist die Anwendungsrelevanz von Wissenschaft und Forschung \cite{simon_2009_wissenschaft_governance}. Sie bezieht sich auf einen Outputfaktor, der sich primär auf den Einsatz der gewonnen wissenschaftlichen Erkenntnisse bezieht und eher auf eine Verwertbarkeit für Produkte oder Patente als auf wissenschaftliche Veröffentlichung abzielt \cite{suchen}.
\item \textbf{Netzwerke und Kontakte} - Netzwerke beschreiben formelle und informelle Verbünde zwischen Wissenschaftlern. Sie erlauben den schnellen Austausch und können Grundlage für Aktivitäten zur Steigerung der wissenschaftlichen Reputation darstellen (gemeinsame Publikationsvorhaben, Austausch von wissenschaftlicher Erkenntnisse usw.). Kontakte und Netzwerke schaffen soziale Beziehungen, die für eine erfolgreiche Integration an der Hochschule und der Fachcommunity Zugang zu wissenschaftlicher Kommunikation ermöglicht und Einfluss auf die Reputation eines Wissenschaftler haben kann.
\item \textbf{öffentliche Aufmerksamkeit} - Die öffentliche Aufmerksamkeit stellt zum einen eine Möglichkeit des Wissenstransfers ausserhalb der wissenschaftlichen (Fach-)Community dar, zum Anderen ermöglicht sie Einflussnahme auf die politische Relevanz wissenschaftlicher Forschungsthemen. Die Veröffentlichung von wissenschaftlichen Informationen zu einem bestimmten Thema des öffentlichen Interesses stellt eine Möglichkeit dar, eben dieses Thema öffentlichkeitswirksam zu katalysieren. Öffentliche Aufmerksamkeit im Rahmen von Wissenschaft und Forschung stellen eine kritisch zu hinterfragende Möglichkeit für die alternative Ressourcengewinnung dar. \cite{suche}
\item \textbf{Politische Relevanz} - Ähnlich wie bei der öffentlichen Aufmerksamkeit stellt auch die politische Relevanz eine Möglichkeit dar wissenschaftliche Inhalte ausserhalb der Wissenschaft relevant zu machen. Im Wissenschaftssystem geht es aber "um Erwerb und Erhalt von Wissen, in der Politik dagegen um Erwerb und Erhalt von Macht" \cite{Mayntz_1996}. Daraus ergeben sich grundsätzliche "Verständigungsprobleme und Interessenkonflikte", da  "Wissenschaft und Politik aufgrund unterschiedlicher Rationalitäten handeln, einander aber zugleich brauchen" \cite{Mayntz_1996}. Die Interessenkonflikte füren zu "gegenseitigen Enttäuschungen", vor allem in der "forschungspolitischen Beziehung" \cite{Mayntz_1996}.
\item \textbf{Renommee der Forschungseinrichtung} -
Das Renommee einer Forschungseinrichtung ist die Wahrnehmung der Einrichtung innerhalb und außerhalb der wissenschaftlichen (Fach-)Community. Sie hat für Wissenschaftler eine besondere Bedeutung \cite{mayntz_2008_wissensproduktion}. Sie basiert auf dem Konzept der "Ansteckung" in dem zum Beispiel rennomierte Professoren den Ruf einer Fakultät und eine rennomierte Fakultät auch den Ruf von Professoren aufbessern können \cite{luhmann_1970_selbststeuerung}. Am Beispiel von Publikationen: Ein Autor profitiert durch das Renommee einer Einreichtung, wenn er durch ihre Publikationsorgane veröffentlicht \cite{lutz_2012_zugang}.
\item \textbf{Renommee von Herausgebern oder Mitautoren} Der Herausgeber organisiert den Begutachtungsprozess und sichert bestimmte Qualitätskriterien mit seiner Reputation und seinem Namen \cite{mueller_2009_peerreview}.
im Rahmen des symbolischen wissenschaftlichen Kapitals
\item \textbf{personelle und materielle Ausstattung, Großgeräte etc.} -
Die materielle Ausstattung beschreibt die Rahmenbedinungen, in der ein Wissenschaftler arbeitet. Die Rahmenbedingungen haben eine herausragende Bedeutung bei der Überlegung von Wissenschaftlern bei der Auswahl des Wirkungsortes \cite{mayntz_2008_wissensproduktion}. Materielle und personelle Ausstattung gemeinsam sind vor allem bei traditionellen Berufungsverfahren deutscher Professorinnen und Professoren von besonderem Belang \cite{himpele_2011_job}, da sie die Arbeitsfähigkeit und die Annerkennung direkt beeinflussen \cite{suche}. Ein weiterer Indikator für die wissenschaftliche Reputation ist die personelle Ausstattung. Wie die materielle Ausstattung gilt auch die personelle Ausstattung als ein reputationsstifdendes Merkmal für den Wissenschaftler und die Institution an der er arbeitet \cite{mayntz_2008_wissensproduktion}. Bei der Ausstattung handelt es sich um einen bilaterale Indikator, der zum einen aus der Bewertung der Wissenschaftlichen Arbeit (im Rahmen der Forschungsförderung) resultiert \cite{Herb_vermessung_2008} und  zum anderen Reputation innerhalb der Community schafft \cite{mayntz_2008_wissensproduktion}.
\item \textbf{Gutachtertätigkeit, Herausgeberschaft und Funktion} - Gutachter werden zum Beispiel in Peer-Review-Verfahren Autoren des entsprechenden Fachgebietes zugeordnet und entscheiden über die Veröffentlichung des Textes oder weisen diesen zurück \cite{Frey_2005}. Dieses Verfahren wird bei manchen Publikationen mehr als drei mal durchlaufen, bevor ein Artikel akzeptiert und daraufhin publiziert wird \cite{Frey_2005}. Die Reputation der mit diesem Verfahren betrauten Gutachter wirkt sich auch auf das Image des Verlages aus und umgekehrt. Die Gutachtertätigkeit ist aber nicht nur Kernbestandteil des wissenschaftlichen Qualitätssicherungs- und interdependenten Reputationssystems \cite{Frey_2005} \cite{mueller_2009_peerreview}, sondern stellt auch einen informellen Weg der Kommunikation über eigene Publikationen mit Verlagen oder Herausgebern dar. Darüber hinaus ermöglichen die Gutachterkeit Vorabsichtung neuster wissenschaftlicher Informationen und Erkenntnisse. Ähnlich wie die Gutachtertätigkeit ist auch die Herausgeberschaft fester Bestandteil des interdependenten wissenschaftlichen Reputationssystems \cite{Frey_2005}. Herausgeber profitieren von den Autoren, beziehungsweise von deren publizierten Ergebnissen in den von ihnen als Herausgeber verantworteten Publikationen sowie von der Reputation dieser. Sie widerum geben Reputation an den Verlag, Autor und die Publikation ab \cite{suchen}. Die jeweilige Funktion oder die (universitäre) Stellenbezeichnung ist ein weiter Faktor für wissenschaftliche Reputation. Zum wissenschaftlichen Personal zählen Professoren, Juniorprofessoren, wissenschaftliche und künstlerische Mitarbeiter, sowie Lehrkräfte \cite{erhardt_2011_hochschulen}. Eine Weiterentwicklung und der "Aufstieg" in der wissenschaftlichen Hierarchie zielt auf das akademische Streben nach einer Professur \cite{Klecha_2008}.
\item \textbf{Awards und Preise} - Preise sind ein Bestandteil des wissenschaftlichen Belohnungs- und Bewertungssystems. "Die Praxis der „Award“-Verleihung beruht auf dem Konzept, dass Ressourcen von unabhängigen Dritten auf Qualität geprüft und (...) zertifiziert werden". Wissenschaftler die Preise oder Awards gewinnen, erfahren große Anerkennung \cite{suchen}, wobei diese anderseits kein "Garant für wissenschaftsrelevante Qualität" darstellen \cite{bargheer_2002_qualitatskriterien}. Die Anerkennung selbst weckt große Erwartungen und die Ehrung mit einem Preis setzt ebenfalls einen stetigen Nachschub an Anerkennung voraus\cite{suchen}.
\end{enumerate}

Wissenschaftliche Reputation wird in diesem Zusammenhang als Währung bezeichnet, mittels derer “Status und Ressourcen verteilt werden” \cite{hanekop_2006}. Sie verteilt sich auf Einrichtungen und einzelne Personen, die wissenschaftlich tätig sind \cite{suchen}. Die Evaluation wissenschaftlicher Einrichtungen findet dabei über “Beobachtungen und Gespräche mit den Wissenschaftlern vor Ort sowie über den Austausch über die Eindrücke innerhalb der Begehungsgruppe und die gemeinsame Verständigung”\cite{Barl_sius_2008} statt.

Die Reputation einzelner Wissenschaftler steht in enger Abhängigkeit zum bestehenden wissenschaftlichen Kommunikationssystem \cite{suchen}. Anstatt durch finanzielle, wird in den Wissenschaften primär mit Aufmerksamkeit entlohnt \cite{suchen}. Vereinfacht lässt sich das System der Wechselbeziehungen der Reputationsverteilung im Rahmen von Publikationen wie folgt darstellen \cite{cite:21a}:

Grafik aus Text von Bernius
http://www.eap-journal.com/archive/v39_i1_8_bernius.pdf

Bernius et al. unterscheiden drei aufeinandertreffende koordinierende Marktmechanismen: die Reputation, die Nutzung wissenschaftlicher Publikationen, sowie den Preis für den Erwerb \cite{suchen}. Während die Reputation ein non-monetärer Aushandlungsmechanismus zwischen wissenschaftlichen Verlagen und wissenschaftlichen Autoren ist, findet die monetäre Preisdefinition zwischen Bibliotheken und Verlagen statt. Der monetäre Aushandlungsprozess zwischen Wissenschaftlern und Bibliotheken fokussiert sich auf die Bedeutung und Nutzung der jeweiligen Publikation \cite{cite:21a}. Nicht jede Publikation hat diesbezüglich die gleiche Wertigkeit \cite{suchen} und damit den gleichen Einfluss auf Reputation des Autors.

Die neuen Möglichkeiten der Verbreitung von Informationen lassen deshalb einen vergleichbaren Veränderungsprozess der wissenschaftlichen Reputation und damit auch Anerkennung vermuten, wie sie durch die Entwicklung des Buchdrucks ausgelöst worden war.\cite{hanekop_2006}.

Dabei identifizierte der US-amerikanische Soziologe Robert K. Merton unter anderem die allgemeine Verfügbarkeitmachung von Forschungsergebnissen als einen "integralen Bestandteil wissenschaftlichen Ethos" \cite{Fangerau_2014}. Er stellte diesen und weitere Grundprinzipien als normative Struktur des Ethos der Wissenschaft vor \cite{Merton_1985}. Der Ethos wird in diesem Zusammenhang als "Komplex von Werten und Normen"\cite{suchen} beziehungsweise "Verhaltensmaßregeln"\cite{suchen} verstanden.

\begin{itemize}
\item Universalismus - Die sozialen Merkmale eines Wissenschaftlers, wie zum Beispiel Nationalität, Geschlecht, Religion, Klasse usw. darf nicht in die Evaluation wissenschaftlicher Ergebnisse einfließen \cite{suchen}.
\item Kommunismus (Kommunalität) - Es gibt eine Pflicht zur Veröffentlichung der Ergebnisse von Wissenschaft und Forschung und sie sind als Allgemeingut zu betrachten. Die wissenschaftliche Anerkennung und Ansehen sind einziges "Besitzrecht"\cite{suchen}.
\item Uneigennützigkeit - Intrinsische "Neugier"\cite{suchen}, "selbstloses Eintreten für das Wohl der Menschheit"\cite{suchen} und der Wissensdrust müssen die vornehmlichen Motivatoren für Wissenschaftler darstellen \cite{suchen}. Eines der weiteren Kriterien erfordert "Objektivität und Desinteresse" an den Ergebnissen der eigenen Forschung \cite{suchen} unabhängig von finanziellem Erfolg und Prestige \cite{suchen}.
\item Organisierter Skeptizismus - Zweifel muss als "grundsätzliches Denkprinzip der Wissenschaft" \cite{suchen} und die "unvoreingenommene Prüfung und Kritik an Wissenschaft, Forschung und Autorität" \cite{suchen} verstanden werden. Dabei gilt es auch den "Matthew Effect" zu vermeiden. Der Matthäus-Effekt ist ein Phänomen auf der Makroebene der Wissenschaft \cite{bonitz_1998_matthaus}. Der „Matthäus-Effekt" ("Wer hat, dem wird gegeben" Mt. 25,29) beschreibt den Umstand, dass Autoren, die bereits häufig zitiert wurden, häufiger zitiert werden als andere Autoren und dadurch noch mehr wissenschaftliche Reputation erlangen, was wiederum zu einer noch höheren Anzahl an Zitationen führt. \cite{Merton_1968} \cite{meier_2009_matthaus}.
\end{itemize}

Merton erkannte damit zwar das Urheberrecht an wissenschaftlichen Ideen und Beiträgen an, allerdings nur insofern, als das Urheberrecht allein auf die Ermöglichung der Anerkennung durch Kollegen und die Achtung der Priorität beschränkt bleibt \cite{Fangerau_2014}.

\subsection{Messbarkeit wissenschaftlicher Qualität vs. Publikationsquantität}
Wissenschaft ist ein Prozess, bei dem aus “unterschiedlichen Inputfaktoren, mittels verschiedener Transformationen Beiträge zur Schaffung neuer wissenschaftlicher Erkenntnisse als Output entstehen” \cite{Jansen_2007}. Die Bewertungen des jeweiligen Outputs führt “zur Ausage über die Forschungsperformanz” \cite{suchen}. Neben den Indikatoren für den Output wissenschaftlicher Perfomanz, müssen aber auch intermediäre Aspekte berücksichtigt werden\cite{schmoch_2009}.

Mit Beginn des 20. Jahrhunderts wurden in der Wissenschaftsforschung Indikatoren überwiegend zur Beschreibung der exponentiellen Wachstumsverläufe von Wissenschaft entwickelt und eingesetzt \cite{Hornbostel_1997}. Nach dem zweiten Weltkrieg etablieren sich die ersten Indikatoren für die Effizienzmessung wissenschaftlicher Wissensproduktion und -verbreitung, die aber "ebenso wie Sozial- und Wirtschaftsindikatoren keine neutralen Realitätsbeschreibungen" \cite{Hornbostel_1997} darstellen. Spätestens seit den 1970er Jahren werden diese Messungen, die die Forschungsleistung quantifizieren sollen, flächendeckend durchgeführt \cite{Hornbostel_1997}.

Seit den 1990er Jahren ist diese Bewertung in Gestalt von Zahlen als unkontrollierte Nebenprodukte digitaler Wissenskommunikation erweitert worden \cite{angermueller_2010}. Heute zählen in der Wissenschaft vor allem die wissenschaftliche Reputation und die als Impact bezeichnete Wirkung wissenschaftlicher Publikationen\cite{herb_open_2013} \cite{Hornbostel_1997}. Die Wirkung wird dabei anhand der Zitationen der jeweiligen Publikation gemessen \cite{suchen}.

Die Kommunikation ist die "Essenz der Wissenschaft"\cite{bonitz_1998_matthaus} und "Zitierungen in ihrer Gesamtheit so etwas, wie die Grundelemente eines weltweiten Expertensystems"\cite{bonitz_1990_sci}. Eine häufige Zitation stellt dabei einen Indikator für einen große Wirkung der wissenschaftlichen Arbeit dar. Ein generalisierter und überzeitlicher Begriff von Qualität wissenschaftlicher Arbeit scheint nicht möglich, weil es Schwierigkeiten mit dem Begriff und ein Grudproblem der Wissenschaftsindikatoren sowie ihrem Ziel der "Abbildung eines Konstruktes, das die Bewertungen einzelner Wissenschaftler oder Experten transzendiert" betrachtet gibt \cite{Hornbostel_1997}.

In den letzten Jahren haben sich die "Umweltbedingungen" für die Qualitätssicherung geändert, dabei haben vor allem die "Anforderungen an Verfügbarkeit von Dokumenten und Transparenz der Begutachtungen" der Open Access Bewegung die Frage aufgebracht, "ob möglicherweise Veränderungen der Review-Praktiken notwendig sind, um exzellente Wissenschaft zu identifizieren und vor allem zu fördern" \cite{suchen_Hornbostel_2006}.

\subsection{Wissenschaftliches Kapital}
Die Wissenschaft ist ein soziales Feld, dessen Strukturen und Praktiken das bestimmten, was als Wissenschaft und als wissenschaftliches Ergebnis gilt \cite{mikl_2010_soziologie}. Im Rahmen der Betrachtung von Steuerungs- und Reputationsmethoden für die Wissenschaft ist der Begriff "wissenschaftliches Kapital" von herausragender Bedeutung \cite{suchen}. Wissenschaftliches Kapital kann als eine Ausprägung des kulturellen Kapitals und als symbolisches, beziehungsweise non-monetäres Kapital \cite{irmer2011} verstanden werden. Symbolisches Kapitel wird von der Sozilogin Mikl-Horke als Besitz an symbolischen Gütern beschrieben, "der besonders in einer Gesellschaft, die auf die Kooperation aller angewiesen ist, sehr kostbar ist"\cite{mikl_2010_soziologie}.

Die "Gewährung wissenschaftlichen Kapitals" basiert heute auf einer engen Verbindung beziehungsweise Kooperation zwischen publizierenden Wissenschaftlern und Verlagen \cite{herb_2006}. Die Wissenschaftler befinden sich in einer Abhängigkeit von den Verlagen. Ulrich Herb definiert mit Hilfe Pierre Bourdieus, "wissenschaftliches Kapital" als “Ergebnis einer Investition (...), die sich auszahlen muss” \cite{herb_2006}. “Diejenigen, die diese Berechtigungsscheine in der Hand halten, verteidigen ihr 'Kapital' und ihre 'Profite', indem sie diejenigen Institutionen verteidigen, die ihnen dieses 'Kapital' garantieren.” \cite{Bourdieu_1992} Herb kommt zu dem Schluss, dass die Öffnung der Wissenschaft bisher nicht wissenschaftlicher Logik folgt, "sondern einer feldunabhängigen Logik der Akkumulation von Kapital" \cite{herb_2006}. Insbesondere das deutsche Wissenschaftssystem ist dabei zunehmend von der Einführung an Output orientierter Anreizsysteme \cite{osterloh2008anreize} und einem Ungleichgewicht in der Kooperation zwischen wissenschaftlicher Kommunikation und wissenschaftlichen Kapital gekennzeichnet.

Der Soziologe Bordieu unterscheidet diesbezüglich zwei Typen von wissenschaftlichen Kapital \cite{Bourdieu_1998}. Den einen, der auf der politischen und insitutitonellen Macht beruht und den anderen, dass durch rein wissenschaftliche Anerkennung ensteht \cite{mikl_2010_soziologie}. Bourdieu nennt Zitationsindexe als einen Indikator für die wissenschaftliche Kapital \cite{Bourdieu_1998}. Die wissenschaftliche Reputation, die aus dem wissenschaftlichen Kapital resultiert, basiert auf der Liste der Publikationen in hoch gerankten Journalen und angesehenen Verlagen \cite{herb_2010}. Diese Bewertung ist symbolischer Natur und basiert "auf der Anerkennung und dem Kredit (...), den die Gesamtheit der Wettbewerber innerhalb des wissenschaftlichen Feldes gewähren" \cite{Bourdieu_1998} \cite{herb_2010}.

"Wissenschaftliches Kapitel" ist zunehmend der Kapitalisierung von Wissenschaft ausgesetzt, bei der um den Einfluss der Ökonomie und den "wissenschaftswidrigen Verwertungsdruck". \cite{suchen_Hornbostel_2006} Als ein Indikator ist die Kopplung des wissenschaftliches Kapitals und an Output-orientierte Anreizsysteme. Ein Beispiel ist der Performanzindikator "Drittmittel" \cite{Jansen_2007}, bei dem neben der Sicherung der Qualität von Forschung und Lehre zunehmend direkte finanzielle und administrative Kontrolle eine Rolle spielt \cite{Barl_sius_2008}. Dem Drittmitteleinkommen wird als Indikator für Forschungsleistung eine hohe Bedeutung zugemessen \cite{Jansen_2007}. Daraus entsteht die Tendenz, das nicht nur die Erwartungen an die Bewertung von Wissenschaft ambitioniert sind, sondern auch, dass die Interessen privater und öffentlicher Drittmittel-Auftraggeber in den Vordergrund rücken. Ähnliches ist im Rahmen der leistungsbezogenen Mittelzuweisungen an die Universitäten zu beobachten \cite{suchen_Hornbostel_2006}. Vor allem die Verknüpfung von wissenschaftlicher Reputation mit der damit einhergehenden Verteilung der Mittel und Stellen stellt eine neuartige Herausforderung an das Wissenschaftsystem dar, “dessen Währung [ursprünglich] nicht Geld ist” \cite{hanekop_2006} \cite{suchen_Hornbostel_2006}.

\subsection{Freiheit von Wissenschaft, Lehre und Forschung}

"Die Autonomie der Wissenschaft wird nach aussen durch die Abhängigkeit der Universität vom Staat und universitätsintern durch die Einheit von Wissenschaft und Forschung gesichert" \cite{Huber_2005}. Diese Wahrung ist im Artikel 5 Abatz. 3 GG garantiertes Grundrechts wie folgt festgehalten: "Wissenschaft, Forschung und Lehre sind frei". Dieses Recht ist nicht nicht nur ein Grundrecht auf wissenschaftliche Meinungsfreiheit, sondern auch eine rechtliche Garantie. "Jeder, der in Wissenschaft, Forschung und Lehre tätig ist, hat - vorbehaltlich der Treuepflicht gemäß Art. 5 Abs. 3 Satz 2 GG - ein Recht auf Abwehr jeder staatlichen Einwirkung auf den Prozeß der Gewinnung und Vermittlung wissenschaftlicher Erkenntnisse", so das Bundesverfassungsfgericht weiter. Dennoch " Das garantiert einerseits die Einrichtung wissenschaftlicher Hochschulen mit Anspruch auf Selbstverwaltung, die staatliche Finanzierung und Sicherung ihrer Arbeit. Andererseits richtet es sich als "Abwehrrecht auf die Abwehr von Eingriffen in die wissenschaftliche Betätigung" gegen staatliche Eingriffe \cite{mayen_grundrechte_forscher} \cite{spindler_2006_rechtloa}. Jede Form der wissenschaftlichen Betätigung ist durch dieses Abwehrrecht geschützt, dazu zählen laut Urteil des Bundesverfassungsgerichts "vor allem die auf wissenschaftlichen Eigengesetzlichkeiten beruhenden Prozesse, Verhaltensweisen und Entscheiden bei dem Auffinden von Erkenntnissen, ihrer Deutung und Weitergabe" - also auch die Möglichkeit zur freien Entscheidung über die Veröffentlichung von Forschungsergebnissen (Publikationsfreiheit) \cite{Fangerau_2014}.

Demgegenüber steht die "Entmythologisierung" der Humboldtschen Idee der "Einheit von Forschung und Lehre" in der Universität hat nicht erst im Rahmen des aktuell steigenden Kosten- und Effizienzdrucks, der "Verwertbarkeit" von Wissenschaft und Forschung, sowie der Modernisierung der Steuerungsmechanismen stattgefunden. Die Einheit von Forschung und Lehre, auf Grundlage des völligen Verzichts auf Differenzierung, lässt sich grundsätzlich nur in Ausnahmefällen realisieren \cite{Schimank_2001}. Als realistische Lesart kann eine situative Differenzierung und die Mittel der Grundausstattung sind nicht nach beiden Aufgaben separiert \cite{Schimank_2001}. Diese realisitsche Lesart der Humboldtsche Idee ist noch immer hegemonialer Rahmen der aktuellen Hochschulreformen \cite{Huber_2005}. Das Recht auf Freiheit von Lehre und Forschung und die humboltsche Idee der Universität wird und wurden dabei immer wieder für die Erhaltung des "organisationellen Status Quo", die Absicherung der "Insitution Universität" und die Wahrung der "Staatsunabhängigkeit" angebracht \cite{Huber_2005}. Schon Kant und Nietsche kritisierten dabei die Ausrichtung der Universität auf die Verwertbarkeit wissenschaftlichen Wissens \cite{Huber_2005}. Diese Autonomie der Wissenschaft und Forschung gilt auch heute als "hohes Gut, das es gegen externe Anforderungen zu verteidigen gilt"\cite{kaldewey_2010}.

In Hinblick auf die wissenschaftliche Publikation kann also festgehalten werden, dass Hochschullehrer nicht von der Hochschule oder anderer staatlicher Institutionen gezwungen werden können, über einen bestimmten Weg oder Kanal zu veröffentlichen \cite{spindler_2006_rechtloa}. Aussnahme stellen hier nur die privatfinanzierten Drittmittelprojekte dar, da sich der Hochschullehrer hier nicht auf die Wissenschaftsfreiheit als Abwehrrecht gegen den Staat berufen kann \cite{spindler_2006_rechtloa}. Wissenschaftlichen Mitarbeiter und Mitarbeiterinnen "müssen ihrer Hochschule die Nutzungsrechte an ihrer Publikation einräumen", es sein denn, sie habe sie nicht nach Weisung des Lehrstuhl- oder Institutsleiters erarbeitet oder es handelt sich um eine Dissertation oder Habiliation \cite{spindler_2006_rechtloa}. Ein direkter staatlicher Eingriff im Rahmen einer Richtlinie zum Publikationszwang über einen bestimmten Weg scheint mit der Wissenschafts- und Publikationsfreiheit nicht vereinbar. Dennoch kann der Staat Anreizsysteme oder Rahmenbedingungen zu schaffen, die die Öffnung des wissenschaftlichen Kommunikations- und Publikationssystems befördern.

\subsection{Ökonomie der wissenschaftlichen Kommunikation}
Die klassische Ökonomie der wissenschaftlichen Kommunikation beruht auf der Durchsetzung von Urheberrechten, die den Zugriff auf und die Wiederverwendung von geschützten Inhalten beschränken, sowie die Zahlung einer Gebühr durch den Leser verlangen, um Zugang zu der Veröffentlichung zu erhalten \cite{CREATe_2014}. Das gilt vor allem für die Veröffentlichung wissenschaftliche Erkenntnisse. Bislang werden dafür "in der Regel wissenschaftliche Arbeiten zwar mit öffentlichen Mitteln finanziert, aber von privaten Verlagen in Fachzeitschriften herausgegeben" \cite{WD_bundestag_2009}. Diese Ökonomie der Wissenschaftsverlage ist nicht neu und hat sich im Laufe der Zeit weiter ausdifferenziert.

Wissenschaftliche Inhalte werden heute über drei Arten zur Verfügung gestellt \cite{cope2014future}:
\begin{enumerate}
\item Wissen als Inhalt zum Verkauf - Der größte Anteil wissenschaftlicher Publikation wird über diese Art vertrieben. Allein für die STM-Fächer (Science, Technology, Medicine) wird in der Literatur von einem Markt von 6 Milliarden Dollar für wissenschaftliche Zeitrschriften ausgegangen \cite{cope2014future}.
\item Wissen als kostenlose Ressource
\item Wissen als bei der Produktion bezhalte Ressource
\end{enumerate}

Das wissenschaftliche Publikationsmodell basiert auf einer "sozial ineffizienten" Ebene \cite{mueller-langer_2010} --- TODO: ZITAT prüfen ----. Die Wahrnehmung der Unverhältnismäßigkeit dieses Systems, insbesondere der Preisgestaltung für wissenschaftliche Publikationen \cite{King_2008} findet erst seit kurzem statt\cite{CREATe_2014}.

Eine weitere wesentliche Besonderheit der Wissenschaftskommunikation ist die Organisation des Marktes, die von spezifischen Akteuren und Prozessen geprägt wird \cite{Hess_2006}. Das wissenschaftliche Publizieren kann als "gesellschaftlich bedingter Kreislauf" \cite{schirmbacher_2009_wisspub} betrachtet werden. Der klassische wissenschaftliche Kommunikationsprozess im Rahmen von Publikationen kann wie folgt unterteilt werden \cite{cite:11b} \cite{Hess_2006}:
\begin{enumerate}
\item Erstellung durch Wissenschaftler - Inhalte erzeugen:
Der Kreislauf beginnt mit der Darstellung der geistigen Werke durch die Autoren\cite{schirmbacher_2009_wisspub}. Nach der Entwicklung eines konkreten Forschungsvorhabens sowie einer wissenschaftlichen Fragestellung enstehen im Rahmen der wissenschaftlichen Forschung oder der jewiligen Untersuchung Informationen\cite{cite:11c}, die im Forschungsprozess gesammelt, analysiert, ausgewertet, aufbereitet und verschriftlicht werden\cite{cite:11d}. Diese Infromationen werden strukturiert zusammengefasst und niedergeschrieben \cite{Hess_2006}.
\item Qualitätskontrolle durch Wissenschaftler - Inhalte bewerten:
Die Qualitätskontrolle ist einer der wesentlichen Bestandteile der wissenschaftlicher Kommunikation. Sie sichert die gewonnen Erkenntnisse\cite{cite:11e} und stellt einen klaren Abrenzungsaspekt zu nicht-wissenschaftlichen Informationen und Erkenntnissen dar\cite{cite:11f}. Sie findet im Kommunikationsprozess an zwei Punkten des Prozesses statt. Hier wird der Bereich addressiert, der vor der Produktion der Informationen in Form einer Publikation, die Erkenntnisse von anderen Wissenschaftlern überprüft und sichert (Peer-Review) \cite{Hess_2006} sowie vom Verlag organisiert wird \cite{schirmbacher_2009_wisspub}.
\item Bündelung durch Verlage - Inhalte auswählen:
Verlage bündeln in Zusammenarbeit mit Wissenschaftlern und kuratieren die wissenschaftlichen Inhalte für die letztendliche Publikation.
\item Publikation durch Verlage - Inhalte distribuieren:
Nach Erstellung und Erkenntnissicherung findet "eigentlichen Publikation" \cite{schirmbacher_2009_wisspub} der Informationen statt. Bis zur Digitalisierung bestand dieser Schritt ausschließlich in dem Druck der Inhalte auf Papier.\cite{cite:11h}
\item Distribution durch die Verlage:
Der Vertrieb und die Verbreitung von Forschungsergebnissen ermöglicht den Zugriff auf die Information durch andere Wissenschaftler. Dieser Schritt stellt einen essenziellen Teil der Zirkulation und Kommunikation des neu gewonnen Wissens dar\cite{cite:11i}. Er sichert die Verfügbarkeit und die Möglichkeit des Zugriffs auf die Informationen und ist Teil des Selektionsprozesses für die Erschaffung neuen Wissens.\cite{cite:11l}
\item Support und Archivierung:
Dieser Schritt beinhaltet die Erschließung, Aufbewahrung und Bereitstellung der Publikation durch Bibliotheken \cite{schirmbacher_2009_wisspub}. Die Bibliotheken unterstützen den Wissenschaftler und die Institution bei einem weiteren wesentelichen Aspekt von Wissenschaft: der Bewahrung und die Archivierung von Wissen \cite{K_lbel_2002}.
\item Konsum, beziehungsweise Rezeption durch Wissenschaftler:
Die Rezeption der veröffentlichten Inhalte durch die wissenschaftliche Gemeinschaft \cite{schirmbacher_2009_wisspub} ist der letzen Schritt des wissenschaftlichen Kommunikationsprozesses. In diesem Schritt entsteht durch den Vergleich neuer Ergebnisse mit bereits publizierten wissenschaftliche Qualität \cite{umstatter_2007_qualitatssicherung}. Aus der Mitte der rezipierenden wissenschaftliche Gemeinschaft schaffen darauf aufbauend andere Autorinnen und Autoren ein neues Werk \cite{cite:11k} \cite{schirmbacher_2009_wisspub} und der Kommunikationsprozess beginnt von vorn.
\end{enumerate}

An diesem Prozess sind drei Gruppen beteiligt: erstens die Wissenschaftler, als Produzenten und Konsumenten der Informationen, zweitens die Verleger, die als Intermediäre wissenschaftliche Informationen sammeln, bündeln und verkaufen, sowie drittens die Bibliotheken, die die Informationen wieder den Wissenschaftlern zur Verfügung stellen \cite{Odlyzko_1997}. Wissenschaftler stehen dabei an einer komfortablen Stelle des wissenschaftlichen Produktions- und Distributionssystems\cite{herb_2010}. So ist für viele wissenschaftliche Autoren und Leser ist Offenheit im Kommunikationssystem kein großes Thema, da sie häufig über sehr gute Zugangsmöglichkeiten zu wissenschaftlichen Informationen durch ihre Forschungsinsititutionen verfügen \cite{cope2014future}. Sie werden an staatlichen, wissenschaftlichen Insitutionen größtenteils durch öffentliche Gelder finanziert und erhalten durch die Bibliotheken ihrer Insitution Zugang zu wissenschaftlichen Publikationen. Sie schreiben Texte für die wissenschaftlichen Verlage, und werden mit im Rahmen der Veröffentlichung mit Reputation "belohnt". Aus der Perspektive der Forschungsförderer, sind die Verlagen sind die einzige voll-privatwirtschaftliche Gruppe und die einzige Gruppe, die Ressourcen aus dem System extrahiert, ohne dass diese Ressourcen dem Kreislauf der Wissenschaftskommunikation vollumfänglich dienen \cite{kiley_2006_open}.

\subsection{Wissenschaftlicher Diskurs nach dem Diskurs- und Machtbegriff}

Die reine Forschung ist nur ein Bestandteil des wissenschaftliche Diskurs\cite{suchen}. Verarbeitung von Forschungsergebnissen, der Anwendung und Neuinterpretation von Ergebnissen, verfassen von Gegenentwürfden und syntheritscehr Gesamtdarstellungen gehören ebenfalls zum wissenschaftlichen Diskurs \cite{suchen}. Jürgen Habermas unterschied kommunikativen Handelns von strategischem Handeln. Im dem "rationalen Diskurs" findet eine Verständigung über problematische Geltungsansprüche statt \cite{suchen}. Dabei entwickelt der Beobachter Methoden und Verfahren um zu einer Verständigung zu kommen. \cite{suchen} Nach Niklas Luhmann operiert der wissenschaftliche Diskurs funktional eigenständig und alles, was durch Wissenschaft kommuniziert wird, ist “entweder wahr oder unwahr” \cite{Luhmann1998}.  Dabei können in einem Diskurs individuelle Dispositionen das Erkenntnis verfälschen.

Im folgenden wird von einem philosophischen Zugang zum Diskursbegriff gewählt und dem Verständnis von einem Diskurs von Michel Foucault geflogt. Dieser versteht unter einem Diskurs "eine Menge von Aussagen, die einem gleichen Formationssystem zugehören"\cite{foucault_archaologie_1981}. Der wissenschaftliche Diskurs gründet sich nur zum Teil auf die Forschung und kann auch nicht nur als “Kontaktglied zwischen dem Denken und dem Sprechen” \cite{foucault_ordnung_2003} definiert werden. Er wird getrieben vom Wille zur Wahrheit der sich durch "die Pädagogik, dem System der Bücher, der Verlage und Bibliotheken, den gelehrten Gesellschaften einstmals und den Laboratorien heute" ständig erneuert \cite{foucault_ordnung_2003}. Abgesichert wird er "durch die Art und Weise, in der das Wissen in einer Gesellschaft eingesetzt wird, in der es gewertet und sortiert, verteilt und zugewiesen wird"\cite{foucault_ordnung_2003}. In der Foucault'schen Diskursanalyse wird der Diskurs als die Fähigkeit definiert, die “Beziehungen” zwischen “Institutionen, ökonomischen und gesellschaftlichen Prozessen, Verhaltensformen, Normsystemen, Techniken, Klassifikationstypen und Charakterisierungsweisen herzustellen”\cite{foucault_archaologie_1981}.

Menschen versuchen mit diversen "Machprozeduren" die "ungeordnete und wuchernde Masse aller Äußerungen" zu reglementieren und zu kontrollieren \cite{Neymeyer_diskurs_2010}. Resultierend daraus entstehen Diskurse, die sich über einen gemeinsamen Gegensatnd definieren, impliziten wie expliziten Regeln gehorchen, speziefischen Fuktionen unterliegen, bestimmte Formen annehmen und die von Machtmechanismen gekennzeichnet sind." \cite{Neymeyer_diskurs_2010}

Foucault beschäftigt sich aber auch mit den Grenzen dieser Diskurse, sowie dessen institutioneller und praktischer Verortung. Im Gegensatz zur innerdiziplinären Betrachtung eignet sich Foucaults “Werkzeugkiste”\cite{Honneth_2003} dabei besonders für die Evlalutation der transdisziplinären Öffnung wissenschaftlicher Prozesse, sowie die damit einhergehende Öffnung des Diskurses theoretisch zu hinterfragen.

---- TODO: Weiter ausarbeiten In diesem Kapitel soll deshalb der Diskursbegriff in den Kontext der Thematik der Öffnung des Zugriff auf den wissenschaftlichen Prozess erläutert werden -----
