\chapter{Grundlagen, Definitionen und Abgrenzungen}

Der theoretische Bezugsrahmen wissenschaftlich gesicherter Modelle, Theorien und Ansätze ermöglicht es, Erklärungen und Handlungsempfehlungen abzuleiten \cite{martin_2007_wissenschaftstheorie}. Er trägt dazu bei, die Fragestellungen in einen Zusammenhang zu stellen, legitimiert die Erforschung dieser Fragen und bildet den Rahmen für die Auswertung gesammelter Erkenntnisse \cite{suchen}. Ziel dieses Kapitels ist es, die theoretischen Grundlagen für die im späteren Verlauf der Arbeit folgende Inhaltsanalyse und die empirischen und experimentellen Ergebnisse zu erarbeiten, sowie die Begriffe, Definitionen und Konzepte, die für das Thema der vorliegenden Arbeit grundlegend sind, einzuführen.

Die Öffnung von Wissenschaft und Forschung wird in dieser Arbeit im Kontext wissenschaftlicher Reputation in Bezug auf ihre technischen, gesellschaftlichen und politischen Aspekte beschrieben und die Betrachtung wird auf die kulturellen Auswirkungen der Medienbrüche im Rahmen wissenschaftlichen Publizierens erweitert. Der historische und gesellschaftliche Kontext ihrer Anwendung wird dargestellt und mittels der Analyse wissenschaftlicher Literatur abgegrenzt. Es wird erläutert, welche Bedeutung sie in der Forschung, der Gesellschaft und der Politik haben. Die Entstehung und Entwicklung der Begriffe wird im Verlauf der Arbeit beschrieben. Um ein möglichst umfassendes Bild zu erhalten, wird "Entwicklung" hier in den drei folgenden Dimensionen erfasst: erstens, als "analytische Kategorie", zweitens als "Forschungsgegenstand" und drittens als "politische Praxis in der moralischen Auseinandersetzung über die Wünschbarkeit von Zuständen" \cite{cite:10}.

Wenn es im aktuellen öffentlichen Diskurs um die Öffnung wissenschaftliche Informationen, Infrastruktur und Arbeiten geht, werden immer öfter Schlagworte mit dem Attribut „Open“, wie Open Access, Open Research und Open Science, verwendet \cite{bunz_2014} \cite{schulze_2013_open}. "Offen" ist dabei nicht mit "kostenlos" gleichzusetzen \cite{grand_2012_open} und bezieht sich üblicherweise auf zwei Kernaspekte: Zum einen die Offenheit des Zugangs zu wissenschaftlichen Text, Daten, Quellcode oder Ergebnissen und zum anderen auf das Gebot der Transparenz, also die Offenlegung, beziehungsweise der Zugriff auf Verfahren, Methoden und Ziele \cite{schulze_2013_open}. "Offenheit" (Openness) wird im Rahmen dieser Arbeit multidimensional adressiert und hat eine rechtliche, wirtschaftliche, technische, politische und kulturelle Dimension.

Während "Openness" bisher vielfach mit den Entwicklungen rund um offene Software assoziiert wird, gibt es andererseits Anknüpfungspunkte von "Offenheit" als Begriff in der wissenschaftlichen Auseinandersetzung die schon früher anzusetzen sind \cite{Tkacz_2014}. So sieht Christopher Kel­ty die ersten Anfänge bereits in den 1980ern \cite{kelty_2008_two_bits}. Andrew Russell sieht die ideologischen Ursprünge von "Offenheit" als Standard schon in der Entwicklung des Telegraphs und weiteren Ingenieurleistungen seit 1860 \cite{Russell_2014} und Könneker und Lugger sehen erste Beispiele einer offenen Wissenschaft bereits im 17. Jahrhundert \cite{Konneker_2013}.

Die Analysen in dieser Arbeit werden aus der Perspektive des Produzenten (Wissenschaftler als Autoren) sowie aus der, damit nicht immer harmonisierenden, Perspektive des Rezipienten, beziehungsweise Medienkonsumenten (Wissenschaftler als Leser) stattfinden. Es wird auch adressiert, inwiefern Macht, regulierende Prinzipien wie die Verknappung, sowie die Ein- und Ausgrenzung im Rahmen wissenschaftlicher Diskurse mit den Modellen Open Access, Open Science und wissenschaftlicher Reputation in der Kommunikation vereinbar sind oder diesen gegenüberstehen.

Die Themenbereiche kollaboratives Arbeiten, Social Media in Wissenschaft und Forschung, Citizen Science und aktuelle Diskurse zu Tools und Diensten werden in dieser Arbeit bewusst nur am Rande und beiläufig andressiert, beziehungsweise nur eingeschlossen, wenn sie die Beantwortung der Forschungsfragen dienen oder diese tangieren.

\section{Wissenschaftliche Kommunikation}

Bevor die Grundlagen für Offenheit in Wissenschaft und Forschung definiert werden, wird eine grundlegende Einordnung von wissenschaftlicher Kommunikation vorgenommen sowie deren Wandel im Rahmen der Digitalisierung beschrieben.

Kommunikation stellt einen wesentlichen Bestandteil des wissenschaftlichen Systems und der wissenschaftlichen Arbeit dar \cite{garvey_2014_communication} \cite[:63]{Luhmann1998}. Sie basiert auf dem Austausch zwischen Wissenschaftlern, die auf einem "gemeinsamen Wissensbestand" zugreifen, "den sie testen, verändern und erweitern" \cite{Gl_ser_2007} und ist eng mit dem "Prozess des Veröffentlichens wissenschaftlichen Publikationen" \cite{weller2011twitter} verknüpft. Sinn und Zweck der Kommunikation beruht auf dem bestmöglichen Austausch zwischen den Mitgliedern der Wissenschaftsgemeinschaft. Er dient der Überprüfung der Zuverlässigkeit von Informationen und ermöglicht die kritische Auseinandersetzung innerhalb der Gemeinschaft \cite{fox_1983_publication}. Jede kommunizierte Erkenntnis trägt dabei theoretisch zur Produktion von Wissen bei \cite{kaden_2009_library}. Grundvoraussetzung dafür ist, dass Wissenschaftler und Wissenschaftlerinnen den Willen zu optimalen Kommunikation untereinander haben.

Es existieren verschiedene Arten wissenschaftlicher Kommunikation und "vielfältige Erscheinungsformen" \cite{graefen2007_wissenschaftliche_artikel}, die sich im Laufe der Zeit immer wieder verändert haben \cite{Konneker_2013}. Grundsätzlich ist die Unterscheidung in \textit{formelle} und \textit{informelle}, sowie die \textit{interne} und \textit{externe} wissenschaftliche Kommunikation gängig.

Was genau als \textit{formell} oder \textit{informell} gilt, hängt unter anderem von der jeweiligen Fachdisziplin ab, "ist historisch gewachsen und damit durchaus unterschiedlich" \cite{Hanekop_2014}. Eine wesentliche Plattform für die wissenschaftliche Kommunikation, Fortschritt und Forschungsförderung bilden Publikationen in Journalen und Monographien \cite{cope2014future} \cite{fox_1983_publication}. Das wissenschaftliche Journal ist (in den meisten wissenschaftlichen Disziplinen) ein wichtiger Kanal für die \textit{formelle} wissenschaftlichen Kommunikation und essenziell für Wissenschaftler und Wissenschaftlerinnen um auf dem Laufenden zu bleiben \cite{cope2014future}.

Die \textit{formelle} Kommunikation wird an bestimmte Bedingungen der wissenschaftlichen Gemeinschaft geknüpft und hat einen direkten Einfluss auf die Reputation der einzelnen Mitglieder der wissenschaftlichen Community. Diese Art der Kommunikation beinhaltet die Einbeziehung Dritter, die die Funktion der Einordnung und Bewertung der Kommunikation übernehmen. Der bisherige Outputkanal für diese Kommunikation ist die gedruckte Publikation, denn "es wird für den Druck geforscht" \cite{luhmann_1997_gesellschaft}. Durch sie "wird festgeschrieben, was nach den Kriterien des jeweiligen Fachs als geprüftes Wissen gelten kann" \cite{bbaw_publizieren_2015}. Ziel dieser Art der Kommunikation ist die Sicherung des Verbleibs und die Positionierung des einzelnen Wissenschaftlers innerhalb der wissenschaftlichen Gemeinschaft. Diese Formalisierung der Kommunikation ist wichtig um das Wissenschaftssystem strukturell und nachhaltig zu sichern und sie macht Erkenntnisprozesse nachweisbar \cite{kaden_2009_library}. Erst mit einer formell begutachteten Publikation wird eine wissenschaftliche Entdeckung als solche erkennbar \cite{brembs2015open}.

Diese Form der wissenschaftlichen Kommunikation wird zumeist anhand quantitativer bibliometrischer Methoden evaluiert und seit der Entwicklung des Science Citation Index (SCI) sowie des Aufkommens systematischer Wissenschaftsevaluation in Form von Rankings zunehmend von Autoren, Wissenschaftlern, Lesern, Verlagen und Herausgebern für die Evaluation der Wirkung der Kommunikation akzeptiert und adoptiert \cite[:2]{haustein_2012_multidimensional}.

\textit{Formelle} wissenschaftliche Kommunikation beruht nach dem Bibiliothekswissenschaftler Ben Kaden auf drei Faktoren \cite{kaden_2009_library}:
\begin{enumerate}
\item \textit{Publizität} meint die Veröffentlichung der Erkenntnisse in einem wissenschaftlichen Fachmedium. Eine Erkenntnis wird durch die Veröffentlichung bekannt gegeben und so für die Community "registriert" \cite{kaden_2009_library}. Sie muss dabei "zeitnah" in einer "wahrnehmbaren" Form vorliegen \cite{Schimank_2012}, damit sie intersubjektiv vermittelbar ist.
\item \textit{Vertrauenswürdigkeit} meint das Vertrauen auf die Einhaltung der Regeln im Kommunikationssystem durch alle Teilnehmer. Das Vertrauen wird bei einer Publikation durch die Überprüfung (Peer-Review) bestätigt und durch Bezugnahme (Zitationen) anderer Wissenschaftler auf die Publikation zu Reputation. Eine Zitation ist - aus Sicht der zitierten Arbeit - eine formelle Erwähnung der Arbeit innerhalb in einer anderen wissenschaftlichen Publikation \cite{weller2011twitter}.
\item \textit{Zugänglichkeit} bezieht sich auf die dauerhafte Sicherung und Zugänglichkeit in einer allgemein verfügbaren Form für die Fachöffentlichkeit.
\end{enumerate}

Die Möglichkeiten der \textit{informelle} Wissenschaftskommunikation sind höchst vielfältig und reichen "vom persönlichen Gespräch über Vorträge, Konferenzen, Zwischen- oder Abschlussberichte aus Projekten, Working Papers und vieles andere mehr"\cite{Hanekop_2014}. \textit{Informelle} Kommunikation umfasst alle Arten der Kommunikation, die dem individuellem Wissenschaftler einen schnellen und direkten Austausch mit Kollegen ermöglichen und die keinen direkten Einfluss auf die wissenschaftliche Reputation des einzelnen Wissenschaftlers haben.

Diese Art der Kommunikation steht im wissenschaftlichen Wertschöpfungsprozess meist am Anfang. Sie umfasst zum Beispiel die Ideenfindung, die Entwicklung von Fragestellungen oder Konkretisierung des Forschungsvorhabens und hilft Wissenschaftlern dabei relevante Ideen für formelle Kommunikation "herauszukristallisieren" \cite{Hanekop_2014}. Informelle Kommunikation ist auf Grund ihrer Heterogenität und impliziten Verankerung weniger präzise differenzierbar und erfassbar \cite{kaden_2009_library}. Die Abgrenzung informeller Kommunikation zu "nicht-wissenschaftlicher Kommunikation" resultiert daraus, dass sie meist auf "die Erzeugung formeller Kommunikation hinarbeitet" \cite{kaden_2009_library}.

Im Gegensatz zur Segmentierung von \textit{formeller} und \textit{informeller} Kommunikation, zielt die Unterscheidung zwischen \textit{interner} und \textit{externer} Kommunikation auf die jeweilige Zielgruppe des Austauschs ab. \textit{Interne} Kommunikation beschreibt alle Prozesse die der Kommunikation innerhalb der wissenschaftlichen Gemeinschaft dienen. \textit{Externe} Kommunikation beschreibt die Kommunikation, die an Akteure ausserhalb der wissenschaftlichen Gemeinschaft gerichtet ist \cite{Konneker_2013}.

\subsection{Digitalisierung der wissenschaftlichen Kommunikation}

Als Digitalisierung werden folgend Fortschritte im Kommunikationssystem bezeichnet, die durch die Entwicklung elektronischer Informations- und Kommunikationstechnologien angestoßen wurden \cite{bbaw_publizieren_2015}. Diese Fortschritte lassen das wissenschaftliche Publikationswesen und die wissenschaftliche Kommunikation nicht unberührt \cite{naeder_2010_open}. Wie bereits in der Einleitung dieser Arbeit dargestellt, üben die Digitalisierung und die dahinterstehenden Technologien einen tiefgreifenden Einfluss auf die wissenschaftliche Prozesse in allen Fachdisziplinen aus, die in Verlauf dieser Arbeit genauer untersucht werden.

Dieser Einfluss ergibt sich aus einer der wichtigsten Unterschiede der digitalen Kommunikation im Vergleich zu analogen Kommunikation. Digital kommunizierten Inhalte sind im Vergleich zu analogen Inhalten weder endgültig noch endlich und weder im Kern noch in Form fixiert, denn sie können leicht geändert werden und das ohne Spur von Löschung oder Korrektur \cite{smith_1999_digitize}. Aus digitalen Informationen können eine endlose Anzahl von identischen Kopien erstellt werden, ohne dass ein Zerfallsprozess eintritt \cite{smith_1999_digitize}. Ergänzt durch die Möglichkeit diese Informationen in einem weltumspannenden Netzwerk in nahezu Echtzeit unabhängig von Lokation und Zeit zu transportieren, haben diese fundamentalen Veränderungen für die Informationsspeicherung, -kommunikation und -verbreitung auch einen direkten Einfluss auf die wissenschaftliche Kommunikation, die bis zu diesem Zeitpunkt ausschließlich auf dem Austausch analoger Medien und Kommunikation basierte. Die weitgehende Verlagerung des wissenschaftlichen Kommunikationsprozesses in die digitale Welt führt dazu, dass mittlerweile über 90 Prozent der englischsprachigen Journale online verfügbar sind und es einen ansteigenden Trend zu Journalen gibt, die nur im digital abrufbar sind \cite{cope2014future} \cite{cite:5}.

Mit diesem digitalen Wandel in der wissenschaftlichen Kommunikation wird die Chance für eine umfassende “Beschleunigung des Wissensumschlages” \cite{Wenzel_2003} und die Möglichkeit einer im Prinzip unbegrenzten Verbreitung aller wissenschaftlichen Publikationen \cite{bbaw_publizieren_2015} \cite{yiotis_2013_open} auch an nicht-wissenschaftliche Zielgruppen \cite{Konneker_2013} verbunden. Durch die Digitalisierung und die neuen Möglichkeiten der Dissemination befindet sich bisher vor allem die \textit{externe} und \textit{informelle} Kommunikation im Wandel. Als Konsequenz dieses Wandels sind Wissenschaftler und Wissenschaftlerinnen heute in der Lage ihre Arbeiten öffentlich auf diversen digitalen Plattformen darzustellen, sich so von der Kommunikation über professionelle, wissenschaftliche Fachmedien zu "befreien" und direkt mit Teilen innerhalb und außerhalb der wissenschaftlichen Gemeinschaft zu interagieren \cite{Konneker_2013}.

Diese ersten Veränderungen sind mit der Hoffnung verknüpft, dass offene Innovation und offene wissenschaftliche Kommunikation, sowie die veränderten Zugriffsmöglichkeiten auf wissenschaftliches Wissen \cite[:109]{naeder_2010_open} den privaten und staatlichen Forschungsbereich offener, integrativer und effizienter machen \cite{harmon_2012_commercialization}. Für die \textit{formelle} wissenschaftliche Kommunikation und das Publikationssystem fasst Johannes Nader das weitere Potenzial der Digitalsierung in folgenden vier Punkten zusammen \cite[:66-76]{naeder_2010_open}:
\begin{enumerate}
\item Ökomnomische Effizienzsteigerung und Kostenersparnis: Platzersparnis und abgesehen von der initalen Digitalsierung analoger Bestände, fallende Kosten für die Bestandserhaltung; verbesserte Verfügbarkeit
\item "Paradigmenwechsel bei der Archivierung": Effizienzsteigerung bei der Bestandserhlatung inklusive besserer Nutzung von Skaleneffekten und Dezentralisierung; identische Kopierbarkeit; Aufhebung der Nutzbestände und der Archivbestände; Trennung der Information von ihrem Trägermedium: nicht mehr die Langzeithaltbarkeit eines physischen Trägermediums ist ausschlaggebend, sondern die Erstellung von identischen Kopien
\item Veränderte und verbesserte Produktions- und Publikationsabläufe: neue Möglichkeiten der Textproduktion, -verarbeitung-, -überarbeitung und -transmission; Anreicherung von Inhalten; Autor zunehmend mit Gestaltung und Schriftsatz beschäftigt
\item Stabilisierung des wissenschaftlichen Kommunikationssystems: Kosteneinsparungen bei Produktion, Distribution, Zugänglichmachung und Archivierung; Funktionsverschiebungen vom Verlag hin zum Autor und Rezipienten lockern starre Publikations- und Wertschöpfungsketten
\end{enumerate}

Diese Vernetzung im Rahmen der Digitalisierung ermöglicht erstmals eine direkten Schnittstelle zwischen Autoren und Rezipienten, die grundsätzlich keiner Vermittlung durch Dritte mehr bedarf. Als Konsequenz dieser Veränderungen obliegt es mehr denn je dem Leser und der Leserin aus einer größeren Menge an theoretisch verfügbaren Werke, die für ihn wichtigen Informationen zu identifizieren, denn "verlagliche Mittler- und Selektionsinstanzen werden dadurch aus ihrer medialen Bindung gelöst und stehen zumindest in ihrer traditionellen Rolle zur Disposition." \cite[:109]{naeder_2010_open}.

Als ganz konkrete Veränderung erfolgte bisher mit der Etablierung der digitalen Kommunikation eine Veränderung der Kategorisierung wissenschaftlicher Kommunikation. Während im Druckzeitalter die formelle und interne wissenschaftliche Kommunikation eng an die bibliometrischen Indikatoren geknüpft war und eindeutig von der informellen und externen abgegrenzt werden konnte, scheinen diese klaren Grenzen im Rahmen der Digitalisierung zu verschwimmen, auch wenn das "jedoch nur vermittelt und mit zeitlicher Verzögerung Wirkungen auf das formelle Publikationssystem zeigt" \cite{Hanekop_2014}. Hanekop definiert diesbezüglich den folgenden Zusammenhang: "Je größer die Abkopplung zwischen informellen und formellen Aspekten der wissenschaftlichen Kommunikation in einem disziplinären, thematischen oder nationalen Wissenschaftsbereich, um so geringer, vermittelter oder langwieriger kann auch die Wirkung des Internets auf diesen Teilbereich des Publikationssystems sein" \cite{Hanekop_2014}. Ben Kaden fasst diese Veränderungen im Kommunikationssystem als \textit{kanalerweiterte Wissenschaftskommunikation} zusammen und erklärt diese als "Form der Wissenschaftskommunikation, die die informelle und formelle ergänzt" und die "individuell affirmativ" als "als eine Art informelles offenes Post Review" verstanden werde kann \cite{kaden_2009_library}.

Diese neuen Formen und Kulturen der Kommunikation führen jedoch auch zu neue Fragen in Bezug auf mögliche Ungleichgewichte und Verzerrungen innerhalb des Wissenschaftssystems und erhöhen damit auch die Herausforderung, die Auswirkungen wissenschaftlicher Kommunikation zu standardisieren und zu messen \cite{gerber_2014_science}. Diese Herausforderungen werden im weiteren Verlauf der Arbeit adressiert.

Einschränkend für den weiteren Verlauf dieser Arbeit ist festzuhalten, dass sich die Bezeichnung "wissenschaftliche Kommunikation" vornehmlich auf die Kommunikation, die formelle und interne Bezugspunkte aufweist, beschränkt und vor allem die Kommunikation gemeint ist, die einen Einfluss auf die wissenschaftliche Reputation des Wissenschaftlers oder der Wissenschaftlerin hat. Der digitale Wandel bezieht sich somit vor allem auf die folgenden drei Bereiche der formellen, internen wissenschaftlichen Kommunikation: die digitale Erstellung von Beiträgen und Texten, das Trägermedium der wissenschaftlichen Information und die Verbreitung, Vermittlung und Rezeption des Wissens \cite{bbaw_publizieren_2015}.

\subsection{Wissenschaftliche Kommunikation als Open-Source-Prozeß}

Im Rahmen der Forderung nach der Öffnung der wissenschaftlichen Kommunikation und wissenschaftlichen Publikationen werden in der Literatur immer wieder Vergleiche zur Open Source-Bewegung gezogen  \cite{cite:9} \cite{Peters_2014} \cite{RIN_2010_open_research} \cite[:423]{mantz_2007_open} \cite{cite:1}. Diese Vergleiche dienen dabei meist beispielhaft dem Verständnis theoretischer Grundlagen im Rahmen der Öffnung von Wissenschaft und Forschung.

"Open Source" ist ein Begriff aus der Softwareentwicklung der als Gegensatz zum “Verfahren der Wissenssicherung” \cite{stallman2002} eine quelloffenen Handhabe von Programmcode beschreibt und in den 1990iger erstmals eingeführt wurde \cite[:5]{hippel_2003_open}. Dieser Begriff wird, auch wenn es im Detail Unterschiede im Konzept gibt \cite[:5]{hippel_2003_open}, häufig - was ideologisch kritisch betrachtet werden kann \cite{stallman2002} - synonym mit “freier Software“ (nicht Freeware) verwendet \cite{naeder_2010_open} \cite[:414]{mantz_2007_open}. Dabei folgt die Open Source-Entwicklung der Maxime, dass die Kernsteuerungsinformationen und -befehle (Quelltext) von Software öffentlich einsehbar und zugänglich sind, sowie je nach gewähltem Lizenzmodell modifiziert, kopiert oder weitergegeben werden können.

Bei der Open Source-Entwicklung veröffentlichen Programmierer den Code einer Software offen im Internet. Andere Programmierer haben die Möglichkeit diesen Code so weiterzuentwickeln und anzupassen, wie es ihnen beliebt. Dadurch entsteht ein offenes Ökosystem an Software, bei dem nicht mehr der Zugriff die Hürde darstellt sondern die Adaption oder der Einsatz der vorhandenen Lösungen. Diese Entwicklungsmethode unterscheidet sich zum traditionellen Modell der Entwicklung von Software mit der Feststellung, dass Open Source-Software das Prinzip der Exklusivität des geistigen Eigentums auf den Kopf stellt, weil diese Software "um das Recht auf Vertrieb konfiguriert, nicht auszuschließen ist" \cite{suchen}. Auch wenn noch immer nicht vollständig geklärt ist, ob Open Source Software wirklich "schneller, besser oder günstiger" ist, hat sich Open Source in den letzten Jahren stark verbreitet hat \cite{Lerner_2001} und an Bedeutung gewonnen.

Die Definition von Open Source beinhaltet festgelegte Kriterien für die Klassifizierung \cite{osd_2003}: Freie Weitergabe ohne zusätzliche Kosten, das Programm muss den Quellcode beinhalten und den Code offen zur Verfügung stellen, die verwendete Lizenz muss Derivate zulassen, die Unversehrtheit des Quellcodes des Autors muss garantiert werden, die Diskriminierung von Personen oder Gruppen muss ausgeschlossen sein, es darf keine Einschränkung des Einsatzfeldes geben, die Lizenz muss weitergegeben werden können und auf das Produktpaket anwendbar sein und die Lizenz darf die Weitergabe des Programmcodes zusammen mit anderer Software nicht einschränken.

Im Vergleich zum klassischen Softwareentwicklungsprozess definiert der Hamburger Wirtschaftsinformatiker Markus Nüttgens folgende charakteristische Merkmale \cite{nuttgens_2014}:
\begin{enumerate}
\item Anzahl der beteiligten Entwickler: Im Vergleich zu traditioneller Softwareentwicklung ist eine weitaus größere Anzahl von Entwicklern beteiligt. Es gibt es keine klare Grenze zwischen Entwicklern und Anwendern, da die Hürden für eine Partizipation im Entwicklungsprozess sehr gering sind. Auch wenn ein großer Teil der Entwicklungsarbeit von Freiwilligen übernommen wird, gibt es dennoch den Trend zum Einsatz bezahlter Entwickler.
\item Zuteilung der Arbeit: Im Open Source Programming (OSP) wird die Entwicklungsarbeit nicht länger von einer definierten Instanz zugeteilt, sondern die Teilnehmer wählen ihre Arbeitspakete selbst aus.
\item Architektur: In der Regel orientierten sich die Teilnehmer eines OSP nicht an einer vorgegebenen System-Architektur. Die Gestaltung der Architektur geschieht dezentral und ist oftmals häufigen Richtungswechseln unterworfen.
\item Koordination: Es gibt wenig oder keine institutionalisierten traditionellen Koordinationsmechanismen, wie z.B. Projekt- und Zeitpläne, Lasten- und Pflichtenhefte u.ä.” \cite{suchen}
\end{enumerate}

Die Verknüpfung der Open-Source Entwicklungsmethode mit der Forderung nach Öffnung der wissenschaftlichen Kommunikation wurde unter anderem von dem Literaturwissenschaftler und Medientheoretiker Friedrich Kittler manifestiert \cite{cite:1}. Open Source Entwicklungsprozesse weisen Konvergenzen mit der Forderung nach der umfassenden Öffnung wissenschaftlicher Kommunikation auf, da es in beiden Fällen nicht nur um den freien und offenen Zugang zum finalen Ergebnis geht, sondern um die Möglichkeit des Zugriffs im gesamten Verlauf des Erstellungsprozesses \cite{kelty_2004}. Die Open Source Entwicklungsprozesse unterscheiden sich von den klassisch-traditionellen (closed-source) Softwareentwicklungsprozessen insbesondere durch die transparente Präsenz und permanente öffentliche Einsehbarkeit. Adaptiert man diese Open Source-Prozesse auf wissenschaftliche Wertschöpfungsprozesse und definiert in diesem Zusammenhang wissenschaftliche Publikationen als Quellcode, ist das Konzept auf den wissenschaftliche Erkenntnisprozess mindestens teilweise übertragbar \cite{garcia_2010_open} \cite{Singh_2008} \cite{Bradley_2008} \cite{mantz_2007_open} \cite{dorschel_2006_open} \cite{Bradley_2007} \cite{Willinsky_2005}. Die Vergleichbarkeit, dass das System der offenen Softwareentwicklung dem System der Erkenntnisgewinnung in der Wissenschaft ähnelt, beruht auch auf der Parallele, dass in der Wissenschaft neues Wissen auf der Grundlage von bereits vorhandenem und verfügbaren Wissen entsteht. Das gilt ebenso für Open-Source Entwicklungen, bei denen Entwickler und Entwicklerinnen häufig auf Softwareteile anderer zurückgreifen.

Ähnlichkeiten bestehen auch bei der Motivation für die Erstellung offener Software und für den wissenschaftlichen Erkenntnisgewinn. Zusammenfassend erstrecken sich die Ähnlichkeiten der beiden Prozesse auf folgende Aspekte:
\begin{enumerate}
\item Wie bei der wissenschaftlichen Kommunikation, baut die Entwicklung vieler Open Source Projekten auf den Inhalten, Steuerungsinformationen und Erfahrungen anderer Projekte auf. Die Projekte profitieren dabei von einem ständigen Austausch von Informationen, gegenseiter Optimierung und Verbesserung. Wie bei Open Source Software streben auch Wissenschaftler nach der größtmöglichen Verbreitung ihrer Inhalte.
\item "Free Software (im Sinne von Open Source), Open Access und Creative Commons sind alles Rechts- und Infrastrukturexperimente"\cite{kelty_2004}. Open Source-Software sollte dabei nicht mit "Shareware" verglichen werden, die zwar kostenlos verbreitet wird, aber deren Quellcode proprietär bleibt \cite{Lerner_2001}
\item Die Kontributoren von Open Source Projekten versprechen sich neue "Karrieremöglichkeiten oder eine Ego-Genugtuung" \cite{Lerner_2001}, Selbstverwirklichung oder Befriedigung der intellektuellen Neugier \cite{Willinsky_2005}, sowie gegenseitige Beurteilung und Anerkennung (non-monetäres Kapital). Das wissenschaftliche System basiert auf ähnlichen Mechanismen beim Karriere- und Reputationssystem.
\item Parallelen ergeben sich auch auch auf Nutzerseite: "Denn hier wie dort gilt es, das Spannungsfeld zwischen dem Prinzip des „offenen Zugangs“ auf der einen Seite und dem Wunsch mancher Urheber, die Nutzung seines Werkes – teils aus ideellen, teils aus ökonomischen Motiven – aufbestimmte „gewünschte“ Nutzungsformen zu beschränken" \cite{dorschel_2006_open}.
\item Wie bei der wissenschaftlichen Kommunikation, geht es bei der Mitarbeit an Open Source Projekten nicht ausschließlich um altruistische Motive \cite{Lerner_2001} und um kollektive bzw. arbeitsteilige Prozesse zur Wissensproduktion.
\item Die Debatte um die Forderung nach Öffnung der wissenschaftlichen Kommunikation kann aus technologisch-entwicklungsmethodischer Sicht mit der Debatte um kostenloser Software (Freeware) versus Open Source Software verglichen werden. Der Vergleich: Freeware und Open Access Publikationen sind zwar kostenlos verfügbar, ihr Erstellungsprozess wird jedoch nicht offen und transparent kommuniziert. Bei Open Science geht es wie bei Open Source um die Offenlegung des gesamten Erstellungsprozesses inklusive Daten \cite{grand_2012_open} und auch den benutzten wissenschaftlichen Code \cite{hey_2015_open}.
\end{enumerate}

Dieser Vergleich der Öffnung von Wissenschaft mit der Open-Source Bewegung wird im Rahmen dieser Arbeit als ein Ansatzpunkt erachtet, um ein mögliches Szenario auzuzeigen, wie in Zukunft die Wissensproduktion frei und öffentlich gestaltet werden kann \cite{Kuhlen_2002_universalaccess}. Dabei gilt es zu berücksichtigen, dass sich auf Seiten der Nutzung Open Access publizierter Werke und Publikationen die Erfahrungen aus dem Bereich der Open Source-Software dienlich sein können, allerdings unterscheiden sich die rechtlichen Fragestellungen und Lösungsansätze auf Anbieterseite  doch erheblich \cite{dorschel_2006_open}. Diese Einschränkung resultiert aus einer zum Teil unterschiedlichen Interessenlage: OpenSource-Software basiert in hohem Maße auf dem Community-Gedanken und ist letztlich altruistischen Motiven geprägt, während bei Open Access die Ressourcenknappheit der öffentlichen Hand sowie die individuellen Renommeeinteressen des Wissenschaftlers im Vordergrund stehen \cite{dorschel_2006_open}.

\subsection{Die Forderung nach Öffnung der wissenschaftlichen Kommunikation}

Das System der wissenschaftlichen Kommunikation, das in der derzeitigen Form seit mehreren hundert Jahren besteht, basiert auf der Forschung, der Begutachtung, dem Druck, der Kommunikation der Ergebnisse in wissenschaftlichen Publikationen, der Verbreitung sowie dem Verkauf an Bibliotheken und andere wissenschaftliche Institutionen \cite{cite:11a} und dem anschließenden Diskurs in der wissenschaftlichen Fachöffentlichkeit \cite{bbaw_publizieren_2015}. Der Fortschritt in diesem System ist demnach maßgeblich durch den freien Austausch und der Verbreitung von Informationen bedingt \cite{cite:11}.

Das aktuell vorherrschende System ist in den 1960er Jahre entwickelt worden und funktionierte am Besten, als die akademischen Ziele und mit den Marktinteressen vereinbar waren. Doch die Rahmenbedingungen wissenschaftlicher Kommunikation haben sich seitdem fundamental verändert \cite{epaa_Weiner_2001}. Infolge des weltweit steigenden Haushaltsdrucks der Bibliotheken und wissenschaftlichen Institutionen, des "ungewöhnlichen Geschäftsmodells" \cite{cite:12} der Wissenschaftsverlage mit immer höheren Margen \cite{albert_2006_open_implications}, der Massifizierung der Universität \cite{binswanger_2014_excellence}, des konstanten Anstiegs des wissenschaftliche Outputs \cite[:23]{haustein_2012_multidimensional} und des Umstandes, dass private Wissenschaftsverlage durch das wissenschaftlichen Reputationssystem über öffentlich finanzierte Wissenschaftlerkarrieren entscheiden \cite{heise_2012}, befindet sich das wissenschaftliche Kommunikationssystem in einer Krise \cite{cite:14}.

Im Rahmen der technologischen Entwicklungen bei der Digitalisierung des wissenschaftlichen Arbeitens und elektronischen Publizierens kann die Öffnung der wissenschaftlichen Kommunikation als eine mögliche Antwort auf diese Krise verstanden werden und setzt bei der Öffnung (Open) und dem freien Zugang (Access) zu wissenschaftlichen Publikationen an und könnte perspektivisch zu einer Öffnung (Open) des Zugriffs auf den Prozess des Forschens (Science) führen. Darüber hinaus werden wissenschaftliche Ergebnisse zunehmend zum Thema massenmedialer Berichterstattung, womit sich der ursprüngliche Publikumsbezug der wissenschaftlichen Kommunikation zur jeweiligen Fachgemeinschaft um den Bezug zur allgemeinen Öffentlichkeit ergänzt" \cite{bbaw_publizieren_2015} .

Diese Entwicklung birgt aber auch Herausforderungen, die der Philosoph Jean-François Lyotard als "Kommerzialisierung des Wissens" \cite{lyotard_1993_postmoderne} bezeichnet. Infolge der Konsequenzen aus der Öffnung der Kommunikation besteht demnach die Gefahr, dass "Wissen immer weniger der Bildung dient, sondern für den Verkauf geschaffen und konsumiert wird" \cite{hagner_2015_sache_buches}. Diese Gefahr beschreibt das Spannungsverhältnis in dem Wissenschaftler und Wissenschaftlerinnen auf der einen Seite zunehmend angehalten sind, die Forschung gemeinsam mit der Industire schnell in Produkte zu übersetzen und auf der anderen Seite das Wissen so schnell wie in der wissenschaftlichen Gemeinschaft verbreitet werden soll, um den wissenschaftlichen Fortschritt zu fördern sowie um die gesellschaftlichen und humanitären Ziele von Wissenschaft zu erfüllen \cite{harmon_2012_commercialization} \cite{Woelfle_2011}. Eine weitere Herausforderung stellt die Frage dar, ob und in wie weit durch die Öffnung der wissenschaftlichen Kommunikation massenmediale Selektionskriterien als Steuerungsmechanismen für Wissenschaft wirksam gemacht werden \cite{bbaw_publizieren_2015}.

Diese Veränderungen in der Kommunikation von Forschung und Wissenschaft sind keine völlig neue Phänomene, denn in gewisser Weise ist die Öffnung der wissenschaftlichen Kommunikationsprozesse eine Rückkehr zu der ursprünglichen Beziehung zwischen Wissenschaft und Öffentlichkeit \cite{Konneker_2013}.

In der gegenwärtigen Literatur finden die Begriffe um "Offenheit" in der wissenschaftlichen Auseinandersetzung auf unterschiedlichste Art und Weise statt \cite{cite:9}. Die Unterscheidung von "Zugang" und "Zugriff" erscheint dabei wesentlich und stellt eine der zentralenn Grundlagen für die Definition und Abgrenzung der hier verwendeten unterschiedlichen "Open"-Begriffe dar:
\begin{itemize}
\item Offener "Zugang" bezieht sich demnach auf einen unbeschränkten Zugang zur finalen wissenschaftlichen Publikation. Zugang meint das "das freie, unwiderrufliche und
weltweite Zugangsrecht" \cite{berliner_erklaerung_2003}. "Unbeschränkt" meint hier vor allem das ausschließliche Lesen der finalen Ergebnispublikation \cite{cite:9a} aber auch die Erstellung von Kopien, sowie Verarbeitung und Benutzung dieser \cite{Lossau_oa_2007} bei Nennung der Urheberschaft. Dieser Open-Access-Ansatz bezieht sich zunächst lediglich auf die Zugangsbedingungen zu den wissenschaftlichen Arbeiten \cite{muller_2010_open}. Dabei bezieht sich dieser Zugang auf einen Zeitpunkt nach dem der eigentliche wissenschaftliche Erkenntnisprozess abgeschlossen ist und die Publikation eingereicht oder veröffentlicht wurde.
\item Offener "Zugriff" soll als erweiterte Nutzung der jeweiligen Wissensressourcen verstanden werden und schließt neben dem "Zugang" zur Publikation sämtliche Informationen und Daten, Quellcode, sowie die Kommunikation hinter und vor der finalen Veröffentlichung \cite{cite:9b} \cite{hey_2015_open} ein. Dieser Zugriff bezieht sich als Erweiterung zu den ersten Forderungen nach "Open Access" auch auf "Daten" als "Gesamtheit der binär codierten, maschinenlesbaren Inskriptionen" und "all das, was auf digitalen Datenträgern gespeichert vorliegt" \cite{Burkhardt_2015}. "Zugriff" beschränkt sich hier also nicht nur auf den reinen Zugang zu wissenschaftlicher Information im Rahmen des Publikationsprozesses, sondern schließt auch den Zugriff auf sämtliche Forschungsdaten, Methoden und wissenschaftlichen Begleitinformationen, die während der wissenschaftliche Arbeit auf dem Weg zur finalen Publikation entstehen \cite{cite:9c} ein und ermöglicht die Weiternutzung, Weiterverarbeitung sowie die Erstellung von Derivaten durch Dritte. Im Unterschied zum "Zugang" geht es dabei auch um einen "Zugriff" auf die Informationen der weit vor den Zeitpunkt der finalen Einreichung oder Publikation liegt und unmittelbar mit Beginn des wissenschaftlichen Erkenntnisprozesses beginnt.
\end{itemize}

\subsubsection{Offener Zugang zur wissenschaftlichen Publikation: Open Access}

\begin{quote}
Der offene Zugang, auch Open Access, bedeutet, dass Peer-Review-Fachliteratur kostenfrei und öffentlich im Internet zugänglich sein sollte, so dass Interessenten die Volltexte lesen, herunterladen, kopieren, verteilen, drucken, in ihnen suchen, auf sie verweisen und sie auch sonst auf jede denkbare legale Weise benutzen können, ohne finanzielle, gesetzliche oder technische Barrieren jenseits von denen, die mit dem Internet-Zugang selbst verbunden sind. In allen Fragen des Wiederabdrucks und der Verteilung und in allen Fragen des Copyrights sollte die einzige Einschränkung darin bestehen, den Autoren Kontrolle über ihre Arbeit zu belassen und deren Recht zu sichern, dass ihre Arbeit angemessen anerkannt und zitiert wird.
\end{quote} \cite{boai_2012}

Der offene Zugang zu wissenschaftlicher Kommunikation ist seit der Entwicklung des gedruckten Wortes eng mit der Frage nach Urheberrechten für wissenschaftliche Informationen verknüpft \cite{Case_2000}. Open Access beschreibt ein wissenschaftliches Kommunikationssystem, in dem der Zugang zu den unterschiedlichsten Formen wissenschaftlicher Publikationen, im Gegensatz zum bestehenden System, unter freien, kostenlosen Bedingungen und ohne finanzielle, gesetzliche oder technische Hürden möglich ist \cite{WD_bundestag_2009}. Dieses System ermöglicht darüber hinaus ein "alternatives Geschäftsmodell"\cite{lewis_2012_inevitability} für wissenschaftliche Publikationen. Was auf Maßgabe beruht, dass der Autor die "Eigentumsrechte an den Artikeln, die bisher für die Publikation in wissenschaftlichen Journals an die jeweiligen Fachverlage abgetreten wurden, (...) nun bei den Autoren der Artikel selbst verbleiben" \cite{Hess_2006}.

"Geringere Kostenbarrieren und damit eine einfachere Verbreitung ihrer eigenen Werke" \cite{WD_bundestag_2009} stellen dabei die Wünsche der wissenschaftlichen Autoren und Urheber an Open Access dar. Der Einsatz (offener) Lizenzen ist dafür ein Haupteinflussfaktor \cite{cite:16}. Open Access hat damit den Zweck, die durch Copyright generierten Barrieren zu überwinden und möglichst schnell und umfassend Zugriff auf neue wissenschaftliche Erkenntnisse zu ermöglichen.

\subsubsection{Offener Zugriff auf den wissenschaftlichen Prozess: Open Science}

"Open Science" knüpft an die Entwicklung der Ideen der Open Access-Bewegung an \cite{garcia_2010_open}. Beschränkte sich die Idee von Open Access vorerst auf die offene Zugänglichkeit zur finalen wissenschaftlichen Publikation, wird Open Science im Folgenden darüber definiert, wie der gesamte wissenschaftliche Wertschöpfungsprozess der Allgemeinheit offen zur Verfügung gestellt werden kann \cite{grand_2012_open}. Open Science kann demnach zum einen als Folge der neuen Möglichkeiten für kollaboratives Arbeiten im Rahmen der Digitalisierung und neuer Kommunikationstechniken und zum Anderen als Schritt hin zu einer "geistigen Allmende" \cite{naeder_2010_open} verstanden werden.

Der Sammelbegriff Open Science erstreckt sich dabei über die gesamte wissenschaftliche Wertschöpfungskette \cite{Scheliga_2014}: Vom offenen Zugang zu Publikationen wissenschaftlicher Forschung (Open Access), sowie den ganzheitlichen wissenschaftlichen Erkenntnisprozess. Unter diesem Gesichtspunkt kann Open Science als eine Weiterentwicklung von Open Access bezeichnet werden. Die diesbezügliche Evolution des Konzepts von Open Access führt zu einem direkten und breiten Weg, Wissenschaft an jedem Schritt der wissenschaftlichen Wertschöpfungskette zu kommunizieren und zu transferieren. Open Science ist die Reaktion auf die Forderung nach offenem Zugriff auf Wissenschaft und Forschung und kann dazu führen, "dass sich die Bedeutung von Forschungsergebnissen zukünftig nicht mehr auf sogenannte klassische wissenschaftliche Publikationen (im Format von Einleitung – Methoden – Ergebnisse – Diskussion), sondern die globale Echtzeitpublikation von Originaldaten stützen wird" \cite{Stengel_2013}.

Wie Open Access hat die Bewegung für Open Science ihre Dynamik der zunehmenden Verbreitung des Internets Anfang der 1990er zu verdanken \cite{Lievrouw_2010} und der neuen Möglichkeiten des kollaborativen Arbeitens sowie des Teilens von Daten und Informationen über das globale Netzwerk \cite{Meyer_2013}. Diese technologischen Entwicklungen ermöglichten jedoch nicht nur das kollaborative Arbeiten zwischen Wissenschaftlern in aller Welt, sondern schafften auch die Möglichkeit die ausgetauschten Informationen nicht nur unter Wissenschaftlern zu teilen, sondern die Verbreitung wissenschaftlicher Informationen an die Gesamtgesellschaft. Befürworter von Open Science sehen hier eine Möglichkeit die gesamten wissenschaftlichen Prozesse, von der Idee bis zur Abschlusspublikation, transparenter, effizienter, nachvollziehbarer und offener zu gestalten \cite{Woelfle_2011} und diese weltweit auch in unterentwickelten Regionen zu verbreiten \cite{yiotis_2013_open}.

Diese Vision einer offenen Wissenschaft steht der Verschlüsselungs- und Patentwut zur Wahrung der Geschlossenheit der wissenschaftlichen Informationen und eines möglichen kommerziellen Vorteils durch Wissenschaft im Rahmen öffentlich-finanzierter Forschung gegenüber und führt zu einer Debatte über die Verfügbarkeit der wissenschaftlichen Arbeit und die Entlohnung der "Erfinder" im wissenschaftlichen System \cite{suchen}.

\section{Wissenschaftliche Reputation}

Wissenschaftliche Reputation kann als eine "Art von Kredit" \cite{luhmann_1970_selbststeuerung} verstanden werden, mittels derer “Status und Ressourcen verteilt werden” \cite{hanekop_2006}. Diese Währung basiert auf der "gegenseitigen Beurteilung und Anerkennung der jeweils neuen Ergebnisse der Fachkollegen (Peers) durch die Wissenschaftler selbst" \cite{Hanekop_2014} \cite{suchen_Hornbostel_2006}, teils auf der "Generalisierung von Einzelleistungen", auf "gegenseitiger Ansteckung" und teils "auf der bloßen Häufigkeit der Publikationen oder der Anwesenheit an renommierten Plätzen" \cite{luhmann_1970_selbststeuerung}. Dabei gesteht auch Luhmann die Existenz von "Nebencodes der Reputation" an \cite{schmoch_2003_hochschulforschung}.

Die Reputation steuert die wissenschaftliche Aufmerksamkeit und die Verteilung motivierender Effekten, die sich durch das reine Streben nach Erkenntnis nicht erzeugen lassen \cite{suchen_luhmann}. Die akademische Reputation „ist die zentrale Kommunikationsform, die das Wissenschaftssystem charakterisiert“ \cite{Rutenfranz_1997}. Die Ergebnisse aus wissenschaftlicher Forschung werden dabei als Publikationen vor allen Mitgliedern der Wissenschaft präsentiert, „um diese intern von der Wissenschaftsgemeinde als wissenschaftlich beziehungsweise unwissenschaftlich zertifizieren zu lassen" \cite{Rutenfranz_1997} und durch einen kontinuierlichen Prozess der Selbstprüfung, wird die Korrektheit der wissenschaftlichen Erkenntnisse sichergestellt \cite{edsall_1976_scientific}. Das Reputationsystem ist demnach ein ausschlaggebender Treiber und Bremser für die Verhaltensweisen der Akteure im wissenschaftlichen System.

Wissenschaftliche Reputation verteilt sich nicht nur auf einzelne Personen sonder auch auf Einrichtungen, die wissenschaftlich tätig sind. Die Evaluation wissenschaftlicher Einrichtungen findet dabei über “Beobachtungen und Gespräche mit den Wissenschaftlern vor Ort sowie über den Austausch über die Eindrücke innerhalb der Begehungsgruppe und die gemeinsame Verständigung”\cite{Barl_sius_2008} statt.

Publikationen bilden im Hinblick auf die Funktion der Reputationsverteilung "eine Art Telos wissenschaftlicher Kommunikation" \cite{hirschauer2004peer}. In Bezug auf die Erlangung von Reputation ist wissenschaftliche Arbeit besonders auf ein funktionierendes Peer-Review-System angewiesen \cite{Luescher_2014}. Das Verfahren hat zwei Funktionen: Erstens die Selektionsfunktion, in deren Rahmen die Auswahl von Personen, Projekten und Texten stattfindet und zweitens eine Konstruktionsfunktion, in der Gutachter "produktiv in den Wissenschaftsprozess eingreifen" \cite{Neidhardt_2010} und die eigenen Fachstandards durchzusetzen. Der Peer Review-Prozess sichert aber nicht nur "Vertrauen" und die Grundlage für die "Anschlusskommunikation" innerhalb der wissenschaftlichen Community, sondern "wirkt überdies auch nach außen und gewährleistet die gesellschaftliche Legitimation des wissenschaftlichen Wissens" \cite{pscheida_2010_wikipedia}.

Als "guter akademischer Forscher" oder gute wissenschaftliche Institution gilt nur "wer viel und in möglichst angesehenen Journalen" \cite{Frey_2005} oder wissenschaftlichen Buchverlagen veröffentlicht. Dabei spielt der so genannte Peer-Review-Prozess eine zentrale Rolle im wissenschaftlichen Prozess \cite{smith_1999_opening} und ist Kernelement der Selbststeuerung von Wissenschaft \cite{Neidhardt_2010}. In dem Peer-Review Prozess "werden eingereichte Beiträge von fachlich versierten Wissenschaftlern (...) beurteilt und gemäß der qualitativen Anforderungen der Forschungs-Community zur Veröffentlichung angenommen oder abgelehnt" \cite{Hess_2006}.

Die Geschichte des Peer Review Verfahrens geht auf das 17. Jahrhundert zurück \cite{Kronick_1978}, etablierte sich aber erst Mitte des 18. Jahrhunderts, als die Royal Society of London ein "Committee of Papers" gründete, dass die Bewertung von Artikeln in seiner Zeitschrift Philosophical Transactions beaufsichtigen sollte \cite{Kronick_1990}. Das Verfahren unterschied sich damals grundlegend von dem, was heute im Einzelfall unter "Peer Review" verstanden wird und auch heute unterscheiden sich die Verfahren und deren Verbreitung in Abhängigkeit zum jeweiligen Fachgebiet. Drei gängige Verfahren der Peer-Review sind heute besonders stark verbreitet \cite{mueller_2009_peerreview}:
\begin{enumerate}
\item Bei der Double Blind Peer Review (DBPR) kennen sich Autoren und Gutachter eines eingereichten Manuskripts nicht
\item Bei der Single Blind Peer Review (SBPR) kennen die Gutachter die Autoren, die Autoren wissen jedoch nicht, wer ihr Manuskript bewertet
\item Open Peer Commentary (OPC) ermöglicht einer vergleichsweise großen Anzahl von Wissenschaftlern die Möglichkeit einge-
räumt wird, an der Bewertung wissenschaftlicher Arbeiten teilzuhaben
\end{enumerate}

Obwohl die meisten Gutachter für ihre Tätigkeit nicht bezahlt werden \cite{yiotis_2013_open}, steckt hinter dem Prozess ein komplexes System bestehend aus Redakteuren, Redaktionen, und der Verwaltung des Peer-Review-Prozess, der meist von Verlagen gesteuert und bezahlt wird \cite{Bargheer_2015} \cite{mueller_2009_peerreview} \cite{Baggs_2006}. Als Herzstück einer autonomen, selbstverwalteten Wissenschaft" \cite{suchen_Hornbostel_2006} beschränkt er sich nicht nur auf den Prozess der Publikation von Texten \cite{mueller_2009_peerreview}, sondern deckt ein breites Spektrum von Aktivitäten über die Fachdisziplinen hinaus ab \cite{Lee_2012}:
\begin{itemize}
\item die Beobachtung der klinischen Praxis (z.B. in der Medizin)
\item Beurteilung des Lehrenden oder der Fähigkeiten der Kollegen
\item Bewertung bei der Forschungsförderung und Stipendien bei Einreichung von Anträgen an staatliche und anderen Förderorganisationen
\item Begutachtung bei Artikeleinreichungen für wissenschaftlichen Zeitschriften
\item Bewertung von Papieren und Plakaten für Konferenzen
\item Bewertung von Buchvorschlägen für Universitätsverlage oder andere Verlage
\item Einschätzungen der Qualität, Anwendbarkeit und Interpretierbarkeit von Datensätzen und wissenschaftlichen Organisationen
\end{itemize}

Das Peer-Review Verfahren ist innerhalb in der Wissenschaft etabliert und weit verbreitet, bleibt aber der Öffentlichkeit weitgehend verborgen \cite{Konneker_2013}. Obwohl diese Verfahren den Kern der wissenschaftlichen Qualitätssicherung darstellen, werden bei den qualitativen Peer Review-Systeme und quantitativen bibliometrische Verfahren viele Mängel vermutet \cite{Peters_2014} \cite{Lee_2012} \cite{bar_2009_wissenschaftliche} \cite{osterloh2008anreize} \cite{ware_2008_peer} \cite{Jansen_2007} \cite{smith_1999_opening}. Die Mängel lassen sich laut Osterloh und Frey wie folgt zusammenfassen \cite{osterloh2008anreize}: Erstens, die "geringe Reliabilität der Gutachter-Urteile", zweitens die "geringe prognostische Qualität von Gutachten" und drittens das "opportunistisches Verhalten der Gutachter und Editoren, sowie das "opportunistisches Verhalten der Autoren". Zusammenfassend kommen die Autoren zu dem Schluss, dass "die Annahme eines Manuskriptes einem Zufallsprozess gleicht" und das "System der qualitativen Peer Reviews (...) auf einer erstaunlich fragwürdigen wissenschaftlichen Grundlage" beruht \cite{osterloh2008anreize}.

---- TODO: Grafik aus Kritik am Peer-Review bauen \cite{mueller_2009_peerreview} ----

Zusammenfassend werden folgende Indikatoren angelehnt an Heidemarie Hanekop \cite{hanekop_2008} für die Erlangung von wissenschaftlicher Reputation für wissenschaftliche Institutionen und Personen in der Literatur genannt:
\begin{enumerate}
\item \textbf{Anzahl der wissenschaftlichen Aufsätze / Beiträge}: Die Anzahl der Texte die Wissenschaftler im Rahmen ihrer Tätigkeit publizieren ist ein wesentlicher Faktor der Bewertung wissenschaftlicher Reputation \cite{Warnke_2012} \cite{CLAPHAM_2005} \cite{luhmann_1970_selbststeuerung}. Zum Beispiel erhöht die Anzahl an Texten die Chance durch die wissenschaftliche Community zitiert zu werden und damit die Möglichkeit auf die Erlangung von Reputation. Durch den zunehmenden Wettbewerb in der Wissenschaft muss sich der einzelne Wissenschaftler entscheiden, "zu publizieren oder im wissenschaftlichen System zu scheitern" \cite{Suess_2006}. Dadurch entsteht im wissenschaftlichen Kommunikationssystem ein konstanter Publikationsdruck, bei dem die Relevanz der publizierten Ergebnisse nicht immer im Vordergrund steht \cite{hamilton_1990_publishing}. Die Anzahl der veröffentlichten Artikel hat einen Einfluss auf die Vergabe von Ressourcen und finanziellen Mittel für weitere Forschung an Institutionen und Individuen \cite{Warnke_2012} \cite{hamilton_1990_publishing}.
\item \textbf{Relevanz der publizierten Ergebnisse}: Die Relevanz der publizierten Ergebnisse ist für das Wissenschaftssystem ein wesentlicher Treiber für den Prozess der Wissensgewinnung. Relevante Erkenntnisse sind die Grundlage für die Produktion von neuem Wissen und damit Grundlage für den gesellschaftlichen Auftrag des Wissenschaftssystems \cite{hanekop_2008}. Das wissenschaftliche System beruht auf der Annahme, dass die Relevanz der publizierten Ergebnisse einen direkten Einfluss auf die wissenschaftliche Reputation hat.
\item \textbf{Anzahl Monografien}: Die Anzahl der veröffentlichten Monographien ist ein wesentlicher Reputationsfaktor. Das gilt für die Disziplinen, in denen diese Publikationsform wichtig ist, wie den Geistes- und Sozialwissenschaften. In den anderen wissenschaftlichen Fachrichtungen spielt die Anzahl der Veröffentlichungen von Artikeln in wissenschaftlichen Journalen eine wichtige Rolle.
\item \textbf{Drittmittelprojekte}: Drittmittel sind, so der deutsche Wissenschaftsrat, "solche Mittel, die zur Förderung der Forschung und Entwicklung sowie des wissenschaftlichen Nachwuchses und der Lehre zusätzlich zum regulären Hochschulhaushalt (Grundausstattung) von öffentlichen oder privaten Stellen eingeworben werden" \cite{wr_2014}. Die Drittmitteleinwerbung hat sich in Deutschland als "meist gebrauchter Maßstab der Messung von Forschungsqualität durchgesetzt" \cite{M_nch_2006}. Diese Entwicklung geht mit einer zunehmenden Finanzierung der Forschung über Drittmittel einher \cite{Neidhardt_2010} \cite{Jansen_2007} \cite{simon_2009_wissenschaft_governance}. Durch die zunehmende Knappheit öffentlicher Ressourcen für Wissenschaft und Forschung, ist die Akquise von Drittmitteln zu einem kritisch zu betrachtenden Kernziel geworden \cite{Jansen_2007}. Das führt, dass zunehmend direkte finanzielle und administrative Kontrolle der Forschung eine Rolle spielen \cite{Barl_sius_2008}. Dabei spielt die Frage eine Rolle, ob die Publikationen, die im Rahmen der Drittmittelfinanzierung als wissenschaftliche Erkenntnisse veröffentlicht werden und ob der Antrag um Drittmitteleinwerbung selbst, "zum Erkenntnisfortschritt in der wissenschaftlichen Gemeinschaft beiträgt" \cite{M_nch_2006}. Die wissenschaftliche Community befürchtet durch die zunehmende Relevanz der Anzahl von Drittmittelprojekten bei der Erlangung von wissenschaftlicher Reputation eine Einschränkung der Freiheit von Wissenschaft und Forschung.
\item \textbf{Patente}: "Unter einem Patent versteht man das vom Staat verliehene Schutzrecht für eine technische Erfindung, welches dem Patentinhaber für eine bestimmte Zeit die ausschließliche wirtschaftliche Nutzung der Erfindung vorbehält." \cite{greif_2003_patente} Die Anzahl dieser Schutzrechte im Hochschulbereich nimmt seit den 1970er konstant zu. \cite{schmoch_2003_hochschulforschung} \cite{Fabrizio_2008}. Vor allem in den technischen Fachdisziplinen wird eine Patentschrift "als funktionales Äquivalent zur wissenschaftlichen Publikation begriffen" und bewertet \cite{mersch_2014_patente}. Die deutsche Hochschulrektorenkonferenz hält fasst die Rolle des Patentwesen an den Hochschulen wie folgt zusammen: "Patente leisten einen Beitrag zur Förderung der Wissenschaft, die Grundlagen des Patentwesens sind daher dem wissenschaftlichen Nachwuchs über entsprechende Lehrangebote zu vermitteln." \cite{suchen-Position-HRK} Die Befürchtung, dass Patente einen negativen Effekt auf die Erstellung und Veröffentlichung von fundamentaler Forschungsergebnisse hat, konnte nicht abschließend bestätigt werden \cite{Fabrizio_2008}.
\item \textbf{Vorträge}: Vorträge dienen der Verbreitung der Forschungserkenntnisse, sowie Zwischenständen und ermöglichen das Vermitteln des Wissens an andere \cite{rassenhoevel_2010_performancemessung}. Vorträge stellen eine informelle und schnelle Form für die Verbreitung neuer wissenschaftlicher Erkenntnisse und Ergebnisse dar. Die in einem Vortrag vermittelten Inhalten müssen meist nicht genauer belegt werden und die kommunizierten Inhalte lassen gegebenenfalls später schriftlich konkretisieren oder korrigieren \cite{haberle_2002_jahrbuch}. Vorträge bieten die Möglichkeit bereits vor der eigentlichen Publikation von wissenschaftlichen Erkenntnissen Anregungen und Reaktion einzuholen.
\item\textbf{Anwendungsrelevanz bzw. Verwertbarkeit}: Ein vergleichsweise neuer Indikator die Reputation von Hochschulen und außeruniversitäre Forschungsinstitute ist die Anwendungsrelevanz von Wissenschaft und Forschung \cite{simon_2009_wissenschaft_governance}. Sie bezieht sich auf einen Outputfaktor, der sich primär auf den Einsatz der gewonnenen wissenschaftlichen Erkenntnisse und auf die Verwertbarkeit für wirtschaftliche Produkte oder Patente als auf die eigentliche wissenschaftliche Veröffentlichung abzielt \cite{suchen}.
\item \textbf{Netzwerke und Kontakte}: Netzwerke beschreiben formelle und informelle Verbundsysteme zwischen Wissenschaftlern. Sie erlauben den schnellen Austausch und können Grundlage für Aktivitäten zur Steigerung der wissenschaftlichen Reputation darstellen. Diese Aktivitäten umfassen zum Beispiel gemeinsame Publikationsvorhaben und den Austausch wissenschaftlicher Erkenntnisse. Kontakte und Netzwerke schaffen soziale Beziehungen, die für eine erfolgreiche Integration an der Hochschule und der Fachcommunity sorgen, Zugang zu wissenschaftlicher Kommunikation ermöglichen und somit einen Einfluss auf die Anerkennung eines Wissenschaftler oder einer Wissenschaftlerin haben können.
\item \textbf{Öffentliche Aufmerksamkeit}: Die öffentliche Aufmerksamkeit stellt zum einen eine Möglichkeit des Wissenstransfers außerhalb der wissenschaftlichen (Fach-)Community dar, zum Anderen ermöglicht sie die Einflussnahme auf die politische Relevanz wissenschaftlicher Forschungsthemen. Die Veröffentlichung von wissenschaftlichen Informationen zu einem bestimmten Thema des öffentlichen Interesses stellt eine Möglichkeit dar, dieses Thema öffentlichkeitswirksam zu katalysieren. Öffentliche Aufmerksamkeit im Rahmen von wissenschaftlicher Tätigkeit stellen eine kritisch zu hinterfragende Möglichkeit für die alternative Ressourcengewinnung dar. \cite{suche}
\item \textbf{Politische Relevanz}: Die wissenschaftliche Tätigkeit mit politischer Relevanz stellt eine weitere Möglichkeit dar, wissenschaftliche Inhalte ausserhalb der Wissenschaft anwendbar zu machen und führt zu Anerkennung der wissenschaftlichen Arbeit. Daraus ergeben sich allerdings grundsätzliche "Verständigungsprobleme und Interessenkonflikte", da  "Wissenschaft und Politik aufgrund unterschiedlicher Rationalitäten handeln, einander aber zugleich brauchen" \cite{Mayntz_1996}. Während es im Wissenschaftssystem "um Erwerb und Erhalt von Wissen" geht, zielt die Politik auf "Erwerb und Erhalt von Macht" \cite{Mayntz_1996} ab. Die daraus resultierenden Interessenkonflikte führen zu "gegenseitigen Enttäuschungen", vor allem in der "forschungspolitischen Beziehung" \cite{Mayntz_1996}.
\item \textbf{Renommee der Forschungseinrichtung}: Das Renommee einer Forschungseinrichtung ist die Wahrnehmung der Einrichtung innerhalb und außerhalb der wissenschaftlichen (Fach-)Community. Sie hat für Wissenschaftler und die Wissenschaftlerin eine besondere Bedeutung \cite{mayntz_2008_wissensproduktion}. Sie basiert auf dem Konzept der "Ansteckung" \cite{luhmann_1970_selbststeuerung}. Diese Ansteckung führt zum Beispiel dazu, dass renommierte Professoren den Ruf einer Fakultät und eine renommierte Fakultät auch den Ruf von Professoren aufbessern können. Übertragen auf das wissenschaftliche Publizieren profitiert ein Autor oder eine Autorin bei der "Ansteckung" von dem Renommee einer Einrichtung, wenn er durch die Publikationsorgane der renommierten Institution veröffentlicht \cite{lutz_2012_zugang}.
\item \textbf{Renommee von Herausgebern oder Mitautoren} Der Herausgeber organisiert den Begutachtungsprozess und sichert bestimmte Qualitätskriterien mit seiner Reputation und seinem Namen \cite{mueller_2009_peerreview}. Auch hier kommt es im Rahmen des symbolischen wissenschaftlichen Kapitals zu einer Übertragung der Reputation der Herausgeber oder Mitautoren auf die anderen veröffentlichenden Autoren.
\item \textbf{personelle und materielle Ausstattung}: Die materielle Ausstattung beschreibt die Rahmenbedingungen, in der ein Wissenschaftler arbeitet. Diese Rahmenbedingungen haben eine herausragende Bedeutung bei der Entscheidung über einen Wirkungsort von Wissenschaftlern \cite{mayntz_2008_wissensproduktion}. Insbesondere die materielle und personelle Ausstattung sind bei traditionellen Berufungsverfahren deutscher Professorinnen und Professoren von besonderem Belang \cite{himpele_2011_job}, da sie die Arbeitsfähigkeit und die Anerkennung direkt beeinflussen \cite{suche}. Wie die materielle Ausstattung gilt auch die personelle Ausstattung als ein reputationstiftendes Merkmal für Wissenschaftler und die Institution, an denen sie arbeiten \cite{mayntz_2008_wissensproduktion}. Bei der Ausstattung handelt es sich um einen bilateralen Indikator, der zum einen aus der Bewertung der wissenschaftlichen Arbeit (im Rahmen der Forschungsförderung) resultiert \cite{Herb_vermessung_2008} und  zum anderen Reputation innerhalb der Community schafft \cite{mayntz_2008_wissensproduktion}.
\item \textbf{Gutachtertätigkeit und Herausgeberschaft}: Gutachter werden zum Beispiel in Peer-Review-Verfahren Autoren des entsprechenden Fachgebietes zugeordnet und entscheiden über die Veröffentlichung des Textes \cite{Frey_2005}. Bei manchen Publikationen wird ein Text mehrmals abgelehnt und eine Überarbeitung durch den Autoren eingefordert, bevor der Artikel final akzeptiert und daraufhin publiziert wird \cite{Frey_2005}. In diesem Zusammenhang wirkt sich die Reputation der mit diesem Verfahren betrauten Gutachter auch auf das Image des Verlages aus und umgekehrt. Die Gutachtertätigkeit ist aber nicht nur Kernbestandteil des wissenschaftlichen Qualitätssicherungs- und interdependenten Reputationssystems, sondern stellt auch einen informellen Weg der Kommunikation dar. Er ermöglicht den Gutachtern die Vorabsichtung neuster wissenschaftlicher Informationen und Erkenntnisse. Ähnlich wie die Gutachtertätigkeit ist auch die Herausgeberschaft fester Bestandteil des interdependenten wissenschaftlichen Reputationssystems \cite{Frey_2005}: Herausgeber profitieren von den publizierten Inhalten und Erkenntnissen der Autoren, Autoren von der Reputation Herausgebern und der Verlag von beiden \cite{suchen}.
\item \textbf{Funktion}: Die jeweilige Funktion oder die (universitäre) Stellenbezeichnung ist ein weiter Faktor für wissenschaftliche Reputation. Zum wissenschaftlichen Personal zählen Professoren, Juniorprofessoren, wissenschaftliche und künstlerische Mitarbeiter, sowie Lehrkräfte \cite{erhardt_2011_hochschulen}. Eine Weiterentwicklung und der "Aufstieg" in der wissenschaftlichen Hierarchie zielt auf das akademische Streben nach einer Professur \cite{Klecha_2008}.
\item \textbf{Awards und Preise}: Preise sind ein weitere Indikator für das wissenschaftliche Belohnungs- und Bewertungssystem. "Die Praxis der Award-Verleihung beruht auf dem Konzept, dass Ressourcen von unabhängigen Dritten auf Qualität geprüft und (...) zertifiziert werden" \cite{bargheer_2002_qualitatskriterien}. Wissenschaftler und Wissenschaftlerinnen, die Preise oder Awards gewinnen, erfahren Anerkennung. Diese Anerkennung können jedoch nicht automatisch als "Garant für wissenschaftsrelevante Qualität"\cite{bargheer_2002_qualitatskriterien} verstanden werden. Die Ehrung mit einem Preis weckt große Erwartungen und führt zu dem Anspruch eines stetigen Nachschubs an Anerkennung für den Wissenschftler oder die Wissenschaftlerin \cite{suchen}.
\end{enumerate}

Das Belohnungssystem für Wissenschaftler und Wissenschaftlerinnen bietet demnach Anreize für die, die als erstes neues Wissen entdecken und veröffentlichen. Das System der wissenschaftliche Reputation baut demnach auf der Verbreitung dieser Ergebnisse in der wissenschaftliche Gemeinschaft auf \cite{Fabrizio_2008}. Die Reputation einzelner Wissenschaftler steht damit in enger Abhängigkeit zum bestehenden wissenschaftlichen Kommunikationssystem  und anstatt finanzieller Entlohnung, wird in der Wissenschaft primär mit Aufmerksamkeit "bezahlt" \cite{suchen}. Vereinfacht lässt sich das System der Wechselbeziehungen der Reputationsverteilung im Rahmen von Publikationen wie folgt darstellen \cite{cite:21a}:

---- TODO: Grafik aus Text von Bernius ----

Bernius et al. unterscheiden drei aufeinandertreffende koordinierende Marktmechanismen: die Reputation, die Nutzung wissenschaftlicher Publikationen, sowie den Preis für den Erwerb der Publikation \cite{cite:21a}. Während die Reputation ein non-monetärer Aushandlungsmechanismus zwischen wissenschaftlichen Verlagen und wissenschaftlichen Autoren ist, findet die monetäre Preisdefinition direkt zwischen Bibliotheken und Verlagen statt \cite{EuropeanCommission_sciencepub_2006}. Der monetäre Aushandlungsprozess zwischen Wissenschaftlern und Bibliotheken wird duch die Bedeutung und Nutzung der jeweiligen Publikation bestimmt \cite{cite:21a}. Nicht jede Publikation hat diesbezüglich die gleiche Wertigkeit \cite{suchen} und damit den gleichen Einfluss auf die Reputation eines Autors.

Die allgemeine Verfügbarkeitmachung von Forschungsergebnissen stellt dabei einen integralen Bestandteil des wissenschaftlichen Ethos dar \cite{Fangerau_2014}. Der US-amerikanische Soziologe Robert K. Merton stellte diesen und weitere Grundprinzipien als normative Grundstruktur des Ethos von Wissenschaft vor \cite{Merton_1985}. Diesem Ethos liegt auch den Annahmen zu grunde, dass es Vorteile für die wissenschaftliche Gemeinschaft bringt, wenn Daten zweitverwertet werden und dass Daten ein wirtschaftliches Gemeingut sind, deren Wert durch breitere Nutzung verbesser wird \cite{RIN_2010_open_research}.

Er hält "Anspruchlosigkeit und Bescheideheit für Grundtugenden" \cite{hagner_2015_sache_buches} des modernen Wissenschaftlers. Der Ethos wird in diesem Zusammenhang als "Komplex von Werten und Normen"\cite{suchen} beziehungsweise "Verhaltensmaßregeln"\cite{suchen} verstanden. Merton unterteilt die Kriterien in die Kategorien:
\begin{itemize}
\item Universalismus: Die sozialen Merkmale eines Wissenschaftlers, wie zum Beispiel Nationalität, Geschlecht, Religion, Klasse usw. dürfen nicht in die Evaluation wissenschaftlicher Ergebnisse einfließen \cite{suchen}.
\item Kommunismus (Kommunalität) - Es gibt eine Pflicht zur Veröffentlichung der Ergebnisse von Wissenschaft und Forschung und sie sind als Allgemeingut zu betrachten. Die wissenschaftliche Anerkennung und das Ansehen sind einziges damit verbundenes "Besitzrecht"\cite{suchen}.
\item Uneigennützigkeit: Intrinsische "Neugier"\cite{suchen}, "selbstloses Eintreten für das Wohl der Menschheit"\cite{suchen} und der Wissensdurst müssen die vornehmlichen Motivatoren für Wissenschaftler darstellen \cite{suchen}.
\item Objektivität und Desinteresse: Eines der weiteren Kriterien erfordert "Objektivität und Desinteresse" an den Ergebnissen der eigenen Forschung \cite{suchen} unabhängig von finanziellem Erfolg und Prestige \cite{suchen}.
\item Organisierter Skeptizismus: Zweifel muss als "grundsätzliches Denkprinzip der Wissenschaft" \cite{suchen} und die "unvoreingenommene Prüfung und Kritik an Wissenschaft, Forschung und Autorität" \cite{suchen} verstanden werden. Dabei gilt es auch den "Matthew Effect" zu vermeiden. Der Matthäus-Effekt ist ein Phänomen auf der Makroebene der Wissenschaft \cite{bonitz_1998_matthaus}. Der „Matthäus-Effekt" ("Wer hat, dem wird gegeben" Mt. 25,29) beschreibt den Umstand, dass Autoren oder Publikationen, die bereits eine hohe Zitationsrate vorweisen können, meist noch häufiger zitiert werden als die Autoren oder Beiträge mit einer geringeren Zitationsrate. Überproportional profitieren in diesem System also die, die besonders häufig zitiert wurden \cite{Merton_1968} \cite{meier_2009_matthaus}.
\end{itemize}

Als Folge dieser Kriterien erkannte Merton das Urheberrecht an wissenschaftlichen Ideen und Beiträgen an, allerdings nur insofern, dass das Urheberrecht allein auf die Ermöglichung der Anerkennung durch Kollegen und die Achtung der Priorität beschränkt bleibt \cite{Fangerau_2014}. Zusammenfassend lassen die neuen Möglichkeiten der Verbreitung von Informationen einen vergleichbaren Veränderungsprozess der wissenschaftlichen Reputation und damit auch Anerkennung vermuten, wie er durch die Entwicklung des Buchdrucks ausgelöst worden war \cite{hanekop_2006}.

\subsection{Messbarkeit wissenschaftlicher Qualität und Publikationsquantität}
Wissenschaft ist ein Prozess, bei dem aus “unterschiedlichen Inputfaktoren, mittels verschiedener Transformationen Beiträge zur Schaffung neuer wissenschaftlicher Erkenntnisse als Output entstehen” \cite{Jansen_2007}. Die Bewertungen des jeweiligen Outputs führt “zur Aussage über die Forschungsperformanz” \cite{suchen}. Neben den Indikatoren für den Output wissenschaftlicher Perfomanz, müssen aber auch intermediäre Aspekte berücksichtigt werden \cite{schmoch_2009}.

Mit Beginn des 20. Jahrhunderts wurden in der Wissenschaftsforschung Indikatoren überwiegend zur Beschreibung der exponentiellen Wachstumsverläufe von Wissenschaft entwickelt und eingesetzt \cite{Hornbostel_1997}. Nach dem zweiten Weltkrieg etablierten sich erstmals Indikatoren für die Effizienzmessung wissenschaftlicher Wissensproduktion und -verbreitung, die aber "ebenso wie Sozial- und Wirtschaftsindikatoren keine neutralen Realitätsbeschreibungen" \cite{Hornbostel_1997} darstellten. Spätestens seit den 1970er Jahren werden diese Messungen, die die Forschungsleistung quantifizieren sollen, flächendeckend durchgeführt \cite{Hornbostel_1997} um Forschungsqualität und Quantität messbar zu machen.

Seit den 1990er Jahren ist diese Bewertung von Wissenschaft in Gestalt von Zahlen als unkontrollierte Nebenprodukte digitaler Wissenskommunikation erweitert worden \cite{angermueller_2010}. Heute zählen in der Wissenschaft vor allem die wissenschaftliche Reputation und die als "Impact" bezeichnete Wirkung wissenschaftlicher Publikationen \cite{herb_open_2013} \cite{Hornbostel_1997}. Die Wirkung der wissenschaftlichen Kommunikation wird anhand der quantitativen Betrachtung der Zitationen der jeweiligen Publikation ermittelt \cite{Brembs_20013} \cite[:16]{haustein_2012_multidimensional} \cite{weller2011twitter}. Diese rein quantitative Betrachtung muss allerdings auch als Proxy für die Bewertung von Wirkung in der "publish or perish" community verstanden werden \cite{peters_2015_research}.

Diese Art der Betrachtung basiert auf der Grundannahme, dass Kommunikation die "Essenz der Wissenschaft"\cite{bonitz_1998_matthaus} ist und "Zitierungen in ihrer Gesamtheit so etwas, wie die Grundelemente eines weltweiten Expertensystems" \cite{bonitz_1990_sci}. Nach dieser Sichtweise stellt eine häufige Zitation einen wesentlichen Indikator für die Wirkung der wissenschaftlichen Arbeit dar \cite{hamilton_1990_publishing}. Ein generalisierter und überzeitlicher Begriff von Qualität wissenschaftlicher Arbeit scheint nicht möglich, weil eine grundlegende Definition der Wissenschaftsindikatoren sowie ihrem Ziel der "Abbildung eines Konstruktes, das die Bewertungen einzelner Wissenschaftler oder Experten transzendiert" nicht möglich erscheint \cite{Hornbostel_1997}.

In den letzten Jahren haben sich neue Möglichkeiten für die Qualitätssicherung und -bewertung herausgebildet \cite{rekdal_2014_academic}. Die "Anforderungen an Verfügbarkeit von Dokumenten und Transparenz der Begutachtungen" der Open Access Bewegung haben die Frage aufgebracht, "ob möglicherweise Veränderungen der Review-Praktiken notwendig sind, um exzellente Wissenschaft zu identifizieren und vor allem zu fördern" \cite{suchen_Hornbostel_2006}. Des Weiteren stellt sich die Frage, ob die Berücksichtigung neuer Metriken für die Bewertung von wissenschaftlicher Qualität eine Antwort auf die Herausforderungen mit den etablierten Messsystem von wissenschaftlicher Qualität und Publikationsquantität sein können.

Bestand die klassische Wirkungsmessung von Wissenschaft in der Ermittlung der Anzahl von Zitationen, ermöglichen die veränderten Bedingungen von wissenschaftlicher Kommunikation im Rahmen der Digitalisierung alternative Erhebungsmöglichkeiten der Wirkung von formeller wissenschaftlicher Kommunikation und damit auch für die Erlangung wissenschaftlicher Qualität und Reputation. In den letzten Jahren wurde es viel einfacher Fälle von Plagiaten und wissenschaftliches Fehlverhalten zu identifizieren, und auch andere Arten von akademischen Abkürzungen zu entdecken und zu sehen, wie erschreckend häufig sie auftreten \cite{rekdal_2014_academic}.

Ergänzend zu den etablierten zitationsbasierten Metriken spielen zunehmend detailliertere Analyse von nutzungsbasierten Metriken und Metriken Basis von Social Indikatoren \cite{peters_2015_research} bei der Bewertung von wissenschaftlicher Kommunikation eine Rolle. Die Befürworter solcher alternativer Metriken erhoffen sich von diesen neuen Verfahren eine unmittelbare, umfassendere und detailliertere Wirkungsmessung wissenschaftlicher Kommunikation und eine gerechtere Verteilung von wissenschaftlicher Reputation \cite{peters_2015_research}.

\subsection{Wissenschaftliches Kapital}

Die Wissenschaft ist ein soziales Feld, dessen Strukturen und Praktiken das bestimmen, was als Wissenschaft und als wissenschaftliches Ergebnis gilt \cite{mikl_2010_soziologie}. Im Rahmen der Betrachtung von Steuerungs- und Reputationsmethoden für die Wissenschaft ist der Begriff "wissenschaftliches Kapital" von herausragender Bedeutung \cite{suchen}. Wissenschaftliches Kapital kann als eine Ausprägung des kulturellen Kapitals und als symbolisches, beziehungsweise non-monetäres Kapital \cite{irmer2011} \cite{hagner_2015_sache_buches} verstanden werden. Symbolisches Kapitel wird von der Sozilogin Mikl-Horke als Besitz an symbolischen Gütern beschrieben, "der besonders in einer Gesellschaft, die auf die Kooperation aller angewiesen ist, sehr kostbar ist"\cite{mikl_2010_soziologie}. Eine genauere Betrachtung des wissenschaftlichen Kapitals ist für das Verständnis der Motivation von Wissenschaftlern zu publizieren und zu kommunizieren, sowie für die Herausarbeitung der Treiber und Bremser für die Öffnung wissenschaftlicher Kommunikation unabdingbar.

Die "Gewährung wissenschaftlichen Kapitals" basiert heute auf der Kooperation zwischen publizierenden Wissenschaftlern und Verlagen \cite{herb_2006}. Die Wissenschaftler befinden sich in einer Abhängigkeit zu den Verlagen. Diese Abhängigkeit wird auch als "Faustischer Pakt" bezeichnet und hinterfragt \cite{hagner_2015_sache_buches} \cite{Parks_2002_acadamic_faust}. Diesen Pakt sind Wissenschaftler notgedrungen eingangen, "um den Preis, dass Barrieren zwischen Autoren und Lesern aufgebaut wurden" \cite{hagner_2015_sache_buches}. "Wissenschaftliches Kapital" kann in diesem Zusammenhang als “Ergebnis einer Investition (...), die sich auszahlen muss” \cite{herb_2006} definiert werden. “Diejenigen, die diese Berechtigungsscheine in der Hand halten, verteidigen ihr 'Kapital' und ihre 'Profite', indem sie diejenigen Institutionen verteidigen, die ihnen dieses 'Kapital' garantieren.” \cite{Bourdieu_1992}

Der Soziologe Bordieu unterscheidet zwei Typen von wissenschaftlichen Kapital \cite{Bourdieu_1998}. Das Kapital, das auf der politischen und institutionellen Macht beruht und das andere, dass aus der rein wissenschaftliche Anerkennung resultiert \cite{mikl_2010_soziologie}. Zitationsindexe sind Indikatoren für das wissenschaftliche Kapital, das durch Anerkennung entsteht \cite{Bourdieu_1998}. Die wissenschaftliche Reputation, die aus dem wissenschaftlichen Kapital resultiert, basiert auf der Liste der Publikationen in hoch gerankten Journalen und angesehenen Verlagen \cite{herb_2010}. Diese Bewertung ist symbolischer Natur und basiert "auf der Anerkennung und dem Kredit (...), den die Gesamtheit der Wettbewerber innerhalb des wissenschaftlichen Feldes gewähren" \cite{Bourdieu_1998} \cite{herb_2010}.

Das wissenschaftliches Kapitel ist zunehmend der Kapitalisierung von Wissenschaft ausgesetzt, bei der um den Einfluss der Ökonomie und den "wissenschaftswidrigen Verwertungsdruck". \cite{suchen_Hornbostel_2006} gerungen wird. Als ein Indikator dafür ist die Kopplung des wissenschaftliches Kapitals und an die output-orientierte Anreizsysteme zu verstehen. Ein Beispiel ist die zunehmende Relevanz des Performanzindikators "Drittmittel" \cite{Fabrizio_2008} \cite{Jansen_2007}, bei dem neben der Sicherung der Qualität von Forschung und Lehre zunehmend direkte finanzielle und administrative Kontrolle eine Rolle spielt \cite{Barl_sius_2008}. Dem Drittmitteleinkommen wird als Indikator für Forschungsleistung eine hohe Bedeutung zugemessen \cite{Jansen_2007}. Daraus entsteht die Tendenz, das nicht nur die Erwartungen an die Bewertung von Wissenschaft sehr ambitioniert sind, sondern auch, dass die Interessen privater und öffentlicher Drittmittel-Auftraggeber in den Vordergrund rücken und die Unabhängigkeit von Wissenschaft und Forschung gefährden. Ähnliches ist im Rahmen der stetigen Ökonomisierung des internationalen Uni- versitätsbetriebes\cite{brembs2015open} und  bei der leistungsbezogenen Mittelzuweisungen an die Universitäten zu beobachten \cite{suchen_Hornbostel_2006}. Vor allem die Verknüpfung von wissenschaftlicher Reputation mit der damit einhergehenden Verteilung von Mitteln und Stellen stellt eine neuartige Herausforderung an das Wissenschaftssystem dar, dessen Währung [ursprünglich] nicht Geld war \cite{hanekop_2006} \cite{suchen_Hornbostel_2006}.

Die Öffnung wissenschaftlicher Kommunikation folgt bisher nicht der wissenschaftlichen Logik, sondern basiert auf einer "feldunabhängigen Logik der Akkumulation von Kapital" \cite{herb_2006}. Insbesondere das deutsche Wissenschaftssystem ist dabei zunehmend von der Einführung an output-orientierter Anreizsysteme \cite{osterloh2008anreize} und einem Ungleichgewicht in der Kooperation zwischen wissenschaftlicher Kommunikation und wissenschaftlichen Kapital gekennzeichnet. Diese Entwicklung wird bei der weiteren Betrachtung der Motivationsfaktoren für Prozesse wissenschaftlicher Kommunikation eine wichtige Rolle spielen.

\subsection{Freiheit von Wissenschaft, Lehre und Forschung}

Die Wissenschaft unterliegt manigfaltigen externen Einflüssen, operiert aber dennoch autonom. So greifen "Andere Funktionssysteme (...) in die Wissenschaft zwar ein, wenn sie in Erfüllung ihrer eigenen Funktionen operieren und ihren eigenen Codes folgen. Aber sie können, jedenfalls unter den Bedingungen der modernen Gesellschaft, nicht selbst festlegen, was wahr und was unwahr ist."  \cite{Luhmann1998}. Dabei bewirkt die vorgabenfreie Erarbeitung und Veröffentlichung neuer Erkenntnisse bewirkt Fortschritt. "Die Autonomie der Wissenschaft wird nach außen durch die Abhängigkeit der Universität vom Staat und universitätsintern durch die Einheit von Wissenschaft und Forschung gesichert" \cite{Huber_2005}. Diese Wahrung ist im Artikel 5 Absatz 3 Grundgesetz als garantiertes Grundrecht wie folgt festgehalten: "Wissenschaft, Forschung und Lehre sind frei" \cite{suchen_GG}.

Dieses Recht ist nicht nur ein Grundrecht auf wissenschaftliche Meinungsfreiheit, sondern auch eine rechtliche Garantie. "Jeder, der in Wissenschaft, Forschung und Lehre tätig ist, hat - vorbehaltlich der Treuepflicht gemäß Art. 5 Absatz 3 Satz 2 GG - ein Recht auf Abwehr jeder staatlichen Einwirkung auf den Prozess der Gewinnung und Vermittlung wissenschaftlicher Erkenntnisse" \cite{suchen_BVG}. "Das garantiert einerseits die Einrichtung wissenschaftlicher Hochschulen mit Anspruch auf Selbstverwaltung, die staatliche Finanzierung und Sicherung ihrer Arbeit"\cite{suchen_BVG}, andererseits richtet es sich als "Abwehrrecht auf die Abwehr von Eingriffen in die wissenschaftliche Betätigung" gegen staatliche Eingriffe \cite{mayen_grundrechte_forscher} \cite{spindler_2006_rechtloa}. Jede Form der wissenschaftlichen Betätigung ist durch dieses Abwehrrecht geschützt. Dazu zählen laut Urteil des Bundesverfassungsgerichts "vor allem die auf wissenschaftlichen Eigengesetzlichkeiten beruhenden Prozesse, Verhaltensweisen und Entscheidungen bei dem Auffinden von Erkenntnissen, ihrer Deutung und Weitergabe" \cite{suchen}. Das betrifft auch die freie Entscheidung über die Art und Weise der Veröffentlichung von Forschungsergebnissen (Publikationsfreiheit) \cite{Fangerau_2014}.

Wissenschaftliche Freiheit bezeiht sich demnach auf der einen Seite auf die selbstbestimmte und unabhägige Wahl der Themen, Methodik, die freie Wahl Verbreitungs- und Publikainskanal und betrifft die Selbstorganisation bei der Durchführung und Steuerung der wissenschaftlichen Arbeit. Auf der anderen Seite beschreibt sie die Freiheit von inhaltlichen und methodischen Richtlinien und Vorgaben. Diese beiden Garantien beziehen sich in abgeleiteter Form auch auf die unterschiedlichen Organisationen und Institutionen von Wissenschaft.

In der praktischen Auslegung dieser Freiheit, wird allerdings von einer Entmythologisierung der Humboldt’schen der "Einheit von Forschung und Lehre" in der Universität gesprochen  \cite{binswanger_2014_excellence} \cite[:299]{Schimank_2001} \cite[:343]{Kruecken_2001}. Diese hat jedoch nicht erst mit dem steigenden Kosten- und Effizienzdruck, der Verwertbarkeit von Wissenschaft und Forschung, sowie der Modernisierung der Steuerungsmechanismen stattgefunden. Schon Imanuel Kant und Friedrich Nietzsche kritisierten eine Ausrichtung der Universität auf die Verwertbarkeit wissenschaftlichen Wissens \cite{Huber_2005}. Die Idee der Einheit von Forschung und Lehre, auf Grundlage eines völligen Verzichts auf Differenzierung \cite{kittler_2004}, lässt sich grundsätzlich nur in Ausnahmefällen realisieren \cite{Schimank_2001}. Als realistische Lesart kann im vorherrschenden System nur eine situative Differenzierung stattfinden bei der die Mittel der Grundausstattung nicht nach beiden Aufgaben separiert sind \cite{Schimank_2001}.

Diese Lesart der Humboldt’schen Idee ist noch immer hegemonialer Rahmen der aktuellen Hochschulreformen \cite{Huber_2005}. Das Recht auf Freiheit von Lehre und Forschung und die humboldtsche Idee der Universität wird und wurde immer für die Erhaltung des "organisationellen Status Quo", die Absicherung der "Institution Universität" und die Wahrung der "Staatsunabhängigkeit" angebracht \cite{Huber_2005}. Diese Autonomie der Wissenschaft und Forschung gilt auch heute als "hohes Gut, das es gegen externe Anforderungen zu verteidigen gilt"\cite{kaldewey_2010}.

In Hinblick auf die wissenschaftliche Publikation kann also festgehalten werden, dass Hochschullehrer nicht von der Hochschule oder anderer staatlicher Institutionen gezwungen werden können, über einen bestimmten Weg oder Kanal zu veröffentlichen \cite{spindler_2006_rechtloa} \cite{dorschel_2006_open}. Ausnahme stellen hier nur die privatfinanzierten Drittmittelprojekte dar, da sich der Hochschullehrer hier nicht auf die Wissenschaftsfreiheit als Abwehrrecht gegen den Staat berufen kann \cite{spindler_2006_rechtloa}. Wissenschaftlichen Mitarbeiter und Mitarbeiterinnen "müssen ihrer Hochschule die Nutzungsrechte an ihrer Publikation einräumen", es sein denn, sie haben sie nicht nach Weisung des Lehrstuhl- oder Institutsleiters erarbeitet oder es handelt sich um eine Dissertation oder Habilitation \cite{spindler_2006_rechtloa}. Ein direkter staatlicher Eingriff im Rahmen einer Richtlinie zum Publikationszwang über einen bestimmten Weg scheint demnach mit der Wissenschafts- und Publikationsfreiheit nicht vereinbar.

Dennoch kann der Staat Anreizsysteme oder Rahmenbedingungen zu schaffen, die die Öffnung des wissenschaftlichen Kommunikations- und Publikationssystems befördern. In der rechtlichen Auseinandersetzung mit dem Thema zielen die diskutierten Ansätze meist darauf ab, "den Autor eines öffentlich finanzierten wissenschaftlichen Werkes zuzwingen, die Allgemeinheit in gewissem Umfang an diesem partizipieren zu lassen und den Verlagen die Möglichkeit zu nehmen, durch einseitige Vertragsgestaltun-gen eine solche (kostenlose) Partizipation zu verhindern" \cite{dorschel_2006_open}.

\subsection{Ökonomie der wissenschaftlichen Kommunikation}

Die klassische Ökonomie der wissenschaftlichen Kommunikation beruht auf der Durchsetzung von Urheberrechten. Diese beschränken den Zugang und Zugriff auf sowie die Wieder- und Weiterverwendung von urheberrechtlich geschützten Inhalten. Leserinnen und Leser können nur gegen die Zahlung einer Gebühr Zugang zu der Veröffentlichung erhalten \cite{CREATe_2014}. Das gilt vor allem für die Veröffentlichung wissenschaftlicher Erkenntnisse. Bislang werden dafür "in der Regel wissenschaftliche Arbeiten zwar mit öffentlichen Mitteln finanziert, aber von privaten Verlagen in Fachzeitschriften herausgegeben" \cite{WD_bundestag_2009}. Diese Ökonomie der Wissenschaftsverlage ist nicht neu und hat sich im Laufe der Zeit weiter ausdifferenziert. Dieses Modell der wissenschaftlichen Publikation basiert auf einer "sozial ineffizienten" System \cite{mueller-langer_2010}. Die Wahrnehmung der Unverhältnismäßigkeit dieses Systems, insbesondere der Preisgestaltung für wissenschaftliche Publikationen \cite{King_2008} findet allerdings erst seit kurzem statt\cite{CREATe_2014}.

Wissenschaftliche Inhalte werden über drei grundlegende Vertriebsarten zur Verfügung gestellt \cite{cope2014future}:
\begin{enumerate}
\item Wissen als Inhalt zum Verkauf: Der größte Anteil wissenschaftlicher Publikation wird über diese Art vertrieben. Allein für die STM-Fächer (Science, Technology, Medicine) wird in der Literatur von einem Markt von 6 Milliarden Dollar für wissenschaftliche Zeitschriften ausgegangen .
\item Wissen als "kostenlose" Ressource: Diese Art des Vertriebs folgt der Maxime, dass wissenschaftlichen trägt die Verantwortung mit sich, die größtmögliche Verbreitung zu erreichen. Damit sind eher theoretische Vertriebsmodelle gemeint, bei denen der Leser kostenlos auf Inhalte zugreifen kann und auch dem Autor keine Kosten entstehen.
\item Wissen als bei der Produktion bezahlte Ressource: Ein wachsendes Modell für die kostenlose Bereitstellung wissenschaftlicher Inhalte, bei dem der Autor (oder die Förderinstitution) die Kosten für die Veröffentlichung und Verbreitung übernimmt.
\end{enumerate}

Das wissenschaftliche Publizieren kann als "gesellschaftlich bedingter Kreislauf" \cite{schirmbacher_2009_wisspub} betrachtet werden. Eine weitere wesentliche Besonderheit der Ökonomie wissenschaftlicher Kommunikation ist die Organisation des Marktes, die von spezifischen Akteuren und Prozessen geprägt wird \cite{Hess_2006}. Im Rahmen der formellen wissenschaftlichen Kommunikation und des wissenschaftlichen Verlagsgeschäfts, "ist es der Staat, der diesen Markt schafft" \cite{Hirschi_2015_buch_oa}. Diese Ökonomie, ihre Akteure und Prozesse können wie folgt unterteilt werden \cite{cite:11b} \cite{Hess_2006}:
\begin{enumerate}
\item Erstellung von Inhalten durch Wissenschaftler und Wissenschaftlerinnen (Erstellung): Der Kreislauf beginnt mit der Anfertigung der geistigen Werke durch die Autoren \cite{schirmbacher_2009_wisspub}. Nach der Entwicklung eines konkreten Forschungsvorhabens sowie einer wissenschaftlichen Fragestellung entstehen im Rahmen der wissenschaftlichen Forschung oder der jeweiligen Untersuchung Daten \cite{cite:11c}, die im Forschungsprozess gesammelt, analysiert, ausgewertet, aufbereitet und verschriftlicht werden \cite{cite:11d}. Die Ergebnisse werden abschließend strukturiert zusammengefasst und niedergeschrieben \cite{Hess_2006}.
\item Qualitätskontrolle und die Bewertung von Inhalten (Bewertung):
Die Qualitätskontrolle ist einer der wesentlichen Bestandteile der wissenschaftlichen Kommunikation. Sie sichert die im ersten Schritt gewonnen Erkenntnisse \cite{cite:11e} und stellt einen klaren Abgrenzungsaspekt zu nicht-wissenschaftlichen Informationen und Erkenntnissen dar\cite{cite:11f}. Sie findet im Kommunikationsprozess an zwei Stellen des Prozesses statt. Bei der initialen Bewertung wird die Publikation der Erkenntnisse vom Verlag organisiert \cite{schirmbacher_2009_wisspub} und von anderen Wissenschaftlern überprüft und gesichert (Peer-Review) \cite{Hess_2006}.
\item Auswahl der Inhalte durch Verlage auswählen (Bündelung):
Die Verlage kuratieren in Zusammenarbeit mit anderen Wissenschaftlern die wissenschaftlichen Inhalte für die Publikation. Bei wissenschaftlichen Journalen werden zum Beispiel die eingereichten Beiträge gebündelt und in einer Ausgabe mit anderen Beiträgen zusammengefasst.
\item Publikation der Inhalte durch Verlage (Druck):
Nach Erstellung und Erkenntnissicherung findet die "eigentlichen Publikation" \cite{schirmbacher_2009_wisspub} der Informationen statt. Bis zur Digitalisierung bestand dieser Schritt ausschließlich aus dem Druck der Inhalte auf Papier.\cite{cite:11h} Im Rahmen der Digitalisierung besteht der Prozess in der Aufbereitung der Beiträge für die digitale Verbreitung.
\item Distribution der Inhalte durch die Verlage (Verbreitung):
Der Vertrieb und die Verbreitung von Forschungsergebnissen an die wissenschaftliche Community ermöglicht den Zugriff auf die Informationen durch andere Wissenschaftler. Dieser Schritt stellt einen essenziellen Teil der Zirkulation und Kommunikation des neu gewonnen Wissens dar\cite{cite:11i}. Er sichert die Verfügbarkeit, die Möglichkeit des Zugriffs auf die Informationen und ist Teil des Selektionsprozesses für die Erschaffung neuen Wissens.\cite{cite:11l}
\item Support und Archivierung (Archivierung): Dieser Schritt beinhaltet die Erschließung, Aufbewahrung und Bereitstellung der Publikation durch Bibliotheken \cite{schirmbacher_2009_wisspub}. Die Bibliotheken unterstützen den Wissenschaftler und die Institution bei der Bewahrung und die Archivierung von Wissen \cite{K_lbel_2002}.
\item Konsum und Rezeption der Inhalte (Aufnahme von Wissen): Die Rezeption der publizierten Inhalte durch die wissenschaftliche Gemeinschaft ist der  Schritt des wissenschaftlichen Kommunikationsprozesses, bevor der ganze Prozess von neuem beginnt \cite{schirmbacher_2009_wisspub}. In diesem Schritt wird durch den Vergleich neuer Ergebnisse mit bereits publizierten Inhalten erneut die wissenschaftliche Qualität gesichert \cite{umstatter_2007_qualitatssicherung}. Aus der Mitte der wissenschaftliche Gemeinschaft entsteht durch die Verschriftlichung der wissenschaftliche Kommunikation der publizierten Ergebnissen neues Wissen \cite{cite:11k} \cite{schirmbacher_2009_wisspub} und der Kommunikationsprozess beginnt von vorn.
\end{enumerate}

An diesem Prozess des wissenschaftlichen Publizierens sind vor allem drei Gruppen beteiligt: erstens die Wissenschaftler, als Produzenten und Konsumenten der Informationen, zweitens die Verleger, die als Intermediäre wissenschaftliche Informationen sammeln, bündeln und verkaufen, sowie drittens die Bibliotheken, die die Informationen wieder den Wissenschaftlern zur Verfügung stellen \cite{Odlyzko_1997}. Wissenschaftler stehen dabei an einer komfortablen Stelle des wissenschaftlichen Produktions- und Distributionssystems \cite{herb_2010}, da sie ausschließlich mit der Verarbeitung und Neuerstellung von Wissens beschäftigt sind. Den Erwerb der Publikationen übernehmen die Bibliotheken, mit der Distribution sind die Verlage befasst. Wissenschaftler und Wissenschaftlerinnen verfügen häufig über sehr gute Zugangsmöglichkeiten zu wissenschaftlichen Informationen durch ihre Forschungsinsititutionen \cite{cope2014future}. Aus dieser Position sind sie als Autoren und als Leser mit den finanziellen Herausforderungen beim Vertrieb von Wissen nicht konfrontiert. Sie werden an staatlichen, wissenschaftlichen Institutionen größtenteils durch öffentliche Gelder finanziert und erhalten durch die Bibliotheken ihrer Institution Zugang zu wissenschaftlichen Publikationen. Sie schreiben Texte für die Publikation in wissenschaftlichen Verlagen, und werden mit im Rahmen der Veröffentlichung mit Reputation "belohnt". In diesem Publikationskreislauf sind die Verlage die einzige voll-privatwirtschaftliche Gruppe, die Ressourcen aus dem System herauszieht, ohne dass diese Ressourcen vollumfänglich dem Kreislauf der Wissenschaftskommunikation wieder zugeführt werden \cite{kiley_2006_open}.

---- TODO: Grafik bauen ----

\subsection{Wissenschaftlicher Diskurs}

\begin{quote}Wissenschaftliche Kommunikation vollzieht sich in Behauptungen, Erklärungen, Prognosen; sie ist nicht nur ein Informationsaustausch. Vielmehr vollzieht sich im wissenschaftlichen Diskurs der kollektive Prozeß des wissenschaftlichen Begreifens. Deshalb ist die wissenschaftliche Sprache als Diskurs nicht bloß ein Medium der Kommunikation, sondern der Ort, an dem sich ein wesentlicher Teil der wissenschaftlichen Arbeit vollzieht, der kollektive Darstellungsraum der Wissenschaft. \cite{bohme_1978_wissenschaftssprachen}\end{quote}

Dennoch muss auch die Erforschung von wissenschaftlichen Fragestellungen als ein Bestandteil des wissenschaftlichen Diskurses\cite{suchen} betrachtet werden. Die Verarbeitung von Forschungsergebnissen, die Anwendung und Neuinterpretation von Ergebnissen und das Verfassen von Gegenentwürfen und synthetischer Gesamtdarstellungen stellen Faktoren für den wissenschaftlichen Diskurs dar \cite{suchen}. Jürgen Habermas unterschied das kommunikatives Handeln von strategischem Handeln. Im dem "rationalen Diskurs" findet dabei vor allem eine Verständigung über problematische Geltungsansprüche statt \cite{suchen}. Der Beobachter entwickelt Methoden und Verfahren um zu einer Verständigung mit seiner Zielgruppe zu kommen \cite{suchen}. Der wissenschaftliche Diskurs operiert in diesem Verständigungsprozess funktional eigenständig und alles, was durch Wissenschaft kommuniziert wird, ist “entweder wahr oder unwahr” \cite{Luhmann1998}.

Michel Foucault versteht unter einem Diskurs "eine Menge von Aussagen, die einem gleichen Formationssystem zugehören"\cite{foucault_archaologie_1981}. Der wissenschaftliche Diskurs gründet sich demnach nur zum Teil auf die Forschung und kann auch nicht nur als “Kontaktglied zwischen dem Denken und dem Sprechen” \cite{foucault_ordnung_2003} definiert werden. Er wird getrieben vom Wille zur Wahrheit der sich durch "die Pädagogik, dem System der Bücher, der Verlage und Bibliotheken, den gelehrten Gesellschaften einstmals und den Laboratorien heute" ständig erneuert \cite{foucault_ordnung_2003}. Abgesichert wird der Diskurs "durch die Art und Weise, in der das Wissen in einer Gesellschaft eingesetzt wird, in der es gewertet und sortiert, verteilt und zugewiesen wird"\cite{foucault_ordnung_2003}. In der Foucault'schen Diskursanalyse wird der Diskurs als die Fähigkeit definiert, die “Beziehungen” zwischen “Institutionen, ökonomischen und gesellschaftlichen Prozessen, Verhaltensformen, Normsystemen, Techniken, Klassifikationstypen und Charakterisierungsweisen herzustellen”\cite{foucault_archaologie_1981}.

Im Rahmen des wissenschaftlichen Diskurses versuchen Menschen mit diversen "Machtprozeduren", die "ungeordnete und wuchernde Masse aller Äußerungen" zu reglementieren und zu kontrollieren \cite{Neymeyer_diskurs_2010}. Resultierend daraus entstehen Diskurse, die sich über einen gemeinsamen Gegenstand definieren. Sie gehorchen "impliziten wie expliziten Regeln", unterliegen "spezifischen Funktionen", nehmen bestimmte Formen an und sind von "Machtmechanismen gekennzeichnet". \cite{Neymeyer_diskurs_2010}
