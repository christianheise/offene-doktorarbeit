\subsubsection{Open Access Modelle}

In der einschlägigen Literatur wird in viele unterschiedliche Formen von Open Access unterteilt und es existieren mehrere Definitionen\cite{CREATe_2014}\cite{albert_2006_open_implications}, sowie unterschiedliche Auffassungen über die verschiedenen Modelle von Open Access\cite{CREATe_2014}\cite{cite:22b}\cite{lewis_2012_inevitability}. Als Grundlage für diese generelle Unterteilung gelten die "three Bs" (siehe Kapitel X), die als angesehensten Definitionen von Open Access gelten. Am Beispiel der Budapest Open Access Initiative werden zwei Wege für Open Access artikuliert\cite{albert_2006_open_implications}: 
\begin{enumerate}
\item Einrichtung von "einer neuen Generation von Fachzeitschriften," die einen kostenfreien und unmittelbaren Zugang zu den Beiträgen ermöglichen (als "goldener" Weg bekannt)
\item öffentlich zugängliche (Selbst-)Archivierung durch den Urheber (als "grüner" Weg bekannt)
\end{enumerate}

Eine zweite Ebene der Unterteilung in hybride, radikale und sonstige Formen von Open Access soll allen weiteren, in der Literatur aufkommenden Formen, gerecht werden. Abschliessend werden auch die Formen genannt, die zwar häufig als Open Access bezeichnet werden, aber den gängigen Deklarationen \cite{boai_2012} und Definitionen nicht gerecht werden. Open Access-Publikationen erheben in der Regel vom Autor Veröffentlichungsgebühr und verzichten nicht auf Peer-Review, um die akademische Reputation zu bewahren \cite{albert_2006_open_implications} \cite{Open_Access_net_2009}.

Der Umstand, dass eine eindeutige Klassifizierung dennoch schwer möglich ist, kann damit begründet werden, dass es "keine formelle Struktur, keine offizelle Organisation und kein ernannter Führer" gibt, der die Open Access Bewegung antreibt\cite{poynder_2011_suber}.

Der grüne Weg beschreibt das Modell, in dem der Autor

Der goldene Weg hingegen stellt die Publikation unmittelbar nach Fertigstellung zur Verfügung. Hierbei gibt es auch die Unterscheidung, dass einige Verlage die Publikation mit Verzögerung zur Verfügung stellen, in der Literatur wird in diesem Zusammenhang von verzögertem goldenen Open Access gesprochen\cite{lewis_2012_inevitability}. Im Rahmen anderer Modelle, vornehmlich bei der Publikation in Zeitschriften, wird dem Autor die Möglichkeit eingeräumt durch zusätzliche Zahlung die Publikation offen und frei zur Verfügung zu stellen\cite{lewis_2012_inevitability}.

Als Kernunterschied zwischen den beiden Modellen kann hervorgehoben werden, dass die grüne und hybride Form sowie das verzögerte Open Access, das klassische Geschäftsmodell der Verlage nicht beeinträchtigt, währen der goldene Weg auch ohne das bisherige Geschäftsmodell des Verlags auskommen kann\cite{lewis_2012_inevitability}.

Weitere, aber wenig genutzte Modelle sind

Hybride Modelle

Open Choice \cite{Hess_2006} 