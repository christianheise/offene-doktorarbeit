\subsubsection{Kritik}
Die Verlage haben mit Hilfe von wissenschaftlichen Journalen ein zentrales Steuerungs- und Bewertungssystem in der Wissenschaft etablieren können. Dabei werden die Grundprinzipien der Wissenschaft für die verlegerischen Verwertungsinteressen genutzt und das obwohl diese “wissenschaftlichen Grundprinzipien und Normen eigentlich ökonomischen Verwertungsinteressen zu widersprechen scheinen” . Spätestens durch die Einführung von Zitationsregistern und Impact Faktoren sowie die Definition der Kernzeitschriften hat den wissenschaftliche Zeitschriftenmarkt extrem unelastisch gemacht und gleichzeitig die Kapazität der kommerziellen Verlagen sowie deren Gewinnmargen ansteigen lassen.  Die Steuerungsmechanismen werden über die Messbarkeit an Hand der in 2.3.4 beschriebenen Methoden direkt oder indirekt ausgeübt. Dabei stehen insbesondere die Methoden, die auf der quantitativen Grundlage der Zitationsraten wissenschaftlicher Publikationen gemessen werden in der Kritik  und auch andere Indikatoren  für die Messung von Forschungsleistungen sind hoch umstritten . Der Hauptkritikpunkt: Die Verfahren, um die Wirkung von Wissenschaft und damit auch die Reputation von Wissenschaftlern zu messen, sind kein eigentliches wissenschaftliches Produkt  und erfassen zum Beispiel die Tätigkeit einzelner Forschergruppen zu stark . Das führt unter anderem dazu, dass der Impact Factor “kein perfektes Werkzeug (ist) um die Qualität der Artikel zu messen” und trotzdem wird er zur Bewertung von Wissenschaft genutzt, denn “(...) es gibt nichts Besseres, und er hat den Vorteil, dass er bereits lange existiert und ist daher eine gute Technik für die wissenschaftliche Bewertung” . Wie “gering der Wirkungsgrad” und die Methoden zur Messung “zur Reproduktion des traditionellen wissenschaftlichen Diskurses ausfall(en), wird von dem Moment an klar, an dem ein neues und offenes Kommunikationsmedium wie das Internet als alternativer Publikations- und Verbreitungskanal für Wissenschaft zur Verfügung steht.  Im folgenden soll aufgezeigt werden, welche Kritik es an dem System der wissenschaftlichen Reputation sowie dem Steuerungsmodell durch die Verlage im Rahmen der Veränderung der wissenschaftlichen Publikationskanälen gibt.