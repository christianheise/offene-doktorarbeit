\chapter{Theoretische Analyse}
Das Geschäftsmodell hinter der wissenschaftlichen Kommunikation ermöglicht den Verlegern Betriebsgewinnmargen von über 35 Prozent \cite{russell_2008_business} und hohe jährliche Wachstumsraten \cite{Wellcome_Trust_2003}. Sucht man nach Gründen für die Unterstützung des bisherigen Modells durch die Wissenschaftsgemeinschaft, wird deutlich, dass vor allem die in --- TODO in welchen Kapitel? --- beschriebene wissenschaftliche Reputation einen zentralen, extrinsischen Motivationsfaktor für Wissenschaftler darstellt \cite{minssen_2012_arbeit}. Die akademische Reputation „ist [dabei] die zentrale Kommunikationsform, die das Wissenschaftssystem charakterisiert“ \cite{Rutenfranz_1997}. Die Ergebnisse aus wissenschaftlicher Forschung werden dabei als Publikationen vor allen Mitgliedern der Wissenschaft präsentiert, „um diese intern von der Wissenschaftsgemeinde als wissenschaftlich beziehungsweise unwissenschaftlich zertifizieren zu lassen" \cite{Rutenfranz_1997}. 

Der aktuelle Forschungsstand zur Öffnung von Wissenschaft, zu den Treibern und Bremsern dieser Entwicklung und dem damiteinhergehenden Paradigmenwechsel mit Fokus auf den Themenbereich der wissenschaftlichen Reputation ist von besonderem Interesse. Grundlage dessen sind ausgewählte relevante und aktuelle Werke der Fachdiskussion die sich mit dem Phänomen Öffnung von Wissenschaft beschäftigen. Ziel des Kapitels ist die Entwicklung einer geeigneten wissenschaftlichen Fragestellung.

\section{Forschungsfragen} 
Folgende Forschungsfragen sollen bei der Inhalsanalyse genauer analysiert werden:
\begin{itemize}
\item Warum kommt es zu der Bestrebungen hin zur Öffnung von Wissenschaft? 
\item Wie können Open Science und Open Access definiert und voneinander abegrenzt werden? 
\item Welche Pro- und Contraargumente gibt es für die Öffnung von Wissenschaft - ist Offenheit in der Wissenschaft gut oder schlecht? 
\item Wo sind die Grenzen der Öffnung? 
\item Warum ist die Öffnung von Wissen in den verschiedenen wissenschaftlichen Disziplinen unterschiedlich weit verbreitet? 
\item Was bedeutet Offenheit und freier Zugang im Rahmen des wissenschaftlichen Diskurs-, Reputations- und Machtbegriffs?
\end{itemize}	

\section{Erhebungsmethode und Umfang} 
tbd

\section{Analyse der Definitionen von Open Access} 
tbd


\section{Analyse der Meinungen und Kritik an Open Access}

In diesem Teil der Arbeit soll im Rahmen der Literaturanalyse eine Auflistung der Kritikpunkte an der Open Access Bewegung in Wissenschaft und Forschung dokumentiert werden. Die Auswahl der berücksichtigten Werke bezieht sich auf dei genannten Fragestellungen und soll als verständlicher Überblick über den vorherrschenden Diskurs im Rahmen von Open Access und Open Science verstanden werden.

\subsubsection{Kritik am ökonomischen Modell}

Ein Kritikpunkt an dem Open Access Modell bezieht sich auf das Kostenargument und die frühe Hoffnung, dass die technologischen Treiber gesteuert und organisiert von der Forschungs Community selbst, anstatt durch Fachverlage, die durchschnittlichen Kosten für einen publizierten Artikel signifikant senke könnten. In einigen Beiträgen wurdne schon früh Kostensenkungen von bis zu 90 Prozent\cite{hilf_2004} prognostiziert. Grundlage dafür war die Fragestellung, dass "aus der Sicht des individuellen Nutzenkalküls von Wissenschaftlern, Verlagen und weiteren Einrichtungen wie Bibliotheken als auch aus Sicht gesamtwirtschaftlicher Wohlfahrtsüberlegunge (...) ob der Markt der Wissenschaftskommunikation nicht effizienter organisiert werden könnte."\cite{Hess_2006} Folgende Punkte schürten darüber hinaus die Hoffung, dass System leistungsfähiger zu machen und "von seinen durch den Papierdruck auferlegten Fesseln" zu befreien \cite{hilf_2004}:
\begin{itemize}
\item langer Zeitverzug vom Einreichen eines Manuskriptes bis zum Gelesen werden,
\item komplizierter Vertriebsweg vom Verlag über Grossisten zu Bibliotheken,
\item horrende Kosten (ca. 3.000,- Euro für die gesamte Verlagsarbeit je Artikel) mit den daraus folgenden horrenden Zeitschriftenpreisen,
\item und daraus folgend wenige Leser, auch noch ungleich in der Welt verteilt (digital divide),
\item unvollständige Information (aus Platzmangel), was Nachnutzungen und das Nachprüfen erschwert und somit auch Fälschungen erleichtert,
\item nur anonymes Referieren vor der Veröffentlichung, was den Missbrauch erleichtert. 
\end{itemize}

\subsubsection{Sicherung von Freiheit von Forschung und Lehre sowie Forschungsdiversität}

Eine Öffnung der wissenschaftlichen Kommunikation hat weitreichende Implikationen, nicht nur auf die Frage wie geforscht wird, sondern auch was geforscht wird \cite{suchen}. Da ein Großteil der Wissenschaft durch die öffentliche Hand finanziert wird, stecken hinter den Steuerungsmechanismen von Wissenschaft und Wissenschaftsförderung immer auch politische Interessen. Zwar soll die Vermischung dieser Interessen in Deutschland durch die Unabhängigkeit der Deutsche Froschungsgemeinschaft verhindert werden und Mittel völlig frei von politischer Couleur verteilt werden \cite{suchen}, dennoch kann, so die Befürchtung einiger Autoren \cite{suchen}, nicht sichergestellt werden, dass eine Einbeziehung der Öffentlichkeit nicht doch einen Einfluss auf die Mittelvergabe hätte. Drastischer ausgedrückt sieht Hagner in dem Beitrag "Open access als Traum der Verwaltungen" dass es im Rahmen des Öffnungsprozesses auch auf eine vollends verwaltete Forschung hinausläuft \cite{suchen}. Grundlagenforschung sowie andere komplexe oder explorative Forschungsbereiche würden in Zukunft weniger Berücksichtigung finden und die Freiheit von Wissenschaft und Forschung endgültig gefährdet \cite{suchen}, so der düstere Ausblick einiger Wissenschaftler \cite{suchen} \cite{suchen}. 

Um diese Aspekte beziehungsweise Prognosen über die Implikationen von Open Access zu evaluieren werden in diesem Teil der Arbeit auf Grundlage von Textbeispielen die Kritik an der Öffnung von Wissenschaft und der (forschungs-)politischen, rechtlichen und freiheitlichen Entwicklungen beleuchtet.

\subsubsection{Beispiel: Der "Heidelberger Apell" für Publikationsfreiheit und die Wahrung der Urheberrechte }

Am 22. März 2009 wurde auf der Webseite der „Frankfurter Allgemeinen Zeitung“ der Artikel "Geistiges Eigentum: Autor darf Freiheit über sein Werk nicht verlieren" \cite{faz_heidelberger_apell_2009} veröffentlicht. Im Anhang zu dem Artikel fand sich ein Aufruf, auch der "Heidelberger Appell" genannt. Vorangegangen war eine öffentlich ausgetragene Diskussion zwischen dem Literaturwissenschaftler Prof. Dr. Roland Reuß sowie weiteren Wissenschaftlern in einem Spezial der Onlineausgabe der Frankfurter Allgemeinen Zeitung: "Die Debatte über Open Access".

Der Appell richtete sich vor allem an "die Bundesregierung und die Regierungen der Länder, das bestehende Urheberrecht, die Publikationsfreiheit und die Freiheit von Forschung und Lehre entschlossen und mit allen zu Gebote stehenden Mitteln zu verteidigen" \cite{ITK_2009}. Die Autoren forderten Politik, Öffentlichkeit und weitere Kreative auf, sich für die "Wahrung der Urheberrechte", unter anderem in Bezug auf die Google Buchsuche "gegen eine angebliche „Enteignung“ der Autoren durch das Vorgehen von Google einerseits und durch das Publikationsmodell Open Access andererseits" \cite{WD_bundestag_2009} zu engagieren. 

Die Kritik am urheberrechtlichem Aspekt der Google Buchsuche soll in dieser Arbeit nicht berücksichtigt werden. Hier soll nur untersucht werden, inwiefern die Kritik am Publikationsmodell Open Access berechtigt ist. Der Apell unterscheidet dabei in zwei Ebenen: \textit{International} kritisieren die Autoren "die nach deutschem Recht illegale Veröffentlichung urheberrechtlich geschützter Werke geistiges Eigentum auf Plattformen wie GoogleBooks und YouTube" und die Entwendung dieser "ohne strafrechtliche Konsequenzen". \textit{National} werden die "Eingriffe in die Presse- und Publikationsfreiheit, deren Folgen grundgesetzwidrig wären" durch die "»Allianz der deutschen Wissenschaftsorganisationen« (Mitglieder: Wissenschaftsrat, Deutsche Forschungsgemeinschaft, Leibniz-Gesellschaft, Max Planck-Institute u. a.)" angeprangert.\cite{ITK_2009}

Die Kritik der Autoren des Heidelberger Apells bezieht laut einer Untersuchung des Wissenschafltichen Diensts des Bundestags insbesondere auf drei Aspekte \cite{WD_bundestag_2009}:
\begin{enumerate}
\item Erzwungene Vertriebswege
"Eine Forschung, der man diktieren könnte, wo ihre Ergebnisse publiziert werden sollen, sei nicht mehr frei." Die Verpflichtung auf "bestimmte Publikationsform (...) dient nicht der Verbesserung der wissenschaftlichen Information" \cite{ITK_2009}.
\item Abhängigkeitsverhältnis
\item Subventionierung von Vertriebswegen
\end{enumerate}

Der Appell "hat eine außergewöhnlich heftige Diskussion über die urheberrechtliche Problematik im Hinblick auf die aktuellen Entwicklungen im Internet ausgelöst. Er hat auch viele Parlamentarier und Politiker für das Thema sensibilisiert"\cite{WD_bundestag_2009}. An vielen Stellen widerlegt der Wissenschaftliche Dienst die Befürchtungen der Autoren des Heidelberger Apells. Beim Kritikpunkt der "Erzwungene Vertriebswege" widerspricht der Wissenschaftliche Dienst mit dem Verweis auf Gudrun Gersmann, weil "auch (Anmerkung: unter Open Access) eine Veröffentlichung bei einem Verlag mit einfachem Nutzungsrecht weiterhin möglich sei". In Bezug auf die im Apell erwähnte Kritik am neuen Abhängigkeitsverhältnis halten die wissenschaftlichen Autoren des Bundestags Reuß entgegen, dass es im bisherigen System "zwischen Autor und Fachzeitschriftverlag oft ein einseitiges Abhängigkeitsverhältnis zu Lasten des Autors gibt" und Wissenschaftler "oftmals alle Rechte an ihren Beiträgen abtreten" \cite{WD_bundestag_2009} müssen. "Der Befürchtung im Heidelberger Appell, das Publikationsmodell Open Access gefährde Fachzeitschriftenverlage", laut Autoren dritter Aspekt der Kritik an Open Access im Apell, "wird entgegengehalten, dass die digitale Plattform auf lange Sicht auch ein Ausweg aus der Zeitschriftenkrise sein könnte" \cite{WD_bundestag_2009}.

Dabei ist die Kritik im Rahmen des Apells mindestens an zwei Punkten berechtigt, so ist es erstens wahr, dass man seitens der Forschungsförderer nicht besonders bemüht war und ist \cite{suchen}, sich "ein genaues Bild von den Nebenwirkungen (Anmerkung: von Open Access)" \cite{Reuss_2009} zu verschaffen und zweitens stellt die Sicherung von Freiheit von Forschung und Lehre sowie die Anpassung der Steuerungsmechanismen eine Herausforderung an die Bestrebungen zur Öffnung von Wissenschaft und Forschung dar \cite{suchen}.

\subsubsection{Analyse der Definitionen von Open Access} 

Eine eindeutige Klassifizierung von Open Access gelingt derzeit nicht. Es "keine formelle Struktur, keine offizelle Organisation und kein ernannter Führer" gibt, der die Open Access Bewegung antreibt\cite{poynder_2011_suber}. Einzig und allein die Open Definition - open definition schreiben -

\subsubsection{Treiber und Bremser für Open Access} 

In den wissenschaftlichen Beiträgen zu Open Access werden viele positive Folgen aufgelistet. Folgende Treiber für eine Veränderung und Öffnung des wissenschaftlichen Kommunikationssystems werden dabei besonders häufig genannt:

\begin{itemize}
\item Verbreitung und Nutzungsmöglichkeiten der digitalen Infrastrukturen
\item Vorteile des grenzüberschreitenden Austauschs im Rahmen der Globalisierung von Wissenschaft und Forschung
\item ...
\end{itemize}

Neben den Aspekten die die Verbreitung von Open Access in den letzten Decaden unterstützt haben, gibt es aber auch einige Kriterien, die entweder zu einer Verlangsamung der Entwicklung geführt haben, oder sie in einigen Teilbereichen ganz zum erliegen gebracht haben. Dazu gehören:

\begin{itemize}
\item Fehlende Richtlinien auf regionaler, nationaler und internationaler Ebene
\item Führungslosigkeit der Open Access Bewegung
\item ...
\end{itemize}

\subsection{Analyse der Definitionen von Open Science} 

--- TODO ---- Michael Nielsen: “Open science is the idea that scientific knowledge of all kinds should be openly shared as early as is practical in the discovery process.”  https://lists.okfn.org/pipermail/open-science/2011-July/000907.html
http://www.openscience.org/blog/?p=454,

Research Information Network: “science carried out and communicated in a manner which allows others to contribute, collaborate and add to the research effort, with all kinds of data, results and protocols made freely available at different stages of the research process.” http://www.rin.ac.uk/our-work/data-management-and-curation/open-science-case-studies

Fecher/Friesike 5 Schulen von Open Science http://blogs.lse.ac.uk/impactofsocialsciences/2013/06/20/open-science-new-perspectives-for-scholarly-communication/ 
Siehe "Open Science"-Teil @ https://docs.google.com/document/d/1qDkQV-M_2VazjWwncRq_udo9tQqrjuZZkdLeKFc3cpI/edit#heading=h.1ahb76xafkbm

"Open science is the concept of making the whole research process as transparent and accessible as possible."\cite{Scheliga_2014}

Open science can be seen as a mechanism of cumulative knowledge production whereby scientists draw upon knowledge derived at by "prior researchers" and make their discoveries available to "future researchers". \cite{Scheliga_2014} auf Grundlage von \cite{Mukherjee_2009}
--- TODO ---

Es gibt zahlreiche Open Science Initiativen \cite{Scheliga_2014} viele von Ihnen erreichen aber keine kritische Masse \cite{wrap_2010} und enden eher als "virtuelle Geisterstädte" \cite{Nielsen_2011}.

\subsection{Analyse der Definitionen Treiber und Bremser für Open Science} 
tbd

\subsection{Analyse der Meinungen und Kritik an Open Science}

Während viele Wissenschaftler und Wissenschaftlerinnen Offenheit in der Forschung als wertvoll erachten, sind nur wenige sind wirklich bereit, die zusätzliche Zeit und Mühe zu investieren und potenziellen Risiken einzugehen, ihre Forschung offen und zugänglich zu machen \cite{Scheliga_2014} \cite{Procter_2010}. Forscherinnen und Forscher, die offene Wissenschaft pratizieren wollen, sind mit einer Reihe von Hindernissen konfrontiert \cite{Scheliga_2014}: 
\begin{enumerate}
\item individuelle Hindernisse: Angst vor Trittbrettfahren, Mehraufwand an Zeit und Mühe, Herausforderungen bei der Nutzung der digitalen Dienste, fehlender Anstoß negative Ergebnisse zu veröffentlichen, Herausforderung den Datenschutz sicherzustellen, Abneigung den Code zu teilen
\item systematische Hindernisse: Evaluationskriterien behindern Offenheit, kulturelle und institutionelle Einschränkungen, ineffektive (politische) Richtlinien, Mangel an Standards für das Teilen von Forschungsmaterialien, Mangel an rechtlicher Klarheit, finanzielle Aspekte der Offenheit
\end{enumerate}

Betrachtet wie Scheliga und Friesike das Phänomen Open Science an Hand des Konzepts des Soziale Dilemmatas, wird deutlich, dass was im kollektiven Interesse der wissenschaftlichen Gemeinschaft ist, nicht unbedingt im Interesse des einzelnen Wissenschaftlers ist und "wenn alle Wissenschaftler ihr Wissen nur in den Situationen teilen, in denen sie erwarten, dass sie selbst davon profitieren, ist die gemeinsame Wissenspool fragmentiert und alle Wissenschaftler stehen schlechter dar"\cite{Scheliga_2014}. 

Demgegenüber stehen dem gegenüber  --- TODO: ausarbeiten ---

\section{Beschreibung des Forschungsstands}
Anhand einer Literaturanalyse wird dargestellt, welche Argumentationen es für und gegen, sowie welche Möglichkeiten und Grenzen für die Öffnung der Wissenschaft angeführt werden. Eine kritische Analyse soll dabei Pro- und Kontraargumente zusammenfassen und einen Überblick über die aktuelle Debatte um Open Science und Open Access ermöglichen. Diese Analyse basiert auf der Annahme, dass sich Open Access in einer Übergangsphase von der reinen offenen Bereitstellung wissenschaftlicher Publikationen und dem damit verbundenen offenen Zugang zur Wissenschaft zur umfassenden und offenen Wissensverteilung und dem damit einherdehnenden Zugriff auf Wissenschaft für die Gesamtgesellschaft (Open Science) befindet. Darüber hinaus sollen medienkulturwissenschaftlich Open Science und Open Access in ihren technischen als auch in ihren gesellschaftlichen und politischen Aspekten sowie die kulturellen Auswirkungen der Medienbrüchen im Rahmen von hybridem Publizieren reflektiert werden. Abschließend werden Treiber und Bremser für die Öffnung von Wissenschaft erhoben und in der Gesamtbetrachtung der Arbeit zusammengeführt.

Die \textbf{Forderung nach Öffnung} adressiert mehrere Unzulänglichkeiten am bestehenden wissenschaftlichen Kommunikationssystem:
\begin{enumerate}
\item Transition-Argument
Die Nutzung der neuen Möglichkeiten für eine offene Wissensverbreitung neben den konventionellen Wegen der nicht-elektronischen Publikationen . Dabei gilt die Grundvoraussetzung der Aufbereitung des Wissens als strukturierte Daten zur Wissensweiterverwendung und -verarbeitung über alle Kanäle.
\item Speed & Circulation-Argument
Wissensverbreitung wird künstlich durch Embargos und ineffiziente Validationssysteme zurückgehalten. Die Digitalsierung und Verbreitung über elektronische Kanäle stellt einen Vorteil für Wissensverbreitung und -verwertung dar. Wenn das Wissen schneller zur Verfügungsteht wird es schneller zirkulieren und effizienter genutzt werden können \cite{Woelfle_2011}.  
\item Higher Impact & Citation-Argument
Ein Hauptargument der Open Acces-Befürworter ist die höhere Zitationsrate von wissenschaftlichen Publikationen, die unter den Kriterien von Open Access veröffentlicht wurden\cite{cite:21a}. In der einschlägigen Literatur findet man viele Untersuchugnen, die das zum Untersuchungsgegenstand gemacht haben und zu einem positiven Ergbniss kommen \cite{Lawrence_2001}\cite{Jeffrey_2008}\cite{Eysenbach_2006}\cite{Antelman_2004}
\item Tax-Payer-Agrument
Durch Steuergelder finanzierte Forschung ist dem Steuerzahler im Rahmen konventioneller wissenschaftlicher Kommunikation nicht immer unentgeldlich zugänglich, obwohl er im Rahmen öffentlich-geförderter Forschungsprogramme die Forschung bereits finanziert hat. Darüber hinaus stellt sich die Frage nach dem bestmöglichen Einsatz der monetären Ressourcen \cite{Glasziou_2014} \cite{altman_1994_scandal}.
\item Economic Promotion Argument
Bisher profitieren wirtschaftliche Unternehmungen nur unzureichend von staatlich-finanzierter wissenschaftlicher Kommunikation, dabei könnte eine schnellere, kommerziell verwertbare und umfassendere Bereitstellung der wissenschaftlichen Inhalte einen eklatanten Beitrag zur non-monetären Wirtschaftsförderung darstellen. Im Rahmen der offenen und schnelleren Verbreitung von wissenschaftlichen Informationen können neue Geschäftmodelle entstehen.
\item Digital Divide Argument
Der offene Zugang zu Publikationen ermöglicht neue Möglichkeiten für die Überwindung der sozialen, nationalen und globalen Wissenskluften  zwischen bildungsfernereren und -affineren Bevölkerungsteilen und -schichten der Welt . Der Mehrwert und die Chance von wissenschaftlichen Informationen für die Bewegung der offenen Bildungsmaterialien ist bisher auch noch nicht ausgeschöpft\cite{heise_lernen_2013}.
\item Validation & Reputation-Argument
Die Entwicklung neuer Verfahren, die die Aktivität und Qualität eines Forschers umfassender, transparenter und demokratischer messbar und kommunizierber machen, als im bestehenden Reputations- und Förderungssystem \cite{chalmers_2009_avoidable_waste}. Wissenschaftsevaluation wird durch Offenheit effizienter.
\item Paradoxon of Information Argument
Überwindung des bestehenden Informationsparadoxons bei der Verbreitung und Vermarktung von wissenschaftlichen Inhalten. Hierbei handelt es sich um das Problem, dass es schwer ist eine Information kommerziell zu verwerten ohne zu viel über Inhalt und Qualität auszusagen. Eine Entkommerzialisierung des Vertriebs von Wissen  würde das Informationsparadoxon aufheben.
\item Science communication Crisis-Argument
Durch die Öffnung von der wissenschaftlichen Kommunikations- und Reputationsprozesse besteht die Möglichkeit der vorherrschende Zeitschriften- und Monographienkrise durch neue Geschäftsmodelle zu begegnen.
\item Interdicipline & International Exchange/Collaboration Argument
Die Globalisierung in der Wissenschaft führt immerstärker zu internationalem Austausch und zur internationalen Zusammenarbeit von Wissenschaftlern . Doch das gilt nicht nur für die grenzenüberschreitende Zusammenarbeit in Bezug auf die lokale Verortung sondern auch für die Interdisziplinarität der Forschungsvorhaben. Die Öffnung von Wissenschaft ermöglicht also auch Fächerfremden Wissenschaftlern Zugruff auf Publikationen und damit auf Wissensressourcen für die eigene Arbeit .
\item Sustainable Access & Archiving Argument
Nur Offenheit im Sinne von Verwertbarkeit ermöglicht es in dezentralen Strukturen wie der des Internets alle Informationen nachhaltig und unabhängig voneinander zu speichern. Im Falle von Natur- oder anderen Katastrophen ermöglicht die digitale Ablage auf mehreren Kontinenten eine präservierung von Wissen undabhängig von lokalen Gegebenheiten oder Bedingungen.
\end{enumerate}

\textbf{Demgegenüber} stehen aber auch Argumente gegen die Öffnung der wissenschaftlichen Prozesse und Publikationen:
\begin{enumerate}
\item 	Quality-Argument
Die Befürchutung, das die Qualität auf Grund von schlechten oder nicht vorhandenen wissenschaftlichen Überprüfungsmechanismen leidet. Hauptargument ist das durch ein Autorengebühren finanziertes Publikationsmodell keinen klaren Anreiz für Ablehnung bietet.
\item Archiving-Argument
Die Sicherstellung der Langzeitarchivierung und die Garantierung der langfristigen Auffindbarkeit sowie Bereitstellung der Dokumente kann im Auge der Kritiker von Offenheit in Wissenschaft und Forschung nicht durch alternative digitale Strukturen gewährleistet werden. 
\item Authenticity-Argument
Forscherinnen und Forscher befürchten durch die dezentrale und offene Handhabung ihrer Texte und Arbeiten, dass diese im Zeitablauf inhaltlich nicht mehr unverändert zuordnenbar ihrem Autor sind.
\item Rightsmanagement-Argument
Hierbei handelt es sich um die Verpflichtung für Mitarbeiter staatlich finanzierter Forschungsinsitutionen alle Texte elektronisch frei und offen zu publizieren. In dem 2009 veröffentlichten "Heidelberger Appell" \cite{faz_heidelberger_apell_2009} kritisieren zahlreiche Autoren, Wissenschaftler, Verleger und Publizisten, dass das “verfassungsmäßig verbürgte Grundrecht von Urhebern auf freie und selbstbestimmte Publikation” … “derzeit massiven Angriffen ausgesetzt und nachhaltig bedroht” ist. Weiter sehen die Unterzeichner „weitreichende Eingriffe in die Presse- und Publikationsfreiheit, deren Folgen grundgesetzwidrig wären“ \cite{ITK_2009} und die Befürchtung, dass die Freiheit von Forschung und Lehre gefährdet ist \cite{Jochum_2009}. 
\item (Re-)Financing-Argument
Die unklare Refinanzierung der Öffnung von Wissenschaft ist eines der Kernargumente gegen das offene Publizieren von Arbeiten und Daten. Die Befürchtung ist, das ein solches System überhaupt nicht finanziert werden kann, konnte bisher nicht ausgeräumt werden.
\item Sustainability-Argument
\item Ressource-Allocation-Argument
Die Befürchtung, dass die Vergabe von Fördermittel und für die Karriere wichtige Aspekte der Reputationsbildung durch offenen System nicht Rechnung getragen wern kann ist eine weiteres Argument der Kritiker der Öffnung von Wissenschaft und Forschung. Eine Mittelvergabe zu gunsten populärer Forschung und damit eine Aushöhlung des wissenschaftlichen Systems in Ihrer Fächer und Facettenvielfalt wäre eine unmittelbare Folge dessen.
\item Open-Caring-Argument
Wissenschaftlerinnen und Wissenschaftler fürchten durch den Zwang zu umfassenderen Bereitstellung von Publikationen und gegebenenfalls soagar Daten einen nicht unwesentlichen zeitlichen Mehraufwand für die Öffnung ihrer Arbeiten. Sie möchten aber möglichst wenig Zeit für die Veröffentlichung, Bereithaltung und Verbreitungung ihrer wissenschaftlichen Arbeiten aufbringen.
	Aufwand für Offenheit im Alltag des Wissenschaftlers
\item Scientific-Freedom/Loss of Idea-Diversity-Argument
Angst dass durch Offenheit und Transparenz Forschungsförderung und Öffentlichkeit nur die wissenschaftlichen Projekte fördern und unterstützen, die von der Öffentlichkeit verstanden werden. Dabei stellt Wissen, vorallem aber Grundlagenwissen ein "öffentliches Gut" dar, "dessen Wert von der Öffentlichkeit nur schwer beurteilt werden kann"\cite{osterloh2008anreize}. Darüber hinaus herrscht die Annahme, dass im Rahmen von zunehmender Kollaboration und der Effizienz der elektronischen Suche die Diversität von wissenschaftlichen Meinungen und Projekten zu einem gleichen oder ähnlichem Thema eingeschränkt wird\cite{Evans_2008}.
\item Interpretations-Argument
Eine der weiteren Ängste der wissenschaftlichen Community ist die Angst vor der Fehlinterpretation ihres kommunizierten Wissens sowie der Verlust der Kontrolle über die Informationen\cite{gibbons_1994}. Dabei steht vor allem die Befürchtung im Vordergrund, dass die offen veröffentlichten Arbeiten genutzt werden um die Arbeit zu miskreditieren oder gezielt zur Falschinfromation der Öffentlichkeit zu nutzen.
\item Transparent-Research-Intentions-Argument
Mit den Forderung nach Offenlegung des gesamten Forschungsprozess erfolgt auch die Forderung nach "Transparenz der Interaktion zwischen Sponsoren (insbesondere kommerzielle Förderer wie die Pharma- und Medizinprodukteindustrie) und Auftragnehmern" \cite{Stengel_2013} 
\end{enumerate}


\section{Defizite}
Viele der Erklärungsansätze für den Paradigmenwechsel hin zur Öffnung der Wissenschaft basieren auf Annahmen, in denen ein direkter Zusammenhang von technischen Entwicklungen unmittelbar auf (wissenschafts-)politische und kulturelle Bewegungen geschlossen werden. Darüber hinaus beschränkt sich die Perspektive primär auf den Zugang zum Ergebnis von Wissenschaft und weniger auf die Öffnung des gesamten Prozesses. Die theoretische Auseinandersetzung mit der Geschlossenheit des wissenschaftlichen Diskurses  auf der Einen und mit den Treibern und Bremsern im realen wissenschaftlichen Prozess werden in der gängigen Literatur auf der anderen Seite, wird nicht genügend berücksichtigt. Hier wird vor allem die Verbindung zwischen wissenschaftlicher Reputation und Geschlossenheit des Wissensproduktionsprozesses nur selten erörtert. Als weiteres Manko kann angeführt werden, "dass die Deliberation bezüglich und die Verbreitung von Wissen ein stabiles Set von Infrastrukturen braucht"\cite{kelty_2004}, nach denen man heute noch vergeblich sucht.
