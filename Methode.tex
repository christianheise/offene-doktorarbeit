\chapter{Methoden und Vorgehen}
Die Verortung der Fragestellung dieser Arbeit, die von den Kulturwissenschaften über die Wirtschaftswissenschaften bis hin zu den Medienwissenschaften reicht, erfordert einen transdisziplinären Zugang zur wissenschaftlichen Bearbeitung. Drei wissenschaftliche Methoden werden in dieser Arbeit angewandt: das Konzeptionelle/Theoretische im Rahmen der qualitativen Inhaltsanalyse für die Begriffsbestimmung und für die Entwicklungen der Fragestellungen, das Ethnographische im Rahmen der quantitativen Befragung zur Identifikation der Treiber und Bremser für die Öffnung wissenschaftlicher Informationen und Prozesse, sowie das Experimentelle im Rahmen der Betrachtung der offenen Erstellung der eigenen Doktorarbeit und bei der Erarbeitung von Handlungsempfehlungen.

\section{Methodenwahl}

Die umfassende Inhaltsanalyse bildet einen wesentlichen Bestandteil der geisteswissenschaftlicher Arbeit. Für den empirischen Teil der Arbeit fiel die Entscheidung für ein quantitatives Vorgehen. Ziel dieser Abfolge ist es, in der theoretischen Phase "Fähigkeiten, Merkmale und Eigenschaften" zu definieren und zu strukturieren und diese in der empierischen Phase zu testen \cite{raab_2012_fragebogen}.

\subsection{Methode der Inhaltsanalyse}
Obwohl in der Literatur die strikte Kontrastierung der qualitiven und quantitativen Inhaltsanaylse als "gegenstandslos" bezeichnet wird \cite{frueh_2011_inhaltsanalyse}, folgt diese Arbeit eher der Methode einer qualitativen Herangehensweise an die Inhaltsanalyse. Diese untersucht Texte nach einem vorher konstuierten Raster auf relevante Informationen, in dem man ihnen in einem systematischen Verfahren Informationen entnimmt \cite{glaser_1999_theorie_analyse}. Dieses Verfahren eignet sich für die Extraktion von Informationen, die getrennt Text weiterverarbeitet werden sollen \cite{glaser_1999_theorie_analyse}.

Die unterschiedliche Verwendung der Begriffe Open Science und Open Access in der wissenschaftlichen Auseinandersetzung machen es notwendig, eine Begriffsbestimmungen für Open Science und Open Access vorzunehmen und zu konkretisieren. Dazu wird sich bei der Sichtung der Literatur die eigene Forschungsidee an andere Ergebnisse orientieren.

In Ergänzug zu bereits erfolgten Analysen von Benedikt Fecher und Sascha Friesike für den Begriff "Open Science"\cite{cite:9} sowie der Analyse von Giancarlo Frosio und Estelle Derclaye "Open Access Publishing" \cite{CREATe_2014} wird für diesen Zweck die Inhaltsanalyse als ergänzendes, extrahierendes Verfahren für die Begriffe "Open Access" und "Open Science" verwendet. Die Analyse der Literatur soll darüber hinaus dazu genutzt werden, um die Treiber und Bremser der Öffnung von wissenschaftlicher Kommunikation im Kontext von "wissenschaftlicher Reputation" zu identifizieren und die Debatte über das Thema darzustellen. Neben Arbeiten aus den Medienwissenschaften im engeren Sinn werden dabei auch Arbeiten aus den Wirtschaftswissenschaften und den Kulturwissenschaften eingeschlossen.

\subsection{Methode der quantitativen teilstandardisierten Datenerhebung}

Ziel der sozialwissenschaftlichen Forschung dieser Arbeit im Rahmen der Empirie ist die Erforschung von Tatbeständen (Exploration) und die Überprüfung von Hypothesen (Überprüfung) \cite{raab_2012_fragebogen}. Um der Entwicklung der Öffnung von Wissenschaft sowie deren Treiber und Bremser nachgehen zu können, wird eine schriftliche Onlinebefragung unter den wissenschaftlichen Akteuren des akademischen Publizierens an wissenschaftlichen Institutionen explorativ durchgeführt. Die hohe praktische Relevanz und vielfältigen Einsatzmöglichkeiten machen den Fragebogen zu der am häufigsten eingesetzten Methode zur Datenerheung in den empirischen Sozailwissenschaften \cite{raab_2012_fragebogen}. Mit der Methode soll es dem Forscher oder der Forscherin ermöglichen Ausschnitte der Realität abzubilden \cite{raab_2012_fragebogen}.

Die Relevanz begründet sich auf theoretischen Vorannahmen im Rahmen der Definition und Abgrenzung sowie der Literaturanalyse. Die bestehenden Hypothesen werden überprüft, beziehungsweise neue Hypothesen generiert werden. Durch einen Vergleich mit der Studie "Neue Formen des Wissenschaftlichen Publizierens" aus dem Jahr 2007 und 2008 vom Soziologisches Forschungsinstitut Göttingen (SOFI) soll darüber hinaus ein Einblick in die historische Entwicklung der Thematik im deutschsprachigen Raum gegeben werden. Die Befragung aus Göttingen bildet die Grundlage für die Fragebogenkonstruktion in der vorliegenden Arbeit.

Die umfrangreiche Befragung aus den Jahren 2007 entstand im Rahmen eines durch das Bundesministerium für Bildung und Forschung (BMBF) geförderten Verbundprojekts zwischen SOFI Göttingen und der Universitätsbibliothek Göttingen. Sie basierte auf einer "Vollerhebung der Wissenschaftler an den Instituten und Einrichtungen an fünf deutschen Standorten, die differenziert nach Fächern, Alters- und Statusgruppen (n=1800) erfasst wurden" \cite{Hanekop_2014}. Ziel der Befragung war es, die "Veränderungen beim Zugang zur Literatur wie auch bei den Veröffentlichungsstrategie" \cite{Hanekop_Wittke_2007_Fragebogen} zu untersuchen. Die Teilnehmer an der Studie wurden anhand von Webseiten der Forschungseinrichtungen identifiziert und per Email um Teilnahme gebeten.

Auf Grundlage und dem Vergleich mit der Vorbefragung in 2007, sowie durch der Überprüfung der Hypothesen in der empirische Arbeit soll das bestehende Wissen im Untersuchungsfeld erweitert werden und die Qualität der Arbeit und der "Neuigkeitswert" sichergestellt werden \cite{raab_2012_fragebogen}.

Fragebögen eignen sich besonders für große homogene Gruppen, wie die der Wissenschaftler und Wissenschaftlerinnen. Da es sich bei der eingesetzten Onlinebefragung um einen teilweise gestaltbaren Ablauf handelt, wird die Art der Befragung als teilstandardisiert bezeichnet \cite{raab_2012_fragebogen}. Um die Vergleichbarkeit zu ermöglichen und zu gewähren, wurde die Befragung dieser Arbeit größtenteils an den Kriterien der Befragung im Jahr 2007 angelegt. Einziger Unterschied ist die ausschließliche Onlinebefragung der Zielgruppen.

--- TODO: weitere eigener Teil für methode----

\subsection{Das Experiment als wissenschaftliche Methode: Offenes Schreiben dieser Arbeit}

Um Handlungsempfehlungen für das offene Schreiben von Dissertationen erstellen zu können, sowie die Kriterien und Argumente für oder gegen das offene Publizieren prüfen zu können, wurde für diese Arbeit selber eine offene Schreibweise gewählt. “Offen” bedeutet in diesem Fall, dass diese Arbeit direkt und unmittelbar während des Erstellungszeitraums Erstellung für jeden, jederzeit frei zugänglich auf einer Webseite im Internet unter einer freien Lizenz (CC-BY-SA) veröffentlicht wurde. Der Stand der Arbeit auf der Webseite ensprach zu jedem Zeitpunkt dem tatsächlichen Stand der Arbeit.

Das im Rahmen dieser Arbeit durchgeführte Experiment unterschiedet sich von klassischen wissenschaflichen Experimenten. Dieses Experiment ist dabei vom klassischen Laborexperiment als "Idealtypus kontrollierten Experimentierens", aber auch von der Feldbeobachtung, "die nicht vorsieht, dass im laufenden Betrieb eingegriffen und experimentiert wird" \cite{FQS196} zu unterscheiden. Experimente bei denen Selbstbeobachtung eine Rolle spielen können nur dann als wissenschaftliche Methode anerkannt werden, wenn sie durch die präzise Rekonstruktion klarstellen wie sie Wissen herstellen, wodurch sich dieses gewonnene Wissen als wissenschaftlichem Wissen auszeichnet \cite{solhdju_2011_selbstexperimente}.

Um trotzdem den Anforderungen der aktuelle geltenden Prüfungsordnung in vollem Umfang gerecht zu werden, wurde in einem Schreiben an die Promotionskomission am 8. Januar 2013 alle betreffende Punkte in der Promotionsordung der Fakultät Kultutwissenschaften (Stand: 02.02.2011) hervorgehoben und versucht zu begründen, warum diese nicht im Widerspruch zur offenen Schreibweise meiner Arbeit stehen. Um die selbstständige, wissenschaftlicher Arbeit sicherzustellen, bestand keine Möglichkeit, den erstellten Inhalt zu editieren oder zu kommentieren. Die Transparenz während der Erstellung stellt in diesem Fall keinen Widerspruch zu der Selbständigkeit bei der Ausarbeitung dar. Im Gegenteil, sie ermöglichte eine neue Form, die Eigenständigkeit direkt während der wissenschaftlichen Arbeit und Erstellung des Inhalts sicherzustellen. Die Promotionskommission hat dem Gesuch die Arbeit "offen" verfassen zu dürfen am 12. Dezember 2013 mehrheitlich entsprochen.

\section{Begründung der Methodenwahl}

Grund für die Wahl der genannten Untersuchungsmethoden ist das Ziel, zu einem vertieften Verständnis der empirischen Ergebnisse zu gelangen und darüber hinaus möglichst viele Verallgemeinerungsmodelle im Rahmen der definierten Fragestellungen erst theoretisch zu entwickeln und dann praktisch zu prüfen.

Das Experiment ist notwendig, da es bisher kein dokumentiertes wissenschaftliches Publikations- oder Promotionsvorhaben im deutschsprachigen Raum gab, welches untersucht werden konnte. Die Erfahrungen der offenen Schreibweise bilden aber einen wesentlichen Anknüpfungspunkt für die Beantwortung der Forschungsfragen und für weitere Forschung in diesem Feld.

\section{Kritische Betrachtung der Vorgehensweise}
