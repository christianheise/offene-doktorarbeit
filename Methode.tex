\chapter{Methoden und Vorgehen}

Die Verortung der Fragestellungen dieser Arbeit von den Kulturwissenschaften über die Wirtschaftswissenschaften, die Politikwissenschaften bis hin zu den Medienwissenschaften, erfordert einen transdisziplinären Zugang zur wissenschaftlichen Bearbeitung. Da die Herangehensweise an die Science and Technology Studies (STS) angelehnt ist, wird neben dem transdisziplinären Zugang auch ein Methodenmix gewählt um das Themenfeld nicht nur zu entdecken, sondern aktiv an der Entwicklung des Themenfelds teilzunehmen \cite{MacKenzie_STS_1999} und die empirischen Realitäten zu verstehen \cite{kelty_2014_freedom}.

Auch wenn die Themenbereiche kollaboratives Arbeiten, Social Media in Wissenschaft und Forschung, Citizen Science sowie Diskurse zu Tools und Diensten eng mit der Digitalisierung und Öffnung der wissenschaftlichen Kommunikation verbunden sind \cite{eu_agenda_open_science_2015}, werden diese in dieser Arbeit bewusst nur am Rande und beiläufig adressiert, beziehungsweise nur eingeschlossen, wenn sie die Beantwortung der Forschungsfragen dienen oder diese tangieren.

\section{Vorüberlegungen zur Methodenwahl}

Es werden folgende wissenschaftliche Erhebungsmethoden angewendet: Die umfassende Literaturrecherche mit analytischen Elementen für die Begriffsbestimmung und für die weitere Ausarbeitung der Fragestellungen, die quantitative Befragung zur Identifikation der Treiber und Bremser für die Öffnung wissenschaftlicher Informationen und Prozesse, sowie das (Auto-)Ethnographische im Rahmen der Betrachtung der offenen Erstellung der eigenen Doktorarbeit um das Zusammenspiel von Wissen und Technologie in gesellschaftlichen Ordnungsprozessen näher zu bestimmen und für die Erarbeitung von Handlungsempfehlungen im Sinne einer Beschreibung der Auswirkungen auf die Kommunikation von Wissenschaftlern.

Ziel dieser Abfolge ist es, in der theoretischen Phase "Fähigkeiten, Merkmale und Eigenschaften" \cite{raab_2012_fragebogen} zu definieren und zu strukturieren, diese in der empirischen Phase zu testen und abschließend mit Hilfe der ethnographische Phase zu überprüfen und zu ergänzen. Damit eignet sich die methodische Herangehensweise der Science and Technology Studies, mit der die Verschränkung von Wissenschaft, Technologie und Gesellschaft im Alltag untersucht und damit auch die Rolle von Wissen und Technologie in gesellschaftlichen Ordnungsprozessen näher bestimmt werden soll \cite{beck_2014_science}.

\section{Generelle Forschungsfragen}

---- TODO: Einleitender Satz ----

Aus dem oben formulierten lassen sich für diese Arbeit folgende zentrale Forschungsfragen ableiten:
\begin{itemize}
\item Wie kam es zur Forderung nach Öffnung der wissenschaftlichen Kommunikation?
\item Welche Herausforderungen bestehen im aktuellen wissenschaftlichen Kommunikationssystem?
\item Welche Aspekte von Open Access und Open Science sind am häufigsten verbreitet?
\item Wie hoch ist das Interesse in der wissenschaftlichen Gemeinschaft an Zugang zu, Zugriff auf und Öffnung der wissenschaftlicher Kommunikation?
\item Welche Argumente spielen in der Debatte für und gegen die Öffnung wissenschaftlicher Kommunikation eine Rolle?
\item Welche sind die Haupteinflussfaktoren für die Entwicklung um die Forderung von Open Access und Open Science?
\item Welche Bedeutung haben die Konzepte um Offenheit und freien Zugang im Rahmen des wissenschaftlichen Reputationsbegriffs?
\item Welche Wege und Möglichkeiten für die Öffnung wissenschaftlicher Kommunikation gibt es?
\item In welchem Zusammenhang haben diese Entwicklungen Einfluss auf die Massifizierung und Neoliberalisierung der Universität?
\item Wie groß ist der Aufwand für die Öffnung des wissenschaftlichen Erkenntnisprozesses im Rahmen einer wissenschaftlichen Publikation (hier Doktorarbeit)?
\item Welche Handlungsempfehlungen können für das Verfassen einer offenen wissenschaftlichen Arbeit gegeben werden?
\end{itemize}

Diese Forschungsfragen basieren auf folgenden Grundannahmen:
\begin{itemize}
\item Die Entwicklung der Forderung nach Öffnung der wissenschaftlichen Kommunikation basieren auf unterschiedliche Zielsetzungen \cite{suchen-Hoffmann-Zugang-undVerwertung-oeffentlicher-Informationen}.
\item Die Herausforderungen im aktuellen wissenschaftlichen Kommunikationssystem resultieren aus sozialen, technischen, rechtlichen und politischen Fehlentwicklungen.
\item Open Access und Open Science können nicht einheitlich definiert werden \cite{naeder_2010_open}.
\item Das Interesse in der wissenschaftlichen Gemeinschaft an der Öffnung wissenschaftlicher Kommunikation ist gering \cite{hagner_2015_sache_buches}.
\item Es besteht eine erhebliche Diskrepanz zwischen der Idee der offenen Wissenschaft und der wissenschaftlichen Realität \cite{Scheliga_2014}.
\item Rechtliche Rahmenbedingungen beeinflussen Maßgeblich die Verbreitung der Öffnung von Wissenschaft und Forschung \cite[:211]{Fehling_2014}.
\item Trotz neuer Initiativen und effizienterer Technologien, werden die wissenschaftlichen Akteure am bestehende Publikationssystem festhalten und die Monografien- und Zeitschriftenkrise wird auf absehbahre Zeit bestehen bleiben \cite{Parks_2002_acadamic_faust} \cite{Goetting_2015}.
\item Die Öffnung des wissenschaftlichen Kommunikation ist in den unterschiedlichen Disziplinen unterschiedlich stark verbreitet \cite{EuropeanCommission_sciencepub_2006}.
\item Die Bedrohung für Publikations- und Forschungsfreiheit ist aus Sicht der wissenschaftlichen Gemeinschaft ein Kernargument gegen die Öffnung wissenschaftlicher Kommunikation im Rahmen der Digitalisierung \cite{siehe_unten}.
\item "Openness" wird in der neoliberalen Rhetorik als effizientes Wettbewerbsmodell für die Massifizierung der Universität eingesetzt \cite{tkacz_2012_open}.
\item Die Umsetzung von Open Access wird in einer weiteren Öffnung des wissenschaftlichen Erkenntnisprozesses münden \cite{siehe_unten}.
\item Die Öffnung des gesamten wissenschaftlichen Erkenntnisprozess im Rahmen einer wissenschaftlichen Publikation ist praxistauglich.
\end{itemize}

\section{Methodenwahl}

---- TODO: Einleitung und Übergang schreiben ----

\subsection{Quantitativen teilstandardisierten Datenerhebung (Online-Befragung)}

Nach der Ausarbeitung der Grundlagen und Definitionen sowie der Literaturstudie, soll durch die Erforschung von Tatbeständen (Exploration), die Überprüfung von Hypothesen (Überprüfung) \cite{raab_2012_fragebogen} und die Erklärung von menschlichen Handeln \cite{atteslander_2008_methoden} der Bestand an gesichertem Wissen in dem Unterschungsbereich erweitert \cite{bortz_Doering_2006_fragestellung} werden. Um der Entwicklung der Öffnung von Wissenschaft sowie deren Treiber und Bremser nachgehen zu können, wird als zweite Methode eine explorative, schriftliche Onlinebefragung unter den wissenschaftlichen Akteuren des akademischen Publizierens an deutschsprachigen, wissenschaftlichen Institutionen durchgeführt.

Fragebögen eignen sich dabei besonders für große homogene Gruppen, wie die der Wissenschaftler und Wissenschaftlerinnen \cite{suchen}. Da es sich bei der eingesetzten Onlinebefragung um einen teilweise gestaltbaren Ablauf handelt, wird die Art der Befragung als teilstandardisiert bezeichnet \cite{raab_2012_fragebogen}. Die hohe praktische Relevanz und vielfältigen Einsatzmöglichkeiten machen den Fragebogen zu der am häufigsten eingesetzten Methode zur Datenerhebung in den empirischen Sozialwissenschaften \cite{raab_2012_fragebogen}. Diese Methode soll es dem Forscher oder der Forscherin ermöglichen Ausschnitte der Realität abzubilden \cite{raab_2012_fragebogen}.

Die bestehenden Grundannahmen werden mit Hilfe der Befragung überprüft und gegebenenfalls neue Hypothesen generiert. Durch den Vergleich mit der Studie "Neue Formen des Wissenschaftlichen Publizierens" aus dem Jahr 2007 vom Soziologisches Forschungsinstitut Göttingen (SOFI) wird die Betrachtung der historischen Entwicklung der Thematik im deutschsprachigen Raum ermöglichen. Des weiteren bildet die Befragung aus Göttingen die Grundlage für die Fragebogenkonstruktion und unterstreicht die Absicherung der wissenschaftlichen Gütekriterien.

Die umfangreiche Befragung aus den Jahren 2007 entstand im Rahmen eines durch das Bundesministerium für Bildung und Forschung (BMBF) geförderten Verbundprojekts zwischen dem SOFI Göttingen und der Universitätsbibliothek Göttingen. Sie basierte auf einer "Vollerhebung der Wissenschaftler an den Instituten und Einrichtungen an fünf deutschen Standorten, die differenziert nach Fächern, Alters- und Statusgruppen (n=1800) erfasst wurden" \cite{Hanekop_2014}. Ziel der Befragung war es, die "Veränderungen beim Zugang zur Literatur wie auch bei den Veröffentlichungsstrategie" \cite{Hanekop_Wittke_2007_Fragebogen} zu untersuchen. Die Teilnehmerinnen und Teilnehmer der Studie wurden anhand von Webseiten der Forschungseinrichtungen identifiziert und per Email um Teilnahme gebeten.

Durch den Vergleich mit der Vorbefragung in 2007, sowie durch die Überprüfung der Hypothesen in der empirische Arbeit soll das bestehende Wissen im Untersuchungsfeld erweitert werden, die Qualität der Arbeit verbessert und der "Neuigkeitswert" sichergestellt werden \cite{raab_2012_fragebogen}. Um diese Vergleichbarkeit zu ermöglichen und zu gewährleisten, wurde die Befragung dieser Arbeit an den Kriterien der Befragung im Jahr 2007 angelegt. Ein Unterschied bei der Durchführung besteht in der ausschließlichen Onlinebefragung der Zielgruppen. Diese Arbeit folgt der Annahme, dass bei Validität und Reliabilität von Online-Befragungen im Vergleich zu "Papier-Bleistift-Befragungen" keine Unterschiede bestehen \cite{Batinic_2013_onlinebefrag}.

\subsection{Das Experiment als wissenschaftliche Methode: Offenes Schreiben dieser Arbeit}

Um Handlungsempfehlungen für das offene Schreiben von Dissertationen erstellen zu können, sowie die Kriterien und Argumente für oder gegen das offene Publizieren prüfen zu können, wurde für diese Arbeit eine offene Schreibweise gewählt. “Offen” bedeutet in diesem Fall, dass diese Arbeit, alle erhobenen Daten sowie Begleitinformationen direkt und möglichst unmittelbar im Zeitraum der Erstellung für jeden, jederzeit frei zugänglich auf einer Webseite im Internet unter einer freien Lizenz (Creative Commons Namensnennung - Weitergabe unter gleichen Bedingungen 3.0 \cite{cc_by_sa_2008})veröffentlicht wurde. Der Stand der Arbeit auf der Webseite entsprach zu jedem Zeitpunkt dem tatsächlichen Stand der Arbeit.

Das im Rahmen dieser Arbeit durchgeführte Experiment unterschiedet sich vom klassischen wissenschaftlichen Experiment als "Idealtypus kontrollierten Experimentierens", aber auch von der Feldbeobachtung, "die nicht vorsieht, dass im laufenden Betrieb eingegriffen und experimentiert wird" \cite{Westermayer_2006}. Die Form des hiesigen Experiments fügt sich in das Konzept des Realexperiments ein und passt somit in die Technik- und Wissenschaftsforschung \cite{Westermayer_2006}. Diese Form des Experiments geht davon aus, "dass man relativ viel über das, was man nicht weiß, wissen kann, und dass das Ausprobieren der effektivste Weg ist, sich selbst zu korrigieren und weiterzukommen" \cite{Krohn_2005}. Realexperimente "sind experimentell orientiert, stehen unter situativ vorgegebenen Randbedingungen und verknüpfen Wissensanwendung und Wissensgenerierung" \cite{Westermayer_2006}. Bei dieser Art der qualitativ orientierten Forschung, ist das Vorgehen sehr spezifisch und auf den jeweiligen Gegenstand bezogen \cite{Krohn_2005}. Sie ermöglicht eine spezielle für diesen Gegenstand entwickelte oder differenzierte Herangehensweise \cite[:119]{Mayring_1999}. Die Anerkennung der Selbstbeobachtungen als wissenschaftliche Methode wird in dieser Arbeit dadurch gewährleistet, dass die präzise und offene Dokumentation sowie der offenen Prozess der Anfertigung sicherstellt, wie das Wissen erzeugt wird und wodurch sich dieses gewonnene Wissen als wissenschaftliches Wissen auszeichnet \cite{solhdju_2011_selbstexperimente}

Für die Auswertung wird auf einen autoethnografischen Ansatz zurückgegriffen. Bei diesem Ansatz wird der Forscher oder die Forscherin zum "teilnehmenden Beobachter"\cite{Ellis_2010}. Er ermöglicht es, "persönliche Erfahrung (auto) zu beschreiben und systematisch zu analysieren (grafie), um kulturelle Erfahrung (ethno) zu verstehen" \cite{Ellis_2010}. Als Forschungsmethode ermöglicht Sie eine Reflexion wie die eigenen Erfahrungen den Forschungszusammenhang beeinflussen \cite{ellis_2011_autoethnography}. Angelehnt an die sozialwissenschaftliche Methode folgt diese Arbeit einem Verständnis einer "Ethnographie über Menschen, die Medien nutzen, konsumieren, distribuieren oder produzieren"\cite{bachmann_2011_ethnographie}.

Das Ziel der autoethnografischen Untersuchungsmethode ist das Bestreben zu einem vertieften Verständnis der empirischen Ergebnisse zu gelangen, den Aufwand, der durch die Öffnung der formellen Kommunikation für Wissenschaftler und Wissenschaftlerinnen entsteht, zu beschreiben und zu analysieren sowie möglichst viele Verallgemeinerungsmodelle im Rahmen der definierten Fragestellungen theoretisch zu entwickeln und praktisch zu prüfen. Diese Zielsetzungen machen es erforderlich, einen methodischen Ansatz zu wählen, der es am Beispiel der eigenen Erstellung einer wissenschaftlichen Arbeit ermöglicht, die kulturelle Praxis des offenen wissenschaftlichen Kommunizierens besser zu verstehen \cite{maso_2001_phenomenology}.

Die autoethnografischen Herangehensweise der Beschreibung und Analyse ist auch deshalb notwendig, da bisher kein dokumentiertes, offen verfasstes wissenschaftliches Publikations- oder Promotionsvorhaben im deutschsprachigen Raum durchgeführt wurde. Die Erfahrungen der offenen Schreibweise bilden darüber hinaus einen wesentlichen Anknüpfungspunkt für die Beantwortung der Forschungsfragen und stellen die Grundlage für weitere Forschung in diesem Feld, sowie für die Erbringung eines Beitrags zum Fortschritt für die Wissenschafts- und Technikforschung, dar.

\section{Begründung der Methodenwahl}

Die Methodenwahl begründet sich zunächst auf den theoretischen Vorannahmen im Rahmen der Literaturrecherche. Wesentlich bei der Wahl der Untersuchungsmethoden sind Herangehensweise, Thema und das zu untersuchende Feld. Das Thema der Öffnung wissenschaftliche Kommunikation in Deutschland ist noch im Anfangsstadium eines Entwicklungsprozesses. Zwar haben sich seit der wissenschaftlichen Revolution im 17. Jahrhundert die methodischen Standards immer wieder gewandelt, doch sind "Sprache, Autorenschaft und Struktur bzw. Beurteilung wissenschaftlicher Artikel (...) Parameter, die jetzt mit der Verschiebung vom Buchdruck zur digitalen Publikation [stattfinden] einmal mehr in Bewegung geraten" \cite{hagner_2015_sache_buches}. Das zu untersuchende Feld, die Teilnehmer des wissenschaftlichen Kommunikationssystems, sehen sich bisher allerdings gar nicht oder erst seit kurzer Zeit mit dem hier behandelten Thema konfrontiert \cite{hagner_2015_sache_buches}.

Für eine Bearbeitung von wenig erforschten Feldern bietet sich die Methodik der qualitativen Sozialforschung an, allerdings begünstigt das Vorhandensein einer Vorbefragung vom SOFI in Göttingen aus dem Jahr 2007 auch die quantitativen Methode der (Online-)Befragung. Die quantitative Herangehensweise hat den Vorteil eines methodisch erklärenden Ansatzes mit einem hohe Messniveau, der für diese Arbeit als erkenntnisgewinnbringend eingeschätzt wird. Als hypothesenprüfende Methode beruht sie auf der Quantifizierung der Beobachtungsrealität \cite{bortz_Doering_2006_Methoden} mit der die dargestellten Vorannahmen geprüft und neue Ansatzpunkte evaluiert werden können. Sie ermöglicht darüber hinaus die differenzierten Erforschung und Bearbeitung, die unvoreingenommene und differenzierte Identifikation von Vorurteilen und Angeboten aber auch von Problemen, Herausforderungen und Alternativen.

Anders als bei der quantitativen Sozialforschung mit sehr elaborierten Methoden,
lässt die qualitative Forschung mit dem klaren Bekenntnis zur Offenheit viel Freiheit und wird im weiteren Verlauf der Arbeit den Forschungsprozess bereichern. Das Experiment als Element der qualitativen Forschung, hilft auf Grundlage der Erkenntnisse aus der Befragung beim "Verstehen" der Materie. Hier werden die Herausforderungen und Praxistauglichkeit der Forderungen für die Öffnung des wissenschaftlichen Prozesses subjektiv, induktiv analysiert. Das hilft zum Aufbau einer Distanz zu den Forderungen von Offenheit und Transparenz im Forschungsprozess und ermöglicht eine praktisch-fundierte Diskussion der Ergebnisse.

Zusammenfassend werden mit Hilfe der Literaturrecherche und der quantitativen Methode der Befragung allgemeine Muster identifiziert und durch die qualitative Methoden des Experiments beispielhaft an den Mechanismen der gesellschaftlichen Konstruktion der Wirklichkeit herausgearbeitet. Ziel dieser Methodenwahl ist es, einen wissenschaftlich fundierten Beitrag zu der Debatte zum Wandel von wissenschaftlicher Kommunikation im Rahmen der Digitalisierung in der Wissenschafts- und Technikforschung (STS) zu liefern.

\section{Kritische Betrachtung der Vorgehensweise}

Die Themen Digitalisierung und wissenschaftliche Kommunikation sind auch bei Eingrenzung auf den deutschsprachigen Raum und unter konkreten Fragestellungen weite Felder. Im Fokus dieser Arbeit steht die Gruppe der Wissenschaftler und Wissenschaftlerinnen. Sie stellen aber nur eine, wenn auch eine wesentliche, von mindestens drei Gruppen des wissenschaftlichen Kommunikationssystems dar. Mitarbeiter und Mitarbeiterinnen von Verlagen und Bibliotheken wurden zwar bewusst nur am Rande adressiert, beziehungsweise nur dann mit eingeschlossen, wenn sie für die Beantwortung der Forschungsfragen dienen oder diese tangieren, dennoch sind sie ebenfalls wichtige Akteure des wissenschaftlichen Kommunikationssystems \cite{suchen}.

Darüber hinaus können die Fächervielfalt und die Unterschiede in den einzelnen Disziplinen im Rahmen der wissenschaftlichen Kommunikation hier nur in Maßen abgebildet werden. Das gilt insbesondere für die autoethnographische Betrachtung der Anfertigung dieser Arbeit, die in den Geisteswissenschaften verordnet ist. Die Erfahrungen und Beobachtungen können nicht verallgemeinert werden und anderen Disziplinen von den hier dargestellten abweichen.

Durch die große Auswahl in der Stichproben, die umfassende Literaturrecherche vorab und die gewählte quantitative Vorgehensweise war die Erhebung zwar im erforderlichen Umfang vergleichbar und statistisch relevant, gab aber nur begrenzt Raum für die Erforschung neuer Herangehensweisen an die Thematik. Dem wurde versucht entgegenzuarbeiten, indem die ethnographische Herangehensweise des Selbstexperiments den Forschungsprozess komplettierte.

Ebenfalls kritisch muss die Vermengung des qualitativen Vorgehens mit dem quantitativen Forschungsprozess betrachtet werden. Zwar ist die Gefahr eines Qualitätsverlusts der Aussagekraft der Ergebnisse durch diese Vorgehensweise möglich \cite{suchen_Lamek_1993_S_198}, allerdings ist diese Kombination im Forschungsalltag nicht unüblich \cite{bortz_Doering_2006_Methoden}. Der Methodenmix eignet sich außerdem bevorzugt im Zusammenhang mit der Herangehensweise der Technik- und Wissenschaftsforschung.

Die Position des Autors als aktiver Teilnehmer um die Debatte um die Öffnung der wissenschaftlichen Kommunikation ist kritisch zu berücksichtigen. Durch die Fokussierung auf die Literatur bei der Extrahierung der Definitionen sowie die Darstellung der Debatten um Open Access und Open Science und die Übernahme der Fragen aus der Vorbefragung vom Soziologischen Forschungsinstitut Göttingen wird sichergestellt, dass die langjährigen Auseinandersetzung des Autors mit dem Thema nicht zu einer suggestiven und unwissenschaftlichen Herangehensweise bei der Bearbeitung der Forschungsfragen führt. Die aktive Auseinandersetzung mit der eigenen Position und das Realexperiment können als effektiver Weg betrachtet werden, sich selbst zu korrigieren und das Thema kritisch zu bearbeiten \cite{Krohn_2005}.
