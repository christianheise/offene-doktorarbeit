\chapter{Methoden und Vorgehen}
Die Verortung der Fragestellungen dieser Arbeit von den Kulturwissenschaften über die Wirtschaftswissenschaften bis hin zu den Medienwissenschaften, erfordert einen transdisziplinären Zugang zur wissenschaftlichen Bearbeitung.
Es werden drei wissenschaftliche Erhebungsmethoden angewendet: die qualitativen Inhaltsanalyse für die Begriffsbestimmung und für die weitere Ausarbeitung der Fragestellungen, die quantitative Befragung zur Identifikation der Treiber und Bremser für die Öffnung wissenschaftlicher Informationen und Prozesse, sowie das (Auto-)Ethnographische im Rahmen der Betrachtung der offenen Erstellung der eigenen Doktorarbeit und für der Erarbeitung von Handlungsempfehlungen.

Ziel dieser Abfolge ist es, in der theoretischen Phase "Fähigkeiten, Merkmale und Eigenschaften" zu definieren und zu strukturieren, diese in der empirischen Phase zu testen \cite{raab_2012_fragebogen} und abschließend in mit Hilfe der ethnographische Phase zu resümieren.

\subsection{Methode der Inhaltsanalyse}
Die Inhaltsanalyse untersucht Texte nach einem vorher konstruierten Raster auf relevante Informationen, in dem man ihnen in einem systematischen Verfahren Informationen entnimmt \cite{glaser_1999_theorie_analyse}. Dieses Verfahren eignet sich für die Extraktion von Informationen, die getrennt Text weiterverarbeitet werden sollen \cite{glaser_1999_theorie_analyse}. Obwohl in der Literatur die strikte Kontrastierung der qualitativen und quantitativen Inhaltsanalyse als "gegenstandslos" bezeichnet wird \cite{frueh_2011_inhaltsanalyse}, folgt diese Arbeit der Methode einer qualitativen Herangehensweise an die Inhaltsanalyse.

Die unterschiedliche Verwendung der Begriffe Open Science und Open Access in der wissenschaftlichen Auseinandersetzung macht es notwendig, eine Begriffsbestimmungen mit Hilfe der Analyse von Texten für Open Science und Open Access vorzunehmen und zu konkretisieren. Dafür werden aus wissenschaftlichen Beiträgen über das Kommunikationssystem in Wissenschaft und Forschung die Informationen extrahiert und verglichen, die sich mit der Öffnung des Systems befassen.

In Bezug zu bereits erfolgten Analysen von Benedikt Fecher und Sascha Friesike zum Begriff "Open Science"\cite{cite:9} sowie der Analyse von Giancarlo Frosio und Estelle Derclaye zu "Open Access Publishing" \cite{CREATe_2014} wird die Inhaltsanalyse als ergänzendes, extrahierendes Verfahren für die Begriffe "Open Access" und "Open Science" verwendet. Die Analyse der Literatur wird darüber hinaus dienen, um die Treiber und Bremser der Öffnung von wissenschaftlicher Kommunikation im Kontext von "wissenschaftlicher Reputation" zu identifizieren und die wissenschaftliche Debatte darzustellen. Neben Arbeiten aus den Medienwissenschaften werden dabei auch Arbeiten aus den Wirtschaftswissenschaften, der Wissenschaftstheorie und den Kulturwissenschaften in die Analyse eingeschlossen.

\subsection{Methode der quantitativen teilstandardisierten Datenerhebung (Umfrage)}

Ziel der Forschung dieser Arbeit im Rahmen der Empirie ist die Erforschung von Tatbeständen (Exploration), die Überprüfung von Hypothesen (Überprüfung) \cite{raab_2012_fragebogen} und zur Erklärung von menschlichen Handelns \cite{suchen_Methoden_d_empirischen_Sozialforschung}. Um der Entwicklung der Öffnung von Wissenschaft sowie deren Treiber und Bremser nachgehen zu können, wird eine schriftliche Onlinebefragung unter den wissenschaftlichen Akteuren des akademischen Publizierens an wissenschaftlichen Institutionen explorativ durchgeführt.

Fragebögen eignen sich besonders für große homogene Gruppen, wie die der Wissenschaftler und Wissenschaftlerinnen. Da es sich bei der eingesetzten Onlinebefragung um einen teilweise gestaltbaren Ablauf handelt, wird die Art der Befragung als teilstandardisiert bezeichnet \cite{raab_2012_fragebogen}. Die hohe praktische Relevanz und vielfältigen Einsatzmöglichkeiten machen den Fragebogen zu der am häufigsten eingesetzten Methode zur Datenerhebung in den empirischen Sozialwissenschaften \cite{raab_2012_fragebogen}. Diese Methode soll es dem Forscher oder der Forscherin ermöglichen Ausschnitte der Realität abzubilden \cite{raab_2012_fragebogen}.

Die Relevanz für diese Methodenwahl begründet sich auf den theoretischen Vorannahmen im Rahmen der Definition und Abgrenzung sowie der Literaturanalyse. Die bestehenden Hypothesen werden überprüft, beziehungsweise neue Hypothesen generiert werden. Durch den Vergleich mit der Studie "Neue Formen des Wissenschaftlichen Publizierens" aus dem Jahr 2007 vom Soziologisches Forschungsinstitut Göttingen (SOFI) soll die Betrachtung der historischen Entwicklung der Thematik im deutschsprachigen Raum ermöglicht werden. Des weiteren bildet die Befragung aus Göttingen die Grundlage für die Fragebogenkonstruktion der Erhebung im Rahmen dieser Arbeit.

Die umfangreiche Befragung aus den Jahren 2007 entstand im Rahmen eines durch das Bundesministerium für Bildung und Forschung (BMBF) geförderten Verbundprojekts zwischen dem SOFI Göttingen und der Universitätsbibliothek Göttingen. Sie basierte auf einer "Vollerhebung der Wissenschaftler an den Instituten und Einrichtungen an fünf deutschen Standorten, die differenziert nach Fächern, Alters- und Statusgruppen (n=1800) erfasst wurden" \cite{Hanekop_2014}. Ziel der Befragung war es, die "Veränderungen beim Zugang zur Literatur wie auch bei den Veröffentlichungsstrategie" \cite{Hanekop_Wittke_2007_Fragebogen} zu untersuchen. Die Teilnehmerinnen und Teilnehmer der Studie wurden anhand von Webseiten der Forschungseinrichtungen identifiziert und per Email um Teilnahme gebeten.

Durch den Vergleich mit der Vorbefragung in 2007, sowie durch die Überprüfung der Hypothesen in der empirische Arbeit soll das bestehende Wissen im Untersuchungsfeld erweitert werden, die Qualität der Arbeit verbessert und der "Neuigkeitswert" sichergestellt werden \cite{raab_2012_fragebogen}. Um diese Vergleichbarkeit zu ermöglichen und zu gewährleisten, wurde die Befragung dieser Arbeit an den Kriterien der Befragung im Jahr 2007 angelegt. Einziger Unterschied bei der Durchführung besteht in der ausschließlichen Onlinebefragung der Zielgruppen. Diese Arbeit folgt der Annahme, dass bei Validität und Reliabilität von Online-Befragungen im Vergleich zu "Papier-Bleistift-Befragungen" keine Unterschiede bestehen \cite{Batinic_2013_onlinebefrag}.

--- TODO: weitere Teil für methode----

\subsection{Das Experiment als wissenschaftliche Methode: Offenes Schreiben dieser Arbeit}

Um Handlungsempfehlungen für das offene Schreiben von Dissertationen erstellen zu können, sowie die Kriterien und Argumente für oder gegen das offene Publizieren prüfen zu können, wurde für diese Arbeit eine offene Schreibweise gewählt. “Offen” bedeutet in diesem Fall, dass diese Arbeit direkt und unmittelbar im Zeitraum der Erstellung für jeden, jederzeit frei zugänglich auf einer Webseite im Internet unter einer freien Lizenz (CC-BY-SA) veröffentlicht wurde. Der Stand der Arbeit auf der Webseite entsprach zu jedem Zeitpunkt dem tatsächlichen Stand der Arbeit.

Das im Rahmen dieser Arbeit durchgeführte Selbstexperiment unterschiedet sich von klassischen wissenschaftlichen Experimenten. Dieses Experiment ist dabei vom klassischen Laborexperiment als "Idealtypus kontrollierten Experimentierens", aber auch von der Feldbeobachtung, "die nicht vorsieht, dass im laufenden Betrieb eingegriffen und experimentiert wird" \cite{FQS196} zu unterscheiden. Experimente, bei denen Selbstbeobachtung eine zentrale Bedeutung haben, können nur dann als wissenschaftliche Methode anerkannt werden, wenn sie durch die präzise Rekonstruktion sicherstellen wie sie Wissen herstellen und wodurch sich dieses gewonnene Wissen als wissenschaftliches Wissen auszeichnet \cite{solhdju_2011_selbstexperimente}.

Methodisch wird für diesen Teil der Arbeit auf einen autoethnografischen Ansatz zurückgegriffen. Bei diesem Ansatz wird der Forscher oder die Forscherin zum "teilnehmenden Beobachter"\cite{Ellis_2010}. Er ermöglicht es, "persönliche Erfahrung (auto) zu beschreiben und systematisch zu analysieren (grafie), um kulturelle Erfahrung (ethno) zu verstehen" \cite{Ellis_2010}.

Um den Anforderungen der aktuell geltenden Prüfungsordnung der Leuphana Universität zu entsprechen, wurden vorab in einem Schreiben an die Promotionskommission die Bedingungen für die offene Erstellung der abgestimmt und die Vereinbarkeit mit der Promotionsordnung geprüft. Um die Eigenleistung und die Selbstständigkeit bei der Erstellung der wissenschaftlichen Qualifikationsarbeit zu gewährleisten, wurde technisch sichergestellt, dass es für andere als den Autoren keine Möglichkeit gab, den erstellten Inhalt zu editieren oder zu kommentieren. Im Gegenteil, sie ermöglicht eine neue Form, die Eigenständigkeit direkt während der wissenschaftlichen Arbeit und Erstellung des Inhalts darzustellen. Die Promotionskommission hat dem Gesuch die Arbeit "offen" verfassen zu dürfen am 12. Dezember 2013 mehrheitlich entsprochen. Die gewonnene Transparenz während des Erstellungsprozesses stellt in diesem Fall keinen Widerspruch zu der Selbständigkeit bei der Ausarbeitung dar.

Der Grund für die Wahl der autoethnografischen Untersuchungsmethode ist das Bestreben zu einem vertieften Verständnis der empirischen Ergebnisse zu gelangen, den Aufwand, der durch die Öffnung der formellen Kommunikation für Wissenschaftler und Wissenschaftlerinnen entsteht, zu beschreiben und zu analysieren sowie möglichst viele Verallgemeinerungsmodelle im Rahmen der definierten Fragestellungen theoretisch zu entwickeln und praktisch zu prüfen. Diese Zielsetzungen machen es erforderlich, einen methodischen Ansatz zu wählen, der es am Beispiel der eigenen Erstellung einer wissenschaftlichen Arbeit ermöglicht, die kulturelle Praxis des offenen wissenschaftlichen Kommunizierens besser zu verstehen \cite{maso_2001_phenomenology}.

Diese autoethnografischen Herangehensweise der Beschreibung und Analyse ist notwendig, da bisher kein dokumentiertes, offen verfasstes wissenschaftliches Publikations- oder Promotionsvorhaben im deutschsprachigen Raum durchgeführt wurde. Die Erfahrungen der offenen Schreibweise bilden einen wesentlichen Anknüpfungspunkt für die Beantwortung der Forschungsfragen und für weitere Forschung in diesem Feld.

\section{Begründung der Methodenwahl}
\section{Kritische Betrachtung der Vorgehensweise}
