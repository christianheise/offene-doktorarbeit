\chapter{Methoden und Vorgehen}

Die Verortung der Fragestellungen dieser Arbeit von den Kulturwissenschaften über die Wirtschaftswissenschaften bis hin zu den Medienwissenschaften, erfordert einen transdisziplinären Zugang zur wissenschaftlichen Bearbeitung. Vorab wird eine umfassende Literaturrecherche durchgeführt.

\section{Vorüberlegungen zur Methodenwahl}

Es werden folgende wissenschaftliche Erhebungsmethoden angewendet: die Inhaltsanalyse für die Begriffsbestimmung und für die weitere Ausarbeitung der Fragestellungen, die quantitative Befragung zur Identifikation der Treiber und Bremser für die Öffnung wissenschaftlicher Informationen und Prozesse, sowie das (Auto-)Ethnographische im Rahmen der Betrachtung der offenen Erstellung der eigenen Doktorarbeit und für der Erarbeitung von Handlungsempfehlungen.

Ziel dieser Abfolge ist es, in der theoretischen Phase "Fähigkeiten, Merkmale und Eigenschaften" zu definieren und zu strukturieren, diese in der empirischen Phase zu testen \cite{raab_2012_fragebogen} und abschließend in mit Hilfe der ethnographische Phase zu resümieren. Damit eignet sich die Methodische Herangehensweise der Science and Technology Studies, mit der die Verschränkung von Wissenschaft, Technologie und Gesellschaft im Alltag untersucht und damit auch die Rolle von Wissen und Technologie in gesellschaftlichen Ordnungsprozessen näher bestimmt werden soll \cite{beck_2014_science}.

In einem ersten Schritt werden dabei anhand der Literaturrecherche Texte nach einem auf relevante Informationen untersucht und Informationen entnommen \cite{glaser_1999_theorie_analyse}. Dieses Verfahren eignet sich für die Extraktion von Informationen, die getrennt Text weiterverarbeitet werden sollen \cite{glaser_1999_theorie_analyse}. Obwohl in der Literatur die strikte Kontrastierung der qualitativen und quantitativen Inhaltsanalyse als "gegenstandslos" bezeichnet wird \cite{frueh_2011_inhaltsanalyse}, folgt diese Arbeit der Methode einer qualitativen Herangehensweise an die Inhaltsanalyse. In der qualitativen Forschung wird die Methodik dabei an die jeweils durchgeführte Untersuchung angepasst.

Die unterschiedliche Verwendung der Begriffe Open Science und Open Access in der wissenschaftlichen Auseinandersetzung macht es notwendig, vorerst eine Begriffsbestimmungen mit Hilfe der Analyse von Texten für Open Science und Open Access vorzunehmen und zu konkretisieren. Dafür werden aus wissenschaftlichen Beiträgen über das Kommunikationssystem in Wissenschaft und Forschung die Informationen extrahiert und verglichen, die sich mit der Öffnung des Systems befassen.

Dafür wird auf zwei herangehensweisen zurückgegriffen, zum Einen werden Bibliothekskataloge,    Online-Datenbanken und relevante Fachzeitschriften systematisch nach den Stichwörtern "wissenschaftliche Kommunikation, "Open Access", "Open Science" durchsucht, zum Anderen werden nach dem "Schneeballsystem" wissenschaftliche Aufsätze und deren Literaturverzeichnisse nach weiteren Quellen analysiert, die wiederum relevante Literaturverweise enthalten können.

In diesem Zusammenhang wird unter anderem in Bezug zu bereits erfolgten Analysen von Benedikt Fecher und Sascha Friesike zum Begriff "Open Science"\cite{cite:9} sowie der Analyse von Giancarlo Frosio und Estelle Derclaye zu "Open Access Publishing" \cite{CREATe_2014} Literaturanalyse als ergänzendes, extrahierendes Verfahren für die Begriffe "Open Access" und "Open Science" verwendet. Diese Recherchen werden darüber hinaus dazu dienen, um Vorannahmen zu prüfen und die Treiber und Bremser der Öffnung von wissenschaftlicher Kommunikation im Kontext von "wissenschaftlicher Reputation" zu identifizieren und die wissenschaftliche Debatte darzustellen. Neben Arbeiten aus den Medienwissenschaften werden dabei auch Arbeiten aus den Wirtschaftswissenschaften, der Wissenschaftstheorie und den Kulturwissenschaften in die Analyse eingeschlossen.

\section{Forschungsfragen}

Aus dem oben formulierten lassen sich für diese Arbeit folgende zentrale Forschungsfragen ableiten:
\begin{itemize}
\item Welche Definition oder Aspekte in Definitionen von Open Access und Open Science sind am häufigsten verbreitet?
\item Wie hoch ist das Interesse in der wissenschaftlichen Gemeinschaft an Zugang zu, Zugriff auf und Öffnung der wissenschaftlicher Kommunikation?
\item Welche Argumente spielen in der Debatte für und gegen die Öffnung wissenschaftlicher Kommunikation im Rahmen der Digitalisierung eine Rolle?
\item Welche sind die Haupteinflussfaktoren für die Entwicklung um die Forderung von Open Access und Open Science?
\item Welche Bedeutung haben die Konzepte um Offenheit und freien Zugang im Rahmen des wissenschaftlichen Reputationsbegriffs?
\item Welche Aufwand bedeutet die Öffnung des wissenschaftlichen Erkenntnisprozess im Rahmen einer wissenschaftlichen Publikation (hier Doktorarbeit) und welche Chancen und Herausforderungen ergeben sich daraus?
\item Welche Handlungsempfehlungen können für das Verfassen einer offenen wissenschaftlichen Arbeit gegeben werden?
\end{itemize}

Diese Forschungsfragen basieren auf folgenden Grundannahmen:
\begin{itemize}
\item Es fehlt an einer klaren Definition von Open Access und Open Science \cite{siehe_unten}
\item Das Interesse in der wissenschaftlichen Gemeinschaft an der Öffnung wissenschaftlicher Kommunikation ist gering \cite{hagner_2015_sache_buches}
\item Es besteht eine erhebliche Diskrepanz zwischen der Idee der offenen Wissenschaft und wissenschaftliche Realität \cite{Scheliga_2014}
\item Trotz neuer Initiativen und effizienterer Technologien, werden sich die Akteure des wissenschaftlichen Publikationssystem nicht ändern und die Monografien- und Zeitschriftenkrise wird bleiben \cite{Parks_2002_acadamic_faust}.
\item Die Öffnung des wissenschaftlichen Kommunikation ist in den unterschiedlichen Disziplinen unterschiedlich stark verbreitet \cite{EuropeanCommission_sciencepub_2006}
\end{itemize}
\item Die Bedrohung für Publikations- und Forschungsfreiheit sind Argumente ist aus Sicht der wissenschaftlichen Gemeinschaft ein Kernargument gegen die Öffnung wissenschaftlicher Kommunikation im Rahmen der Digitalisierung \cite{siehe_unten}
\item Die Umsetzung von Open Access wird in einer weiteren Öffnung des wissenschaftlichen Erkenntnisprozesses münden \cite{siehe_unten}
\item Die Öffnung des wissenschaftlichen Erkenntnisprozess im Rahmen einer wissenschaftlichen Publikation ist praxistauglich \cite{siehe_unten}
\end{itemize}

---- TODO: bearbeiten, Rahmen ausfuehren Quellen für GA aus IH importieren und anpassen ----

\subsection{Methode der Inhaltsanalyse}

Obwohl in der Literatur die strikte Kontrastierung der qualitiven und quantitativen Inhaltsanaylse als "gegenstandslos" bezeichnet wird \cite{frueh_2011_inhaltsanalyse}, folgt diese Arbeit eher der Methode einer qualitativen Herangehensweise an die Inhaltsanalyse. Diese untersucht Texte nach einem vorher konstuierten Raster auf relevante Informationen, in dem man ihnen in einem systematischen Verfahren Informationen entnimmt \cite{glaser_1999_theorie_analyse}. Dieses Verfahren eignet sich für die Extraktion von Informationen, die getrennt Text weiterverarbeitet werden sollen \cite{glaser_1999_theorie_analyse}.

Die unterschiedliche Verwendung der Begriffe Open Science und Open Access in der wissenschaftlichen Auseinandersetzung machen es notwendig, eine Begriffsbestimmungen für Open Science und Open Access vorzunehmen und zu konkretisieren. Bei der Sichtung der Literatur werden sich die eigene Forschungsidee an anderen wissenschaftlichen Ergebnissen orientieren und auch die Debatten um die Herausforderung bei der Öffnung der wissenschaftlichen Kommunikation genauer betrachtet.

\section{Methode der quantitativen teilstandardisierten Datenerhebung (Umfrage)}

Ziel der Forschung dieser Arbeit im Rahmen der Empirie ist die Erforschung von Tatbeständen (Exploration), die Überprüfung von Hypothesen (Überprüfung) \cite{raab_2012_fragebogen} und die Erklärung von menschlichen Handeln \cite{suchen_Methoden_d_empirischen_Sozialforschung} um den Bestand an gesichertem Wissen in dem Unterschungsbereich zu erweitern \cite{bortz_Doering_2006_fragestellung}. Um der Entwicklung der Öffnung von Wissenschaft sowie deren Treiber und Bremser nachgehen zu können, wird eine schriftliche Onlinebefragung unter den wissenschaftlichen Akteuren des akademischen Publizierens an wissenschaftlichen Institutionen explorativ durchgeführt.

Fragebögen eignen sich besonders für große homogene Gruppen, wie die der Wissenschaftler und Wissenschaftlerinnen. Da es sich bei der eingesetzten Onlinebefragung um einen teilweise gestaltbaren Ablauf handelt, wird die Art der Befragung als teilstandardisiert bezeichnet \cite{raab_2012_fragebogen}. Die hohe praktische Relevanz und vielfältigen Einsatzmöglichkeiten machen den Fragebogen zu der am häufigsten eingesetzten Methode zur Datenerhebung in den empirischen Sozialwissenschaften \cite{raab_2012_fragebogen}. Diese Methode soll es dem Forscher oder der Forscherin ermöglichen Ausschnitte der Realität abzubilden \cite{raab_2012_fragebogen}.

Die Relevanz für diese Methodenwahl begründet sich auf den theoretischen Vorannahmen im Rahmen der Definition und Abgrenzung sowie der Literaturrecherche. Die bestehenden Hypothesen werden überprüft, beziehungsweise neue Hypothesen generiert werden. Durch den Vergleich mit der Studie "Neue Formen des Wissenschaftlichen Publizierens" aus dem Jahr 2007 vom Soziologisches Forschungsinstitut Göttingen (SOFI) soll die Betrachtung der historischen Entwicklung der Thematik im deutschsprachigen Raum ermöglicht werden. Des weiteren bildet die Befragung aus Göttingen die Grundlage für die Fragebogenkonstruktion der Erhebung im Rahmen dieser Arbeit.

Die umfangreiche Befragung aus den Jahren 2007 entstand im Rahmen eines durch das Bundesministerium für Bildung und Forschung (BMBF) geförderten Verbundprojekts zwischen dem SOFI Göttingen und der Universitätsbibliothek Göttingen. Sie basierte auf einer "Vollerhebung der Wissenschaftler an den Instituten und Einrichtungen an fünf deutschen Standorten, die differenziert nach Fächern, Alters- und Statusgruppen (n=1800) erfasst wurden" \cite{Hanekop_2014}. Ziel der Befragung war es, die "Veränderungen beim Zugang zur Literatur wie auch bei den Veröffentlichungsstrategie" \cite{Hanekop_Wittke_2007_Fragebogen} zu untersuchen. Die Teilnehmerinnen und Teilnehmer der Studie wurden anhand von Webseiten der Forschungseinrichtungen identifiziert und per Email um Teilnahme gebeten.

Durch den Vergleich mit der Vorbefragung in 2007, sowie durch die Überprüfung der Hypothesen in der empirische Arbeit soll das bestehende Wissen im Untersuchungsfeld erweitert werden, die Qualität der Arbeit verbessert und der "Neuigkeitswert" sichergestellt werden \cite{raab_2012_fragebogen}. Um diese Vergleichbarkeit zu ermöglichen und zu gewährleisten, wurde die Befragung dieser Arbeit an den Kriterien der Befragung im Jahr 2007 angelegt. Einziger Unterschied bei der Durchführung besteht in der ausschließlichen Onlinebefragung der Zielgruppen. Diese Arbeit folgt der Annahme, dass bei Validität und Reliabilität von Online-Befragungen im Vergleich zu "Papier-Bleistift-Befragungen" keine Unterschiede bestehen \cite{Batinic_2013_onlinebefrag}.

Für diese Herangehensweise der quantitativen Forschung reicht in der Regel ein "Hinweis auf die verwendeten Techniken und Messinstrumente aus, denn sie sind ja standardisiert vorgegeben".\cite{Mayring_1999:119}.

---- TODO: weitere Teil für Methode ----

\section{Das Experiment als wissenschaftliche Methode: Offenes Schreiben dieser Arbeit}

Um Handlungsempfehlungen für das offene Schreiben von Dissertationen erstellen zu können, sowie die Kriterien und Argumente für oder gegen das offene Publizieren prüfen zu können, wurde für diese Arbeit eine offene Schreibweise gewählt. “Offen” bedeutet in diesem Fall, dass diese Arbeit direkt und unmittelbar im Zeitraum der Erstellung für jeden, jederzeit frei zugänglich auf einer Webseite im Internet unter einer freien Lizenz (CC-BY-SA) veröffentlicht wurde. Der Stand der Arbeit auf der Webseite entsprach zu jedem Zeitpunkt dem tatsächlichen Stand der Arbeit.

Das im Rahmen dieser Arbeit durchgeführte Selbstexperiment unterschiedet sich von klassischen wissenschaftlichen Experimenten. Dieses Experiment ist dabei vom klassischen Laborexperiment als "Idealtypus kontrollierten Experimentierens", aber auch von der Feldbeobachtung, "die nicht vorsieht, dass im laufenden Betrieb eingegriffen und experimentiert wird" \cite{FQS196} zu unterscheiden. Experimente, bei denen Selbstbeobachtung eine zentrale Bedeutung haben, können nur dann als wissenschaftliche Methode anerkannt werden, wenn sie durch die präzise Rekonstruktion sicherstellen wie sie Wissen herstellen und wodurch sich dieses gewonnene Wissen als wissenschaftliches Wissen auszeichnet \cite{solhdju_2011_selbstexperimente}. Bei dieser Art der qualitativ orientierten Forschung, ist das Vorgehen sehr spezifisch und auf den jeweiligen Gegenstand bezogen. Die die Methoden werden dabei meist speziell für diesen Gegenstand entwickelt oder differenziert\cite{Mayring_1999:119}.

Methodisch wird für diesen Teil der Arbeit auf einen autoethnografischen Ansatz zurückgegriffen. Bei diesem Ansatz wird der Forscher oder die Forscherin zum "teilnehmenden Beobachter"\cite{Ellis_2010}. Er ermöglicht es, "persönliche Erfahrung (auto) zu beschreiben und systematisch zu analysieren (grafie), um kulturelle Erfahrung (ethno) zu verstehen" \cite{Ellis_2010}. Angelehnt an die sozialwissenschaftliche Methode folgt diese Arbeit dem Verständnis einer "Ethnographie über Menschen, die Medien nutzen, konsumieren, distribuieren oder produzieren"\cite{bachmann_2011_ethnographie}.

Um den Anforderungen der aktuell geltenden Prüfungsordnung der Leuphana Universität zu entsprechen, wurden vorab in einem Schreiben an die Promotionskommission die Bedingungen für die offene Erstellung der abgestimmt und die Vereinbarkeit mit der Promotionsordnung geprüft. Um die Eigenleistung und die Selbstständigkeit bei der Erstellung der wissenschaftlichen Qualifikationsarbeit zu gewährleisten, wurde technisch sichergestellt, dass es für andere als den Autoren keine Möglichkeit gab, den erstellten Inhalt zu editieren oder zu kommentieren. Im Gegenteil, sie ermöglicht eine neue Form, die Eigenständigkeit direkt während der wissenschaftlichen Arbeit und Erstellung des Inhalts darzustellen. Die Promotionskommission hat dem Gesuch die Arbeit "offen" verfassen zu dürfen am 12. Dezember 2013 mehrheitlich entsprochen. Die gewonnene Transparenz während des Erstellungsprozesses stellt in diesem Fall keinen Widerspruch zu der Selbständigkeit bei der Ausarbeitung dar.

Der Grund für die Wahl der autoethnografischen Untersuchungsmethode ist das Bestreben zu einem vertieften Verständnis der empirischen Ergebnisse zu gelangen, den Aufwand, der durch die Öffnung der formellen Kommunikation für Wissenschaftler und Wissenschaftlerinnen entsteht, zu beschreiben und zu analysieren sowie möglichst viele Verallgemeinerungsmodelle im Rahmen der definierten Fragestellungen theoretisch zu entwickeln und praktisch zu prüfen. Diese Zielsetzungen machen es erforderlich, einen methodischen Ansatz zu wählen, der es am Beispiel der eigenen Erstellung einer wissenschaftlichen Arbeit ermöglicht, die kulturelle Praxis des offenen wissenschaftlichen Kommunizierens besser zu verstehen \cite{maso_2001_phenomenology}.

Diese autoethnografischen Herangehensweise der Beschreibung und Analyse ist notwendig, da bisher kein dokumentiertes, offen verfasstes wissenschaftliches Publikations- oder Promotionsvorhaben im deutschsprachigen Raum durchgeführt wurde. Die Erfahrungen der offenen Schreibweise bilden einen wesentlichen Anknüpfungspunkt für die Beantwortung der Forschungsfragen und für weitere Forschung in diesem Feld.

\section{Begründung der Methodenwahl}

Wesentlich bei der Wahl der Untersuchungsmethoden sind das Thema und das zu untersuchende Feld. Das Thema der Öffnung wissenschaftliche Kommunikation in Deutschland ist noch im Anfangsstadium eines Entwicklungsprozesses. Zwar hat haben sich seit der wissenschaftlichen Revolution im 17. Jahrhundert die Methodischen Standards immer wieder gewandelt, doch sind "Sprache, Autorenschaft und Struktur bzw. Beurteilung wissenschaftlicher Artikel (...) Parameter, die jetzt mit der Verschiebung vom Buchdruck zur digitalen Publikation einmal mehr in Bewegung geraten" \cite{hagner_2015_sache_buches}. Das zu untersuchende Feld, die Teilnehmer am wissenschaftlichen Kommunikationssystem, sind bisher gar nicht oder erst seit kurzer Zeit mit dem Thema Offenheit in der Wissenschaft konfrontiert \cite{hagner_2015_sache_buches}.

Für die Bearbeitung eines wenig erforschten Feldes bietet sich die Methodik der qualitativen Sozialforschung an, allerdings begünstigt das Vorhandensein einer Vorbefragung vom SOFI in Göttingen auch die quantitativen Methode der Online-Befragung. Diese stellt einen methodisch erklärenden Ansatz mit einem hohe Messniveau dar, der für diese Arbeit als erkenntnisgewinnbringend eingeschätzt wird. Sie wird als hypothesenprüfende Methode beruhend auf der Quantifizierung der Beobachtungsrealität beschrieben \cite{bortz_Doering_2006_Methoden} mit der die Vorannahmen geprüft und neue Ansatzpunkte evaluiert werden können. Die Methode soll helfen einen differenzierten Ansatz zu finden, Vorurteile und Angebote zu identifizieren aber auch Probleme, Herausforderungen und Alternativen zu erforschen.

Anders als bei der quantitativen Sozialforschung mit sehr elaborierten Methoden,
lässt die qualitative Forschung mit dem klaren Bekenntnis zur Offenheit viel Freiheit und wird im weiteren Verlauf der Arbeit den Forschungsprozess bereichern. Das Selbstexperiment als Elemement der qualitativen Forschung, hilft auf Grundlage der Erkenntnisse aus der Befragung beim "Verstehen" der Materie. Hier werden die Herausforderungen und praxistauglichkeit der Forderungen für die öffnung des wissenschaftlichen Prozessen subjektiv, induktiv analysiert. Das hilft zum Aufbau der nötigen Distanz zu den Forderungen von Offenheit und Transparenz im Forschungsprozess.

Zusammenfassend werden mit Hilfe der quantitativen Methode der Befragung allgemeine Muster identifiziert und durch die qualitative Methoden des Experiments Beispielhaft an den Mechanismen der gesellschaftlichen Konstruktion der Wirklichkeit herausgearbeitet \cite{suchen}. Ziel dieser Methodenwahl ist es, einen wissenschaftlich fundierten Beitrag zu der Debatte zum Wandel von wissenschaftlicher Kommunikation im Rahmen der Digitalisierung in der Wissenschafts- und Technikforschung zu liefern.

---- TODO: weiter ausarbeiten auf Grundlage von bortz_Doering_2006_Methoden inklusive Bezug zu STS ----

\section{Kritische Betrachtung der Vorgehensweise}

Die Themen Digitalisierung und wissenschaftliche Kommunikation sind auch bei Eingrenzung auf den deutschsprachigen Raum und unter konkreten Fragestellungen weite Felder. Der Fokus dieser Arbeit stand auf der Gruppe der Wissenschaftler und Wissenschaftlerinnen, die aber nur eine von mindestens drei Gruppen des wissenschaftlichen Kommunikationssystems darstellen.

Durch die große Auswahl in der Stichproben, die umfassende Literaturrecherche vorab und die gewählte quantitative Vorgehensweise war die Erhebung zwar im erforderlichen Umfang vergleichbar, gab aber nur begrenzt Raum für die Erforschung neuer Herangehensweisen. Dem wurde entgegengearbeitet, indem die etnographische Herangehensweise des Selbstexperiments den Forschungsprozess kompletierte.

Die Vermengung des qualitativen Vorgehens mit dem quantitativen Forschungsprozess birgt zwar die Gefahr eines Qualitätsverlusts der Aussagekraft der Ergebnisse \cite{suchen_Lamek_1993:198}, die Kombination ist aber im Forschungsalltag üblich \cite{bortz_Döring_2006_Methoden}.

---- TODO: ausarbeiten ----
