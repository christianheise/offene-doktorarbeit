\chapter{Methoden und Vorgehen bei Empirie und Ethnographie}

Die Verortung der Fragestellungen dieser Arbeit von den Kulturwissenschaften über die Wirtschaftswissenschaften, die Politikwissenschaften bis hin zu den Medienwissenschaften, erfordert einen transdisziplinären Zugang zur wissenschaftlichen Bearbeitung. Da die Herangehensweise an die Science and Technology Studies (STS) angelehnt ist, wird neben dem transdisziplinären Zugang auch ein Methodenmix gewählt um das Themenfeld nicht nur zu entdecken, sondern aktiv an der Entwicklung des Themenfelds teilzunehmen \cite{MacKenzie_1999} und die empirischen Realitäten zu verstehen \cite{Kelty_2014}.

Auch wenn die Themenbereiche kollaboratives Arbeiten, Social Media in Wissenschaft und Forschung, Citizen Science sowie Diskurse zu Tools und Diensten eng mit der Digitalisierung und Öffnung der wissenschaftlichen Kommunikation verbunden sind \cite{European_Commission_2015a}, werden diese in dieser Arbeit bewusst nur am Rande und beiläufig erwähnt beziehungsweise nur eingeschlossen, wenn sie der Beantwortung der Forschungsfragen dienen oder diese tangieren. Bevor die Debatten und die in den Grundlagen herausgearbeiteten Herausforderungen für die Empirie und Ethnographie verdichtet und zusammengefasst werden, sollen Methoden und Vorgehen bei Empirie und Ethnographie dieser Arbeit dargestellt werden.

\section{Vorüberlegungen zur Methodenwahl}

Es werden folgende wissenschaftliche Erhebungsmethoden angewendet: die umfassende Literaturrecherche mit analytischen Elementen für die Begriffsbestimmung und für die weitere Ausarbeitung der Fragestellungen und Debatten sowie die quantitative Befragung zur Identifikation der Treiber und Bremser für die Öffnung wissenschaftlicher Informationen und Prozesse. Darüber hinaus wird die (auto-)ethnographische Methode angewendet, um im Rahmen der Betrachtung der offenen Erstellung der eigenen Doktorarbeit das Zusammenspiel von Wissen und Technologie in gesellschaftlichen Ordnungsprozessen näher zu bestimmen und Handlungsempfehlungen im Sinne einer Beschreibung der Auswirkungen auf die Kommunikation von Wissenschaftlern und Wissenschaftlerinnen zu erarbeiten.

Gegenstand dieser Literaturrecherche sind vor allem Beiträge, die das Thema "wissenschaftliche Kommunikation", "Open Access" und "Open Science" im Wortlaut verwenden. Neben der genauen Betrachtung der Begriffe und der dort genannten Hindernisse oder Katalysatoren für Veränderungsprozesse bei der wissenschaftlichen Kommunikation in den jeweiligen Texten wird der Suchprozess nach Diskurs- und Debattenfragmenten offen durchgeführt, um in der Diskussion möglichst alle Aspekte miteinzubeziehen. Die quantitative Befragung ermöglicht es, diese identifizierten Aussagen bei den wissenschaftlichen Akteuren und die eigenen Grundannahmen zu überprüfen. Das Experiment der offenen Erstellung der eigenen Arbeit ergänzt diese Perspektiven um die Arbeitsperspektive und dient der Beantwortung der Forschungsfragen und der abschließenden Diskussion der Ergebnisse.

Ziel dieser Abfolge ist es, in der theoretischen Phase "Fähigkeiten, Merkmale und Eigenschaften" \cite{Raab-Steiner_2012} zu definieren und zu strukturieren, diese in der empirischen Phase zu testen und abschließend mithilfe der ethnographischen Phase zu überprüfen und zu ergänzen. Dazu eignet sich die methodische Herangehensweise der Science and Technology Studies, mit der die Verschränkung von Wissenschaft, Technologie und Gesellschaft im Alltag untersucht und damit auch die Rolle von Wissen und Technologie in gesellschaftlichen Ordnungsprozessen näher bestimmt werden soll \cite{Beck_2014}.

\section{Generelle Forschungsfragen}

Wie zuvor ausgeführt, liegen trotz hoher Relevanz bisher nur wenige konkrete Untersuchungen und Experimente zur Öffnung wissenschaftlicher Kommunikation, vor allem aus den Geisteswissenschaften, vor. Um mit dieser Arbeit einen Beitrag zum Fortschritt für die Wissenschafts- und Technikforschung zu erzielen, werden folgende zentrale Forschungsfragen aus der einleitenden Betrachtung des Forschungsthemas abgeleitet:
\begin{itemize}
\item Welche Herausforderungen bestehen im aktuellen wissenschaftlichen Kommunikationssystem und wie kam es zu der Forderung nach Öffnung der wissenschaftlichen Kommunikation?
\item Welche Aspekte von Open Access und Open Science sind am häufigsten verbreitet?
\item Wie hoch ist das Interesse in der wissenschaftlichen Gemeinschaft an dem Zugang zu, Zugriff auf und Öffnung der wissenschaftlichen Kommunikation?
\item Wie stark ist die Öffnung der Kommunikation verbreitet?
\item Welche Argumente spielen in der Debatte für und wider die Öffnung wissenschaftlicher Kommunikation eine Rolle?
\item Welche Haupteinflussfaktoren für die Entwicklung von Forderungen nach Open Access und Open Science gibt es?
\item Welche Bedeutung haben die Konzepte um Offenheit und freien Zugang im Rahmen des wissenschaftlichen Reputationsbegriffs?
\item Welcher Aufwand entsteht bei der Öffnung des gesamten wissenschaftlichen Erkenntnisprozesses?
\item Welche Handlungsempfehlungen können für das Verfassen einer offenen wissenschaftlichen Arbeit gegeben werden?
\item Befindet sich die Öffnung des Zugangs zu publizierten wissenschaftlichen Erkenntnissen (Open Access) in einer andauernden Übergangsphase zur Öffnung des Zugriffs auf den gesamten wissenschaftlichen Erkenntnisprozess (Open Science)?
\end{itemize}

Diese Forschungsfragen basieren auf folgenden Vorannahmen, die über die Öffnung wissenschaftlicher Kommunikation im Rahmen der Digitalisierung und unter der Differenzierung zwischen den verschiedenen wissenschaftlichen Disziplinen sowie vor dem Hintergrund wissenschaftlicher Reputation vorliegen: \begin{itemize}
\item Die Forderung nach Öffnung der wissenschaftlichen Kommunikation entstand aus unterschiedliche Interessen und Zielsetzungen \cite{Hofmann_2015}.
\item Die Herausforderungen im aktuellen wissenschaftlichen Kommunikationssystem resultieren aus sozialen, technischen, rechtlichen und politischen Fehlentwicklungen.
\item Open Access und Open Science lassen sich nicht klar und einheitlich definieren \cite{Naeder_2010}.
\item Die Motivation der wissenschaftlichen Gemeinschaft für Veränderungen am System der wissenschaftlichen Kommunikation ist gering \cite{Hagner_2015}.
\item Es besteht eine erhebliche Diskrepanz zwischen dem Interesse an den Ideen der offenen Wissenschaft und der wissenschaftlichen Realität \cite{Scheliga_2014}.
\item Rechtliche Rahmenbedingungen beeinflussen maßgeblich die Etablierung der Öffnung von Wissenschaft und Forschung \cite[:211]{Fehling_2014}.
\item Trotz Initiativen und effizienterer Technologien, die die Öffnung der Kommunikation begünstigen, werden die wissenschaftlichen Akteure am bestehenden Publikationssystem festhalten und die Monografien- und Zeitschriftenkrise wird auf absehbare Zeit bestehen bleiben \cite{Parks_2002} \cite[:146]{Goetting_2015}.
\item Die Bereitschaft zur Öffnung der wissenschaftlichen Kommunikation ist in den unterschiedlichen Disziplinen unterschiedlich stark verbreitet \cite{Hofmann_2015} \cite{European_Commission_2006} \cite{Pansegrau_2011}.
\item Die Bedrohung der Publikations- und Forschungsfreiheit wird aus Sicht der wissenschaftlichen Gemeinschaft als ein Kernargument gegen die politische Forderung nach Öffnung wissenschaftlicher Kommunikation im Rahmen der Digitalisierung angeführt \cite{siehe_unten}.
\item Das Konzept der Offenheit wird in der neoliberalen Rhetorik als effizientes Wettbewerbsmodell im Rahmen der  politischen Steuerung eingesetzt \cite{Tkacz_2012}.
\item Die Öffnung des gesamten wissenschaftlichen Erkenntnisprozesses im Rahmen einer wissenschaftlichen (Qualifikations-)Arbeit ist möglich.
\item Die Öffnung des Zugangs zu finalen wissenschaftlichen Publikationen (Open Access) wird langfristig in einer Öffnung des Zugriffs auf den wissenschaftlichen Erkenntnisprozess (Open Science) münden.
\end{itemize}

\section{Methodenwahl}

Methoden sind die Verfahren und Strategien für die Informationsbeschaffung, die sich bestimmter wissenschaftliche Erhebungsinstrumente bedienen \cite[:309]{Kromrey_2013}. Die Wahl der Methoden ist für die wissenschaftliche Arbeit von großer Bedeutung und muss an die Fragestellungen, die Vorannahmen sowie das Forschungsvorhaben angepasst sein. Dennoch kann die Wahl der Methoden auch ganz pragmatische Gründe haben.

Die Darstellung der Methoden macht es möglich transparent zu beschreiben, wie, wann, wo, welche Daten und Informationen im Rahmen bestimmter Fragestellungen erhoben und analysiert wurden. Diese verschiedenen Verfahrensweisen und Techniken geben dem wissenschaftlichen Erkenntnisprozess einen strukturellen und systematischen Rahmen.

\subsection{Quantitative teilstandardisierte Datenerhebung (Online-Befragung)}

Nach der Ausarbeitung der Grundlagen und Definitionen sowie der Literaturstudie soll durch die Erforschung von Tatbeständen (Exploration), durch die Überprüfung von Hypothesen (Überprüfung) \cite{Raab-Steiner_2012} und die Erklärung von menschlichem Handeln \cite{Atteslander_2008} der Bestand an gesichertem Wissen in dem Untersuchungsbereich erweitert werden \cite{Bortz_2006a}. Um der Entwicklung der Öffnung von Wissenschaft sowie deren Treiber und Bremser nachgehen zu können, wird als zweite Methode eine explorative, schriftliche Online-Befragung unter den wissenschaftlichen Akteuren des akademischen Publizierens an deutschsprachigen wissenschaftlichen Institutionen durchgeführt.

Fragebögen eignen sich dabei besonders für große homogene Gruppen, wie die der Wissenschaftler und Wissenschaftlerinnen \cite{Bortz_2006}. Da es sich bei der eingesetzten Online-Befragung um einen teilweise gestaltbaren Ablauf handelt, wird die Art der Befragung als teilstandardisiert bezeichnet \cite{Raab-Steiner_2012}. Die hohe praktische Relevanz und vielfältige Einsatzmöglichkeiten machen den Fragebogen zu der am häufigsten eingesetzten Methode zur Datenerhebung in den empirischen Sozialwissenschaften \cite{Raab-Steiner_2012}. Diese Methode soll es dem Forscher oder der Forscherin ermöglichen, Ausschnitte der Realität abzubilden \cite{Raab-Steiner_2012}.

Die bestehenden Grundannahmen werden mithilfe der Befragung überprüft und gegebenenfalls neue Hypothesen generiert. Durch den Vergleich mit der Studie "Neue Formen des Wissenschaftlichen Publizierens" aus dem Jahr 2007 vom Soziologischen Forschungsinstitut Göttingen (SOFI) ist eine Betrachtung der historischen Entwicklung der Thematik im deutschsprachigen Raum möglich. Die Befragung aus Göttingen bildet eine Grundlage für die Fragebogenkonstruktion der vorliegenden Arbeit und unterstreicht die Bemühungen zur Absicherung der wissenschaftlichen Güte.

Die umfangreiche Befragung aus dem Jahr 2007 entstand im Rahmen eines durch das Bundesministerium für Bildung und Forschung (BMBF) geförderten Verbundprojekts zwischen dem SOFI Göttingen und der Universitätsbibliothek Göttingen. Sie basierte auf einer "Vollerhebung der Wissenschaftler an den Instituten und Einrichtungen an fünf deutschen Standorten, die differenziert nach Fächern, Alters- und Statusgruppen (n=1800) erfasst wurden" \cite{Hanekop_2014}. Ziel der Befragung war es, die "Veränderungen beim Zugang zur Literatur wie auch bei den Veröffentlichungsstrategien" \cite{SOFI_2007} zu untersuchen. Die Teilnehmer und Teilnehmerinnen der Studie wurden anhand von Webseiten der Forschungseinrichtungen identifiziert und per E-Mail um Teilnahme gebeten.

Durch den Vergleich mit der Vorbefragung im Jahr 2007 sowie durch die Überprüfung der Hypothesen in der empirischen Arbeit soll das bestehende Wissen im Untersuchungsfeld erweitert werden, die Qualität der Arbeit verbessert und der "Neuigkeitswert" sichergestellt werden \cite{Raab-Steiner_2012}. Um diese Vergleichbarkeit zu ermöglichen und zu gewährleisten, wurde die Befragung dieser Arbeit an den Kriterien der Befragung aus dem Jahr 2007 angelegt. Ein Unterschied bei der Durchführung besteht in der ausschließlichen Online-Befragung der Zielgruppen. Diese Arbeit folgt der Annahme, dass bei Validität und Reliabilität von Online-Befragungen im Vergleich zu "Papier-Bleistift-Befragungen" keine Unterschiede bestehen \cite{Batinic_2013}.

\subsection{Das Experiment als wissenschaftliche Methode: Offenes Schreiben dieser Arbeit}

Um Handlungsempfehlungen für das offene Schreiben von Dissertationen erstellen zu können sowie die Kriterien und Argumente für oder wider das offene Publizieren prüfen zu können, wurde für diese Arbeit eine offene Schreibweise gewählt. "Offen" bedeutet in diesem Fall, dass diese Arbeit und alle erhobenen Daten sowie Begleitinformationen direkt und möglichst unmittelbar im Zeitraum der Erstellung für jeden jederzeit frei zugänglich auf einer Webseite im Internet unter einer freien Lizenz (Creative Commons Namensnennung – Weitergabe unter gleichen Bedingungen 3.0 \cite{Creative_Commons_2008a}) veröffentlicht wurde. Der Stand der Arbeit auf der Webseite entsprach zu jedem Zeitpunkt dem tatsächlichen Stand der Arbeit.

Das im Rahmen dieser Arbeit durchgeführte Experiment unterscheidet sich vom klassischen wissenschaftlichen Experiment als "Idealtypus kontrollierten Experimentierens", aber auch von der Feldbeobachtung, "die nicht vorsieht, dass im laufenden Betrieb eingegriffen und experimentiert wird" \cite{Westermayer_2006}. Die Form des hiesigen Experiments fügt sich in das Konzept des Realexperiments ein und passt somit in die Technik- und Wissenschaftsforschung \cite{Westermayer_2006}. Diese Form des Experiments geht davon aus, "dass man relativ viel über das, was man nicht weiß, wissen kann, und dass das Ausprobieren der effektivste Weg ist, sich selbst zu korrigieren und weiterzukommen" \cite{Krohn_2005}. Realexperimente "sind experimentell orientiert, stehen unter situativ vorgegebenen Randbedingungen und verknüpfen Wissensanwendung und Wissensgenerierung" \cite{Westermayer_2006}. Bei dieser Art der qualitativ orientierten Forschung ist das Vorgehen sehr spezifisch und auf den jeweiligen Gegenstand bezogen \cite{Krohn_2005}. Sie ermöglicht eine speziell für diesen Gegenstand entwickelte oder differenzierte Herangehensweise \cite[:119]{Mayring_1999}. Die Anerkennung der Selbstbeobachtung als wissenschaftliche Methode wird in dieser Arbeit dadurch gewährleistet, dass die präzise und offene Dokumentation sowie der offene Prozess der Anfertigung sicherstellt, wie das Wissen erzeugt wird und wodurch sich dieses gewonnene Wissen als wissenschaftliches Wissen auszeichnet \cite{Solhdju_2011}

Für die Auswertung wird auf einen autoethnographischen Ansatz zurückgegriffen. Bei diesem Ansatz wird der Forscher oder die Forscherin zum "teilnehmenden Beobachter"\cite{Ellis_2010}. Er ermöglicht es, "persönliche Erfahrung (auto) zu beschreiben und systematisch zu analysieren (graphie), um kulturelle Erfahrung (ethno) zu verstehen" \cite{Ellis_2010}. Als Forschungsmethode ermöglicht sie eine Reflexion darüber, wie die eigenen Erfahrungen den Forschungszusammenhang beeinflussen \cite{Ellis_2011}. Angelehnt an die sozialwissenschaftliche Methode folgt diese Arbeit einem Verständnis einer "Ethnographie über Menschen, die Medien nutzen, konsumieren, distribuieren oder produzieren"\cite{Bachmann_2011}.

Das Ziel der autoethnographischen Untersuchungsmethode ist das Bestreben zu einem vertieften Verständnis der empirischen Ergebnisse zu gelangen, den Aufwand, der durch die Öffnung der formellen Kommunikation für Wissenschaftler und Wissenschaftlerinnen entsteht, zu beschreiben und zu analysieren sowie möglichst viele Verallgemeinerungsmodelle im Rahmen der definierten Fragestellungen theoretisch zu entwickeln und praktisch zu prüfen. Diese Zielsetzungen machen es erforderlich, einen methodischen Ansatz zu wählen, der es am Beispiel der eigenen Erstellung einer wissenschaftlichen Arbeit ermöglicht, die kulturelle Praxis des offenen wissenschaftlichen Kommunizierens besser zu verstehen \cite{maso_2001_phenomenology}.

Die autoethnographische Herangehensweise der Beschreibung und Analyse ist auch deshalb notwendig, da bisher kein dokumentiertes, offen verfasstes wissenschaftliches Publikations- oder Promotionsvorhaben im deutschsprachigen Raum durchgeführt wurde. Die Erfahrungen der offenen Schreibweise bilden darüber hinaus einen wesentlichen Anknüpfungspunkt für die Beantwortung der Forschungsfragen und stellen die Grundlage für weitere Forschung in diesem Feld sowie für die Erbringung eines Beitrags zum Fortschritt für die Wissenschafts- und Technikforschung dar.

\section{Begründung der Methodenwahl}

Die Methodenwahl begründet sich zunächst auf den theoretischen Vorannahmen im Rahmen der Literaturrecherche. Wesentlich bei der Wahl der Untersuchungsmethoden sind Herangehensweise, Thema und das zu untersuchende Feld. Das Thema der Öffnung wissenschaftlicher Kommunikation in Deutschland befindet sich noch im Anfangsstadium eines Entwicklungsprozesses. Zwar haben sich seit der wissenschaftlichen Revolution im 17. Jahrhundert die methodischen Standards immer wieder gewandelt, doch sind "Sprache, Autorenschaft und Struktur bzw. Beurteilung wissenschaftlicher Artikel (...) Parameter, die jetzt mit der Verschiebung vom Buchdruck zur digitalen Publikation [stattfinden,] einmal mehr in Bewegung geraten" \cite{Hagner_2015}. Das zu untersuchende Feld, die Teilnehmer und Teilnehmerinnen des wissenschaftlichen Kommunikationssystems, sehen sich bisher allerdings gar nicht oder erst seit kurzer Zeit mit dem hier behandelten Thema konfrontiert \cite{Hagner_2015}.

Für eine Bearbeitung wenig erforschter Felder bietet sich die Methodik der qualitativen Sozialforschung an. Allerdings begünstigt das Vorhandensein einer Vorbefragung vom SOFI in Göttingen aus dem Jahr 2007 auch die quantitative Methode der (Online-)Befragung. Die quantitative Herangehensweise hat den Vorteil eines methodisch erklärenden Ansatzes mit einem hohen Messniveau, der für diese Arbeit als erkenntnisgewinnbringend eingeschätzt wird. Als hypothesenprüfende Methode beruht sie auf der Quantifizierung der Beobachtungsrealität \cite{Bortz_2006b}, mit der die dargestellten Vorannahmen geprüft und neue Ansatzpunkte evaluiert werden können. Sie ermöglicht darüber hinaus die differenzierte Erforschung und Bearbeitung sowie die unvoreingenommene und differenzierte Identifikation von Vorurteilen und Angeboten, aber auch von Problemen, Herausforderungen und Alternativen.

Anders als bei der quantitativen Sozialforschung mit sehr elaborierten Methoden, lässt die qualitative Forschung mit dem klaren Bekenntnis zur Offenheit viel Freiheit und wird im weiteren Verlauf der Arbeit den Forschungsprozess bereichern. Das Experiment als Element der qualitativen Forschung hilft auf Grundlage der Erkenntnisse aus der Befragung beim "Verstehen" der Materie. Hier werden die Herausforderungen und die Praxistauglichkeit der Forderungen für die Öffnung des wissenschaftlichen Prozesses subjektiv induktiv analysiert. Das hilft beim Aufbau einer Distanz zu den Forderungen von Offenheit und Transparenz im Forschungsprozess und ermöglicht eine praktisch-fundierte Diskussion der Ergebnisse.

Zusammenfassend werden mithilfe der Literaturrecherche und der quantitativen Methode der Befragung allgemeine Muster identifiziert und durch die qualitative Methode des Experiments beispielhaft an den Mechanismen der gesellschaftlichen Konstruktion der Wirklichkeit herausgearbeitet. Ziel dieser Methodenwahl ist es, einen wissenschaftlich fundierten Beitrag zu der Debatte über den Wandel wissenschaftlicher Kommunikation im Rahmen der Digitalisierung in der Wissenschafts- und Technikforschung (STS) zu liefern.

\section{Kritische Betrachtung der Vorgehensweise}

Die Themen Digitalisierung und wissenschaftliche Kommunikation sind auch bei Eingrenzung auf den deutschsprachigen Raum und unter konkreten Fragestellungen weite Felder. Im Fokus dieser Arbeit steht die Gruppe der Wissenschaftler und Wissenschaftlerinnen. Sie stellen aber nur eine, wenn auch eine wesentliche, von mindestens drei Gruppen des wissenschaftlichen Kommunikationssystems dar. Mitarbeiter und Mitarbeiterinnen von Verlagen und Bibliotheken wurden zwar bewusst nur am Rande berücksichtigt beziehungsweise nur dann mit eingeschlossen, wenn dies der Beantwortung der Forschungsfragen dient oder diese tangieren, dennoch sind sie ebenfalls wichtige Akteure des wissenschaftlichen Kommunikationssystems.

Kritisch betrachtet können die Fächervielfalt und die Unterschiede in den einzelnen anderen Disziplinen im Rahmen der wissenschaftlichen Kommunikation hier nur begrenzt abgebildet werden. Das gilt insbesondere für die autoethnographische Betrachtung der offenen Anfertigung dieser Arbeit, die klar in den Geisteswissenschaften verortet ist. Die Erfahrungen und Beobachtungen können somit nur bedingt übertragen werden und weichen gegebenenfalls von anderen Disziplinen ab. Sie bieten jedoch eine wesentliche Grundlage für Anknüpfungspunkte bei der Beantwortung der Forschungsfragen sowie für weitere Forschung auf diesem Gebiet.

Durch die große Auswahl der Stichproben, die umfassende Literaturrecherche vorab und die gewählte quantitative Vorgehensweise ist die Erhebung zwar im erforderlichen Umfang vergleichbar und statistisch relevant, gibt aber nur begrenzt Raum für die Erforschung neuer Herangehensweisen an die Thematik. Um dem entgegenzuwirken, komplettiert der offene Forschungsprozess im Rahmen des Experiments das methodische Vorgehen und ermöglicht die Ergänzung der quantitativen Erhebung um eigene Beobachtungen und Erfahrungen.

Kritisch kann auch die Vermengung und parallele Anwendung eines qualitativen Vorgehens mit dem quantitativen Ansatz hinterfragt werden. Diese Kombination ist im Forschungsalltag allerdings dennoch nicht unüblich \cite{Bortz_2006b}. Zwar ist die Gefahr eines Qualitätsverlusts der Aussagekraft der Ergebnisse durch diese Vorgehensweise nicht ganz auszuschließen \cite[:198]{Lamnek_1993}. Dennoch eignet sich dieser Methodenmix bevorzugt im Zusammenhang mit der Herangehensweise der Technik- und Wissenschaftsforschung \cite[:8]{Brown_2014}.

Die Position des Autors als aktiver Teilnehmer der Debatte um die Öffnung der wissenschaftlichen Kommunikation ist ebenfalls kritisch zu berücksichtigen. Durch die Fokussierung auf die Literatur bei der Extrahierung der Annäherungen an die Begriffe sowie die Darstellung der Debatten um Open Access und Open Science sowie die Übernahme der Fragen aus der Vorbefragung vom Soziologischen Forschungsinstitut Göttingen wird sichergestellt, dass die langjährige Auseinandersetzung des Autors oder der Autorin mit dem Thema nicht zu einer suggestiven Herangehensweise bei der Bearbeitung der Forschungsfragen führt. Die aktive Auseinandersetzung mit der eigenen Position und das Realexperiment können als effektiver Weg betrachtet werden, sich selbst zu korrigieren und das Thema kritisch zu bearbeiten \cite{Krohn_2005}.
