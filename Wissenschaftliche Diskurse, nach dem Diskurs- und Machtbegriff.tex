\subsubsection{Wissenschaftliche Diskurse, nach dem Diskurs- und Machtbegriff}
Nach Niklas Luhmann operiert der wissenschaftliche Diskurs funktional eigenständig und alles was durch Wissenschaft kommuniziert wird, ist “entweder wahr oder unwahr” . Der wissenschaftliche Diskurs gründet dabei aber nur zum Teil auf der Forschung und kann auch nicht nur als  “Kontaktglied zwischen dem Denken und dem Sprechen”  definiert werden. In der Foucault'schen Diskursanalyse wird der Diskurs deshalb als die Fähigkeit definiert, die “Beziehungen” zwischen “Institutionen, ökonomischen und gesellschaftlichen Prozessen, Verhaltensformen, Normsystemen, Techniken, Klassifikationstypen und Charakterisierungsweisen herzustellen” . Foucault beschäftigt sich in diesem Zusammenhang vor allem mit den Grenzen des Diskurses sowie dessen institutioneller und praktischer Verortung. In diesem Zusammenhang soll in dieser Arbeit auch adressiert werden inwiefern Macht, regulierende Prinzipien wie Verknappung sowie die Ein- und Ausgrenzung bezüglich des wissenschaftlichen Diskurses, nach dem Diskurs- und Machtbegriff von Michel Foucault, mit den Modellen der Open Initiatives in der wissenschaftlichen Kommunikation vereinbar sind oder dem gegenüberstehen. Im Gegensatz zu innerdiziplinären Betrachtung eignet sich Foucaults “Werkzeugkiste”  dabei besonders um die transdisziplinäre Öffnung von wissenschaftlichen Prozessen und den damit einhergehende Öffnung des Diskurses theoretisch zu hinterfragen. 
In diesem Kapitel soll deshalb der Diskursbegriff in den Kontext der Thematik der Öffnung des Zugriff auf den wissenschaftlichen Prozess erläutert werden.
